\chapter{Nota da tradução}

O texto aqui apresentado é resultado de dois mo(vi)mentos: o primeiro
deles remonta ao ano de 2005, quando José Antônio Martins e Oliver Tolle
começaram a fazer a tradução de \emph{O Príncipe} a ser publicado pela
Editora Hedra, na sua coleção de bolso; e o segundo, em 2019, quando
José Antônio Martins, Márcio Roberto do Prado e Liliam Cristina Marins,
estes, professores do Programa de Pós"-graduação em Letras da \versal{UEM},
passaram a fazer a revisão da tradução para essa nova edição.

No primeiro mo(vi)mento, o trabalho de tradução se integrou às pesquisas
de doutorado em filosofia na \versal{USP}. Desde o início, uma questão e uma
exigência se colocavam nesta demanda editorial, a saber: por que fazer
mais uma tradução de \emph{O Príncipe}, sabendo de antemão da existência
de dezenas de edições em língua portuguesa? Para não ser mais uma entre
muitas, tal edição deveria trazer algo que a diferenciasse. De comum
acordo, depois de pesquisar sobre as mais de três dezenas de publicações
até então, verificou"-se que havia ainda espaço para fazer uma tradução a
partir da edição crítica estabelecida por Giorgio Inglese, em 1994. Além
disso, ficou evidente que havia espaço também para uma edição bilíngue,
algo inédito nas edições nacionais e portuguesas, voltada para o
ambiente acadêmico. Para este objetivo, o texto deveria, assim,
apresentar um vocabulário adequado ao meio de circulação desejado,
distanciando"-se de um público maior, neófito no pensamento maquiaveliano
e no pensamento político de modo geral. Importa dizer que muitas das
edições correntes até então não eram traduzidas a partir do original
italiano e também não foram realizadas por especialistas ou
pesquisadores do pensamento maquiaveliano, o que se evidenciava pelas
inúmeras escolhas tradutórias destituídas de um rigor conceitual.

A primeira parte da tradução foi realizada no Brasil e a finalização do
trabalho foi realizada na Itália, durante a estadia de José Antônio na
cidade de Pisa, na Toscana, no âmbito do doutorado sanduíche financiado
pela Capes. Neste período, foi possível se valer das diversas edições da
obra, textos de comentários, léxicos etc., nas melhores bibliotecas da
Itália: a da Scuola Normale di Pisa, da Università di Pisa, e a do
Istituto del Rinascimento Italiano, em Florença. Experiência essa
importante para o contato com esses riquíssimos acervos e com alguns
especialistas italianos, os quais esclareceram sobre o uso do toscano e
das técnicas e estilo de escrita de Maquiavel.

No segundo semestre de 2007, tem"-se a publicação da primeira edição
bilíngue de \emph{O Príncipe} em língua portuguesa, com uma pequena
introdução e notas; em 2009, há a publicação da 2ª edição com algumas
correções no texto e nas notas, e alteração na diagramação; em 2011, a
segunda reimpressão desta 2ª edição.

No entanto, desde a publicação da primeira edição, em 2007, o desejo de
fazer uma introdução mais robusta, que agregasse novas notas ao texto, e
críticas e sugestões recebidas dos colegas acadêmicos, ficou latente. A
intenção de fazer uma edição comemorativa em 2013, por ocasião da
efeméride dos 500 anos da obra maquiaveliana, não foi viabilizada por
diversas razões.

Esse desejo se consolidou apenas no primeiro semestre do ano de 2019,
materializando o segundo mo(vi)mento: o da revisão da tradução. O
resultado foi um trabalho feito a seis mãos, que permitiu alterações
significativas na tradução, ao mesmo tempo que corrigiu ainda algumas
imperfeições textuais, dotando"-a de mais fluidez e adequação ao
português, mas sem comprometer o rigor conceitual. Longe de entrar em
discussões rasas sobre liberdade do tradutor \emph{versus} fidelidade ao
autor na tradução, nosso objetivo se pautou mais consistentemente em uma
prática tradutória -- na qual a revisão tem um papel inquestionável --
do texto filosófico, que tem suas próprias amarras e \emph{modus
operandi}. Além deste foco no rigor conceitual, o processo de revisão
teve como norte o estabelecimento de um estilo de escrita que buscasse
traduzir o texto maquiaveliano para o português do século \versal{XXI}, evitando,
ao máximo, termos arcaizantes, anacronismos e estrangeirismos
excessivos.

Em relação às estratégias tradutórias, houve apenas um caso no qual foi
necessário manter o vocábulo em italiano e este se refere ao termo
\emph{virtù}, que não poderia ser deliberadamente traduzido por
``virtude'' em português, já que esta escolha comprometeria o conceito
filosófico. A fim de não nos desviarmos do propósito de deixar a
tradução mais ``palatável'' para o leitor acadêmico brasileiro em termos
de estilo -- sem nunca perder as travas de segurança conceituais --, a
estratégia tradutória foi a nota de rodapé.

Ainda em relação à presença de estrangeirismo na tradução, mas em termos
estilísticos, tendo em vista que Maquiavel era um grande latinista, foi
necessário recorrer ao modo de construção de orações em latim nos
momentos em que as passagens do texto em italiano se colocavam como
obstáculo à compreensão. Por isso, remetíamos ao latim, para, a partir
deste, trabalhar na revisão da tradução para o português.

Em relação às adaptações linguísticas, tivemos que mudar, em alguns
momentos, a ordem sintática, que, em Maquiavel, apresenta muitas
inversões. No caso do português brasileiro, nosso modelo discursivo é
comumente a ordem direta e, se mantivéssemos a ordem inversa,
dificultaríamos muito a leitura e a construção de sentidos do texto.
Outro aspecto que envolveu a necessidade de uma ``cor local'' se refere
à pontuação que, em toscano, diverge muito daquela da língua portuguesa
contemporânea, de modo que não foi possível tomar o original como ponto
de partida nesse sentido.

Por fim, o elo que une esses dois mo(vi)mentos de tradução e de revisão
é, indubitavelmente, a colaboração e a multivocalidade discursiva: a do
pensador, a do tradutor, a dos revisores, a de outros tradutores etc, 
trabalho esse que enriquece o produto final.

\enlargethispage{\textheight}

\chapter*{}
\addcontentsline{toc}{part}{O Príncipe}
\begin{center}
\begin{vplace}[0.3]
\Large
O Príncipe\\
\textit{De principatibus}
\end{vplace}
\end{center}
\thispagestyle{empty}