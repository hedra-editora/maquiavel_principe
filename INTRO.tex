\chapter*{Introdução}
\addcontentsline{toc}{chapter}{Introdução, por José Martins}

\section{Considerações iniciais}

Dada a quantidade (mais de três dezenas) de traduções e edições de
\emph{O Príncipe} de Nicolau Maquiavel somente no Brasil, sem contar com
as centenas de outras edições nas diversas línguas modernas, essa nova
edição publicada pela Hedra busca atender algumas especificidades e não
repetir aquilo que já foi divulgado à exaustão.

Quanto à edição, trata-se de propiciar ao leitor brasileiro um exemplar
bilíngue, com a melhor edição do texto original italiano -- a
\emph{Edição Crítica Inglese} -- acrescida de introdução e notas
explicativas. A exigência e a novidade de que haja o texto italiano
original, algo presente desde a primeira publicação em 2007, atende a
uma demanda por qualidade nos textos acadêmicos, a saber: permitir que o
leitor possa aprofundar (ou mesmo compreender melhor passagens da obra
que não tenham ficado claras na tradução) por meio do cotejamento com o
original.

A introdução aqui exposta visa a dois objetivos: a) como de praxe nas
boas edições, apresentar ao leitor o autor e os principais aspectos da
obra; b) para além desse aspecto básico, apresentar também um viés
interpretativo que vem se consolidando entre os comentadores, ou seja: a
inclusão de \emph{O Príncipe} no interior da reflexão republicana de
Maquiavel, com destaque para as noções de principado e príncipe. Dito
isso, convém lembrar que essa introdução não pretende ser nem uma
reapresentação da biografia do filósofo, nem uma introdução ao
pensamento político maquiaveliano como um todo, nem uma análise
exaustiva e minuciosa dos diversos aspectos teóricos presentes na obra.
Insistindo e explicitando melhor, apesar de mobilizar informações sobre
a vida de Maquiavel e alguns aspectos de seu pensamento político, essa
introdução busca pensar \emph{O Príncipe} como uma obra peculiar e
fundamental para a compreensão das noções políticas republicanas de
Maquiavel.

Por fim, inserimos algumas notas que visam apresentar personagens, datas
e eventos do contexto histórico no qual o livro está inserido e explicar
conceitos e passagens centrais do texto. Tendo em vista a presença e
natureza da introdução, o uso desse segundo tipo de notas foi restrito
ao mínimo essencial e indispensável.

\section{O contexto histórico maquiaveliano e~os~seus~pressupostos~teóricos}

\subsection{O contexto histórico: as origens e o ingresso~na~Chancelaria~florentina}

Nicolau Maquiavel ou Nicolló Machiavelli nasceu em Florença, Itália, no
dia 3 de maio de 1469, em uma família de pequenas posses. Sabe-se que
seu pai -- Bernardo Machiavelli --, em função de ser escriturário, teve
alguma formação jurídica, mas não ocupou um cargo destacado e nem
exerceu importantes cargos na administração da cidade, tampouco acumulou
grandes posses. Todavia, em função dessa formação jurídica do patriarca,
a casa dos Machiavelli dispunha de obras clássicas de história, de
doutrina política, de jurisprudência, enfim, uma pequena biblioteca nos
moldes de uma sociedade humanista\footnote{É sempre difícil definir com
  poucas palavras grandes manifestações históricas, como foi o caso do
  Humanismo, que teve como seu lugar de nascimento e desenvolvimento a
  Itália dos séculos \versal{XIII} ao século \versal{XVI}. Isso que se entende por
  Humanismo caracterizou-se por uma retomada dos valores culturais da
  Antiguidade e por uma maior valorização do homem. Por isso, busca-se
  ler os autores clássicos nos seus textos originais, seja em grego,
  seja em latim; há uma intensa busca pela redescoberta desses textos,
  quase sempre esquecidos em alguma biblioteca, produzindo-se novas
  edições; uma maior valorização daquilo que diz respeito ao homem e às
  coisas humanas, numa tentativa de contrabalancear a pouca valorização
  desses aspectos pela cultura cristã-medieval. Concomitantemente, há
  uma maior valorização das virtudes ligadas à vida na cidade,
  principalmente as qualidades relacionadas à política, por oposição aos
  valores religiosos cristãos, característicos de grande parte do
  período medieval. Esses aspectos, entre outros, sinalizam a mudança
  que se operará nas sociedades europeias a partir de meados do século
  \versal{XIII} e que será predominante já no século \versal{XV}, preparando o terreno
  para aquilo que se conhecerá como a Era Moderna. Para maiores
  informações cf: \versal{HALE}, 1971; \versal{SKINNER}, 2000, cap. 1-3; \versal{BARON}, 1989.} do
\emph{Quattrocento}.

Sobre a formação de Nicolau Maquiavel, tendo em conta esse ambiente
humanista e a acessibilidade às obras clássicas, sabe-se que ele teve
aulas com professores particulares, não havendo registro sobre uma
possível frequência em escola ou universidade. Segundo Guidi, Maquiavel
teve aulas com os mestres de latim Paolo Sassi, Michele Verino e Pietro
Crinito (\versal{GUIDI}, 2009). A partir de tal formação, deduz-se que Maquiavel
tenha ultrapassado o que seria um nível intermediário, tendo em vista as
exigências acerca do conhecimento da língua latina para o cargo que
ocupou na Chancelaria, sendo que, posteriormente, revelou um grande
conhecimento de autores como Cícero, Lucrécio, Tito Lívio, etc.
Considerando toda a sua produção teórica e literária, comprova-se que
ele conhecia muito bem a sua língua, o toscano, e o latim, assim como
possuía uma vasta compreensão de história e noções de filosofia. Além
dessa formação em língua clássica, ele foi introduzido ao cálculo
matemático, caracterizado pelo ensino do ábaco, e teve alguma formação
em jurisprudência, embora não tenha se tornado bacharel em direito.
Pode-se afirmar com segurança que Maquiavel não aprendeu o grego. É
nesse ambiente cultural e familiar que sua vida se desenvolve até o ano
de 1498, quando começa a trabalhar na Chancelaria florentina, onde
permanecerá pelos próximos catorze anos.

Esse período que Maquiavel passou na Chancelaria foi posto em segundo
plano pelos especialistas por muito tempo. Contudo, mais recentemente,
esse aspecto biográfico passou a ser alvo das atenções, principalmente
em função do verdadeiro alcance dessa experiência diplomática sobre sua
reflexão política. Segundo Guidi, o ingresso na Chancelaria Florentina
ao final do século \versal{XV} era feito por indicação do chefe político -- o
\emph{gonfaloniere} --, indicação essa que deveria ser submetida à
aprovação dos Conselhos políticos superiores (\versal{GUIDI}, 2009). Após a queda
da família Medici, em dezembro de 1494, o novo governo do frei Jerônimo
Savonarola busca reestruturar o regime florentino, criando mecanismos
republicanos de caráter mais popular, particularmente com a reabertura e
o fortalecimento do Conselho Maior (\emph{Consiglio Maggiore}), que era
a instância política na qual todos os cidadãos podiam tomar parte e não
era controlada exclusivamente pela oligarquia florentina. Outra medida
do governo de Savonarola foi reestruturar toda a Chancelaria, trocando
os funcionários e reordenando as incumbências.

Segundo Tafuro (2004, p. 25ss), o mundo político da cidade de Florença
no final do século \versal{XV} era caracterizado por divisões de caráter
econômico e social que estavam na raiz dos grupos políticos, a saber: os
grandes aristocratas, os médios comerciantes, os pequenos comerciantes e
os demais trabalhadores -- esses, em geral, quase sem expressão
política. Maquiavel nomeará esses grupos ao longo de sua \emph{História
de Florença}, respectivamente, como nobres, povo e plebe. A aristocracia
florentina (os nobres, segundo Maquiavel) estava dividida em dois
grandes partidos, desde as lutas políticas do século \versal{XIV} entre os
partidários do papa -- o partido \emph{guelfo} -- e os partidários do
imperador\footnote{Os territórios do centro-norte da península itálica
  foram motivos de disputas entre os séculos \versal{XI} a \versal{XIV} pelo papado e pelo
  Sacro Império Romano-Germânico. Como mostra Skinner (2000, cap. 1),
  nesse período, muitas cidades travaram lutas externas contra esses
  dois domínios em busca de sua independência e também lutas internas
  contra a dominação de um desses grupos ou contra o domínio de tiranos
  locais. A tese aceita pelos intérpretes é que, destas lutas, nasce o
  germe do republicanismo típico do Renascimento italiano, no qual as
  cidades buscaram se organizar como repúblicas independentes e
  autônomas.} -- o partido \emph{ghibelino}. A vitória do partido
\emph{guelfo} em 1289 na sua luta contra os \emph{ghibelinos} instalou
uma estreita relação entre essa parcela da aristocracia e o papado, que
perdurará durante séculos e que terá como fruto a forte influência dos
florentinos nas decisões de Roma, bem como a nomeação de alguns cardeais
florentinos ao trono papal: Leão \versal{X} (Giovanni di Lorenzo de Medici --
papa de 1513 a 1521), Clemente \versal{VII} (Giulio di Guiliano de Medici -- papa
de 1523 a 1534), Clemente \versal{VIII} (Ippolito Aldobrandini, papa de 1592 a
1605), Leão \versal{VI} (Alessandro Ottaviano de Medici, papa por 26 dias, de 1 a
27 de abril de 1605), Urbano \versal{VIII} (Maffeo Barberini, papa de 1623 a
1644) e Clemente \versal{XII} (Lorenzo Corsini, papa de 1730 a 1740).

Apesar dessa importante força política desempenhada por parte da
aristocracia do partido \emph{guelfo}, uma outra parcela também rica, a
aristocracia \emph{ghibelina}, será muito influente e marcante na
cidade, por exemplo, na constante oposição que a família Medici recebeu
enquanto esteve no poder.

Abaixo desses dois grupos políticos aquinhoados, Maquiavel designa as
demais parcelas como ``povo'', embora não se assemelhe muito à nossa
compreensão contemporânea do termo. Essa parcela social e política teve
sua origem nas corporações de ofícios ou \emph{artes}, que eram
associações dos diversos grupos profissionais da cidade que controlavam
o exercício da respectiva atividade profissional. Segundo Tafuro (2004,
p. 27-28), um desses grupos políticos mais abastados, não
necessariamente aristocratas em sua denominação, configuram uma parcela
política dita popular, que ganha riqueza e prestígio e passa a figurar
como um grupo político de destaque. Nesse caso, formam, já no século \versal{XV},
um agrupamento de caráter aristocrático do ponto de vista econômico e
intelectual, mas não tradicional, por não ter ligação com as antigas
aristocracias rurais e urbanas.

Ainda no interior das \emph{Arti} ou povo, segundo a designação
maquiaveliana, verifica-se uma divisão interna entre aquilo que ele
nomeará como \emph{popolo grasso} (povo gordo) e \emph{popolo minuto}
(povo magro), ou seja, entre os setores médios mais aquinhoados e
aqueles com menos poder econômico e, portanto, menor influência
política.

Verifica-se, pois, que o povo (\emph{popolo}), para Maquiavel, é uma
denominação para diferenciar parcelas sociais com riqueza de outras
também com riqueza mediana, mas de origem tradicional. Logo, são
tipificações no interior de grupos caracteristicamente aristocráticos.
Na definição de Tafuro: ``Povo era, em particular, a parcela média dos
artesãos e dos comerciantes reunidos nas associações profissionais''
(2004, p. 40). Ora, quando Maquiavel chama a atenção para a divisão
entre os nobres e o povo, trata-se de uma divisão entre os setores
aristocráticos tradicionais e a classe social que emergiu econômica e
socialmente com o exercício do comércio e do artesanato a partir do
século \versal{XIII}.

Por fim, temos a plebe, que eram os operários ou os assalariados, que
não possuíam qualquer associação ou grupo político no qual pudesse
expressar seus interesses, até a Revolta dos Ciompi, de 20 de julho de
1378, quando eles passam a ter direitos políticos e associativos
(\versal{TAFURO}, 2004, p. 33-34). Sintomática é a importância que Maquiavel
confere a essa revolta política de final do século \versal{XIV}, que marca uma
mudança paradigmática na vida política florentina, conforme ele narra
com cores dramáticas nas \emph{Histórias Florentinas,} livro \versal{III}. A seu
ver, essa revolta política ocupa um papel de destaque na história da
dinâmica política da cidade, visto que implicou não somente uma mudança
institucional, mas, principalmente a partir da visada teórica
maquiaveliana, na dinâmica política e nas correlações de força entre os
grupos políticos.

Nunca é demais lembrar que, apesar da constatação da existência desses
diversos grupos políticos, por outro lado, inúmeras pessoas estavam
alijadas das instâncias decisórias, como os trabalhadores rurais, as
mulheres e os não nascidos em Florença, que, conforme nos relata Gilbert
(1996, p. 25), de uma população estimada em 62.000 em 1494, menos de
15.000 homens possuíam cidadania e direitos políticos.

Nota-se, então, que Florença entre os séculos \versal{XIV} e \versal{XVI} estava dividida
em vários grupos sociais que, por consequência, dividiam-se em grupos
políticos em disputa pelo comando do governo. Esse quadro de divisão e
disputa política entre vários partidos é ressaltada por Maquiavel nos
seus diversos escritos e é uma das marcas de seu pensamento político,
haja vista que é sob essa condição de instabilidade e disputa que se
monta o palco da ação política.

Tal percepção da importância de lidar com os interesses desses diversos
grupos no mundo político Maquiavel experimenta desde o momento de seu
ingresso na Chancelaria e o atingirá ao longo de toda a sua vida
doravante. Com a queda do governo da família Medici no final de 1494 e a
instauração do governo de Savonarola, a administração central passa a
ser controlada pelos partidários deste. Nesse período, sabe-se que
Maquiavel apenas realizou trabalhos avulsos, com os quais travou já
algum contato com o ambiente da Chancelaria, mas que não chegou a
ingressar nela, pois a admissão se fazia por meio de uma indicação do
\emph{gonfaloniere} e era, posteriormente, submetida à votação no
conselho ao qual o cargo se vinculava. Ora, Maquiavel não somente não
teria a indicação de Savonarola como os partidários desse nos diversos
Conselhos da cidade não aceitariam o seu nome, como ocorreu
primeiramente em 18 de fevereiro de 1498, quando seu nome foi recusado
pela primeira vez. Apenas com a queda do governo de Savonarola é que o
nome de Maquiavel é indicado para o cargo de segundo secretário e
admitido em 23 de maio de 1498.

É importante destacar que a Chancelaria não era apenas uma repartição da
burocracia voltada exclusivamente para as relações diplomáticas, mas o
órgão central da administração da república florentina, encarregada das
relações exteriores, da guerra, da arrecadação de tributos e de
assessoramento das instâncias judiciárias. Portanto, ingressar na
Chancelaria não implicava tão somente fazer parte do órgão responsável
pelas relações diplomáticas de Florença, mas trabalhar no coração
político e burocrático da cidade (\versal{GUIDI}, 2009, cap. 1).

O fato em si da indicação para um cargo elevado revela que Maquiavel já
era conhecido do grupo político que assume o poder em 1498, bem como que
ele já conhecia a rotina burocrática. De fato, entre 1494 e 1498,
Maquiavel foi contratado para fazer pequenos serviços para a
Chancelaria, o que foi lhe dando conhecimento sobre as rotinas
administrativas. Então, sua escolha em 15 de junho de 1498 para a função
de secretário da Segunda Chancelaria é fruto de sua formação humanista,
seu conhecimento, ainda que parcial, das rotinas da administração e,
fato apontado como principal pelos estudiosos, sua ligação política com
o grupo que assume o poder em 1498, que eram setores do \emph{povo}, no
caso, os adversários dos Medici e de Savonarola; ou seja, quem apoiou a
indicação de Maquiavel foram em sua maioria os partidários das
\emph{Arti}, especificamente os partidários das \emph{Arti Maggiori}.
Parte das tarefas deste cargo ocupado por Maquiavel implicava assessorar
os ``Dez da Bailia'', que era um órgão encarregado da política externa
e, eventualmente, da guerra e das questões militares (\versal{TAFURO}, 2004, p.
121-122).

A questão que ainda incomoda os estudiosos da obra de Maquiavel diz
respeito a sua atuação política no período em que esteve trabalhando na
administração da república florentina. Por um lado, com base em seus
escritos, Maquiavel apresenta a imagem de um modelo de funcionário
público devotado à causa da cidade e não aos interesses dos grupos no
comando do governo, uma espécie de funcionário de carreira sem ligação
política ou ideológica que estivesse a serviço da cidade\emph{.} Por
outro lado, até mesmo pela sua própria história, Maquiavel foi sempre
identificado a um grupo político, ora como partidário dos republicanos
na visão da oligarquia florentina, ora como partidário dos Medicis,
segundo o grupo republicano por ocasião da restauração da república em
1527. A questão está em saber até que ponto Maquiavel foi de fato um
funcionário, por assim dizer, isento de ligações políticas ou se ele foi
um mentor e articulador do governo de Pier Soderini, ou, até mesmo, se
seria possível conciliar as duas coisas. As pesquisas recentes sobre o
período em que Maquiavel esteve na Chancelaria mostram, por vários
elementos, que ele não somente foi um membro ativo do grupo que esteve
no poder em Florença entre 1498 e 1512, como ocupou um papel destacado
no governo. Segundo Guidi, o ingresso de Maquiavel já em um cargo
elevado na burocracia, a sua participação ativa na reformulação do
governo em 1502, os encargos diplomáticos e militares que ocupou depois
até a queda do governo em 1512 (a demissão oficial de Maquiavel foi em
10 de novembro de 1512), revelam que ele ingressou na Chancelaria também
por seus vínculos políticos e foi, ao longo dos anos, galgando cada vez
mais espaço e importância no governo de Soderini, ao ponto de ser o
maior responsável pela parte militar do governo e pela condução das
guerras após 1506 (\versal{GUIDI}, 2009).

A vinculação aos interesses do governo de Soderini e as suas qualidades
de negociador e analista do contexto político ficarão associadas a
Maquiavel para sempre, pois, mesmo depois de sua saída do governo,
quando foi requisitado, mesmo entre os seus adversários, sempre se nota
esse duplo aspecto: um partidário da causa republicana e um hábil
analista político e militar. Aspecto esse reforçado pelo modo como ele
descreve sem paixão o universo da política em geral e o seu mundo
florentino em particular, no qual os fatores políticos que implicam de
fato as decisões são expostos friamente, sem meandros ou meias palavras.
Enfim, podemos inferir que Maquiavel foi sim um membro leal do grupo de
Solderini, mas isso não o impediu de ser um funcionário a serviço da
cidade que avaliava sem paixão o mundo político, como demonstram a
exaustão suas correspondências diplomáticas e os seus textos políticos.

Embora tenha iniciado no serviço público com 29 anos, uma idade
considerável para a época, e com uma relativa formação humanista,
Maquiavel declara em vários textos que foram esses anos de convivência
cotidiana com o mundo da política que lhe deram uma boa parte de sua
formação política -- a outra parte ele diz ter recebido dos clássicos.
Em seu dia a dia, ele acompanhava, por um lado, o funcionamento das
decisões políticas em Florença, ou seja, observava por dentro os
mecanismos das negociações políticas, como elas eram feitas, os jogos de
intenções e de promessas, e, por outro lado, como se realizavam as
negociações diplomáticas, como os reis, papas, príncipes, comandantes
militares, governantes das republicas negociavam e estabeleciam pactos,
guerras, ou resolviam os conflitos comerciais. Esse conhecimento da vida
política por dentro, aliado a um olhar treinado a não enxergar somente
os aspectos circunstanciais e pessoais do mundo político, mas a
desvendar as reais motivações dos atores e os fatores que regulam o agir
político, deram a Maquiavel uma parte do conhecimento necessário para
entender o universo da política geral, de modo ampliado.

Convém informar que, desde os séculos \versal{XII} e \versal{XIII}, ocorre uma mudança
radical na concepção e organização da vida diplomática e das
chancelarias nas repúblicas italianas (\versal{FUBINI}, 1994). Seja em função das
disputas que deveriam travar contra as grandes potências (não somente o
papado e o Império Germânico, mas também, nos séculos seguintes, contra
os franceses e espanhóis, sem contar a sempre constante ameaça turca que
se amplia após a queda de Constantinopla em 1453), seja em função de seu
pouco poder militar, essas repúblicas reconheceram que os seus quadros
diplomáticos seriam uma grande arma para a defesa de sua condição de
repúblicas livres. A mudança se nota inicialmente na própria missão
delas: não mais defender tão somente os interesses privados do poderoso
no governo, o que implicaria uma prática voltada para os interesses
gerais e não mais como prepostos mercantis. As chancelarias das
repúblicas italianas do Renascimento se diferenciam das medievais, na
medida em que estão mais voltadas a trabalhar pela defesa dos interesses
gerais da cidade, principalmente a defesa da condição de liberdade
política destas repúblicas, e não somente em fazer acordos vantajosos
para as famílias poderosas (\versal{GUIDI}, 2009, p. 35).

Outro aspecto das práticas dessas chancelarias do Renascimento é sua
preocupação em ter um maior controle da vida política e jurídica da
cidade, tornando-se verdadeiras burocracias de controle das práticas
públicas. Neste sentido, adota-se o costume de registrar todas as
reuniões dos conselhos em atas, de catalogar as missões, de documentar
os discursos e exigir de seus emissários a notificação do andamento das
ações em suas missões. Essa massa de material permite ao governo ter
maior controle das decisões e perceber melhor as rotinas e os
encaminhamentos das diversas ações. Segundo Guidi, os chanceleres e os
secretários ``configuram-se como técnicos da administração a serviço do
executivo'' (\versal{GUIDI}, 2009, p. 39-40).

Por isso que fazer parte da Chancelaria florentina não significava tão
somente fazer parte do órgão responsável pelas relações internacionais,
o que por si mesmo já implicava um extraordinário conhecimento do mundo
das negociações políticas. Tendo em vista essa dimensão de controle
burocrático das ações de governo, seus funcionários tinham acesso
privilegiado, por meio da documentação, aos diversos elementos em jogo
na política interna e externa da cidade. Essa documentação torna público
e acessível aquilo que antes eram informações restritas e privilegiadas
para se compreender os reais motivos das tomadas de decisão política.
Portanto, quando Maquiavel diz que aprendeu muito com a vida prática na
Chancelaria, isso não significa que ele esteve presente em todas as
decisões e conhecia muito bem os diversos interesses em jogo (visto que
seria impossível ele estar, por exemplo, fora da cidade e saber dos
detalhes de uma negociação ou jogada política), mas que, por meio dessa
documentação pública, ele pode conhecer o mundo da política. Essa
documentação oficial, também conhecida como \emph{pratiche} (\versal{GILBERT},
1964), relatam as práticas, os diversos procedimentos oficiais, e são,
para um leitor atento, uma fonte preciosa de informações sobre a
dinâmica da vida política. Convém insistir: quando Maquiavel declara,
por várias vezes, que boa parte de sua formação política adveio das
experiências das coisas modernas, da sua convivência com o mundo da
política, inclua-se nesse rol o conhecimento dessa farta e rica
documentação oficial, à qual ele teve acesso por sua condição de Segundo
secretário\footnote{A hierarquia nas chancelarias eram: chanceler,
  secretários e cartorários ou oficiais de chancelaria.}, sendo, pois, o
responsável por produzir e organizar esses documentos\footnote{Não é sem
  razão que as pesquisas sobre o pensamento político de Maquiavel estão
  se voltando cada vez mais para essa documentação produzida no período
  republicano, mais especificamente para os documentos oficiais do seu
  período na Chancelaria, de 1498 a 1512.}. Enfim, as \emph{pratiche}
destacadas por Maquiavel não foram somente as ações políticas, as
práticas políticas, mas também e fundamentalmente a documentação oficial
da República florentina, algo que, ainda que não fosse inédito para
muitos pensadores, foi considerada e analisada de modo especial pelo
Secretário Florentino, configurando-se, enfim, como uma novidade no
mundo do Renascimento.

A outra parte do conhecimento, conforme ele mesmo declara, adquiriu na
leitura dos clássicos. Para compreender melhor esse aspecto, não é
suficiente saber que Maquiavel teve uma formação humanista, mas também
considerar o contexto cultural no qual ele esteve inserido. Conforme
Garin, a Chancelaria de Florença tinha se transformado, desde o final do
século \versal{XIV}, não somente no centro político da cidade, mas no seu centro
intelectual, em função das figuras de destacada importância que
estiveram a serviço da cidade como Coluccio Salutati, Poggio
Bracciollini, Leornardo Bruni, Bartolomeu della Scalla, Lorenzo Valla,
Francesco Guicciardini, fazendo dela uma verdadeira escola do pensamento
político (\versal{GARIN}, 1996). Esses intelectuais marcaram o pensamento
político do Renascimento, entre outros fatores, mas principalmente, por
terem que fazer uma defesa contundente do regime republicano florentino
enquanto ocupavam o cargo de chanceleres. Desde o final do século \versal{XIII}
até o início do século \versal{XVI}, Florença frequentemente se via envolvida em
ameaças de dominação por algum poder exterior ou em disputa contra
tiranos locais. Nessas ocasiões, sempre era necessário mobilizar os
cidadãos para a defesa da liberdade política, que implicava, em última
instância, a defesa do regime republicano. Portanto, diante da ameaça de
dominação, o chanceler do momento comandava uma luta no campo ideológico
e teórico em defesa da liberdade republicana. Ora, esse acúmulo de
reflexão política foi se consolidando ao longo do tempo entre os
ocupantes de cargo, a ponto de se constituir, como interpreta Garin,
numa ``escola'' de pensamento político republicano. Então, para os
postulantes a cargos na Chancelaria, já estava claro, de antemão, que
eles estavam se vinculando a uma instituição que teve e ainda tinha por
missão a defesa da liberdade política e da autonomia da cidade,
encarnada no seu regime republicano. Isso se reflete decisivamente nos
escritos de Maquiavel, nos quais a problemática da liberdade política
perpassa sua reflexão política como um todo (\versal{BIGNOTTO}, 1991).

Associa-se a isso o intenso debate político sobre a melhor forma de
ordenar a república florentina no interior da crise política de final do
século \versal{XV} gerado pela queda dos Medici, pelo curto e conturbado governo
de Savonarola e início do novo governo de Soderini (\versal{BARON}, 1989;
\versal{GILBERT}, 1996; \versal{RUBINSTEIN}, 1998). Ora, num período de mais de 20 anos
(entre 1492 e 1513), Florença passa por quatro governos: o de Piero de
Medici (1492-1494), o de Jerônimo Savonarola (1494-1498), de Pier
Soderini (1498-1512) e doravante novamente com a família Medici até
1527. Essas mudanças na vida política da cidade são acompanhadas de
intensos embates intelectuais, nos quais se podem reconhecer vários
grupos em defesa de seus projetos, mas que podem ser agrupados, em
linhas gerais, em dois grandes partidos, a saber: os defensores de um
modelos republicano de caráter mais aristocrático e os defensores de um
regime republicano de caráter mais popular, com a ampliação da
participação dos diversos segmentos políticos da cidade (\versal{MARTINS}, 2010).

Sobre esse tema, convém recuperar a disputa entre os modelos
republicanos nesse momento histórico na cidade de Florença.

\subsubsection{O contexto maquiaveliano: entre o republicanismo popular~e~o~republicanismo~aristocrático}

Os séculos \versal{XV} e \versal{XVI} foram marcados por intensos debates nas cidades
italianas sobre a melhor forma de regime republicano a ser adotada.
Neste período, vários pensadores se preocuparam em definir modelos de
repúblicas que atendessem às demandas de sua época, consagrando duas
formulações: um modelo de república de caráter mais popular,
caracterizado pelos governos de Savonarola e Soderini em Florença, e um
modelo de república de caráter aristocrático, tendo o regime republicano
da cidade de Veneza como o grande exemplo. É em Florença que esse debate
ganha mais vigor e corpo, fazendo dela um dos centros de produção
intelectual sobre o regime republicano (\versal{GARIN}, 1996). As mudanças
políticas florentinas do final do século \versal{XV} e início do século \versal{XVI}
suscitaram intensos debates sobre os destinos da cidade, num primeiro
momento, e sobre a natureza das repúblicas, num segundo momento. Nesse
contexto de alteração constitucional, a discussão dos fundamentos da
república florentina adquire força nos círculos intelectuais. Dentre as
várias posições assumidas, a defesa da instalação de um regime
republicano inspirado no modelo veneziano foi predominante entre a
aristocracia florentina, grupo político que identificava nos governos
republicanos de Savonarola e Soderini o predomínio dos segmentos
populares, governos esses considerados como ``demasiadamente''
democráticos\footnote{Certamente, caracterizar os governos republicanos
  de Florença que vão de 1494 a 1512 como democráticos é algo
  problemático devido ao poder que a aristocracia deteve nesse período.
  Qualquer afirmação mais contundente no sentido da definição do tipo de
  governo existente em Florença durante esses 18 anos é passível de
  discussão. Sobre a história do período cf. Tenenti (1973) e Tafuro
  (2004, parte \versal{I}).}. Entre os exemplos favoritos da aristocracia para
justificar sua opção política estavam a Roma republicana, a Esparta
concebida por Licurgo e a república veneziana de então. Entretanto, como
afirma Gilbert, o exemplo veneziano era o que mais se destacava: os
aristocratas em particular, ansiosos em limitar o poder do
\emph{Conselho Maior} (no qual todos os cidadãos tinham direito de
participar), colocavam em evidência que, em Veneza, os cidadão discretos
e sábios tinham as possibilidade melhores e mais apropriadas para o
exercício do poder (\versal{GILBERT}, 1977, p. 102-103). A opção pelo modelo
veneziano se deve, principalmente, ao predomínio e controle que a
aristocracia mercantil exercia sobre o governo. Em Florença, o governo
de Savonarola, bem como em certa medida o de Soderini, eram, aos olhos
da aristocracia, muito democráticos, pois neles os poderes decisórios de
seu extrato político estavam limitados pelas forças políticas populares.

É, portanto, no interior dessa luta pela retomada do comando político da
cidade, liderada pela aristocracia, que nasce aquilo que Pocock nomeia
como o ``mito de Veneza'' (\versal{POCOCK}, 1980). Veneza servia como modelo
porque conseguia reunir diversas qualidades almejadas pela aristocracia
florentina, transformando-se num ideal de convivência cívica. A
estabilidade política e a liberdade, bem como a existência de um governo
misto e a virtuosidade de seus gentis-homens, eram entendidas como as
causas principais para a riqueza da república do Norte. O governo
comandado pelo Doge (chefe do executivo de caráter vitalício) e seus
Conselhos (compostos quase que exclusivamente pela aristocracia) seriam
a realização do regime misto idealizado pelos filósofos. Nos escritos de
venezianos da segunda metade do século \versal{XV}, como Francesco Barbaro,
Giorgio da Trebisonda e Bernardo Bembo, Veneza correspondia, até nos
detalhes, à república proposta por Platão, principalmente por conter em
si as três formas de governos particulares ou simples. Ademais, essa
defesa do regime misto, que não está somente nos textos platônicos, mas
também em Aristóteles, Políbio e Cícero, levou esses escritores a
afirmar que Veneza era a realização do modelo clássico de república
ideal (\versal{GILBERT}, 1977).

Na visão da aristocracia florentina, o principal resultado alcançado por
esse governo misto era a ausência de conflitos políticos num ambiente de
grande liberdade cívica, entendida num duplo sentido: como a existência
de um governo não tirânico e não estarem submetidos a outra cidade
(\versal{GILBERT}, 1977). As narrativas que chegavam a Florença sobre a república
veneziana relatavam que ela havia sido instalada há muito tempo e não se
tinha notícia da ocorrência de conjurações ou tumultos políticos que
ameaçassem sua normalidade republicana, fruto, também, da grande
\emph{virtù} de seus cidadãos. Essa fama de Veneza como república
pacífica lhe rendeu a alcunha de ``república sereníssima''.

Tal imagem modelar de Veneza revelou-se um mito na medida em que os
próprios humanistas começaram conhecer melhor a real estruturação do
regime republicano que lá vigorava. Com melhores informações sobre o
funcionamento da república veneziana, descobre-se que se tratava de um
governo tipicamente oligárquico, pois era dominado por um Conselho
estritamente limitado e controlado por um número pequeno de famílias.
Como diz Gilbert (1977), poucas pessoas em Florença conheciam como
realmente se ordenava o regime veneziano. A admiração estava fundada
mais nas narrativas e imagens projetas da cidade do que na realidade
política.

Seja como for, a imagem da república veneziana passou a exercer
relevante influência em Florença antes mesmo da instalação do governo de
Savonarola. Quando do nascimento do governo republicano, em dezembro de
1494, uma das principais inovações do novo regime foi a instauração do
\emph{Conselho Maior}, à semelhança do Grande Conselho de Veneza, com
ampla participação dos vários grupos sociais\footnote{Compreender o
  intrincado funcionamento do regime republicano de Florença é uma
  tarefa difícil. Gilbert nos informa que havia neste período
  aproximadamente 3.300 cargos eletivos, para uma população de não mais
  de 60.000 pessoas. Em termos proporcionais significava dizer que,
  dentre a população masculina com direito a voto, entre 1/4 ou 1/5
  participavam de algum cargo eletivo, o que é significativo em termos
  de participação popular (\versal{GILBERT}, 1996, p. 25, nota 2).}. Neste
sentido, o que era apenas um instrumento de fachada no ordenamento
político veneziano, em Florença, sob o governo republicano, passa a
funcionar de fato\footnote{Notório saber que, quando da restauração do
  governo dos Medici em 1512, um de seus primeiros atos foi a demolição
  do salão onde funcionava o Grande Conselho.}. Apesar dessa modificação
constitucional importante, novas demandas se faziam sentir, levando à
continuidade do debate sobre a melhor forma de governo. Em todos esses
momentos de confronto político por reformas nas instituições
republicanas da cidade, o exemplo veneziano sempre voltava à baila,
tanto que, na reforma de 1502, tem-se a instituição de um
\emph{gonfaloniere a vita}, ou seja, a versão florentina para o
\emph{Doge} veneziano, esse o chefe do executivo.

No \emph{Príncipe} e em vários capítulos dos \emph{Discursos sobre a
primeira década de Tito Lívio}\footnote{Doravante somente citado como
  \emph{Discursos}.}, Maquiavel apresentará afirmações contrárias às
posições teórico-políticas aristocratas. No caso de sua análise sobre a
\emph{História de Roma} de Tito Lívio, aquela discordância ganha uma
característica especial, pois Maquiavel se propôs a tomar como
referência de reflexão a mesma obra sobre a qual o aristocrata Bernardo
Rucellai, cunhado de Giovanni de Medici (que havia governado Florença
até 1492) já havia tecido seus comentários. Partindo dos mesmos métodos
analíticos que Rucellai, Maquiavel retira da história romana conclusões
opostas às dele. Conclusões não somente desfavoráveis à aristocracia,
mas aos ideais e modelos propostos por ela. Como afirma:

\begin{quote}
E além disso, levantar-me-ei contra as a opinião de muitos, segundo a
qual Roma foi uma república tumultuária {[}\ldots{}{]}. Direi que quem
condena os tumultos entre os nobres e a plebe parece censurar as coisas
que foram a causa primeira da liberdade de Roma e considerar mais as
assuadas e a grita que tais tumultos nasciam do que os bons efeitos que
eles geravam (\emph{Discursos}, \versal{I}, \versal{IV}).
\end{quote}

Na sequência, ele insiste no elogio aos conflitos:

\begin{quote}
E não se pode ter razão para chamar de não ordenada uma república
dessas, onde há tantos exemplos de \emph{virtù}; porque os bons exemplos
nascem da boa educação; a boa educação, das boas leis; e as boas leis,
dos tumultos que muitos condenam sem ponderar (\emph{Discursos}, \versal{I},
\versal{IV}).
\end{quote}

Os conflitos políticos são mobilizados num momento do texto no qual
Maquiavel busca uma outra fonte ou origem para os bons ordenamentos,
após ter mostrado que o modelo histórico e determinista polibiano sobre
os destinos dos regimes políticos, também conhecido como a teoria da
\emph{anacyclosis}, não dava mais conta de explicar as mudanças
políticas e a insuficiência do legislador em bem ordenar a cidade
(\emph{Discursos}, \versal{I}, \versal{II}). Os conflitos políticos se apresentam nos
capítulos \versal{III} e \versal{IV} dos \emph{Discursos} como a melhor alternativa para a
fundação de ordenamentos em cidades que não tiveram a sorte de ter um
sábio legislador, como foi Licurgo para Esparta. No caso de Roma, essa
ausência foi suprida ao acaso pelos conflitos políticos. Mais do que
isso, para Maquiavel, a experiência romana mostrou que, dessa maneira, a
fundação das cidades seria mais segura, haja vista que resultaria da
luta dos dois humores ou grupos sociais presentes em todas as cidades:
os grandes (\emph{popolo grasso}) e os pequenos (\emph{popolo minuto}).

Como veremos adiante, Maquiavel demonstrará a maior adequação do regime
republicano calcado nos setores populares contra a posição teórica de um
regime republicano de caráter aristocrático. Assim, se a aristocracia
florentina admirava o regime veneziano, sua estabilidade política, sua
natureza e seu virtuosismo aristocrático, Maquiavel verá nesses mesmos
aspectos fraqueza e enxergará a virtude nos conflitos políticos, na
instabilidade dos regimes republicanos. Ao contrário de pensar no bom
regime como uma república de tipo aristocrático, ele destacará as
qualidades populares das repúblicas. Será justamente na parcela popular
do governo republicano de Roma, e não nos seus quadros aristocráticos,
que esta encontrara sua força, seu vigor e sua grandeza.

Se, para um autor como Leonardo Bruni (séc. \versal{XV}), representante daquilo
que ficou conhecido como o \emph{humanismo cívico} (ou seja, a expressão
do pensamento político do Renascimento italiano), a exaltação de Roma,
acompanhada da afirmação de que Florença era sua filha, implicava a
defesa da liberdade política, um louvor às suas origens e de sua
excelência virtuosa, em Maquiavel cessa o tempo da apologia e começa o
tempo da crítica. Ao pensar em Roma como o modelo que inspiraria
Florença, ele ressaltará os contrastes ao invés dos paralelos, as
diferenças ao invés das semelhanças. Para o Secretário florentino, se a
república de Florença tivesse um ordenamento político assentado mais
sobre o povo e menos sobre a aristocracia, os destinos de sua cidade
poderiam ter sido outros. O que antes se colocava como uma ampla crítica
à aristocracia, agora se restringe à aristocracia florentina, que, nesse
aspecto, foi pior para os destinos da cidade do que a aristocracia
romana. Essa acusação atinge o cerne da ideologia dos aristocratas: a de
que Florença estava revivendo a \emph{virtus civita} da Roma
republicana.

Por outro lado, para os republicanos aristocráticos florentinos, os
conflitos políticos presentes na república romana seriam sinais da
corrupção política e, portanto, da perda da virtude cívica. Com efeito,
é somente numa certa compreensão da \emph{virtus civita}, assentada na
força e no poder romano, ou seja, numa adequação aos ideais humanistas,
que se poderia pensar na virtude como fundamento político (\versal{SKINNER}, 2000
e 2006). Para eles, a corrupção política romana começa quando se
manifestam os tumultos políticos, quando a unidade política da cidade se
vê fraturada pelas contendas entre os grupos. Não é sem fundamento que
vários pensadores, tanto antigos quanto modernos, entenderam a crise e a
decadência das repúblicas como associadas à perda da \emph{virtus
civita}, manifesta pelo conflito, índice maior da corrupção na cidade.

Como apontaram Gilbert (1996), Tafuro (2004, 2005) e Bignotto (1991), o
modelo republicano defendido pela aristocracia florentina representava
uma defesa das posições políticas desse grupo. Ao insistir que as
principais esferas decisórias (os Conselhos superiores da república
florentina) deveriam ser compostas majoritariamente ou totalmente (como
foi o caso do Conselho de Justiça) por membros da aristocracia; ao
defender que um ordenamento político assentando nesse extrato garantiria
a paz e a estabilidade política; e, principalmente, ao sustentar que
essa proeminência da aristocracia devia-se à sua virtuosidade, ao seu
amor e dedicação à pátria, estavam eles construindo uma teoria
republicana de caráter aristocrático. De fato, é nessa qualidade cívica
superior, que se expressa pela \emph{virtù} dos gentis-homens, que se
justifica o destaque dos segmentos aristocratas na vida política de
Florença\footnote{Nesse sentido, discordamos de Araújo (2000, p. 15),
  quando diz que para ``Maquiavel a instabilidade política é
  indesejável''. Ademais, se não se pode falar em toda a tradição dos
  autores republicanos, ao menos em Maquiavel a ampliação do poder
  político, inserindo novos atores para além da aristocracia ou dos
  homens dotados de virtude, ou seja, os segmentos mais populares, é não
  somente desejável, como é a resposta para o fantasma da corrupção
  política.}. Ideais esses que não se mostravam como uma novidade
teórica, haja vista a semelhança deles com algumas formulações de regime
misto no qual a aristocracia é considerada o segmento político mais
relevante na ordenação, como se pode perceber em Platão, Políbio e nos
primeiros escritos de Cícero (\versal{LEPORE}, 1954).

Neste contexto político e cultural, a posição de Maquiavel é
privilegiada, pois não somente ocupou um cargo estratégico (o comando de
uma importante secção da vida política da cidade e das relações
políticas entre os Estados), como pode conhecer a fundo o funcionamento
dessas engrenagens, e também participar desse debate político como um
interlocutor relevante.

\subsubsection{A saída da Chancelaria e o final da vida}

Depois que o governo Soderini cai e Maquiavel é demitido de suas
funções, em setembro de 1512, ele passa a produzir a parte mais
significativa de suas obras, entre as quais: \emph{O Príncipe} (em
1513), os \emph{Discursos sobre a Primeira década de Tito Lívio}
(1515-1517), a \emph{História de Florença} (1520-1525), a \emph{Arte da
Guerra} (1519- 1521) e os opúsculos políticos.

Entretanto, a saída de Maquiavel da Chancelaria é um fato conturbado e
nebuloso, mas que, uma vez compreendido, ajuda a entender a inserção
dele no ambiente político e intelectual florentino. Entretanto, convém
recuar um pouco no tempo.

Em 1502, por pressão da rica aristocracia florentina (os \emph{nobres}
segundo Maquiavel), ocorre uma mudança constitucional com a instauração
do \emph{gonfaloniere a vita}, ou seja, o comandante político da cidade
passa a ser um cargo vitalício, como era o \emph{doge} de Veneza. Isso
implica um ganho de poder por parte do \emph{gonfaloniere} Pier
Soderini, que consolida sua força política perante os outros setores
aristocráticos, bem como mantém seu grande apoio popular. A partir desta
data, conforme Guidi (2009), Maquiavel passa a ocupar uma posição
central na vida política florentina. Mesmo não sendo de família
aristocrática, o que o impedia formalmente de ser um embaixador ou
chanceler, ele fica responsável por diversas negociações importantes,
sendo que os embaixadores florentinos eram chamados ao final apenas para
assinar os acordos e os tratados.

Após 1502, Maquiavel também vai se envolvendo cada vez mais com as
questões militares da cidade, escrevendo e estudando sobre o assunto, a
ponto de, em 1506, ser o responsável pela criação da milícia florentina,
algo até então inexistente, pois a prática era a contratação de
exércitos ou tropas mercenárias para fazer as guerras.

A queda do governo Soderini começa quando as tropas espanholas invadem a
Toscana e tomam a cidade de Prato (vizinha de Florença) em agosto de
1512. A derrota retumbante das tropas florentinas perante a força dos
exércitos espanhóis desmoraliza o governo de Soderini que, em 31 de
agosto de 1512, foge para a cidade de Siena e depois segue para o exílio
na Dalmácia, onde é hoje a Croácia.

Com a queda de Soderini, quem assume o poder é inicialmente seu
adversário Giovan Battista Ridolfi, que inicia a reforma do governo mas
que é substituído, alguns meses depois, por Guiliano de Medici, que em
breve seria eleito papa (\versal{MARTELLI}, 2006, p. 9-11).

Maquiavel é demitido dois meses depois da derrota de Prato e logo em
seguida é preso sob a acusação de conspirar a morte de Giuliano de
Medici. Maquiavel fica dois meses preso, é torturado e solto por um fato
inusitado: a eleição de Giuliano ao papado em março de 1513. O papa
recém-eleito concede, então, anistia aos presos políticos de Florença. O
problema dessa história toda está em entender por que Maquiavel, que foi
demitido de suas funções, preso e torturado pelos partidários dos
Medici, pensa em dedicar a sua obra ao papa (que era sua intenção
inicial), mas depois dedica ao sobrinho deste, Lorenzo de Medici? Mais
ainda, mesmo sendo identificado ao governo de Soderini, do qual de fato
era o mentor intelectual, mas tendo amigos entre os partidários dos
Medici, haja vista a contínua troca de correspondência e favores após
1512, por que esses e o próprio Giuliano de Medici, que bem conheciam
Maquiavel, não entenderam logo que ele não fazia parte da conspiração,
mas que seu nome foi colocado numa lista de conspiradores sem de fato
ele mesmo saber?

A resposta para esses fatos explica muito do contexto florentino de
então. Como escreverá Maquiavel em um texto dessa época, \emph{Riccordo
di Niccolò Machiavelli ai Palleschi} (ou somente \emph{Ai Palleschi,} do
final de 1512), o problema que envolvia Florença e ele em particular
eram as oposições ferozes de parte da oligarquia florentina ao seu nome.
Nesse texto, escrito no calor dos acontecimentos, Maquiavel faz uma
análise deste contexto político se dirigindo aos \emph{palleschi} ou
\emph{pallesco}, ou seja, como ficaram conhecidos os apoiadores dos
Medici. Nele, o Secretário Florentino mostra que o problema político não
era Soderini e seu governo, pois, se assim o fosse, a sua queda teria
dado à cidade a tranquilidade política esperada; todavia, a
instabilidade permanecia. Curioso notar por este opúsculo que ele não
possuía grandes diferenças ou animosidades com os partidários dos
Medici. Na verdade, o texto indica com precisão que os verdadeiros
adversários de Soderini (parte da camada rica da cidade, ou seja, parte
da oligarquia) não serviria de apoio para o novo regime Medici.
Simplificando, Maquiavel avisa que os seus inimigos (pois essa
oligarquia também era inimiga dele) não seriam fiéis apoiadores do
regime dos Medici.

Portanto, Maquiavel não tinha nos Medici e seus apoiadores inimigos ou
adversários políticos, embora não se possa dizer de nenhum modo que ele
fizesse parte desse grupo. O que permite entender agora porque ele não
culpou os Medici pela sua prisão e dedicou a sua obra para o seu líder,
pois sabia que os seus reais adversários eram parte da aristocracia rica
e não o grupo político dos Medici. Esse pequeno texto maquiaveliano é,
então, muito ilustrativo para explicar o episódio da queda do governo de
Soderini e a prisão de Maquiavel, mas é mais explicativo de passagens de
\emph{O Príncipe}, como se verá melhor adiante, quando ele diz que o
príncipe novo não deve manter sua força política apenas apoiada nos
nobres ou grandes, mas, se tiver que escolher entre os nobres e o povo,
que escolha o povo (\emph{Príncipe,} \versal{IX}). Enfim, desde o ingresso no
governo da cidade, passando pela reforma de 1502 até a sua queda,
Maquiavel teve de fato entre seus maiores adversários políticos as
oligarquias ou \emph{nobreza} locais.

Esse panorama no qual se insere a vida pública de Maquiavel revela, por
seu turno, um vivo ambiente de disputas políticas que, como não poderia
deixar de ser, também foi marcado pelo debate intelectual sobre a melhor
forma de governo republicano para Florença. Este ambiente intelectual é
decisivo para a reflexão política maquiaveliana. Mais adiante,
explicaremos melhor a influência desse contexto sobre a produção
intelectual do \emph{Príncipe}, mas, por ora, podemos dizer que, do
ponto de vista geral de sua reflexão teórica, esse contexto de debate
marca os textos maquiavelianos na medida em que perpassa em todos a
busca pela definição de qual a melhor forma de governo republicano e
qual a natureza deste.

Por fim, convém lembrar que a Itália do período maquiaveliano não era um
Estado unificado sob o controle de um único governo, mas, ao contrário,
um conjunto de territórios independentes com governos autônomos. A
cidade de Florença, que possuía total autonomia e independência
política, constituía-se como um importante centro político, econômico e
cultural, no qual se alternavam governos republicanos, como o do período
em que Maquiavel foi diplomata, e autocráticos, caracterizados pelo
domínio da família Medici.

Quando, em 31 de agosto de 1512, o governo republicano de Pier Soderini
cai e os Medici retornam ao poder em Florença, Maquiavel se vê na
iminência de deixar o cargo na Chancelaria. Com efeito, em novembro do
mesmo ano ele é destituído de seu posto. Não bastasse a perda das suas
funções, que tanto prezava, poucos meses após sua demissão, em fevereiro
de 1513, ele é preso e torturado, sob a falsa acusação de participar de
um complô para assassinar um membro da família Medici. Ao sair do
cárcere, Maquiavel deixa a cidade de Florença e vai viver em sua pequena
propriedade rural, em Sant'Andrea in Percussina. É nesse pequeno
vilarejo, a poucos quilômetros de Florença, que Maquiavel passará o
restante de seus dias.

Não tendo mais obrigações diplomáticas, durante o dia ele se ocupa dos
negócios da propriedade, recolhendo-se à noite em seu escritório para o
estudo e a redação de suas obras. Nesse seu exílio forçado, ele escreve
seus principais textos políticos: \emph{O Príncipe}, \emph{Os Discursos,
A Arte da Guerra} e a \emph{História de Florença.}

Em 1515, Maquiavel passa a frequentar um encontro de jovens aristocratas
nos jardins da família Rucellai, em Florença, encontros esses que
ficaram conhecidos como os encontros dos \emph{Orti Oricellari}. São
desses encontros que nascem a maior parte dos \emph{Discursos}. A partir
dessa época, Maquiavel também volta a assumir alguns encargos
particulares, como quando, em 1519, foi representante de interesses dos
comerciantes de Florença em Lucca. Fora esses e outros poucos trabalhos
avulsos, Maquiavel não desenvolve nenhuma atividade diplomática regular.

Em 1520 ele recebe, por meio do \emph{Studio} florentino e do papa Leão
\versal{X} (Guiliano de Medici), o encargo de escrever uma história da cidade de
Florença. Durante quatro anos ele trabalha nesta obra, concluindo-a em
1525, indo pessoalmente a Roma para presenteá-la ao novo Papa Clemente
\versal{VII} (Giulio de Medici), que a recebe com apreço.

Aparentemente, tudo indicava a volta de Maquiavel às suas funções
diplomáticas, principalmente quando, em de maio de 1527, a família
Medici é deposta e é instaurado um novo governo republicano em Florença.
Mas a fortuna também não vem ao seu auxílio desta vez. Por uma grande
ironia do destino, a acusação que em um primeiro momento lhe fez sair do
governo quando das ascensão dos Medici (ser um defensor do regime
republicano), não se fez presente quando a república foi restaurada,
quinze anos depois. Diante desse revés, a sua saúde não resiste e ele
morre pouco mais de um mês depois, a 21 de junho.

\subsection{As edições}

O texto que ora se apresenta como \emph{O Príncipe} de Nicolau Maquiavel
tem na história de sua elaboração e difusão algumas peculiaridades. A
primeira informação sobre sua confecção vem de uma carta de Maquiavel a
Francisco Vettori de 10 de dezembro de 1513, na qual o autor fala da
composição de um opúsculo intitulado \emph{De Principatibus}, como diz:

\begin{quote}
E, como disse Dante, não pode a ciência daquele que não guardou o que
ouviu -- noto aquilo de que pela sua conversação fiz cabedal e compus um
opúsculo, \emph{De Principatibus}, onde me aprofundo quanto posso nas
cogitações deste tema, debatendo o que é principado, de que espécies
são, como eles se conquistam, como eles se mantêm, por que eles se
perdem. {[}\ldots{}{]} Portanto eu o dedico à magnificência de
Juliano\footnote{Carta a Francisco Vettori, 10 de dezembro de 1513.
  \emph{In}: \versal{MAQUIAVEL}, N. \emph{O Príncipe}. Trad. Lívio
  Xavier. São Paulo: Abril Cultural, 1973. p. 119 (Os Pensadores, \versal{IX}).}.
\end{quote}

Sabe-se, pois que, no final do ano de 1513, Maquiavel já havia terminado
o seu pequeno texto sobre ``que coisa é o principado, de quantas
espécies são, como se conquistam e se conservam''. Ainda nessa
carta, Maquiavel declara que este texto já havia sido lido por outro
amigo, Filippo Casavecchia, com o qual ele teve oportunidade de discutir
o texto. Segundo Inglese, certamente, entre o texto enviado a Vettori em
dezembro e a resposta deste em 18 de janeiro de 1514, Maquiavel foi
polindo e retocando o seu texto, que se constitui de fato na primeira
parte do livro. Com efeito, por aquilo que é indicado na carta de
dezembro de 1513, temos o roteiro dos temas tratados entre o capítulo \versal{I}
e o \versal{XI}, de uma obra que contém 26 capítulos.

Depois das dificuldades e negativas de Vettori em entregar o opúsculo de
Maquiavel para o papa, há poucas notícias sobre o manuscrito
maquiaveliano. Ele dará notícia novamente por mais sete vezes em suas
correspondências pessoais, sendo a última, em maio de 1514.

Talvez em função do malogro de entregar ao papa a obra Maquiavel tenha
modificado o destinatário da obra, não mais o papa, Giuliano de Medici,
mas a seu sobrinho, Lorenzo di Piero di Medici, então chefe político de
Florença, tal qual estabelecido pela tradição editorial. Um dado é
certo, esta dedicatória é anterior a outubro de 1516, quando Lorenzo
recebe o título de Duque de Urbino, coisa a que Maquiavel não se refere
na carta dedicatória, mas que jamais o teria feito se Lorenzo já tivesse
recebido o título, pois essa seria uma falta de decoro grave (\versal{INGLESE},
1994, p. 7). Portanto, a partir da própria documentação manuscrita e das
correspondências de Maquiavel, sabe-se que este obra foi composta entre
1513 e outubro de 1516.

Tendo em vista a não aceitação do texto pelo Papa, o opúsculo \emph{De
Principatibus} de Maquiavel passou a circular de forma manuscrita entre
os seus amigos durante muitos anos, pois a primeira edição impressa
sairia somente em 04 de janeiro de 1532. Ou seja, entre o final de 1513
e janeiro de 1532, o texto maquiaveliano circulou de modo informal e
manuscritamente. Ora, tendo em vista que o autor morre em 1527, nasce um
problema incomum para as obras compostas após a invenção da impressa no
século \versal{XV}, a saber: qual é o grau de originalidade do texto que foi
publicado em 1532? Será ele de fato a primeira versão do texto feita por
Maquiavel ou uma cópia de segunda ou terceira mão, com acréscimos,
supressões e demais alterações realizadas de modo costumeiro pelos
copistas? Enfim, qual o grau de originalidade do texto italiano de
\emph{O Príncipe} impresso pela primeira vez em 1532?

Tal dificuldade se amplia tendo em vista que, segundo relata Inglese
(1994, p. 10- 14), após 1514 o texto passou a circular de forma
manuscrita entre os amigos mais próximos de Maquiavel. Uma nova
referência ao texto apenas retorna em 29 de julho de 1517, quando o
jovem Nicollò Guicciardini, escrevendo ao seu pai Luigi, então
comissário florentino na cidade de Arezzo, sugere a ele se comportar
como disse ``Maquiavel em sua obra \emph{De Principatibus}'' (\versal{INGLESE},
1994, p. 14). Luigi Guicciardini conhecia Maquiavel dos tempos da
Chancelaria, tendo recebido certamente uma cópia do texto sobre os
principados. Luigi era irmão de Francesco Guicciardini, também diplomata
e grande pensador político, que escreverá um comentário aos
\emph{Discursos} de Maquiavel e que certamente deve ter lido esse
opúsculo maquiaveliano antes de 1532, pois faz referência a passagens
desse na sua obra \emph{Discorso del modo di assicurare lo stato alla
casa de' Medici}, texto de 1516 (\versal{INGLESE}, 1994, p. 15), bem como existem
passagens do \emph{Reggimento di Firenze}, obra elaborada entre 1521 e
1524, que indicam o conhecimento da argumentação do \emph{De
Principatibus} de Maquiavel (\versal{PROCACCI}, 1995, p. 6). Outra informação
sobre a circulação do texto vem de dois discursos proferidos por
Lodovico Alamanni, expoente do grupo dos Medici que, em 25 de novembro e
27 de dezembro de 1516, pede aos governantes que se comportem de modo
cívico, em uma argumentação que procura imitar a exposição de \emph{O
Principe}, seja no tema e personagens mobilizados, seja no estilo.

Contudo, segundo Inglese, o grande divulgador do \emph{De Principatibus}
de Maquiavel em forma manuscrita foi seu auxiliar dos tempos de
Chancelaria, Biaggio Buonaccorsi (\versal{INGLESE}, 1994, p. 17-18). Ele, que era
amigo de Maquiavel e nutria também um grande interesse pelas questões
políticas, após a saída da Chancelaria, em novembro de 1512, dedica-se,
entre outras coisas, à cópia de textos e monta um ateliê para
isso\footnote{Cumpre lembrar que, mesmo após a invenção da imprensa,
  manteve-se o costume, entre as famílias ricas, de mandar confeccionar
  manuscritos de obras importantes, que eram produzidos com qualidade,
  seja nas folhas utilizadas, seja na encadernação. Certamente o ateliê
  de Buonaccorsi recebeu encomendas de membros da aristocracia
  florentina, sabedores do texto de Maquiavel e curiosos para
  conhecê-lo. Inglese informa ainda que, mesmo depois da publicação da
  edição impressa em 1532, houve ainda a confecção de textos manuscritos
  do \emph{De Principatibus} por alguns anos.}. Conforme os estudos
paleográficos, de seu ateliê saíram ao menos três manuscritos
importantes que compõem o aparato crítico contemporâneo. Mais ainda, dos
27 manuscritos tomados em consideração por Inglese para realizar sua
edição crítica, 16 deles foram cópias feitas a partir desses exemplares
de Buonaccorsi. Isso indica que mais da metade dos manuscritos tem uma
influência direta desses exemplares \emph{buonaccorsianos} e certamente
ele teve acesso ao manuscrito original de Maquiavel.

Antes de prosseguir nessa apresentação do texto, é importante dar
algumas informações básicas sobre paleografia. Toda vez que se tem uma
edição de uma obra impressa não revisada e autorizada pelo autor, nasce
a dificuldade de saber se o texto impresso publicado confere com o
original, também conhecido como \emph{autógrafo}, ou seja, escrito pelo
autor. Quando este é vivo no momento da publicação ou se tem o
manuscrito \emph{autógrafo} ou se tem algum documento (carta,
testamento, etc.) que comprova que aquilo que foi publicado era a
intenção do autor, de modo que não há problemas e donde a primeira
edição impressa, também conhecida como edição \emph{princeps}, tornar-se
a referência para todas as demais publicações.

Contudo, quando a publicação impressa é posterior à morte do autor ou
este ainda em vida não reconhece como verdadeiro aquilo que foi
publicado -- como no caso do filósofo inglês John Locke que, durante
muito tempo, travou uma batalha com os editores de seu \emph{Segundo
Tratado sobre o governo civil} em função das diversas alterações
inseridas pelo editor, e só veio a reconhecer como seu o texto publicado
apenas no final de sua vida (\versal{LASLETT}, 1998) --, ou ainda se a obra foi
divulgada antes do advento da imprensa, então faz-se necessário um
estudo sobre os manuscritos existentes para se estabelecer o texto
fidedigno ou o mais fidedigno possível. O texto publicado que é o
resultado desse trabalho de levantamento e análise dos manuscritos
denomina-se \emph{edição crítica}, cujo texto, se não é idêntico ao
original do autor, ao menos reproduz aquilo que é o mais próximo deste.
Ora, tal trabalho de pesquisa pode ser facilitado ou dificultado
conforme o material que se tenha a disposição. Quando se encontra o
\emph{autógrafo}, a pesquisa se encerra, pois todas as demais
\emph{cópias} manuscritas foram feitas a partir deste primeiro exemplar
ou ele passa a ser a referência para as demais cópias e o texto padrão
da edição impressa. Todavia, quando não há esse exemplar, deve-se
recorrer às técnicas de paleografia e filologia para tentar estabelecer
as relações de dependências entre as cópias e identificar qual seria o
autógrafo ou aquele que mais se aproxima deste, quando ele não existe.

Voltando à difusão do texto do \emph{De Principatibus}, uma outra grande
fonte de informação é o possível plágio de Agostino Nifo, que publicou
em outubro de 1522 o opúsculo \emph{De Regnandi Peritia}, que não
somente mantém a mesma sequência, como reproduz exemplos e argumentos
inteiros do \emph{De Principatibus}, embora o texto de Nifo tenha sido
escrito em latim e o de Maquiavel em italiano (\versal{INGLESE}, 1994, p. 18-22).
Contudo, como explica Larivaille (1989, p. 150-195), falar em plágio de
Nifo sobre o texto de Maquiavel é um tanto equivocado, pois, apesar das
inúmeras semelhanças, o que revela que Nifo leu o texto de Maquiavel,
seu argumento conduz a conclusões diferentes das expostas no \emph{De
Principatibus}, mantendo a tradição de pensamento político de tipo
moralista dos \emph{espelhos de príncipes}\footnote{Apenas adiantando
  algo que será melhor explicado, havia uma tradição de longa data de
  livros de aconselhamentos para os príncipes, também denominados
  \emph{espelhos de príncipes}, que exaltavam a virtude e moralidade
  como as principais qualidades que o príncipe deveria cultivar. Como
  veremos, o texto maquiaveliano é totalmente contrário a isso, não
  sendo possível enquadrá-lo nesta categoria. (\versal{SKINNER}, 2000)}\emph{.} A
conclusão, portanto, é que Nifo leu o texto maquiaveliano e se inspirou
nele, e não o plagiou para escrever sua obra. Ademais, a noção de
plágio, tal qual nós entendemos, não existia neste contexto do
Renascimento, pois os autores se utilizavam de outros sem fazer a devida
referência e até o próprio Maquiavel fará uma verdadeira paráfrase do
livro \versal{VI} das \emph{Histórias} de Políbio nos seus \emph{Discursos} sem
dar qualquer referência ou indicação do texto antigo. Como se verifica
ao longo de várias passagens de \emph{O Príncipe}, Maquiavel cita de
memória passagens de textos históricos romanos, com frequentes erros e
sem qualquer referência.

Sabe-se ainda que Agostino Nifo era um pensador influente e foi
escolhido pelos Medici para lecionar no \emph{Studio} florentino, atual
Universidade de Pisa. Ora, certamente o fato de ele ter lido o texto
maquiaveliano e se baseado nele para escrever o seu próprio texto
demonstra que o \emph{De Principatibus} de Maquiavel era um texto um
tanto conhecido nos círculos eruditos de Florença, contando inclusive
com alguma aceitação e respeitabilidade.

Após essa última informação sobre \emph{O Príncipe} advinda da obra de
Nifo, só teremos notícia do texto maquiaveliano quando de sua publicação
em 04 de abril de 1532 pelo editor Antonio Blado, de Roma, que publica a
obra com o título em italiano, \emph{Il Príncipe}, e não mais em latim,
\emph{De Principatibus.} Logo em seguida, em 08 de maio, o tipógrafo
Bernardo Giunta, de Florença, também publica o texto com o título também
em italiano. Apesar de os estudos paleográficos e filológicos colocarem
em questão a fidedignidade da edição Blado em relação ao
\emph{autógrafo} do \emph{De Principatibus}, ela é reconhecida como a
edição \emph{princeps}, que significa que é primeira edição e sobre o
qual se deve basear a numeração e referência das demais edições.

Como se nota, foi do editor a decisão de alterar o título do livro do
latim para o italiano, cuja tradução mais correta de \emph{De
Principatibus} seria em italiano \emph{Sopra i principati} (Sobre os
principados). Outra mudança significativa entre a carta a Vettori de
dezembro de 1513 e a edição publicada é a inserção de uma parte
reservada às questões militares (capítulos de 12 a 14) e toda uma seção
destinada à figura do príncipe, com uma conclusão exortando à união da
Itália. Essa ampliação do texto sugere que Maquiavel tenha acrescido e
modificado o texto depois de dezembro de 1513. Contudo, a questão é:
quando ele concluiu de fato a obra e, como já foi discutido entre os
comentadores, teria tido \emph{O Príncipe} duas redações?

Quanto à primeira questão, a hipótese mais provável, em função das datas
e personagens citados, é que Maquiavel tenha ampliado e alterado o texto
ao longo de 1514, mas que o concluiu definitivamente antes de 1515 em
função de informações sobre alguns personagens e datas. Segundo Procacci
(1995, cap. 1), tendo em vista a difusão e aceitação dos manuscritos em
seu circulo mais próximo, Maquiavel tentou publicar o \emph{De
Principatibus,} mas em Roma e não em Florença, em função da oposição ao
seu nome entre a aristocracia desta cidade. Em sua última estadia em
Roma, em 1526, Maquiavel encomenda a cópia de um exemplar manuscrito de
seu texto ao ateliê de Ludovico degli Arrighi (o que é o atual
\emph{códice Barberiano 5093} do \emph{De Principatibus}) (\versal{PROCACCI},
1995, p. 8-9). Ora, por que Maquiavel encomendaria um novo exemplar
manuscrito e feito de modo acurado e elegante em um ateliê prestigiado
de Roma se não fosse para publicá-lo ou ofertá-lo a alguém que
patrocinasse tal projeto editorial? Segundo Procacci ainda, as conversas
já estavam adiantadas para isso, entretanto, a publicação não se
realizou em função do ataque que os franceses fizeram a Roma em 1527,
conhecido como o ``Saque de Roma'', que atinge duramente a cidade e suas
finanças. A retomada dos projetos somente ocorrerá após 1530, quando a
situação política parecia estabilizada. Em 1531, um tipógrafo sem muito
prestígio, proprietário de um negócio modesto, Antonio Blado di Asola,
recebe a autorização papal para publicar duas obras de Maquiavel: os
\emph{Discursos} e o \emph{De Principatibus}, cujo título ele modifica.
Blado trabalhou durante muitos anos a serviço da Câmara Apostólica e não
teve um catálogo nem amplo ou com títulos de renome ou significativos
(\versal{PROCACCI}, 1995, p. 9-10). Outro dado curioso é que, na autorização
papal, já se prevê a possibilidade de Blado negociar e permitir outras
edições dessas obras, fato esse que, segundo Procacci, demonstra que o
modesto tipógrafo já pretendia vender os direitos de uma obra que teria
boa repercussão editorial. Tanto é procedente tal interpretação que logo
na sequência tem-se a publicação do texto em Florença pelo editor
Bernardo Giunta e, nos anos seguintes, em Veneza, por três casas
tipográficas distintas (\versal{PROCACCI}, 1995, cap.1). Portanto, conforme a
interpretação de Procacci, a publicação do texto maquiaveliano era um
projeto intencionado desde 1526 por Maquiavel, mas que não ocorreu por
acasos: o ``Saque de Roma'' e sua morte.

A interpretação de Martelli é diferente e agrega alguns dados novos e
complementares. Na edição comentada de \emph{O Príncipe}, Martelli
(2002) explora em muito o fato de o nome de Maquiavel ser rejeitado em
Florença, mas não necessariamente pela família Medici. Isso impediu que
durante muito tempo seu texto fosse publicado e ele mesmo pudesse
retornar ao governo, então sob o comando da família. Mesmo sem ser
considerado um inimigo por parte da família Medici, segundo a
argumentação de Martelli, Maquiavel era um personagem cuja proximidade
provocava desconforto político perante parte da aristocracia florentina.
Do que decorre que sua obra não tenha sido publicada em Florença
inicialmente, mas em Roma e depois de sua morte, quando a oposição havia
em parte diminuído.

Entretanto, em alguns pontos Martelli se distancia das interpretações
precedentes. Segundo ele, o texto maquiaveliano começou a ser escrito em
1513 e foi concluído apenas em 1518 e não em 1514 ou início de 1515 como
defendem Inglese e outros. Além disso, mesmo admitindo que não existisse
o autógrafo do \emph{De Principatibus}, Martelli defende que há um
exemplar que teria sido corrigido por Maquiavel e que poderia ser
denominado como o \emph{arquétipo} do qual derivam todos os outros e que
seria esse manuscrito o mais genuíno\footnote{Segundo Martelli o
  manuscrito \emph{Carpentras}, \emph{Bibliothéque Inguimbertine, 303}
  (denominado como manuscrito A), seria esse arquétipo que foi corrigido
  por Maquiavel e seria o texto mais original, apesar de não ser um
  autógrafo. (\versal{MARTELLI}, 2002, 325-329)}.

Esses elementos invocados por Martelli nos remetem à já mencionada
questão recorrente nos estudos sobre \emph{O Príncipe}: teria ele tido
duas redações? Tal questão foi formulada inicialmente por Federico
Chabod e, depois de discutida longamente, teve sua resposta no estudo de
Sasso (1958) no qual se aponta que Maquiavel redigiu uma primeira parte
do texto em 1513 e o restante, provavelmente, em 1514, não alterando
mais o texto depois de 1515. Uma questão embutida nessa é saber se já no
período da Chancelaria Maquiavel não teria começado o texto,
concluindo-o no período imediatamente posterior a sua saída.

Tal hipótese de uma dupla redação e não de uma redação contínua no
biênio 1513 e 1514 é muito frágil, pois, mesmo que de fato Maquiavel
tenha começado a fazer o texto, ou mesmo que já tivesse um esboço ou
rascunho, certamente ele se valeu dessas informações anteriores -- como
ele mesmo declara na \emph{Carta Dedicatória} --, o que comprova que não
somente havia sim elementos anteriores a 1513, mas que tudo isso foi
reelaborado e ampliado nesse ano de redação. Sem esquecer que ele vinha
pensando nesses temas e argumentos ao longo do tempo, haja vista o
confronto com textos e escritos diplomáticos de análise de contextos,
com os textos posteriores a 1512, nos quais encontramos um autor mais
maduro e consciente de seu pensamento político. Do que se pode concluir
com muita segurança que Maquiavel já vinha de longa data meditando sobre
esses temas e que organizou e colocou no papel tudo isso no biênio de
1513 e 1514. Depois não mais.

Esse quadro informativo nos coloca diante do seguinte questão: o
manuscrito do \emph{De Principatibus} utilizado por Antonio Blado para
fazer a primeira edição impressa de \emph{Il Príncipe} era de fato o
autógrafo de Maquiavel? Mais, tendo em vista a alteração do título, qual
a garantia de que o editor não tenha alterado algo no texto, o que teria
corrompido sua autenticidade? Enfim, parece que a primeira edição
bladiana do texto maquiaveliano não confere nenhuma garantia sobre a
autenticidade do texto.

Essas suspeitas fizeram com que, já no século \versal{XIX}, houvesse uma primeira
tentativa de estabelecer um texto original ou padrão de \emph{Il
Príncipe}, por meio do editor G. Lisio, e editada por Sansoni em
Florença em, 1899. A também conhecida edição Lisio de \emph{Il Príncipe}
foi o primeiro texto com um mínimo de aparato crítico. Depois de algumas
edições que agregaram alguma informação ao texto, mas sem superar ou
corrigir a edição Lisio, Giorgio Inglese publica a edição crítica em
1994. Este, consultando uma gama maior de manuscritos e depois de um
trabalho apurado de análise paleográfica e filológica, apresenta uma
nova edição crítica do texto do \emph{De Principatibus} de Maquiavel,
publicada pelo Istituto Storico Italiano per il Medioevo. Além dessa
importante edição, em 2002 foi publicada uma outra edição de \emph{O
Príncipe}, dentro da coleção \emph{Edizione Nazionalle delle Opere di
Machiavelli}\footnote{Doravante citado apenas como \emph{\versal{EN}.}} pela
Editora Salerno, sob a coordenação de Mario Martelli, e que pretende
discutir e rever pontos não esclarecidos da edição Inglese, valendo-se
de outros manuscritos (que eram do conhecimento de Inglese mas que, por
critérios técnicos, foram por ele descartados) e buscando,
principalmente, refutar a tese de Inglese de que não há um autógrafo de
Maquiavel. Em outros termos, o ponto central da análise crítica de
Inglese é que o autógrafo do \emph{De Principatibus} está ainda perdido
e que a reconstrução dos manuscritos permite estabelecer os dois
principais exemplares dos quais dependem todos os demais e, com esses,
reconstruir o texto o mais próximo do original possível\footnote{A
  hipótese central de Inglese é que os manuscritos D (München.
  Universitätsbibliothek, 4º cod, ms. 787) e G (Gotha, Forschungs- und
  Landesbibliothek, chart. B 70) dependem de um outro manuscrito
  (denominado γ) que seria o arquétipo mais próximo do orignal. Cf.
  \versal{INGLESE}, 1994, p. 150-155.}. A contra-argumentação de Mario Martelli,
tenta, por outra metodologia, mostrar que há um autógrafo ou arquétipo
do texto maquiaveliano e que este foi aquele utilizado por Blado, no
caso, o manuscrito A (\emph{Carpentras}) da lista dos manuscritos
disponíveis.

Ora, não se trata, nesta introdução, de entrar nos detalhes de caráter
filológico ou paleográfico da discussão, visto que não temos elementos
para refutar posições ou novidades a acrescentar. Porém, cumpre declarar
que, no confronto das argumentações, seja em função da metodologia, seja
em função da autenticidade e autoridade dos manuscritos, a exposição de
Giorgio Inglese é mais consistente e o texto por ele organizado é mais
fidedigno. A argumentação de Martelli, ao utilizar-se de uma nova
``metodologia'' na qual a história do texto prevalece, não consegue, ao
fim e ao cabo, oferecer uma comprovação ou modificação substancial no
estatuto de originalidade do texto maquiaveliano. No limite, toda a
argumentação de Martelli apenas polemiza e não traz novos resultados em
relação a edição crítica de Inglese. Tanto é assim que o próprio
Martelli denomina sua edição de \emph{Edição Comentada.}

Donde, na falta de argumentos novos e de elementos concretos que
invalidem sua exposição, somos obrigados a aceitar a edição Inglese de
1994 do \emph{De Principatibus} como o texto mais fidedigno àquele
intencionado e escrito por Nicolau Maquiavel, motivo pelo qual o
escolhemos para ser o texto original italiano dessa edição bilíngue
brasileira.

\subsection{Estrutura do Argumento}

Como foi dito, \emph{O Príncipe} possuía um outro título que depois foi
alterado pelo editor romano, mas que não contradiz totalmente o conteúdo
da obra. Na verdade, pode-se dizer que o texto maquiaveliano contém o
\emph{De Principatibus} e \emph{Il Príncipe} ao mesmo tempo.

A obra está dividida em 26 capítulos, num total de 28.250 palavras. O
primeiro capítulo apresenta o roteiro dos temas tratados até o capítulo
11, que se concentram nos principados, na conquista destes e na
conservação do poder. Os três capítulos seguintes (12, 13 e 14) tratam
da questão militar, acerca da conveniência e inconveniência de se
possuir exércitos auxiliares e próprios, que também são pensados no
interior dos esforços de conservação do poder político. Os capítulos de
15 a 23 tratam das qualidades e cuidados que um príncipe deve possuir no
comando político. Os capítulos 24 e 25 analisam a importância da fortuna
na vida política e como o príncipe deve ficar atento a ela. Por fim, o
capítulo final é uma exortação à unidade da península itálica, algo
desejável sob o comando da família Medici.

Diante dessa disposição temática dos capítulos, uma primeira compreensão
geral da obra, sobre a qual concordam os comentadores, é dividir \emph{O
Príncipe} em duas grandes partes: uma primeira dedicada à análise do
principado e uma segunda voltada para a ação do príncipe. É evidente que
nesta divisão geral não se encaixam muito bem os capítulos finais,
particularmente o último.

Entretanto, seja essa divisão temática da obra, sejam as demais
informações apresentadas, tudo nos remete a uma ordem de problemas que
convém analisar com atenção. Um primeiro âmbito de problemas está
voltado à noção de principado, e pode ser expresso nos seguintes termos:
o que de fato é o principado de Maquiavel? Qual a sua definição e a sua
função no interior do pensamento político maquiaveliano? A resposta mais
comum nos manuais é identificar a noção de principado à moderna noção de
``Estado'', declarando desse modo que Maquiavel é um precursor do
pensamento político moderno. Como se verá, não se trata disso, pois
tanto a definição de principado como a de estado em Maquiavel não se
identificam com a moderna noção política de Estado, embora seja possível
reconhecer neles elementos que apontam para aquilo que o pensamento
político moderno, posterior a Thomas Hobbes, certamente identificou na
noção política de Estado, este entendido como uma entidade política
autônoma.

Uma outra questão que decorre diretamente dessa primeira ordem de
problema é sobre a concepção de príncipe de Maquiavel. A primeira e
imediata associação é com o título nobiliárquico de príncipe das
monarquias. Ora, essa concepção gera, por seu turno, a ideia de que o
texto maquiaveliano é um libelo em defesa da monarquia ou um livro
destinado aos nobres. Tal é o pano de fundo da argumentação de Quentin
Skinner, que no seu livro mais conhecido, \emph{As fundações do
pensamento político moderno}, classifica a obra de Maquiavel dentro de
um estilo que fez fama entre os séculos \versal{XI} e \versal{XVI}, os \emph{espelhos de
príncipes} (\versal{SKINNER}, 2000). Esse estilo de texto, que remonta ao
pensamento político romano, tendo como um de seus textos fundadores o
\emph{Dos Deveres} (\emph{De Officis}) de Marco Túlio Cícero (54 a.C.),
foi retomado ao longo do tempo no pensamento político latino, passando
pela Idade Média e Modernidade (\versal{SENNELART}, 2006)\footnote{No século
  \versal{XVIII} ainda se encontra esse tipo de texto entre os pensadores
  portugueses, o que comprova a fortuna de estilo no pensamento politico
  ocidental.}, e era destinado a auxiliar e instruir os governantes na
tarefa de governar, principalmente os jovens príncipes, que careciam de
experiências políticas, donde a definição ``espelhos de príncipes'': por
ser um manual onde os príncipes deveriam espelhar suas ações.

O desconforto em enquadrar o texto maquiavelino neste estilo reside mais
na definição de príncipe como o filho do rei, como o futuro monarca,
restringindo o escopo de denotação do termo, particularmente quando
tratamos de um autor que, como já foi declarado, assume-se claramente em
defesa do republicanismo. Disso decorre a contradição instalada: teria
um autor que é reconhecidamente defensor do republicanismo escrito um
livro em defesa da monarquia, tendo em vista para quem a obra é
dedicada, para tentar, entre outros motivos, um emprego no novo governo
da família Medici? Teria Maquiavel se destacado por escrever uma obra de
cunho monarquista por que estava desempregado e sem perspectivas? Enfim,
seria Maquiavel um pensador com uma dupla postura teórica: em algumas
obras defensor do regime republicano e em outra -- curiosamente na mais
célebre -- defensor do regime monárquico?

As respostas para tais problemas não são simples e nem fáceis, pois
demandam um esforço de análise e interpretação. O início dessa análise
está, todavia, fora de \emph{O Príncipe}, mas em outra obra, nos
\emph{Discursos.} Nesta obra, em particular, depois de apresentar os
elementos gerais da república, Maquiavel fala da corrupção desta, do seu
fim e como poder-se-ia encontrar alguma solução para a morte certa do
regime republicano. A resposta para essas noções de principado e
príncipe, bem como para a questão de fundo de como articular \emph{O
Príncipe} no interior do pensamento político maquiaveliano, passa por
uma análise sobre o fim da república. Enfim, a dificuldade que se
apresenta agora é de como articular \emph{O Príncipe} com o pensamento
republicano de Maquiavel, com as demais obras políticas de caráter
nitidamente republicano. Para isso será necessário recuperar, ainda que
rapidamente, os delineamentos gerais da noção de república tal qual
Maquiavel faz no início dos \emph{Discursos.}

\subsection{O ``Pequeno tratado sobre as repúblicas''}

No início da exposição dos \emph{Discursos sobre a primeira década de
Tito Lívio}, notadamente no primeiro livro, entre os capítulos 1 e 18,
encontramos um momento privilegiado no qual Maquiavel apresenta os
elementos principais do que entende por república\emph{.} Tais noções
também podem ser encontradas em vários escritos de modo esparso, como na
\emph{História de Florença}, no opúsculo \emph{Discurso sobre as coisas
de Florença}, na \emph{Arte da Guerra}, entre outros; mas em nenhum
deles temos de modo tão claro e estruturado como nos \emph{Discursos}.

Escrito, provavelmente, na sequência de \emph{O} \emph{Príncipe}, os
\emph{Discursos} expõem uma análise sobre parte da obra \emph{História
de Roma} de Tito Lívio. O objetivo do texto maquiaveliano é o de
comentar os fatos narrados pelo historiador romano e recuperar noções
para a elaboração de argumentos em defesa do regime republicano, no caso
particular para a Florença de início do século \versal{XVI}, bem como para uma
teoria em defesa do regime republicano de modo geral.

A \emph{História de Roma} foi a principal obra de Tito Lívio (59 a.C --
17 d.C). Composta originalmente por 142 livros, dos quais restaram
apenas 35, ela narra, em seu conjunto, os feitos romanos desde a sua
origem até o governo de Otávio Augusto (9 a.C.). Ao longo do tempo, os
copistas fizeram uma divisão da obra em grupos de dez livros, bem como
uma sinopse de cada livro que foram reunidas no início da
\emph{História}. A essa reunião dos livros em conjunto de dez, ainda que
nem sempre rígida, deram o nome de \emph{deca} (em italiano), que foi
traduzido por ``décadas'' em português, quando, na verdade, trata-se de
``dezenas''. Os dez primeiros livros ou a ``primeira década'' ou a
``primeira dezena'', foram dos poucos que se conservaram
integralmente e narram os fatos desde as origens de Roma até o ano de
295 a.C., ou seja, a narrativa compreende o governo monárquico e boa
parte do governo republicano. Maquiavel elabora os seus \emph{Discursos}
sobre esses dez primeiros livros da \emph{História de Roma}, uma vez que
nestes há a presença de vários temas que lhe são caros, entre eles, a
conservação da república, as mudanças de regimes e a corrupção das
instituições políticas.

Olhado em sua totalidade, o texto maquiaveliano parece ser um comentário
geral da obra liviana, no qual o comentador vai inserindo em sua análise
as ideias e as concepções políticas que busca defender. O problema maior
nessa visada geral são os primeiros dezoito capítulos do livro \versal{I}, no
qual Maquiavel não segue
\emph{pari passu} o
texto liviano, mas faz uma exposição ampla sobre alguns conceitos
republicanos, inserindo nessa análise exemplos romanos, mas também
exemplos de Esparta, Veneza, Florença e a Roma católica de seu tempo.
Como apontou Gilbert (1953), primeiramente, esses primeiros capítulos
parecem se constituir em um bloco à parte da obra. Doravante se instalou
entre os comentadores uma disputa interpretativa, na qual, entre outros
pontos de discórdia, pairava a questão se Maquiavel escreveu primeiro
esse trecho do livro ou se escreveu conjuntamente com o restante da
obra, entre os anos de 1515 a 1517. Questão essa que se complica, pois,
tendo em vista que Maquiavel teria escrito o \emph{De Principatibus}
antes dos \emph{Discursos}, nasce a dúvida de como entender um trecho
inicial de \emph{O Príncipe}, quando o autor declara:

\begin{quote}
Deixarei de lado a discussão sobre as repúblicas, porque, em outro
momento, dissertei longamente sobre elas. Ocupar-me-ei somente do
principado, e retecerei as urdiduras acima descritas, e demonstrarei
como estes principados podem ser governados e conservados (\emph{O
Príncipe}, cap. \versal{II}, 1-2).
\end{quote}

Resta, pois, a questão de qual seria esse texto republicano de Maquiavel
indicado no texto de 1513. Já nos antecipamos em afirmar que não é
possível, com os dados disponíveis até hoje, indicar se esses dezoitos
capítulos perfazem o texto republicano mencionado no início de \emph{O
Príncipe} e nem a cronologia correta de composição das obras. Tudo o que
se tem são hipóteses, algumas com base nos poucos dados históricos
disponíveis e outras com base na estrutura dos argumentos expostos por
Maquiavel em suas obras.

A polêmica em torno desses dezoito primeiros capítulos dos
\emph{Discursos} não se resume, então, a uma disputa cronológica de
anterioridade ou posteridade em relação a \emph{O} \emph{Príncipe}.
Subjaz a essa discussão a definição do sentido próprio da obra
maquiaveliana cuja interpretação determina a compreensão que ele possuía
das repúblicas, bem como do lugar de \emph{O Príncipe} no interior de
seu pensamento político. O estudo de Gilbert fez com que, sob diferentes
óticas e metodologias, as atenções para a interpretação do pensamento
político maquiaveliano se voltassem para os \emph{Discursos}. Uma vez
que não está ao nosso alcance reconstruir nos detalhes os pontos desse
debate e os meandros dessa contenda, algo que já foi analisado pelos
comentadores e que também abordamos em outro texto (\versal{MARTINS}, 2007, 16),
partiremos aqui do fato concreto de que esse conjunto de capítulos
iniciais perfazem aquilo que Larivaille nomeou como ``\emph{Pequeno
tratado sobre as repúblicas''}. Certamente, neste trecho da obra
maquiaveliana, encontraremos os elementos teóricos para compreender de
que modo pode ser possível pensar a inserção de \emph{O Príncipe} no
pensamento político de seu autor.

Voltando à passagem problemática de \emph{O Príncipe}, quando ele
escreve que não tratará das repúblicas porque já o havia feito em outro
lugar, deduz-se que estivesse se referindo aos \emph{Discursos,} visto
que não deixou um tratado específico sobre esse assunto e que aquela
obra traz uma reflexão sobre a república romana. Todavia, existem vários
indícios que mostram que os \emph{Discursos} foram escritos depois de
1514, quando Maquiavel frequentava os \emph{Orti Oricellari} \footnote{\emph{Orti
  Oricellai} era o nome que se dava aos jardins da família Rucellai,
  que, desde o final do governo dos Medicis, no século \versal{XV}, abrigava
  reuniões de aristocratas florentinos. Após a queda do governo de Pier
  Solderini, em 1512, o neto de Bernardo Rucellai, Cosimo, passou a
  organizar reuniões com jovens aristocratas de ideologia republicana.
  Maquiavel passa a frequentar esses encontros a partir de 1515, momento
  em que se acredita que ele tenha escrito a maior parte dos
  \emph{Discursos}. Todos os comentadores destacam a importância desses
  encontros para a reflexão política maquiaveliana no que diz respeito à
  teoria republicana e ao estudo dos clássicos (\versal{GILBERT}, 1977, p. 15-66;
  \versal{VIROLI}, 2002; \versal{SASSO}, 1986, p. 353-357). Adiante isso será melhor
  explicado.}, entre 1515 e 1517. Tais indícios colocam, pois, o
problema de tentar descobrir a qual texto Maquiavel estava fazendo
referência no início de \emph{O Príncipe}, visto que ele não
havia escrito, ainda, os \emph{Discursos.} Uma outra questão, paralela a
esta, estaria em saber ao certo qual foi o momento de composição dos
\emph{Discursos}: se antes ou depois da composição de \emph{O}
\emph{Príncipe}.

Esses são os pontos principais do debate em torno da datação dos
\emph{Discursos} e da existência ou não de um bloco textual destacado no
início do livro \versal{I}. Apesar dessas divergências, de modo geral, é aceito
que a parte principal do livro tenha sido escrita entre 1515 e 1517, o
que não exclui a possibilidade de que uma primeira parte já tivesse sido
escrita antes desta data, sendo apenas corrigida na época em que
Maquiavel frequentava os \emph{Orti Oricellai}. Depois das contendas
entre os comentadores, uma certa concordância se formou acerca de dois
pontos: a) que os capítulos que tratam dos comentários da \emph{História
de Roma} de Tito Lívio possuem uma unidade analítica a despeito da não
adequação de algum capítulo à essa regra geral, corroborando a tese de
Gilbert de que, nesses casos, as exceções confirmam a regra; b) que os
dezoito primeiros capítulos do livro \versal{I}, mesmo sendo o momento mais
teórico da obra, configuram-se como um anteparo conceitual explicativo
dos eventos que serão comentados. Apesar de não serem comentários
diretos aos livros de Tito Lívio, esses capítulos mostram-se importantes
na economia dos \emph{Discursos} como um todo, na medida em que explicam
as origens e os fundamentos das repúblicas e os ordenamentos, leis e
conflitos que marcam sua vida civil. Enfim, afirmar a existência de um
bloco teórico inicial não depõe contra a harmonia e unidade interna dos
\emph{Discursos}, impossibilitando qualquer afirmação de que a obra é
resultado de uma colagem ou de uma justaposição de textos diversos.

Os \emph{Discursos} estão divididos em três livros, sendo que o livro \versal{I}
tem 60 capítulos, o livro \versal{II} 32 capítulos e o \versal{III} 49 capítulos. Os
primeiros dezoito capítulos do livro \versal{I} perfazem a seguinte sequência
expositiva: a fundação das cidades (capítulo \versal{I}), a natureza e a mudança
dos regimes políticos (capítulo \versal{II}), os conflitos políticos (capítulos
\versal{III} e \versal{IV}), a defesa da liberdade política nas republicas (capítulos \versal{V} e
\versal{VI}), os instrumentos de defesa e acusação pública (capítulos \versal{VII} e
\versal{VIII}), a reforma ou refundação dos estados (capítulos \versal{IX} e \versal{X}), a
importância da religião (capítulos de \versal{XI} a \versal{XV}) e a corrupção nas
repúblicas (capítulos de \versal{XVI} a \versal{XVIII}).

Verifica-se, pois, que esses capítulos apresentam uma unidade teórica e
formam um bloco textual de análise dos fundamentos das repúblicas. Pelo
roteiro dos temas expostos, nota-se a presença de um itinerário
argumentativo cujo movimento vai do nascimento da cidade, passando pela
fundação dos ordenamentos políticos e o modo de defesa do \emph{libere
vivere} político, culminando na corrupção do povo, das instituições
(\emph{ordini}) e das leis. A religião, mobilizada no interior dessa
reflexão, também se apresenta como uma instituição capaz de conservar,
por meio de seus ritos, os valores e os ideais republicanos, ou seja,
ela cumpre o papel de \emph{instrumentum regni,} um instrumento para
governar. Além dessa descrição da vida política das repúblicas, na qual
se revelam as etapas de sua existência, pode-se também afirmar que esses
capítulos são uma introdução teórica aos \emph{Discursos}, uma vez que
definem os elementos essenciais na constituição de uma cidade.

Quando falamos em aspectos teóricos, não devemos ter em vista um certo
modelo de tratado em que os conceitos aparecem de modo destacado por
expressões próprias, como \emph{defino que}, \emph{demonstra-se},
\emph{entendo por} etc. Maquiavel utiliza-se de um outro estilo que não
é em nada menos indicativo de seu objetivo de definir conceitos. A
própria escolha e disposição dos temas é, por si, uma indicação de seus
propósitos teóricos. Assim, no capítulo \versal{I}, ao tratar da fundação das
cidades, ele faz, na verdade, uma descrição dos tipos de cidades que
podem existir e de como seu momento fundador pode ser determinante para
o desenvolvimento ou para a ruína futura, tipificando-as pelo seu modo
de fundação. No capítulo \versal{II}, faz, num primeiro momento, uma exposição
das formas de governo possíveis e de como nelas pode se processar a
mudança, indicando que o motor ou a causa desta não é uma certa lógica
determinista da natureza, mas dos conflitos políticos, tema dos
capítulos \versal{III} e \versal{IV}. Os capítulos \versal{I} e \versal{II} configuram-se, portanto, como
descrições tipológicas, seja do modo como pode se operar a fundação de
uma cidade, seja do modo como os regimes podem se instalar e se
transformar, descrições estas que fazem desses capítulos um preâmbulo
conceitual para a análise que se seguirá.

Na sequência, os capítulos de \versal{III} a \versal{X}, ao apresentarem temas como os
conflitos políticos, a defesa da liberdade política nas repúblicas, os
instrumentos de defesa e de acusação pública e a reforma ou refundação
das cidades, revelam como Maquiavel entende os elementos essenciais das
repúblicas. Como se verá, esses pontos configuram-se como as
\emph{ordini} ou os ordenamentos políticos básicos de uma república,
aspectos estes estruturais da vida de uma cidade ordenada como
república. Mesmo nos exemplos mobilizados, ele não se restringe ao caso
romano, mas fala de Veneza, Esparta, Florença, entre outras cidades,
numa clara indicação de que está apresentando as partes da vida política
em um regime republicano.

A exposição sobre a religião dos capítulos de \versal{XI} a \versal{XV} cumpre também essa
função na medida em que ela é considerada \emph{instrumentum regni}, uma
instituição cujo papel não se limita a ser de ordem religiosa porque é
decisiva na própria organização e no funcionamento da vida política da
cidade. Ao tratar da religião, o silêncio de Maquiavel em relação às
disputas medievais entre o papado e o império, fundamentais seja para o
futuro das cidades do norte da Itália, seja para o próprio período
medieval (\versal{SKINNER}, 2000, cap. 1-3), é indicativo de que a sua
preocupação não é especificamente a religião romana, mas sim analisar a
religião -- entendida eminentemente como prática social -- no que diz
respeito à sua relação com a vida política da cidade, ou seja, ao seu
papel político.

Quanto aos capítulos de \versal{XVI} a \versal{XVIII}, eles não são comentários à
corrupção romana, mas buscam entender como a corrupção pode atingir uma
república, seu povo ou suas instituições. Não se trata de uma explicação
do caso específico romano, mas da corrupção que pode acometer as
repúblicas de um modo geral.

Portanto, os temas mobilizados nesses capítulos iniciais e o modo como
eles são analisados, revelam o quanto Maquiavel fez, ao seu modo, uma
apresentação dos principais elementos políticos de uma república e de
seus fundamentos, já que não há uma análise exclusiva nem do caso
romano, nem dos livros da \emph{História} de Tito Lívio. Ora, é patente
que, nesse primeiro momento dos \emph{Discursos,} não se está fazendo
uma analise \emph{pari passu} do texto liviano, nem um comentário
histórico dos fatos de um modo geral. A coesão presente no objetivo e no
modo de exposição, a escolha dos temas e o itinerário que eles descrevem
-- do nascimento à corrupção da cidade --, confere um caráter unitário a
esse bloco textual. Os capítulos que se seguem ao \versal{XVIII}, nos quais
verifica-se, aí sim, um comentário à obra liviana, não denotam uma
inflexão teórica ou conceitual. A compreensão da \emph{História de Roma}
e o sentido ou o modo como essa análise deve ser feita seguirão os
critérios e as ideias apresentadas no início, mostrando uma imbricação
entre a análise dos fatos romanos à luz dos critérios apresentados. Os
comentários serão, portanto, pautados pelos conceitos e pelo modo de
compreendê-los, indicando uma profunda dependência entre as partes do
livro. Apesar de os dezoito capítulos se constituírem numa unidade
passível de ser analisada autonomamente, os \emph{Discursos} permanecem
um todo em sua estrutura, na medida em que seus capítulos dependem dos
critérios estabelecidos no ``\emph{Pequeno tratado sobre as
repúblicas}'' e este, por sua vez, formula seus fundamentos teóricos
tendo em vista a compreensão da vida em qualquer república, inclusive a
de seu tempo\footnote{As reuniões nos \emph{Orti Oricellai,}
  patrocinadas por jovens aristocratas de ideais republicanos, visavam
  também encontrar meios para restaurar o governo republicano em
  Florença (\versal{GILBERT}, 1977, 15-66).}, e não somente Roma, modelo de
república. Separar o ``\emph{Pequeno tratado}'' dos \emph{Discursos}
violaria, na perspectiva maquiaveliana, a \emph{verità effetuale}, a
comprovação no mundo real, recaindo num argumento feito para
``repúblicas ou cidade imaginadas''.

Segundo Chabod, os \emph{Discursos} constituem a ``origem espiritual''
de \emph{O Príncipe,} uma vez que pode-se perceber a existência
de uma relação direta entre os problemas que subjazem a ambos\footnote{Essa
  hipótese é sugerida primeiramente por Chabod, porém ele não a
  desenvolve. Gennaro Sasso, anos depois, será o primeiro a
  desenvolvê-la, tirando novas conclusões, como se verá nos capítulos
  finais dessa introdução (\versal{CHABOD}, 1993, p. 31-39).}. Ele propõe, não a
partir dos dados históricos sobre os textos, mas a partir da articulação
dos conceitos, que os \emph{Discursos}, ou parte deles, seriam os
pressupostos teóricos para \emph{O Príncipe}. Embora não tenha
desenvolvido essa hipótese, Chabod apontava para uma motivação de ordem
republicana mobilizando os argumentos de \emph{O Príncipe}, de
tal modo que esse deveria ser uma resposta ou continuação de algo
deixado para traz nos \emph{Discursos.}

Avançando nessa interessante sugestão e tendo em vista que há uma
exposição sobre as repúblicas no início dos \emph{Discursos}, nasce a
dificuldade de como se poderia articular essa parte inicial com \emph{O
Príncipe}. Uma hipótese seria verificar a sequência argumentativa quando
a república chega ao seu ápice de corrupção política, conforme exposto
no capítulo \versal{XVIII} do livro \versal{I} dos \emph{Discursos}.

Segundo Gennaro Sasso (1980, cap. \versal{V}), tendo Maquiavel escrito os dezoito
primeiros capítulos dos \emph{Discursos} e chegando ao ponto em que as
repúblicas estão completamente dominadas pela corrupção, em que a ruína
é um fato quase inevitável, a instauração de um principado civil passa a
ser o remédio adequado. Dito de outro modo, quando se verifica, no
capítulo \versal{XVIII} dos \emph{Discursos}, que é quase impossível a uma
república ``\emph{corrompidíssima}'' retomar o \emph{vivere civile}, as
liberdades civis, características dos regimes republicanos sadios, a
solução passa a ser a instauração de um regime fundado em um único
governante para que esse, com sua \emph{virtù}, consiga recuperar a
normalidade política da cidade e impedir a ruína certa. Ademais, ``o
principado representa o remédio que, auxiliado por extraordinária
\emph{virtù}, os legisladores que vêem longe procuram opor à corrupção
das repúblicas'' (\versal{SASSO}, 1980, cap. \versal{V}). Calcado naquilo
que é exposto pelos textos maquiavelianos, há a ``\emph{problemática
passagem}'', como diz Sasso, das repúblicas corrompidas para o regime
régio caracterizado pelo principado civil, na medida em que esse regime
pode oferecer uma resposta eficaz ao problema que se instaura nas
cidades corrompidas. O remédio é sugerido no próprio capítulo \versal{XVIII} dos
\emph{Discursos}, quando Maquiavel afirma que o freio para essa
corrupção total é a instauração de um governo régio, um regime que, com
a sua mão ``régia'', intervenha para reordenar a cidade. Ao final do
capítulo \versal{XVIII}, Maquiavel apresenta a ideia que será dominante no
\emph{Príncipe}, de tal modo que, da perspectiva de quem olha dos
\emph{Discursos}, a boa solução ou o remédio adequado não está nos seus
capítulos seguintes (\versal{XIX}, \versal{XX} etc.), mas no principado civil, tal qual é
apresentado nos capítulos \versal{VIII} e \versal{IX} de \emph{O Príncipe}. Do que podemos
concluir, conforme Sasso: ``os principados pressupõem a crise da
república, e não nascem senão quando essa está tomada pelas formas
extremas da corrupção, da degeneração'' (\versal{SASSO}, 1980, cap. \versal{V}). Com isso,
a origem de \emph{O Príncipe} não se fundaria numa visão
``idealizada''\footnote{A crítica à origem mítica ou idealizada de
  \emph{O Príncipe} é um dos objetivos de Sasso nessa reflexão,
  pois, para ele, carece de fundamento pensar a motivação de um livro
  apenas em pressupostos ideais. (\versal{SASSO}, 1980, p. 316, nota 41).} dos
regimes políticos, mas encontra sua motivação teórica no limite extremo
que se configura com a corrupção das repúblicas. Em uma cidade onde as
\emph{ordini} e as leis estão dominadas pela corrupção, a intervenção do
príncipe novo, tema dominante de todo \emph{O Príncipe}, faz-se
necessária, reformulando, ou melhor, refundando as ordens e as
instituições, reconciliando os humores, enfim, tudo aquilo que também é
preconizado ao longo dos \emph{Discursos}. Desse ângulo, \emph{O
Príncipe} seria tributário do raciocínio desenvolvido no
``\emph{Pequeno tratado sobre as repúblicas}'', pois teria nesse sua
maior motivação teórica. Por outro lado, os \emph{Discursos} manteriam
uma relação de dependência teórica com \emph{O Príncipe}, visto
que a melhor solução para o problema no qual culmina o raciocínio seria
o principado civil. Essa interdependência teórica revela uma estreita
linha de continuidade no interior da reflexão política maquiaveliana,
como afirma o comentador:

\begin{quote}
Precedendo cronologicamente ou seguindo a composição do \emph{Príncipe},
o décimo oitavo capítulo do primeiro livro dos \emph{Discursos} é,
portanto, o ``lugar ideal'' no qual o conceito daquele livro se realiza
nos seus modos próprios (\versal{SASSO}, 1980, p. 327).
\end{quote}

Portanto, cabe agora tentar entender melhor essa passagem da república
corrompida ao principado, ou por outro lado, o que é esse principado e o
seu príncipe, a partir desse pressuposto lançado pela corrupção
republicana.

\section{O Príncipe Civil}

\subsection{Os pressupostos para o poder régio}

Como dito, a sequência argumentativa do \emph{Pequeno tratado sobre as
repúblicas} encaminha-se, a partir do capítulo 16, para o tema da
corrupção na cidade. Primeiro, Maquiavel trata da corrupção do povo ou
da matéria da cidade (cap. \versal{XVI}) e, em seguida, da corrupção dos
ordenamentos republicanos ou da forma (cap. \versal{XVII}). Prosseguindo nessa
escalada da corrupção política, o cap. \versal{XVIII} vai direto ao grau máximo
de corrupção na cidade, quando ela se torna \emph{corrompidíssima}
(sic). Neste estágio de corrupção ampla, as consequências são ou a
mudança de regime ou a dissolução da república como entidade política
dotada de liberdade. Seja como for, qualquer uma das consequências é
contrária à vida política republicana, ao \emph{vivere libero,} visto
que uma condição essencial da vida republicana é a liberdade, ao ponto
do regime republicano ser nomeado, às vezes, como o regime da liberdade.
Na verdade, chegamos ao grande problema enunciado pelo título do
capítulo \versal{XVIII}: ``\emph{De que modo nas cidades corrompidas se podem
conservar um Estado livre, sendo-o; ou, não o sendo, ordená-lo}''
(\emph{Discursos}, \versal{I}, 18, linha 1). A questão está em tentar pensar em
uma solução para aqueles casos nos quais a corrupção não está apenas
localizada numa parte do corpo político ou permanece restrita à matéria
ou à forma, mas quando se encontra disseminada por toda a cidade. Uma
resposta já nos é possível constatar, pois não se pode conservar o
\emph{vivere libero} em condições de extrema corrupção, nas quais o povo
já não mantém a civilidade, e em que as leis são inadequadas e os
ordenamentos não conseguem mais frear as ambições desmedidas dos
diversos grupos políticos. As condições de possibilidades para a
retomada da liberdade republicana já não figuram mais no horizonte.
Diante, então, dessa condição extrema, a possibilidade de retorno, de
uma retomada à normalidade republicana é uma impossibilidade dentro da
lógica de ação política da república, pois, com uma matéria corrompida,
as leis são inadequadas e os ordenamentos políticos ineficazes e,
conforme o grau de corrupção, corrompidos em suas deliberações.
Maquiavel é categórico: De tudo o que dissemos acima provém a
dificuldade ou a impossibilidade de nas cidades corrompidas conservá-las
como republicas ou criá-las de novo (\emph{Discursos}, \versal{I}, 18, linha 28).

Neste contexto, pode-se até perguntar se ainda há ou não liberdade, ou
melhor, se o \emph{vivere libero}, característico da república, ainda
persiste ou se alguma força autoritária teria tomado as rédeas das
decisões políticas. Uma das características dessa corrupção republicana,
talvez a preponderante, está na força política que a aristocracia assume
e como ela passa a deliberar conforme os seus desejos. Pensando numa
cidade em tais condições políticas, mas não somente isso, sendo o povo
impedido de lutar pelos seus direitos, tal quadro é uma descrição de um
caso de corrupção republicana típico. Nestas circunstâncias, extingue-se
a liberdade de uma parte do corpo político, extingue-se a luta política
e um só grupo passa a ditar o caminho. No entanto, a corrupção também
pode extrapolar um grupo político restrito e atingir a todos
(\emph{universale}), circunstância esta caracterizada, entre outros
aspectos, pela perda dos valores cívicos, da civilidade. Também neste
caso não há mais espaço para a luta política, para o \emph{vivere
libero}.

Logo, não importa em que condição se manifeste a corrupção, ela figura
sempre como uma oposição à liberdade, ou como diz Sasso, ``\emph{a
recíproca repugnância entre liberdade e corrupção}'' (\versal{SASSO}, 1987, p.
407). Esta imagem ilustra os termos da dificuldade, pois a vida política
republicana é avessa à corrupção, é o pólo contrário à condição política
corrompida de uma cidade. De fato, se há uma manifestação de corrupção
política, isso implica proporcionalmente na anulação da liberdade; ou,
conforme a corrupção se amplia, por uma proporção inversa, diminui o
grau de liberdade da cidade. O que não quer dizer que a corrupção seja o
antônimo de liberdade, pois, conforme o nível de corrupção, tem-se uma
gradação inversa de liberdade: quando o grau de corrupção da cidade é
baixo, é possível que exista ainda o \emph{vivere libero}. Porém, em
qualquer condição em que haja um aumento de um, automaticamente ocorre o
decréscimo do outro, pois a coexistência de ambos com mesma intensidade
é impossível. Repugnância que não diz respeito apenas à liberdade, mas
pode estender-se à civilidade (entendida como o respeito mínimo às
regras políticas), às regras cívicas, quando se considera a corrupção da
matéria. Ou como dirá Maquiavel no capítulo \versal{LV} desse livro \versal{I} dos
\emph{Discursos}, refletindo acerca da corrupção presente quando os
\emph{gentis-homens} dominam o poder: ``Do que nasce que naquelas
províncias não surja nunca alguma república nem algum \emph{vivere
politico}; porque tal geração de homens são em tudo inimigos de toda
civilidade'' (\emph{Discursos}, \versal{I}, 55, linha 21). A corrupção se
opõe, pois, à república, mas, mais ainda, à civilidade e até mesmo ao
\emph{vivere politico}, de modo geral. Esta afirmação amplia o problema,
pois a corrupção não é somente contrária à liberdade, mas contrária à
vida política, um adversário à normalidade política. Opondo-se a uma
consideração que banaliza o papel que pode chegar a desempenhar a
corrupção no corpo político, Maquiavel confere cores fortes e afirmações
contundentes para descrever a importância das suas consequências para a
vida política da cidade. A manifestação da corrupção não deve ser
tratada como mais um evento possível em uma cidade, mas um grande
problema, um grande perigo para o corpo político como um todo. A
corrupção não é mais uma dificuldade presente no cotidiano político das
repúblicas, mas se torna o problema, a questão a ser tratada.

Todavia, quando esse problema não é passível de solução pelos próprios
mecanismos políticos da república, então, deve-se buscar meios mais
fortes e eficazes para freá-la. Entre as soluções, está a instalação de
um \emph{poder quase régio} ou do \emph{poder régio}.

A cidade diante, pois, de um caso de extrema corrupção, deve mudar o seu
regime, tendo, a princípio, duas possibilidades: o governo \emph{régio}
ou o governo \emph{popular}. Maquiavel reiteradamente identifica na
ambição desmedida da aristocracia a principal causa de corrupção. A
corrupção do povo, quando nasce, é um aspecto secundário, sendo muito
mais fruto da falta de freios à insolência dos grandes do que da perda
de civilidade do povo. Ora, a parcela popular da cidade estaria
habilitada, pelas suas qualidades, para, em tese, assumir o comando da
cidade nas condições de corrupção extrema, desde que não tivesse perdido
também todos os seus valores cívicos. Contudo, o problema não é assumir
o controle da cidade em função da sua capacidade ou por não estar tão
corrompida, a questão que se põe é se esse governo popular seria capaz
de colocar um fim à corrupção endêmica e reordenar a cidade.

Em função da grande insolência que, em geral, assola a cidade
corrompidíssima, a solução dada por Maquiavel não é nem sua conversão
num governo popular e nem num monárquico, mas em algo intermediário: no
poder quase régio. Diz ele:

\begin{quote}
Mas, em se precisando criar ou conservar uma {[}república{]}, seria
necessário, antes, reduzi-la ao estado régio do que ao estado popular;
para que os homens insolentes, que não pudessem ser corrigidos pelas
leis, fossem de algum modo freados pela autoridade quase régia
(\emph{Discursos}, \versal{I}, 18, linha 29).
\end{quote}

A solução pelo governo \emph{quase régio} é, na verdade, a justificação
para a instalação de um ordenamento republicano de Roma: a ditadura. Na
república romana, a figura de um ditador, que concentrava poderes
extraordinários durante um período limitado de tempo, era uma solução
prevista para casos especiais, como guerras e revoltas civis. O ditador
romano era um magistrado especial, escolhido pelo senado com função
específica para realizar alguma missão extraordinária. Com a instalação
do ditador pelo Senado romano, cessariam automaticamente os poderes dos
cônsules e dos outros magistrados, que passavam a subordinar-se ao
ditador (\versal{NICOLET}, 1964; \versal{CIZEK}, 1990). Este ditador romano difere em
muito, contudo, da imagem dos ditadores contemporâneos, pois sua
instalação e sua ação eram reguladas e submetidas à fiscalização e ao
controle do Senado romano, ou seja, ele não teria poderes políticos e
jurídicos absolutos.

Ora, quando Maquiavel pensa num governo \emph{quase régio}, dotado de
poderes extraordinários, ele tem em vista tanto a instalação da ditadura
quanto de um principado nos moldes romanos. No capítulo \versal{XXXIV} do livro
\versal{I}, ele diz:

\begin{quote}
Alguns escritores condenaram os romanos que encontraram um modo de
instituir a ditadura, como algo que, com o tempo, deu ensejo à tirania
em Roma. {[}\ldots{}{]} E vê-se que o ditador, enquanto foi designado segundo
os ordenamentos públicos, e não por autoridade própria, sempre fez bem à
cidade. Pois o que prejudica as repúblicas é fazer magistrados e dar
autoridade por vias extraordinárias, e não a autoridade que se dá por
vias ordinárias: e vê-se que em Roma, durante tanto tempo, nunca ditador
algum fez nada que não fosse o bem à república (\emph{Discursos}, \versal{I},
34, linhas 2; 5-6).
\end{quote}

O problema que pode advir a esses governos com poderes extraordinários
está no modo como nascem. Caso sua autoridade tenha sido delegada por
via ordinária, ou seja, dentro das regras políticas da república, sem
uma exacerbação de força por meio da violência, então não há nenhum
problema maior e os efeitos serão bons. A preocupação de Maquiavel
reside, fundamentalmente, no modo como ocorre a instalação desse
governo, no caso, por um meio não violento, respeitando a dinâmica
política republicana. Por se originar em tal quadro, o ditador detinha
um poder extraordinário, porém limitado, o que era uma garantia de, ao
final de seu mandato, o retorno à normalidade republicana:

\begin{quote}
De modo que, somando-se o breve tempo de sua ditadura, a autoridade
limitada que ele tinha e o fato de o povo romano não ser corrompido, era
impossível que ele saísse de seus limites e prejudicasse a cidade: e
pela experiência se vê que sempre foi proveitoso (\emph{Discursos}, \versal{I},
34, linha 10).
\end{quote}

Uma primeira solução para a república corrompida é a utilização de um
mecanismo republicano, o ditador, que concentra o poder para que possa
dar conta de um problema extraordinário, que, pelas vias ordinárias
republicanas, não poderiam ser sanado. Como sugere Bausi, além desse
ditador ao estilo romano, Maquiavel também tinha em mente como exemplo
desse poder quase régio os \emph{gonfalonieri} florentinos, que foram
governantes com poderes centralizados, mas em repúblicas\footnote{De
  fato, é possível fazer várias aproximações entre as funções e encargos
  dos ditadores romanos e as atribuições iniciais do \emph{gonfaloniere}
  Solderini, em 1494. Contudo, depois da reforma política de 1502 que
  institui o \emph{gonfaloniere a vita}, ou seja, perpétuo, convém
  associá-lo mais ao \emph{princeps rei publicae} do que ao ditador
  romano (\versal{BAUSI}, 2002, p. 117, nota 39).}. Esta solução é sugerida em
outras passagens\footnote{Cf, Livro \versal{I}, \versal{II}, 33; \versal{XXXIV}, 20; Livro \versal{III},
  \versal{XXVIII}, 14}, sinalizando um momento intermediário que, uma vez
fracassado, não deixaria escolha senão a instalação de um regime com um
governante com poderes políticos absolutos, uma autocracia de fato. A
vantagem dessa solução intermediária é que ela garante uma exigência
fundamental para a república corrompida, já que instala um governo de
força sob a égide do regime republicano. Sem abolir totalmente os
valores cívicos do republicanismo, o ditador ou o \emph{gonfaloniere},
por seu caráter extraordinário e temporário, visto que tinha mandatos
definidos que poderiam ou não ser renovados, seria um governo forte em
regimes republicanos enfraquecidos pela corrupção com vistas
exclusivamente à reordenação da cidade, o que por si só é um risco, haja
vista que não se tem a certeza de que eles serão bem sucedidos. De
qualquer modo, a condição extraordinária da corrupção -- pois ela é, no
limite, ruptura da vida política ordinária -- exige uma solução também
extraordinária, que ultrapasse alguns aspectos da normalidade
republicana, a fim de que se restaure a ordem. Os ditadores ou os
\emph{gonfalonieri} são medidas extraordinárias para circunstâncias
políticas extraordinárias. Como diz Maquiavel:

\begin{quote}
Quanto a inovar tais ordenamentos de uma só vez, quando todos reconhecem
que não são boas, digo que essa inutilidade, quando facilmente
reconhecível, é difícil corrigi-la; porque, para tanto, não basta usar
medidas ordinárias, visto que os modos ordinários são ruins; mas é
necessário recorrer ao extraordinário, como a violência e as armas,
tornando-se, antes de mais nada, príncipe em tal cidade, para poder
dispô-la a seu modo (\emph{Discursos}, \versal{I}, 18, linha 26).
\end{quote}

Ou ainda, como diz ao final do capítulo \versal{XVII}:

\begin{quote}
Porque tal corrupção e pouca aptidão à vida livre nascem de uma
desigualdade existente na cidade, e quem quiser dar-lhe igualdade
precisará lançar mão de meios extraordinários {[}grandissimi
straordinari{]}, o que poucos sabem ou desejam fazer (\emph{Discursos},
\versal{I}, 17, linha 16).
\end{quote}

Portanto, mesmo tendo à disposição esse meio extraordinário de reforma,
legítimo e previsto dentro do regime republicano, tal solução, apesar de
possível, não parece ser, contudo, a mais adequada para a cidade
corrompidíssima. Uma outra hipótese é a instalação de um governo que,
apesar de centralizar a autoridade em um indivíduo, consiga conservar um
mínimo de civilidade ou até mesmo recuperar a dinâmica republicana.
Governo esse que pode ser compreendido como um certo tipo de principado,
e não todo e qualquer principado, no caso específico, o principado
civil.

\subsection{Sobre a noção de principado em Maquiavel}

A crise de corrupção das repúblicas nos leva, pois, aos governos de
força, sejam eles régios ou quase régios. Uma das sugestões apontadas
por Maquiavel é o principado e, seguindo esse viés interpretativo,
cumpre pensar o principado a partir desses pressupostos dados pela
corrupção republicana. Esse é um ponto nodal de nossa leitura: como
pensar o principado e o príncipe expostos por Maquiavel em \emph{O
Príncipe} a partir desses pressupostos fornecidos pela corrupção
republicana?

Tendo em vista esse pressuposto interpretativo, algumas questões
formuladas de início tornam à baila: quem é esse príncipe -- que na
definição do texto é antes de tudo um \emph{privato ciptadino} -- que
assume o papel de liderar e conduzir a cidade? Seria ele um típico
monarca ou uma figura política diferente? Se o príncipe não é o monarca,
então, como entender o principado? Seria este o território ou lugar da
ação política do príncipe, identificando-se aquilo que nós entendemos em
nossos dias como principado ou reino? Ou seria ele outra coisa?

Antes mesmo de partir para a busca da resposta sobre o personagem
político que é o príncipe, faz-se necessário indagar antes o que
Maquiavel entende por principado e como essa noção é mobilizada na
reflexão desenvolvida em \emph{O Príncipe}. Convém relembrar,
inicialmente, que o título original da obra era \emph{De Principatibus},
ou seja, \emph{Sobre os Principados}, restando claro que seu autor
pretendia dissertar sobre esse tema ao longo do livro e não sobre a
figura do príncipe em primeiro plano, como o título atual sugere. Em
vista disso tudo, faremos uma explanação em duas etapas: uma primeira
sobre as noções de monarquia e principado que chegam ao contexto do
Renascimento florentino, no qual esse uso vocabular está inserido, ou
seja, uma apresentação do problema em seu contexto discursivo e as
nuances terminológicas que alguns termos possuíam naquele momento. Uma
segunda investigação, buscando extrair do próprio texto maquiaveliano o
que se compreende por principado, a partir dessa noção de príncipe civil
que nos parece ser central.

\subsection{O principado no momento maquiaveliano}

A denominação dos cargos dirigentes e dos detentores desses na Europa do
século \versal{XV} deve ser considerada, inicialmente, nos contextos políticos
particulares de cada território. Como o próprio Maquiavel aplicará ao
longo d'\emph{O Príncipe}, cada lugar possuía suas peculiaridades
em termos de organização política, de modo que unificar ou tentar
generalizar terminologias e designações pode implicar em equívocos
sérios. Por exemplo, nos territórios que hoje conhecemos como a
Alemanha, havia vários príncipes e um imperador, sendo que esses
príncipes não eram filhos desse imperador ou seus sucessores diretos. O
mesmo se diga para o líder político do ainda recente, para os latinos,
governo turco, designado por Maquiavel tão somente como ``O Turco''. Em
geral, assumia-se que o imperador era o governante de um império que
possui vários reinos ou principados. Mas mesmo neste caso, a denominação
é controversa, principalmente entre os regimes que reivindicavam a
herança do Império Romano, pois, tendo em vista a presença até 1453 do
Império Romano do Oriente, também conhecido como o Império bizantino,
todos os demais postulantes latinos à condição de Império disputavam com
Constantinopla esse reconhecimento, incluindo, nesse caso, até o Papado,
que após o século \versal{VIII}, em vários documentos, reivindicava a sua
condição de legítimo herdeiro do Império Romano cristianizado por
Constantino em 313, particularmente pelo documento falsificado
denominado ``Doação de Constantino''. Situação que se complica mais após
o século \versal{VIII}, com o surgimento do Sacro Império Romano, da dinastia
carolíngia, que se transforma posteriormente no Sacro Império Romano
Germânico, dominado pelos imperadores alemães. Historicamente, os
governantes dos impérios oriental e ocidental nunca reconheceram de fato
essas condições por inúmeros motivos, que não vem ao caso dissertar
aqui\footnote{Cf. Ostrogorsk, Georg. \emph{Storia dell'impero
  bizantino.} Torino: Einaudi, 1968. Esse longo estudo sobre o Estado
  bizantino mostra, a partir da ótica dos orientais, como as
  reivindicações dos latinos nunca foram plenamente aceitas por eles,
  donde a disputa constante pela herança do Império Romano durou
  séculos.}. Quadro esse que se amplifica com os novos reinos latinos de
Espanha e Portugal no século \versal{XV}, que se reivindicam também como
impérios. No início do século \versal{XVI}, portanto, um império já não era mais
entendido apenas como um sucessor direto do Império Romano ou como poder
político que se coloca acima dos reinos que governa, passando a ter uma
acepção polissêmica. A clássica designação de que o imperador está acima
do rei já não cabe para certos casos e não possui uma significação
precisa em alguns contextos políticos.

Tal problema de denominação pode ser extrapolado para os termos ``rei''
e ``príncipe'', que passam a ser ter várias acepções, embora, em geral,
verifique-se que o termo ``rei'' se aplica ao governante do reino. Tudo
isso sem levar em conta o sentido jurídico dessas denominações, o que
nos remeteria, por seu turno, aos debates entre os glossadores
medievais.

Tomando-se em conta o contexto histórico italiano do Renascimento,
particularmente, as lutas por autonomia das cidades do norte da
península itálica e a consequente implantação dos regimes republicanos
em várias delas, o uso desses vocábulos políticos ganha novos contornos.
Tendo em vista as guerras travadas pela busca de autonomia dessas
cidades contra as forças imperiais e papais, e depois as próprias lutas
internas contra aqueles que buscavam o domínio da cidade, contra os
\emph{Signori}, a denominação dos cargos políticos apresenta uma
variedade significativa. Ora, no contexto das repúblicas italianas do
Renascimento, o fato histórico da luta pela liberdade contra as tiranias
-- interna ou externa -- fez com que os usos do termo ``príncipe'' sejam
esvaziados dessa ligação ao regime monárquico, quando se refere a
personagens políticos deste contexto.

Isso fica claro, no caso de Florença, quando se analisam os escritos dos
autores políticos, particularmente dos homens da Chancelaria, como
Salutati, Bruni, Valla, Maquiavel, Guicciardini, Vettori, Buonaccorsi
etc. A designação ``príncipe'', aplicada a personagens italianos,
raramente se refere a pessoas ligadas a uma dinastia e trata-se, em
geral, de próceres políticos com cargos executivos. De tal modo que o
termo se aproxima mais da concepção de \emph{prínceps} latino, como o
primeiro entre os iguais, do que o herdeiro de uma dinastia.

Se para o termo ``príncipe'' é possível delimitar esse uso entre os
escritores florentinos, já para o termo ``principado'' parece que temos
ainda uma acepção muito mais próxima de monarquia, do que decorre a sua
definição como uma espécie de regime monárquico. De fato, o termo é
polissêmico e permite essa interpretação, que se reforça quando se leva
em conta os governos da família Medici ao longo do século \versal{XV},
caracterizada como um principado e que, para muitos, se tratava de fato
e de direito de um governo de tipo monárquico na cidade. O que nos
obriga, ao menos, em considerar melhor o que foi esse governo para
entender os possíveis sentidos dessa denominação de principado na
Florença do contexto maquiaveliano.

Como demonstrou Rubinstein (1997) e o próprio Maquiavel na
\emph{História de Florença}, após a revolta popular, também conhecida
como Revolta dos Ciompi, desencadeada em 20 de julho de 1378, uma série
de mudanças no ordenamento político da república florentina são
realizadas. A criação de novos conselhos e o modo como os grupos
poderiam ter acesso a eles foram uma das demandas que se transformaram
em possibilidade política legítima. Contudo, já em início do século, mas
principalmente a partir do governo de Cósimo de Medici, em 1434, o que
se verifica é um aumento da influência política dos setores
aristocráticos mais abastados sobre os demais grupos políticos\footnote{Conforme
  já explicamos no início, a simplificada distinção social em Florença
  entre ricos e pobre (\emph{popolo grasso} e \emph{popolo minuto}) não
  é correta, pois, segundo o próprio Maquiavel, haviam sub- divisões
  entre eles, tornando o equilíbrio de forças políticas mais complicado.}.
O governo sobre o controle da família Medici durante o século \versal{XV} era
formalmente um governo republicano, contudo, era dirigido conforme os
interesses dessa família, que não apenas possuía o cargo de comando, mas
também controlava o acesso aos demais conselhos. Tanto é assim que,
quando Piero di Medici perde o governo da cidade em dezembro de 1494, a
razão maior para isso está no fato dele ter rompido o delicado
equilíbrio de poder que havia entre os diversos setores da aristocracia
(Rubinstein, 2011). E mesmo a volta dessa família ao poder em 1512 tem
como fundamento a reconstrução dos apoios que ela possuía entre a
aristocracia que detinha o controle dos conselhos superiores da cidade.
Ora, em ambos os exemplos históricos, não é possível afirmar que se
tratava de um governo de tipo monárquico, muito menos de um governo
autocrático com outra denominação. De fato, como demonstra Rubinstein
(1997), os Medici dividiam o poder político na cidade e não concentravam
tudo em suas mãos. O fato de exercerem o controle sobre os conselhos não
nos permite dizer que esse fosse um governo autocrático. Nomear, pois,
tais regimes de principados, a partir do exemplo histórico, não pode
implicar uma identificação desses regimes sob o controle da família
Medici como monarquias.

\subsection{O principado de \emph{O Príncipe}}

Para além desses dados históricos, que não resolvem o problema, uma
análise do texto de \emph{O Príncipe} fornece os elementos mais
corroboradores, pois nos apresenta o que de fato Maquiavel entende por
principado.

Logo no início da obra, encontramos uma primeira definição de
principado: \emph{``Todos os estados, todos os domínios que tiveram e
têm autoridade sobre os homens, foram e são ou repúblicas ou
principados.''} {[}cap. 1, linha 1{]}

A definição inicial já nos revela algo, pois o principado é uma
autoridade (império), ou seja, um governo que se exerce sobre os homens.
Em seguida, ele distingue o principado da república, diferenciação esta
que ocorrerá outras vezes, não somente \emph{n'O Príncipe} em outros
momentos, mas nas suas demais obras.

Contudo, se o principado realmente for um domínio sobre os homens,
poder-se-ia declarar que se trata de um governo que exerce sua força
sobre os homens, submetendo-os. Tal constatação é dedutível a partir da
noção de domínio herdada da antiguidade e reelaborada durante o período
medieval, que, grosso modo, é a transferência da relação existente no
interior da casa (\emph{domus}), entre senhor e escravo, para a esfera
pública. Neste caso, o principado é uma dominação política, no sentido
de um governo que não está aberto à interferência e não divide seu
controle ou primazia nas decisões. Entretanto, tais afirmações desse
modo ainda são precipitadas, pois o conceito teve apenas sua primeira
apresentação na obra, falta considerar o restante da primeira parte do
livro.

Na sequência ainda desse capítulo \versal{I}, Maquiavel distingue os principados
em hereditários e novos. Nesse momento, teríamos a primeira aproximação
dos principados com o regime monárquico, pois o principado hereditário
seria aquele no qual o controle está nas mãos de uma mesma família, ou
como diz, ``\emph{nos quais o poder ficou por longo tempo com a família
do príncipe}'' {[}cap. 1, linha 2{]}. Todavia, ainda que essa tradução
seja adequada, uma consideração sobre o texto italiano nos permite
verificar que Maquiavel está falando daqueles governos em que os membros
de uma dinastia foram por longo tempo príncipes (\emph{sia suto lungo
tempo principe}), o que admite também o entendimento de que essa família
liderou a cidade durante este ``longo tempo'', o que é um pouco
diferente do que afirmar que este principado é uma monarquia
hereditária. A distinção estaria na nuance entre liderar politicamente e
ter o domínio político, a autocracia das decisões políticas. Ora, como o
próprio autor não se alongará nessa análise, é temerário tecer hipóteses
sobre um tema não desenvolvido pelo filósofo. Atendo-se à letra do
texto, mesmo nesse principado hereditário, Maquiavel parece destacar a
capacidade de liderança política desta dinastia e não a sua condição
monárquica.

O outro principado em tela, o principado novo, se constituirá, na
verdade, no tema principal do livro, pois, seja nas considerações sobre
o principado, seja nas considerações sobre o príncipe, Maquiavel volta
suas atenções para o principado novo, em suas várias manifestações, e
para o príncipe novo e sua necessidade de conservação do governo.

Neste primeiro momento do livro, sua preocupação está dirigida para a
conquista do principado. O termo ``conquista'' aqui pode ter uma dupla
acepção, em função daquilo que se entende como principado: se principado
for compreendido como um território, então essa conquista é tal qual uma
ocupação fruto de uma campanha militar, por exemplo; nesse sentido, uma
conquista de um local ou território. Mas, haveria ainda a possibilidade
da conquista se referir somente ao controle político do governo, à
esfera política da cidade prioritariamente. Neste caso, trata-se de
ressaltar a dimensão política da cidade, e não seu aspecto territorial,
talvez mais próxima da definição antiga, encontrada, por exemplo, em
Aristóteles (\emph{Política}, \versal{III}, cap. 1-2), no qual a definição de
cidade (\emph{polis}) não se identifica a um território, de tal modo que
a cidadania não é assegurada a quem habita aquele território. O exemplo
do caso da Alemanha citado no capítulo 10 d'\emph{O Príncipe} pode ser
bem ilustrativo disso, pois o rei tem sua esfera de ação política, mas
não detém o controle dos territórios governados pelos príncipes. Assim,
um caso de controle político não implica necessariamente, embora muitas
vezes seja o caso, controle de um território, fornecendo o escopo dessa
dificuldade de compreensão da conquista do principado.

Porém, uma nova ordem de problemas se articula a esses, na verdade, um
problema anterior e mais profundo, que diz respeito ao que entender por
conquista. Talvez aqui esteja o cerne político do problema, pois a
dificuldade está em determinar qual a relação política que esta tomada
do governo estabelece: se é uma relação política na qual o conquistador
domina toda a esfera de comando, exercendo seu poder com domínio
soberano, ou se esta relação implica em um postar-se constantemente na
disputa pelos apoios políticos que permitem o exercício do governo. O
desenvolvimento do texto ater-se-á justamente a esse ponto nevrálgico da
tomada do governo, mais do que ficar listando tipificações dos modos de
se adquirir um principado. É na determinação das relações políticas que
tal investida causa que Maquiavel fixa suas atenções. Logo, não se
atinge os elementos centrais do texto quem se concentra somente na
análise dos tipos de conquista, conforme dá a entender o texto num
primeiro momento, e se esquece de atentar para o modo como as relações
políticas estão sendo construídas, relações estas que formarão os
alicerces desse novo governo.

Desviando um pouco a atenção para a exposição do argumento, é importante
que logo de início o leitor perceba uma característica do estilo de
exposição maquiaveliano. Por ser Maquiavel um escritor hábil, sua função
principal durante todo o período de trabalho na Chancelaria foi escrever
relatos, habilidade esta que ele teve que aprimorar, pois não somente a
clareza deveria ser uma marca dos seus textos, mas, por outro lado,
quando se fazia necessário, convinha produzir um relato que dirigisse os
seus leitores a tomar a decisão que ele entendia a mais adequada. Donde
a necessidade de se usar recursos retóricos, mas com tal sutileza que o
leitor não percebesse essa sub-intenção. Ora, um profissional exercitado
por longos anos nesta arte da escrita e que era conhecido por ser também
teatrólogo -- não nos esqueçamos que em vida Maquiavel foi reconhecido
em Florença mais como o autor da peça \emph{Mandragora} do que como
personagem político --, elabora um texto que deve ser lido com atenção e
levando em conta as diversas dimensões da linguagem e seus efeitos.
(\versal{ADVERSE}, 2009)

Retomando o roteiro argumentativo inicial, o livro se abre, então, com
uma rápida definição de principado, e não de príncipe, o que já é muito
significativo, e passa para sua tipologia de conquistas: pelas armas,
próprias e alheias, pela \emph{virtù} e pela fortuna. Encerrado essa
breve apresentação, que é, na verdade, um roteiro dos temas a serem
tratados ao longo da primeira parte do livro, o parágrafo inicial do
capítulo \versal{II} é também emblemático.

A frase de abertura do capítulo \versal{II} já provocou inúmeras discussões entre
os comentadores, pois diz Maquiavel: ``\emph{Deixarei de lado a
discussão sobre as repúblicas, porque, em outro momento dissertei
longamente sobre elas}'' (cap. 2, linha 1). Não vamos retomar aqui a
discussão relativa a qual livro ele estava se referindo e se esse livro,
em geral entendido como sendo os \emph{Discursos}, foi ou não escrito
antes d'\emph{O Príncipe}, conforme já abordamos. Apenas gostaríamos de
chamar a atenção para o fato de que num livro dedicado aos principados,
em dois momentos iniciais, mais exatamente num espaço de cinco linhas,
Maquiavel faz duas referências diretas à república. Uma primeira
constatação é óbvia, o principado se coloca, já em seu primeiro momento,
como um contraponto à república, o que é inegável. Com efeito,
parece-nos evidente que o principado é um regime diferente da república.
A questão é: por que essa insistência em contrapô-los? Talvez ainda o
texto não nos permita apontar uma resposta suficiente para essa
dificuldade, todavia, a segunda linha do capítulo fala do ``\emph{sangue
do senhor que é por longo tempo príncipe}''. Ora, tal ideia poderia ter
uma outra forma de redação se se tratasse tão somente de monarquias.
Neste caso, bastava dizer que aquela dinastia ou aquela família real
detém o comando político há muito tempo, conforme já tratamos.
Voltando-se para mobilização de república aqui, verifica-se, por apenas
esses elementos, que Maquiavel percebe um paralelo entre principado e
república que exige uma distinção. A diferenciação será apresentada
adiante, porém fica a dúvida acerca do porquê do paralelo. Esse paralelo
ou proximidade entre a república e o principado ficará mais claro quando
da análise do principado civil, mas, pelo exposto, podemos antecipar com
segurança que esse o principado terá na sua estruturação política e na
dinâmica das suas ações algumas semelhanças com as repúblicas,
particularmente no que diz respeito ao lugar dos conflitos políticos, e
se diferenciará, por outro lado, das monarquias ou governos
autocráticos.

Na sequência, Maquiavel apresenta de fato a sua preocupação maior, que
perpassará toda a obra: ``Ocupar-me-ei somente do principado, retecerei
as urdiduras acima descritas, e demonstrarei como estes principados
podem ser governados e conservados'' (cap. 2, linha 2). Na verdade, a
preocupação são duas: o modo de governar os principados e como eles são
conservados, mas que em vários momentos se confundem, pois, no modo de
governar, já devem estar implícitas as estratégias de conservação desse
próprio governo. Agora temos, então, um panorama geral do que será essa
obra: uma análise da conquista dos principados, de como governá-los e
conservá- los. Basicamente sobre esse tripé é que se desdobrarão os
demais temas.

A exposição propriamente dita dos principados começa pelas atenções ao
principado hereditário e aos mistos. Sobre os hereditários, como já
dito, a brevidade da exposição é o dado mais chamativo. Nas poucas
quatro linhas dedicadas ao tema, ele deixa como regra geral que esse
príncipe herdeiro não precisa usar de meios extraordinários para
conservar o seu governo, visto que não teve que conquistá-lo. Ora, basta
a esse que herdou o governo do principado ter uma ``indústria
ordinária'', ou seja, não realizar grandes inovações políticas, seguir o
curso ordinário das coisas. Isso pode soar como um convite a uma conduta
medíocre, como aquele que pauta suas ações pela mediana geral dos
governantes, sem grandes iniciativas políticas. Se for isso, nada mais
disforme ao que será exposto, pois, se é possível dizer algo do
príncipe, é que ele deve ultrapassar o plano do ordinário, da mediania
em termos de ações políticas. Talvez seja então até por isso, por essa
mediocridade inerente ao príncipe herdeiro, que se justifique o fato de
Maquiavel não dedicar maior atenção a ele.

O passo seguinte é uma análise dos principados que são conquistados por
alguém que já detém o comando de um outro principado. Esse novo governo
é, para o conquistador, um principado. Detalhe sutil, mas que indica
muito, pois alguém que já comanda um principado e conquista outro, em
tese, não terá grandes dificuldades na direção deste novo governo, pois
já conhece os modos como governar uma cidade e, portanto, se conservam
esses governos. Entretanto, e novamente vemos o estilo maquiaveliano
desconcertar o leitor, não parece ser tão fácil assim para esse
conquistador conservar o governo dessa sua conquista política. A
dificuldade principal desse novo principado indica o grande problema de
toda a conquista política, até mesmo para alguém que já governa, a
saber: como obter apoios na cidade ocupada?

O problema apresentado no início do capítulo \versal{III} revela a preocupação
política que subjaz a conquista: o governo principesco não se vale por
si só, mas pela capacidade de angariar apoios que o sustentem, aquilo
que contemporaneamente chamamos de legitimidade. Isso fica claro quando
Maquiavel diz que, ``\emph{mesmo que se tenha um fortíssimo exército
seu, sempre se precisa da ajuda dos provincianos para entrar em uma
província}'' (cap. 3, linha 3). Portanto, esse conquistador nunca é tão
poderoso, mesmo quando se vale de um forte exército. Como ele
demonstrará por diversos exemplos, esse conquistador precisa fazer uma
série de ações no intuito de trazer para si apoios políticos que lhe
permitam, de fato, se constituir como um líder político daquela cidade,
o que não se consegue apenas tendo um exército forte, apenas pela força
das armas. Pode-se afirmar, olhado por outro lado, que a força do
governo não está no poderio militar ou nas armas somente, mas se erguerá
também, e principalmente, sobre as alianças e os vínculos políticos que
se consiga estabelecer nesse novo governo.

A busca de sustentação política será, assim, a tônica das preocupações
daquele que ascende à condição de príncipe. Fortuna, \emph{virtù}, armas
próprias e armas alheias, são todos elementos da conquista que remetem
sempre, cada um a sua maneira, ao modo como, num segundo momento, esse
conquistador político recebe o apoio político necessário para a
sustentação do governo. Dentro desta lógica argumentativa, mas por uma
outra perspectiva, nos diversos exemplos históricos mobilizados entre o
capítulo \versal{III} e o capítulo \versal{XI}, quando eles não são apresentados para
corroborar essa necessidade de buscar apoio político para o governo que
está se instalando, esses exemplos cumprem a função contrária, ou seja,
revelam o quanto os governantes sozinhos ou fundados unicamente na força
solitária do príncipe não tem a força política necessária para a
conservação do governo.

Talvez o exemplo mais eloquente dessa necessidade de apoio político do
príncipe novo seja a figura do cidadão (\emph{privato ciptadino}) que
ascende à condição de príncipe. Nesse exemplo, todas as nuances daquilo
que desde as primeiras linhas do livro se apresentavam como necessidade
para a constituição desse governo novo, revelam sua forma mais acabada.
Adiante nos ocuparemos como mais atenção sobre esse tipo político,
todavia, convém aqui apenas apontar alguns aspectos desse exemplo, tendo
em vista nosso objetivo de definição do que seja o principado.

O cidadão que ascende à condição de príncipe, modelo privilegiado que
personifica o príncipe novo por Maquiavel, é certamente alguém que, após
uma série de ações políticas, algumas calcadas na fortuna, mas a maioria
na sua \emph{virtù}, consegue o apoio da cidade para assumir o comando
político. Mesmo que ele tenha se valido da fortuna até a sua chegada ao
governo, doravante ele não se pode valer somente desta para manter-se no
comando político da cidade. O mesmo se diga das armas e das ações
cruéis: elas até podem possibilitar a conquista do governo, mas não se
constituem como fundamento seguro para o exercício dele.

Toda a argumentação culmina para o caso do cidadão que ascende à
condição de príncipe ``com o favor dos outros cidadãos'', ascensão essa
baseada ou no favor do povo ou no favor dos grandes. Esse caso torna-se
emblemático porque expõe a real necessidade deste indivíduo que deseja
ser príncipe. Primeiro, ele deve reconhecer que as forças políticas
estão em disputa para além da sua própria força política, visto que ele
não é a única fonte de força política ou a sede do poder; segundo, que
existem outros atores políticos nesse palco, que podem possuir maior ou
menor influência no jogo conforme as circunstâncias; terceiro, e o mais
importante, que o príncipe novo deve se inserir nessa disputa inerente à
vida política da cidade e saber conduzi-la, seja para a não dissolução
desse regime político, que pode ocorrer por meio da instalação de uma
tirania interna ou externa, seja para a própria conservação da sua
condição de figura política de destaque.

Após ter apresentado o tema a ser dissertado na primeira linha do
capítulo \versal{IX}, os dois períodos seguintes consolidam o argumento, como diz
Maquiavel:

\begin{quote}
Porque em toda cidade se encontram estes dois humores diversos e nasce,
disto, que o povo deseja não ser nem comandado nem oprimido pelos
grandes e os grandes desejam comandar e oprimir o povo. Destes dois
apetites diversos nasce na cidade um destes três efeitos: ou o
principado, ou a liberdade ou a licença. O principado origina-se do povo
ou dos grandes, segundo que uma ou outra destas partes tenha a ocasião,
porque, vendo os grandes que não podem resistir ao povo, começam a
aumentar a reputação e o prestígio de um dos seus e fazem-no príncipe
para poderem, sob sua proteção, desafogar o seu apetite. O povo, também,
vendo que não pode resistir aos grandes, aumenta a reputação de um e o
faz príncipe, para serem defendidos pela sua autoridade (cap. \versal{IX},
linhas 2 e 3).
\end{quote}

A primeira informação importante é que a cidade é composta de dois
humores, ou duas partes antagônicas. Num primeiro olhar, somos tentados
a pensar o principado tomado por esse antagonismo, mas note-se que não
são os regimes políticos e sim a cidade que, em sua constituição, em seu
substrato material, tem esse antagonismo político inerente. Tal oposição
natural -- e aqui convém insistir sobre esse aspecto, visto que as
partes ou humores compõe a natureza da cidade e são, portanto,
indissociáveis -- causa os desejos diversos entre os grandes e o povo:
sendo que os primeiros desejam comandar e os segundos em não ser nem
comandados e nem oprimidos. Ora, a combinação desses humores, tal qual
concebida pela tradição da medicina galênica e da hipocrática (nas quais
a combinação deles gerava os diversos tipos de temperamento), resulta em
três formas de governo: o principado, a liberdade e a licença\footnote{Sobre
  essa relação entre a medicina galênica e seus usos por Maquiavel, cf.
  Nicodimov, 2004.}. Esse trecho resultou numa vasta literatura de
comentários a respeito da teoria dos humores e do conflito político em
Maquiavel, que não vamos retomar aqui\footnote{No Brasil temos um
  capítulo significativo desta discussão. Cf. Ames\ldots{};
  Adverse\ldots{}; Bignotto\ldots{}; Cardoso\ldots{}; Martins\ldots{}}.
Entretanto, para nossas intenções, já é possível ver que o principado
não se confunde com a cidade, mas é um dos modos de sua ordenação em
função da combinação dos humores. Destaque-se que a cidade ainda pode
ser ordenada como liberdade (sinônimo de república) e como licença,
neste último caso, trata-se de um modo de ordenação política desregrado,
que se degenera em uma tirania (\emph{Discursos}, \versal{I}, cap. 2).

Em seguida, Maquiavel mostra como nasce o principado, a saber: ou quando
o povo escolhe alguém e lhe dá o apoio ou quando os grandes escolhem um
dos seus para fazê-lo príncipe. A sequência dos eventos revelará que
esse novo príncipe, na verdade, um cidadão (\emph{privato ciptadino}),
deve saber como buscar sustentação para o seu governo, apoio este que
deve, se possível, fundar-se nas duas partes políticas: o povo e os
grandes.

Neste ponto, já nos é possível entender melhor o que é o principado. Ele
é uma autoridade sobre a cidade que nasce a partir da disputa entre os
humores. Parece evidente, portanto, que o principado é, antes de tudo,
uma forma de governo, dentre as possíveis, para comandar uma cidade. Se
isso está claro, então, se voltarmos ao início do texto e repassarmos a
sequência argumentativa, veremos que Maquiavel trata dos vários modos de
conquista de um governo. A conquista do principado é, pois, a ação de
chegar ao posto de comando da cidade, seja usando as armas, seja usando
a fortuna própria ou a alheia, seja usando a \emph{virtù}. Uma vez
conquistado o comando da cidade, o governo, isso não implica em um
controle total das decisões, esse príncipe novo não é um príncipe
absoluto, pois deve decidir levando em conta essa dinâmica das oposições
e saber lidar com os diversos interesses, seja para não ficar refém de
um desses grupos, seja para não constituir inimigos fortes que ameacem o
seu governo.

Assim, podemos deduzir com certa facilidade que o principado é nomeado
deste modo, porque é a forma de governo sob o comando de um príncipe,
particularmente de um príncipe novo. Todavia ainda resta alguns
problemas lançados antes: por que o paralelo recorrente entre o
principado e a república? Se esse principado não é uma república, o que
parece muito evidente, por que ele também não nos permite identificá-lo
à monarquia, ao governo autocrático?

Há um fator que, de certo modo, responderia as duas questões. Como se
verifica não somente ao longo d'\emph{O Príncipe}, mas também nos
\emph{Discursos}, na \emph{História de Florença} e em diversos opúsculos
políticos maquiavelianos, é em função dos conflitos políticos que a vida
política da cidade deve ser pensada. Se nas repúblicas há mais vida e
mais ódio (\emph{O} \emph{Príncipe}, cap. \versal{V}, 9), é porque as lutas
políticas exigem maior engajamento das pessoas, elas devem tomar partido
nas disputas políticas, visto que o governo da cidade é resultado do
conflito. Ora, se no principado não temos a partilha do comando da
cidade, pois o governo se encontra nas mãos do príncipe, isso não
implica que esse governante, que pode até decidir sozinho, não tome tais
decisões a partir de sua vontade privada. Como se verifica nos vários
exemplos históricos citados, muitas vezes esse príncipe é premido a
tomar decisões contrárias aos seus reais interesses em função da
conservação do governo e do bem estar político da cidade. Talvez, nesse
caso, o exemplo mais eloquente seja o de César Borgia que manda matar
seu braço direito, Ramiro Orco. No principado, então, o príncipe toma a
decisão, indica o rumo político a seguir, mas não o faz necessariamente
de \emph{motu} próprio, decide premido pelas circunstâncias, delimitado
pelos interesses diversos que tensionam o seu governo. O exemplo da
possível conquista do reino francês também é ilustrativo
(\emph{Príncipe}, cap. \versal{IV}), visto que não adianta apenas derrubar o rei,
deve-se ter em conta as disputas políticas que ocorrem abaixo dele entre
os nobres e saber o modo como se inserir nelas para conservar o governo.

Então, tal governo principesco se diferencia de um governo de tipo
monárquico clássico, cujo exemplo é o governo do ``Turco'' citado no
capítulo \versal{IV}. Note-se que o próprio Maquiavel, ao tratar do principado
civil no cap. \versal{IX}, aponta para os riscos desse principado tornar-se um
principado absoluto (\emph{Príncipe}, cap. \versal{IX}, 23-27), no qual o
príncipe concentrasse em si todas as decisões. Ora, seja no caso do
governo otomano, seja no caso desse príncipe absoluto apontado no final
do capítulo \versal{IX}, Maquiavel chama a atenção para o fato de que ele não
consegue ter todo o controle e todo o comando político da cidade. O
Turco precisava dos seus \emph{sandjacs} (administradores políticos
nomeados pelo imperador) e o príncipe absoluto precisa dos magistrados,
o que, em ambos os casos, resulta em partilha ou delegação das decisões
políticas para pessoas que obtiveram apoios que resultarão, no limite,
num enfraquecimento do governo desse príncipe absoluto. Esse
posicionamento político é resultado, assim, de uma análise do contexto
político a partir da dinâmica das lutas entre as partes e de um governo
que é obrigado sempre a buscar apoios para se manter. No interior desse
quadro argumentativo, não há poder que se firme como absoluto, pois o
governante deve sempre agir para obter apoio ou deixar de agir para não
criar oposições.

A monarquia ou o governo absoluto não é uma impossibilidade, segundo
Maquiavel, apenas não é nunca um absoluto em sua plenitude e nem um
governo forte. Pela argumentação construída nessa primeira parte do
livro, não há nenhum principado forte que esteja fundado em um governo
absoluto ou monárquico. Talvez um último exemplo que comprove
definitivamente essa ideia, que não por acaso é o último exemplo de
principado analisado, seja os principados eclesiástico, cujo maior
exemplo é o papado. Nem o Papa possui um controle absoluto do seu
governo, ao contrário do que parece ser a primeira vista. A exposição
desenvolvida no capítulo \versal{XI} é muito ilustrativa de tudo o que se
apresentou até então no livro. Primeiro, conforme o estilo maquiaveliano
de exposição, somos tentados a acreditar que o principado eclesiástico é
sim um principado com uma dinâmica política diferente, visto serem
mantidos pelo próprio Deus. Entretanto, ao longo do capítulo, vai
ficando claro que mesmo o papa tem que lidar com as disputas políticas
entre os dois principais grupos políticos, no caso histórico de início
do século \versal{XVI}, os Collona e os Orsini, que nada mais são do que partidos
em disputa pelo governo. Sem contar a própria disputa entre os cardeais,
que é sempre natural na dinâmica política eclesiástica, de modo que,
mesmo o papa, não tem tanta segurança ou facilidade para governar, caso
se imaginasse que fosse esse principado um tipo de principado absoluto.

Ao final dessa primeira parte do livro, podemos constatar que, por todos
os exemplos mobilizados, o principado não é uma monarquia clássica,
muito menos uma prefiguração das monarquias absolutas que vigorarão ao
longo dos séculos seguintes na Europa. Nas poucas vezes que essa
aproximação ocorre, o principado é retratado como um governo fraco,
noção essa radicalmente diferente das monarquias absolutistas que serão
retratadas como governos fortes, materialização dos governos dotados de
soberania política.

Contudo, ainda resta um último ponto: se os principados não são
monarquias de tipo clássico e nem repúblicas, como entender esse regime
político? A resposta já nos é bem presente. Em síntese, o principado é
um governo sob o comando de um príncipe, que é antes de tudo um líder
político que assume o governo da cidade, mas que isso não implica
necessariamente em um governo monárquico ou autocrático de qualquer
espécie. Tendo em vista o destaque deste tipo muito particular de
príncipe que é um cidadão que assume o governo, que, por isso, necessita
agir no interior das disputas políticas naturais entre os humores, vemos
que a dinâmica das disputas políticas próprias da república se conserva
ainda nesses principados, porém agora sob um novo quadro institucional.
Esse seja talvez o ponto de contato entre principado e república
destacado por Maquiavel, pois ambos regimes possuem essa dinâmica
política inerente à natureza da cidade, mas que, na república, essa
disputa se configura em outros termos e resulta em outro ordenamento
institucional diferente do principado. No principado, também as disputas
existem, contudo, elas são exercidas com outra dinâmica política, com
outras delimitações, que não as mesmas das repúblicas, e geram outros
efeitos em termos de ordenamentos políticos e leis. Como já dito antes,
não é somente os conflitos políticos dos humores antagônicos, mas o
quadro político próprio dos principados é que fazem desse governo um
regime diferente das repúblicas, sem que isso implique em um governo
autocrático.

Por fim, podemos ainda constatar que esse principado fundado na ação
política de um cidadão que se torna príncipe se identifica quase
totalmente ao regime ``quase régio'' apresentado ao final do capítulo
\versal{XVIII} dos \emph{Discursos} (cap. \versal{XVIII}, linha 29). Torna-se, pois,
evidente que a crise da república resulta primeiro nesse governo fundado
neste príncipe, que pode implicar, por seu turno, em três outros
governos: no reestabelecimento do ordenamento republicano (hipótese esta
não explorada n'\emph{O Príncipe}), na conservação do principado civil
por longo tempo e na transformação desse principado em uma monarquia,
possibilidade esta que sempre está posta no horizonte, visto que o
principado civil pode tornar-se absoluto, conforme as ações de
centralização política do príncipe gerar, na sequência, um governo
dinástico, uma monarquia. Assim como no \emph{Pequeno tratado sobre as
repúblicas}, Maquiavel nunca é peremptório ou enfático: os ciclos
políticos são possibilidades de mudanças políticas que se apresentam aos
povos. Isso comprova em outro texto aquilo que havíamos constatado nos
\emph{Discursos} (Martins, 2007): que o pensamento político
maquiaveliano se apresenta sempre como possibilidades de configurações
políticas, jamais em ordenamentos políticos que se realizam
necessariamente, como um ciclo político determinista. Essa é uma das
marcas da reflexão política de Maquiavel, de pensar o mundo da política
como possibilidade, no qual se apresenta sempre aos homens alternativas
para tentar direcionar o curso das coisas que pareceria natural e
determinado. Enfim, probabilidade de realização e não determinação
histórica.

\subsection{A herança dos \emph{espelhos de príncipes} na~noção~de~príncipe~maquiaveliana}

Uma vez reconhecido o que é o principado, importa agora entender o que é
o príncipe para Maquiavel, ou mais especificamente, qual a noção de
príncipe que é apresentada em \emph{O Príncipe}. Essa análise será feita
em duas partes, num primeiro momento, recuperando as noções medievais
que emergem da tradição dos ``\emph{espelhos de príncipes''} e, em
seguida, pela análise desse príncipe como um personagem republicano.
Desde início, convém destacar que estamos tratando desse príncipe que é
um cidadão comum que se torna príncipe novo, donde ser nosso foco:
entender como Maquiavel concebe o papel político desse cidadão comum
(\emph{privato ciptadino}) que é preferencialmente o príncipe novo.
Entretanto, isso não deve implicar que há um único tipo de principado ou
de príncipe para Maquiavel. Enfim, interessa-nos saber melhor a noção de
principado e de príncipe que é ressaltada na obra.

Nesta investigação sobre a noção de príncipe em Maquiavel, ao voltarmos
nossas atenções para suas origens, certamente teremos que voltar ao
pensamento político latino do início da medievalidade, particularmente,
a partir das heranças teóricas legadas pelos autores da Patrística
latina e as elaborações que se seguiram no que se refere à noção de
\emph{regimen}. Tal hipótese foi levantada inicialmente por Senellart em
\emph{As artes de governar}, obra na qual procura demonstrar, em sua
primeira parte, como a reflexão pastoral dos autores da Patrística
latina influenciaram um ramo do pensamento político posterior que
desemboca no gênero dos ``espelhos de príncipes'' (\emph{specula
princeps}), cujo \emph{O Príncipe,} de Maquiavel, figura, por vezes,
como um exemplo.

Sobre o gênero literário, já há uma vasta literatura disponível
(\versal{SENELLART}, 2006; \versal{SKINNER}, 2000), que não pretendemos retomar a
exaustão. Contudo convém lembrar alguns de seus aspectos principais.

Os livros do gênero ``espelhos de príncipes'' eram obras escritas por
eruditos que trabalhavam nas cortes e eram destinadas aos futuros
governantes, apresentando orientações para as práticas de governo. A
origem pode ser remontada ao \emph{``Dos Deveres''} (\emph{De Officis},
escrito no século \versal{I} a.C.), de Cícero, da qual se seguiu uma longa e
numerosa variedade de obras que perpassaram os séculos, tendo o mesmo
intuito de orientação e formação para os regentes. Não há um padrão ou
temática única nessas obras, mas em sua maioria, principalmente durante
o período medieval, esses textos preconizavam a valorização dos aspectos
morais e éticos nas práticas de governos principescas. Conforme
Senellart (2006), que faz um amplo resgate histórico desse gênero --
justamente para entender o possível lugar de \emph{O Príncipe}, de
Maquiavel, nesta tradição -- a ênfase nos aspectos éticos e morais dos
``espelhos de príncipes'' têm nas suas fontes os escritos dos autores da
Patrística latina. Desde Agostinho (séc. \versal{V}), passando por Boécio e
Cassiodoro (séc. \versal{VI}), Isidoro de Sevilha e Gregório Magno (séc. \versal{VII}),
entre outros, a reflexão da Patrística latina valorizou a dimensão ética
na condução dos governos em sobreposição sobre os elementos
determinantes próprios do contexto político.

Seguindo adiante na reflexão exposta por Senellart, verifica-se que a
reflexão política anterior a Maquiavel exerceu uma influência
significativa para a elaboração da noção de \emph{príncipe} como um
condutor ou regente, afastando, desse modo, por outra perspctiva, uma
vertente interpretativa que entende o príncipe maquiaveliano como a
prefiguração do soberano moderno.

Ainda, a título de apresentação, nossa interpretação diverge da de
Senellart, embora não lhe seja contrária, ao não optarmos por explorar a
dimensão ética nas práticas de governo, conforme ele enfatiza, para
comprovar as origens teóricas dos ``espelhos de príncipes'' nos textos
pastorais do início da medievalidade latina. Pretendemos seguir, a
partir de suas premissas, uma outra vertente de explicação na qual fique
mais clara as influências desse pensamento político oriundo da
patrística latina que se corporifica na noção de um príncipe como alguém
responsável, prioritariamente, pela condução e direcionamento do povo,
mais do que aquele que exerce seu poder ou domínio sobre os demais. Esse
é o ponto central de nossa interpretação: a partir dos elementos
teóricos herdados da reflexão política medieval latina sobre o
\emph{regimen}, explorar suas possíveis ligações com a noção
maquiaveliana de príncipe como um condutor ou regente do governo, ao
invés da imagem daquele que exerce o poder soberano ou domínio sobre o
seus súditos.

Assim, somos obrigados a retornar a imagem que a Patrística latina
confere àquele que tem por missão conduzir e guiar os cristãos. Conforme
Senellart (2006, 69), foi no momento histórico situado entre a
transferência da dissolução do Império Romano do Ocidente, com a queda
de Roma em 476, e a instauração do Império Carolíngio por Carlos Magno
que a Igreja latina operou ``\emph{uma inversão espantosa. Em vez de
exortarem os reis a governarem com justiça, sabedoria e bondade,
moderando assim o poder, oriundo da violência, pela doçura de seu
exercício, ela faz do `governo' -- ato de regere, dirigir -- a condição
mesma da realeza} (\emph{regnum})'' (Senellart, 2006, 69). O exercício
do governo implica, pois, uma inflexão conceitual: para além (e não
contra) o discurso pastoral, do governante como pastor de almas, esse
comandante da comunidade política cristã deve exercer um maior controle
dos corpos, para que haja, por consequência, uma maior disciplina das
almas com vista à salvação eterna.

Todavia esse dado histórico precisa ser melhor explorado e especificado,
visto que foi, na verdade, um século antes, com a transposição da
capital do Império de Roma para Constantinopla, que se verifica a
inauguração de uma nova reflexão política sobre o modo de governar e
conduzir as comunidades\footnote{Estamos insistindo aqui no termo
  \emph{comunidade,} visto que esse termo foi largamente empregado pelos
  pensadores latinos e terá grande fortuna na posteridade,
  principalmente após a tradução latina da \emph{Política,} de
  Aristóteles, feita por Guilherme de Moerbeke em 1265, que traduz
  \emph{koinonia politiké} por \emph{communicatio politica}. Sobre as
  consequências dessa tradução, cf: Rubstein (1997); Martins (2011;
  201?)}. Conforme Bertelloni (2005), a mudança da capital imperial
originou uma nova teoria política, principalmente a partir dos escritos
de Eusébio de Cesarea, que buscam justificar o poder do imperador romano
sobre os domínios eclesiásticos, doutrina esta também conhecida como
\emph{cesaropapismo}. Em linhas gerais, o \emph{cesaropapismo}
justificava que o imperador cristão, por ter recebido sua missão
diretamente de Deus -- para isso invocando a visão sobrenatural que o
então general Constantino recebe de Deus para pintar a cruz sobre os
escudos de suas tropas, para com isso, conquistar a vitória e tomar o
governo do Império --, teria uma dupla incumbência: guiar os cristãos na
terra e defender a Igreja, sendo, portanto, a maior figura política e
também o maior dirigente eclesiástico. Essa dupla missão, de
\emph{César} e de Papa, unificava-se, agora, na figura do imperador
cristão e de seus herdeiros doravante. Fato é que, depois do século \versal{IV},
os imperadores cristãos de Constantinopla convocavam concílios, nomeavam
bispos e patriarcas, promulgavam normas, leis e documentos dogmáticos,
exercendo um poder direto nos rumos da cristandade.

Porém, esse quadro de ingerência política do imperador sobre os rumos
dos povos cristãos nunca foi bem aceito pelas autoridades eclesiásticas
latinas, principalmente o bispo de Roma, que não era mais do que um
subordinado do Imperador. Ora, do ponto de vista da reflexão teórica,
não foi a queda de Roma em 476, com a deposição de Rômulo Augusto por
Odoacro, que culmina numa reação dos pensadores latinos. Na verdade,
desde 313, quando o imperador Constantino transfere a capital de Roma
para Constantinopla, que a importância política da cidade cede lugar à
nova capital imperial, situada estrategicamente no ponto de confluência
entre aquilo que posteriormente se entendeu por Ocidente e Oriente.
Juntamente com esse dado da política imperial, também o bispo de Roma,
ainda que sempre fosse reconhecido como sucessor de São Pedro pelos
demais bispos, não possuía influência ou poder político o bastante para
determinar os rumos da cristandade. Importa frisar que, do ponto de
vista político, desde o século \versal{IV} Roma já não possuía mais a importância
política de outrora e figurava como uma província de um Império Romano
cujo centro estava em Constantinopla e cuja parte ocidental decaía
gradativamente em termos econômicos, sociais e políticos . O bispo de
Roma, por seu turno, também não possuía, ainda nesses séculos, o
controle político da cidade e nem uma posição de hierárquica superior
entre os demais bispos, sendo um \emph{``primus inter pares''}, o
primeiro entre os pares, o que não implicava em subordinação política
entre o bispo de Roma e os demais patriarcas orientais, por exemplo.

A primeira reação teórica dos pensadores latinos a essa hegemonia do
pensamento político bizantino (o \emph{cesaropapismo}) ocorreu antes
mesmo da queda de Roma em 476, com a reflexão de Agostinho de Hipona,
notadamente, com o seu \emph{Cidade de Deus}. Agostinho, certamente
influenciado pelo ``Saque de Roma'' de 24 de agosto de 410, perpetrado
por Átila, o Huno, postula na sua obra a teoria das duas cidades, uma
celeste, eterna, a qual os cristãos estão destinados, a Jerusalém
Celeste; e outra, terrestre, mundana, corruptível, cuja imagem da Roma
decadente é sua melhor personificação. Consequência direta dessa ideia
de separação das duas cidades é o modo como os cristãos devem se portar
no mundo, fundamentalmente voltados para a Jerusalém Celeste, seu
destino (\emph{telos}) e lugar da sua realização. A Jerusalém terrestre,
decadente, temporal e corruptível não deve ser objeto de atenção e
preocupação dos cristãos, pois ali não está seu destino, sua realização
ou completude, para usar um vocábulo caro ao agostinismo.

Como consequência dessa teoria de separação das esferas celestial e
terrena, a vinculação do homem cristão aos negócios da cidade fica sem
uma justificativa e gera, por seu turno, uma noção de desqualificação do
mundo político que perpassará boa parte dos pensadores medievais latinos
e verá sua influência até o Renascimento italiano. Com efeito, a
caracterização da cidade terrestre como corruptível implicou uma
desqualificação da política e da história, como enfatiza Pocock (1980).
Os pensadores latinos sempre se defrontarão com a dificuldade de pensar
o mundo da política como algo que não fosse efêmero e decadente. Neste
sentido, um primeiro sintoma desse drama e, \emph{pour cause}, uma
primeira tentativa de resposta já se encontra no século \versal{VI} com a
\emph{Consolação da Filosofia} de Boécio. Esse aristocrata romano e
cristão escreve, enquanto aguardava na prisão a execução de sua pena de
morte, um texto no qual relata as angústias do cristão no âmbito da vida
pública na cidade: não poder negar a sua fé e ser ao mesmo tempo um fiel
e legítimo cidadão romano. Como afirma Pocock, a \emph{Consolação da
Filosofia} não é uma obra de filosofia política, mas contém a filosofia
de um homem político (Pocock, 1980, 127).

Seguindo, pois, o nosso fio condutor, entendemos que já no início do
século \versal{V} e doravante -- então antes de 476, como destaca Senellart -- os
pensadores cristãos latinos começaram a elaborar novas teorias sobre a
inserção do cristão no âmbito dos governos e, nesse sentido, o modo como
governar, distanciando-se, assim, do arcabouço teórico bizantino, num
claro esforço de diferenciação e busca de alguma autonomia em termo de
governo nos seus territórios. Contudo, tal reflexão tinha seus limites
claros, seja na separação das esferas celestes e terrestres, seja nos
limites e autonomias de governo das cidades da Europa latina.

Tento em vista essas limitações, de fato, uma teoria de governo não
deveria se sustentar no território da sua ação de comando, mas em algo
que antecedesse isso, visto que não há essa possibilidade de legitimação
do governo. Para isso, e neste caso bem notado por Senellart, a noção de
governo antecede ao \emph{regnum}, ou seja, a condição de comando
político antecede à materialidade do regime político, no caso, o reino
ou o território ou o povo que se deve governar. Essa noção aparece de
modo claro já em Isidoro de Sevilha, que nas suas \emph{Etimologias}
afirma: ``\emph{A palavra reino vem de rei. Com efeito, do mesmo modo
que rei é tirado de reger, reino é tirado de rei. {[}\ldots{}{]} Rei é tirado
de reger. Do mesmo modo que sacerdote vem de santificar, rei vem de
reger. Ora, não se rege se não se corrige}'' (\emph{Etimologias}, livro
\versal{IX}, 3)\footnote{``Regnum a regibus dictum. Nam sicut reges a regendo
  vocati, ita regnum a regibus. {[}\ldots{}{]} Reges a regendo vocati. Sicut
  enim sacerdos a sacrificando, ita et rex a regendo. Non autem regit,
  qui non corrigit''. (\emph{Etimologias}, livro \versal{IX}, 3).}. A sequência é
clara: reger implica em rei que implica em reino (reger -- rei --
reino). O ato de reger (\emph{regere}) é, pois, o que fundamenta a
condição de rei, que, por seu turno, fundamenta o reino. É sobre o verbo
e a ação que recaem, então, os fundamentos e as atenções; é na ação de
reger, no exercício da regência, que se põe o fundamento da condição do
rei. No limite, é antes no \emph{regere} que no \emph{regnum} que o rei
encontra sua fundamentação política.

Então, o exercício do governo funda o regime político e não o reino
(\emph{regnum}) que determina o governo, como não poderia ser de outro
modo para um autor latino dos séculos \versal{VI} e \versal{VII}. Com efeito, tendo em
vista que as responsabilidades sobre os territórios eram delegações do
imperador romano em Constantinopla, a fundamentação da legitimidade do
exercício de governo somente poderia se apoiar sobre o ato ou o
exercício do governo. Seja o rei de um reino do Império Romano, seja
mesmo um bispo com o governo de uma cidade nos confins da Europa, a
fundamentação de suas ações políticas repousava na ação de governo, no
exercício da direção que dava à cidade. Não havia dinastia, sufrágio,
delegação divina que justificasse a contento o exercício de seu ato
governo, senão o próprio ato de governo em si.

Tal ação de governo nada mais era, então, do que reger, corrigir e,
posteriormente, conduzir os homens pelo caminho reto. Assim, o
governante rege e corrige as pessoas, não os obriga ou exerce o seu
domínio, sua violência, seu poder nas ações de seus comandados. Há,
pois, uma certa sutileza no vocabulário que convém destacar. O rei é rei
não porque tem um reino, mas porque rege, porque corrige e orienta o
povo, tal qual o maestro em relação ao coro ou o marinheiro ao navio.
Sua ação de governo é limitada e circunscrita, não cabe a ele impor pela
violência ou qualquer meio extraodinário as ações políticas da cidade,
assim como o marinheiro não impõe pela violência a ação de navegar do
navio, mas o conduz com sabedoria e corrige conforme as forças em ação
os rumos que o navio deve seguir para chegar ao seu destino. Do mesmo
modo o rei rege e conduz o povo, ora corrigindo as ações, ora
impulsionando, para que a cidade perfaça seu caminho de retidão e
justiça. O rei, mas não somente ele, como também o bispo e o comandante
militar, figuram como os pastores que se põe na condução e direção de
seus comandados e não como seres que exercem seu domínio sobre outros
indivíduos.

O destaque deste aspecto pastoral do governante esvazia ainda mais a
imagem anacrônica do rei que domina o seu povo em função de sua
soberania. Esse rei preconizado pelos textos pastorais da Patrística
latina não tem o \emph{dominium} sobre os homens, isso somente lhe será
atribuído quase mil anos depois. O rei é condutor, assim como o
sacerdote, ou melhor, assim como o bispo em relação aos membros de sua
igreja. Na verdade, está se resgatando aqui a antiga antítese entre
\emph{regere} e \emph{dominari}, que remonta a Cícero (\emph{De
Republica,} \versal{I}, 31), mas que sua gênese poderia ainda ser atribuída aos
filósofos gregos, particularmente a Aristóteles. Já na \emph{Política}
(\versal{III}, 1-2), Aristóteles sublinha a diferença do modo de governo da
cidade (\emph{polis}) e da casa (\emph{oikós}). O governo da casa é de
tipo senhor-escravo (metáfora que fará fortuna na história do pensamento
político), no qual o primeiro exerce sua imposição sobre a vontade do
segundo, dominando-o, numa relação hierarquizada e vertical. Na cidade
as relações entre os cidadãos são relações políticas, de igualdade
(\emph{isonomia}), reciprocidade, na qual não há domínio das vontades,
mas se exige a força do convencimento, da argumentação pública, enfim, o
exercício do \emph{logos} na \emph{Àgora}. Como expõe Vernant (2002,
cap. 3), os governos da casa e da cidade são, para os helenos de
natureza diferentes, sendo o primeiro uma relação de caráter doméstico
(\emph{oikonomico}) e a segunda de caráter político (\emph{politica}). A
transposição dessas categorias para o mundo romano resultará em uma
distinção de governos que regem (\emph{regere}) e os que dominam
(\emph{dominium}). O \emph{dominium} aqui derivado de \emph{domus}
(casa), ou seja, as relações de domínio são a transposição para a esfera
pública das relações próprias da casa, o que distingue o governo do rei
e do tirano. O tirano exerce seu domínio sobre o povo, já o rei rege.
Donde a antítese entre o reger e o dominar no exercício do governo.

Nos textos dos autores da patrística latina, a noção de domínio também
possuía uma relação com a irracionalidade dos homens, pois, conforme
Senellart (2006, p. 101), o domínio se exercia sobre os corpos dos
homens controlados pelo pecado, que estavam como que obscurecido pelos
vícios dele decorrente. Para esse ser humano decaído, que se tornou
servo de seu corpo, apenas um comando forte, tal qual se aplica aos
animais brutos, será adequado. ``Não é o homem enquanto homem, mas o
homem rebaixado pelos impulsos de sua carne à condição do animal, que é
o objeto do governo régio e determina as modalidades de seu exercício''
(\versal{SENELLART}, 2006, p. 101). Donde se evidencia o binômio
dominação/irracionalidade, o governo que domina, se impõe sob a condição
de irracionalidade dos comandados. Neste caso, com mais força do que no
pensamento antigo, no qual, ainda que o \emph{logos} se coloque como a
marca da cidade, sua presença no âmbito do político era condição
\emph{sine qua non} da dinâmica política. Aqui é a antítese que se
apresenta: o domínio do governo se impõe pela ausência de razão, porque
os homens ficaram obscurecidos pelos pecados e pelos vícios.

Associada a essa tarefa de reger e corrigir, o governante tem também uma
outra incumbência na sua tarefa, a de vigiar e assistir os homens.
Herdada da tradição grega, mas sob forte influência cristã, o governante
também é um \emph{episcopus}, da qual resultou o termo português bispo.
O \emph{episcopus vem do grego episkopos}, que se origina, por seu
turno, do grego \emph{skopós}, que é observar, ver, vigiar. Logo, também
é uma função do rei o observar, guardar, vigiar para que possa corrigir
e reger. Tal imagem do governante como aquele que vigia, que olha,
remete a um tema clássico do pensamento político que é o da vigilância
como algo típico da ação política. Essa mesma temática aparecerá nos
textos de Maquiavel, particularmente nos \emph{Discursos}, no livro \versal{I},
capítulos 5 e 6, onde ele destaca o papel de vigilância que a população
tem na defesa da liberdade. Ao invocar a defesa da liberdade (\emph{la
guarda della libertá}), ele não somente está pedindo ao povo para
defender, mas também vigiar, tal qual a sentinela, que vigia e defende a
fortaleza.

Passo seguinte dessa concatenação teórica se fará, a partir do século
\versal{XII}, com a associação à função real do dirigir, liderar. Sendo o rei
aquele que rege, observa e corrige o seu povo, cabe então a ele também
dirigir o povo e o governo. O rei deve também guiar e indicar o caminho,
os rumos a seguir pelo seu reino. Tal capacidade de direção decorre da
racionalidade que deve ser própria da personalidade real, ou seja, em
função de sua razão, que se manifesta na prudência e sabedoria, o rei
tem a qualidade para dirigir e liderar o reino.

Essa imagem do rei que também dirige, segundo Senellart (2006), já
aparece em John de Salisbury no seu \emph{Policraticus}, escrito no
século \versal{XII}, mas se encontra também em Tomás de Aquino (no \emph{De
Regno}) e em Egídio Romano (\emph{De Regimine Principum}), ambos do
século \versal{XIII}. No limite o desenvolvimento do argumento é muito claro: por
reger, corrigir e observar, sendo dotado de sabedoria e prudência, está
o príncipe habilitado a dirigir o povo. O rei, portanto, é agora um
rei-guia, como diz Senellart: ``\emph{Regere} não mais simplesmente
corrigir, mas dirigir. Cabeça do corpo político, o príncipe coordenará a
ação de seus membros em vista de um fim coletivo. \emph{Rex sagittator},
dirá Egídio Romano: o rei, como um arqueiro, olhos fixos no alvo, mira
longa à sua frente''\emph{.} (2006, p. 133)

Todas essas qualidades na ação de governar demonstram a liderança e
proeminência do príncipe em face da cidade, o que consolida cada vez
mais sua condição política destacada, como um primeiro, uma figura
destacada e, por isso, que fica na vanguarda, um precursor. Entretanto,
convém frisar, tudo isso ocorre sem que ele faça valer a força ou a
violência como atos próprios e constantes do seu governo para conservar
a sua condição de príncipe, característica essa que será própria da
Modernidade. Apesar da cólera, natural nos homens, o governante dos
``\emph{espelhos de príncipe}'' deve dissimular seu sentimento de
violência, raiva, rancor, oriundo da bile negra, para que ela não turbe
o seu julgamento, conforme sugere Egídio Romano no \emph{De regimine
principum}. Ora, nem encolerizado o príncipe deve fazer uso da força ou
violência no exercício do governo, para que não seja lhe imputado a
perda da razão, da prudência característica dos grandes líderes. Sem
contar que estamos ainda distantes de uma noção de força ou violência
como veremos, por exemplo, na noção que Thomas Hobbes (séc. \versal{XVII})
conferirá ao Estado\footnote{Não pretendemos fazer aqui uma análise da
  presença ou não da noção de Estado em Maquiavel, algo que possui uma
  vasta literatura e que demandaria um esforço analítico que ultrapassa
  as ambições desse ensaio. Todavia, se tivermos em mente aquilo que
  Ercole define como Estado, como sendo: ``a entidade coletiva soberana
  resultante do ordenamento jurídico de um povo num território sob um
  poder comum e que permanece idêntico a si mesmo através da sucessão e
  a mudança dos indivíduos, dos órgãos e das formas constitucionais''
  (\versal{ERCOLE}, 1926, p.65), notar-se-á que tal concepção não está presente
  em sua plenitude em Maquiavel. Como tentaremos demonstrar, as
  concepções políticas que subjazem a noção de principado e príncipe
  estão calcadas em noções anteriores, e, por isso, diferentes dessa
  noção de Estado própria da modernidade, embora seja possível já
  visualizar alguns elementos conceituais que serão desenvolvidos
  posteriormente. Como nos mostra Rubinstein, antes dos humanistas, já
  se encontrava em vários escritos políticos da medievalidade latina o
  termo \emph{status} em uma acepção política, como regime ou forma de
  governo, como, por exemplo, no \emph{Sententia libri Politicorum,} de
  Tomás de Aquino, e no \emph{De Regime Civitatis,} de Bartollo de
  Sassoferrato. Mas em nenhum deles com o sentido que a modernidade
  conferirá ao termo (Rubinstein, 2004, p. 151-163). Para uma
  apresentação geral do debate, cf: Ames, 2011.}. É essa ausência do
argumento do exercício da força, entendida mais como \emph{dominium},
como essecial e inerente ao governo do príncipe que desloca toda a
atenção para o argumento racional, ou como destacará Skinner (2000),
para a capacidade retórica e de persuasão do príncipe.

Enfim, verifica-se por esses elementos da tradição dos ``espelhos de
príncipe'', que muitas dessas concepções repercutem no texto
maquiaveliano. Como se sabe, a noção de príncipe, que perpassa a obra de
Maquiavel como um todo, apresenta inúmeras semelhanças com esse ideal
herdado do pensamento medieval latino, a saber:

\begin{itemize}
\item
  \begin{quote}
  em quase todos os casos apresentados, a exceção do príncipe herdeiro,
  o príncipe de Maquiavel tem a sua condição de governante fundada antes
  no seu exercício político do que no território ou reino;
  \end{quote}
\item
  \begin{quote}
  Esse príncipe maquiaveliano deve, em função de sua condição e daquilo
  que o legitima no governo, reger, conduzir e liderar a cidade -- algo
  que ganha contornos claros e dramáticos no capítulo final do livro;
  \end{quote}
\item
  \begin{quote}
  No caso do uso da violência, apesar de ser apresentado como
  estratagema para a conquista da condição de príncipe, Maquiavel deixa
  claro que ela não é \emph{virtù} e sua mobilização no texto não tem a
  mesma justificação que terá posteriormente o argumento do uso legítimo
  da violência pelo Estado. Esse, na verdade, é um dos pontos pantanosos
  de \emph{O Príncipe}, pois ainda que não tenhamos a mesma
  caracterização da violência estatal da modernidade, ela se faz
  presente como elemento da ação política do príncipe.
  \end{quote}
\end{itemize}

Mas convém entender melhor esses aspectos no texto maquiaveliano.

Como já foi dito, podemos dividir o livro em duas grandes partes, uma
primeira dedicada ao principado e uma segunda parte dedicada ao
príncipe. Na primeira parte, Maquiavel se detém a tipificar os modos de
ascensão à condição de príncipe, a saber: por herança, por armas
próprias e alheias, por \emph{virtù} e por fortuna. Como diz o próprio
autor no cap. 2, no caso do príncipe herdeiro, as dificuldades na
condução do governo são pequenas, pois:

\begin{quote}
Digo, portanto, que nos estados hereditários e acostumados à
dinastia do seu príncipe são muito menores as dificuldades para
conservá-los do que nos novos, porque basta não preterir os ordenamentos
de seus antecessores e posteriormente contemporizar com os acidentes, de
modo que, se tal príncipe tiver uma indústria ordinária, sempre
conservará o seu estado, a não ser que uma força extraordinária e
excessiva o prive dele. E tendo sido dela privado, reconquista tal
condição na medida em que o conquistador enfrentar alguma
adversidade (\emph{O Príncipe}, cap. 2, linha 3).
\end{quote}

Neste caso, ainda, a legitimidade do governo se faz pelo pertencimento à
família que governa, logo, não por si ou por seu governo que esse
príncipe herdeiro assume a condição de príncipe. Esse breve capítulo
inicial da obra revela, também, uma tópica central da reflexão que será
desenvolvida, a saber: a análise do príncipe novo. Como se verifica,
esse é o personagem principal de Maquiavel, o príncipe que ascende ao
governo e que não o recebe de modo hereditário. Todavia, alguém poderia
argumentar que mesmo o príncipe herdeiro, quando ascende ao governo,
naquele momento ele é novo. Porém, não é dessa condição temporal que
Maquiavel está fazendo menção, mas da condição política nova para ele,
algo que um príncipe herdeiro não tem. O príncipe novo é novo do ponto
de vista político, ou seja, ele está inaugurando, iniciando o seu
exercício político, algo que não ocorre com o príncipe herdeiro. Mais
ainda, o príncipe é novo porque funda o governo, inaugura o novo regime,
ao contrário do herdeiro, pois esse, como explícito pelo próprio texto
maquiaveliano, herda, recebe a sua condição política, logo não funda
nada. Em consonância com o pensamento político maquiaveliano, poderíamos
até dizer que um príncipe herdeiro coloca-se como um príncipe novo, se
ele se apresenta na cena política e faz de seu governo uma novidade em
relação ao governo anterior. Mas, neste caso, como declara Maquiavel,
exige-se dele qualidades extraordinárias, ou seja, ele deverá mudar os
procedimentos políticos ordinários, fundar novos ``modos e
ordenamentos'', e, neste caso, seguir o que vem exposto para o príncipe
novo.

Ora, pela próprio entendimento do que seja aqui o \emph{novo},
verifica-se que Maquiavel está tratando de um tipo particular de
personagem político que assume o comando político da cidade. Nos demais
casos relatados de conquistas de principados, até mesmo para o
principado eclesiástico, aquele que ascende à condição de príncipe tem
que justificar por si mesmo e pelas suas condições sua legitimidade no
governo. Seja por ter a fortuna de conquistar o governo, seja por
possuir um exército próprio ou contar com a ajuda de uma força militar
alheia, seja por possuir \emph{virtù}, o aspirante à condição de
príncipe tem que se valer de seus esforços, de suas qualidades para
obter o governo. Pensando na formula isidoriana, que o rei é rei porque
rege, não tendo nada anteriormente que justifique sua condição de rei,
do mesmo modo esse príncipe novo maquiaveliano, é príncipe \emph{a
posteriori}, ou seja, pode-se afirmar que ele é príncipe apenas quando
está no exercício de seu governo, porque conquistou essa condição e a
exerce. Mais ainda, se diz dele príncipe porque tem o principado,
conquistou o regime principado, ou, no limite, se diz príncipe em função
do principado.

Como foi visto, o principado não é o território ou um reino, como em
geral entendemos. Principado é antes uma forma de governo ou um regime
sob o comando de um príncipe. Então, se o principado não é um reino ou
território e o príncipe se diz príncipe em função do principado, sua
condição se faz pelo próprio ato de governar, pela sua ação de governo,
a semelhança da sentença isidoriana. Donde as ações num primeiro momento
objetivarem a conquista do principado, algo que se faz pelos quatro
modos citados, e que depois se mantém em função da sua qualidade ou
\emph{virtù} como governante, tema da segunda parte do livro, que trata
da conservação do governo.

É evidente, pois, que esse príncipe alcança essa condição em função de
suas qualidades ou \emph{virtù}, ainda que em alguns casos ela não se
faça necessária, mas que certamente ela deve comparecer na conservação
do governo. Ou seja, mais cedo ou mais tarde, o príncipe deve demonstrar
possuir \emph{virtù} para conseguir conservar sua condição de príncipe.

Por outro lado, tal formulação de príncipe fundada no exercício do
governo tão somente, excluindo, desse modo, a necessidade anterior do
reino ou território para a fundamentação de seu governo, reforça a ideia
de que não temos ainda a noção de Estado. Com efeito, na medida em que
esse príncipe funda a sua condição no seu governo, não há algo anterior
a esse governo que estivesse posto como elemento de legitimação de sua
ação política. Em outras palavras, o príncipe, ao conquistar o
principado, estabelece nesse momento o seu governo, inicia de fato o seu
regime e terá sempre na sua ação política -- ou em termos
maquiavelianos, na sua \emph{virtù} -- o principal alicerce da sua
condição de príncipe.

Tal hipótese ajudaria a explicar, ainda, porque Maquiavel confere tanto
destaque à \emph{virtù} do príncipe, a ponto de dedicar metade de sua
obra a essa temática. Note-se que não se trata de uma exposição entre os
capítulos 15 a 24 do fundamento ético e moral da \emph{virtù}, tal qual
num tratado de filosofia moral como muitos que se fizeram durante o
Medioevo, mas de uma exposição da \emph{virtù,} depois de comprovada sua
necessidade para a condição de príncipe, para que esse príncipe conserve
o seu governo. Ou seja, na segunda parte do livro, Maquiavel dedica-se a
pensar a \emph{virtù} em função da sua condição de fundamento e
legitimação do governo do príncipe, porque ela se faz necessária.

Agora talvez fique mais claro o sentido da sentença que o príncipe tem
essa condição em função do exercício de seu governo, pois é a
\emph{virtù} que lhe dá esse fundamento em última instância. Para
comprovar isso, dois exemplos mobilizados na primeira parte da obra
podem nos vir em auxílio: o caso de César Borgia e o \emph{privato
ciptadino}.

César Borgia, filho do papa Alexandre \versal{VI}, é um personagem muito presente
n'\emph{O Príncipe}, podendo até ser confundido com o ideal de príncipe
que Maquiavel deseja apresentar. Contudo, nas mobilizações dos feitos de
Cesar Borgia ao longo do livro, sente-se que há uma certa ambiguidade:
por um lado, ele é apresentado como tendo muita \emph{virtù}, o que lhe
possibilitou fazer inúmeras conquistas, mas, de outro, nota-se que
faltou-lhe \emph{virtù} para conservar tudo aquilo que conquistou. O
capítulo \versal{VII} é certamente o melhor lugar para se perceber essa
ambiguidade que ronda a figura de César Borgia.

Na sequência expositiva sobre a conquista dos principados, depois de ter
analisado como eles são conquistados com armas próprias e com
\emph{virtù}, no capítulo \versal{VII}, Maquiavel passa a dissertar sobre o caso
contrário, quando alguém conquista um principado com armas e fortuna
alheia. O problema inicial desse tipo de conquista é logo de início
apontado:

\begin{quote}
{[}3{]} Estes estão fundados unicamente na vontade e na fortuna
de quem lhes concedeu tal status, duas muito volúveis e instáveis, e não
sabem e não podem se manter naquele posto: não sabem, porque não
sendo um homem de grande engenho e \emph{virtù},
não é razoável que, sempre vivendo como homens de condição
particular, saibam comandar; não podem, porque não
têm forças que lhes possam ser amigas e fiéis (\emph{O Príncipe}, cap.
\versal{VII}, linha 3).
\end{quote}

A dificuldade principal é evidente: aquele que conquista um governo
apoiando- se nas qualidades e na força de outros terá muita dificuldade
de se manter nessa condição de príncipe. O que, para o problema em
questão, é muito ilustrativo, pois, segundo Maquiavel, mesmo que alguém
receba um governo de outro, se ele não tem as qualidades necessárias
para conservar esse governo, perderá tal condição. Com efeito, os
fundamentos desse novo governo estão na vontade e fortuna de quem
outorga, que, como declarado, não perfazem alicerces seguros. A vontade,
embora Maquiavel não disserte sobre ela, sabemos todos que é volúvel,
que pode mudar, donde apoiar um regime na vontade de outrem ser de fato
algo transitório. A fortuna, como ele explicará no capítulo 25 de
\emph{O Príncipe}, é inconstante e difícil, senão impossível de ser
dominada. Logo, começar um governo sustentado por dois elementos
exteriores ao seu controle é permitir a instabilidade. Fato esse que
revela, inicialmente, que nada anteriormente sustenta a condição do
príncipe, a não ser o exercício do governo, no caso, um exercício com
\emph{virtù}.

Entretanto, para corroborar sua posição, logo adiante Maquiavel cita
explicitamente o caso de César Borgia, quando diz:

\begin{quote}
Por outro lado, César Bórgia, chamado pelo povo de duque Valentino,
conquistou o governo com a fortuna do pai e com a
mesma o perdeu, apesar de ter ele usado de todos os recursos e ter feito
todas aquelas coisas que um homem prudente e virtuoso deveria fazer para
deitar suas raízes naqueles governos, que as armas e a fortuna de outros
lhe haviam concedido (\emph{O Príncipe}, cap. \versal{VII}, linha 7).
\end{quote}

O exemplo não poderia ser melhor e mais dramático, visto que César,
mesmo tendo ao seu lado o apoio da condição política do pai, que era
papa, não foi o suficiente para que ele pudesse ter garantias de
exercício tranquilo ou seguro de seu governo sobre os territórios recém
conquistados. Teve ele que realizar inúmeras ações para conquistar a
legitimidade política nesses novos governos, mas ainda assim lhe faltou
algo. Maquiavel chama em causa um exemplo em grau máximo, pois nem a
legitimidade política de alguém como um papa é garantia o suficiente
para o exercício do governo de seu filho. No limite, ele dá a entender
que tal legitimidade política não pode ser transferida ou irradiada para
outro, mas apenas o próprio indivíduo é que tem as condições de
estabelecer os alicerces de seu governo. Na sequência isso fica mais
evidente, quando diz:

\begin{quote}
{[}8{]} Porque, como se disse anteriormente, aquele que não
constrói primeiro os fundamentos, poderia, com uma grande \emph{virtù},
construí-los depois, ainda que se façam com incômodo para o arquiteto e
perigo para o edifício. {[}9{]} Se, então, considerarmos todos os
progressos do duque, veremos que ele construiu grandes fundamentos para
um poder futuro, sobre os quais não julgo supérfluo discorrer, porque
não saberia quais preceitos melhores dar a um príncipe novo, senão o
exemplo de suas ações; e se seus modos de proceder não lhe forem
proveitosos, não será por culpa sua, porque nasce de uma extraordinária
e extrema malignidade da fortuna (\emph{O Príncipe}, cap. \versal{VII}, linha 8 e
9).
\end{quote}

Conforme dito, o exemplo de César torna-se mais emblemático, pois,
sabendo ele que deveria buscar outro fundamento para o seu governo, que
não apenas no prestígio de seu pai, tratou de executar uma série de
medidas políticas para arregimentar os apoios necessários para um
governo seguro. Em todas as suas iniciativas obteve êxito, apenas lhe
faltou a fortuna, que no caso não é somente sorte, mas a condição
favorável para o exercício do governo, que conforme Maquiavel explicará
no penúltimo capítulo, também é passível de obtenção ou ao menos de
contenção. Neste caso de César, faltou conter os desfavores da fortuna
para conservar suas conquistas.

A figura política de César Borgia é ilustrativa em um duplo sentido:
primeiro, mesmo ele sendo filho de um político poderoso e recebendo dele
os governos de algumas cidades, isso não foi garantia o bastante para o
sucesso de seu governo. Num segundo aspecto, suas próprias ações de
governo lhe trariam a segurança política necessária desde que ele
executasse todas as medidas, inclusive conter os ``ventos desfavoráveis
da fortuna''. Em ambos os sentidos, o que fez dele príncipe não foi
outra coisa senão o exercício de seu governo, o seu principado.

Antes de passar para o segundo exemplo, convém notar o uso de um
vocabulário próprio da engenharia de construção sendo mobilizado aqui.
Com efeito, Maquiavel se vale dessa metáfora das edificações para
retratar essa ação em busca dos sustentáculos do governo: fundação,
edifício, arquiteto, obra, etc. A conquista de governo e sua conservação
é apresentada então, como obra em execução, donde a necessidade de
planejamento, fundamentos, alicerces, etc., obra essa que não se faz ou
se sustenta a partir de uma obra anterior, mas em si mesmo. Mais ainda,
o governo é uma fundação política, ou seja, uma obra nova, que necessita
de alicerces e na qual um novo campo do político está sendo erguido.

O outro exemplo citado é o caso de um cidadão comum (\emph{privato
ciptadino}) que se torna príncipe. Neste caso, a dificuldade para esse
governante é maior, pois a conquista do governo requer \emph{virtù},
embora a fortuna e as armas alheias possam permitir a conquista, mas a
conservação do principado exige toda a demonstração de \emph{virtù} por
parte deste príncipe novo.

A princípio o próprio termo \emph{privato} utilizado aqui por Maquiavel
é de difícil tradução para o nosso contexto discursivo, pois
literalmente \emph{privato} deve ser traduzido por \emph{privado} em
português. No capítulo \versal{IX}, ele usará a expressão \emph{privato
ciptadino} (1) e \emph{ciptadino privato} (20), que poderia ser
traduzida literalmente por \emph{cidadão particular}. Entretanto, o
autor está se referindo aqui, bem como nas demais ocorrências que se
seguem {[}capítulos \versal{VI} (27), \versal{VII} (1, 2, 6), \versal{VIII} (1, 4), \versal{IX} (1, 20), \versal{XI}
(15), \versal{XIV} (3){]}, ao cidadão comum, não pertencente à família do
governante, que se torna príncipe de uma cidade. Esse é o caso mais
emblemático até do que o de César Borgia, visto que ele deve fazer muito
mais esforços políticos para conseguir chegar ao governo e conservar- se
nele.

No primeiro caso em análise, Maquiavel já deixa claro o seu modo de
entender a questão:

\begin{quote}
E porque este evento, de passar de cidadão comum a príncipe,
pressupõe ou \emph{virtù} ou fortuna, parece que uma ou outra destas
duas coisas mitiga, em parte, muitas dificuldades. Todavia, aquele que
menos se apoiou na fortuna, manteve-se mais (\emph{O Príncipe}, cap. \versal{VI}, 5).
\end{quote}

Ora, tornar-se príncipe é fundamentalmente um ato de \emph{virtù}, ainda
que a fortuna mitigue muitas dificuldades, ou seja, ela auxilia a
conquista do governo. Entretanto, essa ação, esse evento deve ser
calcado na \emph{virtù,} nas qualidades ou excelências políticas desse
cidadão que comanda a cidade.

Essa contraposição entre a \emph{virtù} e a fortuna para o cidadão comum
que deseja tornar-se príncipe é ressaltada nos capítulos seguintes,
respectiviamente nos capítulos \versal{VII} e \versal{VIII}, sempre destacando a
instabilidade e a fraqueza que a fortuna gera no processo de conquista e
no futuro governo. Maquiavel ressalta, pois, que ainda que a fortuna se
apresente e justifique muitas conquistas políticas, aquele príncipe que
nela se apoiou tão somente perdeu o seu governo. O argumento é claro e
insistentemente enunciado: para o príncipe novo, é necessário a
\emph{virtù} para a conservação do poder. Em suas palavras:

\begin{quote}
Aqueles que somente pela fortuna de cidadão se tornam
príncipes, com pouco esforço conseguem sê-lo e com muito se mantêm. E
não têm nenhuma dificuldade neste caminho, porque voam para esta
condição, mas todas as dificuldades surgem quando a ela chegam
(\emph{O Príncipe}, cap. \versal{VII}, 1).
\end{quote}

Note-se que ele chega a usar o termo \emph{voam}, evidenciando que este
postulante ao governo pula as dificuldades inerentes à conquista sem dar
conta delas. Essa vantagem dada pela fortuna, inicialmente, não
significa uma maior facilidade no exercício do governo, ao contrário,
por justamente não ter passado pelo exercício e prática da \emph{virtù}
política no momento da conquista, falta a este cidadão comuum o domínio
das qualidades políticas, dessa \emph{virtù} que se deve apresentar já
no início do processo de tomada do governo. As dificuldades do governo
são, evidentemente, de natureza política e, portanto, são as mesmas que
ele evitou nesse processo de conquista.

A \emph{virtù} se manifesta, pois, sob uma série de medidas e
procedimentos que já vinham sendo elencados desde o capítulo \versal{III}, quando
de fato temos a exposição daquilo que o príncipe novo precisa para
conquistar e manter o governo. Contudo, no início do capítulo \versal{VIII}, o
problema para o cidadão comum que deseja ser príncipe ganha intensidade
e chega aos seus contornos mais fortes no capítulo seguinte. A
dificuldade em tela diz respeito ao cidadão que não tem nem toda a
fortuna necessária e nem toda a \emph{virtù}, mas uma mescla e
insuficiência das duas, como diz:

\begin{quote}
{[}1{]} Mas porque há ainda dois modos de se passar de cidadão a
príncipe, o que não se pode atribuir de todo ou à fortuna ou à virtù,
não me parece que deva deixá-las de lado, ainda que sobre uma delas se
possa discorrer mais amplamente em se tratando de repúblicas. {[}2{]}
Estes modos são: ou quando por algum meio criminoso e nefasto alguém
ascende ao principado, ou quando um cidadão comum, com o favor de outros
cidadãos, torna-se príncipe da sua pátria (\emph{O Príncipe}, \versal{VIII}, linha 1 e
2).
\end{quote}

Antes de prosseguir na análise, importa chamar a atenção para a ressalva
acerca das repúblicas, na verdade, a terceira do texto até aqui (as
outras duas foram: no início do capítulo \versal{II}, linha 1 e no capítulo \versal{V},
linha 9). A referência à república diz respeito ao exercício da
\emph{virtù}, para a qual apenas a fortuna não basta para a conquista do
governo. Com efeito, nas repúblicas, a presença da \emph{virtù} para as
ações de governos é imperativa, pois, conforme fica evidente tanto nos
\emph{Discursos} e na \emph{História de Florença}, a vida política na
república é marcada pelos conflitos e pela dinâmica dos humores, que
podem ser apenas geridos com a presença da \emph{virtù}. Aqui a
referência é a necessidae, um tanto quanto parcial, diga-se de passagem,
da \emph{virtù} ao governo do principado.

A exposição do argumento maquiaveliano nesta primeira parte \emph{d'O
Príncipe} vem seguindo uma sequência expositiva de desvalorização da
fortuna em razão da valorização da \emph{virtù} para esse cidadão que
deseja o comando do principado. Neste início do capítulo \versal{VIII},
Maquiavel, uma vez tendo deixado patente que a fortuna não é
sustentáculo para a conquista e constituição do governo principesco,
aponta para os aspectos essenciais dessa \emph{virtù} requerida. Num
primeiro momento, poderia até ser \emph{virtù} o uso de uma violência
desmedida para a conquista do governo, mas isso não seria o bastante e
nem adequado, conforme exposto ao longo do capítulo \versal{VIII}. Após o uso da
violência para a conquista do comando político, esse cidadão deveria
usar os estratagemas próprios da conquista política em qualquer
circunstância: saber reconhecer os humores e atuar no interior da
dinâmica própria dessa oposição política intrínseca à cidade, tema do
capítulo \versal{IX}, como declara de início:

\begin{quote}
Voltando à outra parte, quando um cidadão comum, não por meio de
crimes ou outra violência intolerável, mas com o favor dos outros
cidadãos, torna-se príncipe da sua pátria -- que poderia ser chamada de
principado civil: e para sê-lo não é necessário toda virtù ou toda
fortuna, mas, antes, uma astúcia afortunada --, digo que se ascende a
este principado ou com o favor do povo ou com o favor dos
grandes (\emph{O Príncipe}, \versal{IX}, 1).
\end{quote}

Aqui se mostra a \emph{virtù} principal desse cidadão comum que busca
ascender à condição de príncipe: ter uma astúcia afortunada para que
consiga granjear o apoio dos grandes e do povo, os dois humores
políticos da cidade. Esta é a dificuldade que não se pode saltar, o
problema político a ser enfrentado e crucial no pensamento político
maquiaveliano: conquistar o apoio das partes e saber lidar com elas,
desde o momento da conquista e durante todo o governo. Verifica-se que
tal tarefa é sempre um trabalho em construção, uma obra inacabada, pois
nunca se tem a garantia plena de apoio incondicional a ponto de não se
precisar constantemente renovar e conservar os apoios que se tem, ao
mesmo tempo em que se busca obter novos partidários. O cidadão que
deseja ser príncipe tem a partir desse momento inicial uma tarefa sempre
a executar: inserir-se na dinâmica da vida política e movimentar-se
nela. Ao fazer isso, ele demonstra a sua \emph{virtù} política.
\emph{Virtù} essa também sempre em exercício, seja no sentido de um
aprimoramento, se for possível atribuir isso a ela, seja na sua perda,
quando não a usa bem.

Enfim, esse \emph{privato ciptadino}, por tudo isso que foi apresentado,
não possui nada que o legitime e dê fundamento ao exercício de seu
governo \emph{a priori}, mas tão somente a sua \emph{virtù}, que nada
mais é do que o exercício das qualidades políticas, antes e durante o
exercício do governo. Nesse sentido, o que torna esse cidadão comum em
príncipe é o exercício das qualidades políticas que permitem a ascensão
à condição de príncipe, num primeiro momento, e de conservação do
governo, doravante.

Voltando ao nosso problema que nos levou a analisar o caso de César
Borgia e do cidadão comum que ascende à príncipe, verifica-se que nesses
dois casos emblemáticos do livro, não há qualquer poder político
anterior que possa conferir legitimidade ao príncipe novo -- como é o
caso de César -- e que, no limite, todo governante deve fundar o seu
governo no exercício das suas qualidades políticas, na sua \emph{virtù}
política que lhe assegura a legitimidade.

Portanto, comprova-se como em \emph{O Príncipe,} de Maquiavel, ainda se
conserva esse aspecto herdado da tradição dos ``espelhos de príncipes''
no que diz respeito à fundação do poder político sobre o exercício
próprio do governo, donde um cidadão ser dito príncipe em função do
governo ou do principado que exerce, condição política essa alicerçada
na sua \emph{virtù}.

Outro aspecto apontado como uma herança dos ``espelhos de príncipe'' na
reflexão política maquiaveliana diz respeito ao fato do príncipe
conduzir ou liderar a cidade, pois, conforme visto, esse príncipe
maquiaveliano deve, em função de sua condição e daquilo que o legitima
no governo, reger, conduzir e liderar a cidade, o que fica evidente no
capítulo final de \emph{O Príncipe}.

O convite à liderança política da cidade é dirigido a algum membro da
família Medici, definido como um príncipe novo, embora essa família já
tivesse o comando da cidade de Florença em vários momentos e, quando o
livro foi escrito, não somente a cidade, mas inclusive o papado, estavam
nas mãos dos Medici, o que não permitiria afirmar que qualquer membro
dessa família fosse um príncipe novo. Contudo, ao contrário dessa
primeira impressão, conforme visto, os Medici podem se encaixar
perfeitamente nesse tipo político que é o príncipe novo. Evidência disto
vem pela invocação da imagem militar de seguir a bandeira ou o
estandarte do comandante, como diz:

\begin{quote}
Vê"-se ainda toda pronta e disposta a seguir uma bandeira, desde
que haja alguém que a empunhe. {[}\ldots{}{]}Tome, portanto, a sua ilustre
Casa este assunto com aquele ânimo e aquela esperança com que se tomam
as façanhas justas, a fim de que, sob o seu estandarte, esta pátria seja
enobrecida (\emph{O Príncipe}, cap. \versal{XXVI}, linhas 7 e 29).
\end{quote}

Nas técnicas de combate antigas, a organização das tropas nos teatros de
operações se fazia por meio de bandeiras, haja vista a dificuldade de
comunicação do comandante com os seus soldados. Desse modo, os soldados
eram instruídos a se orientar pelas bandeiras, do seu comandante mais
imediato inicialmente, até, no limite, do comandante supremo presente no
campo de batalha. A bandeira era símbolo de orientação da tropa, donde
os soldados se agruparem onde estava posicionado a sua bandeira, a sua
tropa, e seguirem o rumo que esta tomar. A bandeira ficava sempre ou com
o comandante ou ao lado dele, tanto que tomar a bandeira de um exército
era o sinal de que o comando caiu. Notório que, até nos dias de hoje, em
que essa função de orientação pelas bandeiras não se faz mais necessária
nas técnicas de combate (seja pelos modernos meios de comunicação das
forças militares, seja pelo modo como as tropas se dispõem para as
batalhas), a força simbólica da bandeira como fonte de unidade é ainda
essencial, não somente nos meios militares, como na sociedade como um
todo.

Ora, a invocação para que alguém da família Medici ``empunhe uma
bandeira'' remonta a essa imagem clássica de liderança militar.
Maquiavel roga para que os Medici liderem e conduzam os italianos,
retirando-os da submissão aos povos bárbaros.

Mas essa invocação final dirigida aos Medici é apenas a exemplificação
de uma ideia que se mostra ao longo de todo o texto. A liderança
política do príncipe já era apontada seja nos exemplos históricos, como
o de Moisés, que lidera, conduz e dirige os hebreus a Israel, seja no
modo como esse príncipe se instala na cena política. Como será explicado
adiante, esse príncipe novo de Maquiavel é o governante que lidera, ele
é antes de tudo um \emph{princeps}, que conduz e dá a direção dos rumos
políticos que a cidade deve seguir, mas sem que isso implique em domínio
ou imposição pela força, visto que deve angariar apoios, o que não seria
o caso do governante com poderes absolutos. Convém insistir, o príncipe
será reconhecido também como príncipe na medida em que dirige, aponta os
rumos, não em função de um poder dominador sobre o seu povo, mas em
função da sua condição de liderança e destaque.

Portanto, assim como rei deve conduzir e liderar, tal qual Moisés em
relação aos hebreus, também o príncipe maquiaveliano deve conduzir e
liderar politicamente e militarmente a cidade. Todavia, tal proeminência
se coloca como problema quanto à utilização ou não da força nesse
exercício político. Na verdade esse é um dos temas mais embaraçosos no
pensamento político maquiaveliano, visto que as nuances dessa força que
o príncipe exerce parece sugerir dominação, e, portanto, uma noção de
poder como comando e obediência, contudo, em outros casos, não se trata
dessa noção de força como domínio, mas como potência individual do
príncipe que consegue dirigir e impulsionar os seus comandados (\versal{SASSO},
1988; \versal{REALE}, 1974; \versal{CADONI}, 1994; \versal{LARIVAILLE}, 1997; \versal{FRONSINI}, 2005).

Uma análise, ainda que superficial, das ocorrências de três vocábulos
relacionados ao tema do uso da potência política do príncipe nos fornece
boas pistas de como Maquiavel concebe n'\emph{O Príncipe} esse uso da
violência, da crueldade ou da força e quais são os seus efeitos sobre os
comandados.

Comecemos pelo termo ``violência'' (\emph{violenzia}), que ocorre por
três vezes na obra: cap. \versal{VIII}, 6; cap. \versal{IX}, 1; cap. \versal{XXV}, 12. No capítulo
\versal{IX}, ele fala de uma conquista do principado que não se faça por
``violência intolerável'', como foi o exemplo retratado no capítulo
\versal{VIII}. Neste caso, a violência é reprovada e seu qualificativo já nos diz
tudo. No capítulo \versal{XXV}, que trata da fortuna e do como de controlá-la,
ele está contrapondo a violência à arte, ou seja, trata-se daqueles que
fazem o uso da violência para conter os imprevistos, ao invés de usar de
artifício ou da inteligência. Novamente, a violência não é vista como
qualidade, antes como debilidade daquele que não possui engenho o
bastante para controlar a natureza.

Entretanto, a referência principal ao uso da violência é o exemplo do
capítulo \versal{VIII}, quando, ao tratar de Agátocles (que usou de violência
para conquistar o comando da cidade), diz: ``Ao ser investido em tal
posto (o de pretor), decidiu tornar-se príncipe e manter com violência e
sem obrigação a outrem aquilo que lhe tinha sido concedido por um
acordo'' (\emph{O Príncipe}, cap. \versal{VIII}, 6). Neste caso, parece
que o uso da violência não somente gerou o resultado esperado para
Agátocles, a conquista do governo de Siracusa, como parece ter sido
legítimo e necessário, dando a entender que não há problemas no uso da
violência para a conquista do governo. Entretanto, adiante Maquiavel
apresenta a sua posição sobre esse uso da violência:

\begin{quote}
{[}10{]} Não se pode também chamar de \emph{virtù} matar os
seus cidadãos, trair os amigos, agir de má-fé, sem piedade, sem
religião: meios estes que permitem conquistar poder, mas não glória.
{[}11{]} Porque, se se considera a \emph{virtù} de Agátocles ao entrar e
ao sair dos perigos, e a grandeza do seu ânimo ao suportar e superar as
coisas adversas, não se vê porque ele haveria de ser julgado inferior a
qualquer excelentíssimo capitão: todavia, a sua feroz crueldade e
desumanidade, com infinitos crimes, não permitiram que fosse celebrado
entre os excelentíssimos homens (\emph{O Príncipe}, cap. \versal{VIII}, 10-11).
\end{quote}

Para Maquiavel, a \emph{virtù} de Agátocles não é plenamente
\emph{virtù}, falta-lhe a glória, ou seja, falta-lhe a admiração que o
governante deve produzir de sua condição de liderança política. Ainda
que haja respeito por parte dos comandados, ausenta-se, no caso desse
governante, a boa imagem que deveria se projetar do seu governo. Donde
essa \emph{virtù} não ser plenamente \emph{virtù} e de Agátocles não ser
o melhor exemplo de príncipe novo, justamente por esse uso desmedido e
``intolerável'' da violência.

Enfim, pelos usos do termo violência, ela não se insere no repertório
indicado das ações para o príncipe praticar, seja na conquista do
principado, seja na conservação. Porém, ainda resta o caso da crueldade,
como os assassinatos e as execuções relatadas ao longo do livro, que não
geraram nem perda da condição de comando político daqueles que as
praticou, muito menos a sua condenação.

A resposta neste caso nos remete a um ponto delicado e inovador do
pensamento político maquiaveliano acerca do uso da crueldade nas ações
de governo. Mesmo tendo criticado Agátocles, Oliverotto da Fermo e até
mesmo César Borgia pelo uso da crueldade na conquista e conservação do
principado, ao final do capítulo \versal{VIII} e depois com mais atenção no
capítulo \versal{XVII}, Maquiavel afirma que a crueldade não é todo má para o
exercício do governo, ao contrário, podendo ser muito útil para gerar
temor e respeito. Mas em todos os casos citados e explicitamente no
capítulo \versal{XVII}, a crueldade produz o efeito de exemplo para disciplinar
as condutas e não como prática louvável em si. A crueldade é tratada
mais no efeito que gera sobre os cidadãos, tornando-os temerosos e
obedientes, do que pela força em si mesma da ação cruel. Como diz
Maquiavel ao final do capítulo \versal{VIII}:

\begin{quote}
{[}27{]} Donde é de se notar que, ao pilhar um governo, deve o
invasor fazer todas aquelas afrontas que são necessárias, e fazê-las de
uma só vez, para não ter de renovar tudo e para poder, não as renovando,
tranquilizar os homens e ganhá-los ao beneficiá- los. {[}28{]} Quem faz
de outro modo, ou por timidez ou por mau conselho, sempre precisa ter a
faca na mão; também não pode nunca se apoiar nos seus súditos, nem podem
estes, pelas injúrias recentes e contínuas, jamais confiar nele.
{[}29{]} Por isso, as injúrias devem ser feitas todas de uma só vez, a
fim de que se saboreiem menos e afrontem menos; os benefícios se devem
fazer pouco a pouco, afim de serem melhor saboreados (\emph{O Príncipe},
cap. \versal{VIII}, 27-29).
\end{quote}

Enfim, a crueldade pode ser uma prática de governo a ser adotada, na
medida em que é exemplar, mas não como rotina, porque neste caso ela
desperta o ódio, que é diferente do temor e mais perigoso para o
príncipe, visto que gera opositores aguerridos. Raciocínio muito
semelhante ocorre com os usos do termo ``força'' (\emph{forza}) e seus
correlatos: \emph{forzare} e \emph{forzati}. Das treze ocorrências no
livro\footnote{São elas: cap. \versal{II}, 3; \versal{III}, 50; \versal{VI}, 16, 20, 21, 22; \versal{VII}, 43; \versal{VIII}, 30;
  \versal{XI}, 17; \versal{XVI}, 11; \versal{XVIII}, 2; \versal{XIX}, 37 e\versal{XX}, 20.}, em cinco Maquiavel se refere aos que foram forçados, ou
seja, impelidos, sendo pois passivos diante da força de outrem. Em
outros cinco casos, ele se refere à necessidade da força e nos três
restantes, a noção de forçar, como o sentido de ter que se utilizar da
força para conquistar, ou seja, num sentido positivo de tomar a
iniciativa de forçar. Considerando esses últimos oito casos no qual a
ação de força parte do príncipe ou de seu governo, em apenas três casos
ele tece considerações sobre o uso desta força.

Na primeira ocorrência, cap. \versal{III} (50), é dito que os governantes que
fundam principados para outros, tornando esses poderosos, podem gerar a
sua própria ruína, pois esses novos principados, fundados ou pela força
ou pela astúcia, nunca são confiáveis. Maquiavel está criticando a
transferência de força de um principado para outro, como prejuízo para
aquele que doa esse poder. Para o principado novo, a força recebida é,
evidentemente, um fator positivo, ainda que ele não emita qualquer juízo
a respeito, mas cuja conclusão é óbvia: força política não se transfere.

A segunda ocorrência do termo força, cap. \versal{VI} (22), está inserida em uma
argumentação capital para o que estamos tratando aqui, a saber, sobre a
necessidade da força no exercício do governo. O capítulo tem como tema a
conquista de principados por meio das armas próprias e da \emph{virtù},
momento no qual Maquiavel mobiliza quatro exemplos históricos de
príncipes conquistadores: Moisés, Teseu, Ciro e Rômulo.

A dificuldade que nasce na metade da exposição refere-se à necessidade
desse novo príncipe ter que constituir novos ordenamentos políticos,
fonte de conflitos e oposições. O argumento é evidente: ao ter que
instituir novos ordenamentos políticos, ou seja, reordenar o governo da
cidade, modificando funções, criando novas atribuições, extinguindo-o
outras, retirando algumas pessoas de certos cargos, colocando outras,
nessa ação o príncipe ganha novos inimigos e verifica, segundo
Maquiavel, que os seus amigos não são árduos defensores de seu governo.
É nesse momento de crise, no qual se aumenta a oposição e não se tem uma
base de apoio político confiável que o príncipe deve se valer das armas
para dar sustentação aos novos ordenamentos nos quais está fundando. As
armas se fazem necessárias justamente para forçar a aceitação desse novo
ordenamento político. O fundo do argumento de Maquiavel, também exposto
nos \emph{Discursos}, se apoia no fato de que os homens são volúveis e é
preciso medidas de força, e não apenas o convencimento ou o exemplo da
\emph{virtù}, para que esse novo governo consiga se instalar na cidade.
Nos \emph{Discursos,} ele também se vale da religião como forma de
persuasão para o povo, sem contar que estamos em um contexto republicano
na exposição dessa obra. N'\emph{O Príncipe}, então, a necessidade do
uso da força para a implantação de novos ordenamentos está diretamente
ligada à necessidade de exércitos próprios, o que confere novos
contornos a essa noção. Numa primeira leitura, pareceria que já teríamos
em Maquiavel a apresentação da necessidade da força na figura do
príncipe para o exercício do governo, até mesmo para sua constituição.
Todavia, ainda que seja necessário usar de força, esse uso somente terá
um valor positivo para o governo se ele for acompanhado de \emph{virtù}
e armas próprias.

Ora, conforme ele demonstrará nos capítulos 12, 13 e 14, de \emph{O
Príncipe}, das forças militares a disposição de um príncipe, a mais
adequada e aconselhável são as armas próprias, em outros termos, o
príncipe deve se valer, fundamentalmente, dos exércitos compostos por
cidadãos da cidade em sua maioria. O motivo deste exércitos serem os
mais adequados está na sua composição: são os próprios cidadãos que se
tornam soldados, criando desse modo um vínculo de fidelidade à cidade, à
pátria, que não se encontra com mesma intensidade nas outras formas de
forças militares.

Assim, mesmo no caso do príncipe que precise forçar, sua força deve
estar fundada em armas próprias, ou seja, ele deve possuir uma condição
política de líder dos seus concidadãos, proeminência política essa
apoiada também numa força militar formada por cidadãos.

Não vamos explorar aqui esse aspecto, mas ainda que brevemente, importa
lembrar a importância da dimensão política dos exércitos para Maquiavel,
que foi objeto de sua análise na \emph{Arte da Guerra}, notadamente no
livro \versal{I}. Ele não somente defende enfaticamente essa condição do soldado
ser um cidadão, como também mostra os bons efeitos e a necessidade desse
elementos para a vida da cidade. Na verdade, Maquiavel está retomando um
\emph{tópica} clássica do `cidadão soldado' que remonta a Cícero e a
Vergério, e que teve inúmeros outros defensores entre os pensadores
latinos.

Enfim, conforme fica claro no capítulo 14 do \emph{Príncipe}, o exército
próprio não é somente um instrumento de defesa indispensável, mas ele
também produz um engajamento político muito saudável na cidade. No caso
desse príncipe novo que deseja assumir o governo, se ele tiver essa
força militar para lhe apoiar, isso robustecerá sua condição de líder
político e facilitará seu governo.

Entretanto, essa introdução forçada de novos ordenamentos se apresenta
sempre como um problema, como declara na linha 17 desse capítulo \versal{VI}:
``\emph{E deve-se considerar que não há coisa mais difícil de tratar,
nem mais duvidosa em obter, nem mais perigosa em manejar, do que
fazer-se chefe para introduzir} {[}forçar{]} \emph{novos
ordenamentos}''. Maquiavel não rejeita a força, utilizar-se dela para
introduzir algum ordenamento não é tarefa isenta de riscos e
dificuldades, o que poderia soar como uma contradição, visto que o seu
uso parece redundar no engrandecimento da figura política do governante.
Contudo, esse não é o caso, a força por si só não é benéfica, dependerá
dos seus efeitos ou resultados para que se reconheça seu real lugar para
o governo do príncipe novo.

Ora, se articularmos essa exigência de que os exércitos sejam formados
pelos próprios cidadãos, com o que é apresentado ao final do capítulo
\versal{VI}, verifica-se que essa necessidade de introduzir (forçar) novos
ordenamentos apoiados nos exércitos próprios na fase inicial da fundação
do governo, tudo isso esvazia a ideia de que o príncipe novo detém em si
uma força ou domínio que impõe aos seus súditos. Ainda que Maquiavel
diga explicitamente que o príncipe precise introduzir novos
ordenamentos, tal procedimento se faz apoiado nos exércitos formados a
partir dos cidadãos (e por que não considerá-los como seus
partidários?), para que, uma vez modificados os ``costumes políticos'',
esse príncipe novo seja, então, admirado e consolide o seu apoio. O
texto é claro nesse sentido:

\begin{quote}
É necessário, portanto, querendo discorrer bem sobre esta parte,
examinar se estas inovações se sustentam por si mesmas ou se dependem de
outros, isto é, se para conduzir a sua obra, precisa rezar ou pode
forçar. {[}21{]} No primeiro caso, sempre entendem mal e não leva a
coisa alguma, mas, quando dependem de si próprios e podem forçar, então
é que raras vezes correm perigo. Daqui nasce que todos os profetas
armados venceram e os desarmados se arruinaram. {[}22{]} Porque, além
das outras coisas ditas, a natureza dos povos é variada e é fácil
persuadi-los em uma coisa, mas é difícil sustentá-los nesta persuasão.
Porém, convém ser ordenado de modo que, quando não crêem mais, pode-se
fazer crerem pela força. {[}23{]} Moisés, Ciro, Teseu e Rômulo não
teriam podido fazer observar sua constituição longamente caso estivessem
desarmados, como no nosso tempo sucedeu com o frei Jerônimo Savonarola,
o qual arruinou os seus novos ordenamentos, quando a multidão começou a
não acreditar nele, e ele não tinha como manter firmes aqueles que
haviam acreditado nele, nem fazer crer os descrentes. {[}24{]} Porém,
estes tem grande dificuldade no conduzir, e todos os seus perigos estão
no seu caminho, e convém que os superem com a \emph{virtù}. {[}25{]}
Mas, uma vez superadas essas adversidades, começam a ser venerados,
tendo perdido aquela sua qualidade que lhe tinham invejado, permanecendo
fortes, seguros, honrados e felizes (\emph{O Príncipe}, cap. \versal{VI}, 20-25).
\end{quote}

Note-se que a força se coloca tão somente no momento da fundação
política e não como prática constante de governo, menos ainda como um
atributo que emana da pessoa do príncipe. Nem estamos aqui levando em
conta se essa força se realiza com ou sem crueldade, mas apenas como ela
se apresenta no texto. É evidente, portanto, que essa noção de força
possui um estatuto mais fraco do que terá nas definições de soberania em
Jean Bodin ou mesmo nas definições de governo em Thomas Hobbes. Tal
força do príncipe, conforme demonstrado, não se aproxima também da noção
de \emph{dominium}, ao contrário, visto que nos exemplos expostos o
príncipe tem que levar em conta o jogo das forças contrárias dos humores
para poder obter êxito nessa disputa e assim conquistar ou conservar o
poder. Ou, por outro viés, o modo como a força e a crueldade são
apresentadas no texto, levando sempre em conta a sua real necessidade e
o efeito que geram, revelam que ambas, tomadas em si mesmas, não
perfazem em elementos essenciais e permanentes do exercício do governo.

Enfim, conforme extraímos do texto, as noções de força e crueldade,
principalmente do cidadão comum (\emph{privato}) que se torna príncipe
novo, não podem se apresentar como elementos permanentes e constitutivos
do governo, mas como estratagemas para a fundação e, de modo inusual no
caso da crueldade, para conservação do governo. Donde ser possível
afirmar que, no mínimo, esse príncipe novo de Maquiavel aproxima-se
muito à tradição antiga, na medida em que é o exercício do governo que
faz deste indivíduo príncipe.

Todavia, muitos podem argumentar em contrário, mostrando que justamente
por essas referências à força, à crueldade e às armas poder-se-ia
sustentar que em Maquiavel temos os elementos constitutivos
característicos do pensamento político moderno\footnote{Como já
  mencionado, essa discussão está bem apresentada no artigo de Ames
  (2011).}. Mas ainda assim, tais argumentos, em sua maioria, forçam uma
interpretação moderna do pensamento político maquiaveliano, na medida em
que renegam justamente essas heranças do pensamento político latino
sobre o Florentino.

Finalmente, verifica-se que \emph{O Príncipe,} de Maquiavel conserva
elementos importantes da tradição dos `espelhos de príncipes', que
iluminam e fornecem novos contornos para o que se está sendo exposto.
Por essa nova perspectiva, a obra ganha um contorno menos moderno, se
poderíamos dizer assim, na medida em que esse príncipe é mais próximo de
um cidadão comum que assume a regência e a condução, do que a figura
política do monarca moderno que encarna em si a natureza do Estado.

Essa análise sobre a noção de principado nos conduz a uma dúvida sobre
os tipos de principados, que se dividem em: hereditário, mistos, novos e
eclesiásticos. Haveria ainda o principado absoluto, no qual podem todos
eles podem se transformar. Contudo, destes todos, Maquiavel concentra
mais sua atenção ao principado civil, que é uma forma de principado
novo, não somente na primeira parte do livro, como na segunda, quando
trata da figura do príncipe, considerado sempre como príncipe novo.

Sobre o principado civil, analisado no capítulo \versal{IX}, sabemos que algumas
de suas características já vinham sendo apresentadas antes e o seu
modelo de governante, o príncipe novo, é doravante o personagem político
principal da obra. Caso essa atenção dada a esse tipo de regime político
não fosse o bastante para concentrarmos nossas atenções, o lugar
conceitual desse principado no pensamento político maquiaveliano é
certamente um ponto nevrálgico. No limite, esse modelo de principado
apresenta elementos teóricos que ao mesmo tempo que rompem com os
modelos tradicionais de principados e dos ``espelhos de príncipes'',
obrigam o leitor a considerar melhor qual a verdadeira relação deste
regime com o republicanismo defendido por Maquiavel nas suas outras
obras. Essa dificuldade já foi explorada por vários comentadores que não
é conveniente aqui retomar o debate nos seus detalhes\footnote{A lista é
  de fato extensiva, contudo apresentamos algumas indicações para
  orientar o leitor: Sasso, Cadoni, Martins\ldots{}}. Para nosso
interesse, pretendemos explorar alguns aspectos dessa noção de
principado civil que corroboram a nossa tese de que temos em \emph{O
Príncipe}, de Maquiavel, um texto que se coaduna com o seu pensamento
republicano. Isso não deve implicar em dizer que o principado civil é
uma república, pois são coisas diversas e o próprio Maquiavel
informa-nos disso. Porém, encontramos nesse tipo de principado elementos
de uma dinâmica política que, por um lado, pressupõe que esse regime
nasceu de um governo republicano e que conserva inúmeros traços dessa
forma de governo, podendo (e aqui convém insistir no caráter hipotético
do termo) fazer com que a cidade volte ao regime republicano, embora
isso não seja uma necessidade ou destino, mas uma possibilidade conforme
o rumo dos acontecimentos.

Em uma leitura superficial deste capítulo \versal{IX}, alguns aspectos já chamam
a atenção: a sua denominação como civil -- e o que entende-se aqui por
civil --; o seu fundador é um cidadão comum (\emph{privato ciptadino});
esse principado não é fundado pela violência e nem por crime, mas pelo
consenso; esse príncipe é escolhido (alguns comentadores chegam a
declarar que ele é eleito); é neste principado que Maquiavel relata que
há dois humores antagônicos em disputa; há uma apresentação das formas
de governo (principado, liberdade e licença); há uma referência ao final
à transformação em principado absoluto (que não é analisado na obra, mas
que podemos deduzir o que seja pela arquitetônica do argumento); é o
capítulo mais conceitual e com menos exemplos históricos (apenas um).
Todos esses elementos indicam que se trata de um dos momentos mais
conceituais da obra, de maior elaboração teórica. Passar por todos esses
aspectos seria tema de uma tese, e, novamente, lembrando nossa intenção
inicial, desejamos chamar a atenção para alguns aspectos tão somente.

Inicialmente a própria denominação de principado civil é digna de nota,
visto que civil é um termo que remete à condição de civilidade, por
oposição ao súdito. Como é notório ao longo do capítulo, esse principado
conserva uma dinâmica política na qual os membros da cidade estão
envolvidos com os rumos da cidade, tomando partido na determinação do
governante ou fazendo oposição contra esse, ação política essa
engendrada seja pelos grandes, seja pelo povo. Ora, tal quadro revela
que estamos tratando de um contexto político de cidadãos e não súditos
que se inserem na vida política da cidade, algo totalmente distante de
uma monarquia ou regime autocrático e muito próximo da dinâmica política
republicana. Neste sentido, o termo ``civil'' nos mostra a presença da
civilidade, ou seja, embora seja um principado, ele não anula ou
extingue a iniciativa política dos cidadãos, ao contrário, o governo é
expressão dessa luta, visto ser o príncipe alguém que é escolhido.

Como já dito, esse príncipe aqui é o cidadão comum (\emph{privato
ciptadino}) que torna-se príncipe, o que pressupõe sua condição de
liderança política num contexto institucional de igualdade de condições
políticas entre os cidadãos. Mais ainda, conforme enfatizado por
Maquiavel já na primeira linha, esse cidadão comum se torna príncipe sem
o uso da força, da violência, da crueldade, ou seja, sem o recurso às
armas, logo, utilizando-se de meios pacíficos e dentro da normalidade
institucional para alcançar o governo da cidade. O que não significa que
esse cidadão não seja um comandante militar ou tenha uma força armada
que esteja na retaguarda, mas que ela não é usada e não é o sustento de
sua conquista. Esse cidadão peculiar chegará ao comando da cidade,
principalmente, fundado nos apoios políticos que angariou em seu
processo de ascensão ao governo. Enfim, esse cidadão se torna príncipe
dentro das normas de civilidade, ele conquista o governo de forma
cívica.

Esse aspecto não pode ser menosprezado. Ao longo da história da
humanidade como um todo, mas particularmente considerando a história
política dos romanos e dos povos latinos que de Roma descenderam, ou
restringindo mais ainda, tendo em consideração apenas as cidades da
península itálica dos séculos anteriores a Maquiavel, conquistar o
governo da cidade de forma cívica, sem o uso das armas, da violência era
uma fato, no mínimo, inusual. Nos próprios exemplos arrolados no livro,
verifica-se quantos foram os casos de tomada do governo por meio de
assassinatos, mortes, crimes, golpes de estado e quão poucos os casos de
conquista por vias pacíficas e institucionais. Falar da conquista do
governo de um principado sem derramamento de sangue, calcado na escolha,
é algo muito peculiar.

A forma da instalação do governo é um outro dado chamativo, pois diz
Maquiavel que esse principado é originado dos grandes ou do povo. Mais
específico ainda, de um lado, ``\emph{porque, vendo os grandes que não
podem resistir ao povo, começam a aumentar a reputação e o prestígio de
um dos seus e fazem-no príncipe para poderem, sob sua proteção,
desafogar o seu apetite.}'' (cap. \versal{IX}, linha 3). De outro, temos o povo,
que, não resistindo a esse desejo dos grandes, adere a alguém que venha
defendê-los contra as ânsias dos grandes. O governo, portanto, é
originado, causado pelos grupos políticos, tem nesses sua fundação,
tanto no sentido temporal -- originando-se deles --, quanto no sentido
de sustentação. Nesse aspecto é um governo conduzido por um só, mas que
não tem neste personagem nem sua origem (convém insistir neste aspecto,
visto que Maquiavel usa o passivo ``\emph{è causato}'', é originado por
outro e não por aquele que vai governar) e nem sua fundamentação. Alguns
comentadores (Larivaille, 1997, Frosini, 2005) chegam a dizer que esse
príncipe é eleito pelos seus concidadãos. Apesar da ideia não ser
estranha ao modo como o argumento é montado, Maquiavel não utiliza nem o
termo ``eleição'' e nem ``escolha'', o que nos parece uma extrapolação,
pois nos ritos políticos da Florença do tempo de Maquiavel, o que havia
eram regras muito rígidas sobre aqueles que seriam aptos aos cargos
políticos (as magistraturas) que eram sorteados a partir de nomes
colocados em uma bolsa. Totalmente estranho a esse contexto florentino
seria um processo de eleição, muito menos de votação, donde nosso
incômodo com essa terminologia. O que se verifica pela economia do
argumento é que um cidadão é alçado à condição de príncipe com o
sustento político de uma das partes, num claro sentido de escolha ou
predileção de um grupo que conduz esse personagem ao governo. Tão
importante quanto esse dado é a informação, não explícita, que há um
modo de institucionalização do governo que se funda em uma disputa
cívica e pacífica. O povo não coloca alguém no governo por ter armas ou
por meio da força militar e nem os grandes por algum golpe. Ora, então
somos obrigados a admitir que esse cidadão comum chega ao principado
após uma disputa política institucionalizada, no qual as partes se
apresentam na cena pública e sustentam seus prediletos. Há, pois, um
palco de disputa política no qual as partes podem tomar parte, os grupos
podem defender os seus interesses de modo pacífico e institucionalizado,
enfim, há uma dinâmica política.

A argumentação maquiaveliana nos remete ao fim para o seu personagem
político principal: o príncipe. Mas o que é, ao fim, ao cabo, o
\emph{príncipe} de Maquiavel?

\subsection{O príncipe ciceroniano e o príncipe na república}

A noção de príncipe, conforme visto, teve um largo emprego entre os
pensadores desde o período romano, de modo que podemos encontrar várias
acepções e usos nos autores latinos durante vários séculos. Todavia, uma
questão pertinente seria verificar nessa história do conceito uma
acepção do termo \emph{príncipe} que não estivesse ligada
necessariamente ao regime monárquico ou, por outro lado e mais
importante ainda, se haveria algum emprego do termo, anterior a
Maquiavel, que o diferenciasse dessa acepção mais próxima ao
\emph{vivere civile} e distante do \emph{principado absoluto}. Como
estamos argumentando, tendo em vista o quadro conceitual do pensamento
político maquiaveliano como um todo, com suas nítidas posições
republicanas -- que se corroboram, como visto, pelo papel central do
conceito de cidadão comum (\emph{privato ciptadino}) na economia do
argumento de \emph{O Príncipe} --, cumpre encontrar uma possível fonte
da concepção de \emph{príncipe} que divergiria da acepção corrente de
governante em regime monárquico e o aproximaria de noções republicanas,
para que possamos mensurar o grau de inovação ou recuperação conceitual
operado por Maquiavel.

Uma primeira origem, como já mostramos, está nas noções herdadas da
medievalidade latina do \emph{regere}, no qual o governante tem essa
condição em função do exercício da sua ação de governo, não exercendo o
poder político de forma absoluta. Uma outra fonte que vem se associar a
esta é o uso que o termo \emph{príncipe} apresenta no pensador romano
Cícero e como essa noção pode ter servido de fundamento para a reflexão
maquiaveliana. Ora, não somente essa origem do termo \emph{princeps}
definido por Cícero, mas as aproximações que se podem fazer dessa
acepção nos \emph{Discursos} e em \emph{O Príncipe} corroboram a nossa
hipótese de que Maquiavel forja um conceito muito particular de príncipe
como governante, não necessariamente monárquico e compatível com um
regime republicano, ou um governo a ``meio caminho'' entre a monarquia e
a república.

\subsubsection{O \emph{princeps} ciceroniano}

Ettore Lepore (1954), em seu estudo sobre a noção de \emph{princeps} no
pensamento político ciceroniano, ao recuperar as origens do termo, nos
revela semelhanças conceituais entre esse conceito romano, da tarda
república, com o \emph{príncipe} em Maquiavel, que permitem uma
interpretação diversa da noção de príncipe como uma figura tipicamente
monárquica.

Ao analisar nas obras de Cícero o uso e os contextos nos quais é
mobilizado o termo \emph{princeps} e, até mesmo, quem é esse
\emph{princeps}, Lepore apresenta um conceito pertencente ao ordenamento
republicano romano, notadamente, ao contexto da tarda república (séculos
\versal{II} e \versal{I} a.C.). Dado esse que já nos mostra que, durante a república
romana, fonte de inspiração dos republicanismos posteriores, inclusive o
florentino, havia esse personagem político. Donde se constata que, para
Cícero, há um \emph{princeps} no interior do ordenamento político
republicano, que ocupa um papel destacado na estruturação do regime.

A análise de Lepore tem, também, a preocupação de reconhecer se o termo
é meramente um nome diverso -- talvez numa acepção mais literária -- ou
se ele é um conceito, de acordo com o seu emprego nos escritos políticos
ciceronianos. Seus estudos nos permitem dizer, pois, que esse conceito
de \emph{princeps} se revela em seus contornos definitivos nas três
obras políticas principais de Cícero -- \emph{De officis, De Leggibus} e
\emph{De republica} --, momento esse de maturidade intelectual do
pensador romano e de seu distanciamento dos ideais aristocráticos que
marcaram seus primeiros escritos, como diz: ``\emph{o termo `princeps'
está presente em todo o desenvolvimento do pensamento político
ciceroniano como perfeito equivalente aos outros termos com o qual
designa o homem político}'' (\versal{LEPORE}, 1954, p. 34).

Concentrando suas atenções, primeiramente, às ocorrências do termo
\emph{princeps}, Lepore verifica que Cícero se vale de dois termos muito
próximos: \emph{princeps e principes}. O primeiro é usado de vários
modos: \emph{princeps-rector, gubernator, moderator, tutor, procurator,
conservator} etc. (\versal{LEPORE}, 1954, p. 34-35), acepções estas que revelam a
associação do termo a uma função política de comando ou a um cargo ou
magistratura de relevo no ordenamento republicano romano. Em todas essas
ocorrências, mostra-se ainda o predomínio de duas compreensões do
\emph{princeps}: o primeiro, em ordem cronológica, é o melhor em
comparação a um grupo. Ao lado dessa dupla acepção, encontra-se uma
terceira, pois se verifica que o \emph{princeps} é também aquele que
toma a iniciativa da ação política, aquele que lidera e está na
vanguarda, o responsável pelo princípio da investida política. Tais usos
indicam que, para Cícero, o \emph{princeps} é uma figura política de
proa, que possui uma \emph{virtus} política destacada e, por isso, se
põe à frente na ação política. Esse \emph{princeps} é, portanto, o
primeiro a agir, o primeiro ou líder em uma iniciativa política.

Constatado esse primeiro bloco de acepções correntes do termo
\emph{princeps} nos textos ciceronianos, Lepore parte para uma outra
vertente de investigação, no afã de descobrir as fontes teóricas do
termo. Tendo em vista a tradição filosófica grega, da qual Cícero é
herdeiro, e dos usos do termo, o \emph{princeps} se aproxima em muito ao
\emph{politikós} grego. \emph{Politikós} esse que não é tanto o rei, o
\emph{basileu}, mas o homem que tem sua natureza conformada pela
\emph{polis}, ou seja, o \emph{princeps} tem as mesmas funções e as
mesmas incumbências políticas do \emph{politikós}, ou seja, ele é alguém
que deve partilhar as magistraturas na \emph{polis}. Um exemplo notório
é que os generais, os juristas, os filósofos, os oradores, enfim, homens
que não eram necessariamente governantes e não tinham cargos públicos de
destaque são designados como \emph{princeps} (Lepore, 1954, p. 48-49).
Então, assim como o \emph{politikós} era concebido como o político por
excelência numa comunidade de políticos -- sem necessariamente ser o
governante --, conforme definido por Aristóteles na \emph{Política}
(\versal{III}, 2, 1275b19), do mesmo modo o \emph{princeps} é um cidadão dotado
de virtude política que assume a liderança de uma ação política entre
iguais. Tal uso equivalente dos termos mostram que, ``\emph{mesmo
mudando o âmbito da linguagem filosófica grega,} {[}o emprego
ciceroniano{]} \emph{é assimilado perfeitamente pela experiência
lingüística romana}'' (Lepore, 1954, p. 45). Ao \emph{princeps} se
associa, então, um ideal de homem político que é, de um lado, herdeiro
da tradição grega do \emph{politikós}, e, por outro, agrega as
qualidades ou \emph{virtus} própria do cidadão romano.

Em contraposição, o \emph{principes}, diferentemente do \emph{princeps},
era uma conceito de homem político ligado aos antigos valores
aristocráticos, de um contexto próprio da \emph{concordia ordinum},
quando se buscava uma conciliação entre as ordens patrícias ou
senatoriais, como no governo da \emph{nobilitas}, ou seja, como o ideal
de governo da aristocracia do início da república romana. Nesta acepção,
esse \emph{principes} é um típico aristocrata que se coloca entre
aristocratas, num círculo político seleto e restrito. Ora, quando não
mais se está colocada a questão em termos de ordenamentos
aristocráticos, mas num quadro de intensas disputas e dissensões
políticas, há uma mudança nessa conceituação de \emph{principes} para um
novo modelo político, o \emph{princeps.} A mudança no quadro político,
de um contexto de \emph{concordia ordinum} para a preocupação com o
\emph{consensus}, evidenciada depois de 60 a.C., leva Cícero a rever o
modo de conceber o seu ideal de homem político. A distinção de
\emph{principes} e \emph{princeps,} sendo o primeiro um conceito
tradicional, ligado à aristocracia, e o segundo como a prefiguração de
um \emph{novus hominus}, um novo político, é fruto das novas exigências
políticas após 60 a.C. Avançando ainda mais, o novo \emph{princeps} é
associado ao \emph{popularis,} ao \emph{sapienter popularis} (\emph{De
Republica,} \versal{II}, 54), ou seja, ao cidadão dotado de prudência, tal qual
se diz de Péricles (\emph{De Oratore,} \versal{III}, 138), a clássica figura
grega que encarna a prudência política do governante. \emph{Popularis}
esse que, desde um uso anterior a Cícero, era sinônimo de \emph{civis},
donde constata Lepore (1954, p. 216):

\begin{quote}
O vocábulo popularis assumiu na tradição retórica e naquela mais
antiga e redescoberta, aquilo que o faz simplesmente equivalente de
\emph{civis}, tendo valor estritamente técnico, como o encontramos no
âmbito filosófico, para exprimir o complexo de valores inclusos no
grego \emph{politikós}.
\end{quote}

Segundo Lepore (1954, p. 230), tal mudança se deve a uma percepção mais
isocrática da política romana por parte de Cícero, o que na verdade era
antes uma crença para a superação das antigas \emph{ordines}
aristocráticas, por regimes mais moderados ou mistos. Na verdade, depois
das sucessivas crises políticas da república romana, Cícero, ao
contrário da política defendida pelos patrícios, se posiciona em favor
de mudanças no regime republicano no sentido de diminuição da hegemonia
patrícia e das diversas ordens aristocráticas, para a incorporação de
novos atores políticos vinculados à plebe. Ora, a concepção política
aristocrática, que pressupunha uma ordem estática e não dinâmica do
campo político, estava em cheque após o advento dos conflitos políticos
que resultaram nas várias guerras civis da tarda república romana. Em
face da guerra civil, Cícero passa a admitir uma dinâmica do mundo
político, aceitando o conflito como um fato próprio da vida republicana,
mas propondo uma \emph{contentio sapiens}, que discipline essa luta
política contra a possibilidade da sedição. É sob tal ótica que se deve
ler o livro \versal{VI} do \emph{De Republica}, tendo em vista o dissenso
político que exigem um \emph{princeps moderator} e \emph{prudens,} ou
seja, um novo homem político que reconheça as mudanças e não esteja mais
preso aos modelos estáticos e conservadores próprios do patriciado. As
mudanças contra as quais ele deve reagir são aquelas que ameacem a
destruição dos ordenamentos, ``\emph{frutos dos egoísmos e das paixões
dos homens indignos de serem aceitos nos círculos dos princeps''}
(Lepore, 1954, 251). Nota-se, pois, a adoção de uma nova visão, mais
dinâmica e orgânica da vida política, dos seus elementos e dos seus
contrastes como característica fundamental do pensamento ciceroniano,
particularmente no momento de composição do \emph{De republica,} no
período de seu exílio após o Consulado, na década de 50 a.C.

Então, essas considerações sobre o conceito de \emph{princeps}
ciceroniano impedem qualquer associação desse com um ideal político de
tipo monárquico. Nos inúmeros trechos das obras ciceronianas citadas por
Lepore, fica muito difícil, para não dizer impossível, que esse conceito
fizesse remissão ao governo de um só, ao chefe de um corpo político que
concentra em si os poderes decisórios. Ao contrário, a noção de
\emph{princeps} se associou cada vez mais ao conceito de
\emph{politikós} grego ou de \emph{civis} romano, ao cidadão que toma
parte na vida política da cidade e que por vezes lidera ou dá a
iniciativa da ação política, sem ser o ponto de concentração do poder
político. Como diz: ``\emph{Todos os termos até aqui usados não permitem
identificar o ideal de} princeps \emph{com um poder monárquico ou de
qualquer modo a um singular}'' (Lepore, 1954, 71).

Esse retorno às origens do termo \emph{princeps} em Cícero mostra que
não somente é inadequado caracterizá-lo como uma designação de monarca
ou congênere, mas, ao contrário, seu uso, principalmente nas obras de
maturidade, revela um emprego terminológico muito próximo do cidadão em
um contexto republicano. Figura essa inserida num mundo político não
mais caracterizado pela ordenação estática aristocrática, mas regido
pela dinâmica das disputas políticas, que deve lutar contra as sedições
e consequente dissolução do ordenamento político da cidade.

Portanto, no pensamento político ciceroniano, particularmente nos textos
políticos de maturidade, a noção de \emph{princeps} remete de modo
direto e inequívoco a um contexto republicano e não monárquico. Ainda
dentro desse contexto republicano, como visto, o \emph{princeps} se
identifica a um modelo de novo homem político em um contexto não mais
dominado pela lógica aristocrática do \emph{consensus ordinum}, mas
inserido numa configuração política não dominada pela elite senatorial,
e sim pelas lutas e tensões políticas entre os diversos grupos políticos
da tarda república romana. Ora, esse quadro de significações do termo
\emph{princeps} na tradição política ciceroniana influencia o pensamento
político latino posterior e, certamente, Maquiavel. Desse modo, cumpre
entender as possíveis aproximações dessa terminologia ciceroniana com as
noções de príncipe mobilizadas nos capítulos \versal{VIII} e \versal{IX} de \emph{O
Príncipe}, bem como com os \emph{Discursos}, capítulos \versal{IX} e \versal{X}, no qual
ele também faz um largo uso da noção de \emph{príncipe}.

\subsubsection{O príncipe dos \emph{Discursos}}

Comecemos pelos \emph{Discursos,} onde, nos capítulos \versal{IX} e \versal{X} do
livro, Maquiavel aborda o papel que devem desempenhar os
ordenadores de reinos e repúblicas, mobilizando, para isso, uma série de
exemplos de governantes romanos que tiveram êxito ou fracassaram nesse
trabalho. Tema esse que já havia sido tratado em parte no capítulo \versal{II}
deste mesmo livro \versal{I}, quando da análise da fundação das cidades. É
particularmente nesses capítulos que o governante único ou a figura do
príncipe é citada várias vezes como o responsável pela fundação ou
reordenação política da cidade. Assim, antes de entrar na análise desses
capítulos em questão, faz-se necessário resgatar o contexto
argumentativo no qual eles se inserem.

No ``\emph{Pequeno tratado sobre as repúblicas''}, como já indicado, o
centro da reflexão maquiaveliana é, pois, apresentar os fundamentos das
repúblicas, sua estruturação, para, em seguida, ou seja, tendo como
referência tais concepções, interpretar a história romana. Então, tanto
o capítulo \versal{II} quanto os capítulos \versal{IX} e \versal{X} do livro dos \emph{Discursos}
inserem-se nesse itinerário argumentativo que busca determinar os
fundamentos da república, não somente romana -- seu caso exemplar --,
mas da noção de república de modo geral.

No capítulo \versal{II}, depois de ter analisado a fundação da cidade por meio de
um ordenador (no caso, a Esparta de Licurgo), Maquiavel passa a tratar
do caso romano, no qual lamenta o fato desta não ter tido a mesma sorte
daquela. Ao contrário de Esparta, os ordenamentos romanos nascem dos
conflitos entre os nobres e a plebe, sendo isso caracterizado como acaso
ou como acidentes. Ao lado dessa contraposição, uma outra mais
significativa tem lugar, pois, diz ele:

\begin{quote}
Porque Rômulo e todos os outros reis fizeram muitas e boas leis, ainda
em conformidade com a vida livre: mas, como sua finalidade foi fundar um
reino, e não uma república, quando aquela cidade se tornou livre,
faltavam-lhe muitas coisas que cumpria ordenar em favor da liberdade,
coisas que não haviam sido ordenadas por aqueles reis. E, se bem que
aqueles seus reis perdessem o poder pelas razões e nos modos narrados,
aqueles que os depuseram, ao constituírem {[}ordinandovi{]}
imediatamente dois cônsules para içarem no lugar dos reis, na verdade
depuseram em Roma o nome, mas não o poder régio: de tal forma que, como
só tivesse cônsules e senado, aquela república vinha a ser mescla de
duas qualidades das três acima citadas, ou seja, o principado e
optimates. Faltava-lhes apenas dar lugar ao governo popular: motivo
porque, tornando-se a nobreza romana insolente pelas razões que abaixo
se descreverão, o povo sublevou-se contra ela; e, assim, para não perder
tudo, ela foi obrigada a conceder ao povo a sua parte, e, por outro
lado, o senado e os cônsules ficaram com tanta autoridade que puderam
manter suas respectivas posições naquela república (\emph{Discursos},
\versal{I}, 2, 18-19).
\end{quote}

Como destacam Sasso (1987) e Reale (1985), chama atenção a rápida
passagem da forma monárquica para a forma republicana. Passagem esta que
estabelece a república como o lugar das análises que se seguiram nos
demais capítulos. Mas, de qualquer modo, o Rômulo citado no capítulo \versal{II}
é uma figura monárquica que deu os fundamentos para a constituição de
uma república ou de um ordenamento político conforme o \emph{vivere
civile}. Por outro modo, caso se queira considerar que Maquiavel tecia
suas considerações levando em conta uma monarquia, estas se encerram
neste momento final do capítulo \versal{II}.

Entretanto, tendo em vista isso que foi dito no capítulo \versal{II}, a
mobilização de Rômulo no capítulo \versal{IX} ganha novos contornos e
dificuldades. Neste capítulo dos \emph{Discursos}, ao retomar o papel de
Rômulo na dinâmica política romana, a perspectiva de análise é outra,
equiparada a de um fundador em um contexto de disputa política, com
vistas à república. A questão central está em como entender esse
governante único, esse primeiro monarca romano no meio de uma análise
voltada para o estabelecimento dos fundamentos da república. Ora, também
no capítulo \versal{IX}, em seu início, Rômulo é apresentado como um fundador de
cidades, não se levando em conta a dificuldade que perpassa as
considerações sobre o nascimento da república romana.

Porém, tanto no capítulo \versal{II} quanto no \versal{IX}, essa transição em si não é
problematizada, não é analisada a fundo. Concomitantemente a esse pouco
falar ou mesmo não falar da transição constitucional, o regime
republicano se apresenta como um \emph{telos}, uma finalidade à qual
Roma parecia destinada. Apesar da imprevisibilidade sobre o que seria no
futuro, Roma apresenta-se destinada a se transformar numa república,
como se ela estivesse orientada para tanto desde seus momentos
primordiais. Segundo Sasso: ``Como se fosse, na realidade, o
\emph{telos} a constituir, além do fim e ao fim do processo, também o
seu critério, a sua origem, a sua razão de ser, o seu impulso condutor''
(\versal{SASSO}, 1987, p. 128). Assim, à débil análise da transformação
política se contrapõe a profundidade de uma necessidade teleológica de
Roma se tornar uma república. A busca dessa condição política
republicana instala-se na reflexão maquiaveliana e passa a conduzir o
seu raciocínio.

Os argumentos mobilizados, então, têm como objetivo a transformação de
uma monarquia, que nasce tumultuada pelo assassinato de Tito Tazio e
pautada por tumultos, numa república completa em meio às vicissitudes. A
questão que nasce da análise do capítulo \versal{II} e da complementação do \versal{IX}
era que Roma tinha como destino não a instauração de uma monarquia
perfeita, mas de uma república. No quadro apresentado por Maquiavel,
desde o seu nascedouro, Roma estava destinada a se transformar numa
república, pois os eventos convergiam para esse fim. Como ele repete ao
longo desses capítulos, quando se olha para o \emph{fim} e não para o
ato em si, a instauração de um \emph{vivere libero} esteve sempre no
horizonte. Esta era uma motivação encontrada já nos primeiros reis, ou
seja, os ordenamentos políticos iniciais tinham como força indutora a
instalação de um \emph{vivere libero}, forma essa que se completará ou
se realizará perfeitamente no modelo republicano:

\begin{quote}
Mas voltemos a Roma. Embora Roma não tivesse um Licurgo que no princípio
a ordenasse de tal modo que lhe permitisse viver livre por longo tempo,
foram tantos os acontecimentos que nela surgiram, devido à desunião que
havia entre a plebe e o senado, que aquilo que não fora feito por um
ordenador foi feito pelo acaso. Porque, se Roma não teve a primeira
fortuna, teve a segunda; pois se seus primeiros ordenamentos foram
insuficientes, nem por isso o desviaram do bom caminho que a pudesse
levar a perfeição (\emph{Discursos}, \versal{I}, \versal{II}, 30-33).
\end{quote}

Estabelece-se, assim, o quadro conceitual das afirmações feitas por
Maquiavel nas primeiras linhas do capítulo \versal{IX}, nas quais se insere a
análise dos fundadores na sequência da exposição sobre as instituições
políticas, aparentemente não apresentando nenhuma relação com os temas
tratados anteriormente. O que se mostrava inicialmente no capítulo \versal{II}
era uma reflexão sobre a fundação por meio do legislador e suas
possíveis implicações sobre a história romana. No capítulo \versal{IX}, apesar de
retomar a temática da origem constitucional, a chave de leitura não é a
fundação, mas a ordenação, que, do ponto de vista da compreensão da
estrutura política romana, apresenta uma outra perspectiva, visto que já
se considera o substrato material do conflito político.

Tributária dessa compreensão é a figura de Rômulo, que, no capítulo \versal{II},
se assemelha à figura de Licurgo. Nessa tentativa de traçar um paralelo
entre as duas personalidades, Maquiavel ressaltava as carências do rei
romano em comparação com o legislador espartano. No capítulo \versal{IX}, ao
contrário, o que nasce é a figura de um outro Rômulo, não mais a versão
romana e imperfeita de legislador conforme descrição anterior, mas o
responsável pela instalação de um processo de ordenamento constitucional
que fará de Roma uma república poderosa.

Conforme Reale (1985, p. 45), o que seria uma aparente questão retórica,
ganha os contornos de uma questão real, se levarmos em conta o fato de
Maquiavel ter tratado apenas dos fundadores de cidades e não dos
reformadores, o que, no limite, impõe a questão da fundação e da reforma
ou reordenação da cidade. O problema parece não estar restrito à
temática da fundação das cidades, muito menos a uma retomada do papel do
legislador nesse momento inaugural. A afirmação maquiaveliana evidencia
uma sutileza terminológica que se configura como um problema de fundo.
Maquiavel fala em termos de \emph{ordenadores} e não de
\emph{fundadores} ou \emph{legisladores}, que, em um primeiro momento,
poderiam ser compreendidos como sinônimos ou como possuidores de função
igual na origem das cidades. O início do capítulo vem aprimorar a
compreensão do papel do ordenador, que passa a se diferenciar do
legislador, como foi Licurgo (\versal{REALE}, 1985, p. 46). O exemplo seria
Rômulo que, ao ser identificado como o ordenador de Roma, ao mesmo tempo
se diferencia dos fundadores, expostos de início. A figura do fundador
foi apresentada como aquele que concebia a cidade e suas instituições
por critérios racionais, dotando este universo político de mecanismos
estáveis e seguros em função de sua própria racionalidade. Essa
racionalidade, marca distintiva da ordenação designada pelo legislador,
contrapõe-se à ordenação segundo o acaso. Levando-se em conta o
legislador helênico, vê-se que ele disporá a cidade segundo um
``critério geométrico, segundo uma regra racional, em um verdadeiro
cosmo de leis'' (\versal{SASSO}, 1987, p. 121)\footnote{Essa imagem do ordenador
  helênico concebendo a cidades segundos critérios racionais e
  geométricos é um dos pontos centrais da argumentação de Jean-Pierre
  Vernant. Em sua explicação, a racionalização da vida e a
  racionalização do campo político estão extremamente imbricadas no
  mundo grego, cujo melhor exemplo seria Hipodamo de Mileto, contratado
  para reconstruir a sua cidade, o fazendo de modo geométrico, donde
  tudo ser ordenado a partir do centro que é a \emph{ágora}. Como nos
  diz Vernant: ``Ele a reconstrói segundo um plano de conjunto que marca
  uma vontade de racionalizar o espaço urbano.'' Concluindo: ``Ora,
  deve-se constatar que o domínio político aparece tão solidário de uma
  representação do espaço que acentua, de maneira deliberada, o círculo
  e o centro, dando-lhe um significado muito definido. {[}\ldots{}{]} A esse
  respeito, pode-se dizer que o advento da Cidade manifesta-se de início
  por uma transformação do espaço urbano, isto é, do plano das cidades.
  É no mundo grego, sem dúvida, primeiro nas colônias, que aparece um
  \emph{plano novo} {[}grifo nosso{]} de cidade em que todas as
  construções urbanas são centradas ao redor de uma praça que se chama
  ágora. {[}\ldots{}{]} Para que exista uma ágora é preciso um sistema social
  de vida implicando, para todos os negócios comuns, um debate político.
  A existência da ágora é a marca do advento das instituições políticas
  da cidade'' (\versal{VERNANT}, 1995, p. 245).}. Isso permite afirmar que
haveria, de um lado, uma ordenação conforme o \emph{logos} e, de outro,
uma ordenação mediante a fortuna, o que não significa que o \emph{logos}
esteja excluído da fundação das cidades que não tiveram a sua origem
pela mão do legislador, mas apenas que não é esse seu critério
prioritário. Ao final do capítulo \versal{II}, Maquiavel se refere a Rômulo,
mesmo não podendo equipará-lo a Licurgo, como aquele que fez ``muitas e
boas leis, conforme ao \emph{vivere libero}'' (\emph{Discursos}, \versal{I}, \versal{II},
32), ou seja, o primeiro rei romano teve também a intenção de bem
conformar a cidade. Do ponto de vista do projeto, Rômulo também figura
entre os fundadores da cidade, embora nesse momento do texto não fosse
ainda possível perceber se Maquiavel falava de um legislador ou de um
ordenador.

Entretanto, paralelamente a essa fundação conforme o \emph{logos},
Maquiavel chama a atenção para um outro modo de ordenação política da
cidade feita pelos acidentes. O termo ``acidente'' sugere várias
acepções, entre elas imprevistos ou acontecimentos não regulares que
alteram o curso político, bem como, acidente como oposto à essência,
retomando um vocabulário aristotélico. O desenvolvimento do texto tende
a reforçar o primeiro aspecto, haja vista o papel dos tumultos políticos
para a instauração do ordenamento constitucional. Todavia, levando-se em
conta que a fundação pelo legislador é conforme o \emph{logos}, sendo
isto uma busca de modelação da natureza do corpo político conforme a
razão, em tais condições, os tumultos, os acidentes também podem ser
compreendidos como algo que se insere no corpo e o modifica, à
semelhança de uma forma acidental ou causa acidental. Associação que não
deveria ser absurda, pois a tradição aristotélica medieval desenvolveu
tanto o conceito de forma acidental como o de forma substancial, que não
está presente na \emph{Metafísica} de Aristóteles, fazendo deles
conceitos-chave para os sistemas metafísicos medievais, principalmente
depois de Averróis e Tomás de Aquino. Ora, mesmo sabendo não ser muito
adequado utilizar um jargão tributário desses sistemas metafísicos nos
textos políticos maquiavelianos (tendo em vista a ausência de uma
reflexão metafísica por parte de Maquiavel), essa acepção de acidente no
sentido de causa acidental pode ser aceitável se considerarmos as
implicações de uma fundação conforme o \emph{logos}. Pensando na
contraposição evidente entre \emph{logos} e acidente, esses fatos que
alteram a ordenação da cidade e inserem algo de novo em sua natureza --
como é o caso dos tribunos da plebe -- podem também ser compreendidos
como formas acidentais. Independentemente da interpretação que se queira
adotar para a compreensão do termo ``acidente'', o que essa retomada do
papel do ordenador no capítulo \versal{IX} vem problematizar é seu estatuto para
a compreensão da formação das instituições. Conhecido desde o início do
livro o papel do legislador e do \emph{logos}, bem como os problemas
decorrentes de uma tal fundação, as atenções se voltam para a
necessidade de se entender os acidentes na fundação de uma cidade.
Ordenação por acidentes que, ao ser pensada no confronto com a fundação
racional do legislador, adquire novos contornos.

O trabalho realizado pelo legislador de conformar as instituições
políticas segundo um critério racional tem, como uma de suas
consequências, a perenidade dessa instituição. Nesse sentido, a
conformação segundo o \emph{logos} pretende retirar da esfera temporal
as constituições políticas (\versal{SASSO}, 1987, p. 120-132). De fato, a
constituição perfeita, o governo misto, coloca-se fora da circularidade
temporal, perfazendo uma linearidade. Por ser um ordenamento político
acabado, pode-se chamá-lo de perfeito, entendendo-o não como o melhor
dos regimes, mas como aquele que não carece de nada, conforme a
conceituação clássica grega. Tal é a constituição política que nasce do
trabalho do legislador.

Já quanto à fundação ordenada pelo acaso, diferentemente da perfeita,
ela se define por não estar acabada e, consequentemente, por estar
submetida às vicissitudes do tempo, à fortuna. Como demonstra Maquiavel
no capítulo \versal{II} dos \emph{Discursos}, teríamos uma gradação de fundações
em três níveis: a perfeita, a ``menos perfeita'' (mas com possibilidade
de reforma), e uma terceira classe de fundações imperfeitas, sem nenhuma
possibilidade de reforma. Roma encontrar-se-ia nesse segundo grupo,
sendo uma constituição a princípio imperfeita, mas que ao longo do tempo
foi se aperfeiçoando, ou seja, foi se reordenando. Os acidentes, nesse
contexto, são os atributos que aperfeiçoam o regime, que o conduzem à
perfeição; são os agregados que se unem a um corpo político
pré-existente e o modificam. Assim, a perfeição (como ``acabamento'' e
não como ``ausência de defeitos'') pode ser possível para esse segundo
grupo, que não fica refém de um determinismo naturalista que impede que
uma constituição imperfeita se transforme numa perfeita. A
perfectibilidade não é um dado inserido apenas no momento de criação do
regime, mas pode ser, para Maquiavel, uma possibilidade ao longo da
existência, sujeita ao acaso.

A garantia dessa perfectibilidade joga essas constituições inacabadas
para a esfera do tempo, submetidos que estão à fortuna. Embora seja
possível alcançar a perfeição, ela se realiza num quadro de dependência
relacionado à esfera temporal, às subidas e descidas, sem
previsibilidade. Se a fundação segundo o \emph{logos} retira o regime
perfeito das variações temporais, a ordenação segundo os acidentes
insere totalmente o corpo político na História, no tempo. É na História,
no interior do tempo, que essa constituição se perfaz, se aperfeiçoa,
agrega a si aquilo de que carece, no caso, as instituições que a tornam
perfeitas. Essa constituição imperfeita está sujeita, também, à dinâmica
do tempo, à variação dos fatos, dos acidentes e, por isso, deve estar
aberta às mudanças, disposta a incorporar aquilo que o tempo lhe traz
como novidade.

Assim, não é meramente uma questão terminológica a distinção entre a
fundação de uma cidade pelo legislador e uma ordenação segundo os
acidentes. Ao expressar que Roma teve uma ordenação e não uma fundação,
Maquiavel demarca o campo teórico no qual deve ser pensada a
constituição romana, que de nenhum modo pode ser equiparada às
repúblicas conformadas por um legislador.

Retomando a questão de como entender esse príncipe dos capítulos \versal{IX} e \versal{X}
do livro \versal{I} dos \emph{Discursos}, temos agora a constituição de uma outra
imagem que não somente a do governante único. Conforme se depreende da
análise da figura de Rômulo, o governante único desses capítulos está
inserido em um contexto republicano, marcado por lutas e dissensões
políticas entre os dois grupos principais. Importa frisar: nas duas
referências ao rei romano, Maquiavel parece ignorar a passagem da
monarquia para a república, e passa a tratar de Roma num contexto
republicano. Roma, nesses capítulos, é considerada em sua fase
republicana, que, na verdade, esteve no meio de duas formas de governos
centralizados: a monarquia anterior à fundação republicana e o império,
posterior a essa fase. Mesmo quando se faz referência a César no
capítulo \versal{X}, onde Maquiavel expõe toda a sua crítica a ele, o contexto
político é republicano e as críticas se devem em grande parte por ter
César contribuído para a destruição da república. Esse governante único
que foi César, ao invés de recuperar o \emph{vivere civile}, aprofundou
a dominação e retirou o pouco que restava de liberdade da república.

Então, os governantes desses capítulos, tendo em vista as circunstâncias
políticas nas quais estão inseridos e o papel político que devem
desempenhar -- o de ordenar ou reordenar um regime --, são considerados
em termos de governantes executivos em condições republicanas. Logo, não
parece existir a possibilidade de qualificá-los como típicos governantes
monárquicos, que centralizam o poder político na figura do chefe de
governo. Mesmo quando considerados unicamente como reordenadores,
Maquiavel enfatiza que eles devem agir em vista do restabelecimento dos
ordenamentos republicanos. Com efeito, ao declarar que ``um ordenador
prudente e virtuoso não deve deixar por herança a autoridade que tomou''
(\emph{Discursos}, \versal{I}, \versal{IX}, linha 8), que remeteria à importância
de um governante único, ele destaca que a \emph{herança} não pode ser
essa autoridade excepcional, mas cuidar para deixá-la nas mãos de
muitos. Ou seja, mesmo que haja uma reordenação por meio de um só, o
resultado deve ser a instalação do governo de muitos. Neste sentido, o
fim de toda a ordenação visada nessas passagens dos \emph{Discursos} é
um regime republicano e não a perpetuação de uma dinastia.

Essa preocupação com o \emph{vivere civile} é tão importante que
Maquiavel insiste neste mesmo capítulo \versal{IX} e no \versal{X} sobre a ameaça de
instalação de um governo absoluto ou tirânico. Nesses é que se encontra
o grande temor: que, após a reordenação de um regime, o poder fique nas
mãos de um só homem ambicioso que usaria mal aquilo que virtuosamente
foi conquistado (\emph{Discursos}, \versal{I}, \versal{X}, linha 8-10). O mesmo se aplica
a César, que não foi o reordenador da república, mas o seu destruidor,
ao contrário de Rômulo (\emph{Discursos}, \versal{I}, \versal{X}, linha 30).

Então, agora pode ser possível entender porque esse príncipe dos
capítulos \versal{IX} e \versal{X} não é um típico monarca, mas, quando muito, um
reordenador de um regime com vistas à recuperação dos ordenamentos
republicanos. Ele é antes de tudo um líder, um precursor de um processo
político, muito próximo ao ideal de \emph{princeps} de Cícero. Como ele
mesmo diz: ``Aquele que se tornou príncipe nalguma república deve
considerar que, depois de Roma tornar-se Império, mais merecem louvores
os imperadores que viveram de acordo com as leis e como príncipes bons,
do que aqueles viveram de modo oposto'' (\emph{Discursos}, \versal{I}, \versal{X}, linha
16).

Enfim, Maquiavel se vale, nesses capítulos, de uma figura de príncipe em
contexto republicano, o príncipe como um reformador de regime, que não
parece se identificar de nenhum modo com a imagem tradicional de monarca
soberano que centraliza o poder político, muito menos com um governante
tirânico, que é a figura contrária desse príncipe.

Enfim, tanto os paralelos suscitados pela noção de \emph{princeps}
ciceroniano, quanto esse uso da noção de príncipe nos primeiros
capítulos dos \emph{Discursos} nos mostram que o termo se distancia
muito de uma acepção autocrática ou monárquica e se aproxima da figura
política do líder que conduz a cidade, seja reordenando as instituições,
seja fundando novos ordenamentos, estabelecendo uma nova dinâmica
política.

Retornando a \emph{O Príncipe}, quando consideramos aquilo que
Maquiavel atribui ao príncipe civil do capítulo \versal{IX}, conforme já
expusemos (p. 73), mas que convém retomar:

\begin{itemize}
\item
  \begin{quote}
  em quase todos os casos apresentados, à exceção do príncipe herdeiro,
  o príncipe de Maquiavel tem o sua condição de governante fundada antes
  no seu exercício político do que no território ou reino;
  \end{quote}
\item
  \begin{quote}
  o príncipe deve, em função de sua condição e daquilo que o legitima no
  governo, reger, conduzir e liderar a cidade -- algo que ganha
  contornos claros e dramáticos no capítulo final do livro;
  \end{quote}
\item
  \begin{quote}
  No caso do uso da violência, apesar de ser apresentado como
  estratagema para a conquista da condição de príncipe, Maquiavel deixa
  claro que ela não é \emph{virtù,} e sua mobilização no texto não tem a
  mesma justificação que terá posteriormente o argumento do uso legítimo
  da violência pelo Estado. Esse, na verdade, é um dos pontos pantanosos
  de \emph{O Príncipe}, pois, ainda que não tenhamos a mesma
  caracterização da violência estatal da modernidade, ela se faz
  presente como elemento da ação política do príncipe.
  \end{quote}
\end{itemize}

Ora, evidencia-se, enfim, que, nos usos e acepções dados particularmente
ao príncipe civil e nos atributos necessários a este (conforme é exposto
nos capítulos seguintes até o fim da obra), essa personagem não se
indentifica com o monarca, mas com a figura de um lider político em
contexto de disputa política que, se não é de fato republicano, é um
palco no qual o conflito político e a disputa entre as forças
antagônicas se fazem presentes.

Mais ainda, recuperando a noção política do \emph{regere} e como
Maquiavel apresenta a figura do príncipe civil como uma conquista do
cidadão comum (\emph{privato ciptadino}), tudo isso configura de outro
modo esse personagem político central, não somente em \emph{O Príncipe},
mas para o pensamento maquiaveliano. Por todos esses elementos expostos
aqui, verifica-se que o príncipe é uma figura política muito próxima,
senão identificada, ao cidadão (\emph{politikós} ou \emph{civis}), que
assume a liderança política da cidade, tendo como desafio principal
fundar ou reordená-la institucionalmente.

Contudo, convém destacar que Maquiavel está o tempo todo tratando de um
tipo de príncipe, o príncipe civil, que não é o único exemplo e nem um
modelo de conduta política que pode ser universalizado. Nos vários
escritos políticos, ele deixa claro e cita que um príncipe pode se
tornar num tirano e fundar um regime centralizado nele. Enfim, no
pensamento político maquiaveliano não há só esse príncipe civil sendo
retratado, mas é este certamente a figura política central para sua
obra.

\subsubsection{O príncipe civil entre república e principado}

Neste ponto, poderíamos dar por concluída nossa exposição, porém, na
análise do príncipe civil nasce uma última dificuldade. Em todas as
circunstâncias apresentadas aqui desse cidadão que se torna príncipe, há
um pano político de fundo de uma república (donde tratarmos de
``cidadão'' e não de ``súdito''), que parece passar por crises, gerando
a necessidade desse cidadão dotado de \emph{virtù} para liderar a cidade
nessa tarefa de fundação ou reordenação política. Esse é o quadro
político, seja nos \emph{Discursos,} livro \versal{I}, capítulos \versal{IX-X} e
\versal{XVI-XVIII}, seja em \emph{O Príncipe}, na primeira parte do livro,
particularmente, nos capítulos \versal{VIII} e \versal{IX}. No limite, a dificuldade pode
ser enunciada nos seguintes termos: quais são as condições políticas que
geram a necessidade da fundação do principado por esse \emph{princeps}
que é um cidadão dotado de \emph{virtù}? Isso nos remete a recuperar a
questão de como pensar a relação entre a república, modelo político do
qual nasce essa figura do príncipe, e o principado? Não somente como de
uma se passa a outra, no caso da república ao principado, mas, também,
se seria possível que um principado pudesse se tornar uma república.

Associar os capítulos dos \emph{Discursos} que tratam da corrupção
(capítulos \versal{XVI}, \versal{XVII} e \versal{XVIII}) com \emph{O Príncipe},
particularmente os capítulos \versal{VIII} e \versal{IX}, é uma hipótese que pode apontar
para a compreensão da solução daquilo que Maquiavel define como a cidade
corrompidíssima. Conforme diz Sasso:

\begin{quote}
Entre o nono capítulo do \emph{Príncipe} e os capítulos dezesseis,
dezessete e dezoito do primeiro livro dos \emph{Discursos}, existe uma
relação sutil e complexa {[}\ldots{}{]} No nono {[}capítulo{]} do
\emph{Príncipe}, a premissa do raciocínio e da análise teórica, é
fornecida por uma forma republicana que, em vista do `excessivo'
conflito dos `humores', o prevalecer dos `grandes' e, paralelamente, o
desencadear-se das paixões populares estão, depois de te-las restituídas
numa `igual' desigualdade, sempre `aprofundando' mais na corrupção
(\versal{SASSO}, 1987, p. 396-397).
\end{quote}

Seja numa relação de gênese, seja numa relação de consequência, as
imbricações entre essas obras necessitam de maiores considerações, a fim
de que se chegue aos limites reais dessa relação. Tanto na perspectiva
de causa quanto na perspectiva de consequência, o importante é entender,
primeiramente, os termos da relação entre república corrompida e
principado civil. Como também diz Sasso, mais importante que o ângulo
visual, deve-se destacar o contexto da cidade corrompidíssima dos
\emph{Discursos} que precede ou está num

\begin{quote}
{[}\ldots{}{]} tempo anterior àquele considerado no \emph{Príncipe}, no
qual a passagem à forma monárquica ou principesca é já, por assim dizer,
considerada atual e imanente ao consenso que os ``populares'' e os
``grandes'' concederam à iniciativa do ``privado'' na cidade
corrompidíssima (\versal{SASSO}, 1987, p. 398).
\end{quote}

Com efeito, conforme apontamos antes sobre as características do
principado civil, as condições que permitem a um cidadão tornar-se
príncipe num principado civil estão já presentes na cidade republicana
dos \emph{Discursos}. A presença dos dois humores, seus desejos
antagônicos, o conflito político entre ambas, a necessidade do
governante em não se apoiar nos grandes, mas saber controlar os seus
desejos, enfim, todas essas condições que pautam as circunstâncias da
existência do principado civil já estavam presentes na república que se
encaminha para a corrupção.

Retornando ao capítulo \versal{XVIII} dos \emph{Discursos}, ao levarmos em conta
o projeto que Maquiavel tem em vista -- reordenar a cidade
corrompidíssima --, o indivíduo que assume esse empreendimento assume
também para si uma autoridade que compete ao príncipe. Seja ele um
ditador ou um \emph{gonfaloniere}, seja ele o primeiro cidadão dotado de
\emph{virtù}, o ponto central é que a esse indivíduo deve-se atribuir um
tal poder \emph{extraordinário,} estranho ao ordenamento republicano e
muito próximo ao príncipe que assume um principado civil. Ora, se a
solução é extraordinária, a questão se desloca para a transição dos
regimes, ou, em outras palavras, a dificuldade está em como pensar essa
passagem de uma república corrompida para o principado civil.

Antes de tratar da passagem, convém destacar o tipo de principado que se
tem em vista. Estamos tratando do principado civil como o regime
político proposto como solução, mas não seria essa a única opção, pois,
por outro lado, nada impediria o estabelecimento de uma tirania ou de
algum tipo de governo despótico e autoritário. Os regimes tirânicos (ou
nem tão tirânicos, como o governo do Turco, que era um governo
centralizado, mas não necessariamente tirânico ou absoluto) não são
sugeridos em nenhum momento como a solução mais adequada ou a mais
consequente para as condições republicanas. O problema parece ser que
esses principados, que podem ser identificados como absolutos, negam ou
anulam os conflitos políticos ocasionados pelos humores presentes na
cidade. Ao concentrarem todo o fundamento da ação política no
governante, impedem o ``natural'' funcionamento da vida política e, por
consequência, impedem que os grupos ou os humores manifestem seus
desejos pelo meio natural de luta política dentro do corpo político,
aquilo que caracteriza a civilidade política. O principado civil,
conforme visto, é aquele regime que mais assegura o \emph{vivere
libero}, que respeita e garante os conflitos, pois os assume como
inerentes à vida política no principado. Esse será o ponto central: ao
contrário do principado de tipo absoluto, o principado civil conserva os
aspectos básicos da vida política numa república, não anula por completo
o \emph{vivere libero}, a civilidade, o jogo político e os conflitos que
lhe são inerentes; antes os reconhece e os assume como dados essenciais
do principado. O maior problema em se considerar a transição de uma
república corrompida para um principado de tipo absoluto é a
possibilidade de subtração completa das características presentes na
primeira, não mais reconhecidas e existentes nesse tipo de regime. Logo,
o principado de tipo absoluto, apesar de ser uma solução possível, não
pode ser compreendido como a mais adequada para uma cidade que necessita
ou \emph{conservar-se como republica ou criá-la de novo}, que é o
problema central do capítulo \versal{XVIII} dos \emph{Discursos}.

Neste sentido, convém retomar o que Maquiavel expõe no capítulo \versal{X} do
livro \versal{I} dos \emph{Discursos}, quando trata dos reformadores de Roma.
Paralelamente a sua crítica a César, visto como um dos principais
destruidores da república romana, a análise que se segue dos imperadores
romanos visa ressaltar, fundamentalmente, que: aqueles que alcançaram o
império por herança foram maus, ao contrário daqueles que o assumiram
com o apoio dos seus concidadãos (\emph{Discursos}, \versal{I}, \versal{X}, linha 20); que
os \emph{príncipes} (termo do próprio Maquiavel para se referir aos
imperadores que se seguiram a César) que procuraram reordenar o reino e
fazer com que as instituições funcionassem conforme a sua finalidade,
foram mais bem sucedidos em relação àqueles que procuraram, por meio
delas, conquistar glória para si. Donde conclui:

\begin{quote}
E o príncipe que realmente buscar a glória mundana deverá desejar ter
nas mãos uma cidade corrompida, não para destruí-la de todo, como César,
mas para reordená-la, como Rômulo. E realmente, os céus não podem dar
aos homens maior ocasião de glória, nem os homens podem desejar glória
maior. E, se, para bem ordenar uma cidade, houvesse necessidade de depor
o principado, mereceria alguma desculpa quem não a ordenasse para não
cair de tal posição, mas, em sendo possível manter o principado e
ordena-lo, não merece desculpa algum quem não o faça (\emph{Discursos},
\versal{I}, \versal{X}, 30-32).
\end{quote}

Não está descartada, seja nos \emph{Discursos}, seja em \emph{O
Príncipe}, a possibilidade das repúblicas se transformarem em tiranias
ou que os principados tornem-se regimes despóticos. Essas
degenerescências políticas, se pudermos chamarmos assim esses regimes
autocráticos, não estão nas atenções de Maquiavel. Ele está preocupado
com a solução para a recuperação da civilidade política, donde ser o
principado civil o regime que melhor se adequa a esse fim. O modo como é
ordenado o principado civil permite, pois, considerá-lo como o regime
que melhor prepara o povo ou o universal (também denominado como matéria
em vista de uma forma) para o \emph{vivere libero} e o \emph{vivere
civile.} A instalação desse novo governo principesco confirma a hipótese
de que, apesar da derrocada do governo anterior, a cidade na qual ocorre
essa transição política conserva os aspectos essenciais de legalidade
política, de civilidade, de um respeito, ainda que mínimo -- poder-se-ia
conjecturar --, aos ordenamentos políticos, aos valores cívicos.

Aqui cabe ressaltar o porquê de Maquiavel nomear esse tipo de principado
como ``civil''. Evidentemente, é em função dele se instalar reconhecendo
a dinâmica dos conflitos e respeitando esse dado essencial durante o
governo, sem descambar para um governo absoluto ou aristocrático, bem
como numa espécie de populismo ou, anacronicamente falando, numa
``ditadura do proletariado''. O principado civil é civil por instaurar e
estimular a dinâmica política fundamental para a vitalidade da cidade,
por recuperar e conservar a civilidade.

Retomando a análise, verifica-se ainda que o quadro no qual Maquiavel
descreve a origem do principado civil é muito semelhante a uma
república. Tendo em vista a origem não dinástica do príncipe, podemos
questionar até se o regime político anterior ao principado civil era uma
monarquia ou não. A transição que não deve ser calcada na violência, mas
no consenso, a presença dos humores, que tensionam o governo, enfim,
todos esses aspectos elencados em \emph{O Príncipe} remetem à dinâmica
política retratada nos dezoito primeiros capítulos dos \emph{Discursos}.

Por isso, se pretendemos pensar numa gênese do principado civil, devemos
concordar que uma hipótese, talvez a mais provável, é a da cidade
republicana que atinge um certo grau de corrupção e não consegue, por si
só, retomar o seu ordenamento político inicial. O desenrolar do capítulo
\versal{IX} de \emph{O Príncipe} comprova ainda mais essa constatação inicial,
pois Maquiavel mostra como o príncipe novo, que chega ao poder nessas
condições, deve se comportar diante do jogo de interesses e de poder que
permanece após a sua instalação no comando do principado. Pelo controle
dos humores e dos desejos, deve-se tomar todo o cuidado para não ficar
refém dos interesses dos grandes, mas manter um certo equilíbrio entre
os dois grupos principais (grandes e povo) e, quando isso não for
possível, apoiar-se totalmente no povo, ainda que isso implique certos
constrangimentos às suas decisões políticas.

Portanto, a descrição que emerge nesse capítulo \versal{IX} de \emph{O
Príncipe} sobre o principado civil coloca-o muito próximo do
ordenamento da cidade republicana e permite pensar que a transição de um
regime a outro não é uma inferência inadequada. Ao contrário, entre os
modelos de regimes que figuram no horizonte do possível nas descrições
maquiavelianas, o principado civil é o mais adequado às necessidades de
um governo forte exigidas ao final do capítulo dezoito do livro \versal{I} dos
\emph{Discursos}. Por conservar os elementos fundamentais da república
e, também, por manter a presença do essencial da vida política, com seus
humores e os conflitos entre eles, esse regime vem totalmente ao
encontro das exigências que a cidade corrompidíssima solicita para o seu
reordenamento. Seja ao tratar da corrupção nos capítulos \versal{XVI}, \versal{XVII} e
\versal{XVIII} do livro \versal{I} dos \emph{Discursos}, seja nesse capítulo \versal{IX} de
\emph{O Príncipe}, seja, ainda, no capitulo \versal{LV} do mesmo livro \versal{I}
dos \emph{Discursos}, entre as principais causas da corrupção está a
ambição dos grandes em tomar o poder. Em todos esses capítulos, bem como
em inúmeras outras partes, o desejo dos aristocratas em assumir o
comando do poder para si ou instalar um governante que lhe seja
favorável está sempre presente. Ora, mais do que pensar numa corrupção
endêmica e generalizada pela cidade, ao considerar-se a corrupção do
povo (a corrupção da matéria), encontrar-se-á mais um desejo de
usurpação dos grandes e menos uma desobediência às leis por parte do
povo em geral. Quando, pois, numa república dominada pelos grandes, não
se encontram mais meios de impedir esse avanço da aristocracia sobre o
poder, não há outro remédio senão instalar um governo forte, \emph{quase
régio}, sob a forma do principado civil:

\begin{quote}
Razão por que nessas províncias não surgiu nenhuma república nem nenhum
tipo de vida política; porque tais tipos de homens são totalmente
inimigos da civilidade. E não seria possível introduzir uma república em
províncias assim constituídas, mas, para reordena-las -- caso a alguém
coubesse tal arbítrio --, não haveria outro caminho senão constituir um
reino. A razão é que, onde a matéria está tão corrompida, não bastam
leis para contê-la, e é preciso ordenar junto com elas maior força, que
é a mão régia, que com poder absoluto e excessivo, ponha freio à
excessiva ambição e corrupção dos poderosos (\emph{Discursos}, \versal{I}, \versal{LV},
21-23).
\end{quote}

Nessa passagem, como em outras, repetem-se as mesmas exigências
apresentadas para a instalação de um principado civil em substituição à
república corrompida: excessivo poder da aristocracia, ineficácia das
leis e das instituições, um governo forte, mas que se instale sem
violência, a existência de uma parcela, ainda que mínima, de civilidade.
A necessidade de um governo forte não implica necessariamente a fundação
deste com o uso da violência: força e violência não se seguem.

Sobre a passagem do principado civil para uma república não há uma
exposição detalhada de Maquiavel em \emph{O} \emph{Príncipe} e nem na
parte inicial dos \emph{Discursos}. Talvez as considerações dos
capítulos \versal{IX} e \versal{X} do livro \versal{I} dos \emph{Discursos} pudessem fazer uma
alusão a isso, embora saibamos que Maquiavel está tratando da transição
da monarquia romana para uma república. Caso pensemos que esse príncipe
evocado nesses capítulos dos \emph{Discursos} seja um típico príncipe
civil -- hipótese essa não absurda, visto que o objetivo desse príncipe
é fundar ou reordenar as instituições com vista a um regime republicano
e o príncipe novo do principado civil, se não tem esse objetivo, deve ao
menos conservar o mínimo de civilidade que resta à cidade --, então
teríamos sim uma exposição da passagem do principado civil para uma
república.

Junto a essa possibilidade interpretativa, outro argumento que também
poderia auxiliar nessa compreensão da possível passagem do principado
civil para a república, está nos \emph{Discursus rerum Florentinarum},
escrito em 1520 sob encomenda do papa Leão \versal{X} em função da morte de seu
sobrinho e governante de Florença, Lorenzo de Medici. Neste opúsculo,
Maquiavel disserta sobre qual seria a melhor forma de governo a se
instalar na cidade naquele momento de crise do regime comandado pela
família Médici, que, segundo ele, é uma crise no ordenamento político da
cidade de longa data. Após demonstrar que Florença nunca teve de fato
nem uma república e nem um principado, ele sugere dois modos de
reordenar a cidade: uma ordenação verdadeiramente republicana ou uma
reordenação verdadeiramente principesca. Desenvolvendo um pouco mais,
diz ele que, caso esse ordenamento não seja assim, as duas formas de
regime entrarão em dissolução. Donde conclui:

\begin{quote}
{[}\ldots{}{]} digo que não se pode ordenar nenhum regime estável que não
seja um verdadeiro principado ou uma verdadeira república, pois todos os
regimes postos entre estes dois são defeituosos. A razão disso é
claríssima: se o principado tem apenas uma via para sua dissolução, que
é se tornar uma república, e da mesma maneira a república tem uma única
via para se dissolver, que é se tornar um principado, {[}\ldots{}{]}
(\emph{Discursus rerum florentinarum}, § 11).
\end{quote}

Para além de analisar os meandros argumentativos do opúsculo
maquiaveliano, o trecho citado corrobora as duas partes da nossa
hipótese. Já nos era claro que a república corrompida tem a
possibilidade de se transformar em um principado civil. Agora vemos o
próprio Maquiavel enunciar a possibilidade de um principado se
transformar em uma república. Portanto, conforme o pensamento
maquiaveliano, a transição de um principado para uma república é uma
possibilidade.

Acerca dessas transições, há duas coisas a ressaltar: em primeiro lugar,
em ambos os casos estamos tratando de uma possibilidade e não de uma
certeza de mudança de regime. Nesse sentido, é sempre bom lembrar que
não há nenhum caráter determinista ou finalista no pensamento político
maquiaveliano (no que diz respeito às mudanças de regime, isso
implicaria uma associação com os ciclos políticos de Políbio e sua
inexorabilidade). Um segundo aspecto é que Maquiavel não declara, no seu
opúsculo, que esse principado que se transforma em república é um
principado civil. Na verdade, ele se furta em fazer essa análise sobre o
principado, coisa que não ocorrerá com a república. Sobre o principado,
ele apenas tece algumas considerações sobre a condição do povo, se há
igualdade política ou desigualdade política.

Contudo, a mensagem reiterada ao papa é de que, mesmo que se instale um
principado verdadeiro, este deve ter em vista a inserção dos diversos
grupos políticos no governo da cidade, criando uma espécie de ``governo
\emph{largo}'', ou seja, um principado composto de várias magistraturas,
com a participação dos vários extratos políticos. Tal ordenação
prepararia o povo (preparar a matéria) com vistas à fundação de fato de
uma república na cidade, cuja medida principal consistiria numa
redistribuição dos cargos políticos entre os diversos grupos (diversos
humores) e a reabertura do Conselho Maior, que era o espaço político
onde todos podiam tomar parte e que foi fechado pelo governo Medici
instaurado em 1513. Note-se que Maquiavel não defende de imediato a
instalação de uma república, mas um principado que prepare o povo para o
regime republicano. Preparação essa que passa fundamentalmente pela
recuperação dos valores cívicos, o \emph{vivere civile}, que culminará
na fundação de uma república. Ora, isso é quase o que literalmente ele
apresenta nos \emph{Discursos}, no capítulo \versal{IX} e \versal{X} do livro \versal{I}, conforme
foi visto. O príncipe invocado no texto republicano comparece no
opúsculo de igual modo; em ambos os casos, as ações do príncipe
governante direcionam-se no sentido de recuperar a civilidade degradada
ou mesmo perdida. Modos e procedimentos esses que ficam muito claros em
\emph{O Príncipe}.

Portanto, é muito plausível pensar que a república corrompidíssima
encontra sua melhor solução de recuperação no principado civil,
evitando, ainda que precariamente, a instalação de uma tirania ou de um
regime absoluto. Principado esse que deve ter por finalidade a
preparação do povo para a volta ao regime republicano, desde que, e
devemos enfatizar esse aspecto, o príncipe aja com vistas a recuperar a
dinâmica dos conflitos políticos que expressam a saúde e a vitalidade da
cidade. Então, não é sempre que uma república se transforma em um
principado civil e não é necessário que um principado se transforme em
uma república. Em todos os casos há sempre a possibilidade dessas
mudanças, mas não como necessidade, e sim conforme as ações dos atores
políticos, no caso o príncipe, o povo e os grandes.

Finalmente, ressalte-se a coerência da reflexão política maquiaveliana,
possível de ser percebida tanto quando tomamos seus diferentes textos
como quando levamos em conta apenas um deles. A percepção de tal
coerência permite compreender quais são os pontos fundamentais da sua
reflexão política e as suas convicções teóricas mais caras. Unidade que
supera as possíveis objeções advindas de problemas relacionados à
cronologia das obras, e que passa a ser o dado mais relevante para a
afirmação de que, independentemente de qual obra tenha sido escrita
antes ou depois, a reflexão maquiaveliana se mostra coerente e
articulada. Enfim, pelo exposto, parece ser de todo evidente que \emph{O
Príncipe} de Nicolau Maquiavel não é um texto em defesa da monarquia,
mas uma obra que busca compreender as dinâmicas políticas, em perfeita
consonância com sua reflexão republicana.

\chapter*{}
\addcontentsline{toc}{chapter}{O príncipe}
\begin{center}
\begin{vplace}[0.3]
\Large
O príncipe
\end{vplace}
\end{center}
\thispagestyle{empty}