\movetooddpage
\addcontentsline{toc}{chapter}{Bibliografia}
\chapter*{Bibliografia}
\hedramarkboth{bibliografia}{o príncipe}

\begin{bibliohedra}
\section*{Obras de Maquiavel}

\tit{MACHIAVELLI}, Niccolò. \emph{De Principatibus}. Testo critico a cura di
G. Inglese. Roma: Istituto Storico Italiano per il Medio Evo, 1994;

\titidem. \emph{Il Principi.} Opere. V. 1. A cura di Corrado
Vivanti. Torino: Einaud/Gallimard, 1997.

\titidem. \emph{Il Principi}. Edizione Nazionale Delle Opere
-- I/1, a cura de Mario Martelli. Roma: Salerno Editrice, 2006;

\titidem. \emph{Discorsi sopra la prima deca di Tito Lívio.}
Introduzione di Gennaro Sasso, premessa al testo e note di Giorgio
Inglese. Milano: Rizzoli, 1984;

\titidem. \emph{Discursus Florentinarum rerum post mortem
iunioris Laurentii Medices}. Edizione Nazionale Delle Opere -- I/3, a
cura de Jean-Jacques Marchand, Denis Fachard e Giorgio Masi. Roma:
Salerno Editrice, 2001;

\titidem. \emph{Discursos sobre as formas de governo de
Florença}. Introdução, tradução e notas de Gabriel Pancera. Belo
Horizonte: ed. \versal{UFMG}, 2010.

\titidem. \emph{Discursos sobre a primeira década de Tito
Lívio.} Trad. Martins Fontes. São Paulo: Martins Fontes, 2007;

\titidem. \emph{História de Florença}. Trad. e notas de
Nelson Canabarro. São Paulo: Musa, 1994;

\section*{Fontes primárias}

\tit{ARISTÓTELES}. \emph{Política}. {[}edição bilíngue{]} Trad. Antônio C.
Amaral e Carlos C. Gomes. Lisboa: Vega, 1998;

\tit{CICERO}, Marco Tulio. \emph{I Doveri}. {[}edição bilingue{]}. Traduzione
di Anna Resta Berrile. Milano: \versal{BUR}, 2004;

\tit{CICERO}, Marco Tulio. \emph{La Repubblica.} {[}edição bilingue{]}. A cura
de Francesca Nenci. Milano: \versal{BUR}, 2008;

\tit{DANTE}\mbox{ } Aliguieri. \emph{A Divina Comédia.} Paraíso. {[}Edição
bilingue{]}. Tradução e notas de Italo Eugenio Moro. São Paulo: Ed. 34
Letras, 1998.

\tit{POLÍBIO}. \emph{Storie}. Libri \versal{V-VI}, {[}edição bilingue{]}. A cura di
Domenico Musti, traduzione di Manuela Mari, note di John Thornton. Vol.
\versal{III}. Milano: \versal{BUR}, 2002;

\tit{TITO}\mbox{ } Lívio. \emph{História de Roma.} Trad. Paulo Matos Peixoto. São
Paulo: Ed. Paumape, 1989. {[}6 vol.{]}

\section*{Fontes secundárias}

\tit{ADVERSE}, Helton. \emph{Maquiavel, a República e o Desejo de Liberdade.}
Marília: Revista Trans/Form/Ação, v. 30, p. 33--52, 2007;
\emph{Maquiavel. Política e Retórica}. Belo Horizonte: \versal{UFMG}, 2009.

\tit{AMES}, José Luiz. ``A formação do conceito moderno de estado: a
contribuição de Maquiavel'' in \emph{Revista Discurso}, São Paulo, 41,
2011 (293--328);

\titidem. \emph{Conflito e Liberdade.} A vida política para
Maquiavel. Curitiba: \versal{CRV}, 2017.

\titidem. Liberdade e conflito. O confronto dos desejos como
fundamento da ideia de liberdade em Maquiavel\emph{. Kriterion}, Belo
Horizonte, n. 119, p. 179--196, jul. 2009.

\titidem. \emph{Maquiavel: a lógica da ação política.} Cascavel:
Edunioeste, 2002;

\titidem. Transformação do significado de conflito na
\emph{História de Florença} de Maquiavel\emph{. Kriterion}, Belo
Horizonte, n. 129, p. 265--286, jun. 2014.

\tit{BARON}, Hans. \emph{The crisis of the early Italian Renaissance}.
Princenton: Princenton University Press, 1989;

\tit{BAUSI}, Francesco. \emph{I Discorsi di Niccolò Machiavelli}. \emph{Genesi
e structure}. Firenze: Sansoni, 1985;

\titidem. ``Il problema dei `Discorsi''', \emph{Interpress},
\versal{XIX}, (2000) {[}p. 249--261{]};

\tit{BENETTI}, Fabiana. O conceito de Stato em Maquaivel. Dissertação de
mestrado. Toledo: Programa de Pós-graduação em Filosofia -- Unioeste,
2010;

\tit{BERTELLONI}, Francisco. Quando a política começa a ser ciência
(antecedentes históricos e requisitos científicos da teoria política nos
séculos \versal{XIII} e \versal{XIV}), Revista Analytica, v. 09, n. 1, 2005 {[}p.
13--38{]};

\tit{BIGNOTTO}, Newton. \emph{Maquiavel Republicano}. São Paulo: Loyola, 1991;

\titidem. \emph{Origens do republicanismo moderno.} Belo
Horizonte: ed. \versal{UFMG}, 2001;

\titidem. \emph{O aprendizado da força} in Adverse, Helton.
\emph{Reflexões sobre Maquiavel.} 500 anos de O Príncipe. Belo
Horizonte: ed. \versal{UFMG}, 2015 {[}p. 87--107{]}

\titidem. \emph{Antropologia negativa em Maquiavel} in \versal{ANALYTICA}, Rio
de Janeiro, vol 12 nº 2, 2008 {[}p. 7--100{]}.

\tit{BORSELLINO}, N. \emph{Niccolò Machiavelli}, in \emph{Letteratura
Italiana}. Bari: Laterza, 1973. Vol. 4. t.1, {[}p. 35--180{]}.

\tit{CADONI}, Giorgio. \emph{Machiavelli. Regno di Francia e `principato
civile'}. Roma: Bulzoni Editore, 1974;

\tit{CARDOSO}, Sérgio. Em direção ao núcleo da `obra Maquiavel': sobre a
divisão civil e suas interpretações. \emph{Discurso}, São Paulo, n.
45/2, p. 207--247, 2015.

\titidem. Maquiavel: lições das Histórias Florentinas.
\emph{Discurso}, São Paulo, v. 48, n. 1, p. 121--154, 2018.

\tit{CHABOD}, Federico. \emph{Scritti su Machiavelli}. Torino: Einaudi, 1993;

\tit{CIZEK}, Eugen. \emph{Mentalités et institutions politiques romaines.}
Paris: Fayard, 1990;

\tit{CUTINELLI-RÈNDINA}, Emanuele. \emph{Chiesa e Religione in Machiavelli}.
Pisa-Roma: Istituto Editoriali e Poligrafici Internazionali, 1998;

\tit{DUSO}, Giuseppe. \emph{O Poder. História da filosofia política moderna.}
Trad. de Andrea Ciacchi, Líssia da Cruz e Silva e Giuseppe Tosi. São
Paulo: Ed. Vozes, 2005

\tit{ERCOLE}, F. \emph{Lo stato nel pensiero politico di Niccolo'
Machiavelli}, \emph{La politica di N. Machiavelli}, Anonima Romana
Editoriale, Roma, 1926.

\tit{FOURNEL}, Jean-Louis. \emph{Ritorno su una vecchia questione : la
traduzione della parola 'stato' nel 'Principe' di Machiavelli} in
Chroniques italiennes, janvier 2008, série Web n° 13, 1/2008;

\tit{FROSINI}, Fabio. \emph{L'aporia del `principato civil': il problema
politico del `forzare' in Principe \versal{IX}. Filosofia Politica, nao 19, n.2,
p. 199--218,} 2005.

\tit{FUBINI}, Riccardo. \emph{Italia quattrocentesca.} Milano: Franco Angeli,
1994.

\tit{GARIN}, Eugênio. \emph{Ciência e vida civil no renascimento Italiano}.
Trad. Cecília Prada. São Paulo: Ed. Unesp, 1996;

\tit{GILBERT}, Felix. \emph{Machiavel et Guichardin, Politique et histoire à
Florence au \versal{XVI} siècle}. Paris: Seuil, 1996;

\titidem. \emph{Machiavelli il suo tempo}. Bologna: Il Mulino,
1977;

\tit{GUIDI}, Andrea. \emph{Un Segretario militante.} Politica, diplomazia e
armi nel Cancelliere Machiavelli. Bologne: Il Mulino, 2009.

\tit{HALE}, John R. \emph{A Europa durante o Renascimento (1480--1520)}.
Lisboa: Editorial Presença, 1971;

\tit{INGLESE}, Giorgio. Introduzione in Machiavelli, Nicollò. \emph{De
Principatibus.} . Testo critico a cura di G. Inglese. Roma: Istituto
Storico Italiano per il Medio Evo, 1994.

\tit{LARAVAILLE}, Paul. \emph{La pensée politique de Machiavel}, \emph{Les
``Discours sur la Premiére Décade de Tite-Live}. Nancy: \versal{PUN}, 1982;

\titidem. \emph{Introduction} in Machiavel, \emph{Il Principe}
et Agostino Nifo, \emph{De regnandi Peritia}. Introduction, traduction
et notes de Paul Larivaille. Paris: Les Belles Lettres, 2008.

\titidem. \emph{Il capítul \versal{IX} del `Principe' e la crise del
`principato civile'.} In \emph{Cultura e Scritura di Machiavelli}. Roma:
Salerno Editrice, 1997 {[}p. 221--239{]}

\tit{LASLETT}, P. \emph{Introdução} in \emph{Segundo Tratado sobre o governo
civil}. São Paulo: Martins Fontes, 1998

\tit{LEFORT}, Claude. \emph{Le travail de l´ouvre Machiavel}. Paris:
Gallimard, 1972;

\tit{LEPORE}, Ettore. \emph{Il princeps ciceroniano e gli ideali politici
della tarda republica}. Napoli: Istituto Italiano per gli Studi Storici,
1954;

\tit{MARTELLI}, Mario. \emph{Saggio sul Principe.} Roma: Salerno Editrice,
1999.

\titidem. \emph{Introduzione.} In: \emph{Il Principi}. Edizione
Nazionale Delle Opere -- I/1, a cura de Mario Martelli. Roma: Salerno
Editrice, 2006.

\tit{MARTINS}, José Antônio. \emph{Os fundamentos da república e sua corrupção
nos Discursos de Maquiavel}. Tese de doutorado. São Paulo: \versal{FFLCH/USP},
2007.

\titidem. \emph{Sobre as origens do vocabulário político
medieval} in Trans/Form/Ação, vol. 34, n. 3, 2011 {[}p. 51--68{]}

\titidem. ``Sobre a tradução latina da Política de Aristóteles por Moerbeke no século \versal{XIII}'' in Oliveira, Terezinha (org). \emph{Reflexões sobre educação, cultura e política: em cena o medioevo}. Mato Grosso: \versal{UFMT}, 2013 {[}p. 47--59{]}

\titidem. \emph{Sobre o príncipe civil e a soberania em} O
Príncipe \emph{de Maquiavel} in Adverse, Helton. \emph{Reflexões sobre
Maquiavel.} 500 anos de O Príncipe. Belo Horizonte: ed. \versal{UFMG}, 2015 {[}p.
127--152{]}

\tit{NICODIMOV}, Marie Gaille. \emph{Conflit civil et liberté}. Le politique
machiavélienne entre histoire et medicine. Paris: Honoré Champion, 2004.

\tit{NICOLET}, Claude. \emph{Les idées politiques à Rome sous la République}.
Paris: Colin, 1964.

\titidem. \emph{Polybe et les institutions romanes} in
\emph{Entretiens}, tomo \versal{XX}. Geneve: Vandouvres, 1973. {[}p. 222--231{]};

\tit{OSTROGORSK}, Georg. \emph{Storia dell'impero bizantino.} Torino: Einaudi,
1968

\tit{POCOCK}, J.\,G.\,A. \emph{Il momento machiavelliano}. Bologna: Società ed.
Il Mulino, 1980;

\tit{PROCACCI}, Giuliano. \emph{Machiavelli nella cultura europea dell'eta
Moderna}. Bari: Laterza, 1995;

\titidem. \emph{Studi sulla fortuna del Machiavelli}. Roma:
Istituto Storico Italiano. 1965;

\tit{REALE}, Mario. \emph{Machiavelli, la poltica e il problema del tempo. Un
doppio cominciamente della storia romana? A proposito di Romolo in}
Discorsi \emph{I, 9}. in \emph{La Cultura, \versal{XXIII}, nº 1,} 1985 {[}p.
45--123{]};

\tit{RIDOLFI}, Roberto. \emph{Biografia de Nicolau Maquaivel}. Trad. Nelson
Canabarro, São Paulo: Musa, 2005;

\tit{RUBINSTEIN}, Nicolai. \emph{Le allegorie di Ambrogio Lorenzetti nella
Sala della Pace e il pensiero político del suo tempo} (1997) in
\emph{Studies in Italian history in teh middle ages and the
renaissance}. Roma: Edizioni di Stori e Letteratura, 2004.

\titidem. \emph{An unknown version of Machiavelli's Ritratto
delle cose della Magna} (1998) in \emph{Studies in Italian History in
the Middle Ages and Renaissance.} Vol. \versal{III}. Roma: Edizioni di Storia e
Letteratura, 2012.

\titidem. \emph{Studies in Italian History in the Middle Ages
and Renaissance.} Vol. \versal{II}. Politics, Diplomacy and the Constitution in
Florence and Italy. Roma: Edizioni di Storia e Letteratura, 2011.

\titidem. ``The history of the word \emph{politicus} in early
modern Europe'' (1987) in \emph{Studies in Italian history in teh middle
ages and the renaissance}. Roma: Edizioni di Stori e Letteratura, 2004.

\titidem. \emph{Il governo di Firenze sotto i Médici}.
Firenze: Nuova Itália, 1999;

\tit{SASSO}, Gennaro. ``Intorno allá composizione dei Discorsi di Niccolò
Machiavelli'', \emph{Giornale Storico della Letteratura Italiana},
\versal{CXXXIV} (1957), {[}p. 482ss{]} e \versal{CXXXV} (1958), {[}p. 215ss{]};

\titidem. ``Note machiavelliane, \versal{I} (Príncipe, \versal{IX})'', in \emph{La
Cultura,} Vol. \versal{XII}, 1972 {[}p. 123--142{]};

\titidem. \emph{Machiavelli e gli antichi} \emph{e altri saggi}.
Tomo \versal{I} e \versal{II} Milano: Riccardo Ricciardi editore, 1987;

\titidem. \emph{Niccolo Machiavelli}, \emph{storia del suo
pensiero politico}. Bologna: Il Mulino, 1980;

\titidem. \emph{Principato civile e tirannide} e
\emph{Paralipomeni al ``principato civile''} in \emph{Machiavelli e gli
Antichi e altre saggi.} Napoli: Riccardo Ricciardi Editore, 1988;

\titidem. \emph{Studi su Machiavelli}. Napoli: Morano, 1967;

\tit{SENNELART}, Michel. \emph{As artes de governar.} Trad. Paulo Neves. São
Paulo: Ed. 34 Letras, 2006.

\tit{SKINNER}, Quentin. \emph{As fundações do Pensamento Político Moderno}.
Trad. Renato Janine Ribeiro. São Paulo: Cia das Letras, 2000;

\titidem. \emph{Virtù rinascimentali}. Bologna: Il Mulino, 2006

\tit{TAFURO}, Antonio. \emph{La formazione di Niccolò Machiavelli.} Napoli: Dante \&
Descates, 2003, {[}parte I, 1.2 e 1.3{]}

\titidem. \emph{Il reggimento di Firenze secondo Francesco
Guicciardini}. Napoli: Libreria Dante \& Descartes, 2005;

\titidem. \emph{La formazione di Niccolò Machiavelli. Ambiente
fiorentino, esperienza política, vicenda umana}. Napoli: Libreria Dante
\& Descartes, 2004;

\tit{TENENTI}, Alberto. ```Civilità' e civiltà in Machiavelli'' in
\emph{Credenze, ideologie, libertinismi tra Medioevo ed Età moderna}.
Bologna: Il Mulino, 1978 {[}p. 155--173{]};

\tit{VERNANT}, Jean-Pierre. \emph{Mito e Pensamento entre gregos.} Rio de
Janeiro: Paz e Terra, 1995;

\tit{VIROLI}, Maurízio. \emph{O sorriso de Nicolau}. Trad. Valéria Pereira da
Silva. São Paulo: Estação Liberdade, 2002.
\end{bibliohedra}