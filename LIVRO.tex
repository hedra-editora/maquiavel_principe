\documentclass[semcabeco,showtrims,trimframe,12pt,conselho,spreadimages]{memoir}

\usepackage[largepost]{hedraoptions} %% << %%%%%%%%%%%%%%%%
\usepackage[baruch]{hedrastyles}
\usepackage[xetex,chicagofootnotes]{tipografia}
\usepackage[standart,sempontinhos]{toc}
\usepackage{hedraextra}
\usepackage{penalidades}
\usepackage{graficos}
\usepackage{hedralogo}	
\usepackage{hifensextras}
\usepackage{fichatecnica}
\usepackage[standart]{aparatos}
\usepackage{tabelas}
\usepackage{versos}
\usepackage{gitrevisioninfo}
\usepackage{parallel}

\newcommand{\forceindent}{\leavevmode{\parindent=1,4em\indent}}

\linespread{1.15}

\newcommand{\quebra}{\vfil\pagebreak}



%\newcommand{\paragraphbr}[1]{\vfill\pagebreak\paragraph{#1}}
%\renewcommand\theparagraph{\roman{paragraph}}
%\newcommand{\quebra}{\vfil\pagebreak}
%\newcommand{\est}{\vspace{10cm}}
%\newenvironment{mesma}%
   %{\par\samepage}%
   %{\par\pagebreak[0]}

\usepackage{multirow}
\usepackage{graphicx}


%\usepackage{endnotes}
%\renewcommand{\notesname}{Notas}

%\lhead[\fancyplain{}]{}
%\chead[\fancyplain{}]{}
%\rhead[\fancyplain{}]{\cnvt{\thepage} -- \thepage}

%\newcommand*{\cnvt}[1]{\the\numexpr#1-1\relax}

%\fancypagestyle{chapter}{
%\pagestyle{fancy}
%\setlength{\headheight}{5mm}
%\fancyhf{}
%\fancyhead[R]{\thepage}
%\renewcommand{\headrulewidth}{0pt}}


%\usepackage{footmisc}

%\renewcommand*\footnoterule{}
%\fancyhf[RO]{\cnvt{\thepage} -- \thepage}
%\fancyfoot{}
%\renewcommand{\headrulewidth}{0pt}
%\renewcommand{\footrulewidth}{0pt}}

\usepackage{fontspec}

%\usepackage{Formular}
\newfontfamily\Formular{Formular-Regular}[
BoldFont = Formular-Bold.otf]	

%--------------------------------------------ALTERAR DISTÃNCIA ENTRE TÍTULO DO SUMÁRIO E CAPÍTULOS
%\addtocontents{toc}{\vskip-15pt}
%--------------------------------------------
\usepackage{afterpage}

\newcommand\blankpage{%
    \null
    \thispagestyle{empty}%
    \addtocounter{page}{0}%
    \newpage}

%\usepackage{imakeidx} 
%\makeindex[program=xindy, options=-C utf8 -L portuguese]
%\newcommand\gobbleone[1]{}
%\newcommand*{\seeonly}[2]{\ (\emph{\seename} #1)}
%\newcommand*{\also}[2]{\emph{cf.} #1}
%\newcommand{\Also}[2]{\emph{See also} #1}
%\renewcommand\indexname{Índice onomástico}
%\makeindex[intoc]

\setcounter{tocdepth}{0}
\setcounter{secnumdepth}{-2} 
%\linespread{1.08}
\usepackage{commands}

\usepackage{setspace}

\makeatletter
\newenvironment{Parskip}{%
   \par
   \parskip=0.3\baselineskip \advance\parskip by 0pt plus 2pt
   \parindent=\z@
   \def\@listI{\leftmargin\leftmargini
      \topsep\z@ \parsep\parskip \itemsep\z@}
   \let\@listi\@listI
   \@listi
   \def\@listii{\leftmargin\leftmarginii
      \labelwidth\leftmarginii\advance\labelwidth-\labelsep
      \topsep\z@ \parsep\parskip \itemsep\z@}
   \def\@listiii{\leftmargin\leftmarginiii
       \labelwidth\leftmarginiii\advance\labelwidth-\labelsep
       \topsep\z@ \parsep\parskip \itemsep\z@}
   \partopsep=\z@
}{\par}
\makeatother

\makeatletter
\newenvironment{myParskip}{%
   \par
   \parskip=0.2\baselineskip \advance\parskip by 0pt plus 2pt
   \parindent=\z@
   \def\@listI{\leftmargin\leftmargini
      \topsep\z@ \parsep\parskip \itemsep\z@}
   \let\@listi\@listI
   \@listi
   \def\@listii{\leftmargin\leftmarginii
      \labelwidth\leftmarginii\advance\labelwidth-\labelsep
      \topsep\z@ \parsep\parskip \itemsep\z@}
   \def\@listiii{\leftmargin\leftmarginiii
       \labelwidth\leftmarginiii\advance\labelwidth-\labelsep
       \topsep\z@ \parsep\parskip \itemsep\z@}
   \partopsep=\z@
}{\par}
\makeatother

\newcommand{\mystar}{{\fontfamily{lmr}\selectfont$\star$}}

%\makeatletter
%\renewcommand{\@chapapp}{}% Not necessary...
%\newenvironment{chapquote}[2][2em]
%  {\setlength{\@tempdima}{#1}%
%   \def\chapquote@author{#2}%
%   \parshape 1 \@tempdima \dimexpr\textwidth-2\@tempdima\relax%
%   \itshape}
%  {\par\scriptsize\hfill-- \chapquote@author\hspace*{\@tempdima}\par\bigskip}
%\makeatother

%\newcommand\Chapter[2]{\chapter
%  [#1\hfil\hbox{}\protect\linebreak{\itshape#1}]%
%  {#1\\[2ex]\Large\itshape#2}%
%}



\begin{document}
%%!TEX root=./LIVRO.tex
\chapter*{Introdução}
\addcontentsline{toc}{chapter}{Introdução, por José Martins}

\section{Considerações iniciais}

Dada a quantidade (mais de três dezenas) de traduções e edições de
\emph{O Príncipe} de Nicolau Maquiavel somente no Brasil, sem contar com
as centenas de outras edições nas diversas línguas modernas, essa nova
edição publicada pela Hedra busca atender algumas especificidades e não
repetir aquilo que já foi divulgado à exaustão.

Quanto à edição, trata-se de propiciar ao leitor brasileiro um exemplar
bilíngue, com a melhor edição do texto original italiano -- a
\emph{Edição Crítica Inglese} -- acrescida de introdução e notas
explicativas. A exigência e a novidade de que haja o texto italiano
original, algo presente desde a primeira publicação em 2007, atende a
uma demanda por qualidade nos textos acadêmicos, a saber: permitir que o
leitor possa aprofundar (ou mesmo compreender melhor passagens da obra
que não tenham ficado claras na tradução) por meio do cotejamento com o
original.

A introdução aqui exposta visa a dois objetivos: a) como de praxe nas
boas edições, apresentar ao leitor o autor e os principais aspectos da
obra; b) para além desse aspecto básico, apresentar também um viés
interpretativo que vem se consolidando entre os comentadores, ou seja: a
inclusão de \emph{O Príncipe} no interior da reflexão republicana de
Maquiavel, com destaque para as noções de principado e príncipe. Dito
isso, convém lembrar que essa introdução não pretende ser nem uma
reapresentação da biografia do filósofo, nem uma introdução ao
pensamento político maquiaveliano como um todo, nem uma análise
exaustiva e minuciosa dos diversos aspectos teóricos presentes na obra.
Insistindo e explicitando melhor, apesar de mobilizar informações sobre
a vida de Maquiavel e alguns aspectos de seu pensamento político, essa
introdução busca pensar \emph{O Príncipe} como uma obra peculiar e
fundamental para a compreensão das noções políticas republicanas de
Maquiavel.

Por fim, inserimos algumas notas que visam apresentar personagens, datas
e eventos do contexto histórico no qual o livro está inserido e explicar
conceitos e passagens centrais do texto. Tendo em vista a presença e
natureza da introdução, o uso desse segundo tipo de notas foi restrito
ao mínimo essencial e indispensável.

\section{O contexto histórico maquiaveliano e~os~seus~pressupostos~teóricos}

\subsection{O contexto histórico: as origens e o ingresso~na~Chancelaria~florentina}

Nicolau Maquiavel ou Nicolló Machiavelli nasceu em Florença, Itália, no
dia 3 de maio de 1469, em uma família de pequenas posses. Sabe-se que
seu pai -- Bernardo Machiavelli --, em função de ser escriturário, teve
alguma formação jurídica, mas não ocupou um cargo destacado e nem
exerceu importantes cargos na administração da cidade, tampouco acumulou
grandes posses. Todavia, em função dessa formação jurídica do patriarca,
a casa dos Machiavelli dispunha de obras clássicas de história, de
doutrina política, de jurisprudência, enfim, uma pequena biblioteca nos
moldes de uma sociedade humanista\footnote{É sempre difícil definir com
  poucas palavras grandes manifestações históricas, como foi o caso do
  Humanismo, que teve como seu lugar de nascimento e desenvolvimento a
  Itália dos séculos \versal{XIII} ao século \versal{XVI}. Isso que se entende por
  Humanismo caracterizou-se por uma retomada dos valores culturais da
  Antiguidade e por uma maior valorização do homem. Por isso, busca-se
  ler os autores clássicos nos seus textos originais, seja em grego,
  seja em latim; há uma intensa busca pela redescoberta desses textos,
  quase sempre esquecidos em alguma biblioteca, produzindo-se novas
  edições; uma maior valorização daquilo que diz respeito ao homem e às
  coisas humanas, numa tentativa de contrabalancear a pouca valorização
  desses aspectos pela cultura cristã-medieval. Concomitantemente, há
  uma maior valorização das virtudes ligadas à vida na cidade,
  principalmente as qualidades relacionadas à política, por oposição aos
  valores religiosos cristãos, característicos de grande parte do
  período medieval. Esses aspectos, entre outros, sinalizam a mudança
  que se operará nas sociedades europeias a partir de meados do século
  \versal{XIII} e que será predominante já no século \versal{XV}, preparando o terreno
  para aquilo que se conhecerá como a Era Moderna. Para maiores
  informações cf: \versal{HALE}, 1971; \versal{SKINNER}, 2000, cap. 1-3; \versal{BARON}, 1989.} do
\emph{Quattrocento}.

Sobre a formação de Nicolau Maquiavel, tendo em conta esse ambiente
humanista e a acessibilidade às obras clássicas, sabe-se que ele teve
aulas com professores particulares, não havendo registro sobre uma
possível frequência em escola ou universidade. Segundo Guidi, Maquiavel
teve aulas com os mestres de latim Paolo Sassi, Michele Verino e Pietro
Crinito (\versal{GUIDI}, 2009). A partir de tal formação, deduz-se que Maquiavel
tenha ultrapassado o que seria um nível intermediário, tendo em vista as
exigências acerca do conhecimento da língua latina para o cargo que
ocupou na Chancelaria, sendo que, posteriormente, revelou um grande
conhecimento de autores como Cícero, Lucrécio, Tito Lívio, etc.
Considerando toda a sua produção teórica e literária, comprova-se que
ele conhecia muito bem a sua língua, o toscano, e o latim, assim como
possuía uma vasta compreensão de história e noções de filosofia. Além
dessa formação em língua clássica, ele foi introduzido ao cálculo
matemático, caracterizado pelo ensino do ábaco, e teve alguma formação
em jurisprudência, embora não tenha se tornado bacharel em direito.
Pode-se afirmar com segurança que Maquiavel não aprendeu o grego. É
nesse ambiente cultural e familiar que sua vida se desenvolve até o ano
de 1498, quando começa a trabalhar na Chancelaria florentina, onde
permanecerá pelos próximos catorze anos.

Esse período que Maquiavel passou na Chancelaria foi posto em segundo
plano pelos especialistas por muito tempo. Contudo, mais recentemente,
esse aspecto biográfico passou a ser alvo das atenções, principalmente
em função do verdadeiro alcance dessa experiência diplomática sobre sua
reflexão política. Segundo Guidi, o ingresso na Chancelaria Florentina
ao final do século \versal{XV} era feito por indicação do chefe político -- o
\emph{gonfaloniere} --, indicação essa que deveria ser submetida à
aprovação dos Conselhos políticos superiores (\versal{GUIDI}, 2009). Após a queda
da família Medici, em dezembro de 1494, o novo governo do frei Jerônimo
Savonarola busca reestruturar o regime florentino, criando mecanismos
republicanos de caráter mais popular, particularmente com a reabertura e
o fortalecimento do Conselho Maior (\emph{Consiglio Maggiore}), que era
a instância política na qual todos os cidadãos podiam tomar parte e não
era controlada exclusivamente pela oligarquia florentina. Outra medida
do governo de Savonarola foi reestruturar toda a Chancelaria, trocando
os funcionários e reordenando as incumbências.

Segundo Tafuro (2004, p. 25ss), o mundo político da cidade de Florença
no final do século \versal{XV} era caracterizado por divisões de caráter
econômico e social que estavam na raiz dos grupos políticos, a saber: os
grandes aristocratas, os médios comerciantes, os pequenos comerciantes e
os demais trabalhadores -- esses, em geral, quase sem expressão
política. Maquiavel nomeará esses grupos ao longo de sua \emph{História
de Florença}, respectivamente, como nobres, povo e plebe. A aristocracia
florentina (os nobres, segundo Maquiavel) estava dividida em dois
grandes partidos, desde as lutas políticas do século \versal{XIV} entre os
partidários do papa -- o partido \emph{guelfo} -- e os partidários do
imperador\footnote{Os territórios do centro-norte da península itálica
  foram motivos de disputas entre os séculos \versal{XI} a \versal{XIV} pelo papado e pelo
  Sacro Império Romano-Germânico. Como mostra Skinner (2000, cap. 1),
  nesse período, muitas cidades travaram lutas externas contra esses
  dois domínios em busca de sua independência e também lutas internas
  contra a dominação de um desses grupos ou contra o domínio de tiranos
  locais. A tese aceita pelos intérpretes é que, destas lutas, nasce o
  germe do republicanismo típico do Renascimento italiano, no qual as
  cidades buscaram se organizar como repúblicas independentes e
  autônomas.} -- o partido \emph{ghibelino}. A vitória do partido
\emph{guelfo} em 1289 na sua luta contra os \emph{ghibelinos} instalou
uma estreita relação entre essa parcela da aristocracia e o papado, que
perdurará durante séculos e que terá como fruto a forte influência dos
florentinos nas decisões de Roma, bem como a nomeação de alguns cardeais
florentinos ao trono papal: Leão \versal{X} (Giovanni di Lorenzo de Medici --
papa de 1513 a 1521), Clemente \versal{VII} (Giulio di Guiliano de Medici -- papa
de 1523 a 1534), Clemente \versal{VIII} (Ippolito Aldobrandini, papa de 1592 a
1605), Leão \versal{VI} (Alessandro Ottaviano de Medici, papa por 26 dias, de 1 a
27 de abril de 1605), Urbano \versal{VIII} (Maffeo Barberini, papa de 1623 a
1644) e Clemente \versal{XII} (Lorenzo Corsini, papa de 1730 a 1740).

Apesar dessa importante força política desempenhada por parte da
aristocracia do partido \emph{guelfo}, uma outra parcela também rica, a
aristocracia \emph{ghibelina}, será muito influente e marcante na
cidade, por exemplo, na constante oposição que a família Medici recebeu
enquanto esteve no poder.

Abaixo desses dois grupos políticos aquinhoados, Maquiavel designa as
demais parcelas como ``povo'', embora não se assemelhe muito à nossa
compreensão contemporânea do termo. Essa parcela social e política teve
sua origem nas corporações de ofícios ou \emph{artes}, que eram
associações dos diversos grupos profissionais da cidade que controlavam
o exercício da respectiva atividade profissional. Segundo Tafuro (2004,
p. 27-28), um desses grupos políticos mais abastados, não
necessariamente aristocratas em sua denominação, configuram uma parcela
política dita popular, que ganha riqueza e prestígio e passa a figurar
como um grupo político de destaque. Nesse caso, formam, já no século \versal{XV},
um agrupamento de caráter aristocrático do ponto de vista econômico e
intelectual, mas não tradicional, por não ter ligação com as antigas
aristocracias rurais e urbanas.

Ainda no interior das \emph{Arti} ou povo, segundo a designação
maquiaveliana, verifica-se uma divisão interna entre aquilo que ele
nomeará como \emph{popolo grasso} (povo gordo) e \emph{popolo minuto}
(povo magro), ou seja, entre os setores médios mais aquinhoados e
aqueles com menos poder econômico e, portanto, menor influência
política.

Verifica-se, pois, que o povo (\emph{popolo}), para Maquiavel, é uma
denominação para diferenciar parcelas sociais com riqueza de outras
também com riqueza mediana, mas de origem tradicional. Logo, são
tipificações no interior de grupos caracteristicamente aristocráticos.
Na definição de Tafuro: ``Povo era, em particular, a parcela média dos
artesãos e dos comerciantes reunidos nas associações profissionais''
(2004, p. 40). Ora, quando Maquiavel chama a atenção para a divisão
entre os nobres e o povo, trata-se de uma divisão entre os setores
aristocráticos tradicionais e a classe social que emergiu econômica e
socialmente com o exercício do comércio e do artesanato a partir do
século \versal{XIII}.

Por fim, temos a plebe, que eram os operários ou os assalariados, que
não possuíam qualquer associação ou grupo político no qual pudesse
expressar seus interesses, até a Revolta dos Ciompi, de 20 de julho de
1378, quando eles passam a ter direitos políticos e associativos
(\versal{TAFURO}, 2004, p. 33-34). Sintomática é a importância que Maquiavel
confere a essa revolta política de final do século \versal{XIV}, que marca uma
mudança paradigmática na vida política florentina, conforme ele narra
com cores dramáticas nas \emph{Histórias Florentinas,} livro \versal{III}. A seu
ver, essa revolta política ocupa um papel de destaque na história da
dinâmica política da cidade, visto que implicou não somente uma mudança
institucional, mas, principalmente a partir da visada teórica
maquiaveliana, na dinâmica política e nas correlações de força entre os
grupos políticos.

Nunca é demais lembrar que, apesar da constatação da existência desses
diversos grupos políticos, por outro lado, inúmeras pessoas estavam
alijadas das instâncias decisórias, como os trabalhadores rurais, as
mulheres e os não nascidos em Florença, que, conforme nos relata Gilbert
(1996, p. 25), de uma população estimada em 62.000 em 1494, menos de
15.000 homens possuíam cidadania e direitos políticos.

Nota-se, então, que Florença entre os séculos \versal{XIV} e \versal{XVI} estava dividida
em vários grupos sociais que, por consequência, dividiam-se em grupos
políticos em disputa pelo comando do governo. Esse quadro de divisão e
disputa política entre vários partidos é ressaltada por Maquiavel nos
seus diversos escritos e é uma das marcas de seu pensamento político,
haja vista que é sob essa condição de instabilidade e disputa que se
monta o palco da ação política.

Tal percepção da importância de lidar com os interesses desses diversos
grupos no mundo político Maquiavel experimenta desde o momento de seu
ingresso na Chancelaria e o atingirá ao longo de toda a sua vida
doravante. Com a queda do governo da família Medici no final de 1494 e a
instauração do governo de Savonarola, a administração central passa a
ser controlada pelos partidários deste. Nesse período, sabe-se que
Maquiavel apenas realizou trabalhos avulsos, com os quais travou já
algum contato com o ambiente da Chancelaria, mas que não chegou a
ingressar nela, pois a admissão se fazia por meio de uma indicação do
\emph{gonfaloniere} e era, posteriormente, submetida à votação no
conselho ao qual o cargo se vinculava. Ora, Maquiavel não somente não
teria a indicação de Savonarola como os partidários desse nos diversos
Conselhos da cidade não aceitariam o seu nome, como ocorreu
primeiramente em 18 de fevereiro de 1498, quando seu nome foi recusado
pela primeira vez. Apenas com a queda do governo de Savonarola é que o
nome de Maquiavel é indicado para o cargo de segundo secretário e
admitido em 23 de maio de 1498.

É importante destacar que a Chancelaria não era apenas uma repartição da
burocracia voltada exclusivamente para as relações diplomáticas, mas o
órgão central da administração da república florentina, encarregada das
relações exteriores, da guerra, da arrecadação de tributos e de
assessoramento das instâncias judiciárias. Portanto, ingressar na
Chancelaria não implicava tão somente fazer parte do órgão responsável
pelas relações diplomáticas de Florença, mas trabalhar no coração
político e burocrático da cidade (\versal{GUIDI}, 2009, cap. 1).

O fato em si da indicação para um cargo elevado revela que Maquiavel já
era conhecido do grupo político que assume o poder em 1498, bem como que
ele já conhecia a rotina burocrática. De fato, entre 1494 e 1498,
Maquiavel foi contratado para fazer pequenos serviços para a
Chancelaria, o que foi lhe dando conhecimento sobre as rotinas
administrativas. Então, sua escolha em 15 de junho de 1498 para a função
de secretário da Segunda Chancelaria é fruto de sua formação humanista,
seu conhecimento, ainda que parcial, das rotinas da administração e,
fato apontado como principal pelos estudiosos, sua ligação política com
o grupo que assume o poder em 1498, que eram setores do \emph{povo}, no
caso, os adversários dos Medici e de Savonarola; ou seja, quem apoiou a
indicação de Maquiavel foram em sua maioria os partidários das
\emph{Arti}, especificamente os partidários das \emph{Arti Maggiori}.
Parte das tarefas deste cargo ocupado por Maquiavel implicava assessorar
os ``Dez da Bailia'', que era um órgão encarregado da política externa
e, eventualmente, da guerra e das questões militares (\versal{TAFURO}, 2004, p.
121-122).

A questão que ainda incomoda os estudiosos da obra de Maquiavel diz
respeito a sua atuação política no período em que esteve trabalhando na
administração da república florentina. Por um lado, com base em seus
escritos, Maquiavel apresenta a imagem de um modelo de funcionário
público devotado à causa da cidade e não aos interesses dos grupos no
comando do governo, uma espécie de funcionário de carreira sem ligação
política ou ideológica que estivesse a serviço da cidade\emph{.} Por
outro lado, até mesmo pela sua própria história, Maquiavel foi sempre
identificado a um grupo político, ora como partidário dos republicanos
na visão da oligarquia florentina, ora como partidário dos Medicis,
segundo o grupo republicano por ocasião da restauração da república em
1527. A questão está em saber até que ponto Maquiavel foi de fato um
funcionário, por assim dizer, isento de ligações políticas ou se ele foi
um mentor e articulador do governo de Pier Soderini, ou, até mesmo, se
seria possível conciliar as duas coisas. As pesquisas recentes sobre o
período em que Maquiavel esteve na Chancelaria mostram, por vários
elementos, que ele não somente foi um membro ativo do grupo que esteve
no poder em Florença entre 1498 e 1512, como ocupou um papel destacado
no governo. Segundo Guidi, o ingresso de Maquiavel já em um cargo
elevado na burocracia, a sua participação ativa na reformulação do
governo em 1502, os encargos diplomáticos e militares que ocupou depois
até a queda do governo em 1512 (a demissão oficial de Maquiavel foi em
10 de novembro de 1512), revelam que ele ingressou na Chancelaria também
por seus vínculos políticos e foi, ao longo dos anos, galgando cada vez
mais espaço e importância no governo de Soderini, ao ponto de ser o
maior responsável pela parte militar do governo e pela condução das
guerras após 1506 (\versal{GUIDI}, 2009).

A vinculação aos interesses do governo de Soderini e as suas qualidades
de negociador e analista do contexto político ficarão associadas a
Maquiavel para sempre, pois, mesmo depois de sua saída do governo,
quando foi requisitado, mesmo entre os seus adversários, sempre se nota
esse duplo aspecto: um partidário da causa republicana e um hábil
analista político e militar. Aspecto esse reforçado pelo modo como ele
descreve sem paixão o universo da política em geral e o seu mundo
florentino em particular, no qual os fatores políticos que implicam de
fato as decisões são expostos friamente, sem meandros ou meias palavras.
Enfim, podemos inferir que Maquiavel foi sim um membro leal do grupo de
Solderini, mas isso não o impediu de ser um funcionário a serviço da
cidade que avaliava sem paixão o mundo político, como demonstram a
exaustão suas correspondências diplomáticas e os seus textos políticos.

Embora tenha iniciado no serviço público com 29 anos, uma idade
considerável para a época, e com uma relativa formação humanista,
Maquiavel declara em vários textos que foram esses anos de convivência
cotidiana com o mundo da política que lhe deram uma boa parte de sua
formação política -- a outra parte ele diz ter recebido dos clássicos.
Em seu dia a dia, ele acompanhava, por um lado, o funcionamento das
decisões políticas em Florença, ou seja, observava por dentro os
mecanismos das negociações políticas, como elas eram feitas, os jogos de
intenções e de promessas, e, por outro lado, como se realizavam as
negociações diplomáticas, como os reis, papas, príncipes, comandantes
militares, governantes das republicas negociavam e estabeleciam pactos,
guerras, ou resolviam os conflitos comerciais. Esse conhecimento da vida
política por dentro, aliado a um olhar treinado a não enxergar somente
os aspectos circunstanciais e pessoais do mundo político, mas a
desvendar as reais motivações dos atores e os fatores que regulam o agir
político, deram a Maquiavel uma parte do conhecimento necessário para
entender o universo da política geral, de modo ampliado.

Convém informar que, desde os séculos \versal{XII} e \versal{XIII}, ocorre uma mudança
radical na concepção e organização da vida diplomática e das
chancelarias nas repúblicas italianas (\versal{FUBINI}, 1994). Seja em função das
disputas que deveriam travar contra as grandes potências (não somente o
papado e o Império Germânico, mas também, nos séculos seguintes, contra
os franceses e espanhóis, sem contar a sempre constante ameaça turca que
se amplia após a queda de Constantinopla em 1453), seja em função de seu
pouco poder militar, essas repúblicas reconheceram que os seus quadros
diplomáticos seriam uma grande arma para a defesa de sua condição de
repúblicas livres. A mudança se nota inicialmente na própria missão
delas: não mais defender tão somente os interesses privados do poderoso
no governo, o que implicaria uma prática voltada para os interesses
gerais e não mais como prepostos mercantis. As chancelarias das
repúblicas italianas do Renascimento se diferenciam das medievais, na
medida em que estão mais voltadas a trabalhar pela defesa dos interesses
gerais da cidade, principalmente a defesa da condição de liberdade
política destas repúblicas, e não somente em fazer acordos vantajosos
para as famílias poderosas (\versal{GUIDI}, 2009, p. 35).

Outro aspecto das práticas dessas chancelarias do Renascimento é sua
preocupação em ter um maior controle da vida política e jurídica da
cidade, tornando-se verdadeiras burocracias de controle das práticas
públicas. Neste sentido, adota-se o costume de registrar todas as
reuniões dos conselhos em atas, de catalogar as missões, de documentar
os discursos e exigir de seus emissários a notificação do andamento das
ações em suas missões. Essa massa de material permite ao governo ter
maior controle das decisões e perceber melhor as rotinas e os
encaminhamentos das diversas ações. Segundo Guidi, os chanceleres e os
secretários ``configuram-se como técnicos da administração a serviço do
executivo'' (\versal{GUIDI}, 2009, p. 39-40).

Por isso que fazer parte da Chancelaria florentina não significava tão
somente fazer parte do órgão responsável pelas relações internacionais,
o que por si mesmo já implicava um extraordinário conhecimento do mundo
das negociações políticas. Tendo em vista essa dimensão de controle
burocrático das ações de governo, seus funcionários tinham acesso
privilegiado, por meio da documentação, aos diversos elementos em jogo
na política interna e externa da cidade. Essa documentação torna público
e acessível aquilo que antes eram informações restritas e privilegiadas
para se compreender os reais motivos das tomadas de decisão política.
Portanto, quando Maquiavel diz que aprendeu muito com a vida prática na
Chancelaria, isso não significa que ele esteve presente em todas as
decisões e conhecia muito bem os diversos interesses em jogo (visto que
seria impossível ele estar, por exemplo, fora da cidade e saber dos
detalhes de uma negociação ou jogada política), mas que, por meio dessa
documentação pública, ele pode conhecer o mundo da política. Essa
documentação oficial, também conhecida como \emph{pratiche} (\versal{GILBERT},
1964), relatam as práticas, os diversos procedimentos oficiais, e são,
para um leitor atento, uma fonte preciosa de informações sobre a
dinâmica da vida política. Convém insistir: quando Maquiavel declara,
por várias vezes, que boa parte de sua formação política adveio das
experiências das coisas modernas, da sua convivência com o mundo da
política, inclua-se nesse rol o conhecimento dessa farta e rica
documentação oficial, à qual ele teve acesso por sua condição de Segundo
secretário\footnote{A hierarquia nas chancelarias eram: chanceler,
  secretários e cartorários ou oficiais de chancelaria.}, sendo, pois, o
responsável por produzir e organizar esses documentos\footnote{Não é sem
  razão que as pesquisas sobre o pensamento político de Maquiavel estão
  se voltando cada vez mais para essa documentação produzida no período
  republicano, mais especificamente para os documentos oficiais do seu
  período na Chancelaria, de 1498 a 1512.}. Enfim, as \emph{pratiche}
destacadas por Maquiavel não foram somente as ações políticas, as
práticas políticas, mas também e fundamentalmente a documentação oficial
da República florentina, algo que, ainda que não fosse inédito para
muitos pensadores, foi considerada e analisada de modo especial pelo
Secretário Florentino, configurando-se, enfim, como uma novidade no
mundo do Renascimento.

A outra parte do conhecimento, conforme ele mesmo declara, adquiriu na
leitura dos clássicos. Para compreender melhor esse aspecto, não é
suficiente saber que Maquiavel teve uma formação humanista, mas também
considerar o contexto cultural no qual ele esteve inserido. Conforme
Garin, a Chancelaria de Florença tinha se transformado, desde o final do
século \versal{XIV}, não somente no centro político da cidade, mas no seu centro
intelectual, em função das figuras de destacada importância que
estiveram a serviço da cidade como Coluccio Salutati, Poggio
Bracciollini, Leornardo Bruni, Bartolomeu della Scalla, Lorenzo Valla,
Francesco Guicciardini, fazendo dela uma verdadeira escola do pensamento
político (\versal{GARIN}, 1996). Esses intelectuais marcaram o pensamento
político do Renascimento, entre outros fatores, mas principalmente, por
terem que fazer uma defesa contundente do regime republicano florentino
enquanto ocupavam o cargo de chanceleres. Desde o final do século \versal{XIII}
até o início do século \versal{XVI}, Florença frequentemente se via envolvida em
ameaças de dominação por algum poder exterior ou em disputa contra
tiranos locais. Nessas ocasiões, sempre era necessário mobilizar os
cidadãos para a defesa da liberdade política, que implicava, em última
instância, a defesa do regime republicano. Portanto, diante da ameaça de
dominação, o chanceler do momento comandava uma luta no campo ideológico
e teórico em defesa da liberdade republicana. Ora, esse acúmulo de
reflexão política foi se consolidando ao longo do tempo entre os
ocupantes de cargo, a ponto de se constituir, como interpreta Garin,
numa ``escola'' de pensamento político republicano. Então, para os
postulantes a cargos na Chancelaria, já estava claro, de antemão, que
eles estavam se vinculando a uma instituição que teve e ainda tinha por
missão a defesa da liberdade política e da autonomia da cidade,
encarnada no seu regime republicano. Isso se reflete decisivamente nos
escritos de Maquiavel, nos quais a problemática da liberdade política
perpassa sua reflexão política como um todo (\versal{BIGNOTTO}, 1991).

Associa-se a isso o intenso debate político sobre a melhor forma de
ordenar a república florentina no interior da crise política de final do
século \versal{XV} gerado pela queda dos Medici, pelo curto e conturbado governo
de Savonarola e início do novo governo de Soderini (\versal{BARON}, 1989;
\versal{GILBERT}, 1996; \versal{RUBINSTEIN}, 1998). Ora, num período de mais de 20 anos
(entre 1492 e 1513), Florença passa por quatro governos: o de Piero de
Medici (1492-1494), o de Jerônimo Savonarola (1494-1498), de Pier
Soderini (1498-1512) e doravante novamente com a família Medici até
1527. Essas mudanças na vida política da cidade são acompanhadas de
intensos embates intelectuais, nos quais se podem reconhecer vários
grupos em defesa de seus projetos, mas que podem ser agrupados, em
linhas gerais, em dois grandes partidos, a saber: os defensores de um
modelos republicano de caráter mais aristocrático e os defensores de um
regime republicano de caráter mais popular, com a ampliação da
participação dos diversos segmentos políticos da cidade (\versal{MARTINS}, 2010).

Sobre esse tema, convém recuperar a disputa entre os modelos
republicanos nesse momento histórico na cidade de Florença.

\subsubsection{O contexto maquiaveliano: entre o republicanismo popular~e~o~republicanismo~aristocrático}

Os séculos \versal{XV} e \versal{XVI} foram marcados por intensos debates nas cidades
italianas sobre a melhor forma de regime republicano a ser adotada.
Neste período, vários pensadores se preocuparam em definir modelos de
repúblicas que atendessem às demandas de sua época, consagrando duas
formulações: um modelo de república de caráter mais popular,
caracterizado pelos governos de Savonarola e Soderini em Florença, e um
modelo de república de caráter aristocrático, tendo o regime republicano
da cidade de Veneza como o grande exemplo. É em Florença que esse debate
ganha mais vigor e corpo, fazendo dela um dos centros de produção
intelectual sobre o regime republicano (\versal{GARIN}, 1996). As mudanças
políticas florentinas do final do século \versal{XV} e início do século \versal{XVI}
suscitaram intensos debates sobre os destinos da cidade, num primeiro
momento, e sobre a natureza das repúblicas, num segundo momento. Nesse
contexto de alteração constitucional, a discussão dos fundamentos da
república florentina adquire força nos círculos intelectuais. Dentre as
várias posições assumidas, a defesa da instalação de um regime
republicano inspirado no modelo veneziano foi predominante entre a
aristocracia florentina, grupo político que identificava nos governos
republicanos de Savonarola e Soderini o predomínio dos segmentos
populares, governos esses considerados como ``demasiadamente''
democráticos\footnote{Certamente, caracterizar os governos republicanos
  de Florença que vão de 1494 a 1512 como democráticos é algo
  problemático devido ao poder que a aristocracia deteve nesse período.
  Qualquer afirmação mais contundente no sentido da definição do tipo de
  governo existente em Florença durante esses 18 anos é passível de
  discussão. Sobre a história do período cf. Tenenti (1973) e Tafuro
  (2004, parte \versal{I}).}. Entre os exemplos favoritos da aristocracia para
justificar sua opção política estavam a Roma republicana, a Esparta
concebida por Licurgo e a república veneziana de então. Entretanto, como
afirma Gilbert, o exemplo veneziano era o que mais se destacava: os
aristocratas em particular, ansiosos em limitar o poder do
\emph{Conselho Maior} (no qual todos os cidadãos tinham direito de
participar), colocavam em evidência que, em Veneza, os cidadão discretos
e sábios tinham as possibilidade melhores e mais apropriadas para o
exercício do poder (\versal{GILBERT}, 1977, p. 102-103). A opção pelo modelo
veneziano se deve, principalmente, ao predomínio e controle que a
aristocracia mercantil exercia sobre o governo. Em Florença, o governo
de Savonarola, bem como em certa medida o de Soderini, eram, aos olhos
da aristocracia, muito democráticos, pois neles os poderes decisórios de
seu extrato político estavam limitados pelas forças políticas populares.

É, portanto, no interior dessa luta pela retomada do comando político da
cidade, liderada pela aristocracia, que nasce aquilo que Pocock nomeia
como o ``mito de Veneza'' (\versal{POCOCK}, 1980). Veneza servia como modelo
porque conseguia reunir diversas qualidades almejadas pela aristocracia
florentina, transformando-se num ideal de convivência cívica. A
estabilidade política e a liberdade, bem como a existência de um governo
misto e a virtuosidade de seus gentis-homens, eram entendidas como as
causas principais para a riqueza da república do Norte. O governo
comandado pelo Doge (chefe do executivo de caráter vitalício) e seus
Conselhos (compostos quase que exclusivamente pela aristocracia) seriam
a realização do regime misto idealizado pelos filósofos. Nos escritos de
venezianos da segunda metade do século \versal{XV}, como Francesco Barbaro,
Giorgio da Trebisonda e Bernardo Bembo, Veneza correspondia, até nos
detalhes, à república proposta por Platão, principalmente por conter em
si as três formas de governos particulares ou simples. Ademais, essa
defesa do regime misto, que não está somente nos textos platônicos, mas
também em Aristóteles, Políbio e Cícero, levou esses escritores a
afirmar que Veneza era a realização do modelo clássico de república
ideal (\versal{GILBERT}, 1977).

Na visão da aristocracia florentina, o principal resultado alcançado por
esse governo misto era a ausência de conflitos políticos num ambiente de
grande liberdade cívica, entendida num duplo sentido: como a existência
de um governo não tirânico e não estarem submetidos a outra cidade
(\versal{GILBERT}, 1977). As narrativas que chegavam a Florença sobre a república
veneziana relatavam que ela havia sido instalada há muito tempo e não se
tinha notícia da ocorrência de conjurações ou tumultos políticos que
ameaçassem sua normalidade republicana, fruto, também, da grande
\emph{virtù} de seus cidadãos. Essa fama de Veneza como república
pacífica lhe rendeu a alcunha de ``república sereníssima''.

Tal imagem modelar de Veneza revelou-se um mito na medida em que os
próprios humanistas começaram conhecer melhor a real estruturação do
regime republicano que lá vigorava. Com melhores informações sobre o
funcionamento da república veneziana, descobre-se que se tratava de um
governo tipicamente oligárquico, pois era dominado por um Conselho
estritamente limitado e controlado por um número pequeno de famílias.
Como diz Gilbert (1977), poucas pessoas em Florença conheciam como
realmente se ordenava o regime veneziano. A admiração estava fundada
mais nas narrativas e imagens projetas da cidade do que na realidade
política.

Seja como for, a imagem da república veneziana passou a exercer
relevante influência em Florença antes mesmo da instalação do governo de
Savonarola. Quando do nascimento do governo republicano, em dezembro de
1494, uma das principais inovações do novo regime foi a instauração do
\emph{Conselho Maior}, à semelhança do Grande Conselho de Veneza, com
ampla participação dos vários grupos sociais\footnote{Compreender o
  intrincado funcionamento do regime republicano de Florença é uma
  tarefa difícil. Gilbert nos informa que havia neste período
  aproximadamente 3.300 cargos eletivos, para uma população de não mais
  de 60.000 pessoas. Em termos proporcionais significava dizer que,
  dentre a população masculina com direito a voto, entre 1/4 ou 1/5
  participavam de algum cargo eletivo, o que é significativo em termos
  de participação popular (\versal{GILBERT}, 1996, p. 25, nota 2).}. Neste
sentido, o que era apenas um instrumento de fachada no ordenamento
político veneziano, em Florença, sob o governo republicano, passa a
funcionar de fato\footnote{Notório saber que, quando da restauração do
  governo dos Medici em 1512, um de seus primeiros atos foi a demolição
  do salão onde funcionava o Grande Conselho.}. Apesar dessa modificação
constitucional importante, novas demandas se faziam sentir, levando à
continuidade do debate sobre a melhor forma de governo. Em todos esses
momentos de confronto político por reformas nas instituições
republicanas da cidade, o exemplo veneziano sempre voltava à baila,
tanto que, na reforma de 1502, tem-se a instituição de um
\emph{gonfaloniere a vita}, ou seja, a versão florentina para o
\emph{Doge} veneziano, esse o chefe do executivo.

No \emph{Príncipe} e em vários capítulos dos \emph{Discursos sobre a
primeira década de Tito Lívio}\footnote{Doravante somente citado como
  \emph{Discursos}.}, Maquiavel apresentará afirmações contrárias às
posições teórico-políticas aristocratas. No caso de sua análise sobre a
\emph{História de Roma} de Tito Lívio, aquela discordância ganha uma
característica especial, pois Maquiavel se propôs a tomar como
referência de reflexão a mesma obra sobre a qual o aristocrata Bernardo
Rucellai, cunhado de Giovanni de Medici (que havia governado Florença
até 1492) já havia tecido seus comentários. Partindo dos mesmos métodos
analíticos que Rucellai, Maquiavel retira da história romana conclusões
opostas às dele. Conclusões não somente desfavoráveis à aristocracia,
mas aos ideais e modelos propostos por ela. Como afirma:

\begin{quote}
E além disso, levantar-me-ei contra as a opinião de muitos, segundo a
qual Roma foi uma república tumultuária {[}\ldots{}{]}. Direi que quem
condena os tumultos entre os nobres e a plebe parece censurar as coisas
que foram a causa primeira da liberdade de Roma e considerar mais as
assuadas e a grita que tais tumultos nasciam do que os bons efeitos que
eles geravam (\emph{Discursos}, \versal{I}, \versal{IV}).
\end{quote}

Na sequência, ele insiste no elogio aos conflitos:

\begin{quote}
E não se pode ter razão para chamar de não ordenada uma república
dessas, onde há tantos exemplos de \emph{virtù}; porque os bons exemplos
nascem da boa educação; a boa educação, das boas leis; e as boas leis,
dos tumultos que muitos condenam sem ponderar (\emph{Discursos}, \versal{I},
\versal{IV}).
\end{quote}

Os conflitos políticos são mobilizados num momento do texto no qual
Maquiavel busca uma outra fonte ou origem para os bons ordenamentos,
após ter mostrado que o modelo histórico e determinista polibiano sobre
os destinos dos regimes políticos, também conhecido como a teoria da
\emph{anacyclosis}, não dava mais conta de explicar as mudanças
políticas e a insuficiência do legislador em bem ordenar a cidade
(\emph{Discursos}, \versal{I}, \versal{II}). Os conflitos políticos se apresentam nos
capítulos \versal{III} e \versal{IV} dos \emph{Discursos} como a melhor alternativa para a
fundação de ordenamentos em cidades que não tiveram a sorte de ter um
sábio legislador, como foi Licurgo para Esparta. No caso de Roma, essa
ausência foi suprida ao acaso pelos conflitos políticos. Mais do que
isso, para Maquiavel, a experiência romana mostrou que, dessa maneira, a
fundação das cidades seria mais segura, haja vista que resultaria da
luta dos dois humores ou grupos sociais presentes em todas as cidades:
os grandes (\emph{popolo grasso}) e os pequenos (\emph{popolo minuto}).

Como veremos adiante, Maquiavel demonstrará a maior adequação do regime
republicano calcado nos setores populares contra a posição teórica de um
regime republicano de caráter aristocrático. Assim, se a aristocracia
florentina admirava o regime veneziano, sua estabilidade política, sua
natureza e seu virtuosismo aristocrático, Maquiavel verá nesses mesmos
aspectos fraqueza e enxergará a virtude nos conflitos políticos, na
instabilidade dos regimes republicanos. Ao contrário de pensar no bom
regime como uma república de tipo aristocrático, ele destacará as
qualidades populares das repúblicas. Será justamente na parcela popular
do governo republicano de Roma, e não nos seus quadros aristocráticos,
que esta encontrara sua força, seu vigor e sua grandeza.

Se, para um autor como Leonardo Bruni (séc. \versal{XV}), representante daquilo
que ficou conhecido como o \emph{humanismo cívico} (ou seja, a expressão
do pensamento político do Renascimento italiano), a exaltação de Roma,
acompanhada da afirmação de que Florença era sua filha, implicava a
defesa da liberdade política, um louvor às suas origens e de sua
excelência virtuosa, em Maquiavel cessa o tempo da apologia e começa o
tempo da crítica. Ao pensar em Roma como o modelo que inspiraria
Florença, ele ressaltará os contrastes ao invés dos paralelos, as
diferenças ao invés das semelhanças. Para o Secretário florentino, se a
república de Florença tivesse um ordenamento político assentado mais
sobre o povo e menos sobre a aristocracia, os destinos de sua cidade
poderiam ter sido outros. O que antes se colocava como uma ampla crítica
à aristocracia, agora se restringe à aristocracia florentina, que, nesse
aspecto, foi pior para os destinos da cidade do que a aristocracia
romana. Essa acusação atinge o cerne da ideologia dos aristocratas: a de
que Florença estava revivendo a \emph{virtus civita} da Roma
republicana.

Por outro lado, para os republicanos aristocráticos florentinos, os
conflitos políticos presentes na república romana seriam sinais da
corrupção política e, portanto, da perda da virtude cívica. Com efeito,
é somente numa certa compreensão da \emph{virtus civita}, assentada na
força e no poder romano, ou seja, numa adequação aos ideais humanistas,
que se poderia pensar na virtude como fundamento político (\versal{SKINNER}, 2000
e 2006). Para eles, a corrupção política romana começa quando se
manifestam os tumultos políticos, quando a unidade política da cidade se
vê fraturada pelas contendas entre os grupos. Não é sem fundamento que
vários pensadores, tanto antigos quanto modernos, entenderam a crise e a
decadência das repúblicas como associadas à perda da \emph{virtus
civita}, manifesta pelo conflito, índice maior da corrupção na cidade.

Como apontaram Gilbert (1996), Tafuro (2004, 2005) e Bignotto (1991), o
modelo republicano defendido pela aristocracia florentina representava
uma defesa das posições políticas desse grupo. Ao insistir que as
principais esferas decisórias (os Conselhos superiores da república
florentina) deveriam ser compostas majoritariamente ou totalmente (como
foi o caso do Conselho de Justiça) por membros da aristocracia; ao
defender que um ordenamento político assentando nesse extrato garantiria
a paz e a estabilidade política; e, principalmente, ao sustentar que
essa proeminência da aristocracia devia-se à sua virtuosidade, ao seu
amor e dedicação à pátria, estavam eles construindo uma teoria
republicana de caráter aristocrático. De fato, é nessa qualidade cívica
superior, que se expressa pela \emph{virtù} dos gentis-homens, que se
justifica o destaque dos segmentos aristocratas na vida política de
Florença\footnote{Nesse sentido, discordamos de Araújo (2000, p. 15),
  quando diz que para ``Maquiavel a instabilidade política é
  indesejável''. Ademais, se não se pode falar em toda a tradição dos
  autores republicanos, ao menos em Maquiavel a ampliação do poder
  político, inserindo novos atores para além da aristocracia ou dos
  homens dotados de virtude, ou seja, os segmentos mais populares, é não
  somente desejável, como é a resposta para o fantasma da corrupção
  política.}. Ideais esses que não se mostravam como uma novidade
teórica, haja vista a semelhança deles com algumas formulações de regime
misto no qual a aristocracia é considerada o segmento político mais
relevante na ordenação, como se pode perceber em Platão, Políbio e nos
primeiros escritos de Cícero (\versal{LEPORE}, 1954).

Neste contexto político e cultural, a posição de Maquiavel é
privilegiada, pois não somente ocupou um cargo estratégico (o comando de
uma importante secção da vida política da cidade e das relações
políticas entre os Estados), como pode conhecer a fundo o funcionamento
dessas engrenagens, e também participar desse debate político como um
interlocutor relevante.

\subsubsection{A saída da Chancelaria e o final da vida}

Depois que o governo Soderini cai e Maquiavel é demitido de suas
funções, em setembro de 1512, ele passa a produzir a parte mais
significativa de suas obras, entre as quais: \emph{O Príncipe} (em
1513), os \emph{Discursos sobre a Primeira década de Tito Lívio}
(1515-1517), a \emph{História de Florença} (1520-1525), a \emph{Arte da
Guerra} (1519- 1521) e os opúsculos políticos.

Entretanto, a saída de Maquiavel da Chancelaria é um fato conturbado e
nebuloso, mas que, uma vez compreendido, ajuda a entender a inserção
dele no ambiente político e intelectual florentino. Entretanto, convém
recuar um pouco no tempo.

Em 1502, por pressão da rica aristocracia florentina (os \emph{nobres}
segundo Maquiavel), ocorre uma mudança constitucional com a instauração
do \emph{gonfaloniere a vita}, ou seja, o comandante político da cidade
passa a ser um cargo vitalício, como era o \emph{doge} de Veneza. Isso
implica um ganho de poder por parte do \emph{gonfaloniere} Pier
Soderini, que consolida sua força política perante os outros setores
aristocráticos, bem como mantém seu grande apoio popular. A partir desta
data, conforme Guidi (2009), Maquiavel passa a ocupar uma posição
central na vida política florentina. Mesmo não sendo de família
aristocrática, o que o impedia formalmente de ser um embaixador ou
chanceler, ele fica responsável por diversas negociações importantes,
sendo que os embaixadores florentinos eram chamados ao final apenas para
assinar os acordos e os tratados.

Após 1502, Maquiavel também vai se envolvendo cada vez mais com as
questões militares da cidade, escrevendo e estudando sobre o assunto, a
ponto de, em 1506, ser o responsável pela criação da milícia florentina,
algo até então inexistente, pois a prática era a contratação de
exércitos ou tropas mercenárias para fazer as guerras.

A queda do governo Soderini começa quando as tropas espanholas invadem a
Toscana e tomam a cidade de Prato (vizinha de Florença) em agosto de
1512. A derrota retumbante das tropas florentinas perante a força dos
exércitos espanhóis desmoraliza o governo de Soderini que, em 31 de
agosto de 1512, foge para a cidade de Siena e depois segue para o exílio
na Dalmácia, onde é hoje a Croácia.

Com a queda de Soderini, quem assume o poder é inicialmente seu
adversário Giovan Battista Ridolfi, que inicia a reforma do governo mas
que é substituído, alguns meses depois, por Guiliano de Medici, que em
breve seria eleito papa (\versal{MARTELLI}, 2006, p. 9-11).

Maquiavel é demitido dois meses depois da derrota de Prato e logo em
seguida é preso sob a acusação de conspirar a morte de Giuliano de
Medici. Maquiavel fica dois meses preso, é torturado e solto por um fato
inusitado: a eleição de Giuliano ao papado em março de 1513. O papa
recém-eleito concede, então, anistia aos presos políticos de Florença. O
problema dessa história toda está em entender por que Maquiavel, que foi
demitido de suas funções, preso e torturado pelos partidários dos
Medici, pensa em dedicar a sua obra ao papa (que era sua intenção
inicial), mas depois dedica ao sobrinho deste, Lorenzo de Medici? Mais
ainda, mesmo sendo identificado ao governo de Soderini, do qual de fato
era o mentor intelectual, mas tendo amigos entre os partidários dos
Medici, haja vista a contínua troca de correspondência e favores após
1512, por que esses e o próprio Giuliano de Medici, que bem conheciam
Maquiavel, não entenderam logo que ele não fazia parte da conspiração,
mas que seu nome foi colocado numa lista de conspiradores sem de fato
ele mesmo saber?

A resposta para esses fatos explica muito do contexto florentino de
então. Como escreverá Maquiavel em um texto dessa época, \emph{Riccordo
di Niccolò Machiavelli ai Palleschi} (ou somente \emph{Ai Palleschi,} do
final de 1512), o problema que envolvia Florença e ele em particular
eram as oposições ferozes de parte da oligarquia florentina ao seu nome.
Nesse texto, escrito no calor dos acontecimentos, Maquiavel faz uma
análise deste contexto político se dirigindo aos \emph{palleschi} ou
\emph{pallesco}, ou seja, como ficaram conhecidos os apoiadores dos
Medici. Nele, o Secretário Florentino mostra que o problema político não
era Soderini e seu governo, pois, se assim o fosse, a sua queda teria
dado à cidade a tranquilidade política esperada; todavia, a
instabilidade permanecia. Curioso notar por este opúsculo que ele não
possuía grandes diferenças ou animosidades com os partidários dos
Medici. Na verdade, o texto indica com precisão que os verdadeiros
adversários de Soderini (parte da camada rica da cidade, ou seja, parte
da oligarquia) não serviria de apoio para o novo regime Medici.
Simplificando, Maquiavel avisa que os seus inimigos (pois essa
oligarquia também era inimiga dele) não seriam fiéis apoiadores do
regime dos Medici.

Portanto, Maquiavel não tinha nos Medici e seus apoiadores inimigos ou
adversários políticos, embora não se possa dizer de nenhum modo que ele
fizesse parte desse grupo. O que permite entender agora porque ele não
culpou os Medici pela sua prisão e dedicou a sua obra para o seu líder,
pois sabia que os seus reais adversários eram parte da aristocracia rica
e não o grupo político dos Medici. Esse pequeno texto maquiaveliano é,
então, muito ilustrativo para explicar o episódio da queda do governo de
Soderini e a prisão de Maquiavel, mas é mais explicativo de passagens de
\emph{O Príncipe}, como se verá melhor adiante, quando ele diz que o
príncipe novo não deve manter sua força política apenas apoiada nos
nobres ou grandes, mas, se tiver que escolher entre os nobres e o povo,
que escolha o povo (\emph{Príncipe,} \versal{IX}). Enfim, desde o ingresso no
governo da cidade, passando pela reforma de 1502 até a sua queda,
Maquiavel teve de fato entre seus maiores adversários políticos as
oligarquias ou \emph{nobreza} locais.

Esse panorama no qual se insere a vida pública de Maquiavel revela, por
seu turno, um vivo ambiente de disputas políticas que, como não poderia
deixar de ser, também foi marcado pelo debate intelectual sobre a melhor
forma de governo republicano para Florença. Este ambiente intelectual é
decisivo para a reflexão política maquiaveliana. Mais adiante,
explicaremos melhor a influência desse contexto sobre a produção
intelectual do \emph{Príncipe}, mas, por ora, podemos dizer que, do
ponto de vista geral de sua reflexão teórica, esse contexto de debate
marca os textos maquiavelianos na medida em que perpassa em todos a
busca pela definição de qual a melhor forma de governo republicano e
qual a natureza deste.

Por fim, convém lembrar que a Itália do período maquiaveliano não era um
Estado unificado sob o controle de um único governo, mas, ao contrário,
um conjunto de territórios independentes com governos autônomos. A
cidade de Florença, que possuía total autonomia e independência
política, constituía-se como um importante centro político, econômico e
cultural, no qual se alternavam governos republicanos, como o do período
em que Maquiavel foi diplomata, e autocráticos, caracterizados pelo
domínio da família Medici.

Quando, em 31 de agosto de 1512, o governo republicano de Pier Soderini
cai e os Medici retornam ao poder em Florença, Maquiavel se vê na
iminência de deixar o cargo na Chancelaria. Com efeito, em novembro do
mesmo ano ele é destituído de seu posto. Não bastasse a perda das suas
funções, que tanto prezava, poucos meses após sua demissão, em fevereiro
de 1513, ele é preso e torturado, sob a falsa acusação de participar de
um complô para assassinar um membro da família Medici. Ao sair do
cárcere, Maquiavel deixa a cidade de Florença e vai viver em sua pequena
propriedade rural, em Sant'Andrea in Percussina. É nesse pequeno
vilarejo, a poucos quilômetros de Florença, que Maquiavel passará o
restante de seus dias.

Não tendo mais obrigações diplomáticas, durante o dia ele se ocupa dos
negócios da propriedade, recolhendo-se à noite em seu escritório para o
estudo e a redação de suas obras. Nesse seu exílio forçado, ele escreve
seus principais textos políticos: \emph{O Príncipe}, \emph{Os Discursos,
A Arte da Guerra} e a \emph{História de Florença.}

Em 1515, Maquiavel passa a frequentar um encontro de jovens aristocratas
nos jardins da família Rucellai, em Florença, encontros esses que
ficaram conhecidos como os encontros dos \emph{Orti Oricellari}. São
desses encontros que nascem a maior parte dos \emph{Discursos}. A partir
dessa época, Maquiavel também volta a assumir alguns encargos
particulares, como quando, em 1519, foi representante de interesses dos
comerciantes de Florença em Lucca. Fora esses e outros poucos trabalhos
avulsos, Maquiavel não desenvolve nenhuma atividade diplomática regular.

Em 1520 ele recebe, por meio do \emph{Studio} florentino e do papa Leão
\versal{X} (Guiliano de Medici), o encargo de escrever uma história da cidade de
Florença. Durante quatro anos ele trabalha nesta obra, concluindo-a em
1525, indo pessoalmente a Roma para presenteá-la ao novo Papa Clemente
\versal{VII} (Giulio de Medici), que a recebe com apreço.

Aparentemente, tudo indicava a volta de Maquiavel às suas funções
diplomáticas, principalmente quando, em de maio de 1527, a família
Medici é deposta e é instaurado um novo governo republicano em Florença.
Mas a fortuna também não vem ao seu auxílio desta vez. Por uma grande
ironia do destino, a acusação que em um primeiro momento lhe fez sair do
governo quando das ascensão dos Medici (ser um defensor do regime
republicano), não se fez presente quando a república foi restaurada,
quinze anos depois. Diante desse revés, a sua saúde não resiste e ele
morre pouco mais de um mês depois, a 21 de junho.

\subsection{As edições}

O texto que ora se apresenta como \emph{O Príncipe} de Nicolau Maquiavel
tem na história de sua elaboração e difusão algumas peculiaridades. A
primeira informação sobre sua confecção vem de uma carta de Maquiavel a
Francisco Vettori de 10 de dezembro de 1513, na qual o autor fala da
composição de um opúsculo intitulado \emph{De Principatibus}, como diz:

\begin{quote}
E, como disse Dante, não pode a ciência daquele que não guardou o que
ouviu -- noto aquilo de que pela sua conversação fiz cabedal e compus um
opúsculo, \emph{De Principatibus}, onde me aprofundo quanto posso nas
cogitações deste tema, debatendo o que é principado, de que espécies
são, como eles se conquistam, como eles se mantêm, por que eles se
perdem. {[}\ldots{}{]} Portanto eu o dedico à magnificência de
Juliano\footnote{Carta a Francisco Vettori, 10 de dezembro de 1513.
  \emph{In}: \versal{MAQUIAVEL}, N. \emph{O Príncipe}. Trad. Lívio
  Xavier. São Paulo: Abril Cultural, 1973. p. 119 (Os Pensadores, \versal{IX}).}.
\end{quote}

Sabe-se, pois que, no final do ano de 1513, Maquiavel já havia terminado
o seu pequeno texto sobre ``que coisa é o principado, de quantas
espécies são, como se conquistam e se conservam''. Ainda nessa
carta, Maquiavel declara que este texto já havia sido lido por outro
amigo, Filippo Casavecchia, com o qual ele teve oportunidade de discutir
o texto. Segundo Inglese, certamente, entre o texto enviado a Vettori em
dezembro e a resposta deste em 18 de janeiro de 1514, Maquiavel foi
polindo e retocando o seu texto, que se constitui de fato na primeira
parte do livro. Com efeito, por aquilo que é indicado na carta de
dezembro de 1513, temos o roteiro dos temas tratados entre o capítulo \versal{I}
e o \versal{XI}, de uma obra que contém 26 capítulos.

Depois das dificuldades e negativas de Vettori em entregar o opúsculo de
Maquiavel para o papa, há poucas notícias sobre o manuscrito
maquiaveliano. Ele dará notícia novamente por mais sete vezes em suas
correspondências pessoais, sendo a última, em maio de 1514.

Talvez em função do malogro de entregar ao papa a obra Maquiavel tenha
modificado o destinatário da obra, não mais o papa, Giuliano de Medici,
mas a seu sobrinho, Lorenzo di Piero di Medici, então chefe político de
Florença, tal qual estabelecido pela tradição editorial. Um dado é
certo, esta dedicatória é anterior a outubro de 1516, quando Lorenzo
recebe o título de Duque de Urbino, coisa a que Maquiavel não se refere
na carta dedicatória, mas que jamais o teria feito se Lorenzo já tivesse
recebido o título, pois essa seria uma falta de decoro grave (\versal{INGLESE},
1994, p. 7). Portanto, a partir da própria documentação manuscrita e das
correspondências de Maquiavel, sabe-se que este obra foi composta entre
1513 e outubro de 1516.

Tendo em vista a não aceitação do texto pelo Papa, o opúsculo \emph{De
Principatibus} de Maquiavel passou a circular de forma manuscrita entre
os seus amigos durante muitos anos, pois a primeira edição impressa
sairia somente em 04 de janeiro de 1532. Ou seja, entre o final de 1513
e janeiro de 1532, o texto maquiaveliano circulou de modo informal e
manuscritamente. Ora, tendo em vista que o autor morre em 1527, nasce um
problema incomum para as obras compostas após a invenção da impressa no
século \versal{XV}, a saber: qual é o grau de originalidade do texto que foi
publicado em 1532? Será ele de fato a primeira versão do texto feita por
Maquiavel ou uma cópia de segunda ou terceira mão, com acréscimos,
supressões e demais alterações realizadas de modo costumeiro pelos
copistas? Enfim, qual o grau de originalidade do texto italiano de
\emph{O Príncipe} impresso pela primeira vez em 1532?

Tal dificuldade se amplia tendo em vista que, segundo relata Inglese
(1994, p. 10- 14), após 1514 o texto passou a circular de forma
manuscrita entre os amigos mais próximos de Maquiavel. Uma nova
referência ao texto apenas retorna em 29 de julho de 1517, quando o
jovem Nicollò Guicciardini, escrevendo ao seu pai Luigi, então
comissário florentino na cidade de Arezzo, sugere a ele se comportar
como disse ``Maquiavel em sua obra \emph{De Principatibus}'' (\versal{INGLESE},
1994, p. 14). Luigi Guicciardini conhecia Maquiavel dos tempos da
Chancelaria, tendo recebido certamente uma cópia do texto sobre os
principados. Luigi era irmão de Francesco Guicciardini, também diplomata
e grande pensador político, que escreverá um comentário aos
\emph{Discursos} de Maquiavel e que certamente deve ter lido esse
opúsculo maquiaveliano antes de 1532, pois faz referência a passagens
desse na sua obra \emph{Discorso del modo di assicurare lo stato alla
casa de' Medici}, texto de 1516 (\versal{INGLESE}, 1994, p. 15), bem como existem
passagens do \emph{Reggimento di Firenze}, obra elaborada entre 1521 e
1524, que indicam o conhecimento da argumentação do \emph{De
Principatibus} de Maquiavel (\versal{PROCACCI}, 1995, p. 6). Outra informação
sobre a circulação do texto vem de dois discursos proferidos por
Lodovico Alamanni, expoente do grupo dos Medici que, em 25 de novembro e
27 de dezembro de 1516, pede aos governantes que se comportem de modo
cívico, em uma argumentação que procura imitar a exposição de \emph{O
Principe}, seja no tema e personagens mobilizados, seja no estilo.

Contudo, segundo Inglese, o grande divulgador do \emph{De Principatibus}
de Maquiavel em forma manuscrita foi seu auxiliar dos tempos de
Chancelaria, Biaggio Buonaccorsi (\versal{INGLESE}, 1994, p. 17-18). Ele, que era
amigo de Maquiavel e nutria também um grande interesse pelas questões
políticas, após a saída da Chancelaria, em novembro de 1512, dedica-se,
entre outras coisas, à cópia de textos e monta um ateliê para
isso\footnote{Cumpre lembrar que, mesmo após a invenção da imprensa,
  manteve-se o costume, entre as famílias ricas, de mandar confeccionar
  manuscritos de obras importantes, que eram produzidos com qualidade,
  seja nas folhas utilizadas, seja na encadernação. Certamente o ateliê
  de Buonaccorsi recebeu encomendas de membros da aristocracia
  florentina, sabedores do texto de Maquiavel e curiosos para
  conhecê-lo. Inglese informa ainda que, mesmo depois da publicação da
  edição impressa em 1532, houve ainda a confecção de textos manuscritos
  do \emph{De Principatibus} por alguns anos.}. Conforme os estudos
paleográficos, de seu ateliê saíram ao menos três manuscritos
importantes que compõem o aparato crítico contemporâneo. Mais ainda, dos
27 manuscritos tomados em consideração por Inglese para realizar sua
edição crítica, 16 deles foram cópias feitas a partir desses exemplares
de Buonaccorsi. Isso indica que mais da metade dos manuscritos tem uma
influência direta desses exemplares \emph{buonaccorsianos} e certamente
ele teve acesso ao manuscrito original de Maquiavel.

Antes de prosseguir nessa apresentação do texto, é importante dar
algumas informações básicas sobre paleografia. Toda vez que se tem uma
edição de uma obra impressa não revisada e autorizada pelo autor, nasce
a dificuldade de saber se o texto impresso publicado confere com o
original, também conhecido como \emph{autógrafo}, ou seja, escrito pelo
autor. Quando este é vivo no momento da publicação ou se tem o
manuscrito \emph{autógrafo} ou se tem algum documento (carta,
testamento, etc.) que comprova que aquilo que foi publicado era a
intenção do autor, de modo que não há problemas e donde a primeira
edição impressa, também conhecida como edição \emph{princeps}, tornar-se
a referência para todas as demais publicações.

Contudo, quando a publicação impressa é posterior à morte do autor ou
este ainda em vida não reconhece como verdadeiro aquilo que foi
publicado -- como no caso do filósofo inglês John Locke que, durante
muito tempo, travou uma batalha com os editores de seu \emph{Segundo
Tratado sobre o governo civil} em função das diversas alterações
inseridas pelo editor, e só veio a reconhecer como seu o texto publicado
apenas no final de sua vida (\versal{LASLETT}, 1998) --, ou ainda se a obra foi
divulgada antes do advento da imprensa, então faz-se necessário um
estudo sobre os manuscritos existentes para se estabelecer o texto
fidedigno ou o mais fidedigno possível. O texto publicado que é o
resultado desse trabalho de levantamento e análise dos manuscritos
denomina-se \emph{edição crítica}, cujo texto, se não é idêntico ao
original do autor, ao menos reproduz aquilo que é o mais próximo deste.
Ora, tal trabalho de pesquisa pode ser facilitado ou dificultado
conforme o material que se tenha a disposição. Quando se encontra o
\emph{autógrafo}, a pesquisa se encerra, pois todas as demais
\emph{cópias} manuscritas foram feitas a partir deste primeiro exemplar
ou ele passa a ser a referência para as demais cópias e o texto padrão
da edição impressa. Todavia, quando não há esse exemplar, deve-se
recorrer às técnicas de paleografia e filologia para tentar estabelecer
as relações de dependências entre as cópias e identificar qual seria o
autógrafo ou aquele que mais se aproxima deste, quando ele não existe.

Voltando à difusão do texto do \emph{De Principatibus}, uma outra grande
fonte de informação é o possível plágio de Agostino Nifo, que publicou
em outubro de 1522 o opúsculo \emph{De Regnandi Peritia}, que não
somente mantém a mesma sequência, como reproduz exemplos e argumentos
inteiros do \emph{De Principatibus}, embora o texto de Nifo tenha sido
escrito em latim e o de Maquiavel em italiano (\versal{INGLESE}, 1994, p. 18-22).
Contudo, como explica Larivaille (1989, p. 150-195), falar em plágio de
Nifo sobre o texto de Maquiavel é um tanto equivocado, pois, apesar das
inúmeras semelhanças, o que revela que Nifo leu o texto de Maquiavel,
seu argumento conduz a conclusões diferentes das expostas no \emph{De
Principatibus}, mantendo a tradição de pensamento político de tipo
moralista dos \emph{espelhos de príncipes}\footnote{Apenas adiantando
  algo que será melhor explicado, havia uma tradição de longa data de
  livros de aconselhamentos para os príncipes, também denominados
  \emph{espelhos de príncipes}, que exaltavam a virtude e moralidade
  como as principais qualidades que o príncipe deveria cultivar. Como
  veremos, o texto maquiaveliano é totalmente contrário a isso, não
  sendo possível enquadrá-lo nesta categoria. (\versal{SKINNER}, 2000)}\emph{.} A
conclusão, portanto, é que Nifo leu o texto maquiaveliano e se inspirou
nele, e não o plagiou para escrever sua obra. Ademais, a noção de
plágio, tal qual nós entendemos, não existia neste contexto do
Renascimento, pois os autores se utilizavam de outros sem fazer a devida
referência e até o próprio Maquiavel fará uma verdadeira paráfrase do
livro \versal{VI} das \emph{Histórias} de Políbio nos seus \emph{Discursos} sem
dar qualquer referência ou indicação do texto antigo. Como se verifica
ao longo de várias passagens de \emph{O Príncipe}, Maquiavel cita de
memória passagens de textos históricos romanos, com frequentes erros e
sem qualquer referência.

Sabe-se ainda que Agostino Nifo era um pensador influente e foi
escolhido pelos Medici para lecionar no \emph{Studio} florentino, atual
Universidade de Pisa. Ora, certamente o fato de ele ter lido o texto
maquiaveliano e se baseado nele para escrever o seu próprio texto
demonstra que o \emph{De Principatibus} de Maquiavel era um texto um
tanto conhecido nos círculos eruditos de Florença, contando inclusive
com alguma aceitação e respeitabilidade.

Após essa última informação sobre \emph{O Príncipe} advinda da obra de
Nifo, só teremos notícia do texto maquiaveliano quando de sua publicação
em 04 de abril de 1532 pelo editor Antonio Blado, de Roma, que publica a
obra com o título em italiano, \emph{Il Príncipe}, e não mais em latim,
\emph{De Principatibus.} Logo em seguida, em 08 de maio, o tipógrafo
Bernardo Giunta, de Florença, também publica o texto com o título também
em italiano. Apesar de os estudos paleográficos e filológicos colocarem
em questão a fidedignidade da edição Blado em relação ao
\emph{autógrafo} do \emph{De Principatibus}, ela é reconhecida como a
edição \emph{princeps}, que significa que é primeira edição e sobre o
qual se deve basear a numeração e referência das demais edições.

Como se nota, foi do editor a decisão de alterar o título do livro do
latim para o italiano, cuja tradução mais correta de \emph{De
Principatibus} seria em italiano \emph{Sopra i principati} (Sobre os
principados). Outra mudança significativa entre a carta a Vettori de
dezembro de 1513 e a edição publicada é a inserção de uma parte
reservada às questões militares (capítulos de 12 a 14) e toda uma seção
destinada à figura do príncipe, com uma conclusão exortando à união da
Itália. Essa ampliação do texto sugere que Maquiavel tenha acrescido e
modificado o texto depois de dezembro de 1513. Contudo, a questão é:
quando ele concluiu de fato a obra e, como já foi discutido entre os
comentadores, teria tido \emph{O Príncipe} duas redações?

Quanto à primeira questão, a hipótese mais provável, em função das datas
e personagens citados, é que Maquiavel tenha ampliado e alterado o texto
ao longo de 1514, mas que o concluiu definitivamente antes de 1515 em
função de informações sobre alguns personagens e datas. Segundo Procacci
(1995, cap. 1), tendo em vista a difusão e aceitação dos manuscritos em
seu circulo mais próximo, Maquiavel tentou publicar o \emph{De
Principatibus,} mas em Roma e não em Florença, em função da oposição ao
seu nome entre a aristocracia desta cidade. Em sua última estadia em
Roma, em 1526, Maquiavel encomenda a cópia de um exemplar manuscrito de
seu texto ao ateliê de Ludovico degli Arrighi (o que é o atual
\emph{códice Barberiano 5093} do \emph{De Principatibus}) (\versal{PROCACCI},
1995, p. 8-9). Ora, por que Maquiavel encomendaria um novo exemplar
manuscrito e feito de modo acurado e elegante em um ateliê prestigiado
de Roma se não fosse para publicá-lo ou ofertá-lo a alguém que
patrocinasse tal projeto editorial? Segundo Procacci ainda, as conversas
já estavam adiantadas para isso, entretanto, a publicação não se
realizou em função do ataque que os franceses fizeram a Roma em 1527,
conhecido como o ``Saque de Roma'', que atinge duramente a cidade e suas
finanças. A retomada dos projetos somente ocorrerá após 1530, quando a
situação política parecia estabilizada. Em 1531, um tipógrafo sem muito
prestígio, proprietário de um negócio modesto, Antonio Blado di Asola,
recebe a autorização papal para publicar duas obras de Maquiavel: os
\emph{Discursos} e o \emph{De Principatibus}, cujo título ele modifica.
Blado trabalhou durante muitos anos a serviço da Câmara Apostólica e não
teve um catálogo nem amplo ou com títulos de renome ou significativos
(\versal{PROCACCI}, 1995, p. 9-10). Outro dado curioso é que, na autorização
papal, já se prevê a possibilidade de Blado negociar e permitir outras
edições dessas obras, fato esse que, segundo Procacci, demonstra que o
modesto tipógrafo já pretendia vender os direitos de uma obra que teria
boa repercussão editorial. Tanto é procedente tal interpretação que logo
na sequência tem-se a publicação do texto em Florença pelo editor
Bernardo Giunta e, nos anos seguintes, em Veneza, por três casas
tipográficas distintas (\versal{PROCACCI}, 1995, cap.1). Portanto, conforme a
interpretação de Procacci, a publicação do texto maquiaveliano era um
projeto intencionado desde 1526 por Maquiavel, mas que não ocorreu por
acasos: o ``Saque de Roma'' e sua morte.

A interpretação de Martelli é diferente e agrega alguns dados novos e
complementares. Na edição comentada de \emph{O Príncipe}, Martelli
(2002) explora em muito o fato de o nome de Maquiavel ser rejeitado em
Florença, mas não necessariamente pela família Medici. Isso impediu que
durante muito tempo seu texto fosse publicado e ele mesmo pudesse
retornar ao governo, então sob o comando da família. Mesmo sem ser
considerado um inimigo por parte da família Medici, segundo a
argumentação de Martelli, Maquiavel era um personagem cuja proximidade
provocava desconforto político perante parte da aristocracia florentina.
Do que decorre que sua obra não tenha sido publicada em Florença
inicialmente, mas em Roma e depois de sua morte, quando a oposição havia
em parte diminuído.

Entretanto, em alguns pontos Martelli se distancia das interpretações
precedentes. Segundo ele, o texto maquiaveliano começou a ser escrito em
1513 e foi concluído apenas em 1518 e não em 1514 ou início de 1515 como
defendem Inglese e outros. Além disso, mesmo admitindo que não existisse
o autógrafo do \emph{De Principatibus}, Martelli defende que há um
exemplar que teria sido corrigido por Maquiavel e que poderia ser
denominado como o \emph{arquétipo} do qual derivam todos os outros e que
seria esse manuscrito o mais genuíno\footnote{Segundo Martelli o
  manuscrito \emph{Carpentras}, \emph{Bibliothéque Inguimbertine, 303}
  (denominado como manuscrito A), seria esse arquétipo que foi corrigido
  por Maquiavel e seria o texto mais original, apesar de não ser um
  autógrafo. (\versal{MARTELLI}, 2002, 325-329)}.

Esses elementos invocados por Martelli nos remetem à já mencionada
questão recorrente nos estudos sobre \emph{O Príncipe}: teria ele tido
duas redações? Tal questão foi formulada inicialmente por Federico
Chabod e, depois de discutida longamente, teve sua resposta no estudo de
Sasso (1958) no qual se aponta que Maquiavel redigiu uma primeira parte
do texto em 1513 e o restante, provavelmente, em 1514, não alterando
mais o texto depois de 1515. Uma questão embutida nessa é saber se já no
período da Chancelaria Maquiavel não teria começado o texto,
concluindo-o no período imediatamente posterior a sua saída.

Tal hipótese de uma dupla redação e não de uma redação contínua no
biênio 1513 e 1514 é muito frágil, pois, mesmo que de fato Maquiavel
tenha começado a fazer o texto, ou mesmo que já tivesse um esboço ou
rascunho, certamente ele se valeu dessas informações anteriores -- como
ele mesmo declara na \emph{Carta Dedicatória} --, o que comprova que não
somente havia sim elementos anteriores a 1513, mas que tudo isso foi
reelaborado e ampliado nesse ano de redação. Sem esquecer que ele vinha
pensando nesses temas e argumentos ao longo do tempo, haja vista o
confronto com textos e escritos diplomáticos de análise de contextos,
com os textos posteriores a 1512, nos quais encontramos um autor mais
maduro e consciente de seu pensamento político. Do que se pode concluir
com muita segurança que Maquiavel já vinha de longa data meditando sobre
esses temas e que organizou e colocou no papel tudo isso no biênio de
1513 e 1514. Depois não mais.

Esse quadro informativo nos coloca diante do seguinte questão: o
manuscrito do \emph{De Principatibus} utilizado por Antonio Blado para
fazer a primeira edição impressa de \emph{Il Príncipe} era de fato o
autógrafo de Maquiavel? Mais, tendo em vista a alteração do título, qual
a garantia de que o editor não tenha alterado algo no texto, o que teria
corrompido sua autenticidade? Enfim, parece que a primeira edição
bladiana do texto maquiaveliano não confere nenhuma garantia sobre a
autenticidade do texto.

Essas suspeitas fizeram com que, já no século \versal{XIX}, houvesse uma primeira
tentativa de estabelecer um texto original ou padrão de \emph{Il
Príncipe}, por meio do editor G. Lisio, e editada por Sansoni em
Florença em, 1899. A também conhecida edição Lisio de \emph{Il Príncipe}
foi o primeiro texto com um mínimo de aparato crítico. Depois de algumas
edições que agregaram alguma informação ao texto, mas sem superar ou
corrigir a edição Lisio, Giorgio Inglese publica a edição crítica em
1994. Este, consultando uma gama maior de manuscritos e depois de um
trabalho apurado de análise paleográfica e filológica, apresenta uma
nova edição crítica do texto do \emph{De Principatibus} de Maquiavel,
publicada pelo Istituto Storico Italiano per il Medioevo. Além dessa
importante edição, em 2002 foi publicada uma outra edição de \emph{O
Príncipe}, dentro da coleção \emph{Edizione Nazionalle delle Opere di
Machiavelli}\footnote{Doravante citado apenas como \emph{\versal{EN}.}} pela
Editora Salerno, sob a coordenação de Mario Martelli, e que pretende
discutir e rever pontos não esclarecidos da edição Inglese, valendo-se
de outros manuscritos (que eram do conhecimento de Inglese mas que, por
critérios técnicos, foram por ele descartados) e buscando,
principalmente, refutar a tese de Inglese de que não há um autógrafo de
Maquiavel. Em outros termos, o ponto central da análise crítica de
Inglese é que o autógrafo do \emph{De Principatibus} está ainda perdido
e que a reconstrução dos manuscritos permite estabelecer os dois
principais exemplares dos quais dependem todos os demais e, com esses,
reconstruir o texto o mais próximo do original possível\footnote{A
  hipótese central de Inglese é que os manuscritos D (München.
  Universitätsbibliothek, 4º cod, ms. 787) e G (Gotha, Forschungs- und
  Landesbibliothek, chart. B 70) dependem de um outro manuscrito
  (denominado γ) que seria o arquétipo mais próximo do orignal. Cf.
  \versal{INGLESE}, 1994, p. 150-155.}. A contra-argumentação de Mario Martelli,
tenta, por outra metodologia, mostrar que há um autógrafo ou arquétipo
do texto maquiaveliano e que este foi aquele utilizado por Blado, no
caso, o manuscrito A (\emph{Carpentras}) da lista dos manuscritos
disponíveis.

Ora, não se trata, nesta introdução, de entrar nos detalhes de caráter
filológico ou paleográfico da discussão, visto que não temos elementos
para refutar posições ou novidades a acrescentar. Porém, cumpre declarar
que, no confronto das argumentações, seja em função da metodologia, seja
em função da autenticidade e autoridade dos manuscritos, a exposição de
Giorgio Inglese é mais consistente e o texto por ele organizado é mais
fidedigno. A argumentação de Martelli, ao utilizar-se de uma nova
``metodologia'' na qual a história do texto prevalece, não consegue, ao
fim e ao cabo, oferecer uma comprovação ou modificação substancial no
estatuto de originalidade do texto maquiaveliano. No limite, toda a
argumentação de Martelli apenas polemiza e não traz novos resultados em
relação a edição crítica de Inglese. Tanto é assim que o próprio
Martelli denomina sua edição de \emph{Edição Comentada.}

Donde, na falta de argumentos novos e de elementos concretos que
invalidem sua exposição, somos obrigados a aceitar a edição Inglese de
1994 do \emph{De Principatibus} como o texto mais fidedigno àquele
intencionado e escrito por Nicolau Maquiavel, motivo pelo qual o
escolhemos para ser o texto original italiano dessa edição bilíngue
brasileira.

\subsection{Estrutura do Argumento}

Como foi dito, \emph{O Príncipe} possuía um outro título que depois foi
alterado pelo editor romano, mas que não contradiz totalmente o conteúdo
da obra. Na verdade, pode-se dizer que o texto maquiaveliano contém o
\emph{De Principatibus} e \emph{Il Príncipe} ao mesmo tempo.

A obra está dividida em 26 capítulos, num total de 28.250 palavras. O
primeiro capítulo apresenta o roteiro dos temas tratados até o capítulo
11, que se concentram nos principados, na conquista destes e na
conservação do poder. Os três capítulos seguintes (12, 13 e 14) tratam
da questão militar, acerca da conveniência e inconveniência de se
possuir exércitos auxiliares e próprios, que também são pensados no
interior dos esforços de conservação do poder político. Os capítulos de
15 a 23 tratam das qualidades e cuidados que um príncipe deve possuir no
comando político. Os capítulos 24 e 25 analisam a importância da fortuna
na vida política e como o príncipe deve ficar atento a ela. Por fim, o
capítulo final é uma exortação à unidade da península itálica, algo
desejável sob o comando da família Medici.

Diante dessa disposição temática dos capítulos, uma primeira compreensão
geral da obra, sobre a qual concordam os comentadores, é dividir \emph{O
Príncipe} em duas grandes partes: uma primeira dedicada à análise do
principado e uma segunda voltada para a ação do príncipe. É evidente que
nesta divisão geral não se encaixam muito bem os capítulos finais,
particularmente o último.

Entretanto, seja essa divisão temática da obra, sejam as demais
informações apresentadas, tudo nos remete a uma ordem de problemas que
convém analisar com atenção. Um primeiro âmbito de problemas está
voltado à noção de principado, e pode ser expresso nos seguintes termos:
o que de fato é o principado de Maquiavel? Qual a sua definição e a sua
função no interior do pensamento político maquiaveliano? A resposta mais
comum nos manuais é identificar a noção de principado à moderna noção de
``Estado'', declarando desse modo que Maquiavel é um precursor do
pensamento político moderno. Como se verá, não se trata disso, pois
tanto a definição de principado como a de estado em Maquiavel não se
identificam com a moderna noção política de Estado, embora seja possível
reconhecer neles elementos que apontam para aquilo que o pensamento
político moderno, posterior a Thomas Hobbes, certamente identificou na
noção política de Estado, este entendido como uma entidade política
autônoma.

Uma outra questão que decorre diretamente dessa primeira ordem de
problema é sobre a concepção de príncipe de Maquiavel. A primeira e
imediata associação é com o título nobiliárquico de príncipe das
monarquias. Ora, essa concepção gera, por seu turno, a ideia de que o
texto maquiaveliano é um libelo em defesa da monarquia ou um livro
destinado aos nobres. Tal é o pano de fundo da argumentação de Quentin
Skinner, que no seu livro mais conhecido, \emph{As fundações do
pensamento político moderno}, classifica a obra de Maquiavel dentro de
um estilo que fez fama entre os séculos \versal{XI} e \versal{XVI}, os \emph{espelhos de
príncipes} (\versal{SKINNER}, 2000). Esse estilo de texto, que remonta ao
pensamento político romano, tendo como um de seus textos fundadores o
\emph{Dos Deveres} (\emph{De Officis}) de Marco Túlio Cícero (54 a.C.),
foi retomado ao longo do tempo no pensamento político latino, passando
pela Idade Média e Modernidade (\versal{SENNELART}, 2006)\footnote{No século
  \versal{XVIII} ainda se encontra esse tipo de texto entre os pensadores
  portugueses, o que comprova a fortuna de estilo no pensamento politico
  ocidental.}, e era destinado a auxiliar e instruir os governantes na
tarefa de governar, principalmente os jovens príncipes, que careciam de
experiências políticas, donde a definição ``espelhos de príncipes'': por
ser um manual onde os príncipes deveriam espelhar suas ações.

O desconforto em enquadrar o texto maquiavelino neste estilo reside mais
na definição de príncipe como o filho do rei, como o futuro monarca,
restringindo o escopo de denotação do termo, particularmente quando
tratamos de um autor que, como já foi declarado, assume-se claramente em
defesa do republicanismo. Disso decorre a contradição instalada: teria
um autor que é reconhecidamente defensor do republicanismo escrito um
livro em defesa da monarquia, tendo em vista para quem a obra é
dedicada, para tentar, entre outros motivos, um emprego no novo governo
da família Medici? Teria Maquiavel se destacado por escrever uma obra de
cunho monarquista por que estava desempregado e sem perspectivas? Enfim,
seria Maquiavel um pensador com uma dupla postura teórica: em algumas
obras defensor do regime republicano e em outra -- curiosamente na mais
célebre -- defensor do regime monárquico?

As respostas para tais problemas não são simples e nem fáceis, pois
demandam um esforço de análise e interpretação. O início dessa análise
está, todavia, fora de \emph{O Príncipe}, mas em outra obra, nos
\emph{Discursos.} Nesta obra, em particular, depois de apresentar os
elementos gerais da república, Maquiavel fala da corrupção desta, do seu
fim e como poder-se-ia encontrar alguma solução para a morte certa do
regime republicano. A resposta para essas noções de principado e
príncipe, bem como para a questão de fundo de como articular \emph{O
Príncipe} no interior do pensamento político maquiaveliano, passa por
uma análise sobre o fim da república. Enfim, a dificuldade que se
apresenta agora é de como articular \emph{O Príncipe} com o pensamento
republicano de Maquiavel, com as demais obras políticas de caráter
nitidamente republicano. Para isso será necessário recuperar, ainda que
rapidamente, os delineamentos gerais da noção de república tal qual
Maquiavel faz no início dos \emph{Discursos.}

\subsection{O ``Pequeno tratado sobre as repúblicas''}

No início da exposição dos \emph{Discursos sobre a primeira década de
Tito Lívio}, notadamente no primeiro livro, entre os capítulos 1 e 18,
encontramos um momento privilegiado no qual Maquiavel apresenta os
elementos principais do que entende por república\emph{.} Tais noções
também podem ser encontradas em vários escritos de modo esparso, como na
\emph{História de Florença}, no opúsculo \emph{Discurso sobre as coisas
de Florença}, na \emph{Arte da Guerra}, entre outros; mas em nenhum
deles temos de modo tão claro e estruturado como nos \emph{Discursos}.

Escrito, provavelmente, na sequência de \emph{O} \emph{Príncipe}, os
\emph{Discursos} expõem uma análise sobre parte da obra \emph{História
de Roma} de Tito Lívio. O objetivo do texto maquiaveliano é o de
comentar os fatos narrados pelo historiador romano e recuperar noções
para a elaboração de argumentos em defesa do regime republicano, no caso
particular para a Florença de início do século \versal{XVI}, bem como para uma
teoria em defesa do regime republicano de modo geral.

A \emph{História de Roma} foi a principal obra de Tito Lívio (59 a.C --
17 d.C). Composta originalmente por 142 livros, dos quais restaram
apenas 35, ela narra, em seu conjunto, os feitos romanos desde a sua
origem até o governo de Otávio Augusto (9 a.C.). Ao longo do tempo, os
copistas fizeram uma divisão da obra em grupos de dez livros, bem como
uma sinopse de cada livro que foram reunidas no início da
\emph{História}. A essa reunião dos livros em conjunto de dez, ainda que
nem sempre rígida, deram o nome de \emph{deca} (em italiano), que foi
traduzido por ``décadas'' em português, quando, na verdade, trata-se de
``dezenas''. Os dez primeiros livros ou a ``primeira década'' ou a
``primeira dezena'', foram dos poucos que se conservaram
integralmente e narram os fatos desde as origens de Roma até o ano de
295 a.C., ou seja, a narrativa compreende o governo monárquico e boa
parte do governo republicano. Maquiavel elabora os seus \emph{Discursos}
sobre esses dez primeiros livros da \emph{História de Roma}, uma vez que
nestes há a presença de vários temas que lhe são caros, entre eles, a
conservação da república, as mudanças de regimes e a corrupção das
instituições políticas.

Olhado em sua totalidade, o texto maquiaveliano parece ser um comentário
geral da obra liviana, no qual o comentador vai inserindo em sua análise
as ideias e as concepções políticas que busca defender. O problema maior
nessa visada geral são os primeiros dezoito capítulos do livro \versal{I}, no
qual Maquiavel não segue
\emph{pari passu} o
texto liviano, mas faz uma exposição ampla sobre alguns conceitos
republicanos, inserindo nessa análise exemplos romanos, mas também
exemplos de Esparta, Veneza, Florença e a Roma católica de seu tempo.
Como apontou Gilbert (1953), primeiramente, esses primeiros capítulos
parecem se constituir em um bloco à parte da obra. Doravante se instalou
entre os comentadores uma disputa interpretativa, na qual, entre outros
pontos de discórdia, pairava a questão se Maquiavel escreveu primeiro
esse trecho do livro ou se escreveu conjuntamente com o restante da
obra, entre os anos de 1515 a 1517. Questão essa que se complica, pois,
tendo em vista que Maquiavel teria escrito o \emph{De Principatibus}
antes dos \emph{Discursos}, nasce a dúvida de como entender um trecho
inicial de \emph{O Príncipe}, quando o autor declara:

\begin{quote}
Deixarei de lado a discussão sobre as repúblicas, porque, em outro
momento, dissertei longamente sobre elas. Ocupar-me-ei somente do
principado, e retecerei as urdiduras acima descritas, e demonstrarei
como estes principados podem ser governados e conservados (\emph{O
Príncipe}, cap. \versal{II}, 1-2).
\end{quote}

Resta, pois, a questão de qual seria esse texto republicano de Maquiavel
indicado no texto de 1513. Já nos antecipamos em afirmar que não é
possível, com os dados disponíveis até hoje, indicar se esses dezoitos
capítulos perfazem o texto republicano mencionado no início de \emph{O
Príncipe} e nem a cronologia correta de composição das obras. Tudo o que
se tem são hipóteses, algumas com base nos poucos dados históricos
disponíveis e outras com base na estrutura dos argumentos expostos por
Maquiavel em suas obras.

A polêmica em torno desses dezoito primeiros capítulos dos
\emph{Discursos} não se resume, então, a uma disputa cronológica de
anterioridade ou posteridade em relação a \emph{O} \emph{Príncipe}.
Subjaz a essa discussão a definição do sentido próprio da obra
maquiaveliana cuja interpretação determina a compreensão que ele possuía
das repúblicas, bem como do lugar de \emph{O Príncipe} no interior de
seu pensamento político. O estudo de Gilbert fez com que, sob diferentes
óticas e metodologias, as atenções para a interpretação do pensamento
político maquiaveliano se voltassem para os \emph{Discursos}. Uma vez
que não está ao nosso alcance reconstruir nos detalhes os pontos desse
debate e os meandros dessa contenda, algo que já foi analisado pelos
comentadores e que também abordamos em outro texto (\versal{MARTINS}, 2007, 16),
partiremos aqui do fato concreto de que esse conjunto de capítulos
iniciais perfazem aquilo que Larivaille nomeou como ``\emph{Pequeno
tratado sobre as repúblicas''}. Certamente, neste trecho da obra
maquiaveliana, encontraremos os elementos teóricos para compreender de
que modo pode ser possível pensar a inserção de \emph{O Príncipe} no
pensamento político de seu autor.

Voltando à passagem problemática de \emph{O Príncipe}, quando ele
escreve que não tratará das repúblicas porque já o havia feito em outro
lugar, deduz-se que estivesse se referindo aos \emph{Discursos,} visto
que não deixou um tratado específico sobre esse assunto e que aquela
obra traz uma reflexão sobre a república romana. Todavia, existem vários
indícios que mostram que os \emph{Discursos} foram escritos depois de
1514, quando Maquiavel frequentava os \emph{Orti Oricellari} \footnote{\emph{Orti
  Oricellai} era o nome que se dava aos jardins da família Rucellai,
  que, desde o final do governo dos Medicis, no século \versal{XV}, abrigava
  reuniões de aristocratas florentinos. Após a queda do governo de Pier
  Solderini, em 1512, o neto de Bernardo Rucellai, Cosimo, passou a
  organizar reuniões com jovens aristocratas de ideologia republicana.
  Maquiavel passa a frequentar esses encontros a partir de 1515, momento
  em que se acredita que ele tenha escrito a maior parte dos
  \emph{Discursos}. Todos os comentadores destacam a importância desses
  encontros para a reflexão política maquiaveliana no que diz respeito à
  teoria republicana e ao estudo dos clássicos (\versal{GILBERT}, 1977, p. 15-66;
  \versal{VIROLI}, 2002; \versal{SASSO}, 1986, p. 353-357). Adiante isso será melhor
  explicado.}, entre 1515 e 1517. Tais indícios colocam, pois, o
problema de tentar descobrir a qual texto Maquiavel estava fazendo
referência no início de \emph{O Príncipe}, visto que ele não
havia escrito, ainda, os \emph{Discursos.} Uma outra questão, paralela a
esta, estaria em saber ao certo qual foi o momento de composição dos
\emph{Discursos}: se antes ou depois da composição de \emph{O}
\emph{Príncipe}.

Esses são os pontos principais do debate em torno da datação dos
\emph{Discursos} e da existência ou não de um bloco textual destacado no
início do livro \versal{I}. Apesar dessas divergências, de modo geral, é aceito
que a parte principal do livro tenha sido escrita entre 1515 e 1517, o
que não exclui a possibilidade de que uma primeira parte já tivesse sido
escrita antes desta data, sendo apenas corrigida na época em que
Maquiavel frequentava os \emph{Orti Oricellai}. Depois das contendas
entre os comentadores, uma certa concordância se formou acerca de dois
pontos: a) que os capítulos que tratam dos comentários da \emph{História
de Roma} de Tito Lívio possuem uma unidade analítica a despeito da não
adequação de algum capítulo à essa regra geral, corroborando a tese de
Gilbert de que, nesses casos, as exceções confirmam a regra; b) que os
dezoito primeiros capítulos do livro \versal{I}, mesmo sendo o momento mais
teórico da obra, configuram-se como um anteparo conceitual explicativo
dos eventos que serão comentados. Apesar de não serem comentários
diretos aos livros de Tito Lívio, esses capítulos mostram-se importantes
na economia dos \emph{Discursos} como um todo, na medida em que explicam
as origens e os fundamentos das repúblicas e os ordenamentos, leis e
conflitos que marcam sua vida civil. Enfim, afirmar a existência de um
bloco teórico inicial não depõe contra a harmonia e unidade interna dos
\emph{Discursos}, impossibilitando qualquer afirmação de que a obra é
resultado de uma colagem ou de uma justaposição de textos diversos.

Os \emph{Discursos} estão divididos em três livros, sendo que o livro \versal{I}
tem 60 capítulos, o livro \versal{II} 32 capítulos e o \versal{III} 49 capítulos. Os
primeiros dezoito capítulos do livro \versal{I} perfazem a seguinte sequência
expositiva: a fundação das cidades (capítulo \versal{I}), a natureza e a mudança
dos regimes políticos (capítulo \versal{II}), os conflitos políticos (capítulos
\versal{III} e \versal{IV}), a defesa da liberdade política nas republicas (capítulos \versal{V} e
\versal{VI}), os instrumentos de defesa e acusação pública (capítulos \versal{VII} e
\versal{VIII}), a reforma ou refundação dos estados (capítulos \versal{IX} e \versal{X}), a
importância da religião (capítulos de \versal{XI} a \versal{XV}) e a corrupção nas
repúblicas (capítulos de \versal{XVI} a \versal{XVIII}).

Verifica-se, pois, que esses capítulos apresentam uma unidade teórica e
formam um bloco textual de análise dos fundamentos das repúblicas. Pelo
roteiro dos temas expostos, nota-se a presença de um itinerário
argumentativo cujo movimento vai do nascimento da cidade, passando pela
fundação dos ordenamentos políticos e o modo de defesa do \emph{libere
vivere} político, culminando na corrupção do povo, das instituições
(\emph{ordini}) e das leis. A religião, mobilizada no interior dessa
reflexão, também se apresenta como uma instituição capaz de conservar,
por meio de seus ritos, os valores e os ideais republicanos, ou seja,
ela cumpre o papel de \emph{instrumentum regni,} um instrumento para
governar. Além dessa descrição da vida política das repúblicas, na qual
se revelam as etapas de sua existência, pode-se também afirmar que esses
capítulos são uma introdução teórica aos \emph{Discursos}, uma vez que
definem os elementos essenciais na constituição de uma cidade.

Quando falamos em aspectos teóricos, não devemos ter em vista um certo
modelo de tratado em que os conceitos aparecem de modo destacado por
expressões próprias, como \emph{defino que}, \emph{demonstra-se},
\emph{entendo por} etc. Maquiavel utiliza-se de um outro estilo que não
é em nada menos indicativo de seu objetivo de definir conceitos. A
própria escolha e disposição dos temas é, por si, uma indicação de seus
propósitos teóricos. Assim, no capítulo \versal{I}, ao tratar da fundação das
cidades, ele faz, na verdade, uma descrição dos tipos de cidades que
podem existir e de como seu momento fundador pode ser determinante para
o desenvolvimento ou para a ruína futura, tipificando-as pelo seu modo
de fundação. No capítulo \versal{II}, faz, num primeiro momento, uma exposição
das formas de governo possíveis e de como nelas pode se processar a
mudança, indicando que o motor ou a causa desta não é uma certa lógica
determinista da natureza, mas dos conflitos políticos, tema dos
capítulos \versal{III} e \versal{IV}. Os capítulos \versal{I} e \versal{II} configuram-se, portanto, como
descrições tipológicas, seja do modo como pode se operar a fundação de
uma cidade, seja do modo como os regimes podem se instalar e se
transformar, descrições estas que fazem desses capítulos um preâmbulo
conceitual para a análise que se seguirá.

Na sequência, os capítulos de \versal{III} a \versal{X}, ao apresentarem temas como os
conflitos políticos, a defesa da liberdade política nas repúblicas, os
instrumentos de defesa e de acusação pública e a reforma ou refundação
das cidades, revelam como Maquiavel entende os elementos essenciais das
repúblicas. Como se verá, esses pontos configuram-se como as
\emph{ordini} ou os ordenamentos políticos básicos de uma república,
aspectos estes estruturais da vida de uma cidade ordenada como
república. Mesmo nos exemplos mobilizados, ele não se restringe ao caso
romano, mas fala de Veneza, Esparta, Florença, entre outras cidades,
numa clara indicação de que está apresentando as partes da vida política
em um regime republicano.

A exposição sobre a religião dos capítulos de \versal{XI} a \versal{XV} cumpre também essa
função na medida em que ela é considerada \emph{instrumentum regni}, uma
instituição cujo papel não se limita a ser de ordem religiosa porque é
decisiva na própria organização e no funcionamento da vida política da
cidade. Ao tratar da religião, o silêncio de Maquiavel em relação às
disputas medievais entre o papado e o império, fundamentais seja para o
futuro das cidades do norte da Itália, seja para o próprio período
medieval (\versal{SKINNER}, 2000, cap. 1-3), é indicativo de que a sua
preocupação não é especificamente a religião romana, mas sim analisar a
religião -- entendida eminentemente como prática social -- no que diz
respeito à sua relação com a vida política da cidade, ou seja, ao seu
papel político.

Quanto aos capítulos de \versal{XVI} a \versal{XVIII}, eles não são comentários à
corrupção romana, mas buscam entender como a corrupção pode atingir uma
república, seu povo ou suas instituições. Não se trata de uma explicação
do caso específico romano, mas da corrupção que pode acometer as
repúblicas de um modo geral.

Portanto, os temas mobilizados nesses capítulos iniciais e o modo como
eles são analisados, revelam o quanto Maquiavel fez, ao seu modo, uma
apresentação dos principais elementos políticos de uma república e de
seus fundamentos, já que não há uma análise exclusiva nem do caso
romano, nem dos livros da \emph{História} de Tito Lívio. Ora, é patente
que, nesse primeiro momento dos \emph{Discursos,} não se está fazendo
uma analise \emph{pari passu} do texto liviano, nem um comentário
histórico dos fatos de um modo geral. A coesão presente no objetivo e no
modo de exposição, a escolha dos temas e o itinerário que eles descrevem
-- do nascimento à corrupção da cidade --, confere um caráter unitário a
esse bloco textual. Os capítulos que se seguem ao \versal{XVIII}, nos quais
verifica-se, aí sim, um comentário à obra liviana, não denotam uma
inflexão teórica ou conceitual. A compreensão da \emph{História de Roma}
e o sentido ou o modo como essa análise deve ser feita seguirão os
critérios e as ideias apresentadas no início, mostrando uma imbricação
entre a análise dos fatos romanos à luz dos critérios apresentados. Os
comentários serão, portanto, pautados pelos conceitos e pelo modo de
compreendê-los, indicando uma profunda dependência entre as partes do
livro. Apesar de os dezoito capítulos se constituírem numa unidade
passível de ser analisada autonomamente, os \emph{Discursos} permanecem
um todo em sua estrutura, na medida em que seus capítulos dependem dos
critérios estabelecidos no ``\emph{Pequeno tratado sobre as
repúblicas}'' e este, por sua vez, formula seus fundamentos teóricos
tendo em vista a compreensão da vida em qualquer república, inclusive a
de seu tempo\footnote{As reuniões nos \emph{Orti Oricellai,}
  patrocinadas por jovens aristocratas de ideais republicanos, visavam
  também encontrar meios para restaurar o governo republicano em
  Florença (\versal{GILBERT}, 1977, 15-66).}, e não somente Roma, modelo de
república. Separar o ``\emph{Pequeno tratado}'' dos \emph{Discursos}
violaria, na perspectiva maquiaveliana, a \emph{verità effetuale}, a
comprovação no mundo real, recaindo num argumento feito para
``repúblicas ou cidade imaginadas''.

Segundo Chabod, os \emph{Discursos} constituem a ``origem espiritual''
de \emph{O Príncipe,} uma vez que pode-se perceber a existência
de uma relação direta entre os problemas que subjazem a ambos\footnote{Essa
  hipótese é sugerida primeiramente por Chabod, porém ele não a
  desenvolve. Gennaro Sasso, anos depois, será o primeiro a
  desenvolvê-la, tirando novas conclusões, como se verá nos capítulos
  finais dessa introdução (\versal{CHABOD}, 1993, p. 31-39).}. Ele propõe, não a
partir dos dados históricos sobre os textos, mas a partir da articulação
dos conceitos, que os \emph{Discursos}, ou parte deles, seriam os
pressupostos teóricos para \emph{O Príncipe}. Embora não tenha
desenvolvido essa hipótese, Chabod apontava para uma motivação de ordem
republicana mobilizando os argumentos de \emph{O Príncipe}, de
tal modo que esse deveria ser uma resposta ou continuação de algo
deixado para traz nos \emph{Discursos.}

Avançando nessa interessante sugestão e tendo em vista que há uma
exposição sobre as repúblicas no início dos \emph{Discursos}, nasce a
dificuldade de como se poderia articular essa parte inicial com \emph{O
Príncipe}. Uma hipótese seria verificar a sequência argumentativa quando
a república chega ao seu ápice de corrupção política, conforme exposto
no capítulo \versal{XVIII} do livro \versal{I} dos \emph{Discursos}.

Segundo Gennaro Sasso (1980, cap. \versal{V}), tendo Maquiavel escrito os dezoito
primeiros capítulos dos \emph{Discursos} e chegando ao ponto em que as
repúblicas estão completamente dominadas pela corrupção, em que a ruína
é um fato quase inevitável, a instauração de um principado civil passa a
ser o remédio adequado. Dito de outro modo, quando se verifica, no
capítulo \versal{XVIII} dos \emph{Discursos}, que é quase impossível a uma
república ``\emph{corrompidíssima}'' retomar o \emph{vivere civile}, as
liberdades civis, características dos regimes republicanos sadios, a
solução passa a ser a instauração de um regime fundado em um único
governante para que esse, com sua \emph{virtù}, consiga recuperar a
normalidade política da cidade e impedir a ruína certa. Ademais, ``o
principado representa o remédio que, auxiliado por extraordinária
\emph{virtù}, os legisladores que vêem longe procuram opor à corrupção
das repúblicas'' (\versal{SASSO}, 1980, cap. \versal{V}). Calcado naquilo
que é exposto pelos textos maquiavelianos, há a ``\emph{problemática
passagem}'', como diz Sasso, das repúblicas corrompidas para o regime
régio caracterizado pelo principado civil, na medida em que esse regime
pode oferecer uma resposta eficaz ao problema que se instaura nas
cidades corrompidas. O remédio é sugerido no próprio capítulo \versal{XVIII} dos
\emph{Discursos}, quando Maquiavel afirma que o freio para essa
corrupção total é a instauração de um governo régio, um regime que, com
a sua mão ``régia'', intervenha para reordenar a cidade. Ao final do
capítulo \versal{XVIII}, Maquiavel apresenta a ideia que será dominante no
\emph{Príncipe}, de tal modo que, da perspectiva de quem olha dos
\emph{Discursos}, a boa solução ou o remédio adequado não está nos seus
capítulos seguintes (\versal{XIX}, \versal{XX} etc.), mas no principado civil, tal qual é
apresentado nos capítulos \versal{VIII} e \versal{IX} de \emph{O Príncipe}. Do que podemos
concluir, conforme Sasso: ``os principados pressupõem a crise da
república, e não nascem senão quando essa está tomada pelas formas
extremas da corrupção, da degeneração'' (\versal{SASSO}, 1980, cap. \versal{V}). Com isso,
a origem de \emph{O Príncipe} não se fundaria numa visão
``idealizada''\footnote{A crítica à origem mítica ou idealizada de
  \emph{O Príncipe} é um dos objetivos de Sasso nessa reflexão,
  pois, para ele, carece de fundamento pensar a motivação de um livro
  apenas em pressupostos ideais. (\versal{SASSO}, 1980, p. 316, nota 41).} dos
regimes políticos, mas encontra sua motivação teórica no limite extremo
que se configura com a corrupção das repúblicas. Em uma cidade onde as
\emph{ordini} e as leis estão dominadas pela corrupção, a intervenção do
príncipe novo, tema dominante de todo \emph{O Príncipe}, faz-se
necessária, reformulando, ou melhor, refundando as ordens e as
instituições, reconciliando os humores, enfim, tudo aquilo que também é
preconizado ao longo dos \emph{Discursos}. Desse ângulo, \emph{O
Príncipe} seria tributário do raciocínio desenvolvido no
``\emph{Pequeno tratado sobre as repúblicas}'', pois teria nesse sua
maior motivação teórica. Por outro lado, os \emph{Discursos} manteriam
uma relação de dependência teórica com \emph{O Príncipe}, visto
que a melhor solução para o problema no qual culmina o raciocínio seria
o principado civil. Essa interdependência teórica revela uma estreita
linha de continuidade no interior da reflexão política maquiaveliana,
como afirma o comentador:

\begin{quote}
Precedendo cronologicamente ou seguindo a composição do \emph{Príncipe},
o décimo oitavo capítulo do primeiro livro dos \emph{Discursos} é,
portanto, o ``lugar ideal'' no qual o conceito daquele livro se realiza
nos seus modos próprios (\versal{SASSO}, 1980, p. 327).
\end{quote}

Portanto, cabe agora tentar entender melhor essa passagem da república
corrompida ao principado, ou por outro lado, o que é esse principado e o
seu príncipe, a partir desse pressuposto lançado pela corrupção
republicana.

\section{O Príncipe Civil}

\subsection{Os pressupostos para o poder régio}

Como dito, a sequência argumentativa do \emph{Pequeno tratado sobre as
repúblicas} encaminha-se, a partir do capítulo 16, para o tema da
corrupção na cidade. Primeiro, Maquiavel trata da corrupção do povo ou
da matéria da cidade (cap. \versal{XVI}) e, em seguida, da corrupção dos
ordenamentos republicanos ou da forma (cap. \versal{XVII}). Prosseguindo nessa
escalada da corrupção política, o cap. \versal{XVIII} vai direto ao grau máximo
de corrupção na cidade, quando ela se torna \emph{corrompidíssima}
(sic). Neste estágio de corrupção ampla, as consequências são ou a
mudança de regime ou a dissolução da república como entidade política
dotada de liberdade. Seja como for, qualquer uma das consequências é
contrária à vida política republicana, ao \emph{vivere libero,} visto
que uma condição essencial da vida republicana é a liberdade, ao ponto
do regime republicano ser nomeado, às vezes, como o regime da liberdade.
Na verdade, chegamos ao grande problema enunciado pelo título do
capítulo \versal{XVIII}: ``\emph{De que modo nas cidades corrompidas se podem
conservar um Estado livre, sendo-o; ou, não o sendo, ordená-lo}''
(\emph{Discursos}, \versal{I}, 18, linha 1). A questão está em tentar pensar em
uma solução para aqueles casos nos quais a corrupção não está apenas
localizada numa parte do corpo político ou permanece restrita à matéria
ou à forma, mas quando se encontra disseminada por toda a cidade. Uma
resposta já nos é possível constatar, pois não se pode conservar o
\emph{vivere libero} em condições de extrema corrupção, nas quais o povo
já não mantém a civilidade, e em que as leis são inadequadas e os
ordenamentos não conseguem mais frear as ambições desmedidas dos
diversos grupos políticos. As condições de possibilidades para a
retomada da liberdade republicana já não figuram mais no horizonte.
Diante, então, dessa condição extrema, a possibilidade de retorno, de
uma retomada à normalidade republicana é uma impossibilidade dentro da
lógica de ação política da república, pois, com uma matéria corrompida,
as leis são inadequadas e os ordenamentos políticos ineficazes e,
conforme o grau de corrupção, corrompidos em suas deliberações.
Maquiavel é categórico: De tudo o que dissemos acima provém a
dificuldade ou a impossibilidade de nas cidades corrompidas conservá-las
como republicas ou criá-las de novo (\emph{Discursos}, \versal{I}, 18, linha 28).

Neste contexto, pode-se até perguntar se ainda há ou não liberdade, ou
melhor, se o \emph{vivere libero}, característico da república, ainda
persiste ou se alguma força autoritária teria tomado as rédeas das
decisões políticas. Uma das características dessa corrupção republicana,
talvez a preponderante, está na força política que a aristocracia assume
e como ela passa a deliberar conforme os seus desejos. Pensando numa
cidade em tais condições políticas, mas não somente isso, sendo o povo
impedido de lutar pelos seus direitos, tal quadro é uma descrição de um
caso de corrupção republicana típico. Nestas circunstâncias, extingue-se
a liberdade de uma parte do corpo político, extingue-se a luta política
e um só grupo passa a ditar o caminho. No entanto, a corrupção também
pode extrapolar um grupo político restrito e atingir a todos
(\emph{universale}), circunstância esta caracterizada, entre outros
aspectos, pela perda dos valores cívicos, da civilidade. Também neste
caso não há mais espaço para a luta política, para o \emph{vivere
libero}.

Logo, não importa em que condição se manifeste a corrupção, ela figura
sempre como uma oposição à liberdade, ou como diz Sasso, ``\emph{a
recíproca repugnância entre liberdade e corrupção}'' (\versal{SASSO}, 1987, p.
407). Esta imagem ilustra os termos da dificuldade, pois a vida política
republicana é avessa à corrupção, é o pólo contrário à condição política
corrompida de uma cidade. De fato, se há uma manifestação de corrupção
política, isso implica proporcionalmente na anulação da liberdade; ou,
conforme a corrupção se amplia, por uma proporção inversa, diminui o
grau de liberdade da cidade. O que não quer dizer que a corrupção seja o
antônimo de liberdade, pois, conforme o nível de corrupção, tem-se uma
gradação inversa de liberdade: quando o grau de corrupção da cidade é
baixo, é possível que exista ainda o \emph{vivere libero}. Porém, em
qualquer condição em que haja um aumento de um, automaticamente ocorre o
decréscimo do outro, pois a coexistência de ambos com mesma intensidade
é impossível. Repugnância que não diz respeito apenas à liberdade, mas
pode estender-se à civilidade (entendida como o respeito mínimo às
regras políticas), às regras cívicas, quando se considera a corrupção da
matéria. Ou como dirá Maquiavel no capítulo \versal{LV} desse livro \versal{I} dos
\emph{Discursos}, refletindo acerca da corrupção presente quando os
\emph{gentis-homens} dominam o poder: ``Do que nasce que naquelas
províncias não surja nunca alguma república nem algum \emph{vivere
politico}; porque tal geração de homens são em tudo inimigos de toda
civilidade'' (\emph{Discursos}, \versal{I}, 55, linha 21). A corrupção se
opõe, pois, à república, mas, mais ainda, à civilidade e até mesmo ao
\emph{vivere politico}, de modo geral. Esta afirmação amplia o problema,
pois a corrupção não é somente contrária à liberdade, mas contrária à
vida política, um adversário à normalidade política. Opondo-se a uma
consideração que banaliza o papel que pode chegar a desempenhar a
corrupção no corpo político, Maquiavel confere cores fortes e afirmações
contundentes para descrever a importância das suas consequências para a
vida política da cidade. A manifestação da corrupção não deve ser
tratada como mais um evento possível em uma cidade, mas um grande
problema, um grande perigo para o corpo político como um todo. A
corrupção não é mais uma dificuldade presente no cotidiano político das
repúblicas, mas se torna o problema, a questão a ser tratada.

Todavia, quando esse problema não é passível de solução pelos próprios
mecanismos políticos da república, então, deve-se buscar meios mais
fortes e eficazes para freá-la. Entre as soluções, está a instalação de
um \emph{poder quase régio} ou do \emph{poder régio}.

A cidade diante, pois, de um caso de extrema corrupção, deve mudar o seu
regime, tendo, a princípio, duas possibilidades: o governo \emph{régio}
ou o governo \emph{popular}. Maquiavel reiteradamente identifica na
ambição desmedida da aristocracia a principal causa de corrupção. A
corrupção do povo, quando nasce, é um aspecto secundário, sendo muito
mais fruto da falta de freios à insolência dos grandes do que da perda
de civilidade do povo. Ora, a parcela popular da cidade estaria
habilitada, pelas suas qualidades, para, em tese, assumir o comando da
cidade nas condições de corrupção extrema, desde que não tivesse perdido
também todos os seus valores cívicos. Contudo, o problema não é assumir
o controle da cidade em função da sua capacidade ou por não estar tão
corrompida, a questão que se põe é se esse governo popular seria capaz
de colocar um fim à corrupção endêmica e reordenar a cidade.

Em função da grande insolência que, em geral, assola a cidade
corrompidíssima, a solução dada por Maquiavel não é nem sua conversão
num governo popular e nem num monárquico, mas em algo intermediário: no
poder quase régio. Diz ele:

\begin{quote}
Mas, em se precisando criar ou conservar uma {[}república{]}, seria
necessário, antes, reduzi-la ao estado régio do que ao estado popular;
para que os homens insolentes, que não pudessem ser corrigidos pelas
leis, fossem de algum modo freados pela autoridade quase régia
(\emph{Discursos}, \versal{I}, 18, linha 29).
\end{quote}

A solução pelo governo \emph{quase régio} é, na verdade, a justificação
para a instalação de um ordenamento republicano de Roma: a ditadura. Na
república romana, a figura de um ditador, que concentrava poderes
extraordinários durante um período limitado de tempo, era uma solução
prevista para casos especiais, como guerras e revoltas civis. O ditador
romano era um magistrado especial, escolhido pelo senado com função
específica para realizar alguma missão extraordinária. Com a instalação
do ditador pelo Senado romano, cessariam automaticamente os poderes dos
cônsules e dos outros magistrados, que passavam a subordinar-se ao
ditador (\versal{NICOLET}, 1964; \versal{CIZEK}, 1990). Este ditador romano difere em
muito, contudo, da imagem dos ditadores contemporâneos, pois sua
instalação e sua ação eram reguladas e submetidas à fiscalização e ao
controle do Senado romano, ou seja, ele não teria poderes políticos e
jurídicos absolutos.

Ora, quando Maquiavel pensa num governo \emph{quase régio}, dotado de
poderes extraordinários, ele tem em vista tanto a instalação da ditadura
quanto de um principado nos moldes romanos. No capítulo \versal{XXXIV} do livro
\versal{I}, ele diz:

\begin{quote}
Alguns escritores condenaram os romanos que encontraram um modo de
instituir a ditadura, como algo que, com o tempo, deu ensejo à tirania
em Roma. {[}\ldots{}{]} E vê-se que o ditador, enquanto foi designado segundo
os ordenamentos públicos, e não por autoridade própria, sempre fez bem à
cidade. Pois o que prejudica as repúblicas é fazer magistrados e dar
autoridade por vias extraordinárias, e não a autoridade que se dá por
vias ordinárias: e vê-se que em Roma, durante tanto tempo, nunca ditador
algum fez nada que não fosse o bem à república (\emph{Discursos}, \versal{I},
34, linhas 2; 5-6).
\end{quote}

O problema que pode advir a esses governos com poderes extraordinários
está no modo como nascem. Caso sua autoridade tenha sido delegada por
via ordinária, ou seja, dentro das regras políticas da república, sem
uma exacerbação de força por meio da violência, então não há nenhum
problema maior e os efeitos serão bons. A preocupação de Maquiavel
reside, fundamentalmente, no modo como ocorre a instalação desse
governo, no caso, por um meio não violento, respeitando a dinâmica
política republicana. Por se originar em tal quadro, o ditador detinha
um poder extraordinário, porém limitado, o que era uma garantia de, ao
final de seu mandato, o retorno à normalidade republicana:

\begin{quote}
De modo que, somando-se o breve tempo de sua ditadura, a autoridade
limitada que ele tinha e o fato de o povo romano não ser corrompido, era
impossível que ele saísse de seus limites e prejudicasse a cidade: e
pela experiência se vê que sempre foi proveitoso (\emph{Discursos}, \versal{I},
34, linha 10).
\end{quote}

Uma primeira solução para a república corrompida é a utilização de um
mecanismo republicano, o ditador, que concentra o poder para que possa
dar conta de um problema extraordinário, que, pelas vias ordinárias
republicanas, não poderiam ser sanado. Como sugere Bausi, além desse
ditador ao estilo romano, Maquiavel também tinha em mente como exemplo
desse poder quase régio os \emph{gonfalonieri} florentinos, que foram
governantes com poderes centralizados, mas em repúblicas\footnote{De
  fato, é possível fazer várias aproximações entre as funções e encargos
  dos ditadores romanos e as atribuições iniciais do \emph{gonfaloniere}
  Solderini, em 1494. Contudo, depois da reforma política de 1502 que
  institui o \emph{gonfaloniere a vita}, ou seja, perpétuo, convém
  associá-lo mais ao \emph{princeps rei publicae} do que ao ditador
  romano (\versal{BAUSI}, 2002, p. 117, nota 39).}. Esta solução é sugerida em
outras passagens\footnote{Cf, Livro \versal{I}, \versal{II}, 33; \versal{XXXIV}, 20; Livro \versal{III},
  \versal{XXVIII}, 14}, sinalizando um momento intermediário que, uma vez
fracassado, não deixaria escolha senão a instalação de um regime com um
governante com poderes políticos absolutos, uma autocracia de fato. A
vantagem dessa solução intermediária é que ela garante uma exigência
fundamental para a república corrompida, já que instala um governo de
força sob a égide do regime republicano. Sem abolir totalmente os
valores cívicos do republicanismo, o ditador ou o \emph{gonfaloniere},
por seu caráter extraordinário e temporário, visto que tinha mandatos
definidos que poderiam ou não ser renovados, seria um governo forte em
regimes republicanos enfraquecidos pela corrupção com vistas
exclusivamente à reordenação da cidade, o que por si só é um risco, haja
vista que não se tem a certeza de que eles serão bem sucedidos. De
qualquer modo, a condição extraordinária da corrupção -- pois ela é, no
limite, ruptura da vida política ordinária -- exige uma solução também
extraordinária, que ultrapasse alguns aspectos da normalidade
republicana, a fim de que se restaure a ordem. Os ditadores ou os
\emph{gonfalonieri} são medidas extraordinárias para circunstâncias
políticas extraordinárias. Como diz Maquiavel:

\begin{quote}
Quanto a inovar tais ordenamentos de uma só vez, quando todos reconhecem
que não são boas, digo que essa inutilidade, quando facilmente
reconhecível, é difícil corrigi-la; porque, para tanto, não basta usar
medidas ordinárias, visto que os modos ordinários são ruins; mas é
necessário recorrer ao extraordinário, como a violência e as armas,
tornando-se, antes de mais nada, príncipe em tal cidade, para poder
dispô-la a seu modo (\emph{Discursos}, \versal{I}, 18, linha 26).
\end{quote}

Ou ainda, como diz ao final do capítulo \versal{XVII}:

\begin{quote}
Porque tal corrupção e pouca aptidão à vida livre nascem de uma
desigualdade existente na cidade, e quem quiser dar-lhe igualdade
precisará lançar mão de meios extraordinários {[}grandissimi
straordinari{]}, o que poucos sabem ou desejam fazer (\emph{Discursos},
\versal{I}, 17, linha 16).
\end{quote}

Portanto, mesmo tendo à disposição esse meio extraordinário de reforma,
legítimo e previsto dentro do regime republicano, tal solução, apesar de
possível, não parece ser, contudo, a mais adequada para a cidade
corrompidíssima. Uma outra hipótese é a instalação de um governo que,
apesar de centralizar a autoridade em um indivíduo, consiga conservar um
mínimo de civilidade ou até mesmo recuperar a dinâmica republicana.
Governo esse que pode ser compreendido como um certo tipo de principado,
e não todo e qualquer principado, no caso específico, o principado
civil.

\subsection{Sobre a noção de principado em Maquiavel}

A crise de corrupção das repúblicas nos leva, pois, aos governos de
força, sejam eles régios ou quase régios. Uma das sugestões apontadas
por Maquiavel é o principado e, seguindo esse viés interpretativo,
cumpre pensar o principado a partir desses pressupostos dados pela
corrupção republicana. Esse é um ponto nodal de nossa leitura: como
pensar o principado e o príncipe expostos por Maquiavel em \emph{O
Príncipe} a partir desses pressupostos fornecidos pela corrupção
republicana?

Tendo em vista esse pressuposto interpretativo, algumas questões
formuladas de início tornam à baila: quem é esse príncipe -- que na
definição do texto é antes de tudo um \emph{privato ciptadino} -- que
assume o papel de liderar e conduzir a cidade? Seria ele um típico
monarca ou uma figura política diferente? Se o príncipe não é o monarca,
então, como entender o principado? Seria este o território ou lugar da
ação política do príncipe, identificando-se aquilo que nós entendemos em
nossos dias como principado ou reino? Ou seria ele outra coisa?

Antes mesmo de partir para a busca da resposta sobre o personagem
político que é o príncipe, faz-se necessário indagar antes o que
Maquiavel entende por principado e como essa noção é mobilizada na
reflexão desenvolvida em \emph{O Príncipe}. Convém relembrar,
inicialmente, que o título original da obra era \emph{De Principatibus},
ou seja, \emph{Sobre os Principados}, restando claro que seu autor
pretendia dissertar sobre esse tema ao longo do livro e não sobre a
figura do príncipe em primeiro plano, como o título atual sugere. Em
vista disso tudo, faremos uma explanação em duas etapas: uma primeira
sobre as noções de monarquia e principado que chegam ao contexto do
Renascimento florentino, no qual esse uso vocabular está inserido, ou
seja, uma apresentação do problema em seu contexto discursivo e as
nuances terminológicas que alguns termos possuíam naquele momento. Uma
segunda investigação, buscando extrair do próprio texto maquiaveliano o
que se compreende por principado, a partir dessa noção de príncipe civil
que nos parece ser central.

\subsection{O principado no momento maquiaveliano}

A denominação dos cargos dirigentes e dos detentores desses na Europa do
século \versal{XV} deve ser considerada, inicialmente, nos contextos políticos
particulares de cada território. Como o próprio Maquiavel aplicará ao
longo d'\emph{O Príncipe}, cada lugar possuía suas peculiaridades
em termos de organização política, de modo que unificar ou tentar
generalizar terminologias e designações pode implicar em equívocos
sérios. Por exemplo, nos territórios que hoje conhecemos como a
Alemanha, havia vários príncipes e um imperador, sendo que esses
príncipes não eram filhos desse imperador ou seus sucessores diretos. O
mesmo se diga para o líder político do ainda recente, para os latinos,
governo turco, designado por Maquiavel tão somente como ``O Turco''. Em
geral, assumia-se que o imperador era o governante de um império que
possui vários reinos ou principados. Mas mesmo neste caso, a denominação
é controversa, principalmente entre os regimes que reivindicavam a
herança do Império Romano, pois, tendo em vista a presença até 1453 do
Império Romano do Oriente, também conhecido como o Império bizantino,
todos os demais postulantes latinos à condição de Império disputavam com
Constantinopla esse reconhecimento, incluindo, nesse caso, até o Papado,
que após o século \versal{VIII}, em vários documentos, reivindicava a sua
condição de legítimo herdeiro do Império Romano cristianizado por
Constantino em 313, particularmente pelo documento falsificado
denominado ``Doação de Constantino''. Situação que se complica mais após
o século \versal{VIII}, com o surgimento do Sacro Império Romano, da dinastia
carolíngia, que se transforma posteriormente no Sacro Império Romano
Germânico, dominado pelos imperadores alemães. Historicamente, os
governantes dos impérios oriental e ocidental nunca reconheceram de fato
essas condições por inúmeros motivos, que não vem ao caso dissertar
aqui\footnote{Cf. Ostrogorsk, Georg. \emph{Storia dell'impero
  bizantino.} Torino: Einaudi, 1968. Esse longo estudo sobre o Estado
  bizantino mostra, a partir da ótica dos orientais, como as
  reivindicações dos latinos nunca foram plenamente aceitas por eles,
  donde a disputa constante pela herança do Império Romano durou
  séculos.}. Quadro esse que se amplifica com os novos reinos latinos de
Espanha e Portugal no século \versal{XV}, que se reivindicam também como
impérios. No início do século \versal{XVI}, portanto, um império já não era mais
entendido apenas como um sucessor direto do Império Romano ou como poder
político que se coloca acima dos reinos que governa, passando a ter uma
acepção polissêmica. A clássica designação de que o imperador está acima
do rei já não cabe para certos casos e não possui uma significação
precisa em alguns contextos políticos.

Tal problema de denominação pode ser extrapolado para os termos ``rei''
e ``príncipe'', que passam a ser ter várias acepções, embora, em geral,
verifique-se que o termo ``rei'' se aplica ao governante do reino. Tudo
isso sem levar em conta o sentido jurídico dessas denominações, o que
nos remeteria, por seu turno, aos debates entre os glossadores
medievais.

Tomando-se em conta o contexto histórico italiano do Renascimento,
particularmente, as lutas por autonomia das cidades do norte da
península itálica e a consequente implantação dos regimes republicanos
em várias delas, o uso desses vocábulos políticos ganha novos contornos.
Tendo em vista as guerras travadas pela busca de autonomia dessas
cidades contra as forças imperiais e papais, e depois as próprias lutas
internas contra aqueles que buscavam o domínio da cidade, contra os
\emph{Signori}, a denominação dos cargos políticos apresenta uma
variedade significativa. Ora, no contexto das repúblicas italianas do
Renascimento, o fato histórico da luta pela liberdade contra as tiranias
-- interna ou externa -- fez com que os usos do termo ``príncipe'' sejam
esvaziados dessa ligação ao regime monárquico, quando se refere a
personagens políticos deste contexto.

Isso fica claro, no caso de Florença, quando se analisam os escritos dos
autores políticos, particularmente dos homens da Chancelaria, como
Salutati, Bruni, Valla, Maquiavel, Guicciardini, Vettori, Buonaccorsi
etc. A designação ``príncipe'', aplicada a personagens italianos,
raramente se refere a pessoas ligadas a uma dinastia e trata-se, em
geral, de próceres políticos com cargos executivos. De tal modo que o
termo se aproxima mais da concepção de \emph{prínceps} latino, como o
primeiro entre os iguais, do que o herdeiro de uma dinastia.

Se para o termo ``príncipe'' é possível delimitar esse uso entre os
escritores florentinos, já para o termo ``principado'' parece que temos
ainda uma acepção muito mais próxima de monarquia, do que decorre a sua
definição como uma espécie de regime monárquico. De fato, o termo é
polissêmico e permite essa interpretação, que se reforça quando se leva
em conta os governos da família Medici ao longo do século \versal{XV},
caracterizada como um principado e que, para muitos, se tratava de fato
e de direito de um governo de tipo monárquico na cidade. O que nos
obriga, ao menos, em considerar melhor o que foi esse governo para
entender os possíveis sentidos dessa denominação de principado na
Florença do contexto maquiaveliano.

Como demonstrou Rubinstein (1997) e o próprio Maquiavel na
\emph{História de Florença}, após a revolta popular, também conhecida
como Revolta dos Ciompi, desencadeada em 20 de julho de 1378, uma série
de mudanças no ordenamento político da república florentina são
realizadas. A criação de novos conselhos e o modo como os grupos
poderiam ter acesso a eles foram uma das demandas que se transformaram
em possibilidade política legítima. Contudo, já em início do século, mas
principalmente a partir do governo de Cósimo de Medici, em 1434, o que
se verifica é um aumento da influência política dos setores
aristocráticos mais abastados sobre os demais grupos políticos\footnote{Conforme
  já explicamos no início, a simplificada distinção social em Florença
  entre ricos e pobre (\emph{popolo grasso} e \emph{popolo minuto}) não
  é correta, pois, segundo o próprio Maquiavel, haviam sub- divisões
  entre eles, tornando o equilíbrio de forças políticas mais complicado.}.
O governo sobre o controle da família Medici durante o século \versal{XV} era
formalmente um governo republicano, contudo, era dirigido conforme os
interesses dessa família, que não apenas possuía o cargo de comando, mas
também controlava o acesso aos demais conselhos. Tanto é assim que,
quando Piero di Medici perde o governo da cidade em dezembro de 1494, a
razão maior para isso está no fato dele ter rompido o delicado
equilíbrio de poder que havia entre os diversos setores da aristocracia
(Rubinstein, 2011). E mesmo a volta dessa família ao poder em 1512 tem
como fundamento a reconstrução dos apoios que ela possuía entre a
aristocracia que detinha o controle dos conselhos superiores da cidade.
Ora, em ambos os exemplos históricos, não é possível afirmar que se
tratava de um governo de tipo monárquico, muito menos de um governo
autocrático com outra denominação. De fato, como demonstra Rubinstein
(1997), os Medici dividiam o poder político na cidade e não concentravam
tudo em suas mãos. O fato de exercerem o controle sobre os conselhos não
nos permite dizer que esse fosse um governo autocrático. Nomear, pois,
tais regimes de principados, a partir do exemplo histórico, não pode
implicar uma identificação desses regimes sob o controle da família
Medici como monarquias.

\subsection{O principado de \emph{O Príncipe}}

Para além desses dados históricos, que não resolvem o problema, uma
análise do texto de \emph{O Príncipe} fornece os elementos mais
corroboradores, pois nos apresenta o que de fato Maquiavel entende por
principado.

Logo no início da obra, encontramos uma primeira definição de
principado: \emph{``Todos os estados, todos os domínios que tiveram e
têm autoridade sobre os homens, foram e são ou repúblicas ou
principados.''} {[}cap. 1, linha 1{]}

A definição inicial já nos revela algo, pois o principado é uma
autoridade (império), ou seja, um governo que se exerce sobre os homens.
Em seguida, ele distingue o principado da república, diferenciação esta
que ocorrerá outras vezes, não somente \emph{n'O Príncipe} em outros
momentos, mas nas suas demais obras.

Contudo, se o principado realmente for um domínio sobre os homens,
poder-se-ia declarar que se trata de um governo que exerce sua força
sobre os homens, submetendo-os. Tal constatação é dedutível a partir da
noção de domínio herdada da antiguidade e reelaborada durante o período
medieval, que, grosso modo, é a transferência da relação existente no
interior da casa (\emph{domus}), entre senhor e escravo, para a esfera
pública. Neste caso, o principado é uma dominação política, no sentido
de um governo que não está aberto à interferência e não divide seu
controle ou primazia nas decisões. Entretanto, tais afirmações desse
modo ainda são precipitadas, pois o conceito teve apenas sua primeira
apresentação na obra, falta considerar o restante da primeira parte do
livro.

Na sequência ainda desse capítulo \versal{I}, Maquiavel distingue os principados
em hereditários e novos. Nesse momento, teríamos a primeira aproximação
dos principados com o regime monárquico, pois o principado hereditário
seria aquele no qual o controle está nas mãos de uma mesma família, ou
como diz, ``\emph{nos quais o poder ficou por longo tempo com a família
do príncipe}'' {[}cap. 1, linha 2{]}. Todavia, ainda que essa tradução
seja adequada, uma consideração sobre o texto italiano nos permite
verificar que Maquiavel está falando daqueles governos em que os membros
de uma dinastia foram por longo tempo príncipes (\emph{sia suto lungo
tempo principe}), o que admite também o entendimento de que essa família
liderou a cidade durante este ``longo tempo'', o que é um pouco
diferente do que afirmar que este principado é uma monarquia
hereditária. A distinção estaria na nuance entre liderar politicamente e
ter o domínio político, a autocracia das decisões políticas. Ora, como o
próprio autor não se alongará nessa análise, é temerário tecer hipóteses
sobre um tema não desenvolvido pelo filósofo. Atendo-se à letra do
texto, mesmo nesse principado hereditário, Maquiavel parece destacar a
capacidade de liderança política desta dinastia e não a sua condição
monárquica.

O outro principado em tela, o principado novo, se constituirá, na
verdade, no tema principal do livro, pois, seja nas considerações sobre
o principado, seja nas considerações sobre o príncipe, Maquiavel volta
suas atenções para o principado novo, em suas várias manifestações, e
para o príncipe novo e sua necessidade de conservação do governo.

Neste primeiro momento do livro, sua preocupação está dirigida para a
conquista do principado. O termo ``conquista'' aqui pode ter uma dupla
acepção, em função daquilo que se entende como principado: se principado
for compreendido como um território, então essa conquista é tal qual uma
ocupação fruto de uma campanha militar, por exemplo; nesse sentido, uma
conquista de um local ou território. Mas, haveria ainda a possibilidade
da conquista se referir somente ao controle político do governo, à
esfera política da cidade prioritariamente. Neste caso, trata-se de
ressaltar a dimensão política da cidade, e não seu aspecto territorial,
talvez mais próxima da definição antiga, encontrada, por exemplo, em
Aristóteles (\emph{Política}, \versal{III}, cap. 1-2), no qual a definição de
cidade (\emph{polis}) não se identifica a um território, de tal modo que
a cidadania não é assegurada a quem habita aquele território. O exemplo
do caso da Alemanha citado no capítulo 10 d'\emph{O Príncipe} pode ser
bem ilustrativo disso, pois o rei tem sua esfera de ação política, mas
não detém o controle dos territórios governados pelos príncipes. Assim,
um caso de controle político não implica necessariamente, embora muitas
vezes seja o caso, controle de um território, fornecendo o escopo dessa
dificuldade de compreensão da conquista do principado.

Porém, uma nova ordem de problemas se articula a esses, na verdade, um
problema anterior e mais profundo, que diz respeito ao que entender por
conquista. Talvez aqui esteja o cerne político do problema, pois a
dificuldade está em determinar qual a relação política que esta tomada
do governo estabelece: se é uma relação política na qual o conquistador
domina toda a esfera de comando, exercendo seu poder com domínio
soberano, ou se esta relação implica em um postar-se constantemente na
disputa pelos apoios políticos que permitem o exercício do governo. O
desenvolvimento do texto ater-se-á justamente a esse ponto nevrálgico da
tomada do governo, mais do que ficar listando tipificações dos modos de
se adquirir um principado. É na determinação das relações políticas que
tal investida causa que Maquiavel fixa suas atenções. Logo, não se
atinge os elementos centrais do texto quem se concentra somente na
análise dos tipos de conquista, conforme dá a entender o texto num
primeiro momento, e se esquece de atentar para o modo como as relações
políticas estão sendo construídas, relações estas que formarão os
alicerces desse novo governo.

Desviando um pouco a atenção para a exposição do argumento, é importante
que logo de início o leitor perceba uma característica do estilo de
exposição maquiaveliano. Por ser Maquiavel um escritor hábil, sua função
principal durante todo o período de trabalho na Chancelaria foi escrever
relatos, habilidade esta que ele teve que aprimorar, pois não somente a
clareza deveria ser uma marca dos seus textos, mas, por outro lado,
quando se fazia necessário, convinha produzir um relato que dirigisse os
seus leitores a tomar a decisão que ele entendia a mais adequada. Donde
a necessidade de se usar recursos retóricos, mas com tal sutileza que o
leitor não percebesse essa sub-intenção. Ora, um profissional exercitado
por longos anos nesta arte da escrita e que era conhecido por ser também
teatrólogo -- não nos esqueçamos que em vida Maquiavel foi reconhecido
em Florença mais como o autor da peça \emph{Mandragora} do que como
personagem político --, elabora um texto que deve ser lido com atenção e
levando em conta as diversas dimensões da linguagem e seus efeitos.
(\versal{ADVERSE}, 2009)

Retomando o roteiro argumentativo inicial, o livro se abre, então, com
uma rápida definição de principado, e não de príncipe, o que já é muito
significativo, e passa para sua tipologia de conquistas: pelas armas,
próprias e alheias, pela \emph{virtù} e pela fortuna. Encerrado essa
breve apresentação, que é, na verdade, um roteiro dos temas a serem
tratados ao longo da primeira parte do livro, o parágrafo inicial do
capítulo \versal{II} é também emblemático.

A frase de abertura do capítulo \versal{II} já provocou inúmeras discussões entre
os comentadores, pois diz Maquiavel: ``\emph{Deixarei de lado a
discussão sobre as repúblicas, porque, em outro momento dissertei
longamente sobre elas}'' (cap. 2, linha 1). Não vamos retomar aqui a
discussão relativa a qual livro ele estava se referindo e se esse livro,
em geral entendido como sendo os \emph{Discursos}, foi ou não escrito
antes d'\emph{O Príncipe}, conforme já abordamos. Apenas gostaríamos de
chamar a atenção para o fato de que num livro dedicado aos principados,
em dois momentos iniciais, mais exatamente num espaço de cinco linhas,
Maquiavel faz duas referências diretas à república. Uma primeira
constatação é óbvia, o principado se coloca, já em seu primeiro momento,
como um contraponto à república, o que é inegável. Com efeito,
parece-nos evidente que o principado é um regime diferente da república.
A questão é: por que essa insistência em contrapô-los? Talvez ainda o
texto não nos permita apontar uma resposta suficiente para essa
dificuldade, todavia, a segunda linha do capítulo fala do ``\emph{sangue
do senhor que é por longo tempo príncipe}''. Ora, tal ideia poderia ter
uma outra forma de redação se se tratasse tão somente de monarquias.
Neste caso, bastava dizer que aquela dinastia ou aquela família real
detém o comando político há muito tempo, conforme já tratamos.
Voltando-se para mobilização de república aqui, verifica-se, por apenas
esses elementos, que Maquiavel percebe um paralelo entre principado e
república que exige uma distinção. A diferenciação será apresentada
adiante, porém fica a dúvida acerca do porquê do paralelo. Esse paralelo
ou proximidade entre a república e o principado ficará mais claro quando
da análise do principado civil, mas, pelo exposto, podemos antecipar com
segurança que esse o principado terá na sua estruturação política e na
dinâmica das suas ações algumas semelhanças com as repúblicas,
particularmente no que diz respeito ao lugar dos conflitos políticos, e
se diferenciará, por outro lado, das monarquias ou governos
autocráticos.

Na sequência, Maquiavel apresenta de fato a sua preocupação maior, que
perpassará toda a obra: ``Ocupar-me-ei somente do principado, retecerei
as urdiduras acima descritas, e demonstrarei como estes principados
podem ser governados e conservados'' (cap. 2, linha 2). Na verdade, a
preocupação são duas: o modo de governar os principados e como eles são
conservados, mas que em vários momentos se confundem, pois, no modo de
governar, já devem estar implícitas as estratégias de conservação desse
próprio governo. Agora temos, então, um panorama geral do que será essa
obra: uma análise da conquista dos principados, de como governá-los e
conservá- los. Basicamente sobre esse tripé é que se desdobrarão os
demais temas.

A exposição propriamente dita dos principados começa pelas atenções ao
principado hereditário e aos mistos. Sobre os hereditários, como já
dito, a brevidade da exposição é o dado mais chamativo. Nas poucas
quatro linhas dedicadas ao tema, ele deixa como regra geral que esse
príncipe herdeiro não precisa usar de meios extraordinários para
conservar o seu governo, visto que não teve que conquistá-lo. Ora, basta
a esse que herdou o governo do principado ter uma ``indústria
ordinária'', ou seja, não realizar grandes inovações políticas, seguir o
curso ordinário das coisas. Isso pode soar como um convite a uma conduta
medíocre, como aquele que pauta suas ações pela mediana geral dos
governantes, sem grandes iniciativas políticas. Se for isso, nada mais
disforme ao que será exposto, pois, se é possível dizer algo do
príncipe, é que ele deve ultrapassar o plano do ordinário, da mediania
em termos de ações políticas. Talvez seja então até por isso, por essa
mediocridade inerente ao príncipe herdeiro, que se justifique o fato de
Maquiavel não dedicar maior atenção a ele.

O passo seguinte é uma análise dos principados que são conquistados por
alguém que já detém o comando de um outro principado. Esse novo governo
é, para o conquistador, um principado. Detalhe sutil, mas que indica
muito, pois alguém que já comanda um principado e conquista outro, em
tese, não terá grandes dificuldades na direção deste novo governo, pois
já conhece os modos como governar uma cidade e, portanto, se conservam
esses governos. Entretanto, e novamente vemos o estilo maquiaveliano
desconcertar o leitor, não parece ser tão fácil assim para esse
conquistador conservar o governo dessa sua conquista política. A
dificuldade principal desse novo principado indica o grande problema de
toda a conquista política, até mesmo para alguém que já governa, a
saber: como obter apoios na cidade ocupada?

O problema apresentado no início do capítulo \versal{III} revela a preocupação
política que subjaz a conquista: o governo principesco não se vale por
si só, mas pela capacidade de angariar apoios que o sustentem, aquilo
que contemporaneamente chamamos de legitimidade. Isso fica claro quando
Maquiavel diz que, ``\emph{mesmo que se tenha um fortíssimo exército
seu, sempre se precisa da ajuda dos provincianos para entrar em uma
província}'' (cap. 3, linha 3). Portanto, esse conquistador nunca é tão
poderoso, mesmo quando se vale de um forte exército. Como ele
demonstrará por diversos exemplos, esse conquistador precisa fazer uma
série de ações no intuito de trazer para si apoios políticos que lhe
permitam, de fato, se constituir como um líder político daquela cidade,
o que não se consegue apenas tendo um exército forte, apenas pela força
das armas. Pode-se afirmar, olhado por outro lado, que a força do
governo não está no poderio militar ou nas armas somente, mas se erguerá
também, e principalmente, sobre as alianças e os vínculos políticos que
se consiga estabelecer nesse novo governo.

A busca de sustentação política será, assim, a tônica das preocupações
daquele que ascende à condição de príncipe. Fortuna, \emph{virtù}, armas
próprias e armas alheias, são todos elementos da conquista que remetem
sempre, cada um a sua maneira, ao modo como, num segundo momento, esse
conquistador político recebe o apoio político necessário para a
sustentação do governo. Dentro desta lógica argumentativa, mas por uma
outra perspectiva, nos diversos exemplos históricos mobilizados entre o
capítulo \versal{III} e o capítulo \versal{XI}, quando eles não são apresentados para
corroborar essa necessidade de buscar apoio político para o governo que
está se instalando, esses exemplos cumprem a função contrária, ou seja,
revelam o quanto os governantes sozinhos ou fundados unicamente na força
solitária do príncipe não tem a força política necessária para a
conservação do governo.

Talvez o exemplo mais eloquente dessa necessidade de apoio político do
príncipe novo seja a figura do cidadão (\emph{privato ciptadino}) que
ascende à condição de príncipe. Nesse exemplo, todas as nuances daquilo
que desde as primeiras linhas do livro se apresentavam como necessidade
para a constituição desse governo novo, revelam sua forma mais acabada.
Adiante nos ocuparemos como mais atenção sobre esse tipo político,
todavia, convém aqui apenas apontar alguns aspectos desse exemplo, tendo
em vista nosso objetivo de definição do que seja o principado.

O cidadão que ascende à condição de príncipe, modelo privilegiado que
personifica o príncipe novo por Maquiavel, é certamente alguém que, após
uma série de ações políticas, algumas calcadas na fortuna, mas a maioria
na sua \emph{virtù}, consegue o apoio da cidade para assumir o comando
político. Mesmo que ele tenha se valido da fortuna até a sua chegada ao
governo, doravante ele não se pode valer somente desta para manter-se no
comando político da cidade. O mesmo se diga das armas e das ações
cruéis: elas até podem possibilitar a conquista do governo, mas não se
constituem como fundamento seguro para o exercício dele.

Toda a argumentação culmina para o caso do cidadão que ascende à
condição de príncipe ``com o favor dos outros cidadãos'', ascensão essa
baseada ou no favor do povo ou no favor dos grandes. Esse caso torna-se
emblemático porque expõe a real necessidade deste indivíduo que deseja
ser príncipe. Primeiro, ele deve reconhecer que as forças políticas
estão em disputa para além da sua própria força política, visto que ele
não é a única fonte de força política ou a sede do poder; segundo, que
existem outros atores políticos nesse palco, que podem possuir maior ou
menor influência no jogo conforme as circunstâncias; terceiro, e o mais
importante, que o príncipe novo deve se inserir nessa disputa inerente à
vida política da cidade e saber conduzi-la, seja para a não dissolução
desse regime político, que pode ocorrer por meio da instalação de uma
tirania interna ou externa, seja para a própria conservação da sua
condição de figura política de destaque.

Após ter apresentado o tema a ser dissertado na primeira linha do
capítulo \versal{IX}, os dois períodos seguintes consolidam o argumento, como diz
Maquiavel:

\begin{quote}
Porque em toda cidade se encontram estes dois humores diversos e nasce,
disto, que o povo deseja não ser nem comandado nem oprimido pelos
grandes e os grandes desejam comandar e oprimir o povo. Destes dois
apetites diversos nasce na cidade um destes três efeitos: ou o
principado, ou a liberdade ou a licença. O principado origina-se do povo
ou dos grandes, segundo que uma ou outra destas partes tenha a ocasião,
porque, vendo os grandes que não podem resistir ao povo, começam a
aumentar a reputação e o prestígio de um dos seus e fazem-no príncipe
para poderem, sob sua proteção, desafogar o seu apetite. O povo, também,
vendo que não pode resistir aos grandes, aumenta a reputação de um e o
faz príncipe, para serem defendidos pela sua autoridade (cap. \versal{IX},
linhas 2 e 3).
\end{quote}

A primeira informação importante é que a cidade é composta de dois
humores, ou duas partes antagônicas. Num primeiro olhar, somos tentados
a pensar o principado tomado por esse antagonismo, mas note-se que não
são os regimes políticos e sim a cidade que, em sua constituição, em seu
substrato material, tem esse antagonismo político inerente. Tal oposição
natural -- e aqui convém insistir sobre esse aspecto, visto que as
partes ou humores compõe a natureza da cidade e são, portanto,
indissociáveis -- causa os desejos diversos entre os grandes e o povo:
sendo que os primeiros desejam comandar e os segundos em não ser nem
comandados e nem oprimidos. Ora, a combinação desses humores, tal qual
concebida pela tradição da medicina galênica e da hipocrática (nas quais
a combinação deles gerava os diversos tipos de temperamento), resulta em
três formas de governo: o principado, a liberdade e a licença\footnote{Sobre
  essa relação entre a medicina galênica e seus usos por Maquiavel, cf.
  Nicodimov, 2004.}. Esse trecho resultou numa vasta literatura de
comentários a respeito da teoria dos humores e do conflito político em
Maquiavel, que não vamos retomar aqui\footnote{No Brasil temos um
  capítulo significativo desta discussão. Cf. Ames\ldots{};
  Adverse\ldots{}; Bignotto\ldots{}; Cardoso\ldots{}; Martins\ldots{}}.
Entretanto, para nossas intenções, já é possível ver que o principado
não se confunde com a cidade, mas é um dos modos de sua ordenação em
função da combinação dos humores. Destaque-se que a cidade ainda pode
ser ordenada como liberdade (sinônimo de república) e como licença,
neste último caso, trata-se de um modo de ordenação política desregrado,
que se degenera em uma tirania (\emph{Discursos}, \versal{I}, cap. 2).

Em seguida, Maquiavel mostra como nasce o principado, a saber: ou quando
o povo escolhe alguém e lhe dá o apoio ou quando os grandes escolhem um
dos seus para fazê-lo príncipe. A sequência dos eventos revelará que
esse novo príncipe, na verdade, um cidadão (\emph{privato ciptadino}),
deve saber como buscar sustentação para o seu governo, apoio este que
deve, se possível, fundar-se nas duas partes políticas: o povo e os
grandes.

Neste ponto, já nos é possível entender melhor o que é o principado. Ele
é uma autoridade sobre a cidade que nasce a partir da disputa entre os
humores. Parece evidente, portanto, que o principado é, antes de tudo,
uma forma de governo, dentre as possíveis, para comandar uma cidade. Se
isso está claro, então, se voltarmos ao início do texto e repassarmos a
sequência argumentativa, veremos que Maquiavel trata dos vários modos de
conquista de um governo. A conquista do principado é, pois, a ação de
chegar ao posto de comando da cidade, seja usando as armas, seja usando
a fortuna própria ou a alheia, seja usando a \emph{virtù}. Uma vez
conquistado o comando da cidade, o governo, isso não implica em um
controle total das decisões, esse príncipe novo não é um príncipe
absoluto, pois deve decidir levando em conta essa dinâmica das oposições
e saber lidar com os diversos interesses, seja para não ficar refém de
um desses grupos, seja para não constituir inimigos fortes que ameacem o
seu governo.

Assim, podemos deduzir com certa facilidade que o principado é nomeado
deste modo, porque é a forma de governo sob o comando de um príncipe,
particularmente de um príncipe novo. Todavia ainda resta alguns
problemas lançados antes: por que o paralelo recorrente entre o
principado e a república? Se esse principado não é uma república, o que
parece muito evidente, por que ele também não nos permite identificá-lo
à monarquia, ao governo autocrático?

Há um fator que, de certo modo, responderia as duas questões. Como se
verifica não somente ao longo d'\emph{O Príncipe}, mas também nos
\emph{Discursos}, na \emph{História de Florença} e em diversos opúsculos
políticos maquiavelianos, é em função dos conflitos políticos que a vida
política da cidade deve ser pensada. Se nas repúblicas há mais vida e
mais ódio (\emph{O} \emph{Príncipe}, cap. \versal{V}, 9), é porque as lutas
políticas exigem maior engajamento das pessoas, elas devem tomar partido
nas disputas políticas, visto que o governo da cidade é resultado do
conflito. Ora, se no principado não temos a partilha do comando da
cidade, pois o governo se encontra nas mãos do príncipe, isso não
implica que esse governante, que pode até decidir sozinho, não tome tais
decisões a partir de sua vontade privada. Como se verifica nos vários
exemplos históricos citados, muitas vezes esse príncipe é premido a
tomar decisões contrárias aos seus reais interesses em função da
conservação do governo e do bem estar político da cidade. Talvez, nesse
caso, o exemplo mais eloquente seja o de César Borgia que manda matar
seu braço direito, Ramiro Orco. No principado, então, o príncipe toma a
decisão, indica o rumo político a seguir, mas não o faz necessariamente
de \emph{motu} próprio, decide premido pelas circunstâncias, delimitado
pelos interesses diversos que tensionam o seu governo. O exemplo da
possível conquista do reino francês também é ilustrativo
(\emph{Príncipe}, cap. \versal{IV}), visto que não adianta apenas derrubar o rei,
deve-se ter em conta as disputas políticas que ocorrem abaixo dele entre
os nobres e saber o modo como se inserir nelas para conservar o governo.

Então, tal governo principesco se diferencia de um governo de tipo
monárquico clássico, cujo exemplo é o governo do ``Turco'' citado no
capítulo \versal{IV}. Note-se que o próprio Maquiavel, ao tratar do principado
civil no cap. \versal{IX}, aponta para os riscos desse principado tornar-se um
principado absoluto (\emph{Príncipe}, cap. \versal{IX}, 23-27), no qual o
príncipe concentrasse em si todas as decisões. Ora, seja no caso do
governo otomano, seja no caso desse príncipe absoluto apontado no final
do capítulo \versal{IX}, Maquiavel chama a atenção para o fato de que ele não
consegue ter todo o controle e todo o comando político da cidade. O
Turco precisava dos seus \emph{sandjacs} (administradores políticos
nomeados pelo imperador) e o príncipe absoluto precisa dos magistrados,
o que, em ambos os casos, resulta em partilha ou delegação das decisões
políticas para pessoas que obtiveram apoios que resultarão, no limite,
num enfraquecimento do governo desse príncipe absoluto. Esse
posicionamento político é resultado, assim, de uma análise do contexto
político a partir da dinâmica das lutas entre as partes e de um governo
que é obrigado sempre a buscar apoios para se manter. No interior desse
quadro argumentativo, não há poder que se firme como absoluto, pois o
governante deve sempre agir para obter apoio ou deixar de agir para não
criar oposições.

A monarquia ou o governo absoluto não é uma impossibilidade, segundo
Maquiavel, apenas não é nunca um absoluto em sua plenitude e nem um
governo forte. Pela argumentação construída nessa primeira parte do
livro, não há nenhum principado forte que esteja fundado em um governo
absoluto ou monárquico. Talvez um último exemplo que comprove
definitivamente essa ideia, que não por acaso é o último exemplo de
principado analisado, seja os principados eclesiástico, cujo maior
exemplo é o papado. Nem o Papa possui um controle absoluto do seu
governo, ao contrário do que parece ser a primeira vista. A exposição
desenvolvida no capítulo \versal{XI} é muito ilustrativa de tudo o que se
apresentou até então no livro. Primeiro, conforme o estilo maquiaveliano
de exposição, somos tentados a acreditar que o principado eclesiástico é
sim um principado com uma dinâmica política diferente, visto serem
mantidos pelo próprio Deus. Entretanto, ao longo do capítulo, vai
ficando claro que mesmo o papa tem que lidar com as disputas políticas
entre os dois principais grupos políticos, no caso histórico de início
do século \versal{XVI}, os Collona e os Orsini, que nada mais são do que partidos
em disputa pelo governo. Sem contar a própria disputa entre os cardeais,
que é sempre natural na dinâmica política eclesiástica, de modo que,
mesmo o papa, não tem tanta segurança ou facilidade para governar, caso
se imaginasse que fosse esse principado um tipo de principado absoluto.

Ao final dessa primeira parte do livro, podemos constatar que, por todos
os exemplos mobilizados, o principado não é uma monarquia clássica,
muito menos uma prefiguração das monarquias absolutas que vigorarão ao
longo dos séculos seguintes na Europa. Nas poucas vezes que essa
aproximação ocorre, o principado é retratado como um governo fraco,
noção essa radicalmente diferente das monarquias absolutistas que serão
retratadas como governos fortes, materialização dos governos dotados de
soberania política.

Contudo, ainda resta um último ponto: se os principados não são
monarquias de tipo clássico e nem repúblicas, como entender esse regime
político? A resposta já nos é bem presente. Em síntese, o principado é
um governo sob o comando de um príncipe, que é antes de tudo um líder
político que assume o governo da cidade, mas que isso não implica
necessariamente em um governo monárquico ou autocrático de qualquer
espécie. Tendo em vista o destaque deste tipo muito particular de
príncipe que é um cidadão que assume o governo, que, por isso, necessita
agir no interior das disputas políticas naturais entre os humores, vemos
que a dinâmica das disputas políticas próprias da república se conserva
ainda nesses principados, porém agora sob um novo quadro institucional.
Esse seja talvez o ponto de contato entre principado e república
destacado por Maquiavel, pois ambos regimes possuem essa dinâmica
política inerente à natureza da cidade, mas que, na república, essa
disputa se configura em outros termos e resulta em outro ordenamento
institucional diferente do principado. No principado, também as disputas
existem, contudo, elas são exercidas com outra dinâmica política, com
outras delimitações, que não as mesmas das repúblicas, e geram outros
efeitos em termos de ordenamentos políticos e leis. Como já dito antes,
não é somente os conflitos políticos dos humores antagônicos, mas o
quadro político próprio dos principados é que fazem desse governo um
regime diferente das repúblicas, sem que isso implique em um governo
autocrático.

Por fim, podemos ainda constatar que esse principado fundado na ação
política de um cidadão que se torna príncipe se identifica quase
totalmente ao regime ``quase régio'' apresentado ao final do capítulo
\versal{XVIII} dos \emph{Discursos} (cap. \versal{XVIII}, linha 29). Torna-se, pois,
evidente que a crise da república resulta primeiro nesse governo fundado
neste príncipe, que pode implicar, por seu turno, em três outros
governos: no reestabelecimento do ordenamento republicano (hipótese esta
não explorada n'\emph{O Príncipe}), na conservação do principado civil
por longo tempo e na transformação desse principado em uma monarquia,
possibilidade esta que sempre está posta no horizonte, visto que o
principado civil pode tornar-se absoluto, conforme as ações de
centralização política do príncipe gerar, na sequência, um governo
dinástico, uma monarquia. Assim como no \emph{Pequeno tratado sobre as
repúblicas}, Maquiavel nunca é peremptório ou enfático: os ciclos
políticos são possibilidades de mudanças políticas que se apresentam aos
povos. Isso comprova em outro texto aquilo que havíamos constatado nos
\emph{Discursos} (Martins, 2007): que o pensamento político
maquiaveliano se apresenta sempre como possibilidades de configurações
políticas, jamais em ordenamentos políticos que se realizam
necessariamente, como um ciclo político determinista. Essa é uma das
marcas da reflexão política de Maquiavel, de pensar o mundo da política
como possibilidade, no qual se apresenta sempre aos homens alternativas
para tentar direcionar o curso das coisas que pareceria natural e
determinado. Enfim, probabilidade de realização e não determinação
histórica.

\subsection{A herança dos \emph{espelhos de príncipes} na~noção~de~príncipe~maquiaveliana}

Uma vez reconhecido o que é o principado, importa agora entender o que é
o príncipe para Maquiavel, ou mais especificamente, qual a noção de
príncipe que é apresentada em \emph{O Príncipe}. Essa análise será feita
em duas partes, num primeiro momento, recuperando as noções medievais
que emergem da tradição dos ``\emph{espelhos de príncipes''} e, em
seguida, pela análise desse príncipe como um personagem republicano.
Desde início, convém destacar que estamos tratando desse príncipe que é
um cidadão comum que se torna príncipe novo, donde ser nosso foco:
entender como Maquiavel concebe o papel político desse cidadão comum
(\emph{privato ciptadino}) que é preferencialmente o príncipe novo.
Entretanto, isso não deve implicar que há um único tipo de principado ou
de príncipe para Maquiavel. Enfim, interessa-nos saber melhor a noção de
principado e de príncipe que é ressaltada na obra.

Nesta investigação sobre a noção de príncipe em Maquiavel, ao voltarmos
nossas atenções para suas origens, certamente teremos que voltar ao
pensamento político latino do início da medievalidade, particularmente,
a partir das heranças teóricas legadas pelos autores da Patrística
latina e as elaborações que se seguiram no que se refere à noção de
\emph{regimen}. Tal hipótese foi levantada inicialmente por Senellart em
\emph{As artes de governar}, obra na qual procura demonstrar, em sua
primeira parte, como a reflexão pastoral dos autores da Patrística
latina influenciaram um ramo do pensamento político posterior que
desemboca no gênero dos ``espelhos de príncipes'' (\emph{specula
princeps}), cujo \emph{O Príncipe,} de Maquiavel, figura, por vezes,
como um exemplo.

Sobre o gênero literário, já há uma vasta literatura disponível
(\versal{SENELLART}, 2006; \versal{SKINNER}, 2000), que não pretendemos retomar a
exaustão. Contudo convém lembrar alguns de seus aspectos principais.

Os livros do gênero ``espelhos de príncipes'' eram obras escritas por
eruditos que trabalhavam nas cortes e eram destinadas aos futuros
governantes, apresentando orientações para as práticas de governo. A
origem pode ser remontada ao \emph{``Dos Deveres''} (\emph{De Officis},
escrito no século \versal{I} a.C.), de Cícero, da qual se seguiu uma longa e
numerosa variedade de obras que perpassaram os séculos, tendo o mesmo
intuito de orientação e formação para os regentes. Não há um padrão ou
temática única nessas obras, mas em sua maioria, principalmente durante
o período medieval, esses textos preconizavam a valorização dos aspectos
morais e éticos nas práticas de governos principescas. Conforme
Senellart (2006), que faz um amplo resgate histórico desse gênero --
justamente para entender o possível lugar de \emph{O Príncipe}, de
Maquiavel, nesta tradição -- a ênfase nos aspectos éticos e morais dos
``espelhos de príncipes'' têm nas suas fontes os escritos dos autores da
Patrística latina. Desde Agostinho (séc. \versal{V}), passando por Boécio e
Cassiodoro (séc. \versal{VI}), Isidoro de Sevilha e Gregório Magno (séc. \versal{VII}),
entre outros, a reflexão da Patrística latina valorizou a dimensão ética
na condução dos governos em sobreposição sobre os elementos
determinantes próprios do contexto político.

Seguindo adiante na reflexão exposta por Senellart, verifica-se que a
reflexão política anterior a Maquiavel exerceu uma influência
significativa para a elaboração da noção de \emph{príncipe} como um
condutor ou regente, afastando, desse modo, por outra perspctiva, uma
vertente interpretativa que entende o príncipe maquiaveliano como a
prefiguração do soberano moderno.

Ainda, a título de apresentação, nossa interpretação diverge da de
Senellart, embora não lhe seja contrária, ao não optarmos por explorar a
dimensão ética nas práticas de governo, conforme ele enfatiza, para
comprovar as origens teóricas dos ``espelhos de príncipes'' nos textos
pastorais do início da medievalidade latina. Pretendemos seguir, a
partir de suas premissas, uma outra vertente de explicação na qual fique
mais clara as influências desse pensamento político oriundo da
patrística latina que se corporifica na noção de um príncipe como alguém
responsável, prioritariamente, pela condução e direcionamento do povo,
mais do que aquele que exerce seu poder ou domínio sobre os demais. Esse
é o ponto central de nossa interpretação: a partir dos elementos
teóricos herdados da reflexão política medieval latina sobre o
\emph{regimen}, explorar suas possíveis ligações com a noção
maquiaveliana de príncipe como um condutor ou regente do governo, ao
invés da imagem daquele que exerce o poder soberano ou domínio sobre o
seus súditos.

Assim, somos obrigados a retornar a imagem que a Patrística latina
confere àquele que tem por missão conduzir e guiar os cristãos. Conforme
Senellart (2006, 69), foi no momento histórico situado entre a
transferência da dissolução do Império Romano do Ocidente, com a queda
de Roma em 476, e a instauração do Império Carolíngio por Carlos Magno
que a Igreja latina operou ``\emph{uma inversão espantosa. Em vez de
exortarem os reis a governarem com justiça, sabedoria e bondade,
moderando assim o poder, oriundo da violência, pela doçura de seu
exercício, ela faz do `governo' -- ato de regere, dirigir -- a condição
mesma da realeza} (\emph{regnum})'' (Senellart, 2006, 69). O exercício
do governo implica, pois, uma inflexão conceitual: para além (e não
contra) o discurso pastoral, do governante como pastor de almas, esse
comandante da comunidade política cristã deve exercer um maior controle
dos corpos, para que haja, por consequência, uma maior disciplina das
almas com vista à salvação eterna.

Todavia esse dado histórico precisa ser melhor explorado e especificado,
visto que foi, na verdade, um século antes, com a transposição da
capital do Império de Roma para Constantinopla, que se verifica a
inauguração de uma nova reflexão política sobre o modo de governar e
conduzir as comunidades\footnote{Estamos insistindo aqui no termo
  \emph{comunidade,} visto que esse termo foi largamente empregado pelos
  pensadores latinos e terá grande fortuna na posteridade,
  principalmente após a tradução latina da \emph{Política,} de
  Aristóteles, feita por Guilherme de Moerbeke em 1265, que traduz
  \emph{koinonia politiké} por \emph{communicatio politica}. Sobre as
  consequências dessa tradução, cf: Rubstein (1997); Martins (2011;
  201?)}. Conforme Bertelloni (2005), a mudança da capital imperial
originou uma nova teoria política, principalmente a partir dos escritos
de Eusébio de Cesarea, que buscam justificar o poder do imperador romano
sobre os domínios eclesiásticos, doutrina esta também conhecida como
\emph{cesaropapismo}. Em linhas gerais, o \emph{cesaropapismo}
justificava que o imperador cristão, por ter recebido sua missão
diretamente de Deus -- para isso invocando a visão sobrenatural que o
então general Constantino recebe de Deus para pintar a cruz sobre os
escudos de suas tropas, para com isso, conquistar a vitória e tomar o
governo do Império --, teria uma dupla incumbência: guiar os cristãos na
terra e defender a Igreja, sendo, portanto, a maior figura política e
também o maior dirigente eclesiástico. Essa dupla missão, de
\emph{César} e de Papa, unificava-se, agora, na figura do imperador
cristão e de seus herdeiros doravante. Fato é que, depois do século \versal{IV},
os imperadores cristãos de Constantinopla convocavam concílios, nomeavam
bispos e patriarcas, promulgavam normas, leis e documentos dogmáticos,
exercendo um poder direto nos rumos da cristandade.

Porém, esse quadro de ingerência política do imperador sobre os rumos
dos povos cristãos nunca foi bem aceito pelas autoridades eclesiásticas
latinas, principalmente o bispo de Roma, que não era mais do que um
subordinado do Imperador. Ora, do ponto de vista da reflexão teórica,
não foi a queda de Roma em 476, com a deposição de Rômulo Augusto por
Odoacro, que culmina numa reação dos pensadores latinos. Na verdade,
desde 313, quando o imperador Constantino transfere a capital de Roma
para Constantinopla, que a importância política da cidade cede lugar à
nova capital imperial, situada estrategicamente no ponto de confluência
entre aquilo que posteriormente se entendeu por Ocidente e Oriente.
Juntamente com esse dado da política imperial, também o bispo de Roma,
ainda que sempre fosse reconhecido como sucessor de São Pedro pelos
demais bispos, não possuía influência ou poder político o bastante para
determinar os rumos da cristandade. Importa frisar que, do ponto de
vista político, desde o século \versal{IV} Roma já não possuía mais a importância
política de outrora e figurava como uma província de um Império Romano
cujo centro estava em Constantinopla e cuja parte ocidental decaía
gradativamente em termos econômicos, sociais e políticos . O bispo de
Roma, por seu turno, também não possuía, ainda nesses séculos, o
controle político da cidade e nem uma posição de hierárquica superior
entre os demais bispos, sendo um \emph{``primus inter pares''}, o
primeiro entre os pares, o que não implicava em subordinação política
entre o bispo de Roma e os demais patriarcas orientais, por exemplo.

A primeira reação teórica dos pensadores latinos a essa hegemonia do
pensamento político bizantino (o \emph{cesaropapismo}) ocorreu antes
mesmo da queda de Roma em 476, com a reflexão de Agostinho de Hipona,
notadamente, com o seu \emph{Cidade de Deus}. Agostinho, certamente
influenciado pelo ``Saque de Roma'' de 24 de agosto de 410, perpetrado
por Átila, o Huno, postula na sua obra a teoria das duas cidades, uma
celeste, eterna, a qual os cristãos estão destinados, a Jerusalém
Celeste; e outra, terrestre, mundana, corruptível, cuja imagem da Roma
decadente é sua melhor personificação. Consequência direta dessa ideia
de separação das duas cidades é o modo como os cristãos devem se portar
no mundo, fundamentalmente voltados para a Jerusalém Celeste, seu
destino (\emph{telos}) e lugar da sua realização. A Jerusalém terrestre,
decadente, temporal e corruptível não deve ser objeto de atenção e
preocupação dos cristãos, pois ali não está seu destino, sua realização
ou completude, para usar um vocábulo caro ao agostinismo.

Como consequência dessa teoria de separação das esferas celestial e
terrena, a vinculação do homem cristão aos negócios da cidade fica sem
uma justificativa e gera, por seu turno, uma noção de desqualificação do
mundo político que perpassará boa parte dos pensadores medievais latinos
e verá sua influência até o Renascimento italiano. Com efeito, a
caracterização da cidade terrestre como corruptível implicou uma
desqualificação da política e da história, como enfatiza Pocock (1980).
Os pensadores latinos sempre se defrontarão com a dificuldade de pensar
o mundo da política como algo que não fosse efêmero e decadente. Neste
sentido, um primeiro sintoma desse drama e, \emph{pour cause}, uma
primeira tentativa de resposta já se encontra no século \versal{VI} com a
\emph{Consolação da Filosofia} de Boécio. Esse aristocrata romano e
cristão escreve, enquanto aguardava na prisão a execução de sua pena de
morte, um texto no qual relata as angústias do cristão no âmbito da vida
pública na cidade: não poder negar a sua fé e ser ao mesmo tempo um fiel
e legítimo cidadão romano. Como afirma Pocock, a \emph{Consolação da
Filosofia} não é uma obra de filosofia política, mas contém a filosofia
de um homem político (Pocock, 1980, 127).

Seguindo, pois, o nosso fio condutor, entendemos que já no início do
século \versal{V} e doravante -- então antes de 476, como destaca Senellart -- os
pensadores cristãos latinos começaram a elaborar novas teorias sobre a
inserção do cristão no âmbito dos governos e, nesse sentido, o modo como
governar, distanciando-se, assim, do arcabouço teórico bizantino, num
claro esforço de diferenciação e busca de alguma autonomia em termo de
governo nos seus territórios. Contudo, tal reflexão tinha seus limites
claros, seja na separação das esferas celestes e terrestres, seja nos
limites e autonomias de governo das cidades da Europa latina.

Tento em vista essas limitações, de fato, uma teoria de governo não
deveria se sustentar no território da sua ação de comando, mas em algo
que antecedesse isso, visto que não há essa possibilidade de legitimação
do governo. Para isso, e neste caso bem notado por Senellart, a noção de
governo antecede ao \emph{regnum}, ou seja, a condição de comando
político antecede à materialidade do regime político, no caso, o reino
ou o território ou o povo que se deve governar. Essa noção aparece de
modo claro já em Isidoro de Sevilha, que nas suas \emph{Etimologias}
afirma: ``\emph{A palavra reino vem de rei. Com efeito, do mesmo modo
que rei é tirado de reger, reino é tirado de rei. {[}\ldots{}{]} Rei é tirado
de reger. Do mesmo modo que sacerdote vem de santificar, rei vem de
reger. Ora, não se rege se não se corrige}'' (\emph{Etimologias}, livro
\versal{IX}, 3)\footnote{``Regnum a regibus dictum. Nam sicut reges a regendo
  vocati, ita regnum a regibus. {[}\ldots{}{]} Reges a regendo vocati. Sicut
  enim sacerdos a sacrificando, ita et rex a regendo. Non autem regit,
  qui non corrigit''. (\emph{Etimologias}, livro \versal{IX}, 3).}. A sequência é
clara: reger implica em rei que implica em reino (reger -- rei --
reino). O ato de reger (\emph{regere}) é, pois, o que fundamenta a
condição de rei, que, por seu turno, fundamenta o reino. É sobre o verbo
e a ação que recaem, então, os fundamentos e as atenções; é na ação de
reger, no exercício da regência, que se põe o fundamento da condição do
rei. No limite, é antes no \emph{regere} que no \emph{regnum} que o rei
encontra sua fundamentação política.

Então, o exercício do governo funda o regime político e não o reino
(\emph{regnum}) que determina o governo, como não poderia ser de outro
modo para um autor latino dos séculos \versal{VI} e \versal{VII}. Com efeito, tendo em
vista que as responsabilidades sobre os territórios eram delegações do
imperador romano em Constantinopla, a fundamentação da legitimidade do
exercício de governo somente poderia se apoiar sobre o ato ou o
exercício do governo. Seja o rei de um reino do Império Romano, seja
mesmo um bispo com o governo de uma cidade nos confins da Europa, a
fundamentação de suas ações políticas repousava na ação de governo, no
exercício da direção que dava à cidade. Não havia dinastia, sufrágio,
delegação divina que justificasse a contento o exercício de seu ato
governo, senão o próprio ato de governo em si.

Tal ação de governo nada mais era, então, do que reger, corrigir e,
posteriormente, conduzir os homens pelo caminho reto. Assim, o
governante rege e corrige as pessoas, não os obriga ou exerce o seu
domínio, sua violência, seu poder nas ações de seus comandados. Há,
pois, uma certa sutileza no vocabulário que convém destacar. O rei é rei
não porque tem um reino, mas porque rege, porque corrige e orienta o
povo, tal qual o maestro em relação ao coro ou o marinheiro ao navio.
Sua ação de governo é limitada e circunscrita, não cabe a ele impor pela
violência ou qualquer meio extraodinário as ações políticas da cidade,
assim como o marinheiro não impõe pela violência a ação de navegar do
navio, mas o conduz com sabedoria e corrige conforme as forças em ação
os rumos que o navio deve seguir para chegar ao seu destino. Do mesmo
modo o rei rege e conduz o povo, ora corrigindo as ações, ora
impulsionando, para que a cidade perfaça seu caminho de retidão e
justiça. O rei, mas não somente ele, como também o bispo e o comandante
militar, figuram como os pastores que se põe na condução e direção de
seus comandados e não como seres que exercem seu domínio sobre outros
indivíduos.

O destaque deste aspecto pastoral do governante esvazia ainda mais a
imagem anacrônica do rei que domina o seu povo em função de sua
soberania. Esse rei preconizado pelos textos pastorais da Patrística
latina não tem o \emph{dominium} sobre os homens, isso somente lhe será
atribuído quase mil anos depois. O rei é condutor, assim como o
sacerdote, ou melhor, assim como o bispo em relação aos membros de sua
igreja. Na verdade, está se resgatando aqui a antiga antítese entre
\emph{regere} e \emph{dominari}, que remonta a Cícero (\emph{De
Republica,} \versal{I}, 31), mas que sua gênese poderia ainda ser atribuída aos
filósofos gregos, particularmente a Aristóteles. Já na \emph{Política}
(\versal{III}, 1-2), Aristóteles sublinha a diferença do modo de governo da
cidade (\emph{polis}) e da casa (\emph{oikós}). O governo da casa é de
tipo senhor-escravo (metáfora que fará fortuna na história do pensamento
político), no qual o primeiro exerce sua imposição sobre a vontade do
segundo, dominando-o, numa relação hierarquizada e vertical. Na cidade
as relações entre os cidadãos são relações políticas, de igualdade
(\emph{isonomia}), reciprocidade, na qual não há domínio das vontades,
mas se exige a força do convencimento, da argumentação pública, enfim, o
exercício do \emph{logos} na \emph{Àgora}. Como expõe Vernant (2002,
cap. 3), os governos da casa e da cidade são, para os helenos de
natureza diferentes, sendo o primeiro uma relação de caráter doméstico
(\emph{oikonomico}) e a segunda de caráter político (\emph{politica}). A
transposição dessas categorias para o mundo romano resultará em uma
distinção de governos que regem (\emph{regere}) e os que dominam
(\emph{dominium}). O \emph{dominium} aqui derivado de \emph{domus}
(casa), ou seja, as relações de domínio são a transposição para a esfera
pública das relações próprias da casa, o que distingue o governo do rei
e do tirano. O tirano exerce seu domínio sobre o povo, já o rei rege.
Donde a antítese entre o reger e o dominar no exercício do governo.

Nos textos dos autores da patrística latina, a noção de domínio também
possuía uma relação com a irracionalidade dos homens, pois, conforme
Senellart (2006, p. 101), o domínio se exercia sobre os corpos dos
homens controlados pelo pecado, que estavam como que obscurecido pelos
vícios dele decorrente. Para esse ser humano decaído, que se tornou
servo de seu corpo, apenas um comando forte, tal qual se aplica aos
animais brutos, será adequado. ``Não é o homem enquanto homem, mas o
homem rebaixado pelos impulsos de sua carne à condição do animal, que é
o objeto do governo régio e determina as modalidades de seu exercício''
(\versal{SENELLART}, 2006, p. 101). Donde se evidencia o binômio
dominação/irracionalidade, o governo que domina, se impõe sob a condição
de irracionalidade dos comandados. Neste caso, com mais força do que no
pensamento antigo, no qual, ainda que o \emph{logos} se coloque como a
marca da cidade, sua presença no âmbito do político era condição
\emph{sine qua non} da dinâmica política. Aqui é a antítese que se
apresenta: o domínio do governo se impõe pela ausência de razão, porque
os homens ficaram obscurecidos pelos pecados e pelos vícios.

Associada a essa tarefa de reger e corrigir, o governante tem também uma
outra incumbência na sua tarefa, a de vigiar e assistir os homens.
Herdada da tradição grega, mas sob forte influência cristã, o governante
também é um \emph{episcopus}, da qual resultou o termo português bispo.
O \emph{episcopus vem do grego episkopos}, que se origina, por seu
turno, do grego \emph{skopós}, que é observar, ver, vigiar. Logo, também
é uma função do rei o observar, guardar, vigiar para que possa corrigir
e reger. Tal imagem do governante como aquele que vigia, que olha,
remete a um tema clássico do pensamento político que é o da vigilância
como algo típico da ação política. Essa mesma temática aparecerá nos
textos de Maquiavel, particularmente nos \emph{Discursos}, no livro \versal{I},
capítulos 5 e 6, onde ele destaca o papel de vigilância que a população
tem na defesa da liberdade. Ao invocar a defesa da liberdade (\emph{la
guarda della libertá}), ele não somente está pedindo ao povo para
defender, mas também vigiar, tal qual a sentinela, que vigia e defende a
fortaleza.

Passo seguinte dessa concatenação teórica se fará, a partir do século
\versal{XII}, com a associação à função real do dirigir, liderar. Sendo o rei
aquele que rege, observa e corrige o seu povo, cabe então a ele também
dirigir o povo e o governo. O rei deve também guiar e indicar o caminho,
os rumos a seguir pelo seu reino. Tal capacidade de direção decorre da
racionalidade que deve ser própria da personalidade real, ou seja, em
função de sua razão, que se manifesta na prudência e sabedoria, o rei
tem a qualidade para dirigir e liderar o reino.

Essa imagem do rei que também dirige, segundo Senellart (2006), já
aparece em John de Salisbury no seu \emph{Policraticus}, escrito no
século \versal{XII}, mas se encontra também em Tomás de Aquino (no \emph{De
Regno}) e em Egídio Romano (\emph{De Regimine Principum}), ambos do
século \versal{XIII}. No limite o desenvolvimento do argumento é muito claro: por
reger, corrigir e observar, sendo dotado de sabedoria e prudência, está
o príncipe habilitado a dirigir o povo. O rei, portanto, é agora um
rei-guia, como diz Senellart: ``\emph{Regere} não mais simplesmente
corrigir, mas dirigir. Cabeça do corpo político, o príncipe coordenará a
ação de seus membros em vista de um fim coletivo. \emph{Rex sagittator},
dirá Egídio Romano: o rei, como um arqueiro, olhos fixos no alvo, mira
longa à sua frente''\emph{.} (2006, p. 133)

Todas essas qualidades na ação de governar demonstram a liderança e
proeminência do príncipe em face da cidade, o que consolida cada vez
mais sua condição política destacada, como um primeiro, uma figura
destacada e, por isso, que fica na vanguarda, um precursor. Entretanto,
convém frisar, tudo isso ocorre sem que ele faça valer a força ou a
violência como atos próprios e constantes do seu governo para conservar
a sua condição de príncipe, característica essa que será própria da
Modernidade. Apesar da cólera, natural nos homens, o governante dos
``\emph{espelhos de príncipe}'' deve dissimular seu sentimento de
violência, raiva, rancor, oriundo da bile negra, para que ela não turbe
o seu julgamento, conforme sugere Egídio Romano no \emph{De regimine
principum}. Ora, nem encolerizado o príncipe deve fazer uso da força ou
violência no exercício do governo, para que não seja lhe imputado a
perda da razão, da prudência característica dos grandes líderes. Sem
contar que estamos ainda distantes de uma noção de força ou violência
como veremos, por exemplo, na noção que Thomas Hobbes (séc. \versal{XVII})
conferirá ao Estado\footnote{Não pretendemos fazer aqui uma análise da
  presença ou não da noção de Estado em Maquiavel, algo que possui uma
  vasta literatura e que demandaria um esforço analítico que ultrapassa
  as ambições desse ensaio. Todavia, se tivermos em mente aquilo que
  Ercole define como Estado, como sendo: ``a entidade coletiva soberana
  resultante do ordenamento jurídico de um povo num território sob um
  poder comum e que permanece idêntico a si mesmo através da sucessão e
  a mudança dos indivíduos, dos órgãos e das formas constitucionais''
  (\versal{ERCOLE}, 1926, p.65), notar-se-á que tal concepção não está presente
  em sua plenitude em Maquiavel. Como tentaremos demonstrar, as
  concepções políticas que subjazem a noção de principado e príncipe
  estão calcadas em noções anteriores, e, por isso, diferentes dessa
  noção de Estado própria da modernidade, embora seja possível já
  visualizar alguns elementos conceituais que serão desenvolvidos
  posteriormente. Como nos mostra Rubinstein, antes dos humanistas, já
  se encontrava em vários escritos políticos da medievalidade latina o
  termo \emph{status} em uma acepção política, como regime ou forma de
  governo, como, por exemplo, no \emph{Sententia libri Politicorum,} de
  Tomás de Aquino, e no \emph{De Regime Civitatis,} de Bartollo de
  Sassoferrato. Mas em nenhum deles com o sentido que a modernidade
  conferirá ao termo (Rubinstein, 2004, p. 151-163). Para uma
  apresentação geral do debate, cf: Ames, 2011.}. É essa ausência do
argumento do exercício da força, entendida mais como \emph{dominium},
como essecial e inerente ao governo do príncipe que desloca toda a
atenção para o argumento racional, ou como destacará Skinner (2000),
para a capacidade retórica e de persuasão do príncipe.

Enfim, verifica-se por esses elementos da tradição dos ``espelhos de
príncipe'', que muitas dessas concepções repercutem no texto
maquiaveliano. Como se sabe, a noção de príncipe, que perpassa a obra de
Maquiavel como um todo, apresenta inúmeras semelhanças com esse ideal
herdado do pensamento medieval latino, a saber:

\begin{itemize}
\item
  \begin{quote}
  em quase todos os casos apresentados, a exceção do príncipe herdeiro,
  o príncipe de Maquiavel tem a sua condição de governante fundada antes
  no seu exercício político do que no território ou reino;
  \end{quote}
\item
  \begin{quote}
  Esse príncipe maquiaveliano deve, em função de sua condição e daquilo
  que o legitima no governo, reger, conduzir e liderar a cidade -- algo
  que ganha contornos claros e dramáticos no capítulo final do livro;
  \end{quote}
\item
  \begin{quote}
  No caso do uso da violência, apesar de ser apresentado como
  estratagema para a conquista da condição de príncipe, Maquiavel deixa
  claro que ela não é \emph{virtù} e sua mobilização no texto não tem a
  mesma justificação que terá posteriormente o argumento do uso legítimo
  da violência pelo Estado. Esse, na verdade, é um dos pontos pantanosos
  de \emph{O Príncipe}, pois ainda que não tenhamos a mesma
  caracterização da violência estatal da modernidade, ela se faz
  presente como elemento da ação política do príncipe.
  \end{quote}
\end{itemize}

Mas convém entender melhor esses aspectos no texto maquiaveliano.

Como já foi dito, podemos dividir o livro em duas grandes partes, uma
primeira dedicada ao principado e uma segunda parte dedicada ao
príncipe. Na primeira parte, Maquiavel se detém a tipificar os modos de
ascensão à condição de príncipe, a saber: por herança, por armas
próprias e alheias, por \emph{virtù} e por fortuna. Como diz o próprio
autor no cap. 2, no caso do príncipe herdeiro, as dificuldades na
condução do governo são pequenas, pois:

\begin{quote}
Digo, portanto, que nos estados hereditários e acostumados à
dinastia do seu príncipe são muito menores as dificuldades para
conservá-los do que nos novos, porque basta não preterir os ordenamentos
de seus antecessores e posteriormente contemporizar com os acidentes, de
modo que, se tal príncipe tiver uma indústria ordinária, sempre
conservará o seu estado, a não ser que uma força extraordinária e
excessiva o prive dele. E tendo sido dela privado, reconquista tal
condição na medida em que o conquistador enfrentar alguma
adversidade (\emph{O Príncipe}, cap. 2, linha 3).
\end{quote}

Neste caso, ainda, a legitimidade do governo se faz pelo pertencimento à
família que governa, logo, não por si ou por seu governo que esse
príncipe herdeiro assume a condição de príncipe. Esse breve capítulo
inicial da obra revela, também, uma tópica central da reflexão que será
desenvolvida, a saber: a análise do príncipe novo. Como se verifica,
esse é o personagem principal de Maquiavel, o príncipe que ascende ao
governo e que não o recebe de modo hereditário. Todavia, alguém poderia
argumentar que mesmo o príncipe herdeiro, quando ascende ao governo,
naquele momento ele é novo. Porém, não é dessa condição temporal que
Maquiavel está fazendo menção, mas da condição política nova para ele,
algo que um príncipe herdeiro não tem. O príncipe novo é novo do ponto
de vista político, ou seja, ele está inaugurando, iniciando o seu
exercício político, algo que não ocorre com o príncipe herdeiro. Mais
ainda, o príncipe é novo porque funda o governo, inaugura o novo regime,
ao contrário do herdeiro, pois esse, como explícito pelo próprio texto
maquiaveliano, herda, recebe a sua condição política, logo não funda
nada. Em consonância com o pensamento político maquiaveliano, poderíamos
até dizer que um príncipe herdeiro coloca-se como um príncipe novo, se
ele se apresenta na cena política e faz de seu governo uma novidade em
relação ao governo anterior. Mas, neste caso, como declara Maquiavel,
exige-se dele qualidades extraordinárias, ou seja, ele deverá mudar os
procedimentos políticos ordinários, fundar novos ``modos e
ordenamentos'', e, neste caso, seguir o que vem exposto para o príncipe
novo.

Ora, pela próprio entendimento do que seja aqui o \emph{novo},
verifica-se que Maquiavel está tratando de um tipo particular de
personagem político que assume o comando político da cidade. Nos demais
casos relatados de conquistas de principados, até mesmo para o
principado eclesiástico, aquele que ascende à condição de príncipe tem
que justificar por si mesmo e pelas suas condições sua legitimidade no
governo. Seja por ter a fortuna de conquistar o governo, seja por
possuir um exército próprio ou contar com a ajuda de uma força militar
alheia, seja por possuir \emph{virtù}, o aspirante à condição de
príncipe tem que se valer de seus esforços, de suas qualidades para
obter o governo. Pensando na formula isidoriana, que o rei é rei porque
rege, não tendo nada anteriormente que justifique sua condição de rei,
do mesmo modo esse príncipe novo maquiaveliano, é príncipe \emph{a
posteriori}, ou seja, pode-se afirmar que ele é príncipe apenas quando
está no exercício de seu governo, porque conquistou essa condição e a
exerce. Mais ainda, se diz dele príncipe porque tem o principado,
conquistou o regime principado, ou, no limite, se diz príncipe em função
do principado.

Como foi visto, o principado não é o território ou um reino, como em
geral entendemos. Principado é antes uma forma de governo ou um regime
sob o comando de um príncipe. Então, se o principado não é um reino ou
território e o príncipe se diz príncipe em função do principado, sua
condição se faz pelo próprio ato de governar, pela sua ação de governo,
a semelhança da sentença isidoriana. Donde as ações num primeiro momento
objetivarem a conquista do principado, algo que se faz pelos quatro
modos citados, e que depois se mantém em função da sua qualidade ou
\emph{virtù} como governante, tema da segunda parte do livro, que trata
da conservação do governo.

É evidente, pois, que esse príncipe alcança essa condição em função de
suas qualidades ou \emph{virtù}, ainda que em alguns casos ela não se
faça necessária, mas que certamente ela deve comparecer na conservação
do governo. Ou seja, mais cedo ou mais tarde, o príncipe deve demonstrar
possuir \emph{virtù} para conseguir conservar sua condição de príncipe.

Por outro lado, tal formulação de príncipe fundada no exercício do
governo tão somente, excluindo, desse modo, a necessidade anterior do
reino ou território para a fundamentação de seu governo, reforça a ideia
de que não temos ainda a noção de Estado. Com efeito, na medida em que
esse príncipe funda a sua condição no seu governo, não há algo anterior
a esse governo que estivesse posto como elemento de legitimação de sua
ação política. Em outras palavras, o príncipe, ao conquistar o
principado, estabelece nesse momento o seu governo, inicia de fato o seu
regime e terá sempre na sua ação política -- ou em termos
maquiavelianos, na sua \emph{virtù} -- o principal alicerce da sua
condição de príncipe.

Tal hipótese ajudaria a explicar, ainda, porque Maquiavel confere tanto
destaque à \emph{virtù} do príncipe, a ponto de dedicar metade de sua
obra a essa temática. Note-se que não se trata de uma exposição entre os
capítulos 15 a 24 do fundamento ético e moral da \emph{virtù}, tal qual
num tratado de filosofia moral como muitos que se fizeram durante o
Medioevo, mas de uma exposição da \emph{virtù,} depois de comprovada sua
necessidade para a condição de príncipe, para que esse príncipe conserve
o seu governo. Ou seja, na segunda parte do livro, Maquiavel dedica-se a
pensar a \emph{virtù} em função da sua condição de fundamento e
legitimação do governo do príncipe, porque ela se faz necessária.

Agora talvez fique mais claro o sentido da sentença que o príncipe tem
essa condição em função do exercício de seu governo, pois é a
\emph{virtù} que lhe dá esse fundamento em última instância. Para
comprovar isso, dois exemplos mobilizados na primeira parte da obra
podem nos vir em auxílio: o caso de César Borgia e o \emph{privato
ciptadino}.

César Borgia, filho do papa Alexandre \versal{VI}, é um personagem muito presente
n'\emph{O Príncipe}, podendo até ser confundido com o ideal de príncipe
que Maquiavel deseja apresentar. Contudo, nas mobilizações dos feitos de
Cesar Borgia ao longo do livro, sente-se que há uma certa ambiguidade:
por um lado, ele é apresentado como tendo muita \emph{virtù}, o que lhe
possibilitou fazer inúmeras conquistas, mas, de outro, nota-se que
faltou-lhe \emph{virtù} para conservar tudo aquilo que conquistou. O
capítulo \versal{VII} é certamente o melhor lugar para se perceber essa
ambiguidade que ronda a figura de César Borgia.

Na sequência expositiva sobre a conquista dos principados, depois de ter
analisado como eles são conquistados com armas próprias e com
\emph{virtù}, no capítulo \versal{VII}, Maquiavel passa a dissertar sobre o caso
contrário, quando alguém conquista um principado com armas e fortuna
alheia. O problema inicial desse tipo de conquista é logo de início
apontado:

\begin{quote}
{[}3{]} Estes estão fundados unicamente na vontade e na fortuna
de quem lhes concedeu tal status, duas muito volúveis e instáveis, e não
sabem e não podem se manter naquele posto: não sabem, porque não
sendo um homem de grande engenho e \emph{virtù},
não é razoável que, sempre vivendo como homens de condição
particular, saibam comandar; não podem, porque não
têm forças que lhes possam ser amigas e fiéis (\emph{O Príncipe}, cap.
\versal{VII}, linha 3).
\end{quote}

A dificuldade principal é evidente: aquele que conquista um governo
apoiando- se nas qualidades e na força de outros terá muita dificuldade
de se manter nessa condição de príncipe. O que, para o problema em
questão, é muito ilustrativo, pois, segundo Maquiavel, mesmo que alguém
receba um governo de outro, se ele não tem as qualidades necessárias
para conservar esse governo, perderá tal condição. Com efeito, os
fundamentos desse novo governo estão na vontade e fortuna de quem
outorga, que, como declarado, não perfazem alicerces seguros. A vontade,
embora Maquiavel não disserte sobre ela, sabemos todos que é volúvel,
que pode mudar, donde apoiar um regime na vontade de outrem ser de fato
algo transitório. A fortuna, como ele explicará no capítulo 25 de
\emph{O Príncipe}, é inconstante e difícil, senão impossível de ser
dominada. Logo, começar um governo sustentado por dois elementos
exteriores ao seu controle é permitir a instabilidade. Fato esse que
revela, inicialmente, que nada anteriormente sustenta a condição do
príncipe, a não ser o exercício do governo, no caso, um exercício com
\emph{virtù}.

Entretanto, para corroborar sua posição, logo adiante Maquiavel cita
explicitamente o caso de César Borgia, quando diz:

\begin{quote}
Por outro lado, César Bórgia, chamado pelo povo de duque Valentino,
conquistou o governo com a fortuna do pai e com a
mesma o perdeu, apesar de ter ele usado de todos os recursos e ter feito
todas aquelas coisas que um homem prudente e virtuoso deveria fazer para
deitar suas raízes naqueles governos, que as armas e a fortuna de outros
lhe haviam concedido (\emph{O Príncipe}, cap. \versal{VII}, linha 7).
\end{quote}

O exemplo não poderia ser melhor e mais dramático, visto que César,
mesmo tendo ao seu lado o apoio da condição política do pai, que era
papa, não foi o suficiente para que ele pudesse ter garantias de
exercício tranquilo ou seguro de seu governo sobre os territórios recém
conquistados. Teve ele que realizar inúmeras ações para conquistar a
legitimidade política nesses novos governos, mas ainda assim lhe faltou
algo. Maquiavel chama em causa um exemplo em grau máximo, pois nem a
legitimidade política de alguém como um papa é garantia o suficiente
para o exercício do governo de seu filho. No limite, ele dá a entender
que tal legitimidade política não pode ser transferida ou irradiada para
outro, mas apenas o próprio indivíduo é que tem as condições de
estabelecer os alicerces de seu governo. Na sequência isso fica mais
evidente, quando diz:

\begin{quote}
{[}8{]} Porque, como se disse anteriormente, aquele que não
constrói primeiro os fundamentos, poderia, com uma grande \emph{virtù},
construí-los depois, ainda que se façam com incômodo para o arquiteto e
perigo para o edifício. {[}9{]} Se, então, considerarmos todos os
progressos do duque, veremos que ele construiu grandes fundamentos para
um poder futuro, sobre os quais não julgo supérfluo discorrer, porque
não saberia quais preceitos melhores dar a um príncipe novo, senão o
exemplo de suas ações; e se seus modos de proceder não lhe forem
proveitosos, não será por culpa sua, porque nasce de uma extraordinária
e extrema malignidade da fortuna (\emph{O Príncipe}, cap. \versal{VII}, linha 8 e
9).
\end{quote}

Conforme dito, o exemplo de César torna-se mais emblemático, pois,
sabendo ele que deveria buscar outro fundamento para o seu governo, que
não apenas no prestígio de seu pai, tratou de executar uma série de
medidas políticas para arregimentar os apoios necessários para um
governo seguro. Em todas as suas iniciativas obteve êxito, apenas lhe
faltou a fortuna, que no caso não é somente sorte, mas a condição
favorável para o exercício do governo, que conforme Maquiavel explicará
no penúltimo capítulo, também é passível de obtenção ou ao menos de
contenção. Neste caso de César, faltou conter os desfavores da fortuna
para conservar suas conquistas.

A figura política de César Borgia é ilustrativa em um duplo sentido:
primeiro, mesmo ele sendo filho de um político poderoso e recebendo dele
os governos de algumas cidades, isso não foi garantia o bastante para o
sucesso de seu governo. Num segundo aspecto, suas próprias ações de
governo lhe trariam a segurança política necessária desde que ele
executasse todas as medidas, inclusive conter os ``ventos desfavoráveis
da fortuna''. Em ambos os sentidos, o que fez dele príncipe não foi
outra coisa senão o exercício de seu governo, o seu principado.

Antes de passar para o segundo exemplo, convém notar o uso de um
vocabulário próprio da engenharia de construção sendo mobilizado aqui.
Com efeito, Maquiavel se vale dessa metáfora das edificações para
retratar essa ação em busca dos sustentáculos do governo: fundação,
edifício, arquiteto, obra, etc. A conquista de governo e sua conservação
é apresentada então, como obra em execução, donde a necessidade de
planejamento, fundamentos, alicerces, etc., obra essa que não se faz ou
se sustenta a partir de uma obra anterior, mas em si mesmo. Mais ainda,
o governo é uma fundação política, ou seja, uma obra nova, que necessita
de alicerces e na qual um novo campo do político está sendo erguido.

O outro exemplo citado é o caso de um cidadão comum (\emph{privato
ciptadino}) que se torna príncipe. Neste caso, a dificuldade para esse
governante é maior, pois a conquista do governo requer \emph{virtù},
embora a fortuna e as armas alheias possam permitir a conquista, mas a
conservação do principado exige toda a demonstração de \emph{virtù} por
parte deste príncipe novo.

A princípio o próprio termo \emph{privato} utilizado aqui por Maquiavel
é de difícil tradução para o nosso contexto discursivo, pois
literalmente \emph{privato} deve ser traduzido por \emph{privado} em
português. No capítulo \versal{IX}, ele usará a expressão \emph{privato
ciptadino} (1) e \emph{ciptadino privato} (20), que poderia ser
traduzida literalmente por \emph{cidadão particular}. Entretanto, o
autor está se referindo aqui, bem como nas demais ocorrências que se
seguem {[}capítulos \versal{VI} (27), \versal{VII} (1, 2, 6), \versal{VIII} (1, 4), \versal{IX} (1, 20), \versal{XI}
(15), \versal{XIV} (3){]}, ao cidadão comum, não pertencente à família do
governante, que se torna príncipe de uma cidade. Esse é o caso mais
emblemático até do que o de César Borgia, visto que ele deve fazer muito
mais esforços políticos para conseguir chegar ao governo e conservar- se
nele.

No primeiro caso em análise, Maquiavel já deixa claro o seu modo de
entender a questão:

\begin{quote}
E porque este evento, de passar de cidadão comum a príncipe,
pressupõe ou \emph{virtù} ou fortuna, parece que uma ou outra destas
duas coisas mitiga, em parte, muitas dificuldades. Todavia, aquele que
menos se apoiou na fortuna, manteve-se mais (\emph{O Príncipe}, cap. \versal{VI}, 5).
\end{quote}

Ora, tornar-se príncipe é fundamentalmente um ato de \emph{virtù}, ainda
que a fortuna mitigue muitas dificuldades, ou seja, ela auxilia a
conquista do governo. Entretanto, essa ação, esse evento deve ser
calcado na \emph{virtù,} nas qualidades ou excelências políticas desse
cidadão que comanda a cidade.

Essa contraposição entre a \emph{virtù} e a fortuna para o cidadão comum
que deseja tornar-se príncipe é ressaltada nos capítulos seguintes,
respectiviamente nos capítulos \versal{VII} e \versal{VIII}, sempre destacando a
instabilidade e a fraqueza que a fortuna gera no processo de conquista e
no futuro governo. Maquiavel ressalta, pois, que ainda que a fortuna se
apresente e justifique muitas conquistas políticas, aquele príncipe que
nela se apoiou tão somente perdeu o seu governo. O argumento é claro e
insistentemente enunciado: para o príncipe novo, é necessário a
\emph{virtù} para a conservação do poder. Em suas palavras:

\begin{quote}
Aqueles que somente pela fortuna de cidadão se tornam
príncipes, com pouco esforço conseguem sê-lo e com muito se mantêm. E
não têm nenhuma dificuldade neste caminho, porque voam para esta
condição, mas todas as dificuldades surgem quando a ela chegam
(\emph{O Príncipe}, cap. \versal{VII}, 1).
\end{quote}

Note-se que ele chega a usar o termo \emph{voam}, evidenciando que este
postulante ao governo pula as dificuldades inerentes à conquista sem dar
conta delas. Essa vantagem dada pela fortuna, inicialmente, não
significa uma maior facilidade no exercício do governo, ao contrário,
por justamente não ter passado pelo exercício e prática da \emph{virtù}
política no momento da conquista, falta a este cidadão comuum o domínio
das qualidades políticas, dessa \emph{virtù} que se deve apresentar já
no início do processo de tomada do governo. As dificuldades do governo
são, evidentemente, de natureza política e, portanto, são as mesmas que
ele evitou nesse processo de conquista.

A \emph{virtù} se manifesta, pois, sob uma série de medidas e
procedimentos que já vinham sendo elencados desde o capítulo \versal{III}, quando
de fato temos a exposição daquilo que o príncipe novo precisa para
conquistar e manter o governo. Contudo, no início do capítulo \versal{VIII}, o
problema para o cidadão comum que deseja ser príncipe ganha intensidade
e chega aos seus contornos mais fortes no capítulo seguinte. A
dificuldade em tela diz respeito ao cidadão que não tem nem toda a
fortuna necessária e nem toda a \emph{virtù}, mas uma mescla e
insuficiência das duas, como diz:

\begin{quote}
{[}1{]} Mas porque há ainda dois modos de se passar de cidadão a
príncipe, o que não se pode atribuir de todo ou à fortuna ou à virtù,
não me parece que deva deixá-las de lado, ainda que sobre uma delas se
possa discorrer mais amplamente em se tratando de repúblicas. {[}2{]}
Estes modos são: ou quando por algum meio criminoso e nefasto alguém
ascende ao principado, ou quando um cidadão comum, com o favor de outros
cidadãos, torna-se príncipe da sua pátria (\emph{O Príncipe}, \versal{VIII}, linha 1 e
2).
\end{quote}

Antes de prosseguir na análise, importa chamar a atenção para a ressalva
acerca das repúblicas, na verdade, a terceira do texto até aqui (as
outras duas foram: no início do capítulo \versal{II}, linha 1 e no capítulo \versal{V},
linha 9). A referência à república diz respeito ao exercício da
\emph{virtù}, para a qual apenas a fortuna não basta para a conquista do
governo. Com efeito, nas repúblicas, a presença da \emph{virtù} para as
ações de governos é imperativa, pois, conforme fica evidente tanto nos
\emph{Discursos} e na \emph{História de Florença}, a vida política na
república é marcada pelos conflitos e pela dinâmica dos humores, que
podem ser apenas geridos com a presença da \emph{virtù}. Aqui a
referência é a necessidae, um tanto quanto parcial, diga-se de passagem,
da \emph{virtù} ao governo do principado.

A exposição do argumento maquiaveliano nesta primeira parte \emph{d'O
Príncipe} vem seguindo uma sequência expositiva de desvalorização da
fortuna em razão da valorização da \emph{virtù} para esse cidadão que
deseja o comando do principado. Neste início do capítulo \versal{VIII},
Maquiavel, uma vez tendo deixado patente que a fortuna não é
sustentáculo para a conquista e constituição do governo principesco,
aponta para os aspectos essenciais dessa \emph{virtù} requerida. Num
primeiro momento, poderia até ser \emph{virtù} o uso de uma violência
desmedida para a conquista do governo, mas isso não seria o bastante e
nem adequado, conforme exposto ao longo do capítulo \versal{VIII}. Após o uso da
violência para a conquista do comando político, esse cidadão deveria
usar os estratagemas próprios da conquista política em qualquer
circunstância: saber reconhecer os humores e atuar no interior da
dinâmica própria dessa oposição política intrínseca à cidade, tema do
capítulo \versal{IX}, como declara de início:

\begin{quote}
Voltando à outra parte, quando um cidadão comum, não por meio de
crimes ou outra violência intolerável, mas com o favor dos outros
cidadãos, torna-se príncipe da sua pátria -- que poderia ser chamada de
principado civil: e para sê-lo não é necessário toda virtù ou toda
fortuna, mas, antes, uma astúcia afortunada --, digo que se ascende a
este principado ou com o favor do povo ou com o favor dos
grandes (\emph{O Príncipe}, \versal{IX}, 1).
\end{quote}

Aqui se mostra a \emph{virtù} principal desse cidadão comum que busca
ascender à condição de príncipe: ter uma astúcia afortunada para que
consiga granjear o apoio dos grandes e do povo, os dois humores
políticos da cidade. Esta é a dificuldade que não se pode saltar, o
problema político a ser enfrentado e crucial no pensamento político
maquiaveliano: conquistar o apoio das partes e saber lidar com elas,
desde o momento da conquista e durante todo o governo. Verifica-se que
tal tarefa é sempre um trabalho em construção, uma obra inacabada, pois
nunca se tem a garantia plena de apoio incondicional a ponto de não se
precisar constantemente renovar e conservar os apoios que se tem, ao
mesmo tempo em que se busca obter novos partidários. O cidadão que
deseja ser príncipe tem a partir desse momento inicial uma tarefa sempre
a executar: inserir-se na dinâmica da vida política e movimentar-se
nela. Ao fazer isso, ele demonstra a sua \emph{virtù} política.
\emph{Virtù} essa também sempre em exercício, seja no sentido de um
aprimoramento, se for possível atribuir isso a ela, seja na sua perda,
quando não a usa bem.

Enfim, esse \emph{privato ciptadino}, por tudo isso que foi apresentado,
não possui nada que o legitime e dê fundamento ao exercício de seu
governo \emph{a priori}, mas tão somente a sua \emph{virtù}, que nada
mais é do que o exercício das qualidades políticas, antes e durante o
exercício do governo. Nesse sentido, o que torna esse cidadão comum em
príncipe é o exercício das qualidades políticas que permitem a ascensão
à condição de príncipe, num primeiro momento, e de conservação do
governo, doravante.

Voltando ao nosso problema que nos levou a analisar o caso de César
Borgia e do cidadão comum que ascende à príncipe, verifica-se que nesses
dois casos emblemáticos do livro, não há qualquer poder político
anterior que possa conferir legitimidade ao príncipe novo -- como é o
caso de César -- e que, no limite, todo governante deve fundar o seu
governo no exercício das suas qualidades políticas, na sua \emph{virtù}
política que lhe assegura a legitimidade.

Portanto, comprova-se como em \emph{O Príncipe,} de Maquiavel, ainda se
conserva esse aspecto herdado da tradição dos ``espelhos de príncipes''
no que diz respeito à fundação do poder político sobre o exercício
próprio do governo, donde um cidadão ser dito príncipe em função do
governo ou do principado que exerce, condição política essa alicerçada
na sua \emph{virtù}.

Outro aspecto apontado como uma herança dos ``espelhos de príncipe'' na
reflexão política maquiaveliana diz respeito ao fato do príncipe
conduzir ou liderar a cidade, pois, conforme visto, esse príncipe
maquiaveliano deve, em função de sua condição e daquilo que o legitima
no governo, reger, conduzir e liderar a cidade, o que fica evidente no
capítulo final de \emph{O Príncipe}.

O convite à liderança política da cidade é dirigido a algum membro da
família Medici, definido como um príncipe novo, embora essa família já
tivesse o comando da cidade de Florença em vários momentos e, quando o
livro foi escrito, não somente a cidade, mas inclusive o papado, estavam
nas mãos dos Medici, o que não permitiria afirmar que qualquer membro
dessa família fosse um príncipe novo. Contudo, ao contrário dessa
primeira impressão, conforme visto, os Medici podem se encaixar
perfeitamente nesse tipo político que é o príncipe novo. Evidência disto
vem pela invocação da imagem militar de seguir a bandeira ou o
estandarte do comandante, como diz:

\begin{quote}
Vê"-se ainda toda pronta e disposta a seguir uma bandeira, desde
que haja alguém que a empunhe. {[}\ldots{}{]}Tome, portanto, a sua ilustre
Casa este assunto com aquele ânimo e aquela esperança com que se tomam
as façanhas justas, a fim de que, sob o seu estandarte, esta pátria seja
enobrecida (\emph{O Príncipe}, cap. \versal{XXVI}, linhas 7 e 29).
\end{quote}

Nas técnicas de combate antigas, a organização das tropas nos teatros de
operações se fazia por meio de bandeiras, haja vista a dificuldade de
comunicação do comandante com os seus soldados. Desse modo, os soldados
eram instruídos a se orientar pelas bandeiras, do seu comandante mais
imediato inicialmente, até, no limite, do comandante supremo presente no
campo de batalha. A bandeira era símbolo de orientação da tropa, donde
os soldados se agruparem onde estava posicionado a sua bandeira, a sua
tropa, e seguirem o rumo que esta tomar. A bandeira ficava sempre ou com
o comandante ou ao lado dele, tanto que tomar a bandeira de um exército
era o sinal de que o comando caiu. Notório que, até nos dias de hoje, em
que essa função de orientação pelas bandeiras não se faz mais necessária
nas técnicas de combate (seja pelos modernos meios de comunicação das
forças militares, seja pelo modo como as tropas se dispõem para as
batalhas), a força simbólica da bandeira como fonte de unidade é ainda
essencial, não somente nos meios militares, como na sociedade como um
todo.

Ora, a invocação para que alguém da família Medici ``empunhe uma
bandeira'' remonta a essa imagem clássica de liderança militar.
Maquiavel roga para que os Medici liderem e conduzam os italianos,
retirando-os da submissão aos povos bárbaros.

Mas essa invocação final dirigida aos Medici é apenas a exemplificação
de uma ideia que se mostra ao longo de todo o texto. A liderança
política do príncipe já era apontada seja nos exemplos históricos, como
o de Moisés, que lidera, conduz e dirige os hebreus a Israel, seja no
modo como esse príncipe se instala na cena política. Como será explicado
adiante, esse príncipe novo de Maquiavel é o governante que lidera, ele
é antes de tudo um \emph{princeps}, que conduz e dá a direção dos rumos
políticos que a cidade deve seguir, mas sem que isso implique em domínio
ou imposição pela força, visto que deve angariar apoios, o que não seria
o caso do governante com poderes absolutos. Convém insistir, o príncipe
será reconhecido também como príncipe na medida em que dirige, aponta os
rumos, não em função de um poder dominador sobre o seu povo, mas em
função da sua condição de liderança e destaque.

Portanto, assim como rei deve conduzir e liderar, tal qual Moisés em
relação aos hebreus, também o príncipe maquiaveliano deve conduzir e
liderar politicamente e militarmente a cidade. Todavia, tal proeminência
se coloca como problema quanto à utilização ou não da força nesse
exercício político. Na verdade esse é um dos temas mais embaraçosos no
pensamento político maquiaveliano, visto que as nuances dessa força que
o príncipe exerce parece sugerir dominação, e, portanto, uma noção de
poder como comando e obediência, contudo, em outros casos, não se trata
dessa noção de força como domínio, mas como potência individual do
príncipe que consegue dirigir e impulsionar os seus comandados (\versal{SASSO},
1988; \versal{REALE}, 1974; \versal{CADONI}, 1994; \versal{LARIVAILLE}, 1997; \versal{FRONSINI}, 2005).

Uma análise, ainda que superficial, das ocorrências de três vocábulos
relacionados ao tema do uso da potência política do príncipe nos fornece
boas pistas de como Maquiavel concebe n'\emph{O Príncipe} esse uso da
violência, da crueldade ou da força e quais são os seus efeitos sobre os
comandados.

Comecemos pelo termo ``violência'' (\emph{violenzia}), que ocorre por
três vezes na obra: cap. \versal{VIII}, 6; cap. \versal{IX}, 1; cap. \versal{XXV}, 12. No capítulo
\versal{IX}, ele fala de uma conquista do principado que não se faça por
``violência intolerável'', como foi o exemplo retratado no capítulo
\versal{VIII}. Neste caso, a violência é reprovada e seu qualificativo já nos diz
tudo. No capítulo \versal{XXV}, que trata da fortuna e do como de controlá-la,
ele está contrapondo a violência à arte, ou seja, trata-se daqueles que
fazem o uso da violência para conter os imprevistos, ao invés de usar de
artifício ou da inteligência. Novamente, a violência não é vista como
qualidade, antes como debilidade daquele que não possui engenho o
bastante para controlar a natureza.

Entretanto, a referência principal ao uso da violência é o exemplo do
capítulo \versal{VIII}, quando, ao tratar de Agátocles (que usou de violência
para conquistar o comando da cidade), diz: ``Ao ser investido em tal
posto (o de pretor), decidiu tornar-se príncipe e manter com violência e
sem obrigação a outrem aquilo que lhe tinha sido concedido por um
acordo'' (\emph{O Príncipe}, cap. \versal{VIII}, 6). Neste caso, parece
que o uso da violência não somente gerou o resultado esperado para
Agátocles, a conquista do governo de Siracusa, como parece ter sido
legítimo e necessário, dando a entender que não há problemas no uso da
violência para a conquista do governo. Entretanto, adiante Maquiavel
apresenta a sua posição sobre esse uso da violência:

\begin{quote}
{[}10{]} Não se pode também chamar de \emph{virtù} matar os
seus cidadãos, trair os amigos, agir de má-fé, sem piedade, sem
religião: meios estes que permitem conquistar poder, mas não glória.
{[}11{]} Porque, se se considera a \emph{virtù} de Agátocles ao entrar e
ao sair dos perigos, e a grandeza do seu ânimo ao suportar e superar as
coisas adversas, não se vê porque ele haveria de ser julgado inferior a
qualquer excelentíssimo capitão: todavia, a sua feroz crueldade e
desumanidade, com infinitos crimes, não permitiram que fosse celebrado
entre os excelentíssimos homens (\emph{O Príncipe}, cap. \versal{VIII}, 10-11).
\end{quote}

Para Maquiavel, a \emph{virtù} de Agátocles não é plenamente
\emph{virtù}, falta-lhe a glória, ou seja, falta-lhe a admiração que o
governante deve produzir de sua condição de liderança política. Ainda
que haja respeito por parte dos comandados, ausenta-se, no caso desse
governante, a boa imagem que deveria se projetar do seu governo. Donde
essa \emph{virtù} não ser plenamente \emph{virtù} e de Agátocles não ser
o melhor exemplo de príncipe novo, justamente por esse uso desmedido e
``intolerável'' da violência.

Enfim, pelos usos do termo violência, ela não se insere no repertório
indicado das ações para o príncipe praticar, seja na conquista do
principado, seja na conservação. Porém, ainda resta o caso da crueldade,
como os assassinatos e as execuções relatadas ao longo do livro, que não
geraram nem perda da condição de comando político daqueles que as
praticou, muito menos a sua condenação.

A resposta neste caso nos remete a um ponto delicado e inovador do
pensamento político maquiaveliano acerca do uso da crueldade nas ações
de governo. Mesmo tendo criticado Agátocles, Oliverotto da Fermo e até
mesmo César Borgia pelo uso da crueldade na conquista e conservação do
principado, ao final do capítulo \versal{VIII} e depois com mais atenção no
capítulo \versal{XVII}, Maquiavel afirma que a crueldade não é todo má para o
exercício do governo, ao contrário, podendo ser muito útil para gerar
temor e respeito. Mas em todos os casos citados e explicitamente no
capítulo \versal{XVII}, a crueldade produz o efeito de exemplo para disciplinar
as condutas e não como prática louvável em si. A crueldade é tratada
mais no efeito que gera sobre os cidadãos, tornando-os temerosos e
obedientes, do que pela força em si mesma da ação cruel. Como diz
Maquiavel ao final do capítulo \versal{VIII}:

\begin{quote}
{[}27{]} Donde é de se notar que, ao pilhar um governo, deve o
invasor fazer todas aquelas afrontas que são necessárias, e fazê-las de
uma só vez, para não ter de renovar tudo e para poder, não as renovando,
tranquilizar os homens e ganhá-los ao beneficiá- los. {[}28{]} Quem faz
de outro modo, ou por timidez ou por mau conselho, sempre precisa ter a
faca na mão; também não pode nunca se apoiar nos seus súditos, nem podem
estes, pelas injúrias recentes e contínuas, jamais confiar nele.
{[}29{]} Por isso, as injúrias devem ser feitas todas de uma só vez, a
fim de que se saboreiem menos e afrontem menos; os benefícios se devem
fazer pouco a pouco, afim de serem melhor saboreados (\emph{O Príncipe},
cap. \versal{VIII}, 27-29).
\end{quote}

Enfim, a crueldade pode ser uma prática de governo a ser adotada, na
medida em que é exemplar, mas não como rotina, porque neste caso ela
desperta o ódio, que é diferente do temor e mais perigoso para o
príncipe, visto que gera opositores aguerridos. Raciocínio muito
semelhante ocorre com os usos do termo ``força'' (\emph{forza}) e seus
correlatos: \emph{forzare} e \emph{forzati}. Das treze ocorrências no
livro\footnote{São elas: cap. \versal{II}, 3; \versal{III}, 50; \versal{VI}, 16, 20, 21, 22; \versal{VII}, 43; \versal{VIII}, 30;
  \versal{XI}, 17; \versal{XVI}, 11; \versal{XVIII}, 2; \versal{XIX}, 37 e\versal{XX}, 20.}, em cinco Maquiavel se refere aos que foram forçados, ou
seja, impelidos, sendo pois passivos diante da força de outrem. Em
outros cinco casos, ele se refere à necessidade da força e nos três
restantes, a noção de forçar, como o sentido de ter que se utilizar da
força para conquistar, ou seja, num sentido positivo de tomar a
iniciativa de forçar. Considerando esses últimos oito casos no qual a
ação de força parte do príncipe ou de seu governo, em apenas três casos
ele tece considerações sobre o uso desta força.

Na primeira ocorrência, cap. \versal{III} (50), é dito que os governantes que
fundam principados para outros, tornando esses poderosos, podem gerar a
sua própria ruína, pois esses novos principados, fundados ou pela força
ou pela astúcia, nunca são confiáveis. Maquiavel está criticando a
transferência de força de um principado para outro, como prejuízo para
aquele que doa esse poder. Para o principado novo, a força recebida é,
evidentemente, um fator positivo, ainda que ele não emita qualquer juízo
a respeito, mas cuja conclusão é óbvia: força política não se transfere.

A segunda ocorrência do termo força, cap. \versal{VI} (22), está inserida em uma
argumentação capital para o que estamos tratando aqui, a saber, sobre a
necessidade da força no exercício do governo. O capítulo tem como tema a
conquista de principados por meio das armas próprias e da \emph{virtù},
momento no qual Maquiavel mobiliza quatro exemplos históricos de
príncipes conquistadores: Moisés, Teseu, Ciro e Rômulo.

A dificuldade que nasce na metade da exposição refere-se à necessidade
desse novo príncipe ter que constituir novos ordenamentos políticos,
fonte de conflitos e oposições. O argumento é evidente: ao ter que
instituir novos ordenamentos políticos, ou seja, reordenar o governo da
cidade, modificando funções, criando novas atribuições, extinguindo-o
outras, retirando algumas pessoas de certos cargos, colocando outras,
nessa ação o príncipe ganha novos inimigos e verifica, segundo
Maquiavel, que os seus amigos não são árduos defensores de seu governo.
É nesse momento de crise, no qual se aumenta a oposição e não se tem uma
base de apoio político confiável que o príncipe deve se valer das armas
para dar sustentação aos novos ordenamentos nos quais está fundando. As
armas se fazem necessárias justamente para forçar a aceitação desse novo
ordenamento político. O fundo do argumento de Maquiavel, também exposto
nos \emph{Discursos}, se apoia no fato de que os homens são volúveis e é
preciso medidas de força, e não apenas o convencimento ou o exemplo da
\emph{virtù}, para que esse novo governo consiga se instalar na cidade.
Nos \emph{Discursos,} ele também se vale da religião como forma de
persuasão para o povo, sem contar que estamos em um contexto republicano
na exposição dessa obra. N'\emph{O Príncipe}, então, a necessidade do
uso da força para a implantação de novos ordenamentos está diretamente
ligada à necessidade de exércitos próprios, o que confere novos
contornos a essa noção. Numa primeira leitura, pareceria que já teríamos
em Maquiavel a apresentação da necessidade da força na figura do
príncipe para o exercício do governo, até mesmo para sua constituição.
Todavia, ainda que seja necessário usar de força, esse uso somente terá
um valor positivo para o governo se ele for acompanhado de \emph{virtù}
e armas próprias.

Ora, conforme ele demonstrará nos capítulos 12, 13 e 14, de \emph{O
Príncipe}, das forças militares a disposição de um príncipe, a mais
adequada e aconselhável são as armas próprias, em outros termos, o
príncipe deve se valer, fundamentalmente, dos exércitos compostos por
cidadãos da cidade em sua maioria. O motivo deste exércitos serem os
mais adequados está na sua composição: são os próprios cidadãos que se
tornam soldados, criando desse modo um vínculo de fidelidade à cidade, à
pátria, que não se encontra com mesma intensidade nas outras formas de
forças militares.

Assim, mesmo no caso do príncipe que precise forçar, sua força deve
estar fundada em armas próprias, ou seja, ele deve possuir uma condição
política de líder dos seus concidadãos, proeminência política essa
apoiada também numa força militar formada por cidadãos.

Não vamos explorar aqui esse aspecto, mas ainda que brevemente, importa
lembrar a importância da dimensão política dos exércitos para Maquiavel,
que foi objeto de sua análise na \emph{Arte da Guerra}, notadamente no
livro \versal{I}. Ele não somente defende enfaticamente essa condição do soldado
ser um cidadão, como também mostra os bons efeitos e a necessidade desse
elementos para a vida da cidade. Na verdade, Maquiavel está retomando um
\emph{tópica} clássica do `cidadão soldado' que remonta a Cícero e a
Vergério, e que teve inúmeros outros defensores entre os pensadores
latinos.

Enfim, conforme fica claro no capítulo 14 do \emph{Príncipe}, o exército
próprio não é somente um instrumento de defesa indispensável, mas ele
também produz um engajamento político muito saudável na cidade. No caso
desse príncipe novo que deseja assumir o governo, se ele tiver essa
força militar para lhe apoiar, isso robustecerá sua condição de líder
político e facilitará seu governo.

Entretanto, essa introdução forçada de novos ordenamentos se apresenta
sempre como um problema, como declara na linha 17 desse capítulo \versal{VI}:
``\emph{E deve-se considerar que não há coisa mais difícil de tratar,
nem mais duvidosa em obter, nem mais perigosa em manejar, do que
fazer-se chefe para introduzir} {[}forçar{]} \emph{novos
ordenamentos}''. Maquiavel não rejeita a força, utilizar-se dela para
introduzir algum ordenamento não é tarefa isenta de riscos e
dificuldades, o que poderia soar como uma contradição, visto que o seu
uso parece redundar no engrandecimento da figura política do governante.
Contudo, esse não é o caso, a força por si só não é benéfica, dependerá
dos seus efeitos ou resultados para que se reconheça seu real lugar para
o governo do príncipe novo.

Ora, se articularmos essa exigência de que os exércitos sejam formados
pelos próprios cidadãos, com o que é apresentado ao final do capítulo
\versal{VI}, verifica-se que essa necessidade de introduzir (forçar) novos
ordenamentos apoiados nos exércitos próprios na fase inicial da fundação
do governo, tudo isso esvazia a ideia de que o príncipe novo detém em si
uma força ou domínio que impõe aos seus súditos. Ainda que Maquiavel
diga explicitamente que o príncipe precise introduzir novos
ordenamentos, tal procedimento se faz apoiado nos exércitos formados a
partir dos cidadãos (e por que não considerá-los como seus
partidários?), para que, uma vez modificados os ``costumes políticos'',
esse príncipe novo seja, então, admirado e consolide o seu apoio. O
texto é claro nesse sentido:

\begin{quote}
É necessário, portanto, querendo discorrer bem sobre esta parte,
examinar se estas inovações se sustentam por si mesmas ou se dependem de
outros, isto é, se para conduzir a sua obra, precisa rezar ou pode
forçar. {[}21{]} No primeiro caso, sempre entendem mal e não leva a
coisa alguma, mas, quando dependem de si próprios e podem forçar, então
é que raras vezes correm perigo. Daqui nasce que todos os profetas
armados venceram e os desarmados se arruinaram. {[}22{]} Porque, além
das outras coisas ditas, a natureza dos povos é variada e é fácil
persuadi-los em uma coisa, mas é difícil sustentá-los nesta persuasão.
Porém, convém ser ordenado de modo que, quando não crêem mais, pode-se
fazer crerem pela força. {[}23{]} Moisés, Ciro, Teseu e Rômulo não
teriam podido fazer observar sua constituição longamente caso estivessem
desarmados, como no nosso tempo sucedeu com o frei Jerônimo Savonarola,
o qual arruinou os seus novos ordenamentos, quando a multidão começou a
não acreditar nele, e ele não tinha como manter firmes aqueles que
haviam acreditado nele, nem fazer crer os descrentes. {[}24{]} Porém,
estes tem grande dificuldade no conduzir, e todos os seus perigos estão
no seu caminho, e convém que os superem com a \emph{virtù}. {[}25{]}
Mas, uma vez superadas essas adversidades, começam a ser venerados,
tendo perdido aquela sua qualidade que lhe tinham invejado, permanecendo
fortes, seguros, honrados e felizes (\emph{O Príncipe}, cap. \versal{VI}, 20-25).
\end{quote}

Note-se que a força se coloca tão somente no momento da fundação
política e não como prática constante de governo, menos ainda como um
atributo que emana da pessoa do príncipe. Nem estamos aqui levando em
conta se essa força se realiza com ou sem crueldade, mas apenas como ela
se apresenta no texto. É evidente, portanto, que essa noção de força
possui um estatuto mais fraco do que terá nas definições de soberania em
Jean Bodin ou mesmo nas definições de governo em Thomas Hobbes. Tal
força do príncipe, conforme demonstrado, não se aproxima também da noção
de \emph{dominium}, ao contrário, visto que nos exemplos expostos o
príncipe tem que levar em conta o jogo das forças contrárias dos humores
para poder obter êxito nessa disputa e assim conquistar ou conservar o
poder. Ou, por outro viés, o modo como a força e a crueldade são
apresentadas no texto, levando sempre em conta a sua real necessidade e
o efeito que geram, revelam que ambas, tomadas em si mesmas, não
perfazem em elementos essenciais e permanentes do exercício do governo.

Enfim, conforme extraímos do texto, as noções de força e crueldade,
principalmente do cidadão comum (\emph{privato}) que se torna príncipe
novo, não podem se apresentar como elementos permanentes e constitutivos
do governo, mas como estratagemas para a fundação e, de modo inusual no
caso da crueldade, para conservação do governo. Donde ser possível
afirmar que, no mínimo, esse príncipe novo de Maquiavel aproxima-se
muito à tradição antiga, na medida em que é o exercício do governo que
faz deste indivíduo príncipe.

Todavia, muitos podem argumentar em contrário, mostrando que justamente
por essas referências à força, à crueldade e às armas poder-se-ia
sustentar que em Maquiavel temos os elementos constitutivos
característicos do pensamento político moderno\footnote{Como já
  mencionado, essa discussão está bem apresentada no artigo de Ames
  (2011).}. Mas ainda assim, tais argumentos, em sua maioria, forçam uma
interpretação moderna do pensamento político maquiaveliano, na medida em
que renegam justamente essas heranças do pensamento político latino
sobre o Florentino.

Finalmente, verifica-se que \emph{O Príncipe,} de Maquiavel conserva
elementos importantes da tradição dos `espelhos de príncipes', que
iluminam e fornecem novos contornos para o que se está sendo exposto.
Por essa nova perspectiva, a obra ganha um contorno menos moderno, se
poderíamos dizer assim, na medida em que esse príncipe é mais próximo de
um cidadão comum que assume a regência e a condução, do que a figura
política do monarca moderno que encarna em si a natureza do Estado.

Essa análise sobre a noção de principado nos conduz a uma dúvida sobre
os tipos de principados, que se dividem em: hereditário, mistos, novos e
eclesiásticos. Haveria ainda o principado absoluto, no qual podem todos
eles podem se transformar. Contudo, destes todos, Maquiavel concentra
mais sua atenção ao principado civil, que é uma forma de principado
novo, não somente na primeira parte do livro, como na segunda, quando
trata da figura do príncipe, considerado sempre como príncipe novo.

Sobre o principado civil, analisado no capítulo \versal{IX}, sabemos que algumas
de suas características já vinham sendo apresentadas antes e o seu
modelo de governante, o príncipe novo, é doravante o personagem político
principal da obra. Caso essa atenção dada a esse tipo de regime político
não fosse o bastante para concentrarmos nossas atenções, o lugar
conceitual desse principado no pensamento político maquiaveliano é
certamente um ponto nevrálgico. No limite, esse modelo de principado
apresenta elementos teóricos que ao mesmo tempo que rompem com os
modelos tradicionais de principados e dos ``espelhos de príncipes'',
obrigam o leitor a considerar melhor qual a verdadeira relação deste
regime com o republicanismo defendido por Maquiavel nas suas outras
obras. Essa dificuldade já foi explorada por vários comentadores que não
é conveniente aqui retomar o debate nos seus detalhes\footnote{A lista é
  de fato extensiva, contudo apresentamos algumas indicações para
  orientar o leitor: Sasso, Cadoni, Martins\ldots{}}. Para nosso
interesse, pretendemos explorar alguns aspectos dessa noção de
principado civil que corroboram a nossa tese de que temos em \emph{O
Príncipe}, de Maquiavel, um texto que se coaduna com o seu pensamento
republicano. Isso não deve implicar em dizer que o principado civil é
uma república, pois são coisas diversas e o próprio Maquiavel
informa-nos disso. Porém, encontramos nesse tipo de principado elementos
de uma dinâmica política que, por um lado, pressupõe que esse regime
nasceu de um governo republicano e que conserva inúmeros traços dessa
forma de governo, podendo (e aqui convém insistir no caráter hipotético
do termo) fazer com que a cidade volte ao regime republicano, embora
isso não seja uma necessidade ou destino, mas uma possibilidade conforme
o rumo dos acontecimentos.

Em uma leitura superficial deste capítulo \versal{IX}, alguns aspectos já chamam
a atenção: a sua denominação como civil -- e o que entende-se aqui por
civil --; o seu fundador é um cidadão comum (\emph{privato ciptadino});
esse principado não é fundado pela violência e nem por crime, mas pelo
consenso; esse príncipe é escolhido (alguns comentadores chegam a
declarar que ele é eleito); é neste principado que Maquiavel relata que
há dois humores antagônicos em disputa; há uma apresentação das formas
de governo (principado, liberdade e licença); há uma referência ao final
à transformação em principado absoluto (que não é analisado na obra, mas
que podemos deduzir o que seja pela arquitetônica do argumento); é o
capítulo mais conceitual e com menos exemplos históricos (apenas um).
Todos esses elementos indicam que se trata de um dos momentos mais
conceituais da obra, de maior elaboração teórica. Passar por todos esses
aspectos seria tema de uma tese, e, novamente, lembrando nossa intenção
inicial, desejamos chamar a atenção para alguns aspectos tão somente.

Inicialmente a própria denominação de principado civil é digna de nota,
visto que civil é um termo que remete à condição de civilidade, por
oposição ao súdito. Como é notório ao longo do capítulo, esse principado
conserva uma dinâmica política na qual os membros da cidade estão
envolvidos com os rumos da cidade, tomando partido na determinação do
governante ou fazendo oposição contra esse, ação política essa
engendrada seja pelos grandes, seja pelo povo. Ora, tal quadro revela
que estamos tratando de um contexto político de cidadãos e não súditos
que se inserem na vida política da cidade, algo totalmente distante de
uma monarquia ou regime autocrático e muito próximo da dinâmica política
republicana. Neste sentido, o termo ``civil'' nos mostra a presença da
civilidade, ou seja, embora seja um principado, ele não anula ou
extingue a iniciativa política dos cidadãos, ao contrário, o governo é
expressão dessa luta, visto ser o príncipe alguém que é escolhido.

Como já dito, esse príncipe aqui é o cidadão comum (\emph{privato
ciptadino}) que torna-se príncipe, o que pressupõe sua condição de
liderança política num contexto institucional de igualdade de condições
políticas entre os cidadãos. Mais ainda, conforme enfatizado por
Maquiavel já na primeira linha, esse cidadão comum se torna príncipe sem
o uso da força, da violência, da crueldade, ou seja, sem o recurso às
armas, logo, utilizando-se de meios pacíficos e dentro da normalidade
institucional para alcançar o governo da cidade. O que não significa que
esse cidadão não seja um comandante militar ou tenha uma força armada
que esteja na retaguarda, mas que ela não é usada e não é o sustento de
sua conquista. Esse cidadão peculiar chegará ao comando da cidade,
principalmente, fundado nos apoios políticos que angariou em seu
processo de ascensão ao governo. Enfim, esse cidadão se torna príncipe
dentro das normas de civilidade, ele conquista o governo de forma
cívica.

Esse aspecto não pode ser menosprezado. Ao longo da história da
humanidade como um todo, mas particularmente considerando a história
política dos romanos e dos povos latinos que de Roma descenderam, ou
restringindo mais ainda, tendo em consideração apenas as cidades da
península itálica dos séculos anteriores a Maquiavel, conquistar o
governo da cidade de forma cívica, sem o uso das armas, da violência era
uma fato, no mínimo, inusual. Nos próprios exemplos arrolados no livro,
verifica-se quantos foram os casos de tomada do governo por meio de
assassinatos, mortes, crimes, golpes de estado e quão poucos os casos de
conquista por vias pacíficas e institucionais. Falar da conquista do
governo de um principado sem derramamento de sangue, calcado na escolha,
é algo muito peculiar.

A forma da instalação do governo é um outro dado chamativo, pois diz
Maquiavel que esse principado é originado dos grandes ou do povo. Mais
específico ainda, de um lado, ``\emph{porque, vendo os grandes que não
podem resistir ao povo, começam a aumentar a reputação e o prestígio de
um dos seus e fazem-no príncipe para poderem, sob sua proteção,
desafogar o seu apetite.}'' (cap. \versal{IX}, linha 3). De outro, temos o povo,
que, não resistindo a esse desejo dos grandes, adere a alguém que venha
defendê-los contra as ânsias dos grandes. O governo, portanto, é
originado, causado pelos grupos políticos, tem nesses sua fundação,
tanto no sentido temporal -- originando-se deles --, quanto no sentido
de sustentação. Nesse aspecto é um governo conduzido por um só, mas que
não tem neste personagem nem sua origem (convém insistir neste aspecto,
visto que Maquiavel usa o passivo ``\emph{è causato}'', é originado por
outro e não por aquele que vai governar) e nem sua fundamentação. Alguns
comentadores (Larivaille, 1997, Frosini, 2005) chegam a dizer que esse
príncipe é eleito pelos seus concidadãos. Apesar da ideia não ser
estranha ao modo como o argumento é montado, Maquiavel não utiliza nem o
termo ``eleição'' e nem ``escolha'', o que nos parece uma extrapolação,
pois nos ritos políticos da Florença do tempo de Maquiavel, o que havia
eram regras muito rígidas sobre aqueles que seriam aptos aos cargos
políticos (as magistraturas) que eram sorteados a partir de nomes
colocados em uma bolsa. Totalmente estranho a esse contexto florentino
seria um processo de eleição, muito menos de votação, donde nosso
incômodo com essa terminologia. O que se verifica pela economia do
argumento é que um cidadão é alçado à condição de príncipe com o
sustento político de uma das partes, num claro sentido de escolha ou
predileção de um grupo que conduz esse personagem ao governo. Tão
importante quanto esse dado é a informação, não explícita, que há um
modo de institucionalização do governo que se funda em uma disputa
cívica e pacífica. O povo não coloca alguém no governo por ter armas ou
por meio da força militar e nem os grandes por algum golpe. Ora, então
somos obrigados a admitir que esse cidadão comum chega ao principado
após uma disputa política institucionalizada, no qual as partes se
apresentam na cena pública e sustentam seus prediletos. Há, pois, um
palco de disputa política no qual as partes podem tomar parte, os grupos
podem defender os seus interesses de modo pacífico e institucionalizado,
enfim, há uma dinâmica política.

A argumentação maquiaveliana nos remete ao fim para o seu personagem
político principal: o príncipe. Mas o que é, ao fim, ao cabo, o
\emph{príncipe} de Maquiavel?

\subsection{O príncipe ciceroniano e o príncipe na república}

A noção de príncipe, conforme visto, teve um largo emprego entre os
pensadores desde o período romano, de modo que podemos encontrar várias
acepções e usos nos autores latinos durante vários séculos. Todavia, uma
questão pertinente seria verificar nessa história do conceito uma
acepção do termo \emph{príncipe} que não estivesse ligada
necessariamente ao regime monárquico ou, por outro lado e mais
importante ainda, se haveria algum emprego do termo, anterior a
Maquiavel, que o diferenciasse dessa acepção mais próxima ao
\emph{vivere civile} e distante do \emph{principado absoluto}. Como
estamos argumentando, tendo em vista o quadro conceitual do pensamento
político maquiaveliano como um todo, com suas nítidas posições
republicanas -- que se corroboram, como visto, pelo papel central do
conceito de cidadão comum (\emph{privato ciptadino}) na economia do
argumento de \emph{O Príncipe} --, cumpre encontrar uma possível fonte
da concepção de \emph{príncipe} que divergiria da acepção corrente de
governante em regime monárquico e o aproximaria de noções republicanas,
para que possamos mensurar o grau de inovação ou recuperação conceitual
operado por Maquiavel.

Uma primeira origem, como já mostramos, está nas noções herdadas da
medievalidade latina do \emph{regere}, no qual o governante tem essa
condição em função do exercício da sua ação de governo, não exercendo o
poder político de forma absoluta. Uma outra fonte que vem se associar a
esta é o uso que o termo \emph{príncipe} apresenta no pensador romano
Cícero e como essa noção pode ter servido de fundamento para a reflexão
maquiaveliana. Ora, não somente essa origem do termo \emph{princeps}
definido por Cícero, mas as aproximações que se podem fazer dessa
acepção nos \emph{Discursos} e em \emph{O Príncipe} corroboram a nossa
hipótese de que Maquiavel forja um conceito muito particular de príncipe
como governante, não necessariamente monárquico e compatível com um
regime republicano, ou um governo a ``meio caminho'' entre a monarquia e
a república.

\subsubsection{O \emph{princeps} ciceroniano}

Ettore Lepore (1954), em seu estudo sobre a noção de \emph{princeps} no
pensamento político ciceroniano, ao recuperar as origens do termo, nos
revela semelhanças conceituais entre esse conceito romano, da tarda
república, com o \emph{príncipe} em Maquiavel, que permitem uma
interpretação diversa da noção de príncipe como uma figura tipicamente
monárquica.

Ao analisar nas obras de Cícero o uso e os contextos nos quais é
mobilizado o termo \emph{princeps} e, até mesmo, quem é esse
\emph{princeps}, Lepore apresenta um conceito pertencente ao ordenamento
republicano romano, notadamente, ao contexto da tarda república (séculos
\versal{II} e \versal{I} a.C.). Dado esse que já nos mostra que, durante a república
romana, fonte de inspiração dos republicanismos posteriores, inclusive o
florentino, havia esse personagem político. Donde se constata que, para
Cícero, há um \emph{princeps} no interior do ordenamento político
republicano, que ocupa um papel destacado na estruturação do regime.

A análise de Lepore tem, também, a preocupação de reconhecer se o termo
é meramente um nome diverso -- talvez numa acepção mais literária -- ou
se ele é um conceito, de acordo com o seu emprego nos escritos políticos
ciceronianos. Seus estudos nos permitem dizer, pois, que esse conceito
de \emph{princeps} se revela em seus contornos definitivos nas três
obras políticas principais de Cícero -- \emph{De officis, De Leggibus} e
\emph{De republica} --, momento esse de maturidade intelectual do
pensador romano e de seu distanciamento dos ideais aristocráticos que
marcaram seus primeiros escritos, como diz: ``\emph{o termo `princeps'
está presente em todo o desenvolvimento do pensamento político
ciceroniano como perfeito equivalente aos outros termos com o qual
designa o homem político}'' (\versal{LEPORE}, 1954, p. 34).

Concentrando suas atenções, primeiramente, às ocorrências do termo
\emph{princeps}, Lepore verifica que Cícero se vale de dois termos muito
próximos: \emph{princeps e principes}. O primeiro é usado de vários
modos: \emph{princeps-rector, gubernator, moderator, tutor, procurator,
conservator} etc. (\versal{LEPORE}, 1954, p. 34-35), acepções estas que revelam a
associação do termo a uma função política de comando ou a um cargo ou
magistratura de relevo no ordenamento republicano romano. Em todas essas
ocorrências, mostra-se ainda o predomínio de duas compreensões do
\emph{princeps}: o primeiro, em ordem cronológica, é o melhor em
comparação a um grupo. Ao lado dessa dupla acepção, encontra-se uma
terceira, pois se verifica que o \emph{princeps} é também aquele que
toma a iniciativa da ação política, aquele que lidera e está na
vanguarda, o responsável pelo princípio da investida política. Tais usos
indicam que, para Cícero, o \emph{princeps} é uma figura política de
proa, que possui uma \emph{virtus} política destacada e, por isso, se
põe à frente na ação política. Esse \emph{princeps} é, portanto, o
primeiro a agir, o primeiro ou líder em uma iniciativa política.

Constatado esse primeiro bloco de acepções correntes do termo
\emph{princeps} nos textos ciceronianos, Lepore parte para uma outra
vertente de investigação, no afã de descobrir as fontes teóricas do
termo. Tendo em vista a tradição filosófica grega, da qual Cícero é
herdeiro, e dos usos do termo, o \emph{princeps} se aproxima em muito ao
\emph{politikós} grego. \emph{Politikós} esse que não é tanto o rei, o
\emph{basileu}, mas o homem que tem sua natureza conformada pela
\emph{polis}, ou seja, o \emph{princeps} tem as mesmas funções e as
mesmas incumbências políticas do \emph{politikós}, ou seja, ele é alguém
que deve partilhar as magistraturas na \emph{polis}. Um exemplo notório
é que os generais, os juristas, os filósofos, os oradores, enfim, homens
que não eram necessariamente governantes e não tinham cargos públicos de
destaque são designados como \emph{princeps} (Lepore, 1954, p. 48-49).
Então, assim como o \emph{politikós} era concebido como o político por
excelência numa comunidade de políticos -- sem necessariamente ser o
governante --, conforme definido por Aristóteles na \emph{Política}
(\versal{III}, 2, 1275b19), do mesmo modo o \emph{princeps} é um cidadão dotado
de virtude política que assume a liderança de uma ação política entre
iguais. Tal uso equivalente dos termos mostram que, ``\emph{mesmo
mudando o âmbito da linguagem filosófica grega,} {[}o emprego
ciceroniano{]} \emph{é assimilado perfeitamente pela experiência
lingüística romana}'' (Lepore, 1954, p. 45). Ao \emph{princeps} se
associa, então, um ideal de homem político que é, de um lado, herdeiro
da tradição grega do \emph{politikós}, e, por outro, agrega as
qualidades ou \emph{virtus} própria do cidadão romano.

Em contraposição, o \emph{principes}, diferentemente do \emph{princeps},
era uma conceito de homem político ligado aos antigos valores
aristocráticos, de um contexto próprio da \emph{concordia ordinum},
quando se buscava uma conciliação entre as ordens patrícias ou
senatoriais, como no governo da \emph{nobilitas}, ou seja, como o ideal
de governo da aristocracia do início da república romana. Nesta acepção,
esse \emph{principes} é um típico aristocrata que se coloca entre
aristocratas, num círculo político seleto e restrito. Ora, quando não
mais se está colocada a questão em termos de ordenamentos
aristocráticos, mas num quadro de intensas disputas e dissensões
políticas, há uma mudança nessa conceituação de \emph{principes} para um
novo modelo político, o \emph{princeps.} A mudança no quadro político,
de um contexto de \emph{concordia ordinum} para a preocupação com o
\emph{consensus}, evidenciada depois de 60 a.C., leva Cícero a rever o
modo de conceber o seu ideal de homem político. A distinção de
\emph{principes} e \emph{princeps,} sendo o primeiro um conceito
tradicional, ligado à aristocracia, e o segundo como a prefiguração de
um \emph{novus hominus}, um novo político, é fruto das novas exigências
políticas após 60 a.C. Avançando ainda mais, o novo \emph{princeps} é
associado ao \emph{popularis,} ao \emph{sapienter popularis} (\emph{De
Republica,} \versal{II}, 54), ou seja, ao cidadão dotado de prudência, tal qual
se diz de Péricles (\emph{De Oratore,} \versal{III}, 138), a clássica figura
grega que encarna a prudência política do governante. \emph{Popularis}
esse que, desde um uso anterior a Cícero, era sinônimo de \emph{civis},
donde constata Lepore (1954, p. 216):

\begin{quote}
O vocábulo popularis assumiu na tradição retórica e naquela mais
antiga e redescoberta, aquilo que o faz simplesmente equivalente de
\emph{civis}, tendo valor estritamente técnico, como o encontramos no
âmbito filosófico, para exprimir o complexo de valores inclusos no
grego \emph{politikós}.
\end{quote}

Segundo Lepore (1954, p. 230), tal mudança se deve a uma percepção mais
isocrática da política romana por parte de Cícero, o que na verdade era
antes uma crença para a superação das antigas \emph{ordines}
aristocráticas, por regimes mais moderados ou mistos. Na verdade, depois
das sucessivas crises políticas da república romana, Cícero, ao
contrário da política defendida pelos patrícios, se posiciona em favor
de mudanças no regime republicano no sentido de diminuição da hegemonia
patrícia e das diversas ordens aristocráticas, para a incorporação de
novos atores políticos vinculados à plebe. Ora, a concepção política
aristocrática, que pressupunha uma ordem estática e não dinâmica do
campo político, estava em cheque após o advento dos conflitos políticos
que resultaram nas várias guerras civis da tarda república romana. Em
face da guerra civil, Cícero passa a admitir uma dinâmica do mundo
político, aceitando o conflito como um fato próprio da vida republicana,
mas propondo uma \emph{contentio sapiens}, que discipline essa luta
política contra a possibilidade da sedição. É sob tal ótica que se deve
ler o livro \versal{VI} do \emph{De Republica}, tendo em vista o dissenso
político que exigem um \emph{princeps moderator} e \emph{prudens,} ou
seja, um novo homem político que reconheça as mudanças e não esteja mais
preso aos modelos estáticos e conservadores próprios do patriciado. As
mudanças contra as quais ele deve reagir são aquelas que ameacem a
destruição dos ordenamentos, ``\emph{frutos dos egoísmos e das paixões
dos homens indignos de serem aceitos nos círculos dos princeps''}
(Lepore, 1954, 251). Nota-se, pois, a adoção de uma nova visão, mais
dinâmica e orgânica da vida política, dos seus elementos e dos seus
contrastes como característica fundamental do pensamento ciceroniano,
particularmente no momento de composição do \emph{De republica,} no
período de seu exílio após o Consulado, na década de 50 a.C.

Então, essas considerações sobre o conceito de \emph{princeps}
ciceroniano impedem qualquer associação desse com um ideal político de
tipo monárquico. Nos inúmeros trechos das obras ciceronianas citadas por
Lepore, fica muito difícil, para não dizer impossível, que esse conceito
fizesse remissão ao governo de um só, ao chefe de um corpo político que
concentra em si os poderes decisórios. Ao contrário, a noção de
\emph{princeps} se associou cada vez mais ao conceito de
\emph{politikós} grego ou de \emph{civis} romano, ao cidadão que toma
parte na vida política da cidade e que por vezes lidera ou dá a
iniciativa da ação política, sem ser o ponto de concentração do poder
político. Como diz: ``\emph{Todos os termos até aqui usados não permitem
identificar o ideal de} princeps \emph{com um poder monárquico ou de
qualquer modo a um singular}'' (Lepore, 1954, 71).

Esse retorno às origens do termo \emph{princeps} em Cícero mostra que
não somente é inadequado caracterizá-lo como uma designação de monarca
ou congênere, mas, ao contrário, seu uso, principalmente nas obras de
maturidade, revela um emprego terminológico muito próximo do cidadão em
um contexto republicano. Figura essa inserida num mundo político não
mais caracterizado pela ordenação estática aristocrática, mas regido
pela dinâmica das disputas políticas, que deve lutar contra as sedições
e consequente dissolução do ordenamento político da cidade.

Portanto, no pensamento político ciceroniano, particularmente nos textos
políticos de maturidade, a noção de \emph{princeps} remete de modo
direto e inequívoco a um contexto republicano e não monárquico. Ainda
dentro desse contexto republicano, como visto, o \emph{princeps} se
identifica a um modelo de novo homem político em um contexto não mais
dominado pela lógica aristocrática do \emph{consensus ordinum}, mas
inserido numa configuração política não dominada pela elite senatorial,
e sim pelas lutas e tensões políticas entre os diversos grupos políticos
da tarda república romana. Ora, esse quadro de significações do termo
\emph{princeps} na tradição política ciceroniana influencia o pensamento
político latino posterior e, certamente, Maquiavel. Desse modo, cumpre
entender as possíveis aproximações dessa terminologia ciceroniana com as
noções de príncipe mobilizadas nos capítulos \versal{VIII} e \versal{IX} de \emph{O
Príncipe}, bem como com os \emph{Discursos}, capítulos \versal{IX} e \versal{X}, no qual
ele também faz um largo uso da noção de \emph{príncipe}.

\subsubsection{O príncipe dos \emph{Discursos}}

Comecemos pelos \emph{Discursos,} onde, nos capítulos \versal{IX} e \versal{X} do
livro, Maquiavel aborda o papel que devem desempenhar os
ordenadores de reinos e repúblicas, mobilizando, para isso, uma série de
exemplos de governantes romanos que tiveram êxito ou fracassaram nesse
trabalho. Tema esse que já havia sido tratado em parte no capítulo \versal{II}
deste mesmo livro \versal{I}, quando da análise da fundação das cidades. É
particularmente nesses capítulos que o governante único ou a figura do
príncipe é citada várias vezes como o responsável pela fundação ou
reordenação política da cidade. Assim, antes de entrar na análise desses
capítulos em questão, faz-se necessário resgatar o contexto
argumentativo no qual eles se inserem.

No ``\emph{Pequeno tratado sobre as repúblicas''}, como já indicado, o
centro da reflexão maquiaveliana é, pois, apresentar os fundamentos das
repúblicas, sua estruturação, para, em seguida, ou seja, tendo como
referência tais concepções, interpretar a história romana. Então, tanto
o capítulo \versal{II} quanto os capítulos \versal{IX} e \versal{X} do livro dos \emph{Discursos}
inserem-se nesse itinerário argumentativo que busca determinar os
fundamentos da república, não somente romana -- seu caso exemplar --,
mas da noção de república de modo geral.

No capítulo \versal{II}, depois de ter analisado a fundação da cidade por meio de
um ordenador (no caso, a Esparta de Licurgo), Maquiavel passa a tratar
do caso romano, no qual lamenta o fato desta não ter tido a mesma sorte
daquela. Ao contrário de Esparta, os ordenamentos romanos nascem dos
conflitos entre os nobres e a plebe, sendo isso caracterizado como acaso
ou como acidentes. Ao lado dessa contraposição, uma outra mais
significativa tem lugar, pois, diz ele:

\begin{quote}
Porque Rômulo e todos os outros reis fizeram muitas e boas leis, ainda
em conformidade com a vida livre: mas, como sua finalidade foi fundar um
reino, e não uma república, quando aquela cidade se tornou livre,
faltavam-lhe muitas coisas que cumpria ordenar em favor da liberdade,
coisas que não haviam sido ordenadas por aqueles reis. E, se bem que
aqueles seus reis perdessem o poder pelas razões e nos modos narrados,
aqueles que os depuseram, ao constituírem {[}ordinandovi{]}
imediatamente dois cônsules para içarem no lugar dos reis, na verdade
depuseram em Roma o nome, mas não o poder régio: de tal forma que, como
só tivesse cônsules e senado, aquela república vinha a ser mescla de
duas qualidades das três acima citadas, ou seja, o principado e
optimates. Faltava-lhes apenas dar lugar ao governo popular: motivo
porque, tornando-se a nobreza romana insolente pelas razões que abaixo
se descreverão, o povo sublevou-se contra ela; e, assim, para não perder
tudo, ela foi obrigada a conceder ao povo a sua parte, e, por outro
lado, o senado e os cônsules ficaram com tanta autoridade que puderam
manter suas respectivas posições naquela república (\emph{Discursos},
\versal{I}, 2, 18-19).
\end{quote}

Como destacam Sasso (1987) e Reale (1985), chama atenção a rápida
passagem da forma monárquica para a forma republicana. Passagem esta que
estabelece a república como o lugar das análises que se seguiram nos
demais capítulos. Mas, de qualquer modo, o Rômulo citado no capítulo \versal{II}
é uma figura monárquica que deu os fundamentos para a constituição de
uma república ou de um ordenamento político conforme o \emph{vivere
civile}. Por outro modo, caso se queira considerar que Maquiavel tecia
suas considerações levando em conta uma monarquia, estas se encerram
neste momento final do capítulo \versal{II}.

Entretanto, tendo em vista isso que foi dito no capítulo \versal{II}, a
mobilização de Rômulo no capítulo \versal{IX} ganha novos contornos e
dificuldades. Neste capítulo dos \emph{Discursos}, ao retomar o papel de
Rômulo na dinâmica política romana, a perspectiva de análise é outra,
equiparada a de um fundador em um contexto de disputa política, com
vistas à república. A questão central está em como entender esse
governante único, esse primeiro monarca romano no meio de uma análise
voltada para o estabelecimento dos fundamentos da república. Ora, também
no capítulo \versal{IX}, em seu início, Rômulo é apresentado como um fundador de
cidades, não se levando em conta a dificuldade que perpassa as
considerações sobre o nascimento da república romana.

Porém, tanto no capítulo \versal{II} quanto no \versal{IX}, essa transição em si não é
problematizada, não é analisada a fundo. Concomitantemente a esse pouco
falar ou mesmo não falar da transição constitucional, o regime
republicano se apresenta como um \emph{telos}, uma finalidade à qual
Roma parecia destinada. Apesar da imprevisibilidade sobre o que seria no
futuro, Roma apresenta-se destinada a se transformar numa república,
como se ela estivesse orientada para tanto desde seus momentos
primordiais. Segundo Sasso: ``Como se fosse, na realidade, o
\emph{telos} a constituir, além do fim e ao fim do processo, também o
seu critério, a sua origem, a sua razão de ser, o seu impulso condutor''
(\versal{SASSO}, 1987, p. 128). Assim, à débil análise da transformação
política se contrapõe a profundidade de uma necessidade teleológica de
Roma se tornar uma república. A busca dessa condição política
republicana instala-se na reflexão maquiaveliana e passa a conduzir o
seu raciocínio.

Os argumentos mobilizados, então, têm como objetivo a transformação de
uma monarquia, que nasce tumultuada pelo assassinato de Tito Tazio e
pautada por tumultos, numa república completa em meio às vicissitudes. A
questão que nasce da análise do capítulo \versal{II} e da complementação do \versal{IX}
era que Roma tinha como destino não a instauração de uma monarquia
perfeita, mas de uma república. No quadro apresentado por Maquiavel,
desde o seu nascedouro, Roma estava destinada a se transformar numa
república, pois os eventos convergiam para esse fim. Como ele repete ao
longo desses capítulos, quando se olha para o \emph{fim} e não para o
ato em si, a instauração de um \emph{vivere libero} esteve sempre no
horizonte. Esta era uma motivação encontrada já nos primeiros reis, ou
seja, os ordenamentos políticos iniciais tinham como força indutora a
instalação de um \emph{vivere libero}, forma essa que se completará ou
se realizará perfeitamente no modelo republicano:

\begin{quote}
Mas voltemos a Roma. Embora Roma não tivesse um Licurgo que no princípio
a ordenasse de tal modo que lhe permitisse viver livre por longo tempo,
foram tantos os acontecimentos que nela surgiram, devido à desunião que
havia entre a plebe e o senado, que aquilo que não fora feito por um
ordenador foi feito pelo acaso. Porque, se Roma não teve a primeira
fortuna, teve a segunda; pois se seus primeiros ordenamentos foram
insuficientes, nem por isso o desviaram do bom caminho que a pudesse
levar a perfeição (\emph{Discursos}, \versal{I}, \versal{II}, 30-33).
\end{quote}

Estabelece-se, assim, o quadro conceitual das afirmações feitas por
Maquiavel nas primeiras linhas do capítulo \versal{IX}, nas quais se insere a
análise dos fundadores na sequência da exposição sobre as instituições
políticas, aparentemente não apresentando nenhuma relação com os temas
tratados anteriormente. O que se mostrava inicialmente no capítulo \versal{II}
era uma reflexão sobre a fundação por meio do legislador e suas
possíveis implicações sobre a história romana. No capítulo \versal{IX}, apesar de
retomar a temática da origem constitucional, a chave de leitura não é a
fundação, mas a ordenação, que, do ponto de vista da compreensão da
estrutura política romana, apresenta uma outra perspectiva, visto que já
se considera o substrato material do conflito político.

Tributária dessa compreensão é a figura de Rômulo, que, no capítulo \versal{II},
se assemelha à figura de Licurgo. Nessa tentativa de traçar um paralelo
entre as duas personalidades, Maquiavel ressaltava as carências do rei
romano em comparação com o legislador espartano. No capítulo \versal{IX}, ao
contrário, o que nasce é a figura de um outro Rômulo, não mais a versão
romana e imperfeita de legislador conforme descrição anterior, mas o
responsável pela instalação de um processo de ordenamento constitucional
que fará de Roma uma república poderosa.

Conforme Reale (1985, p. 45), o que seria uma aparente questão retórica,
ganha os contornos de uma questão real, se levarmos em conta o fato de
Maquiavel ter tratado apenas dos fundadores de cidades e não dos
reformadores, o que, no limite, impõe a questão da fundação e da reforma
ou reordenação da cidade. O problema parece não estar restrito à
temática da fundação das cidades, muito menos a uma retomada do papel do
legislador nesse momento inaugural. A afirmação maquiaveliana evidencia
uma sutileza terminológica que se configura como um problema de fundo.
Maquiavel fala em termos de \emph{ordenadores} e não de
\emph{fundadores} ou \emph{legisladores}, que, em um primeiro momento,
poderiam ser compreendidos como sinônimos ou como possuidores de função
igual na origem das cidades. O início do capítulo vem aprimorar a
compreensão do papel do ordenador, que passa a se diferenciar do
legislador, como foi Licurgo (\versal{REALE}, 1985, p. 46). O exemplo seria
Rômulo que, ao ser identificado como o ordenador de Roma, ao mesmo tempo
se diferencia dos fundadores, expostos de início. A figura do fundador
foi apresentada como aquele que concebia a cidade e suas instituições
por critérios racionais, dotando este universo político de mecanismos
estáveis e seguros em função de sua própria racionalidade. Essa
racionalidade, marca distintiva da ordenação designada pelo legislador,
contrapõe-se à ordenação segundo o acaso. Levando-se em conta o
legislador helênico, vê-se que ele disporá a cidade segundo um
``critério geométrico, segundo uma regra racional, em um verdadeiro
cosmo de leis'' (\versal{SASSO}, 1987, p. 121)\footnote{Essa imagem do ordenador
  helênico concebendo a cidades segundos critérios racionais e
  geométricos é um dos pontos centrais da argumentação de Jean-Pierre
  Vernant. Em sua explicação, a racionalização da vida e a
  racionalização do campo político estão extremamente imbricadas no
  mundo grego, cujo melhor exemplo seria Hipodamo de Mileto, contratado
  para reconstruir a sua cidade, o fazendo de modo geométrico, donde
  tudo ser ordenado a partir do centro que é a \emph{ágora}. Como nos
  diz Vernant: ``Ele a reconstrói segundo um plano de conjunto que marca
  uma vontade de racionalizar o espaço urbano.'' Concluindo: ``Ora,
  deve-se constatar que o domínio político aparece tão solidário de uma
  representação do espaço que acentua, de maneira deliberada, o círculo
  e o centro, dando-lhe um significado muito definido. {[}\ldots{}{]} A esse
  respeito, pode-se dizer que o advento da Cidade manifesta-se de início
  por uma transformação do espaço urbano, isto é, do plano das cidades.
  É no mundo grego, sem dúvida, primeiro nas colônias, que aparece um
  \emph{plano novo} {[}grifo nosso{]} de cidade em que todas as
  construções urbanas são centradas ao redor de uma praça que se chama
  ágora. {[}\ldots{}{]} Para que exista uma ágora é preciso um sistema social
  de vida implicando, para todos os negócios comuns, um debate político.
  A existência da ágora é a marca do advento das instituições políticas
  da cidade'' (\versal{VERNANT}, 1995, p. 245).}. Isso permite afirmar que
haveria, de um lado, uma ordenação conforme o \emph{logos} e, de outro,
uma ordenação mediante a fortuna, o que não significa que o \emph{logos}
esteja excluído da fundação das cidades que não tiveram a sua origem
pela mão do legislador, mas apenas que não é esse seu critério
prioritário. Ao final do capítulo \versal{II}, Maquiavel se refere a Rômulo,
mesmo não podendo equipará-lo a Licurgo, como aquele que fez ``muitas e
boas leis, conforme ao \emph{vivere libero}'' (\emph{Discursos}, \versal{I}, \versal{II},
32), ou seja, o primeiro rei romano teve também a intenção de bem
conformar a cidade. Do ponto de vista do projeto, Rômulo também figura
entre os fundadores da cidade, embora nesse momento do texto não fosse
ainda possível perceber se Maquiavel falava de um legislador ou de um
ordenador.

Entretanto, paralelamente a essa fundação conforme o \emph{logos},
Maquiavel chama a atenção para um outro modo de ordenação política da
cidade feita pelos acidentes. O termo ``acidente'' sugere várias
acepções, entre elas imprevistos ou acontecimentos não regulares que
alteram o curso político, bem como, acidente como oposto à essência,
retomando um vocabulário aristotélico. O desenvolvimento do texto tende
a reforçar o primeiro aspecto, haja vista o papel dos tumultos políticos
para a instauração do ordenamento constitucional. Todavia, levando-se em
conta que a fundação pelo legislador é conforme o \emph{logos}, sendo
isto uma busca de modelação da natureza do corpo político conforme a
razão, em tais condições, os tumultos, os acidentes também podem ser
compreendidos como algo que se insere no corpo e o modifica, à
semelhança de uma forma acidental ou causa acidental. Associação que não
deveria ser absurda, pois a tradição aristotélica medieval desenvolveu
tanto o conceito de forma acidental como o de forma substancial, que não
está presente na \emph{Metafísica} de Aristóteles, fazendo deles
conceitos-chave para os sistemas metafísicos medievais, principalmente
depois de Averróis e Tomás de Aquino. Ora, mesmo sabendo não ser muito
adequado utilizar um jargão tributário desses sistemas metafísicos nos
textos políticos maquiavelianos (tendo em vista a ausência de uma
reflexão metafísica por parte de Maquiavel), essa acepção de acidente no
sentido de causa acidental pode ser aceitável se considerarmos as
implicações de uma fundação conforme o \emph{logos}. Pensando na
contraposição evidente entre \emph{logos} e acidente, esses fatos que
alteram a ordenação da cidade e inserem algo de novo em sua natureza --
como é o caso dos tribunos da plebe -- podem também ser compreendidos
como formas acidentais. Independentemente da interpretação que se queira
adotar para a compreensão do termo ``acidente'', o que essa retomada do
papel do ordenador no capítulo \versal{IX} vem problematizar é seu estatuto para
a compreensão da formação das instituições. Conhecido desde o início do
livro o papel do legislador e do \emph{logos}, bem como os problemas
decorrentes de uma tal fundação, as atenções se voltam para a
necessidade de se entender os acidentes na fundação de uma cidade.
Ordenação por acidentes que, ao ser pensada no confronto com a fundação
racional do legislador, adquire novos contornos.

O trabalho realizado pelo legislador de conformar as instituições
políticas segundo um critério racional tem, como uma de suas
consequências, a perenidade dessa instituição. Nesse sentido, a
conformação segundo o \emph{logos} pretende retirar da esfera temporal
as constituições políticas (\versal{SASSO}, 1987, p. 120-132). De fato, a
constituição perfeita, o governo misto, coloca-se fora da circularidade
temporal, perfazendo uma linearidade. Por ser um ordenamento político
acabado, pode-se chamá-lo de perfeito, entendendo-o não como o melhor
dos regimes, mas como aquele que não carece de nada, conforme a
conceituação clássica grega. Tal é a constituição política que nasce do
trabalho do legislador.

Já quanto à fundação ordenada pelo acaso, diferentemente da perfeita,
ela se define por não estar acabada e, consequentemente, por estar
submetida às vicissitudes do tempo, à fortuna. Como demonstra Maquiavel
no capítulo \versal{II} dos \emph{Discursos}, teríamos uma gradação de fundações
em três níveis: a perfeita, a ``menos perfeita'' (mas com possibilidade
de reforma), e uma terceira classe de fundações imperfeitas, sem nenhuma
possibilidade de reforma. Roma encontrar-se-ia nesse segundo grupo,
sendo uma constituição a princípio imperfeita, mas que ao longo do tempo
foi se aperfeiçoando, ou seja, foi se reordenando. Os acidentes, nesse
contexto, são os atributos que aperfeiçoam o regime, que o conduzem à
perfeição; são os agregados que se unem a um corpo político
pré-existente e o modificam. Assim, a perfeição (como ``acabamento'' e
não como ``ausência de defeitos'') pode ser possível para esse segundo
grupo, que não fica refém de um determinismo naturalista que impede que
uma constituição imperfeita se transforme numa perfeita. A
perfectibilidade não é um dado inserido apenas no momento de criação do
regime, mas pode ser, para Maquiavel, uma possibilidade ao longo da
existência, sujeita ao acaso.

A garantia dessa perfectibilidade joga essas constituições inacabadas
para a esfera do tempo, submetidos que estão à fortuna. Embora seja
possível alcançar a perfeição, ela se realiza num quadro de dependência
relacionado à esfera temporal, às subidas e descidas, sem
previsibilidade. Se a fundação segundo o \emph{logos} retira o regime
perfeito das variações temporais, a ordenação segundo os acidentes
insere totalmente o corpo político na História, no tempo. É na História,
no interior do tempo, que essa constituição se perfaz, se aperfeiçoa,
agrega a si aquilo de que carece, no caso, as instituições que a tornam
perfeitas. Essa constituição imperfeita está sujeita, também, à dinâmica
do tempo, à variação dos fatos, dos acidentes e, por isso, deve estar
aberta às mudanças, disposta a incorporar aquilo que o tempo lhe traz
como novidade.

Assim, não é meramente uma questão terminológica a distinção entre a
fundação de uma cidade pelo legislador e uma ordenação segundo os
acidentes. Ao expressar que Roma teve uma ordenação e não uma fundação,
Maquiavel demarca o campo teórico no qual deve ser pensada a
constituição romana, que de nenhum modo pode ser equiparada às
repúblicas conformadas por um legislador.

Retomando a questão de como entender esse príncipe dos capítulos \versal{IX} e \versal{X}
do livro \versal{I} dos \emph{Discursos}, temos agora a constituição de uma outra
imagem que não somente a do governante único. Conforme se depreende da
análise da figura de Rômulo, o governante único desses capítulos está
inserido em um contexto republicano, marcado por lutas e dissensões
políticas entre os dois grupos principais. Importa frisar: nas duas
referências ao rei romano, Maquiavel parece ignorar a passagem da
monarquia para a república, e passa a tratar de Roma num contexto
republicano. Roma, nesses capítulos, é considerada em sua fase
republicana, que, na verdade, esteve no meio de duas formas de governos
centralizados: a monarquia anterior à fundação republicana e o império,
posterior a essa fase. Mesmo quando se faz referência a César no
capítulo \versal{X}, onde Maquiavel expõe toda a sua crítica a ele, o contexto
político é republicano e as críticas se devem em grande parte por ter
César contribuído para a destruição da república. Esse governante único
que foi César, ao invés de recuperar o \emph{vivere civile}, aprofundou
a dominação e retirou o pouco que restava de liberdade da república.

Então, os governantes desses capítulos, tendo em vista as circunstâncias
políticas nas quais estão inseridos e o papel político que devem
desempenhar -- o de ordenar ou reordenar um regime --, são considerados
em termos de governantes executivos em condições republicanas. Logo, não
parece existir a possibilidade de qualificá-los como típicos governantes
monárquicos, que centralizam o poder político na figura do chefe de
governo. Mesmo quando considerados unicamente como reordenadores,
Maquiavel enfatiza que eles devem agir em vista do restabelecimento dos
ordenamentos republicanos. Com efeito, ao declarar que ``um ordenador
prudente e virtuoso não deve deixar por herança a autoridade que tomou''
(\emph{Discursos}, \versal{I}, \versal{IX}, linha 8), que remeteria à importância
de um governante único, ele destaca que a \emph{herança} não pode ser
essa autoridade excepcional, mas cuidar para deixá-la nas mãos de
muitos. Ou seja, mesmo que haja uma reordenação por meio de um só, o
resultado deve ser a instalação do governo de muitos. Neste sentido, o
fim de toda a ordenação visada nessas passagens dos \emph{Discursos} é
um regime republicano e não a perpetuação de uma dinastia.

Essa preocupação com o \emph{vivere civile} é tão importante que
Maquiavel insiste neste mesmo capítulo \versal{IX} e no \versal{X} sobre a ameaça de
instalação de um governo absoluto ou tirânico. Nesses é que se encontra
o grande temor: que, após a reordenação de um regime, o poder fique nas
mãos de um só homem ambicioso que usaria mal aquilo que virtuosamente
foi conquistado (\emph{Discursos}, \versal{I}, \versal{X}, linha 8-10). O mesmo se aplica
a César, que não foi o reordenador da república, mas o seu destruidor,
ao contrário de Rômulo (\emph{Discursos}, \versal{I}, \versal{X}, linha 30).

Então, agora pode ser possível entender porque esse príncipe dos
capítulos \versal{IX} e \versal{X} não é um típico monarca, mas, quando muito, um
reordenador de um regime com vistas à recuperação dos ordenamentos
republicanos. Ele é antes de tudo um líder, um precursor de um processo
político, muito próximo ao ideal de \emph{princeps} de Cícero. Como ele
mesmo diz: ``Aquele que se tornou príncipe nalguma república deve
considerar que, depois de Roma tornar-se Império, mais merecem louvores
os imperadores que viveram de acordo com as leis e como príncipes bons,
do que aqueles viveram de modo oposto'' (\emph{Discursos}, \versal{I}, \versal{X}, linha
16).

Enfim, Maquiavel se vale, nesses capítulos, de uma figura de príncipe em
contexto republicano, o príncipe como um reformador de regime, que não
parece se identificar de nenhum modo com a imagem tradicional de monarca
soberano que centraliza o poder político, muito menos com um governante
tirânico, que é a figura contrária desse príncipe.

Enfim, tanto os paralelos suscitados pela noção de \emph{princeps}
ciceroniano, quanto esse uso da noção de príncipe nos primeiros
capítulos dos \emph{Discursos} nos mostram que o termo se distancia
muito de uma acepção autocrática ou monárquica e se aproxima da figura
política do líder que conduz a cidade, seja reordenando as instituições,
seja fundando novos ordenamentos, estabelecendo uma nova dinâmica
política.

Retornando a \emph{O Príncipe}, quando consideramos aquilo que
Maquiavel atribui ao príncipe civil do capítulo \versal{IX}, conforme já
expusemos (p. 73), mas que convém retomar:

\begin{itemize}
\item
  \begin{quote}
  em quase todos os casos apresentados, à exceção do príncipe herdeiro,
  o príncipe de Maquiavel tem o sua condição de governante fundada antes
  no seu exercício político do que no território ou reino;
  \end{quote}
\item
  \begin{quote}
  o príncipe deve, em função de sua condição e daquilo que o legitima no
  governo, reger, conduzir e liderar a cidade -- algo que ganha
  contornos claros e dramáticos no capítulo final do livro;
  \end{quote}
\item
  \begin{quote}
  No caso do uso da violência, apesar de ser apresentado como
  estratagema para a conquista da condição de príncipe, Maquiavel deixa
  claro que ela não é \emph{virtù,} e sua mobilização no texto não tem a
  mesma justificação que terá posteriormente o argumento do uso legítimo
  da violência pelo Estado. Esse, na verdade, é um dos pontos pantanosos
  de \emph{O Príncipe}, pois, ainda que não tenhamos a mesma
  caracterização da violência estatal da modernidade, ela se faz
  presente como elemento da ação política do príncipe.
  \end{quote}
\end{itemize}

Ora, evidencia-se, enfim, que, nos usos e acepções dados particularmente
ao príncipe civil e nos atributos necessários a este (conforme é exposto
nos capítulos seguintes até o fim da obra), essa personagem não se
indentifica com o monarca, mas com a figura de um lider político em
contexto de disputa política que, se não é de fato republicano, é um
palco no qual o conflito político e a disputa entre as forças
antagônicas se fazem presentes.

Mais ainda, recuperando a noção política do \emph{regere} e como
Maquiavel apresenta a figura do príncipe civil como uma conquista do
cidadão comum (\emph{privato ciptadino}), tudo isso configura de outro
modo esse personagem político central, não somente em \emph{O Príncipe},
mas para o pensamento maquiaveliano. Por todos esses elementos expostos
aqui, verifica-se que o príncipe é uma figura política muito próxima,
senão identificada, ao cidadão (\emph{politikós} ou \emph{civis}), que
assume a liderança política da cidade, tendo como desafio principal
fundar ou reordená-la institucionalmente.

Contudo, convém destacar que Maquiavel está o tempo todo tratando de um
tipo de príncipe, o príncipe civil, que não é o único exemplo e nem um
modelo de conduta política que pode ser universalizado. Nos vários
escritos políticos, ele deixa claro e cita que um príncipe pode se
tornar num tirano e fundar um regime centralizado nele. Enfim, no
pensamento político maquiaveliano não há só esse príncipe civil sendo
retratado, mas é este certamente a figura política central para sua
obra.

\subsubsection{O príncipe civil entre república e principado}

Neste ponto, poderíamos dar por concluída nossa exposição, porém, na
análise do príncipe civil nasce uma última dificuldade. Em todas as
circunstâncias apresentadas aqui desse cidadão que se torna príncipe, há
um pano político de fundo de uma república (donde tratarmos de
``cidadão'' e não de ``súdito''), que parece passar por crises, gerando
a necessidade desse cidadão dotado de \emph{virtù} para liderar a cidade
nessa tarefa de fundação ou reordenação política. Esse é o quadro
político, seja nos \emph{Discursos,} livro \versal{I}, capítulos \versal{IX-X} e
\versal{XVI-XVIII}, seja em \emph{O Príncipe}, na primeira parte do livro,
particularmente, nos capítulos \versal{VIII} e \versal{IX}. No limite, a dificuldade pode
ser enunciada nos seguintes termos: quais são as condições políticas que
geram a necessidade da fundação do principado por esse \emph{princeps}
que é um cidadão dotado de \emph{virtù}? Isso nos remete a recuperar a
questão de como pensar a relação entre a república, modelo político do
qual nasce essa figura do príncipe, e o principado? Não somente como de
uma se passa a outra, no caso da república ao principado, mas, também,
se seria possível que um principado pudesse se tornar uma república.

Associar os capítulos dos \emph{Discursos} que tratam da corrupção
(capítulos \versal{XVI}, \versal{XVII} e \versal{XVIII}) com \emph{O Príncipe},
particularmente os capítulos \versal{VIII} e \versal{IX}, é uma hipótese que pode apontar
para a compreensão da solução daquilo que Maquiavel define como a cidade
corrompidíssima. Conforme diz Sasso:

\begin{quote}
Entre o nono capítulo do \emph{Príncipe} e os capítulos dezesseis,
dezessete e dezoito do primeiro livro dos \emph{Discursos}, existe uma
relação sutil e complexa {[}\ldots{}{]} No nono {[}capítulo{]} do
\emph{Príncipe}, a premissa do raciocínio e da análise teórica, é
fornecida por uma forma republicana que, em vista do `excessivo'
conflito dos `humores', o prevalecer dos `grandes' e, paralelamente, o
desencadear-se das paixões populares estão, depois de te-las restituídas
numa `igual' desigualdade, sempre `aprofundando' mais na corrupção
(\versal{SASSO}, 1987, p. 396-397).
\end{quote}

Seja numa relação de gênese, seja numa relação de consequência, as
imbricações entre essas obras necessitam de maiores considerações, a fim
de que se chegue aos limites reais dessa relação. Tanto na perspectiva
de causa quanto na perspectiva de consequência, o importante é entender,
primeiramente, os termos da relação entre república corrompida e
principado civil. Como também diz Sasso, mais importante que o ângulo
visual, deve-se destacar o contexto da cidade corrompidíssima dos
\emph{Discursos} que precede ou está num

\begin{quote}
{[}\ldots{}{]} tempo anterior àquele considerado no \emph{Príncipe}, no
qual a passagem à forma monárquica ou principesca é já, por assim dizer,
considerada atual e imanente ao consenso que os ``populares'' e os
``grandes'' concederam à iniciativa do ``privado'' na cidade
corrompidíssima (\versal{SASSO}, 1987, p. 398).
\end{quote}

Com efeito, conforme apontamos antes sobre as características do
principado civil, as condições que permitem a um cidadão tornar-se
príncipe num principado civil estão já presentes na cidade republicana
dos \emph{Discursos}. A presença dos dois humores, seus desejos
antagônicos, o conflito político entre ambas, a necessidade do
governante em não se apoiar nos grandes, mas saber controlar os seus
desejos, enfim, todas essas condições que pautam as circunstâncias da
existência do principado civil já estavam presentes na república que se
encaminha para a corrupção.

Retornando ao capítulo \versal{XVIII} dos \emph{Discursos}, ao levarmos em conta
o projeto que Maquiavel tem em vista -- reordenar a cidade
corrompidíssima --, o indivíduo que assume esse empreendimento assume
também para si uma autoridade que compete ao príncipe. Seja ele um
ditador ou um \emph{gonfaloniere}, seja ele o primeiro cidadão dotado de
\emph{virtù}, o ponto central é que a esse indivíduo deve-se atribuir um
tal poder \emph{extraordinário,} estranho ao ordenamento republicano e
muito próximo ao príncipe que assume um principado civil. Ora, se a
solução é extraordinária, a questão se desloca para a transição dos
regimes, ou, em outras palavras, a dificuldade está em como pensar essa
passagem de uma república corrompida para o principado civil.

Antes de tratar da passagem, convém destacar o tipo de principado que se
tem em vista. Estamos tratando do principado civil como o regime
político proposto como solução, mas não seria essa a única opção, pois,
por outro lado, nada impediria o estabelecimento de uma tirania ou de
algum tipo de governo despótico e autoritário. Os regimes tirânicos (ou
nem tão tirânicos, como o governo do Turco, que era um governo
centralizado, mas não necessariamente tirânico ou absoluto) não são
sugeridos em nenhum momento como a solução mais adequada ou a mais
consequente para as condições republicanas. O problema parece ser que
esses principados, que podem ser identificados como absolutos, negam ou
anulam os conflitos políticos ocasionados pelos humores presentes na
cidade. Ao concentrarem todo o fundamento da ação política no
governante, impedem o ``natural'' funcionamento da vida política e, por
consequência, impedem que os grupos ou os humores manifestem seus
desejos pelo meio natural de luta política dentro do corpo político,
aquilo que caracteriza a civilidade política. O principado civil,
conforme visto, é aquele regime que mais assegura o \emph{vivere
libero}, que respeita e garante os conflitos, pois os assume como
inerentes à vida política no principado. Esse será o ponto central: ao
contrário do principado de tipo absoluto, o principado civil conserva os
aspectos básicos da vida política numa república, não anula por completo
o \emph{vivere libero}, a civilidade, o jogo político e os conflitos que
lhe são inerentes; antes os reconhece e os assume como dados essenciais
do principado. O maior problema em se considerar a transição de uma
república corrompida para um principado de tipo absoluto é a
possibilidade de subtração completa das características presentes na
primeira, não mais reconhecidas e existentes nesse tipo de regime. Logo,
o principado de tipo absoluto, apesar de ser uma solução possível, não
pode ser compreendido como a mais adequada para uma cidade que necessita
ou \emph{conservar-se como republica ou criá-la de novo}, que é o
problema central do capítulo \versal{XVIII} dos \emph{Discursos}.

Neste sentido, convém retomar o que Maquiavel expõe no capítulo \versal{X} do
livro \versal{I} dos \emph{Discursos}, quando trata dos reformadores de Roma.
Paralelamente a sua crítica a César, visto como um dos principais
destruidores da república romana, a análise que se segue dos imperadores
romanos visa ressaltar, fundamentalmente, que: aqueles que alcançaram o
império por herança foram maus, ao contrário daqueles que o assumiram
com o apoio dos seus concidadãos (\emph{Discursos}, \versal{I}, \versal{X}, linha 20); que
os \emph{príncipes} (termo do próprio Maquiavel para se referir aos
imperadores que se seguiram a César) que procuraram reordenar o reino e
fazer com que as instituições funcionassem conforme a sua finalidade,
foram mais bem sucedidos em relação àqueles que procuraram, por meio
delas, conquistar glória para si. Donde conclui:

\begin{quote}
E o príncipe que realmente buscar a glória mundana deverá desejar ter
nas mãos uma cidade corrompida, não para destruí-la de todo, como César,
mas para reordená-la, como Rômulo. E realmente, os céus não podem dar
aos homens maior ocasião de glória, nem os homens podem desejar glória
maior. E, se, para bem ordenar uma cidade, houvesse necessidade de depor
o principado, mereceria alguma desculpa quem não a ordenasse para não
cair de tal posição, mas, em sendo possível manter o principado e
ordena-lo, não merece desculpa algum quem não o faça (\emph{Discursos},
\versal{I}, \versal{X}, 30-32).
\end{quote}

Não está descartada, seja nos \emph{Discursos}, seja em \emph{O
Príncipe}, a possibilidade das repúblicas se transformarem em tiranias
ou que os principados tornem-se regimes despóticos. Essas
degenerescências políticas, se pudermos chamarmos assim esses regimes
autocráticos, não estão nas atenções de Maquiavel. Ele está preocupado
com a solução para a recuperação da civilidade política, donde ser o
principado civil o regime que melhor se adequa a esse fim. O modo como é
ordenado o principado civil permite, pois, considerá-lo como o regime
que melhor prepara o povo ou o universal (também denominado como matéria
em vista de uma forma) para o \emph{vivere libero} e o \emph{vivere
civile.} A instalação desse novo governo principesco confirma a hipótese
de que, apesar da derrocada do governo anterior, a cidade na qual ocorre
essa transição política conserva os aspectos essenciais de legalidade
política, de civilidade, de um respeito, ainda que mínimo -- poder-se-ia
conjecturar --, aos ordenamentos políticos, aos valores cívicos.

Aqui cabe ressaltar o porquê de Maquiavel nomear esse tipo de principado
como ``civil''. Evidentemente, é em função dele se instalar reconhecendo
a dinâmica dos conflitos e respeitando esse dado essencial durante o
governo, sem descambar para um governo absoluto ou aristocrático, bem
como numa espécie de populismo ou, anacronicamente falando, numa
``ditadura do proletariado''. O principado civil é civil por instaurar e
estimular a dinâmica política fundamental para a vitalidade da cidade,
por recuperar e conservar a civilidade.

Retomando a análise, verifica-se ainda que o quadro no qual Maquiavel
descreve a origem do principado civil é muito semelhante a uma
república. Tendo em vista a origem não dinástica do príncipe, podemos
questionar até se o regime político anterior ao principado civil era uma
monarquia ou não. A transição que não deve ser calcada na violência, mas
no consenso, a presença dos humores, que tensionam o governo, enfim,
todos esses aspectos elencados em \emph{O Príncipe} remetem à dinâmica
política retratada nos dezoito primeiros capítulos dos \emph{Discursos}.

Por isso, se pretendemos pensar numa gênese do principado civil, devemos
concordar que uma hipótese, talvez a mais provável, é a da cidade
republicana que atinge um certo grau de corrupção e não consegue, por si
só, retomar o seu ordenamento político inicial. O desenrolar do capítulo
\versal{IX} de \emph{O Príncipe} comprova ainda mais essa constatação inicial,
pois Maquiavel mostra como o príncipe novo, que chega ao poder nessas
condições, deve se comportar diante do jogo de interesses e de poder que
permanece após a sua instalação no comando do principado. Pelo controle
dos humores e dos desejos, deve-se tomar todo o cuidado para não ficar
refém dos interesses dos grandes, mas manter um certo equilíbrio entre
os dois grupos principais (grandes e povo) e, quando isso não for
possível, apoiar-se totalmente no povo, ainda que isso implique certos
constrangimentos às suas decisões políticas.

Portanto, a descrição que emerge nesse capítulo \versal{IX} de \emph{O
Príncipe} sobre o principado civil coloca-o muito próximo do
ordenamento da cidade republicana e permite pensar que a transição de um
regime a outro não é uma inferência inadequada. Ao contrário, entre os
modelos de regimes que figuram no horizonte do possível nas descrições
maquiavelianas, o principado civil é o mais adequado às necessidades de
um governo forte exigidas ao final do capítulo dezoito do livro \versal{I} dos
\emph{Discursos}. Por conservar os elementos fundamentais da república
e, também, por manter a presença do essencial da vida política, com seus
humores e os conflitos entre eles, esse regime vem totalmente ao
encontro das exigências que a cidade corrompidíssima solicita para o seu
reordenamento. Seja ao tratar da corrupção nos capítulos \versal{XVI}, \versal{XVII} e
\versal{XVIII} do livro \versal{I} dos \emph{Discursos}, seja nesse capítulo \versal{IX} de
\emph{O Príncipe}, seja, ainda, no capitulo \versal{LV} do mesmo livro \versal{I}
dos \emph{Discursos}, entre as principais causas da corrupção está a
ambição dos grandes em tomar o poder. Em todos esses capítulos, bem como
em inúmeras outras partes, o desejo dos aristocratas em assumir o
comando do poder para si ou instalar um governante que lhe seja
favorável está sempre presente. Ora, mais do que pensar numa corrupção
endêmica e generalizada pela cidade, ao considerar-se a corrupção do
povo (a corrupção da matéria), encontrar-se-á mais um desejo de
usurpação dos grandes e menos uma desobediência às leis por parte do
povo em geral. Quando, pois, numa república dominada pelos grandes, não
se encontram mais meios de impedir esse avanço da aristocracia sobre o
poder, não há outro remédio senão instalar um governo forte, \emph{quase
régio}, sob a forma do principado civil:

\begin{quote}
Razão por que nessas províncias não surgiu nenhuma república nem nenhum
tipo de vida política; porque tais tipos de homens são totalmente
inimigos da civilidade. E não seria possível introduzir uma república em
províncias assim constituídas, mas, para reordena-las -- caso a alguém
coubesse tal arbítrio --, não haveria outro caminho senão constituir um
reino. A razão é que, onde a matéria está tão corrompida, não bastam
leis para contê-la, e é preciso ordenar junto com elas maior força, que
é a mão régia, que com poder absoluto e excessivo, ponha freio à
excessiva ambição e corrupção dos poderosos (\emph{Discursos}, \versal{I}, \versal{LV},
21-23).
\end{quote}

Nessa passagem, como em outras, repetem-se as mesmas exigências
apresentadas para a instalação de um principado civil em substituição à
república corrompida: excessivo poder da aristocracia, ineficácia das
leis e das instituições, um governo forte, mas que se instale sem
violência, a existência de uma parcela, ainda que mínima, de civilidade.
A necessidade de um governo forte não implica necessariamente a fundação
deste com o uso da violência: força e violência não se seguem.

Sobre a passagem do principado civil para uma república não há uma
exposição detalhada de Maquiavel em \emph{O} \emph{Príncipe} e nem na
parte inicial dos \emph{Discursos}. Talvez as considerações dos
capítulos \versal{IX} e \versal{X} do livro \versal{I} dos \emph{Discursos} pudessem fazer uma
alusão a isso, embora saibamos que Maquiavel está tratando da transição
da monarquia romana para uma república. Caso pensemos que esse príncipe
evocado nesses capítulos dos \emph{Discursos} seja um típico príncipe
civil -- hipótese essa não absurda, visto que o objetivo desse príncipe
é fundar ou reordenar as instituições com vista a um regime republicano
e o príncipe novo do principado civil, se não tem esse objetivo, deve ao
menos conservar o mínimo de civilidade que resta à cidade --, então
teríamos sim uma exposição da passagem do principado civil para uma
república.

Junto a essa possibilidade interpretativa, outro argumento que também
poderia auxiliar nessa compreensão da possível passagem do principado
civil para a república, está nos \emph{Discursus rerum Florentinarum},
escrito em 1520 sob encomenda do papa Leão \versal{X} em função da morte de seu
sobrinho e governante de Florença, Lorenzo de Medici. Neste opúsculo,
Maquiavel disserta sobre qual seria a melhor forma de governo a se
instalar na cidade naquele momento de crise do regime comandado pela
família Médici, que, segundo ele, é uma crise no ordenamento político da
cidade de longa data. Após demonstrar que Florença nunca teve de fato
nem uma república e nem um principado, ele sugere dois modos de
reordenar a cidade: uma ordenação verdadeiramente republicana ou uma
reordenação verdadeiramente principesca. Desenvolvendo um pouco mais,
diz ele que, caso esse ordenamento não seja assim, as duas formas de
regime entrarão em dissolução. Donde conclui:

\begin{quote}
{[}\ldots{}{]} digo que não se pode ordenar nenhum regime estável que não
seja um verdadeiro principado ou uma verdadeira república, pois todos os
regimes postos entre estes dois são defeituosos. A razão disso é
claríssima: se o principado tem apenas uma via para sua dissolução, que
é se tornar uma república, e da mesma maneira a república tem uma única
via para se dissolver, que é se tornar um principado, {[}\ldots{}{]}
(\emph{Discursus rerum florentinarum}, § 11).
\end{quote}

Para além de analisar os meandros argumentativos do opúsculo
maquiaveliano, o trecho citado corrobora as duas partes da nossa
hipótese. Já nos era claro que a república corrompida tem a
possibilidade de se transformar em um principado civil. Agora vemos o
próprio Maquiavel enunciar a possibilidade de um principado se
transformar em uma república. Portanto, conforme o pensamento
maquiaveliano, a transição de um principado para uma república é uma
possibilidade.

Acerca dessas transições, há duas coisas a ressaltar: em primeiro lugar,
em ambos os casos estamos tratando de uma possibilidade e não de uma
certeza de mudança de regime. Nesse sentido, é sempre bom lembrar que
não há nenhum caráter determinista ou finalista no pensamento político
maquiaveliano (no que diz respeito às mudanças de regime, isso
implicaria uma associação com os ciclos políticos de Políbio e sua
inexorabilidade). Um segundo aspecto é que Maquiavel não declara, no seu
opúsculo, que esse principado que se transforma em república é um
principado civil. Na verdade, ele se furta em fazer essa análise sobre o
principado, coisa que não ocorrerá com a república. Sobre o principado,
ele apenas tece algumas considerações sobre a condição do povo, se há
igualdade política ou desigualdade política.

Contudo, a mensagem reiterada ao papa é de que, mesmo que se instale um
principado verdadeiro, este deve ter em vista a inserção dos diversos
grupos políticos no governo da cidade, criando uma espécie de ``governo
\emph{largo}'', ou seja, um principado composto de várias magistraturas,
com a participação dos vários extratos políticos. Tal ordenação
prepararia o povo (preparar a matéria) com vistas à fundação de fato de
uma república na cidade, cuja medida principal consistiria numa
redistribuição dos cargos políticos entre os diversos grupos (diversos
humores) e a reabertura do Conselho Maior, que era o espaço político
onde todos podiam tomar parte e que foi fechado pelo governo Medici
instaurado em 1513. Note-se que Maquiavel não defende de imediato a
instalação de uma república, mas um principado que prepare o povo para o
regime republicano. Preparação essa que passa fundamentalmente pela
recuperação dos valores cívicos, o \emph{vivere civile}, que culminará
na fundação de uma república. Ora, isso é quase o que literalmente ele
apresenta nos \emph{Discursos}, no capítulo \versal{IX} e \versal{X} do livro \versal{I}, conforme
foi visto. O príncipe invocado no texto republicano comparece no
opúsculo de igual modo; em ambos os casos, as ações do príncipe
governante direcionam-se no sentido de recuperar a civilidade degradada
ou mesmo perdida. Modos e procedimentos esses que ficam muito claros em
\emph{O Príncipe}.

Portanto, é muito plausível pensar que a república corrompidíssima
encontra sua melhor solução de recuperação no principado civil,
evitando, ainda que precariamente, a instalação de uma tirania ou de um
regime absoluto. Principado esse que deve ter por finalidade a
preparação do povo para a volta ao regime republicano, desde que, e
devemos enfatizar esse aspecto, o príncipe aja com vistas a recuperar a
dinâmica dos conflitos políticos que expressam a saúde e a vitalidade da
cidade. Então, não é sempre que uma república se transforma em um
principado civil e não é necessário que um principado se transforme em
uma república. Em todos os casos há sempre a possibilidade dessas
mudanças, mas não como necessidade, e sim conforme as ações dos atores
políticos, no caso o príncipe, o povo e os grandes.

Finalmente, ressalte-se a coerência da reflexão política maquiaveliana,
possível de ser percebida tanto quando tomamos seus diferentes textos
como quando levamos em conta apenas um deles. A percepção de tal
coerência permite compreender quais são os pontos fundamentais da sua
reflexão política e as suas convicções teóricas mais caras. Unidade que
supera as possíveis objeções advindas de problemas relacionados à
cronologia das obras, e que passa a ser o dado mais relevante para a
afirmação de que, independentemente de qual obra tenha sido escrita
antes ou depois, a reflexão maquiaveliana se mostra coerente e
articulada. Enfim, pelo exposto, parece ser de todo evidente que \emph{O
Príncipe} de Nicolau Maquiavel não é um texto em defesa da monarquia,
mas uma obra que busca compreender as dinâmicas políticas, em perfeita
consonância com sua reflexão republicana.

\chapter*{}
\addcontentsline{toc}{chapter}{O príncipe}
\begin{center}
\begin{vplace}[0.3]
\Large
O príncipe
\end{vplace}
\end{center}
\thispagestyle{empty}
\input{TEXTO.tex}

\chapter{Nota da tradução}

O texto aqui apresentado é resultado de dois mo(vi)mentos: o primeiro
deles remonta ao ano de 2005, quando José Antônio Martins e Oliver Tolle
começaram a fazer a tradução de \emph{O Príncipe} a ser publicado pela
Editora Hedra, na sua coleção de bolso; e o segundo, em 2019, quando
José Antônio Martins, Márcio Roberto do Prado e Liliam Cristina Marins,
estes, professores do Programa de Pós"-graduação em Letras da \versal{UEM},
passaram a fazer a revisão da tradução para essa nova edição.

No primeiro mo(vi)mento, o trabalho de tradução se integrou às pesquisas
de doutorado em filosofia na \versal{USP}. Desde o início, uma questão e uma
exigência se colocavam nesta demanda editorial, a saber: por que fazer
mais uma tradução de \emph{O Príncipe}, sabendo de antemão da existência
de dezenas de edições em língua portuguesa? Para não ser mais uma entre
muitas, tal edição deveria trazer algo que a diferenciasse. De comum
acordo, depois de pesquisar sobre as mais de três dezenas de publicações
até então, verificou"-se que havia ainda espaço para fazer uma tradução a
partir da edição crítica estabelecida por Giorgio Inglese, em 1994. Além
disso, ficou evidente que havia espaço também para uma edição bilíngue,
algo inédito nas edições nacionais e portuguesas, voltada para o
ambiente acadêmico. Para este objetivo, o texto deveria, assim,
apresentar um vocabulário adequado ao meio de circulação desejado,
distanciando"-se de um público maior, neófito no pensamento maquiaveliano
e no pensamento político de modo geral. Importa dizer que muitas das
edições correntes até então não eram traduzidas a partir do original
italiano e também não foram realizadas por especialistas ou
pesquisadores do pensamento maquiaveliano, o que se evidenciava pelas
inúmeras escolhas tradutórias destituídas de um rigor conceitual.

A primeira parte da tradução foi realizada no Brasil e a finalização do
trabalho foi realizada na Itália, durante a estadia de José Antônio na
cidade de Pisa, na Toscana, no âmbito do doutorado sanduíche financiado
pela Capes. Neste período, foi possível se valer das diversas edições da
obra, textos de comentários, léxicos etc., nas melhores bibliotecas da
Itália: a da Scuola Normale di Pisa, da Università di Pisa, e a do
Istituto del Rinascimento Italiano, em Florença. Experiência essa
importante para o contato com esses riquíssimos acervos e com alguns
especialistas italianos, os quais esclareceram sobre o uso do toscano e
das técnicas e estilo de escrita de Maquiavel.

No segundo semestre de 2007, tem"-se a publicação da primeira edição
bilíngue de \emph{O Príncipe} em língua portuguesa, com uma pequena
introdução e notas; em 2009, há a publicação da 2ª edição com algumas
correções no texto e nas notas, e alteração na diagramação; em 2011, a
segunda reimpressão desta 2ª edição.

No entanto, desde a publicação da primeira edição, em 2007, o desejo de
fazer uma introdução mais robusta, que agregasse novas notas ao texto, e
críticas e sugestões recebidas dos colegas acadêmicos, ficou latente. A
intenção de fazer uma edição comemorativa em 2013, por ocasião da
efeméride dos 500 anos da obra maquiaveliana, não foi viabilizada por
diversas razões.

Esse desejo se consolidou apenas no primeiro semestre do ano de 2019,
materializando o segundo mo(vi)mento: o da revisão da tradução. O
resultado foi um trabalho feito a seis mãos, que permitiu alterações
significativas na tradução, ao mesmo tempo que corrigiu ainda algumas
imperfeições textuais, dotando"-a de mais fluidez e adequação ao
português, mas sem comprometer o rigor conceitual. Longe de entrar em
discussões rasas sobre liberdade do tradutor \emph{versus} fidelidade ao
autor na tradução, nosso objetivo se pautou mais consistentemente em uma
prática tradutória --- na qual a revisão tem um papel inquestionável ---
do texto filosófico, que tem suas próprias amarras e \emph{modus
operandi}. Além deste foco no rigor conceitual, o processo de revisão
teve como norte o estabelecimento de um estilo de escrita que buscasse
traduzir o texto maquiaveliano para o português do século \versal{XXI}, evitando,
ao máximo, termos arcaizantes, anacronismos e estrangeirismos
excessivos.

Em relação às estratégias tradutórias, houve apenas um caso no qual foi
necessário manter o vocábulo em italiano e este se refere ao termo
\emph{virtù}, que não poderia ser deliberadamente traduzido por
``virtude'' em português, já que esta escolha comprometeria o conceito
filosófico. A fim de não nos desviarmos do propósito de deixar a
tradução mais ``palatável'' para o leitor acadêmico brasileiro em termos
de estilo --- sem nunca perder as travas de segurança conceituais ---, a
estratégia tradutória foi a nota de rodapé.

Ainda em relação à presença de estrangeirismo na tradução, mas em termos
estilísticos, tendo em vista que Maquiavel era um grande latinista, foi
necessário recorrer ao modo de construção de orações em latim nos
momentos em que as passagens do texto em italiano se colocavam como
obstáculo à compreensão. Por isso, remetíamos ao latim, para, a partir
deste, trabalhar na revisão da tradução para o português.

Em relação às adaptações linguísticas, tivemos que mudar, em alguns
momentos, a ordem sintática, que, em Maquiavel, apresenta muitas
inversões. No caso do português brasileiro, nosso modelo discursivo é
comumente a ordem direta e, se mantivéssemos a ordem inversa,
dificultaríamos muito a leitura e a construção de sentidos do texto.
Outro aspecto que envolveu a necessidade de uma ``cor local'' se refere
à pontuação que, em toscano, diverge muito daquela da língua portuguesa
contemporânea, de modo que não foi possível tomar o original como ponto
de partida nesse sentido.

Por fim, o elo que une esses dois mo(vi)mentos de tradução e de revisão
é, indubitavelmente, a colaboração e a multivocalidade discursiva: a do
pensador, a do tradutor, a dos revisores, a de outros tradutores etc, 
trabalho esse que enriquece o produto final.



\chapter*{Introdução}
\addcontentsline{toc}{chapter}{Introdução, por José Martins}

\section{Considerações iniciais}

Dada a quantidade (mais de três dezenas) de traduções e edições de
\emph{O Príncipe} de Nicolau Maquiavel somente no Brasil, sem contar com
as centenas de outras edições nas diversas línguas modernas, essa nova
edição publicada pela Hedra busca atender algumas especificidades e não
repetir aquilo que já foi divulgado à exaustão.

Quanto à edição, trata-se de propiciar ao leitor brasileiro um exemplar
bilíngue, com a melhor edição do texto original italiano -- a
\emph{Edição Crítica Inglese} -- acrescida de introdução e notas
explicativas. A exigência e a novidade de que haja o texto italiano
original, algo presente desde a primeira publicação em 2007, atende a
uma demanda por qualidade nos textos acadêmicos, a saber: permitir que o
leitor possa aprofundar (ou mesmo compreender melhor passagens da obra
que não tenham ficado claras na tradução) por meio do cotejamento com o
original.

A introdução aqui exposta visa a dois objetivos: a) como de praxe nas
boas edições, apresentar ao leitor o autor e os principais aspectos da
obra; b) para além desse aspecto básico, apresentar também um viés
interpretativo que vem se consolidando entre os comentadores, ou seja: a
inclusão de \emph{O Príncipe} no interior da reflexão republicana de
Maquiavel, com destaque para as noções de principado e príncipe. Dito
isso, convém lembrar que essa introdução não pretende ser nem uma
reapresentação da biografia do filósofo, nem uma introdução ao
pensamento político maquiaveliano como um todo, nem uma análise
exaustiva e minuciosa dos diversos aspectos teóricos presentes na obra.
Insistindo e explicitando melhor, apesar de mobilizar informações sobre
a vida de Maquiavel e alguns aspectos de seu pensamento político, essa
introdução busca pensar \emph{O Príncipe} como uma obra peculiar e
fundamental para a compreensão das noções políticas republicanas de
Maquiavel.

Por fim, inserimos algumas notas que visam apresentar personagens, datas
e eventos do contexto histórico no qual o livro está inserido e explicar
conceitos e passagens centrais do texto. Tendo em vista a presença e
natureza da introdução, o uso desse segundo tipo de notas foi restrito
ao mínimo essencial e indispensável.

\section{O contexto histórico maquiaveliano e~os~seus~pressupostos~teóricos}

\subsection{O contexto histórico: as origens e o ingresso~na~Chancelaria~florentina}

Nicolau Maquiavel ou Nicolló Machiavelli nasceu em Florença, Itália, no
dia 3 de maio de 1469, em uma família de pequenas posses. Sabe-se que
seu pai -- Bernardo Machiavelli --, em função de ser escriturário, teve
alguma formação jurídica, mas não ocupou um cargo destacado e nem
exerceu importantes cargos na administração da cidade, tampouco acumulou
grandes posses. Todavia, em função dessa formação jurídica do patriarca,
a casa dos Machiavelli dispunha de obras clássicas de história, de
doutrina política, de jurisprudência, enfim, uma pequena biblioteca nos
moldes de uma sociedade humanista\footnote{É sempre difícil definir com
  poucas palavras grandes manifestações históricas, como foi o caso do
  Humanismo, que teve como seu lugar de nascimento e desenvolvimento a
  Itália dos séculos \versal{XIII} ao século \versal{XVI}. Isso que se entende por
  Humanismo caracterizou-se por uma retomada dos valores culturais da
  Antiguidade e por uma maior valorização do homem. Por isso, busca-se
  ler os autores clássicos nos seus textos originais, seja em grego,
  seja em latim; há uma intensa busca pela redescoberta desses textos,
  quase sempre esquecidos em alguma biblioteca, produzindo-se novas
  edições; uma maior valorização daquilo que diz respeito ao homem e às
  coisas humanas, numa tentativa de contrabalancear a pouca valorização
  desses aspectos pela cultura cristã-medieval. Concomitantemente, há
  uma maior valorização das virtudes ligadas à vida na cidade,
  principalmente as qualidades relacionadas à política, por oposição aos
  valores religiosos cristãos, característicos de grande parte do
  período medieval. Esses aspectos, entre outros, sinalizam a mudança
  que se operará nas sociedades europeias a partir de meados do século
  \versal{XIII} e que será predominante já no século \versal{XV}, preparando o terreno
  para aquilo que se conhecerá como a Era Moderna. Para maiores
  informações cf: \versal{HALE}, 1971; \versal{SKINNER}, 2000, cap. 1-3; \versal{BARON}, 1989.} do
\emph{Quattrocento}.

Sobre a formação de Nicolau Maquiavel, tendo em conta esse ambiente
humanista e a acessibilidade às obras clássicas, sabe-se que ele teve
aulas com professores particulares, não havendo registro sobre uma
possível frequência em escola ou universidade. Segundo Guidi, Maquiavel
teve aulas com os mestres de latim Paolo Sassi, Michele Verino e Pietro
Crinito (\versal{GUIDI}, 2009). A partir de tal formação, deduz-se que Maquiavel
tenha ultrapassado o que seria um nível intermediário, tendo em vista as
exigências acerca do conhecimento da língua latina para o cargo que
ocupou na Chancelaria, sendo que, posteriormente, revelou um grande
conhecimento de autores como Cícero, Lucrécio, Tito Lívio, etc.
Considerando toda a sua produção teórica e literária, comprova-se que
ele conhecia muito bem a sua língua, o toscano, e o latim, assim como
possuía uma vasta compreensão de história e noções de filosofia. Além
dessa formação em língua clássica, ele foi introduzido ao cálculo
matemático, caracterizado pelo ensino do ábaco, e teve alguma formação
em jurisprudência, embora não tenha se tornado bacharel em direito.
Pode-se afirmar com segurança que Maquiavel não aprendeu o grego. É
nesse ambiente cultural e familiar que sua vida se desenvolve até o ano
de 1498, quando começa a trabalhar na Chancelaria florentina, onde
permanecerá pelos próximos catorze anos.

Esse período que Maquiavel passou na Chancelaria foi posto em segundo
plano pelos especialistas por muito tempo. Contudo, mais recentemente,
esse aspecto biográfico passou a ser alvo das atenções, principalmente
em função do verdadeiro alcance dessa experiência diplomática sobre sua
reflexão política. Segundo Guidi, o ingresso na Chancelaria Florentina
ao final do século \versal{XV} era feito por indicação do chefe político -- o
\emph{gonfaloniere} --, indicação essa que deveria ser submetida à
aprovação dos Conselhos políticos superiores (\versal{GUIDI}, 2009). Após a queda
da família Medici, em dezembro de 1494, o novo governo do frei Jerônimo
Savonarola busca reestruturar o regime florentino, criando mecanismos
republicanos de caráter mais popular, particularmente com a reabertura e
o fortalecimento do Conselho Maior (\emph{Consiglio Maggiore}), que era
a instância política na qual todos os cidadãos podiam tomar parte e não
era controlada exclusivamente pela oligarquia florentina. Outra medida
do governo de Savonarola foi reestruturar toda a Chancelaria, trocando
os funcionários e reordenando as incumbências.

Segundo Tafuro (2004, p. 25ss), o mundo político da cidade de Florença
no final do século \versal{XV} era caracterizado por divisões de caráter
econômico e social que estavam na raiz dos grupos políticos, a saber: os
grandes aristocratas, os médios comerciantes, os pequenos comerciantes e
os demais trabalhadores -- esses, em geral, quase sem expressão
política. Maquiavel nomeará esses grupos ao longo de sua \emph{História
de Florença}, respectivamente, como nobres, povo e plebe. A aristocracia
florentina (os nobres, segundo Maquiavel) estava dividida em dois
grandes partidos, desde as lutas políticas do século \versal{XIV} entre os
partidários do papa -- o partido \emph{guelfo} -- e os partidários do
imperador\footnote{Os territórios do centro-norte da península itálica
  foram motivos de disputas entre os séculos \versal{XI} a \versal{XIV} pelo papado e pelo
  Sacro Império Romano-Germânico. Como mostra Skinner (2000, cap. 1),
  nesse período, muitas cidades travaram lutas externas contra esses
  dois domínios em busca de sua independência e também lutas internas
  contra a dominação de um desses grupos ou contra o domínio de tiranos
  locais. A tese aceita pelos intérpretes é que, destas lutas, nasce o
  germe do republicanismo típico do Renascimento italiano, no qual as
  cidades buscaram se organizar como repúblicas independentes e
  autônomas.} -- o partido \emph{ghibelino}. A vitória do partido
\emph{guelfo} em 1289 na sua luta contra os \emph{ghibelinos} instalou
uma estreita relação entre essa parcela da aristocracia e o papado, que
perdurará durante séculos e que terá como fruto a forte influência dos
florentinos nas decisões de Roma, bem como a nomeação de alguns cardeais
florentinos ao trono papal: Leão \versal{X} (Giovanni di Lorenzo de Medici --
papa de 1513 a 1521), Clemente \versal{VII} (Giulio di Guiliano de Medici -- papa
de 1523 a 1534), Clemente \versal{VIII} (Ippolito Aldobrandini, papa de 1592 a
1605), Leão \versal{VI} (Alessandro Ottaviano de Medici, papa por 26 dias, de 1 a
27 de abril de 1605), Urbano \versal{VIII} (Maffeo Barberini, papa de 1623 a
1644) e Clemente \versal{XII} (Lorenzo Corsini, papa de 1730 a 1740).

Apesar dessa importante força política desempenhada por parte da
aristocracia do partido \emph{guelfo}, uma outra parcela também rica, a
aristocracia \emph{ghibelina}, será muito influente e marcante na
cidade, por exemplo, na constante oposição que a família Medici recebeu
enquanto esteve no poder.

Abaixo desses dois grupos políticos aquinhoados, Maquiavel designa as
demais parcelas como ``povo'', embora não se assemelhe muito à nossa
compreensão contemporânea do termo. Essa parcela social e política teve
sua origem nas corporações de ofícios ou \emph{artes}, que eram
associações dos diversos grupos profissionais da cidade que controlavam
o exercício da respectiva atividade profissional. Segundo Tafuro (2004,
p. 27-28), um desses grupos políticos mais abastados, não
necessariamente aristocratas em sua denominação, configuram uma parcela
política dita popular, que ganha riqueza e prestígio e passa a figurar
como um grupo político de destaque. Nesse caso, formam, já no século \versal{XV},
um agrupamento de caráter aristocrático do ponto de vista econômico e
intelectual, mas não tradicional, por não ter ligação com as antigas
aristocracias rurais e urbanas.

Ainda no interior das \emph{Arti} ou povo, segundo a designação
maquiaveliana, verifica-se uma divisão interna entre aquilo que ele
nomeará como \emph{popolo grasso} (povo gordo) e \emph{popolo minuto}
(povo magro), ou seja, entre os setores médios mais aquinhoados e
aqueles com menos poder econômico e, portanto, menor influência
política.

Verifica-se, pois, que o povo (\emph{popolo}), para Maquiavel, é uma
denominação para diferenciar parcelas sociais com riqueza de outras
também com riqueza mediana, mas de origem tradicional. Logo, são
tipificações no interior de grupos caracteristicamente aristocráticos.
Na definição de Tafuro: ``Povo era, em particular, a parcela média dos
artesãos e dos comerciantes reunidos nas associações profissionais''
(2004, p. 40). Ora, quando Maquiavel chama a atenção para a divisão
entre os nobres e o povo, trata-se de uma divisão entre os setores
aristocráticos tradicionais e a classe social que emergiu econômica e
socialmente com o exercício do comércio e do artesanato a partir do
século \versal{XIII}.

Por fim, temos a plebe, que eram os operários ou os assalariados, que
não possuíam qualquer associação ou grupo político no qual pudesse
expressar seus interesses, até a Revolta dos Ciompi, de 20 de julho de
1378, quando eles passam a ter direitos políticos e associativos
(\versal{TAFURO}, 2004, p. 33-34). Sintomática é a importância que Maquiavel
confere a essa revolta política de final do século \versal{XIV}, que marca uma
mudança paradigmática na vida política florentina, conforme ele narra
com cores dramáticas nas \emph{Histórias Florentinas,} livro \versal{III}. A seu
ver, essa revolta política ocupa um papel de destaque na história da
dinâmica política da cidade, visto que implicou não somente uma mudança
institucional, mas, principalmente a partir da visada teórica
maquiaveliana, na dinâmica política e nas correlações de força entre os
grupos políticos.

Nunca é demais lembrar que, apesar da constatação da existência desses
diversos grupos políticos, por outro lado, inúmeras pessoas estavam
alijadas das instâncias decisórias, como os trabalhadores rurais, as
mulheres e os não nascidos em Florença, que, conforme nos relata Gilbert
(1996, p. 25), de uma população estimada em 62.000 em 1494, menos de
15.000 homens possuíam cidadania e direitos políticos.

Nota-se, então, que Florença entre os séculos \versal{XIV} e \versal{XVI} estava dividida
em vários grupos sociais que, por consequência, dividiam-se em grupos
políticos em disputa pelo comando do governo. Esse quadro de divisão e
disputa política entre vários partidos é ressaltada por Maquiavel nos
seus diversos escritos e é uma das marcas de seu pensamento político,
haja vista que é sob essa condição de instabilidade e disputa que se
monta o palco da ação política.

Tal percepção da importância de lidar com os interesses desses diversos
grupos no mundo político Maquiavel experimenta desde o momento de seu
ingresso na Chancelaria e o atingirá ao longo de toda a sua vida
doravante. Com a queda do governo da família Medici no final de 1494 e a
instauração do governo de Savonarola, a administração central passa a
ser controlada pelos partidários deste. Nesse período, sabe-se que
Maquiavel apenas realizou trabalhos avulsos, com os quais travou já
algum contato com o ambiente da Chancelaria, mas que não chegou a
ingressar nela, pois a admissão se fazia por meio de uma indicação do
\emph{gonfaloniere} e era, posteriormente, submetida à votação no
conselho ao qual o cargo se vinculava. Ora, Maquiavel não somente não
teria a indicação de Savonarola como os partidários desse nos diversos
Conselhos da cidade não aceitariam o seu nome, como ocorreu
primeiramente em 18 de fevereiro de 1498, quando seu nome foi recusado
pela primeira vez. Apenas com a queda do governo de Savonarola é que o
nome de Maquiavel é indicado para o cargo de segundo secretário e
admitido em 23 de maio de 1498.

É importante destacar que a Chancelaria não era apenas uma repartição da
burocracia voltada exclusivamente para as relações diplomáticas, mas o
órgão central da administração da república florentina, encarregada das
relações exteriores, da guerra, da arrecadação de tributos e de
assessoramento das instâncias judiciárias. Portanto, ingressar na
Chancelaria não implicava tão somente fazer parte do órgão responsável
pelas relações diplomáticas de Florença, mas trabalhar no coração
político e burocrático da cidade (\versal{GUIDI}, 2009, cap. 1).

O fato em si da indicação para um cargo elevado revela que Maquiavel já
era conhecido do grupo político que assume o poder em 1498, bem como que
ele já conhecia a rotina burocrática. De fato, entre 1494 e 1498,
Maquiavel foi contratado para fazer pequenos serviços para a
Chancelaria, o que foi lhe dando conhecimento sobre as rotinas
administrativas. Então, sua escolha em 15 de junho de 1498 para a função
de secretário da Segunda Chancelaria é fruto de sua formação humanista,
seu conhecimento, ainda que parcial, das rotinas da administração e,
fato apontado como principal pelos estudiosos, sua ligação política com
o grupo que assume o poder em 1498, que eram setores do \emph{povo}, no
caso, os adversários dos Medici e de Savonarola; ou seja, quem apoiou a
indicação de Maquiavel foram em sua maioria os partidários das
\emph{Arti}, especificamente os partidários das \emph{Arti Maggiori}.
Parte das tarefas deste cargo ocupado por Maquiavel implicava assessorar
os ``Dez da Bailia'', que era um órgão encarregado da política externa
e, eventualmente, da guerra e das questões militares (\versal{TAFURO}, 2004, p.
121-122).

A questão que ainda incomoda os estudiosos da obra de Maquiavel diz
respeito a sua atuação política no período em que esteve trabalhando na
administração da república florentina. Por um lado, com base em seus
escritos, Maquiavel apresenta a imagem de um modelo de funcionário
público devotado à causa da cidade e não aos interesses dos grupos no
comando do governo, uma espécie de funcionário de carreira sem ligação
política ou ideológica que estivesse a serviço da cidade\emph{.} Por
outro lado, até mesmo pela sua própria história, Maquiavel foi sempre
identificado a um grupo político, ora como partidário dos republicanos
na visão da oligarquia florentina, ora como partidário dos Medicis,
segundo o grupo republicano por ocasião da restauração da república em
1527. A questão está em saber até que ponto Maquiavel foi de fato um
funcionário, por assim dizer, isento de ligações políticas ou se ele foi
um mentor e articulador do governo de Pier Soderini, ou, até mesmo, se
seria possível conciliar as duas coisas. As pesquisas recentes sobre o
período em que Maquiavel esteve na Chancelaria mostram, por vários
elementos, que ele não somente foi um membro ativo do grupo que esteve
no poder em Florença entre 1498 e 1512, como ocupou um papel destacado
no governo. Segundo Guidi, o ingresso de Maquiavel já em um cargo
elevado na burocracia, a sua participação ativa na reformulação do
governo em 1502, os encargos diplomáticos e militares que ocupou depois
até a queda do governo em 1512 (a demissão oficial de Maquiavel foi em
10 de novembro de 1512), revelam que ele ingressou na Chancelaria também
por seus vínculos políticos e foi, ao longo dos anos, galgando cada vez
mais espaço e importância no governo de Soderini, ao ponto de ser o
maior responsável pela parte militar do governo e pela condução das
guerras após 1506 (\versal{GUIDI}, 2009).

A vinculação aos interesses do governo de Soderini e as suas qualidades
de negociador e analista do contexto político ficarão associadas a
Maquiavel para sempre, pois, mesmo depois de sua saída do governo,
quando foi requisitado, mesmo entre os seus adversários, sempre se nota
esse duplo aspecto: um partidário da causa republicana e um hábil
analista político e militar. Aspecto esse reforçado pelo modo como ele
descreve sem paixão o universo da política em geral e o seu mundo
florentino em particular, no qual os fatores políticos que implicam de
fato as decisões são expostos friamente, sem meandros ou meias palavras.
Enfim, podemos inferir que Maquiavel foi sim um membro leal do grupo de
Solderini, mas isso não o impediu de ser um funcionário a serviço da
cidade que avaliava sem paixão o mundo político, como demonstram a
exaustão suas correspondências diplomáticas e os seus textos políticos.

Embora tenha iniciado no serviço público com 29 anos, uma idade
considerável para a época, e com uma relativa formação humanista,
Maquiavel declara em vários textos que foram esses anos de convivência
cotidiana com o mundo da política que lhe deram uma boa parte de sua
formação política -- a outra parte ele diz ter recebido dos clássicos.
Em seu dia a dia, ele acompanhava, por um lado, o funcionamento das
decisões políticas em Florença, ou seja, observava por dentro os
mecanismos das negociações políticas, como elas eram feitas, os jogos de
intenções e de promessas, e, por outro lado, como se realizavam as
negociações diplomáticas, como os reis, papas, príncipes, comandantes
militares, governantes das republicas negociavam e estabeleciam pactos,
guerras, ou resolviam os conflitos comerciais. Esse conhecimento da vida
política por dentro, aliado a um olhar treinado a não enxergar somente
os aspectos circunstanciais e pessoais do mundo político, mas a
desvendar as reais motivações dos atores e os fatores que regulam o agir
político, deram a Maquiavel uma parte do conhecimento necessário para
entender o universo da política geral, de modo ampliado.

Convém informar que, desde os séculos \versal{XII} e \versal{XIII}, ocorre uma mudança
radical na concepção e organização da vida diplomática e das
chancelarias nas repúblicas italianas (\versal{FUBINI}, 1994). Seja em função das
disputas que deveriam travar contra as grandes potências (não somente o
papado e o Império Germânico, mas também, nos séculos seguintes, contra
os franceses e espanhóis, sem contar a sempre constante ameaça turca que
se amplia após a queda de Constantinopla em 1453), seja em função de seu
pouco poder militar, essas repúblicas reconheceram que os seus quadros
diplomáticos seriam uma grande arma para a defesa de sua condição de
repúblicas livres. A mudança se nota inicialmente na própria missão
delas: não mais defender tão somente os interesses privados do poderoso
no governo, o que implicaria uma prática voltada para os interesses
gerais e não mais como prepostos mercantis. As chancelarias das
repúblicas italianas do Renascimento se diferenciam das medievais, na
medida em que estão mais voltadas a trabalhar pela defesa dos interesses
gerais da cidade, principalmente a defesa da condição de liberdade
política destas repúblicas, e não somente em fazer acordos vantajosos
para as famílias poderosas (\versal{GUIDI}, 2009, p. 35).

Outro aspecto das práticas dessas chancelarias do Renascimento é sua
preocupação em ter um maior controle da vida política e jurídica da
cidade, tornando-se verdadeiras burocracias de controle das práticas
públicas. Neste sentido, adota-se o costume de registrar todas as
reuniões dos conselhos em atas, de catalogar as missões, de documentar
os discursos e exigir de seus emissários a notificação do andamento das
ações em suas missões. Essa massa de material permite ao governo ter
maior controle das decisões e perceber melhor as rotinas e os
encaminhamentos das diversas ações. Segundo Guidi, os chanceleres e os
secretários ``configuram-se como técnicos da administração a serviço do
executivo'' (\versal{GUIDI}, 2009, p. 39-40).

Por isso que fazer parte da Chancelaria florentina não significava tão
somente fazer parte do órgão responsável pelas relações internacionais,
o que por si mesmo já implicava um extraordinário conhecimento do mundo
das negociações políticas. Tendo em vista essa dimensão de controle
burocrático das ações de governo, seus funcionários tinham acesso
privilegiado, por meio da documentação, aos diversos elementos em jogo
na política interna e externa da cidade. Essa documentação torna público
e acessível aquilo que antes eram informações restritas e privilegiadas
para se compreender os reais motivos das tomadas de decisão política.
Portanto, quando Maquiavel diz que aprendeu muito com a vida prática na
Chancelaria, isso não significa que ele esteve presente em todas as
decisões e conhecia muito bem os diversos interesses em jogo (visto que
seria impossível ele estar, por exemplo, fora da cidade e saber dos
detalhes de uma negociação ou jogada política), mas que, por meio dessa
documentação pública, ele pode conhecer o mundo da política. Essa
documentação oficial, também conhecida como \emph{pratiche} (\versal{GILBERT},
1964), relatam as práticas, os diversos procedimentos oficiais, e são,
para um leitor atento, uma fonte preciosa de informações sobre a
dinâmica da vida política. Convém insistir: quando Maquiavel declara,
por várias vezes, que boa parte de sua formação política adveio das
experiências das coisas modernas, da sua convivência com o mundo da
política, inclua-se nesse rol o conhecimento dessa farta e rica
documentação oficial, à qual ele teve acesso por sua condição de Segundo
secretário\footnote{A hierarquia nas chancelarias eram: chanceler,
  secretários e cartorários ou oficiais de chancelaria.}, sendo, pois, o
responsável por produzir e organizar esses documentos\footnote{Não é sem
  razão que as pesquisas sobre o pensamento político de Maquiavel estão
  se voltando cada vez mais para essa documentação produzida no período
  republicano, mais especificamente para os documentos oficiais do seu
  período na Chancelaria, de 1498 a 1512.}. Enfim, as \emph{pratiche}
destacadas por Maquiavel não foram somente as ações políticas, as
práticas políticas, mas também e fundamentalmente a documentação oficial
da República florentina, algo que, ainda que não fosse inédito para
muitos pensadores, foi considerada e analisada de modo especial pelo
Secretário Florentino, configurando-se, enfim, como uma novidade no
mundo do Renascimento.

A outra parte do conhecimento, conforme ele mesmo declara, adquiriu na
leitura dos clássicos. Para compreender melhor esse aspecto, não é
suficiente saber que Maquiavel teve uma formação humanista, mas também
considerar o contexto cultural no qual ele esteve inserido. Conforme
Garin, a Chancelaria de Florença tinha se transformado, desde o final do
século \versal{XIV}, não somente no centro político da cidade, mas no seu centro
intelectual, em função das figuras de destacada importância que
estiveram a serviço da cidade como Coluccio Salutati, Poggio
Bracciollini, Leornardo Bruni, Bartolomeu della Scalla, Lorenzo Valla,
Francesco Guicciardini, fazendo dela uma verdadeira escola do pensamento
político (\versal{GARIN}, 1996). Esses intelectuais marcaram o pensamento
político do Renascimento, entre outros fatores, mas principalmente, por
terem que fazer uma defesa contundente do regime republicano florentino
enquanto ocupavam o cargo de chanceleres. Desde o final do século \versal{XIII}
até o início do século \versal{XVI}, Florença frequentemente se via envolvida em
ameaças de dominação por algum poder exterior ou em disputa contra
tiranos locais. Nessas ocasiões, sempre era necessário mobilizar os
cidadãos para a defesa da liberdade política, que implicava, em última
instância, a defesa do regime republicano. Portanto, diante da ameaça de
dominação, o chanceler do momento comandava uma luta no campo ideológico
e teórico em defesa da liberdade republicana. Ora, esse acúmulo de
reflexão política foi se consolidando ao longo do tempo entre os
ocupantes de cargo, a ponto de se constituir, como interpreta Garin,
numa ``escola'' de pensamento político republicano. Então, para os
postulantes a cargos na Chancelaria, já estava claro, de antemão, que
eles estavam se vinculando a uma instituição que teve e ainda tinha por
missão a defesa da liberdade política e da autonomia da cidade,
encarnada no seu regime republicano. Isso se reflete decisivamente nos
escritos de Maquiavel, nos quais a problemática da liberdade política
perpassa sua reflexão política como um todo (\versal{BIGNOTTO}, 1991).

Associa-se a isso o intenso debate político sobre a melhor forma de
ordenar a república florentina no interior da crise política de final do
século \versal{XV} gerado pela queda dos Medici, pelo curto e conturbado governo
de Savonarola e início do novo governo de Soderini (\versal{BARON}, 1989;
\versal{GILBERT}, 1996; \versal{RUBINSTEIN}, 1998). Ora, num período de mais de 20 anos
(entre 1492 e 1513), Florença passa por quatro governos: o de Piero de
Medici (1492-1494), o de Jerônimo Savonarola (1494-1498), de Pier
Soderini (1498-1512) e doravante novamente com a família Medici até
1527. Essas mudanças na vida política da cidade são acompanhadas de
intensos embates intelectuais, nos quais se podem reconhecer vários
grupos em defesa de seus projetos, mas que podem ser agrupados, em
linhas gerais, em dois grandes partidos, a saber: os defensores de um
modelos republicano de caráter mais aristocrático e os defensores de um
regime republicano de caráter mais popular, com a ampliação da
participação dos diversos segmentos políticos da cidade (\versal{MARTINS}, 2010).

Sobre esse tema, convém recuperar a disputa entre os modelos
republicanos nesse momento histórico na cidade de Florença.

\subsubsection{O contexto maquiaveliano: entre o republicanismo popular~e~o~republicanismo~aristocrático}

Os séculos \versal{XV} e \versal{XVI} foram marcados por intensos debates nas cidades
italianas sobre a melhor forma de regime republicano a ser adotada.
Neste período, vários pensadores se preocuparam em definir modelos de
repúblicas que atendessem às demandas de sua época, consagrando duas
formulações: um modelo de república de caráter mais popular,
caracterizado pelos governos de Savonarola e Soderini em Florença, e um
modelo de república de caráter aristocrático, tendo o regime republicano
da cidade de Veneza como o grande exemplo. É em Florença que esse debate
ganha mais vigor e corpo, fazendo dela um dos centros de produção
intelectual sobre o regime republicano (\versal{GARIN}, 1996). As mudanças
políticas florentinas do final do século \versal{XV} e início do século \versal{XVI}
suscitaram intensos debates sobre os destinos da cidade, num primeiro
momento, e sobre a natureza das repúblicas, num segundo momento. Nesse
contexto de alteração constitucional, a discussão dos fundamentos da
república florentina adquire força nos círculos intelectuais. Dentre as
várias posições assumidas, a defesa da instalação de um regime
republicano inspirado no modelo veneziano foi predominante entre a
aristocracia florentina, grupo político que identificava nos governos
republicanos de Savonarola e Soderini o predomínio dos segmentos
populares, governos esses considerados como ``demasiadamente''
democráticos\footnote{Certamente, caracterizar os governos republicanos
  de Florença que vão de 1494 a 1512 como democráticos é algo
  problemático devido ao poder que a aristocracia deteve nesse período.
  Qualquer afirmação mais contundente no sentido da definição do tipo de
  governo existente em Florença durante esses 18 anos é passível de
  discussão. Sobre a história do período cf. Tenenti (1973) e Tafuro
  (2004, parte \versal{I}).}. Entre os exemplos favoritos da aristocracia para
justificar sua opção política estavam a Roma republicana, a Esparta
concebida por Licurgo e a república veneziana de então. Entretanto, como
afirma Gilbert, o exemplo veneziano era o que mais se destacava: os
aristocratas em particular, ansiosos em limitar o poder do
\emph{Conselho Maior} (no qual todos os cidadãos tinham direito de
participar), colocavam em evidência que, em Veneza, os cidadão discretos
e sábios tinham as possibilidade melhores e mais apropriadas para o
exercício do poder (\versal{GILBERT}, 1977, p. 102-103). A opção pelo modelo
veneziano se deve, principalmente, ao predomínio e controle que a
aristocracia mercantil exercia sobre o governo. Em Florença, o governo
de Savonarola, bem como em certa medida o de Soderini, eram, aos olhos
da aristocracia, muito democráticos, pois neles os poderes decisórios de
seu extrato político estavam limitados pelas forças políticas populares.

É, portanto, no interior dessa luta pela retomada do comando político da
cidade, liderada pela aristocracia, que nasce aquilo que Pocock nomeia
como o ``mito de Veneza'' (\versal{POCOCK}, 1980). Veneza servia como modelo
porque conseguia reunir diversas qualidades almejadas pela aristocracia
florentina, transformando-se num ideal de convivência cívica. A
estabilidade política e a liberdade, bem como a existência de um governo
misto e a virtuosidade de seus gentis-homens, eram entendidas como as
causas principais para a riqueza da república do Norte. O governo
comandado pelo Doge (chefe do executivo de caráter vitalício) e seus
Conselhos (compostos quase que exclusivamente pela aristocracia) seriam
a realização do regime misto idealizado pelos filósofos. Nos escritos de
venezianos da segunda metade do século \versal{XV}, como Francesco Barbaro,
Giorgio da Trebisonda e Bernardo Bembo, Veneza correspondia, até nos
detalhes, à república proposta por Platão, principalmente por conter em
si as três formas de governos particulares ou simples. Ademais, essa
defesa do regime misto, que não está somente nos textos platônicos, mas
também em Aristóteles, Políbio e Cícero, levou esses escritores a
afirmar que Veneza era a realização do modelo clássico de república
ideal (\versal{GILBERT}, 1977).

Na visão da aristocracia florentina, o principal resultado alcançado por
esse governo misto era a ausência de conflitos políticos num ambiente de
grande liberdade cívica, entendida num duplo sentido: como a existência
de um governo não tirânico e não estarem submetidos a outra cidade
(\versal{GILBERT}, 1977). As narrativas que chegavam a Florença sobre a república
veneziana relatavam que ela havia sido instalada há muito tempo e não se
tinha notícia da ocorrência de conjurações ou tumultos políticos que
ameaçassem sua normalidade republicana, fruto, também, da grande
\emph{virtù} de seus cidadãos. Essa fama de Veneza como república
pacífica lhe rendeu a alcunha de ``república sereníssima''.

Tal imagem modelar de Veneza revelou-se um mito na medida em que os
próprios humanistas começaram conhecer melhor a real estruturação do
regime republicano que lá vigorava. Com melhores informações sobre o
funcionamento da república veneziana, descobre-se que se tratava de um
governo tipicamente oligárquico, pois era dominado por um Conselho
estritamente limitado e controlado por um número pequeno de famílias.
Como diz Gilbert (1977), poucas pessoas em Florença conheciam como
realmente se ordenava o regime veneziano. A admiração estava fundada
mais nas narrativas e imagens projetas da cidade do que na realidade
política.

Seja como for, a imagem da república veneziana passou a exercer
relevante influência em Florença antes mesmo da instalação do governo de
Savonarola. Quando do nascimento do governo republicano, em dezembro de
1494, uma das principais inovações do novo regime foi a instauração do
\emph{Conselho Maior}, à semelhança do Grande Conselho de Veneza, com
ampla participação dos vários grupos sociais\footnote{Compreender o
  intrincado funcionamento do regime republicano de Florença é uma
  tarefa difícil. Gilbert nos informa que havia neste período
  aproximadamente 3.300 cargos eletivos, para uma população de não mais
  de 60.000 pessoas. Em termos proporcionais significava dizer que,
  dentre a população masculina com direito a voto, entre 1/4 ou 1/5
  participavam de algum cargo eletivo, o que é significativo em termos
  de participação popular (\versal{GILBERT}, 1996, p. 25, nota 2).}. Neste
sentido, o que era apenas um instrumento de fachada no ordenamento
político veneziano, em Florença, sob o governo republicano, passa a
funcionar de fato\footnote{Notório saber que, quando da restauração do
  governo dos Medici em 1512, um de seus primeiros atos foi a demolição
  do salão onde funcionava o Grande Conselho.}. Apesar dessa modificação
constitucional importante, novas demandas se faziam sentir, levando à
continuidade do debate sobre a melhor forma de governo. Em todos esses
momentos de confronto político por reformas nas instituições
republicanas da cidade, o exemplo veneziano sempre voltava à baila,
tanto que, na reforma de 1502, tem-se a instituição de um
\emph{gonfaloniere a vita}, ou seja, a versão florentina para o
\emph{Doge} veneziano, esse o chefe do executivo.

No \emph{Príncipe} e em vários capítulos dos \emph{Discursos sobre a
primeira década de Tito Lívio}\footnote{Doravante somente citado como
  \emph{Discursos}.}, Maquiavel apresentará afirmações contrárias às
posições teórico-políticas aristocratas. No caso de sua análise sobre a
\emph{História de Roma} de Tito Lívio, aquela discordância ganha uma
característica especial, pois Maquiavel se propôs a tomar como
referência de reflexão a mesma obra sobre a qual o aristocrata Bernardo
Rucellai, cunhado de Giovanni de Medici (que havia governado Florença
até 1492) já havia tecido seus comentários. Partindo dos mesmos métodos
analíticos que Rucellai, Maquiavel retira da história romana conclusões
opostas às dele. Conclusões não somente desfavoráveis à aristocracia,
mas aos ideais e modelos propostos por ela. Como afirma:

\begin{quote}
E além disso, levantar-me-ei contra as a opinião de muitos, segundo a
qual Roma foi uma república tumultuária {[}\ldots{}{]}. Direi que quem
condena os tumultos entre os nobres e a plebe parece censurar as coisas
que foram a causa primeira da liberdade de Roma e considerar mais as
assuadas e a grita que tais tumultos nasciam do que os bons efeitos que
eles geravam (\emph{Discursos}, \versal{I}, \versal{IV}).
\end{quote}

Na sequência, ele insiste no elogio aos conflitos:

\begin{quote}
E não se pode ter razão para chamar de não ordenada uma república
dessas, onde há tantos exemplos de \emph{virtù}; porque os bons exemplos
nascem da boa educação; a boa educação, das boas leis; e as boas leis,
dos tumultos que muitos condenam sem ponderar (\emph{Discursos}, \versal{I},
\versal{IV}).
\end{quote}

Os conflitos políticos são mobilizados num momento do texto no qual
Maquiavel busca uma outra fonte ou origem para os bons ordenamentos,
após ter mostrado que o modelo histórico e determinista polibiano sobre
os destinos dos regimes políticos, também conhecido como a teoria da
\emph{anacyclosis}, não dava mais conta de explicar as mudanças
políticas e a insuficiência do legislador em bem ordenar a cidade
(\emph{Discursos}, \versal{I}, \versal{II}). Os conflitos políticos se apresentam nos
capítulos \versal{III} e \versal{IV} dos \emph{Discursos} como a melhor alternativa para a
fundação de ordenamentos em cidades que não tiveram a sorte de ter um
sábio legislador, como foi Licurgo para Esparta. No caso de Roma, essa
ausência foi suprida ao acaso pelos conflitos políticos. Mais do que
isso, para Maquiavel, a experiência romana mostrou que, dessa maneira, a
fundação das cidades seria mais segura, haja vista que resultaria da
luta dos dois humores ou grupos sociais presentes em todas as cidades:
os grandes (\emph{popolo grasso}) e os pequenos (\emph{popolo minuto}).

Como veremos adiante, Maquiavel demonstrará a maior adequação do regime
republicano calcado nos setores populares contra a posição teórica de um
regime republicano de caráter aristocrático. Assim, se a aristocracia
florentina admirava o regime veneziano, sua estabilidade política, sua
natureza e seu virtuosismo aristocrático, Maquiavel verá nesses mesmos
aspectos fraqueza e enxergará a virtude nos conflitos políticos, na
instabilidade dos regimes republicanos. Ao contrário de pensar no bom
regime como uma república de tipo aristocrático, ele destacará as
qualidades populares das repúblicas. Será justamente na parcela popular
do governo republicano de Roma, e não nos seus quadros aristocráticos,
que esta encontrara sua força, seu vigor e sua grandeza.

Se, para um autor como Leonardo Bruni (séc. \versal{XV}), representante daquilo
que ficou conhecido como o \emph{humanismo cívico} (ou seja, a expressão
do pensamento político do Renascimento italiano), a exaltação de Roma,
acompanhada da afirmação de que Florença era sua filha, implicava a
defesa da liberdade política, um louvor às suas origens e de sua
excelência virtuosa, em Maquiavel cessa o tempo da apologia e começa o
tempo da crítica. Ao pensar em Roma como o modelo que inspiraria
Florença, ele ressaltará os contrastes ao invés dos paralelos, as
diferenças ao invés das semelhanças. Para o Secretário florentino, se a
república de Florença tivesse um ordenamento político assentado mais
sobre o povo e menos sobre a aristocracia, os destinos de sua cidade
poderiam ter sido outros. O que antes se colocava como uma ampla crítica
à aristocracia, agora se restringe à aristocracia florentina, que, nesse
aspecto, foi pior para os destinos da cidade do que a aristocracia
romana. Essa acusação atinge o cerne da ideologia dos aristocratas: a de
que Florença estava revivendo a \emph{virtus civita} da Roma
republicana.

Por outro lado, para os republicanos aristocráticos florentinos, os
conflitos políticos presentes na república romana seriam sinais da
corrupção política e, portanto, da perda da virtude cívica. Com efeito,
é somente numa certa compreensão da \emph{virtus civita}, assentada na
força e no poder romano, ou seja, numa adequação aos ideais humanistas,
que se poderia pensar na virtude como fundamento político (\versal{SKINNER}, 2000
e 2006). Para eles, a corrupção política romana começa quando se
manifestam os tumultos políticos, quando a unidade política da cidade se
vê fraturada pelas contendas entre os grupos. Não é sem fundamento que
vários pensadores, tanto antigos quanto modernos, entenderam a crise e a
decadência das repúblicas como associadas à perda da \emph{virtus
civita}, manifesta pelo conflito, índice maior da corrupção na cidade.

Como apontaram Gilbert (1996), Tafuro (2004, 2005) e Bignotto (1991), o
modelo republicano defendido pela aristocracia florentina representava
uma defesa das posições políticas desse grupo. Ao insistir que as
principais esferas decisórias (os Conselhos superiores da república
florentina) deveriam ser compostas majoritariamente ou totalmente (como
foi o caso do Conselho de Justiça) por membros da aristocracia; ao
defender que um ordenamento político assentando nesse extrato garantiria
a paz e a estabilidade política; e, principalmente, ao sustentar que
essa proeminência da aristocracia devia-se à sua virtuosidade, ao seu
amor e dedicação à pátria, estavam eles construindo uma teoria
republicana de caráter aristocrático. De fato, é nessa qualidade cívica
superior, que se expressa pela \emph{virtù} dos gentis-homens, que se
justifica o destaque dos segmentos aristocratas na vida política de
Florença\footnote{Nesse sentido, discordamos de Araújo (2000, p. 15),
  quando diz que para ``Maquiavel a instabilidade política é
  indesejável''. Ademais, se não se pode falar em toda a tradição dos
  autores republicanos, ao menos em Maquiavel a ampliação do poder
  político, inserindo novos atores para além da aristocracia ou dos
  homens dotados de virtude, ou seja, os segmentos mais populares, é não
  somente desejável, como é a resposta para o fantasma da corrupção
  política.}. Ideais esses que não se mostravam como uma novidade
teórica, haja vista a semelhança deles com algumas formulações de regime
misto no qual a aristocracia é considerada o segmento político mais
relevante na ordenação, como se pode perceber em Platão, Políbio e nos
primeiros escritos de Cícero (\versal{LEPORE}, 1954).

Neste contexto político e cultural, a posição de Maquiavel é
privilegiada, pois não somente ocupou um cargo estratégico (o comando de
uma importante secção da vida política da cidade e das relações
políticas entre os Estados), como pode conhecer a fundo o funcionamento
dessas engrenagens, e também participar desse debate político como um
interlocutor relevante.

\subsubsection{A saída da Chancelaria e o final da vida}

Depois que o governo Soderini cai e Maquiavel é demitido de suas
funções, em setembro de 1512, ele passa a produzir a parte mais
significativa de suas obras, entre as quais: \emph{O Príncipe} (em
1513), os \emph{Discursos sobre a Primeira década de Tito Lívio}
(1515-1517), a \emph{História de Florença} (1520-1525), a \emph{Arte da
Guerra} (1519- 1521) e os opúsculos políticos.

Entretanto, a saída de Maquiavel da Chancelaria é um fato conturbado e
nebuloso, mas que, uma vez compreendido, ajuda a entender a inserção
dele no ambiente político e intelectual florentino. Entretanto, convém
recuar um pouco no tempo.

Em 1502, por pressão da rica aristocracia florentina (os \emph{nobres}
segundo Maquiavel), ocorre uma mudança constitucional com a instauração
do \emph{gonfaloniere a vita}, ou seja, o comandante político da cidade
passa a ser um cargo vitalício, como era o \emph{doge} de Veneza. Isso
implica um ganho de poder por parte do \emph{gonfaloniere} Pier
Soderini, que consolida sua força política perante os outros setores
aristocráticos, bem como mantém seu grande apoio popular. A partir desta
data, conforme Guidi (2009), Maquiavel passa a ocupar uma posição
central na vida política florentina. Mesmo não sendo de família
aristocrática, o que o impedia formalmente de ser um embaixador ou
chanceler, ele fica responsável por diversas negociações importantes,
sendo que os embaixadores florentinos eram chamados ao final apenas para
assinar os acordos e os tratados.

Após 1502, Maquiavel também vai se envolvendo cada vez mais com as
questões militares da cidade, escrevendo e estudando sobre o assunto, a
ponto de, em 1506, ser o responsável pela criação da milícia florentina,
algo até então inexistente, pois a prática era a contratação de
exércitos ou tropas mercenárias para fazer as guerras.

A queda do governo Soderini começa quando as tropas espanholas invadem a
Toscana e tomam a cidade de Prato (vizinha de Florença) em agosto de
1512. A derrota retumbante das tropas florentinas perante a força dos
exércitos espanhóis desmoraliza o governo de Soderini que, em 31 de
agosto de 1512, foge para a cidade de Siena e depois segue para o exílio
na Dalmácia, onde é hoje a Croácia.

Com a queda de Soderini, quem assume o poder é inicialmente seu
adversário Giovan Battista Ridolfi, que inicia a reforma do governo mas
que é substituído, alguns meses depois, por Guiliano de Medici, que em
breve seria eleito papa (\versal{MARTELLI}, 2006, p. 9-11).

Maquiavel é demitido dois meses depois da derrota de Prato e logo em
seguida é preso sob a acusação de conspirar a morte de Giuliano de
Medici. Maquiavel fica dois meses preso, é torturado e solto por um fato
inusitado: a eleição de Giuliano ao papado em março de 1513. O papa
recém-eleito concede, então, anistia aos presos políticos de Florença. O
problema dessa história toda está em entender por que Maquiavel, que foi
demitido de suas funções, preso e torturado pelos partidários dos
Medici, pensa em dedicar a sua obra ao papa (que era sua intenção
inicial), mas depois dedica ao sobrinho deste, Lorenzo de Medici? Mais
ainda, mesmo sendo identificado ao governo de Soderini, do qual de fato
era o mentor intelectual, mas tendo amigos entre os partidários dos
Medici, haja vista a contínua troca de correspondência e favores após
1512, por que esses e o próprio Giuliano de Medici, que bem conheciam
Maquiavel, não entenderam logo que ele não fazia parte da conspiração,
mas que seu nome foi colocado numa lista de conspiradores sem de fato
ele mesmo saber?

A resposta para esses fatos explica muito do contexto florentino de
então. Como escreverá Maquiavel em um texto dessa época, \emph{Riccordo
di Niccolò Machiavelli ai Palleschi} (ou somente \emph{Ai Palleschi,} do
final de 1512), o problema que envolvia Florença e ele em particular
eram as oposições ferozes de parte da oligarquia florentina ao seu nome.
Nesse texto, escrito no calor dos acontecimentos, Maquiavel faz uma
análise deste contexto político se dirigindo aos \emph{palleschi} ou
\emph{pallesco}, ou seja, como ficaram conhecidos os apoiadores dos
Medici. Nele, o Secretário Florentino mostra que o problema político não
era Soderini e seu governo, pois, se assim o fosse, a sua queda teria
dado à cidade a tranquilidade política esperada; todavia, a
instabilidade permanecia. Curioso notar por este opúsculo que ele não
possuía grandes diferenças ou animosidades com os partidários dos
Medici. Na verdade, o texto indica com precisão que os verdadeiros
adversários de Soderini (parte da camada rica da cidade, ou seja, parte
da oligarquia) não serviria de apoio para o novo regime Medici.
Simplificando, Maquiavel avisa que os seus inimigos (pois essa
oligarquia também era inimiga dele) não seriam fiéis apoiadores do
regime dos Medici.

Portanto, Maquiavel não tinha nos Medici e seus apoiadores inimigos ou
adversários políticos, embora não se possa dizer de nenhum modo que ele
fizesse parte desse grupo. O que permite entender agora porque ele não
culpou os Medici pela sua prisão e dedicou a sua obra para o seu líder,
pois sabia que os seus reais adversários eram parte da aristocracia rica
e não o grupo político dos Medici. Esse pequeno texto maquiaveliano é,
então, muito ilustrativo para explicar o episódio da queda do governo de
Soderini e a prisão de Maquiavel, mas é mais explicativo de passagens de
\emph{O Príncipe}, como se verá melhor adiante, quando ele diz que o
príncipe novo não deve manter sua força política apenas apoiada nos
nobres ou grandes, mas, se tiver que escolher entre os nobres e o povo,
que escolha o povo (\emph{Príncipe,} \versal{IX}). Enfim, desde o ingresso no
governo da cidade, passando pela reforma de 1502 até a sua queda,
Maquiavel teve de fato entre seus maiores adversários políticos as
oligarquias ou \emph{nobreza} locais.

Esse panorama no qual se insere a vida pública de Maquiavel revela, por
seu turno, um vivo ambiente de disputas políticas que, como não poderia
deixar de ser, também foi marcado pelo debate intelectual sobre a melhor
forma de governo republicano para Florença. Este ambiente intelectual é
decisivo para a reflexão política maquiaveliana. Mais adiante,
explicaremos melhor a influência desse contexto sobre a produção
intelectual do \emph{Príncipe}, mas, por ora, podemos dizer que, do
ponto de vista geral de sua reflexão teórica, esse contexto de debate
marca os textos maquiavelianos na medida em que perpassa em todos a
busca pela definição de qual a melhor forma de governo republicano e
qual a natureza deste.

Por fim, convém lembrar que a Itália do período maquiaveliano não era um
Estado unificado sob o controle de um único governo, mas, ao contrário,
um conjunto de territórios independentes com governos autônomos. A
cidade de Florença, que possuía total autonomia e independência
política, constituía-se como um importante centro político, econômico e
cultural, no qual se alternavam governos republicanos, como o do período
em que Maquiavel foi diplomata, e autocráticos, caracterizados pelo
domínio da família Medici.

Quando, em 31 de agosto de 1512, o governo republicano de Pier Soderini
cai e os Medici retornam ao poder em Florença, Maquiavel se vê na
iminência de deixar o cargo na Chancelaria. Com efeito, em novembro do
mesmo ano ele é destituído de seu posto. Não bastasse a perda das suas
funções, que tanto prezava, poucos meses após sua demissão, em fevereiro
de 1513, ele é preso e torturado, sob a falsa acusação de participar de
um complô para assassinar um membro da família Medici. Ao sair do
cárcere, Maquiavel deixa a cidade de Florença e vai viver em sua pequena
propriedade rural, em Sant'Andrea in Percussina. É nesse pequeno
vilarejo, a poucos quilômetros de Florença, que Maquiavel passará o
restante de seus dias.

Não tendo mais obrigações diplomáticas, durante o dia ele se ocupa dos
negócios da propriedade, recolhendo-se à noite em seu escritório para o
estudo e a redação de suas obras. Nesse seu exílio forçado, ele escreve
seus principais textos políticos: \emph{O Príncipe}, \emph{Os Discursos,
A Arte da Guerra} e a \emph{História de Florença.}

Em 1515, Maquiavel passa a frequentar um encontro de jovens aristocratas
nos jardins da família Rucellai, em Florença, encontros esses que
ficaram conhecidos como os encontros dos \emph{Orti Oricellari}. São
desses encontros que nascem a maior parte dos \emph{Discursos}. A partir
dessa época, Maquiavel também volta a assumir alguns encargos
particulares, como quando, em 1519, foi representante de interesses dos
comerciantes de Florença em Lucca. Fora esses e outros poucos trabalhos
avulsos, Maquiavel não desenvolve nenhuma atividade diplomática regular.

Em 1520 ele recebe, por meio do \emph{Studio} florentino e do papa Leão
\versal{X} (Guiliano de Medici), o encargo de escrever uma história da cidade de
Florença. Durante quatro anos ele trabalha nesta obra, concluindo-a em
1525, indo pessoalmente a Roma para presenteá-la ao novo Papa Clemente
\versal{VII} (Giulio de Medici), que a recebe com apreço.

Aparentemente, tudo indicava a volta de Maquiavel às suas funções
diplomáticas, principalmente quando, em de maio de 1527, a família
Medici é deposta e é instaurado um novo governo republicano em Florença.
Mas a fortuna também não vem ao seu auxílio desta vez. Por uma grande
ironia do destino, a acusação que em um primeiro momento lhe fez sair do
governo quando das ascensão dos Medici (ser um defensor do regime
republicano), não se fez presente quando a república foi restaurada,
quinze anos depois. Diante desse revés, a sua saúde não resiste e ele
morre pouco mais de um mês depois, a 21 de junho.

\subsection{As edições}

O texto que ora se apresenta como \emph{O Príncipe} de Nicolau Maquiavel
tem na história de sua elaboração e difusão algumas peculiaridades. A
primeira informação sobre sua confecção vem de uma carta de Maquiavel a
Francisco Vettori de 10 de dezembro de 1513, na qual o autor fala da
composição de um opúsculo intitulado \emph{De Principatibus}, como diz:

\begin{quote}
E, como disse Dante, não pode a ciência daquele que não guardou o que
ouviu -- noto aquilo de que pela sua conversação fiz cabedal e compus um
opúsculo, \emph{De Principatibus}, onde me aprofundo quanto posso nas
cogitações deste tema, debatendo o que é principado, de que espécies
são, como eles se conquistam, como eles se mantêm, por que eles se
perdem. {[}\ldots{}{]} Portanto eu o dedico à magnificência de
Juliano\footnote{Carta a Francisco Vettori, 10 de dezembro de 1513.
  \emph{In}: \versal{MAQUIAVEL}, N. \emph{O Príncipe}. Trad. Lívio
  Xavier. São Paulo: Abril Cultural, 1973. p. 119 (Os Pensadores, \versal{IX}).}.
\end{quote}

Sabe-se, pois que, no final do ano de 1513, Maquiavel já havia terminado
o seu pequeno texto sobre ``que coisa é o principado, de quantas
espécies são, como se conquistam e se conservam''. Ainda nessa
carta, Maquiavel declara que este texto já havia sido lido por outro
amigo, Filippo Casavecchia, com o qual ele teve oportunidade de discutir
o texto. Segundo Inglese, certamente, entre o texto enviado a Vettori em
dezembro e a resposta deste em 18 de janeiro de 1514, Maquiavel foi
polindo e retocando o seu texto, que se constitui de fato na primeira
parte do livro. Com efeito, por aquilo que é indicado na carta de
dezembro de 1513, temos o roteiro dos temas tratados entre o capítulo \versal{I}
e o \versal{XI}, de uma obra que contém 26 capítulos.

Depois das dificuldades e negativas de Vettori em entregar o opúsculo de
Maquiavel para o papa, há poucas notícias sobre o manuscrito
maquiaveliano. Ele dará notícia novamente por mais sete vezes em suas
correspondências pessoais, sendo a última, em maio de 1514.

Talvez em função do malogro de entregar ao papa a obra Maquiavel tenha
modificado o destinatário da obra, não mais o papa, Giuliano de Medici,
mas a seu sobrinho, Lorenzo di Piero di Medici, então chefe político de
Florença, tal qual estabelecido pela tradição editorial. Um dado é
certo, esta dedicatória é anterior a outubro de 1516, quando Lorenzo
recebe o título de Duque de Urbino, coisa a que Maquiavel não se refere
na carta dedicatória, mas que jamais o teria feito se Lorenzo já tivesse
recebido o título, pois essa seria uma falta de decoro grave (\versal{INGLESE},
1994, p. 7). Portanto, a partir da própria documentação manuscrita e das
correspondências de Maquiavel, sabe-se que este obra foi composta entre
1513 e outubro de 1516.

Tendo em vista a não aceitação do texto pelo Papa, o opúsculo \emph{De
Principatibus} de Maquiavel passou a circular de forma manuscrita entre
os seus amigos durante muitos anos, pois a primeira edição impressa
sairia somente em 04 de janeiro de 1532. Ou seja, entre o final de 1513
e janeiro de 1532, o texto maquiaveliano circulou de modo informal e
manuscritamente. Ora, tendo em vista que o autor morre em 1527, nasce um
problema incomum para as obras compostas após a invenção da impressa no
século \versal{XV}, a saber: qual é o grau de originalidade do texto que foi
publicado em 1532? Será ele de fato a primeira versão do texto feita por
Maquiavel ou uma cópia de segunda ou terceira mão, com acréscimos,
supressões e demais alterações realizadas de modo costumeiro pelos
copistas? Enfim, qual o grau de originalidade do texto italiano de
\emph{O Príncipe} impresso pela primeira vez em 1532?

Tal dificuldade se amplia tendo em vista que, segundo relata Inglese
(1994, p. 10- 14), após 1514 o texto passou a circular de forma
manuscrita entre os amigos mais próximos de Maquiavel. Uma nova
referência ao texto apenas retorna em 29 de julho de 1517, quando o
jovem Nicollò Guicciardini, escrevendo ao seu pai Luigi, então
comissário florentino na cidade de Arezzo, sugere a ele se comportar
como disse ``Maquiavel em sua obra \emph{De Principatibus}'' (\versal{INGLESE},
1994, p. 14). Luigi Guicciardini conhecia Maquiavel dos tempos da
Chancelaria, tendo recebido certamente uma cópia do texto sobre os
principados. Luigi era irmão de Francesco Guicciardini, também diplomata
e grande pensador político, que escreverá um comentário aos
\emph{Discursos} de Maquiavel e que certamente deve ter lido esse
opúsculo maquiaveliano antes de 1532, pois faz referência a passagens
desse na sua obra \emph{Discorso del modo di assicurare lo stato alla
casa de' Medici}, texto de 1516 (\versal{INGLESE}, 1994, p. 15), bem como existem
passagens do \emph{Reggimento di Firenze}, obra elaborada entre 1521 e
1524, que indicam o conhecimento da argumentação do \emph{De
Principatibus} de Maquiavel (\versal{PROCACCI}, 1995, p. 6). Outra informação
sobre a circulação do texto vem de dois discursos proferidos por
Lodovico Alamanni, expoente do grupo dos Medici que, em 25 de novembro e
27 de dezembro de 1516, pede aos governantes que se comportem de modo
cívico, em uma argumentação que procura imitar a exposição de \emph{O
Principe}, seja no tema e personagens mobilizados, seja no estilo.

Contudo, segundo Inglese, o grande divulgador do \emph{De Principatibus}
de Maquiavel em forma manuscrita foi seu auxiliar dos tempos de
Chancelaria, Biaggio Buonaccorsi (\versal{INGLESE}, 1994, p. 17-18). Ele, que era
amigo de Maquiavel e nutria também um grande interesse pelas questões
políticas, após a saída da Chancelaria, em novembro de 1512, dedica-se,
entre outras coisas, à cópia de textos e monta um ateliê para
isso\footnote{Cumpre lembrar que, mesmo após a invenção da imprensa,
  manteve-se o costume, entre as famílias ricas, de mandar confeccionar
  manuscritos de obras importantes, que eram produzidos com qualidade,
  seja nas folhas utilizadas, seja na encadernação. Certamente o ateliê
  de Buonaccorsi recebeu encomendas de membros da aristocracia
  florentina, sabedores do texto de Maquiavel e curiosos para
  conhecê-lo. Inglese informa ainda que, mesmo depois da publicação da
  edição impressa em 1532, houve ainda a confecção de textos manuscritos
  do \emph{De Principatibus} por alguns anos.}. Conforme os estudos
paleográficos, de seu ateliê saíram ao menos três manuscritos
importantes que compõem o aparato crítico contemporâneo. Mais ainda, dos
27 manuscritos tomados em consideração por Inglese para realizar sua
edição crítica, 16 deles foram cópias feitas a partir desses exemplares
de Buonaccorsi. Isso indica que mais da metade dos manuscritos tem uma
influência direta desses exemplares \emph{buonaccorsianos} e certamente
ele teve acesso ao manuscrito original de Maquiavel.

Antes de prosseguir nessa apresentação do texto, é importante dar
algumas informações básicas sobre paleografia. Toda vez que se tem uma
edição de uma obra impressa não revisada e autorizada pelo autor, nasce
a dificuldade de saber se o texto impresso publicado confere com o
original, também conhecido como \emph{autógrafo}, ou seja, escrito pelo
autor. Quando este é vivo no momento da publicação ou se tem o
manuscrito \emph{autógrafo} ou se tem algum documento (carta,
testamento, etc.) que comprova que aquilo que foi publicado era a
intenção do autor, de modo que não há problemas e donde a primeira
edição impressa, também conhecida como edição \emph{princeps}, tornar-se
a referência para todas as demais publicações.

Contudo, quando a publicação impressa é posterior à morte do autor ou
este ainda em vida não reconhece como verdadeiro aquilo que foi
publicado -- como no caso do filósofo inglês John Locke que, durante
muito tempo, travou uma batalha com os editores de seu \emph{Segundo
Tratado sobre o governo civil} em função das diversas alterações
inseridas pelo editor, e só veio a reconhecer como seu o texto publicado
apenas no final de sua vida (\versal{LASLETT}, 1998) --, ou ainda se a obra foi
divulgada antes do advento da imprensa, então faz-se necessário um
estudo sobre os manuscritos existentes para se estabelecer o texto
fidedigno ou o mais fidedigno possível. O texto publicado que é o
resultado desse trabalho de levantamento e análise dos manuscritos
denomina-se \emph{edição crítica}, cujo texto, se não é idêntico ao
original do autor, ao menos reproduz aquilo que é o mais próximo deste.
Ora, tal trabalho de pesquisa pode ser facilitado ou dificultado
conforme o material que se tenha a disposição. Quando se encontra o
\emph{autógrafo}, a pesquisa se encerra, pois todas as demais
\emph{cópias} manuscritas foram feitas a partir deste primeiro exemplar
ou ele passa a ser a referência para as demais cópias e o texto padrão
da edição impressa. Todavia, quando não há esse exemplar, deve-se
recorrer às técnicas de paleografia e filologia para tentar estabelecer
as relações de dependências entre as cópias e identificar qual seria o
autógrafo ou aquele que mais se aproxima deste, quando ele não existe.

Voltando à difusão do texto do \emph{De Principatibus}, uma outra grande
fonte de informação é o possível plágio de Agostino Nifo, que publicou
em outubro de 1522 o opúsculo \emph{De Regnandi Peritia}, que não
somente mantém a mesma sequência, como reproduz exemplos e argumentos
inteiros do \emph{De Principatibus}, embora o texto de Nifo tenha sido
escrito em latim e o de Maquiavel em italiano (\versal{INGLESE}, 1994, p. 18-22).
Contudo, como explica Larivaille (1989, p. 150-195), falar em plágio de
Nifo sobre o texto de Maquiavel é um tanto equivocado, pois, apesar das
inúmeras semelhanças, o que revela que Nifo leu o texto de Maquiavel,
seu argumento conduz a conclusões diferentes das expostas no \emph{De
Principatibus}, mantendo a tradição de pensamento político de tipo
moralista dos \emph{espelhos de príncipes}\footnote{Apenas adiantando
  algo que será melhor explicado, havia uma tradição de longa data de
  livros de aconselhamentos para os príncipes, também denominados
  \emph{espelhos de príncipes}, que exaltavam a virtude e moralidade
  como as principais qualidades que o príncipe deveria cultivar. Como
  veremos, o texto maquiaveliano é totalmente contrário a isso, não
  sendo possível enquadrá-lo nesta categoria. (\versal{SKINNER}, 2000)}\emph{.} A
conclusão, portanto, é que Nifo leu o texto maquiaveliano e se inspirou
nele, e não o plagiou para escrever sua obra. Ademais, a noção de
plágio, tal qual nós entendemos, não existia neste contexto do
Renascimento, pois os autores se utilizavam de outros sem fazer a devida
referência e até o próprio Maquiavel fará uma verdadeira paráfrase do
livro \versal{VI} das \emph{Histórias} de Políbio nos seus \emph{Discursos} sem
dar qualquer referência ou indicação do texto antigo. Como se verifica
ao longo de várias passagens de \emph{O Príncipe}, Maquiavel cita de
memória passagens de textos históricos romanos, com frequentes erros e
sem qualquer referência.

Sabe-se ainda que Agostino Nifo era um pensador influente e foi
escolhido pelos Medici para lecionar no \emph{Studio} florentino, atual
Universidade de Pisa. Ora, certamente o fato de ele ter lido o texto
maquiaveliano e se baseado nele para escrever o seu próprio texto
demonstra que o \emph{De Principatibus} de Maquiavel era um texto um
tanto conhecido nos círculos eruditos de Florença, contando inclusive
com alguma aceitação e respeitabilidade.

Após essa última informação sobre \emph{O Príncipe} advinda da obra de
Nifo, só teremos notícia do texto maquiaveliano quando de sua publicação
em 04 de abril de 1532 pelo editor Antonio Blado, de Roma, que publica a
obra com o título em italiano, \emph{Il Príncipe}, e não mais em latim,
\emph{De Principatibus.} Logo em seguida, em 08 de maio, o tipógrafo
Bernardo Giunta, de Florença, também publica o texto com o título também
em italiano. Apesar de os estudos paleográficos e filológicos colocarem
em questão a fidedignidade da edição Blado em relação ao
\emph{autógrafo} do \emph{De Principatibus}, ela é reconhecida como a
edição \emph{princeps}, que significa que é primeira edição e sobre o
qual se deve basear a numeração e referência das demais edições.

Como se nota, foi do editor a decisão de alterar o título do livro do
latim para o italiano, cuja tradução mais correta de \emph{De
Principatibus} seria em italiano \emph{Sopra i principati} (Sobre os
principados). Outra mudança significativa entre a carta a Vettori de
dezembro de 1513 e a edição publicada é a inserção de uma parte
reservada às questões militares (capítulos de 12 a 14) e toda uma seção
destinada à figura do príncipe, com uma conclusão exortando à união da
Itália. Essa ampliação do texto sugere que Maquiavel tenha acrescido e
modificado o texto depois de dezembro de 1513. Contudo, a questão é:
quando ele concluiu de fato a obra e, como já foi discutido entre os
comentadores, teria tido \emph{O Príncipe} duas redações?

Quanto à primeira questão, a hipótese mais provável, em função das datas
e personagens citados, é que Maquiavel tenha ampliado e alterado o texto
ao longo de 1514, mas que o concluiu definitivamente antes de 1515 em
função de informações sobre alguns personagens e datas. Segundo Procacci
(1995, cap. 1), tendo em vista a difusão e aceitação dos manuscritos em
seu circulo mais próximo, Maquiavel tentou publicar o \emph{De
Principatibus,} mas em Roma e não em Florença, em função da oposição ao
seu nome entre a aristocracia desta cidade. Em sua última estadia em
Roma, em 1526, Maquiavel encomenda a cópia de um exemplar manuscrito de
seu texto ao ateliê de Ludovico degli Arrighi (o que é o atual
\emph{códice Barberiano 5093} do \emph{De Principatibus}) (\versal{PROCACCI},
1995, p. 8-9). Ora, por que Maquiavel encomendaria um novo exemplar
manuscrito e feito de modo acurado e elegante em um ateliê prestigiado
de Roma se não fosse para publicá-lo ou ofertá-lo a alguém que
patrocinasse tal projeto editorial? Segundo Procacci ainda, as conversas
já estavam adiantadas para isso, entretanto, a publicação não se
realizou em função do ataque que os franceses fizeram a Roma em 1527,
conhecido como o ``Saque de Roma'', que atinge duramente a cidade e suas
finanças. A retomada dos projetos somente ocorrerá após 1530, quando a
situação política parecia estabilizada. Em 1531, um tipógrafo sem muito
prestígio, proprietário de um negócio modesto, Antonio Blado di Asola,
recebe a autorização papal para publicar duas obras de Maquiavel: os
\emph{Discursos} e o \emph{De Principatibus}, cujo título ele modifica.
Blado trabalhou durante muitos anos a serviço da Câmara Apostólica e não
teve um catálogo nem amplo ou com títulos de renome ou significativos
(\versal{PROCACCI}, 1995, p. 9-10). Outro dado curioso é que, na autorização
papal, já se prevê a possibilidade de Blado negociar e permitir outras
edições dessas obras, fato esse que, segundo Procacci, demonstra que o
modesto tipógrafo já pretendia vender os direitos de uma obra que teria
boa repercussão editorial. Tanto é procedente tal interpretação que logo
na sequência tem-se a publicação do texto em Florença pelo editor
Bernardo Giunta e, nos anos seguintes, em Veneza, por três casas
tipográficas distintas (\versal{PROCACCI}, 1995, cap.1). Portanto, conforme a
interpretação de Procacci, a publicação do texto maquiaveliano era um
projeto intencionado desde 1526 por Maquiavel, mas que não ocorreu por
acasos: o ``Saque de Roma'' e sua morte.

A interpretação de Martelli é diferente e agrega alguns dados novos e
complementares. Na edição comentada de \emph{O Príncipe}, Martelli
(2002) explora em muito o fato de o nome de Maquiavel ser rejeitado em
Florença, mas não necessariamente pela família Medici. Isso impediu que
durante muito tempo seu texto fosse publicado e ele mesmo pudesse
retornar ao governo, então sob o comando da família. Mesmo sem ser
considerado um inimigo por parte da família Medici, segundo a
argumentação de Martelli, Maquiavel era um personagem cuja proximidade
provocava desconforto político perante parte da aristocracia florentina.
Do que decorre que sua obra não tenha sido publicada em Florença
inicialmente, mas em Roma e depois de sua morte, quando a oposição havia
em parte diminuído.

Entretanto, em alguns pontos Martelli se distancia das interpretações
precedentes. Segundo ele, o texto maquiaveliano começou a ser escrito em
1513 e foi concluído apenas em 1518 e não em 1514 ou início de 1515 como
defendem Inglese e outros. Além disso, mesmo admitindo que não existisse
o autógrafo do \emph{De Principatibus}, Martelli defende que há um
exemplar que teria sido corrigido por Maquiavel e que poderia ser
denominado como o \emph{arquétipo} do qual derivam todos os outros e que
seria esse manuscrito o mais genuíno\footnote{Segundo Martelli o
  manuscrito \emph{Carpentras}, \emph{Bibliothéque Inguimbertine, 303}
  (denominado como manuscrito A), seria esse arquétipo que foi corrigido
  por Maquiavel e seria o texto mais original, apesar de não ser um
  autógrafo. (\versal{MARTELLI}, 2002, 325-329)}.

Esses elementos invocados por Martelli nos remetem à já mencionada
questão recorrente nos estudos sobre \emph{O Príncipe}: teria ele tido
duas redações? Tal questão foi formulada inicialmente por Federico
Chabod e, depois de discutida longamente, teve sua resposta no estudo de
Sasso (1958) no qual se aponta que Maquiavel redigiu uma primeira parte
do texto em 1513 e o restante, provavelmente, em 1514, não alterando
mais o texto depois de 1515. Uma questão embutida nessa é saber se já no
período da Chancelaria Maquiavel não teria começado o texto,
concluindo-o no período imediatamente posterior a sua saída.

Tal hipótese de uma dupla redação e não de uma redação contínua no
biênio 1513 e 1514 é muito frágil, pois, mesmo que de fato Maquiavel
tenha começado a fazer o texto, ou mesmo que já tivesse um esboço ou
rascunho, certamente ele se valeu dessas informações anteriores -- como
ele mesmo declara na \emph{Carta Dedicatória} --, o que comprova que não
somente havia sim elementos anteriores a 1513, mas que tudo isso foi
reelaborado e ampliado nesse ano de redação. Sem esquecer que ele vinha
pensando nesses temas e argumentos ao longo do tempo, haja vista o
confronto com textos e escritos diplomáticos de análise de contextos,
com os textos posteriores a 1512, nos quais encontramos um autor mais
maduro e consciente de seu pensamento político. Do que se pode concluir
com muita segurança que Maquiavel já vinha de longa data meditando sobre
esses temas e que organizou e colocou no papel tudo isso no biênio de
1513 e 1514. Depois não mais.

Esse quadro informativo nos coloca diante do seguinte questão: o
manuscrito do \emph{De Principatibus} utilizado por Antonio Blado para
fazer a primeira edição impressa de \emph{Il Príncipe} era de fato o
autógrafo de Maquiavel? Mais, tendo em vista a alteração do título, qual
a garantia de que o editor não tenha alterado algo no texto, o que teria
corrompido sua autenticidade? Enfim, parece que a primeira edição
bladiana do texto maquiaveliano não confere nenhuma garantia sobre a
autenticidade do texto.

Essas suspeitas fizeram com que, já no século \versal{XIX}, houvesse uma primeira
tentativa de estabelecer um texto original ou padrão de \emph{Il
Príncipe}, por meio do editor G. Lisio, e editada por Sansoni em
Florença em, 1899. A também conhecida edição Lisio de \emph{Il Príncipe}
foi o primeiro texto com um mínimo de aparato crítico. Depois de algumas
edições que agregaram alguma informação ao texto, mas sem superar ou
corrigir a edição Lisio, Giorgio Inglese publica a edição crítica em
1994. Este, consultando uma gama maior de manuscritos e depois de um
trabalho apurado de análise paleográfica e filológica, apresenta uma
nova edição crítica do texto do \emph{De Principatibus} de Maquiavel,
publicada pelo Istituto Storico Italiano per il Medioevo. Além dessa
importante edição, em 2002 foi publicada uma outra edição de \emph{O
Príncipe}, dentro da coleção \emph{Edizione Nazionalle delle Opere di
Machiavelli}\footnote{Doravante citado apenas como \emph{\versal{EN}.}} pela
Editora Salerno, sob a coordenação de Mario Martelli, e que pretende
discutir e rever pontos não esclarecidos da edição Inglese, valendo-se
de outros manuscritos (que eram do conhecimento de Inglese mas que, por
critérios técnicos, foram por ele descartados) e buscando,
principalmente, refutar a tese de Inglese de que não há um autógrafo de
Maquiavel. Em outros termos, o ponto central da análise crítica de
Inglese é que o autógrafo do \emph{De Principatibus} está ainda perdido
e que a reconstrução dos manuscritos permite estabelecer os dois
principais exemplares dos quais dependem todos os demais e, com esses,
reconstruir o texto o mais próximo do original possível\footnote{A
  hipótese central de Inglese é que os manuscritos D (München.
  Universitätsbibliothek, 4º cod, ms. 787) e G (Gotha, Forschungs- und
  Landesbibliothek, chart. B 70) dependem de um outro manuscrito
  (denominado γ) que seria o arquétipo mais próximo do orignal. Cf.
  \versal{INGLESE}, 1994, p. 150-155.}. A contra-argumentação de Mario Martelli,
tenta, por outra metodologia, mostrar que há um autógrafo ou arquétipo
do texto maquiaveliano e que este foi aquele utilizado por Blado, no
caso, o manuscrito A (\emph{Carpentras}) da lista dos manuscritos
disponíveis.

Ora, não se trata, nesta introdução, de entrar nos detalhes de caráter
filológico ou paleográfico da discussão, visto que não temos elementos
para refutar posições ou novidades a acrescentar. Porém, cumpre declarar
que, no confronto das argumentações, seja em função da metodologia, seja
em função da autenticidade e autoridade dos manuscritos, a exposição de
Giorgio Inglese é mais consistente e o texto por ele organizado é mais
fidedigno. A argumentação de Martelli, ao utilizar-se de uma nova
``metodologia'' na qual a história do texto prevalece, não consegue, ao
fim e ao cabo, oferecer uma comprovação ou modificação substancial no
estatuto de originalidade do texto maquiaveliano. No limite, toda a
argumentação de Martelli apenas polemiza e não traz novos resultados em
relação a edição crítica de Inglese. Tanto é assim que o próprio
Martelli denomina sua edição de \emph{Edição Comentada.}

Donde, na falta de argumentos novos e de elementos concretos que
invalidem sua exposição, somos obrigados a aceitar a edição Inglese de
1994 do \emph{De Principatibus} como o texto mais fidedigno àquele
intencionado e escrito por Nicolau Maquiavel, motivo pelo qual o
escolhemos para ser o texto original italiano dessa edição bilíngue
brasileira.

\subsection{Estrutura do Argumento}

Como foi dito, \emph{O Príncipe} possuía um outro título que depois foi
alterado pelo editor romano, mas que não contradiz totalmente o conteúdo
da obra. Na verdade, pode-se dizer que o texto maquiaveliano contém o
\emph{De Principatibus} e \emph{Il Príncipe} ao mesmo tempo.

A obra está dividida em 26 capítulos, num total de 28.250 palavras. O
primeiro capítulo apresenta o roteiro dos temas tratados até o capítulo
11, que se concentram nos principados, na conquista destes e na
conservação do poder. Os três capítulos seguintes (12, 13 e 14) tratam
da questão militar, acerca da conveniência e inconveniência de se
possuir exércitos auxiliares e próprios, que também são pensados no
interior dos esforços de conservação do poder político. Os capítulos de
15 a 23 tratam das qualidades e cuidados que um príncipe deve possuir no
comando político. Os capítulos 24 e 25 analisam a importância da fortuna
na vida política e como o príncipe deve ficar atento a ela. Por fim, o
capítulo final é uma exortação à unidade da península itálica, algo
desejável sob o comando da família Medici.

Diante dessa disposição temática dos capítulos, uma primeira compreensão
geral da obra, sobre a qual concordam os comentadores, é dividir \emph{O
Príncipe} em duas grandes partes: uma primeira dedicada à análise do
principado e uma segunda voltada para a ação do príncipe. É evidente que
nesta divisão geral não se encaixam muito bem os capítulos finais,
particularmente o último.

Entretanto, seja essa divisão temática da obra, sejam as demais
informações apresentadas, tudo nos remete a uma ordem de problemas que
convém analisar com atenção. Um primeiro âmbito de problemas está
voltado à noção de principado, e pode ser expresso nos seguintes termos:
o que de fato é o principado de Maquiavel? Qual a sua definição e a sua
função no interior do pensamento político maquiaveliano? A resposta mais
comum nos manuais é identificar a noção de principado à moderna noção de
``Estado'', declarando desse modo que Maquiavel é um precursor do
pensamento político moderno. Como se verá, não se trata disso, pois
tanto a definição de principado como a de estado em Maquiavel não se
identificam com a moderna noção política de Estado, embora seja possível
reconhecer neles elementos que apontam para aquilo que o pensamento
político moderno, posterior a Thomas Hobbes, certamente identificou na
noção política de Estado, este entendido como uma entidade política
autônoma.

Uma outra questão que decorre diretamente dessa primeira ordem de
problema é sobre a concepção de príncipe de Maquiavel. A primeira e
imediata associação é com o título nobiliárquico de príncipe das
monarquias. Ora, essa concepção gera, por seu turno, a ideia de que o
texto maquiaveliano é um libelo em defesa da monarquia ou um livro
destinado aos nobres. Tal é o pano de fundo da argumentação de Quentin
Skinner, que no seu livro mais conhecido, \emph{As fundações do
pensamento político moderno}, classifica a obra de Maquiavel dentro de
um estilo que fez fama entre os séculos \versal{XI} e \versal{XVI}, os \emph{espelhos de
príncipes} (\versal{SKINNER}, 2000). Esse estilo de texto, que remonta ao
pensamento político romano, tendo como um de seus textos fundadores o
\emph{Dos Deveres} (\emph{De Officis}) de Marco Túlio Cícero (54 a.C.),
foi retomado ao longo do tempo no pensamento político latino, passando
pela Idade Média e Modernidade (\versal{SENNELART}, 2006)\footnote{No século
  \versal{XVIII} ainda se encontra esse tipo de texto entre os pensadores
  portugueses, o que comprova a fortuna de estilo no pensamento politico
  ocidental.}, e era destinado a auxiliar e instruir os governantes na
tarefa de governar, principalmente os jovens príncipes, que careciam de
experiências políticas, donde a definição ``espelhos de príncipes'': por
ser um manual onde os príncipes deveriam espelhar suas ações.

O desconforto em enquadrar o texto maquiavelino neste estilo reside mais
na definição de príncipe como o filho do rei, como o futuro monarca,
restringindo o escopo de denotação do termo, particularmente quando
tratamos de um autor que, como já foi declarado, assume-se claramente em
defesa do republicanismo. Disso decorre a contradição instalada: teria
um autor que é reconhecidamente defensor do republicanismo escrito um
livro em defesa da monarquia, tendo em vista para quem a obra é
dedicada, para tentar, entre outros motivos, um emprego no novo governo
da família Medici? Teria Maquiavel se destacado por escrever uma obra de
cunho monarquista por que estava desempregado e sem perspectivas? Enfim,
seria Maquiavel um pensador com uma dupla postura teórica: em algumas
obras defensor do regime republicano e em outra -- curiosamente na mais
célebre -- defensor do regime monárquico?

As respostas para tais problemas não são simples e nem fáceis, pois
demandam um esforço de análise e interpretação. O início dessa análise
está, todavia, fora de \emph{O Príncipe}, mas em outra obra, nos
\emph{Discursos.} Nesta obra, em particular, depois de apresentar os
elementos gerais da república, Maquiavel fala da corrupção desta, do seu
fim e como poder-se-ia encontrar alguma solução para a morte certa do
regime republicano. A resposta para essas noções de principado e
príncipe, bem como para a questão de fundo de como articular \emph{O
Príncipe} no interior do pensamento político maquiaveliano, passa por
uma análise sobre o fim da república. Enfim, a dificuldade que se
apresenta agora é de como articular \emph{O Príncipe} com o pensamento
republicano de Maquiavel, com as demais obras políticas de caráter
nitidamente republicano. Para isso será necessário recuperar, ainda que
rapidamente, os delineamentos gerais da noção de república tal qual
Maquiavel faz no início dos \emph{Discursos.}

\subsection{O ``Pequeno tratado sobre as repúblicas''}

No início da exposição dos \emph{Discursos sobre a primeira década de
Tito Lívio}, notadamente no primeiro livro, entre os capítulos 1 e 18,
encontramos um momento privilegiado no qual Maquiavel apresenta os
elementos principais do que entende por república\emph{.} Tais noções
também podem ser encontradas em vários escritos de modo esparso, como na
\emph{História de Florença}, no opúsculo \emph{Discurso sobre as coisas
de Florença}, na \emph{Arte da Guerra}, entre outros; mas em nenhum
deles temos de modo tão claro e estruturado como nos \emph{Discursos}.

Escrito, provavelmente, na sequência de \emph{O} \emph{Príncipe}, os
\emph{Discursos} expõem uma análise sobre parte da obra \emph{História
de Roma} de Tito Lívio. O objetivo do texto maquiaveliano é o de
comentar os fatos narrados pelo historiador romano e recuperar noções
para a elaboração de argumentos em defesa do regime republicano, no caso
particular para a Florença de início do século \versal{XVI}, bem como para uma
teoria em defesa do regime republicano de modo geral.

A \emph{História de Roma} foi a principal obra de Tito Lívio (59 a.C --
17 d.C). Composta originalmente por 142 livros, dos quais restaram
apenas 35, ela narra, em seu conjunto, os feitos romanos desde a sua
origem até o governo de Otávio Augusto (9 a.C.). Ao longo do tempo, os
copistas fizeram uma divisão da obra em grupos de dez livros, bem como
uma sinopse de cada livro que foram reunidas no início da
\emph{História}. A essa reunião dos livros em conjunto de dez, ainda que
nem sempre rígida, deram o nome de \emph{deca} (em italiano), que foi
traduzido por ``décadas'' em português, quando, na verdade, trata-se de
``dezenas''. Os dez primeiros livros ou a ``primeira década'' ou a
``primeira dezena'', foram dos poucos que se conservaram
integralmente e narram os fatos desde as origens de Roma até o ano de
295 a.C., ou seja, a narrativa compreende o governo monárquico e boa
parte do governo republicano. Maquiavel elabora os seus \emph{Discursos}
sobre esses dez primeiros livros da \emph{História de Roma}, uma vez que
nestes há a presença de vários temas que lhe são caros, entre eles, a
conservação da república, as mudanças de regimes e a corrupção das
instituições políticas.

Olhado em sua totalidade, o texto maquiaveliano parece ser um comentário
geral da obra liviana, no qual o comentador vai inserindo em sua análise
as ideias e as concepções políticas que busca defender. O problema maior
nessa visada geral são os primeiros dezoito capítulos do livro \versal{I}, no
qual Maquiavel não segue
\emph{pari passu} o
texto liviano, mas faz uma exposição ampla sobre alguns conceitos
republicanos, inserindo nessa análise exemplos romanos, mas também
exemplos de Esparta, Veneza, Florença e a Roma católica de seu tempo.
Como apontou Gilbert (1953), primeiramente, esses primeiros capítulos
parecem se constituir em um bloco à parte da obra. Doravante se instalou
entre os comentadores uma disputa interpretativa, na qual, entre outros
pontos de discórdia, pairava a questão se Maquiavel escreveu primeiro
esse trecho do livro ou se escreveu conjuntamente com o restante da
obra, entre os anos de 1515 a 1517. Questão essa que se complica, pois,
tendo em vista que Maquiavel teria escrito o \emph{De Principatibus}
antes dos \emph{Discursos}, nasce a dúvida de como entender um trecho
inicial de \emph{O Príncipe}, quando o autor declara:

\begin{quote}
Deixarei de lado a discussão sobre as repúblicas, porque, em outro
momento, dissertei longamente sobre elas. Ocupar-me-ei somente do
principado, e retecerei as urdiduras acima descritas, e demonstrarei
como estes principados podem ser governados e conservados (\emph{O
Príncipe}, cap. \versal{II}, 1-2).
\end{quote}

Resta, pois, a questão de qual seria esse texto republicano de Maquiavel
indicado no texto de 1513. Já nos antecipamos em afirmar que não é
possível, com os dados disponíveis até hoje, indicar se esses dezoitos
capítulos perfazem o texto republicano mencionado no início de \emph{O
Príncipe} e nem a cronologia correta de composição das obras. Tudo o que
se tem são hipóteses, algumas com base nos poucos dados históricos
disponíveis e outras com base na estrutura dos argumentos expostos por
Maquiavel em suas obras.

A polêmica em torno desses dezoito primeiros capítulos dos
\emph{Discursos} não se resume, então, a uma disputa cronológica de
anterioridade ou posteridade em relação a \emph{O} \emph{Príncipe}.
Subjaz a essa discussão a definição do sentido próprio da obra
maquiaveliana cuja interpretação determina a compreensão que ele possuía
das repúblicas, bem como do lugar de \emph{O Príncipe} no interior de
seu pensamento político. O estudo de Gilbert fez com que, sob diferentes
óticas e metodologias, as atenções para a interpretação do pensamento
político maquiaveliano se voltassem para os \emph{Discursos}. Uma vez
que não está ao nosso alcance reconstruir nos detalhes os pontos desse
debate e os meandros dessa contenda, algo que já foi analisado pelos
comentadores e que também abordamos em outro texto (\versal{MARTINS}, 2007, 16),
partiremos aqui do fato concreto de que esse conjunto de capítulos
iniciais perfazem aquilo que Larivaille nomeou como ``\emph{Pequeno
tratado sobre as repúblicas''}. Certamente, neste trecho da obra
maquiaveliana, encontraremos os elementos teóricos para compreender de
que modo pode ser possível pensar a inserção de \emph{O Príncipe} no
pensamento político de seu autor.

Voltando à passagem problemática de \emph{O Príncipe}, quando ele
escreve que não tratará das repúblicas porque já o havia feito em outro
lugar, deduz-se que estivesse se referindo aos \emph{Discursos,} visto
que não deixou um tratado específico sobre esse assunto e que aquela
obra traz uma reflexão sobre a república romana. Todavia, existem vários
indícios que mostram que os \emph{Discursos} foram escritos depois de
1514, quando Maquiavel frequentava os \emph{Orti Oricellari} \footnote{\emph{Orti
  Oricellai} era o nome que se dava aos jardins da família Rucellai,
  que, desde o final do governo dos Medicis, no século \versal{XV}, abrigava
  reuniões de aristocratas florentinos. Após a queda do governo de Pier
  Solderini, em 1512, o neto de Bernardo Rucellai, Cosimo, passou a
  organizar reuniões com jovens aristocratas de ideologia republicana.
  Maquiavel passa a frequentar esses encontros a partir de 1515, momento
  em que se acredita que ele tenha escrito a maior parte dos
  \emph{Discursos}. Todos os comentadores destacam a importância desses
  encontros para a reflexão política maquiaveliana no que diz respeito à
  teoria republicana e ao estudo dos clássicos (\versal{GILBERT}, 1977, p. 15-66;
  \versal{VIROLI}, 2002; \versal{SASSO}, 1986, p. 353-357). Adiante isso será melhor
  explicado.}, entre 1515 e 1517. Tais indícios colocam, pois, o
problema de tentar descobrir a qual texto Maquiavel estava fazendo
referência no início de \emph{O Príncipe}, visto que ele não
havia escrito, ainda, os \emph{Discursos.} Uma outra questão, paralela a
esta, estaria em saber ao certo qual foi o momento de composição dos
\emph{Discursos}: se antes ou depois da composição de \emph{O}
\emph{Príncipe}.

Esses são os pontos principais do debate em torno da datação dos
\emph{Discursos} e da existência ou não de um bloco textual destacado no
início do livro \versal{I}. Apesar dessas divergências, de modo geral, é aceito
que a parte principal do livro tenha sido escrita entre 1515 e 1517, o
que não exclui a possibilidade de que uma primeira parte já tivesse sido
escrita antes desta data, sendo apenas corrigida na época em que
Maquiavel frequentava os \emph{Orti Oricellai}. Depois das contendas
entre os comentadores, uma certa concordância se formou acerca de dois
pontos: a) que os capítulos que tratam dos comentários da \emph{História
de Roma} de Tito Lívio possuem uma unidade analítica a despeito da não
adequação de algum capítulo à essa regra geral, corroborando a tese de
Gilbert de que, nesses casos, as exceções confirmam a regra; b) que os
dezoito primeiros capítulos do livro \versal{I}, mesmo sendo o momento mais
teórico da obra, configuram-se como um anteparo conceitual explicativo
dos eventos que serão comentados. Apesar de não serem comentários
diretos aos livros de Tito Lívio, esses capítulos mostram-se importantes
na economia dos \emph{Discursos} como um todo, na medida em que explicam
as origens e os fundamentos das repúblicas e os ordenamentos, leis e
conflitos que marcam sua vida civil. Enfim, afirmar a existência de um
bloco teórico inicial não depõe contra a harmonia e unidade interna dos
\emph{Discursos}, impossibilitando qualquer afirmação de que a obra é
resultado de uma colagem ou de uma justaposição de textos diversos.

Os \emph{Discursos} estão divididos em três livros, sendo que o livro \versal{I}
tem 60 capítulos, o livro \versal{II} 32 capítulos e o \versal{III} 49 capítulos. Os
primeiros dezoito capítulos do livro \versal{I} perfazem a seguinte sequência
expositiva: a fundação das cidades (capítulo \versal{I}), a natureza e a mudança
dos regimes políticos (capítulo \versal{II}), os conflitos políticos (capítulos
\versal{III} e \versal{IV}), a defesa da liberdade política nas republicas (capítulos \versal{V} e
\versal{VI}), os instrumentos de defesa e acusação pública (capítulos \versal{VII} e
\versal{VIII}), a reforma ou refundação dos estados (capítulos \versal{IX} e \versal{X}), a
importância da religião (capítulos de \versal{XI} a \versal{XV}) e a corrupção nas
repúblicas (capítulos de \versal{XVI} a \versal{XVIII}).

Verifica-se, pois, que esses capítulos apresentam uma unidade teórica e
formam um bloco textual de análise dos fundamentos das repúblicas. Pelo
roteiro dos temas expostos, nota-se a presença de um itinerário
argumentativo cujo movimento vai do nascimento da cidade, passando pela
fundação dos ordenamentos políticos e o modo de defesa do \emph{libere
vivere} político, culminando na corrupção do povo, das instituições
(\emph{ordini}) e das leis. A religião, mobilizada no interior dessa
reflexão, também se apresenta como uma instituição capaz de conservar,
por meio de seus ritos, os valores e os ideais republicanos, ou seja,
ela cumpre o papel de \emph{instrumentum regni,} um instrumento para
governar. Além dessa descrição da vida política das repúblicas, na qual
se revelam as etapas de sua existência, pode-se também afirmar que esses
capítulos são uma introdução teórica aos \emph{Discursos}, uma vez que
definem os elementos essenciais na constituição de uma cidade.

Quando falamos em aspectos teóricos, não devemos ter em vista um certo
modelo de tratado em que os conceitos aparecem de modo destacado por
expressões próprias, como \emph{defino que}, \emph{demonstra-se},
\emph{entendo por} etc. Maquiavel utiliza-se de um outro estilo que não
é em nada menos indicativo de seu objetivo de definir conceitos. A
própria escolha e disposição dos temas é, por si, uma indicação de seus
propósitos teóricos. Assim, no capítulo \versal{I}, ao tratar da fundação das
cidades, ele faz, na verdade, uma descrição dos tipos de cidades que
podem existir e de como seu momento fundador pode ser determinante para
o desenvolvimento ou para a ruína futura, tipificando-as pelo seu modo
de fundação. No capítulo \versal{II}, faz, num primeiro momento, uma exposição
das formas de governo possíveis e de como nelas pode se processar a
mudança, indicando que o motor ou a causa desta não é uma certa lógica
determinista da natureza, mas dos conflitos políticos, tema dos
capítulos \versal{III} e \versal{IV}. Os capítulos \versal{I} e \versal{II} configuram-se, portanto, como
descrições tipológicas, seja do modo como pode se operar a fundação de
uma cidade, seja do modo como os regimes podem se instalar e se
transformar, descrições estas que fazem desses capítulos um preâmbulo
conceitual para a análise que se seguirá.

Na sequência, os capítulos de \versal{III} a \versal{X}, ao apresentarem temas como os
conflitos políticos, a defesa da liberdade política nas repúblicas, os
instrumentos de defesa e de acusação pública e a reforma ou refundação
das cidades, revelam como Maquiavel entende os elementos essenciais das
repúblicas. Como se verá, esses pontos configuram-se como as
\emph{ordini} ou os ordenamentos políticos básicos de uma república,
aspectos estes estruturais da vida de uma cidade ordenada como
república. Mesmo nos exemplos mobilizados, ele não se restringe ao caso
romano, mas fala de Veneza, Esparta, Florença, entre outras cidades,
numa clara indicação de que está apresentando as partes da vida política
em um regime republicano.

A exposição sobre a religião dos capítulos de \versal{XI} a \versal{XV} cumpre também essa
função na medida em que ela é considerada \emph{instrumentum regni}, uma
instituição cujo papel não se limita a ser de ordem religiosa porque é
decisiva na própria organização e no funcionamento da vida política da
cidade. Ao tratar da religião, o silêncio de Maquiavel em relação às
disputas medievais entre o papado e o império, fundamentais seja para o
futuro das cidades do norte da Itália, seja para o próprio período
medieval (\versal{SKINNER}, 2000, cap. 1-3), é indicativo de que a sua
preocupação não é especificamente a religião romana, mas sim analisar a
religião -- entendida eminentemente como prática social -- no que diz
respeito à sua relação com a vida política da cidade, ou seja, ao seu
papel político.

Quanto aos capítulos de \versal{XVI} a \versal{XVIII}, eles não são comentários à
corrupção romana, mas buscam entender como a corrupção pode atingir uma
república, seu povo ou suas instituições. Não se trata de uma explicação
do caso específico romano, mas da corrupção que pode acometer as
repúblicas de um modo geral.

Portanto, os temas mobilizados nesses capítulos iniciais e o modo como
eles são analisados, revelam o quanto Maquiavel fez, ao seu modo, uma
apresentação dos principais elementos políticos de uma república e de
seus fundamentos, já que não há uma análise exclusiva nem do caso
romano, nem dos livros da \emph{História} de Tito Lívio. Ora, é patente
que, nesse primeiro momento dos \emph{Discursos,} não se está fazendo
uma analise \emph{pari passu} do texto liviano, nem um comentário
histórico dos fatos de um modo geral. A coesão presente no objetivo e no
modo de exposição, a escolha dos temas e o itinerário que eles descrevem
-- do nascimento à corrupção da cidade --, confere um caráter unitário a
esse bloco textual. Os capítulos que se seguem ao \versal{XVIII}, nos quais
verifica-se, aí sim, um comentário à obra liviana, não denotam uma
inflexão teórica ou conceitual. A compreensão da \emph{História de Roma}
e o sentido ou o modo como essa análise deve ser feita seguirão os
critérios e as ideias apresentadas no início, mostrando uma imbricação
entre a análise dos fatos romanos à luz dos critérios apresentados. Os
comentários serão, portanto, pautados pelos conceitos e pelo modo de
compreendê-los, indicando uma profunda dependência entre as partes do
livro. Apesar de os dezoito capítulos se constituírem numa unidade
passível de ser analisada autonomamente, os \emph{Discursos} permanecem
um todo em sua estrutura, na medida em que seus capítulos dependem dos
critérios estabelecidos no ``\emph{Pequeno tratado sobre as
repúblicas}'' e este, por sua vez, formula seus fundamentos teóricos
tendo em vista a compreensão da vida em qualquer república, inclusive a
de seu tempo\footnote{As reuniões nos \emph{Orti Oricellai,}
  patrocinadas por jovens aristocratas de ideais republicanos, visavam
  também encontrar meios para restaurar o governo republicano em
  Florença (\versal{GILBERT}, 1977, 15-66).}, e não somente Roma, modelo de
república. Separar o ``\emph{Pequeno tratado}'' dos \emph{Discursos}
violaria, na perspectiva maquiaveliana, a \emph{verità effetuale}, a
comprovação no mundo real, recaindo num argumento feito para
``repúblicas ou cidade imaginadas''.

Segundo Chabod, os \emph{Discursos} constituem a ``origem espiritual''
de \emph{O Príncipe,} uma vez que pode-se perceber a existência
de uma relação direta entre os problemas que subjazem a ambos\footnote{Essa
  hipótese é sugerida primeiramente por Chabod, porém ele não a
  desenvolve. Gennaro Sasso, anos depois, será o primeiro a
  desenvolvê-la, tirando novas conclusões, como se verá nos capítulos
  finais dessa introdução (\versal{CHABOD}, 1993, p. 31-39).}. Ele propõe, não a
partir dos dados históricos sobre os textos, mas a partir da articulação
dos conceitos, que os \emph{Discursos}, ou parte deles, seriam os
pressupostos teóricos para \emph{O Príncipe}. Embora não tenha
desenvolvido essa hipótese, Chabod apontava para uma motivação de ordem
republicana mobilizando os argumentos de \emph{O Príncipe}, de
tal modo que esse deveria ser uma resposta ou continuação de algo
deixado para traz nos \emph{Discursos.}

Avançando nessa interessante sugestão e tendo em vista que há uma
exposição sobre as repúblicas no início dos \emph{Discursos}, nasce a
dificuldade de como se poderia articular essa parte inicial com \emph{O
Príncipe}. Uma hipótese seria verificar a sequência argumentativa quando
a república chega ao seu ápice de corrupção política, conforme exposto
no capítulo \versal{XVIII} do livro \versal{I} dos \emph{Discursos}.

Segundo Gennaro Sasso (1980, cap. \versal{V}), tendo Maquiavel escrito os dezoito
primeiros capítulos dos \emph{Discursos} e chegando ao ponto em que as
repúblicas estão completamente dominadas pela corrupção, em que a ruína
é um fato quase inevitável, a instauração de um principado civil passa a
ser o remédio adequado. Dito de outro modo, quando se verifica, no
capítulo \versal{XVIII} dos \emph{Discursos}, que é quase impossível a uma
república ``\emph{corrompidíssima}'' retomar o \emph{vivere civile}, as
liberdades civis, características dos regimes republicanos sadios, a
solução passa a ser a instauração de um regime fundado em um único
governante para que esse, com sua \emph{virtù}, consiga recuperar a
normalidade política da cidade e impedir a ruína certa. Ademais, ``o
principado representa o remédio que, auxiliado por extraordinária
\emph{virtù}, os legisladores que vêem longe procuram opor à corrupção
das repúblicas'' (\versal{SASSO}, 1980, cap. \versal{V}). Calcado naquilo
que é exposto pelos textos maquiavelianos, há a ``\emph{problemática
passagem}'', como diz Sasso, das repúblicas corrompidas para o regime
régio caracterizado pelo principado civil, na medida em que esse regime
pode oferecer uma resposta eficaz ao problema que se instaura nas
cidades corrompidas. O remédio é sugerido no próprio capítulo \versal{XVIII} dos
\emph{Discursos}, quando Maquiavel afirma que o freio para essa
corrupção total é a instauração de um governo régio, um regime que, com
a sua mão ``régia'', intervenha para reordenar a cidade. Ao final do
capítulo \versal{XVIII}, Maquiavel apresenta a ideia que será dominante no
\emph{Príncipe}, de tal modo que, da perspectiva de quem olha dos
\emph{Discursos}, a boa solução ou o remédio adequado não está nos seus
capítulos seguintes (\versal{XIX}, \versal{XX} etc.), mas no principado civil, tal qual é
apresentado nos capítulos \versal{VIII} e \versal{IX} de \emph{O Príncipe}. Do que podemos
concluir, conforme Sasso: ``os principados pressupõem a crise da
república, e não nascem senão quando essa está tomada pelas formas
extremas da corrupção, da degeneração'' (\versal{SASSO}, 1980, cap. \versal{V}). Com isso,
a origem de \emph{O Príncipe} não se fundaria numa visão
``idealizada''\footnote{A crítica à origem mítica ou idealizada de
  \emph{O Príncipe} é um dos objetivos de Sasso nessa reflexão,
  pois, para ele, carece de fundamento pensar a motivação de um livro
  apenas em pressupostos ideais. (\versal{SASSO}, 1980, p. 316, nota 41).} dos
regimes políticos, mas encontra sua motivação teórica no limite extremo
que se configura com a corrupção das repúblicas. Em uma cidade onde as
\emph{ordini} e as leis estão dominadas pela corrupção, a intervenção do
príncipe novo, tema dominante de todo \emph{O Príncipe}, faz-se
necessária, reformulando, ou melhor, refundando as ordens e as
instituições, reconciliando os humores, enfim, tudo aquilo que também é
preconizado ao longo dos \emph{Discursos}. Desse ângulo, \emph{O
Príncipe} seria tributário do raciocínio desenvolvido no
``\emph{Pequeno tratado sobre as repúblicas}'', pois teria nesse sua
maior motivação teórica. Por outro lado, os \emph{Discursos} manteriam
uma relação de dependência teórica com \emph{O Príncipe}, visto
que a melhor solução para o problema no qual culmina o raciocínio seria
o principado civil. Essa interdependência teórica revela uma estreita
linha de continuidade no interior da reflexão política maquiaveliana,
como afirma o comentador:

\begin{quote}
Precedendo cronologicamente ou seguindo a composição do \emph{Príncipe},
o décimo oitavo capítulo do primeiro livro dos \emph{Discursos} é,
portanto, o ``lugar ideal'' no qual o conceito daquele livro se realiza
nos seus modos próprios (\versal{SASSO}, 1980, p. 327).
\end{quote}

Portanto, cabe agora tentar entender melhor essa passagem da república
corrompida ao principado, ou por outro lado, o que é esse principado e o
seu príncipe, a partir desse pressuposto lançado pela corrupção
republicana.

\section{O Príncipe Civil}

\subsection{Os pressupostos para o poder régio}

Como dito, a sequência argumentativa do \emph{Pequeno tratado sobre as
repúblicas} encaminha-se, a partir do capítulo 16, para o tema da
corrupção na cidade. Primeiro, Maquiavel trata da corrupção do povo ou
da matéria da cidade (cap. \versal{XVI}) e, em seguida, da corrupção dos
ordenamentos republicanos ou da forma (cap. \versal{XVII}). Prosseguindo nessa
escalada da corrupção política, o cap. \versal{XVIII} vai direto ao grau máximo
de corrupção na cidade, quando ela se torna \emph{corrompidíssima}
(sic). Neste estágio de corrupção ampla, as consequências são ou a
mudança de regime ou a dissolução da república como entidade política
dotada de liberdade. Seja como for, qualquer uma das consequências é
contrária à vida política republicana, ao \emph{vivere libero,} visto
que uma condição essencial da vida republicana é a liberdade, ao ponto
do regime republicano ser nomeado, às vezes, como o regime da liberdade.
Na verdade, chegamos ao grande problema enunciado pelo título do
capítulo \versal{XVIII}: ``\emph{De que modo nas cidades corrompidas se podem
conservar um Estado livre, sendo-o; ou, não o sendo, ordená-lo}''
(\emph{Discursos}, \versal{I}, 18, linha 1). A questão está em tentar pensar em
uma solução para aqueles casos nos quais a corrupção não está apenas
localizada numa parte do corpo político ou permanece restrita à matéria
ou à forma, mas quando se encontra disseminada por toda a cidade. Uma
resposta já nos é possível constatar, pois não se pode conservar o
\emph{vivere libero} em condições de extrema corrupção, nas quais o povo
já não mantém a civilidade, e em que as leis são inadequadas e os
ordenamentos não conseguem mais frear as ambições desmedidas dos
diversos grupos políticos. As condições de possibilidades para a
retomada da liberdade republicana já não figuram mais no horizonte.
Diante, então, dessa condição extrema, a possibilidade de retorno, de
uma retomada à normalidade republicana é uma impossibilidade dentro da
lógica de ação política da república, pois, com uma matéria corrompida,
as leis são inadequadas e os ordenamentos políticos ineficazes e,
conforme o grau de corrupção, corrompidos em suas deliberações.
Maquiavel é categórico: De tudo o que dissemos acima provém a
dificuldade ou a impossibilidade de nas cidades corrompidas conservá-las
como republicas ou criá-las de novo (\emph{Discursos}, \versal{I}, 18, linha 28).

Neste contexto, pode-se até perguntar se ainda há ou não liberdade, ou
melhor, se o \emph{vivere libero}, característico da república, ainda
persiste ou se alguma força autoritária teria tomado as rédeas das
decisões políticas. Uma das características dessa corrupção republicana,
talvez a preponderante, está na força política que a aristocracia assume
e como ela passa a deliberar conforme os seus desejos. Pensando numa
cidade em tais condições políticas, mas não somente isso, sendo o povo
impedido de lutar pelos seus direitos, tal quadro é uma descrição de um
caso de corrupção republicana típico. Nestas circunstâncias, extingue-se
a liberdade de uma parte do corpo político, extingue-se a luta política
e um só grupo passa a ditar o caminho. No entanto, a corrupção também
pode extrapolar um grupo político restrito e atingir a todos
(\emph{universale}), circunstância esta caracterizada, entre outros
aspectos, pela perda dos valores cívicos, da civilidade. Também neste
caso não há mais espaço para a luta política, para o \emph{vivere
libero}.

Logo, não importa em que condição se manifeste a corrupção, ela figura
sempre como uma oposição à liberdade, ou como diz Sasso, ``\emph{a
recíproca repugnância entre liberdade e corrupção}'' (\versal{SASSO}, 1987, p.
407). Esta imagem ilustra os termos da dificuldade, pois a vida política
republicana é avessa à corrupção, é o pólo contrário à condição política
corrompida de uma cidade. De fato, se há uma manifestação de corrupção
política, isso implica proporcionalmente na anulação da liberdade; ou,
conforme a corrupção se amplia, por uma proporção inversa, diminui o
grau de liberdade da cidade. O que não quer dizer que a corrupção seja o
antônimo de liberdade, pois, conforme o nível de corrupção, tem-se uma
gradação inversa de liberdade: quando o grau de corrupção da cidade é
baixo, é possível que exista ainda o \emph{vivere libero}. Porém, em
qualquer condição em que haja um aumento de um, automaticamente ocorre o
decréscimo do outro, pois a coexistência de ambos com mesma intensidade
é impossível. Repugnância que não diz respeito apenas à liberdade, mas
pode estender-se à civilidade (entendida como o respeito mínimo às
regras políticas), às regras cívicas, quando se considera a corrupção da
matéria. Ou como dirá Maquiavel no capítulo \versal{LV} desse livro \versal{I} dos
\emph{Discursos}, refletindo acerca da corrupção presente quando os
\emph{gentis-homens} dominam o poder: ``Do que nasce que naquelas
províncias não surja nunca alguma república nem algum \emph{vivere
politico}; porque tal geração de homens são em tudo inimigos de toda
civilidade'' (\emph{Discursos}, \versal{I}, 55, linha 21). A corrupção se
opõe, pois, à república, mas, mais ainda, à civilidade e até mesmo ao
\emph{vivere politico}, de modo geral. Esta afirmação amplia o problema,
pois a corrupção não é somente contrária à liberdade, mas contrária à
vida política, um adversário à normalidade política. Opondo-se a uma
consideração que banaliza o papel que pode chegar a desempenhar a
corrupção no corpo político, Maquiavel confere cores fortes e afirmações
contundentes para descrever a importância das suas consequências para a
vida política da cidade. A manifestação da corrupção não deve ser
tratada como mais um evento possível em uma cidade, mas um grande
problema, um grande perigo para o corpo político como um todo. A
corrupção não é mais uma dificuldade presente no cotidiano político das
repúblicas, mas se torna o problema, a questão a ser tratada.

Todavia, quando esse problema não é passível de solução pelos próprios
mecanismos políticos da república, então, deve-se buscar meios mais
fortes e eficazes para freá-la. Entre as soluções, está a instalação de
um \emph{poder quase régio} ou do \emph{poder régio}.

A cidade diante, pois, de um caso de extrema corrupção, deve mudar o seu
regime, tendo, a princípio, duas possibilidades: o governo \emph{régio}
ou o governo \emph{popular}. Maquiavel reiteradamente identifica na
ambição desmedida da aristocracia a principal causa de corrupção. A
corrupção do povo, quando nasce, é um aspecto secundário, sendo muito
mais fruto da falta de freios à insolência dos grandes do que da perda
de civilidade do povo. Ora, a parcela popular da cidade estaria
habilitada, pelas suas qualidades, para, em tese, assumir o comando da
cidade nas condições de corrupção extrema, desde que não tivesse perdido
também todos os seus valores cívicos. Contudo, o problema não é assumir
o controle da cidade em função da sua capacidade ou por não estar tão
corrompida, a questão que se põe é se esse governo popular seria capaz
de colocar um fim à corrupção endêmica e reordenar a cidade.

Em função da grande insolência que, em geral, assola a cidade
corrompidíssima, a solução dada por Maquiavel não é nem sua conversão
num governo popular e nem num monárquico, mas em algo intermediário: no
poder quase régio. Diz ele:

\begin{quote}
Mas, em se precisando criar ou conservar uma {[}república{]}, seria
necessário, antes, reduzi-la ao estado régio do que ao estado popular;
para que os homens insolentes, que não pudessem ser corrigidos pelas
leis, fossem de algum modo freados pela autoridade quase régia
(\emph{Discursos}, \versal{I}, 18, linha 29).
\end{quote}

A solução pelo governo \emph{quase régio} é, na verdade, a justificação
para a instalação de um ordenamento republicano de Roma: a ditadura. Na
república romana, a figura de um ditador, que concentrava poderes
extraordinários durante um período limitado de tempo, era uma solução
prevista para casos especiais, como guerras e revoltas civis. O ditador
romano era um magistrado especial, escolhido pelo senado com função
específica para realizar alguma missão extraordinária. Com a instalação
do ditador pelo Senado romano, cessariam automaticamente os poderes dos
cônsules e dos outros magistrados, que passavam a subordinar-se ao
ditador (\versal{NICOLET}, 1964; \versal{CIZEK}, 1990). Este ditador romano difere em
muito, contudo, da imagem dos ditadores contemporâneos, pois sua
instalação e sua ação eram reguladas e submetidas à fiscalização e ao
controle do Senado romano, ou seja, ele não teria poderes políticos e
jurídicos absolutos.

Ora, quando Maquiavel pensa num governo \emph{quase régio}, dotado de
poderes extraordinários, ele tem em vista tanto a instalação da ditadura
quanto de um principado nos moldes romanos. No capítulo \versal{XXXIV} do livro
\versal{I}, ele diz:

\begin{quote}
Alguns escritores condenaram os romanos que encontraram um modo de
instituir a ditadura, como algo que, com o tempo, deu ensejo à tirania
em Roma. {[}\ldots{}{]} E vê-se que o ditador, enquanto foi designado segundo
os ordenamentos públicos, e não por autoridade própria, sempre fez bem à
cidade. Pois o que prejudica as repúblicas é fazer magistrados e dar
autoridade por vias extraordinárias, e não a autoridade que se dá por
vias ordinárias: e vê-se que em Roma, durante tanto tempo, nunca ditador
algum fez nada que não fosse o bem à república (\emph{Discursos}, \versal{I},
34, linhas 2; 5-6).
\end{quote}

O problema que pode advir a esses governos com poderes extraordinários
está no modo como nascem. Caso sua autoridade tenha sido delegada por
via ordinária, ou seja, dentro das regras políticas da república, sem
uma exacerbação de força por meio da violência, então não há nenhum
problema maior e os efeitos serão bons. A preocupação de Maquiavel
reside, fundamentalmente, no modo como ocorre a instalação desse
governo, no caso, por um meio não violento, respeitando a dinâmica
política republicana. Por se originar em tal quadro, o ditador detinha
um poder extraordinário, porém limitado, o que era uma garantia de, ao
final de seu mandato, o retorno à normalidade republicana:

\begin{quote}
De modo que, somando-se o breve tempo de sua ditadura, a autoridade
limitada que ele tinha e o fato de o povo romano não ser corrompido, era
impossível que ele saísse de seus limites e prejudicasse a cidade: e
pela experiência se vê que sempre foi proveitoso (\emph{Discursos}, \versal{I},
34, linha 10).
\end{quote}

Uma primeira solução para a república corrompida é a utilização de um
mecanismo republicano, o ditador, que concentra o poder para que possa
dar conta de um problema extraordinário, que, pelas vias ordinárias
republicanas, não poderiam ser sanado. Como sugere Bausi, além desse
ditador ao estilo romano, Maquiavel também tinha em mente como exemplo
desse poder quase régio os \emph{gonfalonieri} florentinos, que foram
governantes com poderes centralizados, mas em repúblicas\footnote{De
  fato, é possível fazer várias aproximações entre as funções e encargos
  dos ditadores romanos e as atribuições iniciais do \emph{gonfaloniere}
  Solderini, em 1494. Contudo, depois da reforma política de 1502 que
  institui o \emph{gonfaloniere a vita}, ou seja, perpétuo, convém
  associá-lo mais ao \emph{princeps rei publicae} do que ao ditador
  romano (\versal{BAUSI}, 2002, p. 117, nota 39).}. Esta solução é sugerida em
outras passagens\footnote{Cf, Livro \versal{I}, \versal{II}, 33; \versal{XXXIV}, 20; Livro \versal{III},
  \versal{XXVIII}, 14}, sinalizando um momento intermediário que, uma vez
fracassado, não deixaria escolha senão a instalação de um regime com um
governante com poderes políticos absolutos, uma autocracia de fato. A
vantagem dessa solução intermediária é que ela garante uma exigência
fundamental para a república corrompida, já que instala um governo de
força sob a égide do regime republicano. Sem abolir totalmente os
valores cívicos do republicanismo, o ditador ou o \emph{gonfaloniere},
por seu caráter extraordinário e temporário, visto que tinha mandatos
definidos que poderiam ou não ser renovados, seria um governo forte em
regimes republicanos enfraquecidos pela corrupção com vistas
exclusivamente à reordenação da cidade, o que por si só é um risco, haja
vista que não se tem a certeza de que eles serão bem sucedidos. De
qualquer modo, a condição extraordinária da corrupção -- pois ela é, no
limite, ruptura da vida política ordinária -- exige uma solução também
extraordinária, que ultrapasse alguns aspectos da normalidade
republicana, a fim de que se restaure a ordem. Os ditadores ou os
\emph{gonfalonieri} são medidas extraordinárias para circunstâncias
políticas extraordinárias. Como diz Maquiavel:

\begin{quote}
Quanto a inovar tais ordenamentos de uma só vez, quando todos reconhecem
que não são boas, digo que essa inutilidade, quando facilmente
reconhecível, é difícil corrigi-la; porque, para tanto, não basta usar
medidas ordinárias, visto que os modos ordinários são ruins; mas é
necessário recorrer ao extraordinário, como a violência e as armas,
tornando-se, antes de mais nada, príncipe em tal cidade, para poder
dispô-la a seu modo (\emph{Discursos}, \versal{I}, 18, linha 26).
\end{quote}

Ou ainda, como diz ao final do capítulo \versal{XVII}:

\begin{quote}
Porque tal corrupção e pouca aptidão à vida livre nascem de uma
desigualdade existente na cidade, e quem quiser dar-lhe igualdade
precisará lançar mão de meios extraordinários {[}grandissimi
straordinari{]}, o que poucos sabem ou desejam fazer (\emph{Discursos},
\versal{I}, 17, linha 16).
\end{quote}

Portanto, mesmo tendo à disposição esse meio extraordinário de reforma,
legítimo e previsto dentro do regime republicano, tal solução, apesar de
possível, não parece ser, contudo, a mais adequada para a cidade
corrompidíssima. Uma outra hipótese é a instalação de um governo que,
apesar de centralizar a autoridade em um indivíduo, consiga conservar um
mínimo de civilidade ou até mesmo recuperar a dinâmica republicana.
Governo esse que pode ser compreendido como um certo tipo de principado,
e não todo e qualquer principado, no caso específico, o principado
civil.

\subsection{Sobre a noção de principado em Maquiavel}

A crise de corrupção das repúblicas nos leva, pois, aos governos de
força, sejam eles régios ou quase régios. Uma das sugestões apontadas
por Maquiavel é o principado e, seguindo esse viés interpretativo,
cumpre pensar o principado a partir desses pressupostos dados pela
corrupção republicana. Esse é um ponto nodal de nossa leitura: como
pensar o principado e o príncipe expostos por Maquiavel em \emph{O
Príncipe} a partir desses pressupostos fornecidos pela corrupção
republicana?

Tendo em vista esse pressuposto interpretativo, algumas questões
formuladas de início tornam à baila: quem é esse príncipe -- que na
definição do texto é antes de tudo um \emph{privato ciptadino} -- que
assume o papel de liderar e conduzir a cidade? Seria ele um típico
monarca ou uma figura política diferente? Se o príncipe não é o monarca,
então, como entender o principado? Seria este o território ou lugar da
ação política do príncipe, identificando-se aquilo que nós entendemos em
nossos dias como principado ou reino? Ou seria ele outra coisa?

Antes mesmo de partir para a busca da resposta sobre o personagem
político que é o príncipe, faz-se necessário indagar antes o que
Maquiavel entende por principado e como essa noção é mobilizada na
reflexão desenvolvida em \emph{O Príncipe}. Convém relembrar,
inicialmente, que o título original da obra era \emph{De Principatibus},
ou seja, \emph{Sobre os Principados}, restando claro que seu autor
pretendia dissertar sobre esse tema ao longo do livro e não sobre a
figura do príncipe em primeiro plano, como o título atual sugere. Em
vista disso tudo, faremos uma explanação em duas etapas: uma primeira
sobre as noções de monarquia e principado que chegam ao contexto do
Renascimento florentino, no qual esse uso vocabular está inserido, ou
seja, uma apresentação do problema em seu contexto discursivo e as
nuances terminológicas que alguns termos possuíam naquele momento. Uma
segunda investigação, buscando extrair do próprio texto maquiaveliano o
que se compreende por principado, a partir dessa noção de príncipe civil
que nos parece ser central.

\subsection{O principado no momento maquiaveliano}

A denominação dos cargos dirigentes e dos detentores desses na Europa do
século \versal{XV} deve ser considerada, inicialmente, nos contextos políticos
particulares de cada território. Como o próprio Maquiavel aplicará ao
longo d'\emph{O Príncipe}, cada lugar possuía suas peculiaridades
em termos de organização política, de modo que unificar ou tentar
generalizar terminologias e designações pode implicar em equívocos
sérios. Por exemplo, nos territórios que hoje conhecemos como a
Alemanha, havia vários príncipes e um imperador, sendo que esses
príncipes não eram filhos desse imperador ou seus sucessores diretos. O
mesmo se diga para o líder político do ainda recente, para os latinos,
governo turco, designado por Maquiavel tão somente como ``O Turco''. Em
geral, assumia-se que o imperador era o governante de um império que
possui vários reinos ou principados. Mas mesmo neste caso, a denominação
é controversa, principalmente entre os regimes que reivindicavam a
herança do Império Romano, pois, tendo em vista a presença até 1453 do
Império Romano do Oriente, também conhecido como o Império bizantino,
todos os demais postulantes latinos à condição de Império disputavam com
Constantinopla esse reconhecimento, incluindo, nesse caso, até o Papado,
que após o século \versal{VIII}, em vários documentos, reivindicava a sua
condição de legítimo herdeiro do Império Romano cristianizado por
Constantino em 313, particularmente pelo documento falsificado
denominado ``Doação de Constantino''. Situação que se complica mais após
o século \versal{VIII}, com o surgimento do Sacro Império Romano, da dinastia
carolíngia, que se transforma posteriormente no Sacro Império Romano
Germânico, dominado pelos imperadores alemães. Historicamente, os
governantes dos impérios oriental e ocidental nunca reconheceram de fato
essas condições por inúmeros motivos, que não vem ao caso dissertar
aqui\footnote{Cf. Ostrogorsk, Georg. \emph{Storia dell'impero
  bizantino.} Torino: Einaudi, 1968. Esse longo estudo sobre o Estado
  bizantino mostra, a partir da ótica dos orientais, como as
  reivindicações dos latinos nunca foram plenamente aceitas por eles,
  donde a disputa constante pela herança do Império Romano durou
  séculos.}. Quadro esse que se amplifica com os novos reinos latinos de
Espanha e Portugal no século \versal{XV}, que se reivindicam também como
impérios. No início do século \versal{XVI}, portanto, um império já não era mais
entendido apenas como um sucessor direto do Império Romano ou como poder
político que se coloca acima dos reinos que governa, passando a ter uma
acepção polissêmica. A clássica designação de que o imperador está acima
do rei já não cabe para certos casos e não possui uma significação
precisa em alguns contextos políticos.

Tal problema de denominação pode ser extrapolado para os termos ``rei''
e ``príncipe'', que passam a ser ter várias acepções, embora, em geral,
verifique-se que o termo ``rei'' se aplica ao governante do reino. Tudo
isso sem levar em conta o sentido jurídico dessas denominações, o que
nos remeteria, por seu turno, aos debates entre os glossadores
medievais.

Tomando-se em conta o contexto histórico italiano do Renascimento,
particularmente, as lutas por autonomia das cidades do norte da
península itálica e a consequente implantação dos regimes republicanos
em várias delas, o uso desses vocábulos políticos ganha novos contornos.
Tendo em vista as guerras travadas pela busca de autonomia dessas
cidades contra as forças imperiais e papais, e depois as próprias lutas
internas contra aqueles que buscavam o domínio da cidade, contra os
\emph{Signori}, a denominação dos cargos políticos apresenta uma
variedade significativa. Ora, no contexto das repúblicas italianas do
Renascimento, o fato histórico da luta pela liberdade contra as tiranias
-- interna ou externa -- fez com que os usos do termo ``príncipe'' sejam
esvaziados dessa ligação ao regime monárquico, quando se refere a
personagens políticos deste contexto.

Isso fica claro, no caso de Florença, quando se analisam os escritos dos
autores políticos, particularmente dos homens da Chancelaria, como
Salutati, Bruni, Valla, Maquiavel, Guicciardini, Vettori, Buonaccorsi
etc. A designação ``príncipe'', aplicada a personagens italianos,
raramente se refere a pessoas ligadas a uma dinastia e trata-se, em
geral, de próceres políticos com cargos executivos. De tal modo que o
termo se aproxima mais da concepção de \emph{prínceps} latino, como o
primeiro entre os iguais, do que o herdeiro de uma dinastia.

Se para o termo ``príncipe'' é possível delimitar esse uso entre os
escritores florentinos, já para o termo ``principado'' parece que temos
ainda uma acepção muito mais próxima de monarquia, do que decorre a sua
definição como uma espécie de regime monárquico. De fato, o termo é
polissêmico e permite essa interpretação, que se reforça quando se leva
em conta os governos da família Medici ao longo do século \versal{XV},
caracterizada como um principado e que, para muitos, se tratava de fato
e de direito de um governo de tipo monárquico na cidade. O que nos
obriga, ao menos, em considerar melhor o que foi esse governo para
entender os possíveis sentidos dessa denominação de principado na
Florença do contexto maquiaveliano.

Como demonstrou Rubinstein (1997) e o próprio Maquiavel na
\emph{História de Florença}, após a revolta popular, também conhecida
como Revolta dos Ciompi, desencadeada em 20 de julho de 1378, uma série
de mudanças no ordenamento político da república florentina são
realizadas. A criação de novos conselhos e o modo como os grupos
poderiam ter acesso a eles foram uma das demandas que se transformaram
em possibilidade política legítima. Contudo, já em início do século, mas
principalmente a partir do governo de Cósimo de Medici, em 1434, o que
se verifica é um aumento da influência política dos setores
aristocráticos mais abastados sobre os demais grupos políticos\footnote{Conforme
  já explicamos no início, a simplificada distinção social em Florença
  entre ricos e pobre (\emph{popolo grasso} e \emph{popolo minuto}) não
  é correta, pois, segundo o próprio Maquiavel, haviam sub- divisões
  entre eles, tornando o equilíbrio de forças políticas mais complicado.}.
O governo sobre o controle da família Medici durante o século \versal{XV} era
formalmente um governo republicano, contudo, era dirigido conforme os
interesses dessa família, que não apenas possuía o cargo de comando, mas
também controlava o acesso aos demais conselhos. Tanto é assim que,
quando Piero di Medici perde o governo da cidade em dezembro de 1494, a
razão maior para isso está no fato dele ter rompido o delicado
equilíbrio de poder que havia entre os diversos setores da aristocracia
(Rubinstein, 2011). E mesmo a volta dessa família ao poder em 1512 tem
como fundamento a reconstrução dos apoios que ela possuía entre a
aristocracia que detinha o controle dos conselhos superiores da cidade.
Ora, em ambos os exemplos históricos, não é possível afirmar que se
tratava de um governo de tipo monárquico, muito menos de um governo
autocrático com outra denominação. De fato, como demonstra Rubinstein
(1997), os Medici dividiam o poder político na cidade e não concentravam
tudo em suas mãos. O fato de exercerem o controle sobre os conselhos não
nos permite dizer que esse fosse um governo autocrático. Nomear, pois,
tais regimes de principados, a partir do exemplo histórico, não pode
implicar uma identificação desses regimes sob o controle da família
Medici como monarquias.

\subsection{O principado de \emph{O Príncipe}}

Para além desses dados históricos, que não resolvem o problema, uma
análise do texto de \emph{O Príncipe} fornece os elementos mais
corroboradores, pois nos apresenta o que de fato Maquiavel entende por
principado.

Logo no início da obra, encontramos uma primeira definição de
principado: \emph{``Todos os estados, todos os domínios que tiveram e
têm autoridade sobre os homens, foram e são ou repúblicas ou
principados.''} {[}cap. 1, linha 1{]}

A definição inicial já nos revela algo, pois o principado é uma
autoridade (império), ou seja, um governo que se exerce sobre os homens.
Em seguida, ele distingue o principado da república, diferenciação esta
que ocorrerá outras vezes, não somente \emph{n'O Príncipe} em outros
momentos, mas nas suas demais obras.

Contudo, se o principado realmente for um domínio sobre os homens,
poder-se-ia declarar que se trata de um governo que exerce sua força
sobre os homens, submetendo-os. Tal constatação é dedutível a partir da
noção de domínio herdada da antiguidade e reelaborada durante o período
medieval, que, grosso modo, é a transferência da relação existente no
interior da casa (\emph{domus}), entre senhor e escravo, para a esfera
pública. Neste caso, o principado é uma dominação política, no sentido
de um governo que não está aberto à interferência e não divide seu
controle ou primazia nas decisões. Entretanto, tais afirmações desse
modo ainda são precipitadas, pois o conceito teve apenas sua primeira
apresentação na obra, falta considerar o restante da primeira parte do
livro.

Na sequência ainda desse capítulo \versal{I}, Maquiavel distingue os principados
em hereditários e novos. Nesse momento, teríamos a primeira aproximação
dos principados com o regime monárquico, pois o principado hereditário
seria aquele no qual o controle está nas mãos de uma mesma família, ou
como diz, ``\emph{nos quais o poder ficou por longo tempo com a família
do príncipe}'' {[}cap. 1, linha 2{]}. Todavia, ainda que essa tradução
seja adequada, uma consideração sobre o texto italiano nos permite
verificar que Maquiavel está falando daqueles governos em que os membros
de uma dinastia foram por longo tempo príncipes (\emph{sia suto lungo
tempo principe}), o que admite também o entendimento de que essa família
liderou a cidade durante este ``longo tempo'', o que é um pouco
diferente do que afirmar que este principado é uma monarquia
hereditária. A distinção estaria na nuance entre liderar politicamente e
ter o domínio político, a autocracia das decisões políticas. Ora, como o
próprio autor não se alongará nessa análise, é temerário tecer hipóteses
sobre um tema não desenvolvido pelo filósofo. Atendo-se à letra do
texto, mesmo nesse principado hereditário, Maquiavel parece destacar a
capacidade de liderança política desta dinastia e não a sua condição
monárquica.

O outro principado em tela, o principado novo, se constituirá, na
verdade, no tema principal do livro, pois, seja nas considerações sobre
o principado, seja nas considerações sobre o príncipe, Maquiavel volta
suas atenções para o principado novo, em suas várias manifestações, e
para o príncipe novo e sua necessidade de conservação do governo.

Neste primeiro momento do livro, sua preocupação está dirigida para a
conquista do principado. O termo ``conquista'' aqui pode ter uma dupla
acepção, em função daquilo que se entende como principado: se principado
for compreendido como um território, então essa conquista é tal qual uma
ocupação fruto de uma campanha militar, por exemplo; nesse sentido, uma
conquista de um local ou território. Mas, haveria ainda a possibilidade
da conquista se referir somente ao controle político do governo, à
esfera política da cidade prioritariamente. Neste caso, trata-se de
ressaltar a dimensão política da cidade, e não seu aspecto territorial,
talvez mais próxima da definição antiga, encontrada, por exemplo, em
Aristóteles (\emph{Política}, \versal{III}, cap. 1-2), no qual a definição de
cidade (\emph{polis}) não se identifica a um território, de tal modo que
a cidadania não é assegurada a quem habita aquele território. O exemplo
do caso da Alemanha citado no capítulo 10 d'\emph{O Príncipe} pode ser
bem ilustrativo disso, pois o rei tem sua esfera de ação política, mas
não detém o controle dos territórios governados pelos príncipes. Assim,
um caso de controle político não implica necessariamente, embora muitas
vezes seja o caso, controle de um território, fornecendo o escopo dessa
dificuldade de compreensão da conquista do principado.

Porém, uma nova ordem de problemas se articula a esses, na verdade, um
problema anterior e mais profundo, que diz respeito ao que entender por
conquista. Talvez aqui esteja o cerne político do problema, pois a
dificuldade está em determinar qual a relação política que esta tomada
do governo estabelece: se é uma relação política na qual o conquistador
domina toda a esfera de comando, exercendo seu poder com domínio
soberano, ou se esta relação implica em um postar-se constantemente na
disputa pelos apoios políticos que permitem o exercício do governo. O
desenvolvimento do texto ater-se-á justamente a esse ponto nevrálgico da
tomada do governo, mais do que ficar listando tipificações dos modos de
se adquirir um principado. É na determinação das relações políticas que
tal investida causa que Maquiavel fixa suas atenções. Logo, não se
atinge os elementos centrais do texto quem se concentra somente na
análise dos tipos de conquista, conforme dá a entender o texto num
primeiro momento, e se esquece de atentar para o modo como as relações
políticas estão sendo construídas, relações estas que formarão os
alicerces desse novo governo.

Desviando um pouco a atenção para a exposição do argumento, é importante
que logo de início o leitor perceba uma característica do estilo de
exposição maquiaveliano. Por ser Maquiavel um escritor hábil, sua função
principal durante todo o período de trabalho na Chancelaria foi escrever
relatos, habilidade esta que ele teve que aprimorar, pois não somente a
clareza deveria ser uma marca dos seus textos, mas, por outro lado,
quando se fazia necessário, convinha produzir um relato que dirigisse os
seus leitores a tomar a decisão que ele entendia a mais adequada. Donde
a necessidade de se usar recursos retóricos, mas com tal sutileza que o
leitor não percebesse essa sub-intenção. Ora, um profissional exercitado
por longos anos nesta arte da escrita e que era conhecido por ser também
teatrólogo -- não nos esqueçamos que em vida Maquiavel foi reconhecido
em Florença mais como o autor da peça \emph{Mandragora} do que como
personagem político --, elabora um texto que deve ser lido com atenção e
levando em conta as diversas dimensões da linguagem e seus efeitos.
(\versal{ADVERSE}, 2009)

Retomando o roteiro argumentativo inicial, o livro se abre, então, com
uma rápida definição de principado, e não de príncipe, o que já é muito
significativo, e passa para sua tipologia de conquistas: pelas armas,
próprias e alheias, pela \emph{virtù} e pela fortuna. Encerrado essa
breve apresentação, que é, na verdade, um roteiro dos temas a serem
tratados ao longo da primeira parte do livro, o parágrafo inicial do
capítulo \versal{II} é também emblemático.

A frase de abertura do capítulo \versal{II} já provocou inúmeras discussões entre
os comentadores, pois diz Maquiavel: ``\emph{Deixarei de lado a
discussão sobre as repúblicas, porque, em outro momento dissertei
longamente sobre elas}'' (cap. 2, linha 1). Não vamos retomar aqui a
discussão relativa a qual livro ele estava se referindo e se esse livro,
em geral entendido como sendo os \emph{Discursos}, foi ou não escrito
antes d'\emph{O Príncipe}, conforme já abordamos. Apenas gostaríamos de
chamar a atenção para o fato de que num livro dedicado aos principados,
em dois momentos iniciais, mais exatamente num espaço de cinco linhas,
Maquiavel faz duas referências diretas à república. Uma primeira
constatação é óbvia, o principado se coloca, já em seu primeiro momento,
como um contraponto à república, o que é inegável. Com efeito,
parece-nos evidente que o principado é um regime diferente da república.
A questão é: por que essa insistência em contrapô-los? Talvez ainda o
texto não nos permita apontar uma resposta suficiente para essa
dificuldade, todavia, a segunda linha do capítulo fala do ``\emph{sangue
do senhor que é por longo tempo príncipe}''. Ora, tal ideia poderia ter
uma outra forma de redação se se tratasse tão somente de monarquias.
Neste caso, bastava dizer que aquela dinastia ou aquela família real
detém o comando político há muito tempo, conforme já tratamos.
Voltando-se para mobilização de república aqui, verifica-se, por apenas
esses elementos, que Maquiavel percebe um paralelo entre principado e
república que exige uma distinção. A diferenciação será apresentada
adiante, porém fica a dúvida acerca do porquê do paralelo. Esse paralelo
ou proximidade entre a república e o principado ficará mais claro quando
da análise do principado civil, mas, pelo exposto, podemos antecipar com
segurança que esse o principado terá na sua estruturação política e na
dinâmica das suas ações algumas semelhanças com as repúblicas,
particularmente no que diz respeito ao lugar dos conflitos políticos, e
se diferenciará, por outro lado, das monarquias ou governos
autocráticos.

Na sequência, Maquiavel apresenta de fato a sua preocupação maior, que
perpassará toda a obra: ``Ocupar-me-ei somente do principado, retecerei
as urdiduras acima descritas, e demonstrarei como estes principados
podem ser governados e conservados'' (cap. 2, linha 2). Na verdade, a
preocupação são duas: o modo de governar os principados e como eles são
conservados, mas que em vários momentos se confundem, pois, no modo de
governar, já devem estar implícitas as estratégias de conservação desse
próprio governo. Agora temos, então, um panorama geral do que será essa
obra: uma análise da conquista dos principados, de como governá-los e
conservá- los. Basicamente sobre esse tripé é que se desdobrarão os
demais temas.

A exposição propriamente dita dos principados começa pelas atenções ao
principado hereditário e aos mistos. Sobre os hereditários, como já
dito, a brevidade da exposição é o dado mais chamativo. Nas poucas
quatro linhas dedicadas ao tema, ele deixa como regra geral que esse
príncipe herdeiro não precisa usar de meios extraordinários para
conservar o seu governo, visto que não teve que conquistá-lo. Ora, basta
a esse que herdou o governo do principado ter uma ``indústria
ordinária'', ou seja, não realizar grandes inovações políticas, seguir o
curso ordinário das coisas. Isso pode soar como um convite a uma conduta
medíocre, como aquele que pauta suas ações pela mediana geral dos
governantes, sem grandes iniciativas políticas. Se for isso, nada mais
disforme ao que será exposto, pois, se é possível dizer algo do
príncipe, é que ele deve ultrapassar o plano do ordinário, da mediania
em termos de ações políticas. Talvez seja então até por isso, por essa
mediocridade inerente ao príncipe herdeiro, que se justifique o fato de
Maquiavel não dedicar maior atenção a ele.

O passo seguinte é uma análise dos principados que são conquistados por
alguém que já detém o comando de um outro principado. Esse novo governo
é, para o conquistador, um principado. Detalhe sutil, mas que indica
muito, pois alguém que já comanda um principado e conquista outro, em
tese, não terá grandes dificuldades na direção deste novo governo, pois
já conhece os modos como governar uma cidade e, portanto, se conservam
esses governos. Entretanto, e novamente vemos o estilo maquiaveliano
desconcertar o leitor, não parece ser tão fácil assim para esse
conquistador conservar o governo dessa sua conquista política. A
dificuldade principal desse novo principado indica o grande problema de
toda a conquista política, até mesmo para alguém que já governa, a
saber: como obter apoios na cidade ocupada?

O problema apresentado no início do capítulo \versal{III} revela a preocupação
política que subjaz a conquista: o governo principesco não se vale por
si só, mas pela capacidade de angariar apoios que o sustentem, aquilo
que contemporaneamente chamamos de legitimidade. Isso fica claro quando
Maquiavel diz que, ``\emph{mesmo que se tenha um fortíssimo exército
seu, sempre se precisa da ajuda dos provincianos para entrar em uma
província}'' (cap. 3, linha 3). Portanto, esse conquistador nunca é tão
poderoso, mesmo quando se vale de um forte exército. Como ele
demonstrará por diversos exemplos, esse conquistador precisa fazer uma
série de ações no intuito de trazer para si apoios políticos que lhe
permitam, de fato, se constituir como um líder político daquela cidade,
o que não se consegue apenas tendo um exército forte, apenas pela força
das armas. Pode-se afirmar, olhado por outro lado, que a força do
governo não está no poderio militar ou nas armas somente, mas se erguerá
também, e principalmente, sobre as alianças e os vínculos políticos que
se consiga estabelecer nesse novo governo.

A busca de sustentação política será, assim, a tônica das preocupações
daquele que ascende à condição de príncipe. Fortuna, \emph{virtù}, armas
próprias e armas alheias, são todos elementos da conquista que remetem
sempre, cada um a sua maneira, ao modo como, num segundo momento, esse
conquistador político recebe o apoio político necessário para a
sustentação do governo. Dentro desta lógica argumentativa, mas por uma
outra perspectiva, nos diversos exemplos históricos mobilizados entre o
capítulo \versal{III} e o capítulo \versal{XI}, quando eles não são apresentados para
corroborar essa necessidade de buscar apoio político para o governo que
está se instalando, esses exemplos cumprem a função contrária, ou seja,
revelam o quanto os governantes sozinhos ou fundados unicamente na força
solitária do príncipe não tem a força política necessária para a
conservação do governo.

Talvez o exemplo mais eloquente dessa necessidade de apoio político do
príncipe novo seja a figura do cidadão (\emph{privato ciptadino}) que
ascende à condição de príncipe. Nesse exemplo, todas as nuances daquilo
que desde as primeiras linhas do livro se apresentavam como necessidade
para a constituição desse governo novo, revelam sua forma mais acabada.
Adiante nos ocuparemos como mais atenção sobre esse tipo político,
todavia, convém aqui apenas apontar alguns aspectos desse exemplo, tendo
em vista nosso objetivo de definição do que seja o principado.

O cidadão que ascende à condição de príncipe, modelo privilegiado que
personifica o príncipe novo por Maquiavel, é certamente alguém que, após
uma série de ações políticas, algumas calcadas na fortuna, mas a maioria
na sua \emph{virtù}, consegue o apoio da cidade para assumir o comando
político. Mesmo que ele tenha se valido da fortuna até a sua chegada ao
governo, doravante ele não se pode valer somente desta para manter-se no
comando político da cidade. O mesmo se diga das armas e das ações
cruéis: elas até podem possibilitar a conquista do governo, mas não se
constituem como fundamento seguro para o exercício dele.

Toda a argumentação culmina para o caso do cidadão que ascende à
condição de príncipe ``com o favor dos outros cidadãos'', ascensão essa
baseada ou no favor do povo ou no favor dos grandes. Esse caso torna-se
emblemático porque expõe a real necessidade deste indivíduo que deseja
ser príncipe. Primeiro, ele deve reconhecer que as forças políticas
estão em disputa para além da sua própria força política, visto que ele
não é a única fonte de força política ou a sede do poder; segundo, que
existem outros atores políticos nesse palco, que podem possuir maior ou
menor influência no jogo conforme as circunstâncias; terceiro, e o mais
importante, que o príncipe novo deve se inserir nessa disputa inerente à
vida política da cidade e saber conduzi-la, seja para a não dissolução
desse regime político, que pode ocorrer por meio da instalação de uma
tirania interna ou externa, seja para a própria conservação da sua
condição de figura política de destaque.

Após ter apresentado o tema a ser dissertado na primeira linha do
capítulo \versal{IX}, os dois períodos seguintes consolidam o argumento, como diz
Maquiavel:

\begin{quote}
Porque em toda cidade se encontram estes dois humores diversos e nasce,
disto, que o povo deseja não ser nem comandado nem oprimido pelos
grandes e os grandes desejam comandar e oprimir o povo. Destes dois
apetites diversos nasce na cidade um destes três efeitos: ou o
principado, ou a liberdade ou a licença. O principado origina-se do povo
ou dos grandes, segundo que uma ou outra destas partes tenha a ocasião,
porque, vendo os grandes que não podem resistir ao povo, começam a
aumentar a reputação e o prestígio de um dos seus e fazem-no príncipe
para poderem, sob sua proteção, desafogar o seu apetite. O povo, também,
vendo que não pode resistir aos grandes, aumenta a reputação de um e o
faz príncipe, para serem defendidos pela sua autoridade (cap. \versal{IX},
linhas 2 e 3).
\end{quote}

A primeira informação importante é que a cidade é composta de dois
humores, ou duas partes antagônicas. Num primeiro olhar, somos tentados
a pensar o principado tomado por esse antagonismo, mas note-se que não
são os regimes políticos e sim a cidade que, em sua constituição, em seu
substrato material, tem esse antagonismo político inerente. Tal oposição
natural -- e aqui convém insistir sobre esse aspecto, visto que as
partes ou humores compõe a natureza da cidade e são, portanto,
indissociáveis -- causa os desejos diversos entre os grandes e o povo:
sendo que os primeiros desejam comandar e os segundos em não ser nem
comandados e nem oprimidos. Ora, a combinação desses humores, tal qual
concebida pela tradição da medicina galênica e da hipocrática (nas quais
a combinação deles gerava os diversos tipos de temperamento), resulta em
três formas de governo: o principado, a liberdade e a licença\footnote{Sobre
  essa relação entre a medicina galênica e seus usos por Maquiavel, cf.
  Nicodimov, 2004.}. Esse trecho resultou numa vasta literatura de
comentários a respeito da teoria dos humores e do conflito político em
Maquiavel, que não vamos retomar aqui\footnote{No Brasil temos um
  capítulo significativo desta discussão. Cf. Ames\ldots{};
  Adverse\ldots{}; Bignotto\ldots{}; Cardoso\ldots{}; Martins\ldots{}}.
Entretanto, para nossas intenções, já é possível ver que o principado
não se confunde com a cidade, mas é um dos modos de sua ordenação em
função da combinação dos humores. Destaque-se que a cidade ainda pode
ser ordenada como liberdade (sinônimo de república) e como licença,
neste último caso, trata-se de um modo de ordenação política desregrado,
que se degenera em uma tirania (\emph{Discursos}, \versal{I}, cap. 2).

Em seguida, Maquiavel mostra como nasce o principado, a saber: ou quando
o povo escolhe alguém e lhe dá o apoio ou quando os grandes escolhem um
dos seus para fazê-lo príncipe. A sequência dos eventos revelará que
esse novo príncipe, na verdade, um cidadão (\emph{privato ciptadino}),
deve saber como buscar sustentação para o seu governo, apoio este que
deve, se possível, fundar-se nas duas partes políticas: o povo e os
grandes.

Neste ponto, já nos é possível entender melhor o que é o principado. Ele
é uma autoridade sobre a cidade que nasce a partir da disputa entre os
humores. Parece evidente, portanto, que o principado é, antes de tudo,
uma forma de governo, dentre as possíveis, para comandar uma cidade. Se
isso está claro, então, se voltarmos ao início do texto e repassarmos a
sequência argumentativa, veremos que Maquiavel trata dos vários modos de
conquista de um governo. A conquista do principado é, pois, a ação de
chegar ao posto de comando da cidade, seja usando as armas, seja usando
a fortuna própria ou a alheia, seja usando a \emph{virtù}. Uma vez
conquistado o comando da cidade, o governo, isso não implica em um
controle total das decisões, esse príncipe novo não é um príncipe
absoluto, pois deve decidir levando em conta essa dinâmica das oposições
e saber lidar com os diversos interesses, seja para não ficar refém de
um desses grupos, seja para não constituir inimigos fortes que ameacem o
seu governo.

Assim, podemos deduzir com certa facilidade que o principado é nomeado
deste modo, porque é a forma de governo sob o comando de um príncipe,
particularmente de um príncipe novo. Todavia ainda resta alguns
problemas lançados antes: por que o paralelo recorrente entre o
principado e a república? Se esse principado não é uma república, o que
parece muito evidente, por que ele também não nos permite identificá-lo
à monarquia, ao governo autocrático?

Há um fator que, de certo modo, responderia as duas questões. Como se
verifica não somente ao longo d'\emph{O Príncipe}, mas também nos
\emph{Discursos}, na \emph{História de Florença} e em diversos opúsculos
políticos maquiavelianos, é em função dos conflitos políticos que a vida
política da cidade deve ser pensada. Se nas repúblicas há mais vida e
mais ódio (\emph{O} \emph{Príncipe}, cap. \versal{V}, 9), é porque as lutas
políticas exigem maior engajamento das pessoas, elas devem tomar partido
nas disputas políticas, visto que o governo da cidade é resultado do
conflito. Ora, se no principado não temos a partilha do comando da
cidade, pois o governo se encontra nas mãos do príncipe, isso não
implica que esse governante, que pode até decidir sozinho, não tome tais
decisões a partir de sua vontade privada. Como se verifica nos vários
exemplos históricos citados, muitas vezes esse príncipe é premido a
tomar decisões contrárias aos seus reais interesses em função da
conservação do governo e do bem estar político da cidade. Talvez, nesse
caso, o exemplo mais eloquente seja o de César Borgia que manda matar
seu braço direito, Ramiro Orco. No principado, então, o príncipe toma a
decisão, indica o rumo político a seguir, mas não o faz necessariamente
de \emph{motu} próprio, decide premido pelas circunstâncias, delimitado
pelos interesses diversos que tensionam o seu governo. O exemplo da
possível conquista do reino francês também é ilustrativo
(\emph{Príncipe}, cap. \versal{IV}), visto que não adianta apenas derrubar o rei,
deve-se ter em conta as disputas políticas que ocorrem abaixo dele entre
os nobres e saber o modo como se inserir nelas para conservar o governo.

Então, tal governo principesco se diferencia de um governo de tipo
monárquico clássico, cujo exemplo é o governo do ``Turco'' citado no
capítulo \versal{IV}. Note-se que o próprio Maquiavel, ao tratar do principado
civil no cap. \versal{IX}, aponta para os riscos desse principado tornar-se um
principado absoluto (\emph{Príncipe}, cap. \versal{IX}, 23-27), no qual o
príncipe concentrasse em si todas as decisões. Ora, seja no caso do
governo otomano, seja no caso desse príncipe absoluto apontado no final
do capítulo \versal{IX}, Maquiavel chama a atenção para o fato de que ele não
consegue ter todo o controle e todo o comando político da cidade. O
Turco precisava dos seus \emph{sandjacs} (administradores políticos
nomeados pelo imperador) e o príncipe absoluto precisa dos magistrados,
o que, em ambos os casos, resulta em partilha ou delegação das decisões
políticas para pessoas que obtiveram apoios que resultarão, no limite,
num enfraquecimento do governo desse príncipe absoluto. Esse
posicionamento político é resultado, assim, de uma análise do contexto
político a partir da dinâmica das lutas entre as partes e de um governo
que é obrigado sempre a buscar apoios para se manter. No interior desse
quadro argumentativo, não há poder que se firme como absoluto, pois o
governante deve sempre agir para obter apoio ou deixar de agir para não
criar oposições.

A monarquia ou o governo absoluto não é uma impossibilidade, segundo
Maquiavel, apenas não é nunca um absoluto em sua plenitude e nem um
governo forte. Pela argumentação construída nessa primeira parte do
livro, não há nenhum principado forte que esteja fundado em um governo
absoluto ou monárquico. Talvez um último exemplo que comprove
definitivamente essa ideia, que não por acaso é o último exemplo de
principado analisado, seja os principados eclesiástico, cujo maior
exemplo é o papado. Nem o Papa possui um controle absoluto do seu
governo, ao contrário do que parece ser a primeira vista. A exposição
desenvolvida no capítulo \versal{XI} é muito ilustrativa de tudo o que se
apresentou até então no livro. Primeiro, conforme o estilo maquiaveliano
de exposição, somos tentados a acreditar que o principado eclesiástico é
sim um principado com uma dinâmica política diferente, visto serem
mantidos pelo próprio Deus. Entretanto, ao longo do capítulo, vai
ficando claro que mesmo o papa tem que lidar com as disputas políticas
entre os dois principais grupos políticos, no caso histórico de início
do século \versal{XVI}, os Collona e os Orsini, que nada mais são do que partidos
em disputa pelo governo. Sem contar a própria disputa entre os cardeais,
que é sempre natural na dinâmica política eclesiástica, de modo que,
mesmo o papa, não tem tanta segurança ou facilidade para governar, caso
se imaginasse que fosse esse principado um tipo de principado absoluto.

Ao final dessa primeira parte do livro, podemos constatar que, por todos
os exemplos mobilizados, o principado não é uma monarquia clássica,
muito menos uma prefiguração das monarquias absolutas que vigorarão ao
longo dos séculos seguintes na Europa. Nas poucas vezes que essa
aproximação ocorre, o principado é retratado como um governo fraco,
noção essa radicalmente diferente das monarquias absolutistas que serão
retratadas como governos fortes, materialização dos governos dotados de
soberania política.

Contudo, ainda resta um último ponto: se os principados não são
monarquias de tipo clássico e nem repúblicas, como entender esse regime
político? A resposta já nos é bem presente. Em síntese, o principado é
um governo sob o comando de um príncipe, que é antes de tudo um líder
político que assume o governo da cidade, mas que isso não implica
necessariamente em um governo monárquico ou autocrático de qualquer
espécie. Tendo em vista o destaque deste tipo muito particular de
príncipe que é um cidadão que assume o governo, que, por isso, necessita
agir no interior das disputas políticas naturais entre os humores, vemos
que a dinâmica das disputas políticas próprias da república se conserva
ainda nesses principados, porém agora sob um novo quadro institucional.
Esse seja talvez o ponto de contato entre principado e república
destacado por Maquiavel, pois ambos regimes possuem essa dinâmica
política inerente à natureza da cidade, mas que, na república, essa
disputa se configura em outros termos e resulta em outro ordenamento
institucional diferente do principado. No principado, também as disputas
existem, contudo, elas são exercidas com outra dinâmica política, com
outras delimitações, que não as mesmas das repúblicas, e geram outros
efeitos em termos de ordenamentos políticos e leis. Como já dito antes,
não é somente os conflitos políticos dos humores antagônicos, mas o
quadro político próprio dos principados é que fazem desse governo um
regime diferente das repúblicas, sem que isso implique em um governo
autocrático.

Por fim, podemos ainda constatar que esse principado fundado na ação
política de um cidadão que se torna príncipe se identifica quase
totalmente ao regime ``quase régio'' apresentado ao final do capítulo
\versal{XVIII} dos \emph{Discursos} (cap. \versal{XVIII}, linha 29). Torna-se, pois,
evidente que a crise da república resulta primeiro nesse governo fundado
neste príncipe, que pode implicar, por seu turno, em três outros
governos: no reestabelecimento do ordenamento republicano (hipótese esta
não explorada n'\emph{O Príncipe}), na conservação do principado civil
por longo tempo e na transformação desse principado em uma monarquia,
possibilidade esta que sempre está posta no horizonte, visto que o
principado civil pode tornar-se absoluto, conforme as ações de
centralização política do príncipe gerar, na sequência, um governo
dinástico, uma monarquia. Assim como no \emph{Pequeno tratado sobre as
repúblicas}, Maquiavel nunca é peremptório ou enfático: os ciclos
políticos são possibilidades de mudanças políticas que se apresentam aos
povos. Isso comprova em outro texto aquilo que havíamos constatado nos
\emph{Discursos} (Martins, 2007): que o pensamento político
maquiaveliano se apresenta sempre como possibilidades de configurações
políticas, jamais em ordenamentos políticos que se realizam
necessariamente, como um ciclo político determinista. Essa é uma das
marcas da reflexão política de Maquiavel, de pensar o mundo da política
como possibilidade, no qual se apresenta sempre aos homens alternativas
para tentar direcionar o curso das coisas que pareceria natural e
determinado. Enfim, probabilidade de realização e não determinação
histórica.

\subsection{A herança dos \emph{espelhos de príncipes} na~noção~de~príncipe~maquiaveliana}

Uma vez reconhecido o que é o principado, importa agora entender o que é
o príncipe para Maquiavel, ou mais especificamente, qual a noção de
príncipe que é apresentada em \emph{O Príncipe}. Essa análise será feita
em duas partes, num primeiro momento, recuperando as noções medievais
que emergem da tradição dos ``\emph{espelhos de príncipes''} e, em
seguida, pela análise desse príncipe como um personagem republicano.
Desde início, convém destacar que estamos tratando desse príncipe que é
um cidadão comum que se torna príncipe novo, donde ser nosso foco:
entender como Maquiavel concebe o papel político desse cidadão comum
(\emph{privato ciptadino}) que é preferencialmente o príncipe novo.
Entretanto, isso não deve implicar que há um único tipo de principado ou
de príncipe para Maquiavel. Enfim, interessa-nos saber melhor a noção de
principado e de príncipe que é ressaltada na obra.

Nesta investigação sobre a noção de príncipe em Maquiavel, ao voltarmos
nossas atenções para suas origens, certamente teremos que voltar ao
pensamento político latino do início da medievalidade, particularmente,
a partir das heranças teóricas legadas pelos autores da Patrística
latina e as elaborações que se seguiram no que se refere à noção de
\emph{regimen}. Tal hipótese foi levantada inicialmente por Senellart em
\emph{As artes de governar}, obra na qual procura demonstrar, em sua
primeira parte, como a reflexão pastoral dos autores da Patrística
latina influenciaram um ramo do pensamento político posterior que
desemboca no gênero dos ``espelhos de príncipes'' (\emph{specula
princeps}), cujo \emph{O Príncipe,} de Maquiavel, figura, por vezes,
como um exemplo.

Sobre o gênero literário, já há uma vasta literatura disponível
(\versal{SENELLART}, 2006; \versal{SKINNER}, 2000), que não pretendemos retomar a
exaustão. Contudo convém lembrar alguns de seus aspectos principais.

Os livros do gênero ``espelhos de príncipes'' eram obras escritas por
eruditos que trabalhavam nas cortes e eram destinadas aos futuros
governantes, apresentando orientações para as práticas de governo. A
origem pode ser remontada ao \emph{``Dos Deveres''} (\emph{De Officis},
escrito no século \versal{I} a.C.), de Cícero, da qual se seguiu uma longa e
numerosa variedade de obras que perpassaram os séculos, tendo o mesmo
intuito de orientação e formação para os regentes. Não há um padrão ou
temática única nessas obras, mas em sua maioria, principalmente durante
o período medieval, esses textos preconizavam a valorização dos aspectos
morais e éticos nas práticas de governos principescas. Conforme
Senellart (2006), que faz um amplo resgate histórico desse gênero --
justamente para entender o possível lugar de \emph{O Príncipe}, de
Maquiavel, nesta tradição -- a ênfase nos aspectos éticos e morais dos
``espelhos de príncipes'' têm nas suas fontes os escritos dos autores da
Patrística latina. Desde Agostinho (séc. \versal{V}), passando por Boécio e
Cassiodoro (séc. \versal{VI}), Isidoro de Sevilha e Gregório Magno (séc. \versal{VII}),
entre outros, a reflexão da Patrística latina valorizou a dimensão ética
na condução dos governos em sobreposição sobre os elementos
determinantes próprios do contexto político.

Seguindo adiante na reflexão exposta por Senellart, verifica-se que a
reflexão política anterior a Maquiavel exerceu uma influência
significativa para a elaboração da noção de \emph{príncipe} como um
condutor ou regente, afastando, desse modo, por outra perspctiva, uma
vertente interpretativa que entende o príncipe maquiaveliano como a
prefiguração do soberano moderno.

Ainda, a título de apresentação, nossa interpretação diverge da de
Senellart, embora não lhe seja contrária, ao não optarmos por explorar a
dimensão ética nas práticas de governo, conforme ele enfatiza, para
comprovar as origens teóricas dos ``espelhos de príncipes'' nos textos
pastorais do início da medievalidade latina. Pretendemos seguir, a
partir de suas premissas, uma outra vertente de explicação na qual fique
mais clara as influências desse pensamento político oriundo da
patrística latina que se corporifica na noção de um príncipe como alguém
responsável, prioritariamente, pela condução e direcionamento do povo,
mais do que aquele que exerce seu poder ou domínio sobre os demais. Esse
é o ponto central de nossa interpretação: a partir dos elementos
teóricos herdados da reflexão política medieval latina sobre o
\emph{regimen}, explorar suas possíveis ligações com a noção
maquiaveliana de príncipe como um condutor ou regente do governo, ao
invés da imagem daquele que exerce o poder soberano ou domínio sobre o
seus súditos.

Assim, somos obrigados a retornar a imagem que a Patrística latina
confere àquele que tem por missão conduzir e guiar os cristãos. Conforme
Senellart (2006, 69), foi no momento histórico situado entre a
transferência da dissolução do Império Romano do Ocidente, com a queda
de Roma em 476, e a instauração do Império Carolíngio por Carlos Magno
que a Igreja latina operou ``\emph{uma inversão espantosa. Em vez de
exortarem os reis a governarem com justiça, sabedoria e bondade,
moderando assim o poder, oriundo da violência, pela doçura de seu
exercício, ela faz do `governo' -- ato de regere, dirigir -- a condição
mesma da realeza} (\emph{regnum})'' (Senellart, 2006, 69). O exercício
do governo implica, pois, uma inflexão conceitual: para além (e não
contra) o discurso pastoral, do governante como pastor de almas, esse
comandante da comunidade política cristã deve exercer um maior controle
dos corpos, para que haja, por consequência, uma maior disciplina das
almas com vista à salvação eterna.

Todavia esse dado histórico precisa ser melhor explorado e especificado,
visto que foi, na verdade, um século antes, com a transposição da
capital do Império de Roma para Constantinopla, que se verifica a
inauguração de uma nova reflexão política sobre o modo de governar e
conduzir as comunidades\footnote{Estamos insistindo aqui no termo
  \emph{comunidade,} visto que esse termo foi largamente empregado pelos
  pensadores latinos e terá grande fortuna na posteridade,
  principalmente após a tradução latina da \emph{Política,} de
  Aristóteles, feita por Guilherme de Moerbeke em 1265, que traduz
  \emph{koinonia politiké} por \emph{communicatio politica}. Sobre as
  consequências dessa tradução, cf: Rubstein (1997); Martins (2011;
  201?)}. Conforme Bertelloni (2005), a mudança da capital imperial
originou uma nova teoria política, principalmente a partir dos escritos
de Eusébio de Cesarea, que buscam justificar o poder do imperador romano
sobre os domínios eclesiásticos, doutrina esta também conhecida como
\emph{cesaropapismo}. Em linhas gerais, o \emph{cesaropapismo}
justificava que o imperador cristão, por ter recebido sua missão
diretamente de Deus -- para isso invocando a visão sobrenatural que o
então general Constantino recebe de Deus para pintar a cruz sobre os
escudos de suas tropas, para com isso, conquistar a vitória e tomar o
governo do Império --, teria uma dupla incumbência: guiar os cristãos na
terra e defender a Igreja, sendo, portanto, a maior figura política e
também o maior dirigente eclesiástico. Essa dupla missão, de
\emph{César} e de Papa, unificava-se, agora, na figura do imperador
cristão e de seus herdeiros doravante. Fato é que, depois do século \versal{IV},
os imperadores cristãos de Constantinopla convocavam concílios, nomeavam
bispos e patriarcas, promulgavam normas, leis e documentos dogmáticos,
exercendo um poder direto nos rumos da cristandade.

Porém, esse quadro de ingerência política do imperador sobre os rumos
dos povos cristãos nunca foi bem aceito pelas autoridades eclesiásticas
latinas, principalmente o bispo de Roma, que não era mais do que um
subordinado do Imperador. Ora, do ponto de vista da reflexão teórica,
não foi a queda de Roma em 476, com a deposição de Rômulo Augusto por
Odoacro, que culmina numa reação dos pensadores latinos. Na verdade,
desde 313, quando o imperador Constantino transfere a capital de Roma
para Constantinopla, que a importância política da cidade cede lugar à
nova capital imperial, situada estrategicamente no ponto de confluência
entre aquilo que posteriormente se entendeu por Ocidente e Oriente.
Juntamente com esse dado da política imperial, também o bispo de Roma,
ainda que sempre fosse reconhecido como sucessor de São Pedro pelos
demais bispos, não possuía influência ou poder político o bastante para
determinar os rumos da cristandade. Importa frisar que, do ponto de
vista político, desde o século \versal{IV} Roma já não possuía mais a importância
política de outrora e figurava como uma província de um Império Romano
cujo centro estava em Constantinopla e cuja parte ocidental decaía
gradativamente em termos econômicos, sociais e políticos . O bispo de
Roma, por seu turno, também não possuía, ainda nesses séculos, o
controle político da cidade e nem uma posição de hierárquica superior
entre os demais bispos, sendo um \emph{``primus inter pares''}, o
primeiro entre os pares, o que não implicava em subordinação política
entre o bispo de Roma e os demais patriarcas orientais, por exemplo.

A primeira reação teórica dos pensadores latinos a essa hegemonia do
pensamento político bizantino (o \emph{cesaropapismo}) ocorreu antes
mesmo da queda de Roma em 476, com a reflexão de Agostinho de Hipona,
notadamente, com o seu \emph{Cidade de Deus}. Agostinho, certamente
influenciado pelo ``Saque de Roma'' de 24 de agosto de 410, perpetrado
por Átila, o Huno, postula na sua obra a teoria das duas cidades, uma
celeste, eterna, a qual os cristãos estão destinados, a Jerusalém
Celeste; e outra, terrestre, mundana, corruptível, cuja imagem da Roma
decadente é sua melhor personificação. Consequência direta dessa ideia
de separação das duas cidades é o modo como os cristãos devem se portar
no mundo, fundamentalmente voltados para a Jerusalém Celeste, seu
destino (\emph{telos}) e lugar da sua realização. A Jerusalém terrestre,
decadente, temporal e corruptível não deve ser objeto de atenção e
preocupação dos cristãos, pois ali não está seu destino, sua realização
ou completude, para usar um vocábulo caro ao agostinismo.

Como consequência dessa teoria de separação das esferas celestial e
terrena, a vinculação do homem cristão aos negócios da cidade fica sem
uma justificativa e gera, por seu turno, uma noção de desqualificação do
mundo político que perpassará boa parte dos pensadores medievais latinos
e verá sua influência até o Renascimento italiano. Com efeito, a
caracterização da cidade terrestre como corruptível implicou uma
desqualificação da política e da história, como enfatiza Pocock (1980).
Os pensadores latinos sempre se defrontarão com a dificuldade de pensar
o mundo da política como algo que não fosse efêmero e decadente. Neste
sentido, um primeiro sintoma desse drama e, \emph{pour cause}, uma
primeira tentativa de resposta já se encontra no século \versal{VI} com a
\emph{Consolação da Filosofia} de Boécio. Esse aristocrata romano e
cristão escreve, enquanto aguardava na prisão a execução de sua pena de
morte, um texto no qual relata as angústias do cristão no âmbito da vida
pública na cidade: não poder negar a sua fé e ser ao mesmo tempo um fiel
e legítimo cidadão romano. Como afirma Pocock, a \emph{Consolação da
Filosofia} não é uma obra de filosofia política, mas contém a filosofia
de um homem político (Pocock, 1980, 127).

Seguindo, pois, o nosso fio condutor, entendemos que já no início do
século \versal{V} e doravante -- então antes de 476, como destaca Senellart -- os
pensadores cristãos latinos começaram a elaborar novas teorias sobre a
inserção do cristão no âmbito dos governos e, nesse sentido, o modo como
governar, distanciando-se, assim, do arcabouço teórico bizantino, num
claro esforço de diferenciação e busca de alguma autonomia em termo de
governo nos seus territórios. Contudo, tal reflexão tinha seus limites
claros, seja na separação das esferas celestes e terrestres, seja nos
limites e autonomias de governo das cidades da Europa latina.

Tento em vista essas limitações, de fato, uma teoria de governo não
deveria se sustentar no território da sua ação de comando, mas em algo
que antecedesse isso, visto que não há essa possibilidade de legitimação
do governo. Para isso, e neste caso bem notado por Senellart, a noção de
governo antecede ao \emph{regnum}, ou seja, a condição de comando
político antecede à materialidade do regime político, no caso, o reino
ou o território ou o povo que se deve governar. Essa noção aparece de
modo claro já em Isidoro de Sevilha, que nas suas \emph{Etimologias}
afirma: ``\emph{A palavra reino vem de rei. Com efeito, do mesmo modo
que rei é tirado de reger, reino é tirado de rei. {[}\ldots{}{]} Rei é tirado
de reger. Do mesmo modo que sacerdote vem de santificar, rei vem de
reger. Ora, não se rege se não se corrige}'' (\emph{Etimologias}, livro
\versal{IX}, 3)\footnote{``Regnum a regibus dictum. Nam sicut reges a regendo
  vocati, ita regnum a regibus. {[}\ldots{}{]} Reges a regendo vocati. Sicut
  enim sacerdos a sacrificando, ita et rex a regendo. Non autem regit,
  qui non corrigit''. (\emph{Etimologias}, livro \versal{IX}, 3).}. A sequência é
clara: reger implica em rei que implica em reino (reger -- rei --
reino). O ato de reger (\emph{regere}) é, pois, o que fundamenta a
condição de rei, que, por seu turno, fundamenta o reino. É sobre o verbo
e a ação que recaem, então, os fundamentos e as atenções; é na ação de
reger, no exercício da regência, que se põe o fundamento da condição do
rei. No limite, é antes no \emph{regere} que no \emph{regnum} que o rei
encontra sua fundamentação política.

Então, o exercício do governo funda o regime político e não o reino
(\emph{regnum}) que determina o governo, como não poderia ser de outro
modo para um autor latino dos séculos \versal{VI} e \versal{VII}. Com efeito, tendo em
vista que as responsabilidades sobre os territórios eram delegações do
imperador romano em Constantinopla, a fundamentação da legitimidade do
exercício de governo somente poderia se apoiar sobre o ato ou o
exercício do governo. Seja o rei de um reino do Império Romano, seja
mesmo um bispo com o governo de uma cidade nos confins da Europa, a
fundamentação de suas ações políticas repousava na ação de governo, no
exercício da direção que dava à cidade. Não havia dinastia, sufrágio,
delegação divina que justificasse a contento o exercício de seu ato
governo, senão o próprio ato de governo em si.

Tal ação de governo nada mais era, então, do que reger, corrigir e,
posteriormente, conduzir os homens pelo caminho reto. Assim, o
governante rege e corrige as pessoas, não os obriga ou exerce o seu
domínio, sua violência, seu poder nas ações de seus comandados. Há,
pois, uma certa sutileza no vocabulário que convém destacar. O rei é rei
não porque tem um reino, mas porque rege, porque corrige e orienta o
povo, tal qual o maestro em relação ao coro ou o marinheiro ao navio.
Sua ação de governo é limitada e circunscrita, não cabe a ele impor pela
violência ou qualquer meio extraodinário as ações políticas da cidade,
assim como o marinheiro não impõe pela violência a ação de navegar do
navio, mas o conduz com sabedoria e corrige conforme as forças em ação
os rumos que o navio deve seguir para chegar ao seu destino. Do mesmo
modo o rei rege e conduz o povo, ora corrigindo as ações, ora
impulsionando, para que a cidade perfaça seu caminho de retidão e
justiça. O rei, mas não somente ele, como também o bispo e o comandante
militar, figuram como os pastores que se põe na condução e direção de
seus comandados e não como seres que exercem seu domínio sobre outros
indivíduos.

O destaque deste aspecto pastoral do governante esvazia ainda mais a
imagem anacrônica do rei que domina o seu povo em função de sua
soberania. Esse rei preconizado pelos textos pastorais da Patrística
latina não tem o \emph{dominium} sobre os homens, isso somente lhe será
atribuído quase mil anos depois. O rei é condutor, assim como o
sacerdote, ou melhor, assim como o bispo em relação aos membros de sua
igreja. Na verdade, está se resgatando aqui a antiga antítese entre
\emph{regere} e \emph{dominari}, que remonta a Cícero (\emph{De
Republica,} \versal{I}, 31), mas que sua gênese poderia ainda ser atribuída aos
filósofos gregos, particularmente a Aristóteles. Já na \emph{Política}
(\versal{III}, 1-2), Aristóteles sublinha a diferença do modo de governo da
cidade (\emph{polis}) e da casa (\emph{oikós}). O governo da casa é de
tipo senhor-escravo (metáfora que fará fortuna na história do pensamento
político), no qual o primeiro exerce sua imposição sobre a vontade do
segundo, dominando-o, numa relação hierarquizada e vertical. Na cidade
as relações entre os cidadãos são relações políticas, de igualdade
(\emph{isonomia}), reciprocidade, na qual não há domínio das vontades,
mas se exige a força do convencimento, da argumentação pública, enfim, o
exercício do \emph{logos} na \emph{Àgora}. Como expõe Vernant (2002,
cap. 3), os governos da casa e da cidade são, para os helenos de
natureza diferentes, sendo o primeiro uma relação de caráter doméstico
(\emph{oikonomico}) e a segunda de caráter político (\emph{politica}). A
transposição dessas categorias para o mundo romano resultará em uma
distinção de governos que regem (\emph{regere}) e os que dominam
(\emph{dominium}). O \emph{dominium} aqui derivado de \emph{domus}
(casa), ou seja, as relações de domínio são a transposição para a esfera
pública das relações próprias da casa, o que distingue o governo do rei
e do tirano. O tirano exerce seu domínio sobre o povo, já o rei rege.
Donde a antítese entre o reger e o dominar no exercício do governo.

Nos textos dos autores da patrística latina, a noção de domínio também
possuía uma relação com a irracionalidade dos homens, pois, conforme
Senellart (2006, p. 101), o domínio se exercia sobre os corpos dos
homens controlados pelo pecado, que estavam como que obscurecido pelos
vícios dele decorrente. Para esse ser humano decaído, que se tornou
servo de seu corpo, apenas um comando forte, tal qual se aplica aos
animais brutos, será adequado. ``Não é o homem enquanto homem, mas o
homem rebaixado pelos impulsos de sua carne à condição do animal, que é
o objeto do governo régio e determina as modalidades de seu exercício''
(\versal{SENELLART}, 2006, p. 101). Donde se evidencia o binômio
dominação/irracionalidade, o governo que domina, se impõe sob a condição
de irracionalidade dos comandados. Neste caso, com mais força do que no
pensamento antigo, no qual, ainda que o \emph{logos} se coloque como a
marca da cidade, sua presença no âmbito do político era condição
\emph{sine qua non} da dinâmica política. Aqui é a antítese que se
apresenta: o domínio do governo se impõe pela ausência de razão, porque
os homens ficaram obscurecidos pelos pecados e pelos vícios.

Associada a essa tarefa de reger e corrigir, o governante tem também uma
outra incumbência na sua tarefa, a de vigiar e assistir os homens.
Herdada da tradição grega, mas sob forte influência cristã, o governante
também é um \emph{episcopus}, da qual resultou o termo português bispo.
O \emph{episcopus vem do grego episkopos}, que se origina, por seu
turno, do grego \emph{skopós}, que é observar, ver, vigiar. Logo, também
é uma função do rei o observar, guardar, vigiar para que possa corrigir
e reger. Tal imagem do governante como aquele que vigia, que olha,
remete a um tema clássico do pensamento político que é o da vigilância
como algo típico da ação política. Essa mesma temática aparecerá nos
textos de Maquiavel, particularmente nos \emph{Discursos}, no livro \versal{I},
capítulos 5 e 6, onde ele destaca o papel de vigilância que a população
tem na defesa da liberdade. Ao invocar a defesa da liberdade (\emph{la
guarda della libertá}), ele não somente está pedindo ao povo para
defender, mas também vigiar, tal qual a sentinela, que vigia e defende a
fortaleza.

Passo seguinte dessa concatenação teórica se fará, a partir do século
\versal{XII}, com a associação à função real do dirigir, liderar. Sendo o rei
aquele que rege, observa e corrige o seu povo, cabe então a ele também
dirigir o povo e o governo. O rei deve também guiar e indicar o caminho,
os rumos a seguir pelo seu reino. Tal capacidade de direção decorre da
racionalidade que deve ser própria da personalidade real, ou seja, em
função de sua razão, que se manifesta na prudência e sabedoria, o rei
tem a qualidade para dirigir e liderar o reino.

Essa imagem do rei que também dirige, segundo Senellart (2006), já
aparece em John de Salisbury no seu \emph{Policraticus}, escrito no
século \versal{XII}, mas se encontra também em Tomás de Aquino (no \emph{De
Regno}) e em Egídio Romano (\emph{De Regimine Principum}), ambos do
século \versal{XIII}. No limite o desenvolvimento do argumento é muito claro: por
reger, corrigir e observar, sendo dotado de sabedoria e prudência, está
o príncipe habilitado a dirigir o povo. O rei, portanto, é agora um
rei-guia, como diz Senellart: ``\emph{Regere} não mais simplesmente
corrigir, mas dirigir. Cabeça do corpo político, o príncipe coordenará a
ação de seus membros em vista de um fim coletivo. \emph{Rex sagittator},
dirá Egídio Romano: o rei, como um arqueiro, olhos fixos no alvo, mira
longa à sua frente''\emph{.} (2006, p. 133)

Todas essas qualidades na ação de governar demonstram a liderança e
proeminência do príncipe em face da cidade, o que consolida cada vez
mais sua condição política destacada, como um primeiro, uma figura
destacada e, por isso, que fica na vanguarda, um precursor. Entretanto,
convém frisar, tudo isso ocorre sem que ele faça valer a força ou a
violência como atos próprios e constantes do seu governo para conservar
a sua condição de príncipe, característica essa que será própria da
Modernidade. Apesar da cólera, natural nos homens, o governante dos
``\emph{espelhos de príncipe}'' deve dissimular seu sentimento de
violência, raiva, rancor, oriundo da bile negra, para que ela não turbe
o seu julgamento, conforme sugere Egídio Romano no \emph{De regimine
principum}. Ora, nem encolerizado o príncipe deve fazer uso da força ou
violência no exercício do governo, para que não seja lhe imputado a
perda da razão, da prudência característica dos grandes líderes. Sem
contar que estamos ainda distantes de uma noção de força ou violência
como veremos, por exemplo, na noção que Thomas Hobbes (séc. \versal{XVII})
conferirá ao Estado\footnote{Não pretendemos fazer aqui uma análise da
  presença ou não da noção de Estado em Maquiavel, algo que possui uma
  vasta literatura e que demandaria um esforço analítico que ultrapassa
  as ambições desse ensaio. Todavia, se tivermos em mente aquilo que
  Ercole define como Estado, como sendo: ``a entidade coletiva soberana
  resultante do ordenamento jurídico de um povo num território sob um
  poder comum e que permanece idêntico a si mesmo através da sucessão e
  a mudança dos indivíduos, dos órgãos e das formas constitucionais''
  (\versal{ERCOLE}, 1926, p.65), notar-se-á que tal concepção não está presente
  em sua plenitude em Maquiavel. Como tentaremos demonstrar, as
  concepções políticas que subjazem a noção de principado e príncipe
  estão calcadas em noções anteriores, e, por isso, diferentes dessa
  noção de Estado própria da modernidade, embora seja possível já
  visualizar alguns elementos conceituais que serão desenvolvidos
  posteriormente. Como nos mostra Rubinstein, antes dos humanistas, já
  se encontrava em vários escritos políticos da medievalidade latina o
  termo \emph{status} em uma acepção política, como regime ou forma de
  governo, como, por exemplo, no \emph{Sententia libri Politicorum,} de
  Tomás de Aquino, e no \emph{De Regime Civitatis,} de Bartollo de
  Sassoferrato. Mas em nenhum deles com o sentido que a modernidade
  conferirá ao termo (Rubinstein, 2004, p. 151-163). Para uma
  apresentação geral do debate, cf: Ames, 2011.}. É essa ausência do
argumento do exercício da força, entendida mais como \emph{dominium},
como essecial e inerente ao governo do príncipe que desloca toda a
atenção para o argumento racional, ou como destacará Skinner (2000),
para a capacidade retórica e de persuasão do príncipe.

Enfim, verifica-se por esses elementos da tradição dos ``espelhos de
príncipe'', que muitas dessas concepções repercutem no texto
maquiaveliano. Como se sabe, a noção de príncipe, que perpassa a obra de
Maquiavel como um todo, apresenta inúmeras semelhanças com esse ideal
herdado do pensamento medieval latino, a saber:

\begin{itemize}
\item
  \begin{quote}
  em quase todos os casos apresentados, a exceção do príncipe herdeiro,
  o príncipe de Maquiavel tem a sua condição de governante fundada antes
  no seu exercício político do que no território ou reino;
  \end{quote}
\item
  \begin{quote}
  Esse príncipe maquiaveliano deve, em função de sua condição e daquilo
  que o legitima no governo, reger, conduzir e liderar a cidade -- algo
  que ganha contornos claros e dramáticos no capítulo final do livro;
  \end{quote}
\item
  \begin{quote}
  No caso do uso da violência, apesar de ser apresentado como
  estratagema para a conquista da condição de príncipe, Maquiavel deixa
  claro que ela não é \emph{virtù} e sua mobilização no texto não tem a
  mesma justificação que terá posteriormente o argumento do uso legítimo
  da violência pelo Estado. Esse, na verdade, é um dos pontos pantanosos
  de \emph{O Príncipe}, pois ainda que não tenhamos a mesma
  caracterização da violência estatal da modernidade, ela se faz
  presente como elemento da ação política do príncipe.
  \end{quote}
\end{itemize}

Mas convém entender melhor esses aspectos no texto maquiaveliano.

Como já foi dito, podemos dividir o livro em duas grandes partes, uma
primeira dedicada ao principado e uma segunda parte dedicada ao
príncipe. Na primeira parte, Maquiavel se detém a tipificar os modos de
ascensão à condição de príncipe, a saber: por herança, por armas
próprias e alheias, por \emph{virtù} e por fortuna. Como diz o próprio
autor no cap. 2, no caso do príncipe herdeiro, as dificuldades na
condução do governo são pequenas, pois:

\begin{quote}
Digo, portanto, que nos estados hereditários e acostumados à
dinastia do seu príncipe são muito menores as dificuldades para
conservá-los do que nos novos, porque basta não preterir os ordenamentos
de seus antecessores e posteriormente contemporizar com os acidentes, de
modo que, se tal príncipe tiver uma indústria ordinária, sempre
conservará o seu estado, a não ser que uma força extraordinária e
excessiva o prive dele. E tendo sido dela privado, reconquista tal
condição na medida em que o conquistador enfrentar alguma
adversidade (\emph{O Príncipe}, cap. 2, linha 3).
\end{quote}

Neste caso, ainda, a legitimidade do governo se faz pelo pertencimento à
família que governa, logo, não por si ou por seu governo que esse
príncipe herdeiro assume a condição de príncipe. Esse breve capítulo
inicial da obra revela, também, uma tópica central da reflexão que será
desenvolvida, a saber: a análise do príncipe novo. Como se verifica,
esse é o personagem principal de Maquiavel, o príncipe que ascende ao
governo e que não o recebe de modo hereditário. Todavia, alguém poderia
argumentar que mesmo o príncipe herdeiro, quando ascende ao governo,
naquele momento ele é novo. Porém, não é dessa condição temporal que
Maquiavel está fazendo menção, mas da condição política nova para ele,
algo que um príncipe herdeiro não tem. O príncipe novo é novo do ponto
de vista político, ou seja, ele está inaugurando, iniciando o seu
exercício político, algo que não ocorre com o príncipe herdeiro. Mais
ainda, o príncipe é novo porque funda o governo, inaugura o novo regime,
ao contrário do herdeiro, pois esse, como explícito pelo próprio texto
maquiaveliano, herda, recebe a sua condição política, logo não funda
nada. Em consonância com o pensamento político maquiaveliano, poderíamos
até dizer que um príncipe herdeiro coloca-se como um príncipe novo, se
ele se apresenta na cena política e faz de seu governo uma novidade em
relação ao governo anterior. Mas, neste caso, como declara Maquiavel,
exige-se dele qualidades extraordinárias, ou seja, ele deverá mudar os
procedimentos políticos ordinários, fundar novos ``modos e
ordenamentos'', e, neste caso, seguir o que vem exposto para o príncipe
novo.

Ora, pela próprio entendimento do que seja aqui o \emph{novo},
verifica-se que Maquiavel está tratando de um tipo particular de
personagem político que assume o comando político da cidade. Nos demais
casos relatados de conquistas de principados, até mesmo para o
principado eclesiástico, aquele que ascende à condição de príncipe tem
que justificar por si mesmo e pelas suas condições sua legitimidade no
governo. Seja por ter a fortuna de conquistar o governo, seja por
possuir um exército próprio ou contar com a ajuda de uma força militar
alheia, seja por possuir \emph{virtù}, o aspirante à condição de
príncipe tem que se valer de seus esforços, de suas qualidades para
obter o governo. Pensando na formula isidoriana, que o rei é rei porque
rege, não tendo nada anteriormente que justifique sua condição de rei,
do mesmo modo esse príncipe novo maquiaveliano, é príncipe \emph{a
posteriori}, ou seja, pode-se afirmar que ele é príncipe apenas quando
está no exercício de seu governo, porque conquistou essa condição e a
exerce. Mais ainda, se diz dele príncipe porque tem o principado,
conquistou o regime principado, ou, no limite, se diz príncipe em função
do principado.

Como foi visto, o principado não é o território ou um reino, como em
geral entendemos. Principado é antes uma forma de governo ou um regime
sob o comando de um príncipe. Então, se o principado não é um reino ou
território e o príncipe se diz príncipe em função do principado, sua
condição se faz pelo próprio ato de governar, pela sua ação de governo,
a semelhança da sentença isidoriana. Donde as ações num primeiro momento
objetivarem a conquista do principado, algo que se faz pelos quatro
modos citados, e que depois se mantém em função da sua qualidade ou
\emph{virtù} como governante, tema da segunda parte do livro, que trata
da conservação do governo.

É evidente, pois, que esse príncipe alcança essa condição em função de
suas qualidades ou \emph{virtù}, ainda que em alguns casos ela não se
faça necessária, mas que certamente ela deve comparecer na conservação
do governo. Ou seja, mais cedo ou mais tarde, o príncipe deve demonstrar
possuir \emph{virtù} para conseguir conservar sua condição de príncipe.

Por outro lado, tal formulação de príncipe fundada no exercício do
governo tão somente, excluindo, desse modo, a necessidade anterior do
reino ou território para a fundamentação de seu governo, reforça a ideia
de que não temos ainda a noção de Estado. Com efeito, na medida em que
esse príncipe funda a sua condição no seu governo, não há algo anterior
a esse governo que estivesse posto como elemento de legitimação de sua
ação política. Em outras palavras, o príncipe, ao conquistar o
principado, estabelece nesse momento o seu governo, inicia de fato o seu
regime e terá sempre na sua ação política -- ou em termos
maquiavelianos, na sua \emph{virtù} -- o principal alicerce da sua
condição de príncipe.

Tal hipótese ajudaria a explicar, ainda, porque Maquiavel confere tanto
destaque à \emph{virtù} do príncipe, a ponto de dedicar metade de sua
obra a essa temática. Note-se que não se trata de uma exposição entre os
capítulos 15 a 24 do fundamento ético e moral da \emph{virtù}, tal qual
num tratado de filosofia moral como muitos que se fizeram durante o
Medioevo, mas de uma exposição da \emph{virtù,} depois de comprovada sua
necessidade para a condição de príncipe, para que esse príncipe conserve
o seu governo. Ou seja, na segunda parte do livro, Maquiavel dedica-se a
pensar a \emph{virtù} em função da sua condição de fundamento e
legitimação do governo do príncipe, porque ela se faz necessária.

Agora talvez fique mais claro o sentido da sentença que o príncipe tem
essa condição em função do exercício de seu governo, pois é a
\emph{virtù} que lhe dá esse fundamento em última instância. Para
comprovar isso, dois exemplos mobilizados na primeira parte da obra
podem nos vir em auxílio: o caso de César Borgia e o \emph{privato
ciptadino}.

César Borgia, filho do papa Alexandre \versal{VI}, é um personagem muito presente
n'\emph{O Príncipe}, podendo até ser confundido com o ideal de príncipe
que Maquiavel deseja apresentar. Contudo, nas mobilizações dos feitos de
Cesar Borgia ao longo do livro, sente-se que há uma certa ambiguidade:
por um lado, ele é apresentado como tendo muita \emph{virtù}, o que lhe
possibilitou fazer inúmeras conquistas, mas, de outro, nota-se que
faltou-lhe \emph{virtù} para conservar tudo aquilo que conquistou. O
capítulo \versal{VII} é certamente o melhor lugar para se perceber essa
ambiguidade que ronda a figura de César Borgia.

Na sequência expositiva sobre a conquista dos principados, depois de ter
analisado como eles são conquistados com armas próprias e com
\emph{virtù}, no capítulo \versal{VII}, Maquiavel passa a dissertar sobre o caso
contrário, quando alguém conquista um principado com armas e fortuna
alheia. O problema inicial desse tipo de conquista é logo de início
apontado:

\begin{quote}
{[}3{]} Estes estão fundados unicamente na vontade e na fortuna
de quem lhes concedeu tal status, duas muito volúveis e instáveis, e não
sabem e não podem se manter naquele posto: não sabem, porque não
sendo um homem de grande engenho e \emph{virtù},
não é razoável que, sempre vivendo como homens de condição
particular, saibam comandar; não podem, porque não
têm forças que lhes possam ser amigas e fiéis (\emph{O Príncipe}, cap.
\versal{VII}, linha 3).
\end{quote}

A dificuldade principal é evidente: aquele que conquista um governo
apoiando- se nas qualidades e na força de outros terá muita dificuldade
de se manter nessa condição de príncipe. O que, para o problema em
questão, é muito ilustrativo, pois, segundo Maquiavel, mesmo que alguém
receba um governo de outro, se ele não tem as qualidades necessárias
para conservar esse governo, perderá tal condição. Com efeito, os
fundamentos desse novo governo estão na vontade e fortuna de quem
outorga, que, como declarado, não perfazem alicerces seguros. A vontade,
embora Maquiavel não disserte sobre ela, sabemos todos que é volúvel,
que pode mudar, donde apoiar um regime na vontade de outrem ser de fato
algo transitório. A fortuna, como ele explicará no capítulo 25 de
\emph{O Príncipe}, é inconstante e difícil, senão impossível de ser
dominada. Logo, começar um governo sustentado por dois elementos
exteriores ao seu controle é permitir a instabilidade. Fato esse que
revela, inicialmente, que nada anteriormente sustenta a condição do
príncipe, a não ser o exercício do governo, no caso, um exercício com
\emph{virtù}.

Entretanto, para corroborar sua posição, logo adiante Maquiavel cita
explicitamente o caso de César Borgia, quando diz:

\begin{quote}
Por outro lado, César Bórgia, chamado pelo povo de duque Valentino,
conquistou o governo com a fortuna do pai e com a
mesma o perdeu, apesar de ter ele usado de todos os recursos e ter feito
todas aquelas coisas que um homem prudente e virtuoso deveria fazer para
deitar suas raízes naqueles governos, que as armas e a fortuna de outros
lhe haviam concedido (\emph{O Príncipe}, cap. \versal{VII}, linha 7).
\end{quote}

O exemplo não poderia ser melhor e mais dramático, visto que César,
mesmo tendo ao seu lado o apoio da condição política do pai, que era
papa, não foi o suficiente para que ele pudesse ter garantias de
exercício tranquilo ou seguro de seu governo sobre os territórios recém
conquistados. Teve ele que realizar inúmeras ações para conquistar a
legitimidade política nesses novos governos, mas ainda assim lhe faltou
algo. Maquiavel chama em causa um exemplo em grau máximo, pois nem a
legitimidade política de alguém como um papa é garantia o suficiente
para o exercício do governo de seu filho. No limite, ele dá a entender
que tal legitimidade política não pode ser transferida ou irradiada para
outro, mas apenas o próprio indivíduo é que tem as condições de
estabelecer os alicerces de seu governo. Na sequência isso fica mais
evidente, quando diz:

\begin{quote}
{[}8{]} Porque, como se disse anteriormente, aquele que não
constrói primeiro os fundamentos, poderia, com uma grande \emph{virtù},
construí-los depois, ainda que se façam com incômodo para o arquiteto e
perigo para o edifício. {[}9{]} Se, então, considerarmos todos os
progressos do duque, veremos que ele construiu grandes fundamentos para
um poder futuro, sobre os quais não julgo supérfluo discorrer, porque
não saberia quais preceitos melhores dar a um príncipe novo, senão o
exemplo de suas ações; e se seus modos de proceder não lhe forem
proveitosos, não será por culpa sua, porque nasce de uma extraordinária
e extrema malignidade da fortuna (\emph{O Príncipe}, cap. \versal{VII}, linha 8 e
9).
\end{quote}

Conforme dito, o exemplo de César torna-se mais emblemático, pois,
sabendo ele que deveria buscar outro fundamento para o seu governo, que
não apenas no prestígio de seu pai, tratou de executar uma série de
medidas políticas para arregimentar os apoios necessários para um
governo seguro. Em todas as suas iniciativas obteve êxito, apenas lhe
faltou a fortuna, que no caso não é somente sorte, mas a condição
favorável para o exercício do governo, que conforme Maquiavel explicará
no penúltimo capítulo, também é passível de obtenção ou ao menos de
contenção. Neste caso de César, faltou conter os desfavores da fortuna
para conservar suas conquistas.

A figura política de César Borgia é ilustrativa em um duplo sentido:
primeiro, mesmo ele sendo filho de um político poderoso e recebendo dele
os governos de algumas cidades, isso não foi garantia o bastante para o
sucesso de seu governo. Num segundo aspecto, suas próprias ações de
governo lhe trariam a segurança política necessária desde que ele
executasse todas as medidas, inclusive conter os ``ventos desfavoráveis
da fortuna''. Em ambos os sentidos, o que fez dele príncipe não foi
outra coisa senão o exercício de seu governo, o seu principado.

Antes de passar para o segundo exemplo, convém notar o uso de um
vocabulário próprio da engenharia de construção sendo mobilizado aqui.
Com efeito, Maquiavel se vale dessa metáfora das edificações para
retratar essa ação em busca dos sustentáculos do governo: fundação,
edifício, arquiteto, obra, etc. A conquista de governo e sua conservação
é apresentada então, como obra em execução, donde a necessidade de
planejamento, fundamentos, alicerces, etc., obra essa que não se faz ou
se sustenta a partir de uma obra anterior, mas em si mesmo. Mais ainda,
o governo é uma fundação política, ou seja, uma obra nova, que necessita
de alicerces e na qual um novo campo do político está sendo erguido.

O outro exemplo citado é o caso de um cidadão comum (\emph{privato
ciptadino}) que se torna príncipe. Neste caso, a dificuldade para esse
governante é maior, pois a conquista do governo requer \emph{virtù},
embora a fortuna e as armas alheias possam permitir a conquista, mas a
conservação do principado exige toda a demonstração de \emph{virtù} por
parte deste príncipe novo.

A princípio o próprio termo \emph{privato} utilizado aqui por Maquiavel
é de difícil tradução para o nosso contexto discursivo, pois
literalmente \emph{privato} deve ser traduzido por \emph{privado} em
português. No capítulo \versal{IX}, ele usará a expressão \emph{privato
ciptadino} (1) e \emph{ciptadino privato} (20), que poderia ser
traduzida literalmente por \emph{cidadão particular}. Entretanto, o
autor está se referindo aqui, bem como nas demais ocorrências que se
seguem {[}capítulos \versal{VI} (27), \versal{VII} (1, 2, 6), \versal{VIII} (1, 4), \versal{IX} (1, 20), \versal{XI}
(15), \versal{XIV} (3){]}, ao cidadão comum, não pertencente à família do
governante, que se torna príncipe de uma cidade. Esse é o caso mais
emblemático até do que o de César Borgia, visto que ele deve fazer muito
mais esforços políticos para conseguir chegar ao governo e conservar- se
nele.

No primeiro caso em análise, Maquiavel já deixa claro o seu modo de
entender a questão:

\begin{quote}
E porque este evento, de passar de cidadão comum a príncipe,
pressupõe ou \emph{virtù} ou fortuna, parece que uma ou outra destas
duas coisas mitiga, em parte, muitas dificuldades. Todavia, aquele que
menos se apoiou na fortuna, manteve-se mais (\emph{O Príncipe}, cap. \versal{VI}, 5).
\end{quote}

Ora, tornar-se príncipe é fundamentalmente um ato de \emph{virtù}, ainda
que a fortuna mitigue muitas dificuldades, ou seja, ela auxilia a
conquista do governo. Entretanto, essa ação, esse evento deve ser
calcado na \emph{virtù,} nas qualidades ou excelências políticas desse
cidadão que comanda a cidade.

Essa contraposição entre a \emph{virtù} e a fortuna para o cidadão comum
que deseja tornar-se príncipe é ressaltada nos capítulos seguintes,
respectiviamente nos capítulos \versal{VII} e \versal{VIII}, sempre destacando a
instabilidade e a fraqueza que a fortuna gera no processo de conquista e
no futuro governo. Maquiavel ressalta, pois, que ainda que a fortuna se
apresente e justifique muitas conquistas políticas, aquele príncipe que
nela se apoiou tão somente perdeu o seu governo. O argumento é claro e
insistentemente enunciado: para o príncipe novo, é necessário a
\emph{virtù} para a conservação do poder. Em suas palavras:

\begin{quote}
Aqueles que somente pela fortuna de cidadão se tornam
príncipes, com pouco esforço conseguem sê-lo e com muito se mantêm. E
não têm nenhuma dificuldade neste caminho, porque voam para esta
condição, mas todas as dificuldades surgem quando a ela chegam
(\emph{O Príncipe}, cap. \versal{VII}, 1).
\end{quote}

Note-se que ele chega a usar o termo \emph{voam}, evidenciando que este
postulante ao governo pula as dificuldades inerentes à conquista sem dar
conta delas. Essa vantagem dada pela fortuna, inicialmente, não
significa uma maior facilidade no exercício do governo, ao contrário,
por justamente não ter passado pelo exercício e prática da \emph{virtù}
política no momento da conquista, falta a este cidadão comuum o domínio
das qualidades políticas, dessa \emph{virtù} que se deve apresentar já
no início do processo de tomada do governo. As dificuldades do governo
são, evidentemente, de natureza política e, portanto, são as mesmas que
ele evitou nesse processo de conquista.

A \emph{virtù} se manifesta, pois, sob uma série de medidas e
procedimentos que já vinham sendo elencados desde o capítulo \versal{III}, quando
de fato temos a exposição daquilo que o príncipe novo precisa para
conquistar e manter o governo. Contudo, no início do capítulo \versal{VIII}, o
problema para o cidadão comum que deseja ser príncipe ganha intensidade
e chega aos seus contornos mais fortes no capítulo seguinte. A
dificuldade em tela diz respeito ao cidadão que não tem nem toda a
fortuna necessária e nem toda a \emph{virtù}, mas uma mescla e
insuficiência das duas, como diz:

\begin{quote}
{[}1{]} Mas porque há ainda dois modos de se passar de cidadão a
príncipe, o que não se pode atribuir de todo ou à fortuna ou à virtù,
não me parece que deva deixá-las de lado, ainda que sobre uma delas se
possa discorrer mais amplamente em se tratando de repúblicas. {[}2{]}
Estes modos são: ou quando por algum meio criminoso e nefasto alguém
ascende ao principado, ou quando um cidadão comum, com o favor de outros
cidadãos, torna-se príncipe da sua pátria (\emph{O Príncipe}, \versal{VIII}, linha 1 e
2).
\end{quote}

Antes de prosseguir na análise, importa chamar a atenção para a ressalva
acerca das repúblicas, na verdade, a terceira do texto até aqui (as
outras duas foram: no início do capítulo \versal{II}, linha 1 e no capítulo \versal{V},
linha 9). A referência à república diz respeito ao exercício da
\emph{virtù}, para a qual apenas a fortuna não basta para a conquista do
governo. Com efeito, nas repúblicas, a presença da \emph{virtù} para as
ações de governos é imperativa, pois, conforme fica evidente tanto nos
\emph{Discursos} e na \emph{História de Florença}, a vida política na
república é marcada pelos conflitos e pela dinâmica dos humores, que
podem ser apenas geridos com a presença da \emph{virtù}. Aqui a
referência é a necessidae, um tanto quanto parcial, diga-se de passagem,
da \emph{virtù} ao governo do principado.

A exposição do argumento maquiaveliano nesta primeira parte \emph{d'O
Príncipe} vem seguindo uma sequência expositiva de desvalorização da
fortuna em razão da valorização da \emph{virtù} para esse cidadão que
deseja o comando do principado. Neste início do capítulo \versal{VIII},
Maquiavel, uma vez tendo deixado patente que a fortuna não é
sustentáculo para a conquista e constituição do governo principesco,
aponta para os aspectos essenciais dessa \emph{virtù} requerida. Num
primeiro momento, poderia até ser \emph{virtù} o uso de uma violência
desmedida para a conquista do governo, mas isso não seria o bastante e
nem adequado, conforme exposto ao longo do capítulo \versal{VIII}. Após o uso da
violência para a conquista do comando político, esse cidadão deveria
usar os estratagemas próprios da conquista política em qualquer
circunstância: saber reconhecer os humores e atuar no interior da
dinâmica própria dessa oposição política intrínseca à cidade, tema do
capítulo \versal{IX}, como declara de início:

\begin{quote}
Voltando à outra parte, quando um cidadão comum, não por meio de
crimes ou outra violência intolerável, mas com o favor dos outros
cidadãos, torna-se príncipe da sua pátria -- que poderia ser chamada de
principado civil: e para sê-lo não é necessário toda virtù ou toda
fortuna, mas, antes, uma astúcia afortunada --, digo que se ascende a
este principado ou com o favor do povo ou com o favor dos
grandes (\emph{O Príncipe}, \versal{IX}, 1).
\end{quote}

Aqui se mostra a \emph{virtù} principal desse cidadão comum que busca
ascender à condição de príncipe: ter uma astúcia afortunada para que
consiga granjear o apoio dos grandes e do povo, os dois humores
políticos da cidade. Esta é a dificuldade que não se pode saltar, o
problema político a ser enfrentado e crucial no pensamento político
maquiaveliano: conquistar o apoio das partes e saber lidar com elas,
desde o momento da conquista e durante todo o governo. Verifica-se que
tal tarefa é sempre um trabalho em construção, uma obra inacabada, pois
nunca se tem a garantia plena de apoio incondicional a ponto de não se
precisar constantemente renovar e conservar os apoios que se tem, ao
mesmo tempo em que se busca obter novos partidários. O cidadão que
deseja ser príncipe tem a partir desse momento inicial uma tarefa sempre
a executar: inserir-se na dinâmica da vida política e movimentar-se
nela. Ao fazer isso, ele demonstra a sua \emph{virtù} política.
\emph{Virtù} essa também sempre em exercício, seja no sentido de um
aprimoramento, se for possível atribuir isso a ela, seja na sua perda,
quando não a usa bem.

Enfim, esse \emph{privato ciptadino}, por tudo isso que foi apresentado,
não possui nada que o legitime e dê fundamento ao exercício de seu
governo \emph{a priori}, mas tão somente a sua \emph{virtù}, que nada
mais é do que o exercício das qualidades políticas, antes e durante o
exercício do governo. Nesse sentido, o que torna esse cidadão comum em
príncipe é o exercício das qualidades políticas que permitem a ascensão
à condição de príncipe, num primeiro momento, e de conservação do
governo, doravante.

Voltando ao nosso problema que nos levou a analisar o caso de César
Borgia e do cidadão comum que ascende à príncipe, verifica-se que nesses
dois casos emblemáticos do livro, não há qualquer poder político
anterior que possa conferir legitimidade ao príncipe novo -- como é o
caso de César -- e que, no limite, todo governante deve fundar o seu
governo no exercício das suas qualidades políticas, na sua \emph{virtù}
política que lhe assegura a legitimidade.

Portanto, comprova-se como em \emph{O Príncipe,} de Maquiavel, ainda se
conserva esse aspecto herdado da tradição dos ``espelhos de príncipes''
no que diz respeito à fundação do poder político sobre o exercício
próprio do governo, donde um cidadão ser dito príncipe em função do
governo ou do principado que exerce, condição política essa alicerçada
na sua \emph{virtù}.

Outro aspecto apontado como uma herança dos ``espelhos de príncipe'' na
reflexão política maquiaveliana diz respeito ao fato do príncipe
conduzir ou liderar a cidade, pois, conforme visto, esse príncipe
maquiaveliano deve, em função de sua condição e daquilo que o legitima
no governo, reger, conduzir e liderar a cidade, o que fica evidente no
capítulo final de \emph{O Príncipe}.

O convite à liderança política da cidade é dirigido a algum membro da
família Medici, definido como um príncipe novo, embora essa família já
tivesse o comando da cidade de Florença em vários momentos e, quando o
livro foi escrito, não somente a cidade, mas inclusive o papado, estavam
nas mãos dos Medici, o que não permitiria afirmar que qualquer membro
dessa família fosse um príncipe novo. Contudo, ao contrário dessa
primeira impressão, conforme visto, os Medici podem se encaixar
perfeitamente nesse tipo político que é o príncipe novo. Evidência disto
vem pela invocação da imagem militar de seguir a bandeira ou o
estandarte do comandante, como diz:

\begin{quote}
Vê"-se ainda toda pronta e disposta a seguir uma bandeira, desde
que haja alguém que a empunhe. {[}\ldots{}{]}Tome, portanto, a sua ilustre
Casa este assunto com aquele ânimo e aquela esperança com que se tomam
as façanhas justas, a fim de que, sob o seu estandarte, esta pátria seja
enobrecida (\emph{O Príncipe}, cap. \versal{XXVI}, linhas 7 e 29).
\end{quote}

Nas técnicas de combate antigas, a organização das tropas nos teatros de
operações se fazia por meio de bandeiras, haja vista a dificuldade de
comunicação do comandante com os seus soldados. Desse modo, os soldados
eram instruídos a se orientar pelas bandeiras, do seu comandante mais
imediato inicialmente, até, no limite, do comandante supremo presente no
campo de batalha. A bandeira era símbolo de orientação da tropa, donde
os soldados se agruparem onde estava posicionado a sua bandeira, a sua
tropa, e seguirem o rumo que esta tomar. A bandeira ficava sempre ou com
o comandante ou ao lado dele, tanto que tomar a bandeira de um exército
era o sinal de que o comando caiu. Notório que, até nos dias de hoje, em
que essa função de orientação pelas bandeiras não se faz mais necessária
nas técnicas de combate (seja pelos modernos meios de comunicação das
forças militares, seja pelo modo como as tropas se dispõem para as
batalhas), a força simbólica da bandeira como fonte de unidade é ainda
essencial, não somente nos meios militares, como na sociedade como um
todo.

Ora, a invocação para que alguém da família Medici ``empunhe uma
bandeira'' remonta a essa imagem clássica de liderança militar.
Maquiavel roga para que os Medici liderem e conduzam os italianos,
retirando-os da submissão aos povos bárbaros.

Mas essa invocação final dirigida aos Medici é apenas a exemplificação
de uma ideia que se mostra ao longo de todo o texto. A liderança
política do príncipe já era apontada seja nos exemplos históricos, como
o de Moisés, que lidera, conduz e dirige os hebreus a Israel, seja no
modo como esse príncipe se instala na cena política. Como será explicado
adiante, esse príncipe novo de Maquiavel é o governante que lidera, ele
é antes de tudo um \emph{princeps}, que conduz e dá a direção dos rumos
políticos que a cidade deve seguir, mas sem que isso implique em domínio
ou imposição pela força, visto que deve angariar apoios, o que não seria
o caso do governante com poderes absolutos. Convém insistir, o príncipe
será reconhecido também como príncipe na medida em que dirige, aponta os
rumos, não em função de um poder dominador sobre o seu povo, mas em
função da sua condição de liderança e destaque.

Portanto, assim como rei deve conduzir e liderar, tal qual Moisés em
relação aos hebreus, também o príncipe maquiaveliano deve conduzir e
liderar politicamente e militarmente a cidade. Todavia, tal proeminência
se coloca como problema quanto à utilização ou não da força nesse
exercício político. Na verdade esse é um dos temas mais embaraçosos no
pensamento político maquiaveliano, visto que as nuances dessa força que
o príncipe exerce parece sugerir dominação, e, portanto, uma noção de
poder como comando e obediência, contudo, em outros casos, não se trata
dessa noção de força como domínio, mas como potência individual do
príncipe que consegue dirigir e impulsionar os seus comandados (\versal{SASSO},
1988; \versal{REALE}, 1974; \versal{CADONI}, 1994; \versal{LARIVAILLE}, 1997; \versal{FRONSINI}, 2005).

Uma análise, ainda que superficial, das ocorrências de três vocábulos
relacionados ao tema do uso da potência política do príncipe nos fornece
boas pistas de como Maquiavel concebe n'\emph{O Príncipe} esse uso da
violência, da crueldade ou da força e quais são os seus efeitos sobre os
comandados.

Comecemos pelo termo ``violência'' (\emph{violenzia}), que ocorre por
três vezes na obra: cap. \versal{VIII}, 6; cap. \versal{IX}, 1; cap. \versal{XXV}, 12. No capítulo
\versal{IX}, ele fala de uma conquista do principado que não se faça por
``violência intolerável'', como foi o exemplo retratado no capítulo
\versal{VIII}. Neste caso, a violência é reprovada e seu qualificativo já nos diz
tudo. No capítulo \versal{XXV}, que trata da fortuna e do como de controlá-la,
ele está contrapondo a violência à arte, ou seja, trata-se daqueles que
fazem o uso da violência para conter os imprevistos, ao invés de usar de
artifício ou da inteligência. Novamente, a violência não é vista como
qualidade, antes como debilidade daquele que não possui engenho o
bastante para controlar a natureza.

Entretanto, a referência principal ao uso da violência é o exemplo do
capítulo \versal{VIII}, quando, ao tratar de Agátocles (que usou de violência
para conquistar o comando da cidade), diz: ``Ao ser investido em tal
posto (o de pretor), decidiu tornar-se príncipe e manter com violência e
sem obrigação a outrem aquilo que lhe tinha sido concedido por um
acordo'' (\emph{O Príncipe}, cap. \versal{VIII}, 6). Neste caso, parece
que o uso da violência não somente gerou o resultado esperado para
Agátocles, a conquista do governo de Siracusa, como parece ter sido
legítimo e necessário, dando a entender que não há problemas no uso da
violência para a conquista do governo. Entretanto, adiante Maquiavel
apresenta a sua posição sobre esse uso da violência:

\begin{quote}
{[}10{]} Não se pode também chamar de \emph{virtù} matar os
seus cidadãos, trair os amigos, agir de má-fé, sem piedade, sem
religião: meios estes que permitem conquistar poder, mas não glória.
{[}11{]} Porque, se se considera a \emph{virtù} de Agátocles ao entrar e
ao sair dos perigos, e a grandeza do seu ânimo ao suportar e superar as
coisas adversas, não se vê porque ele haveria de ser julgado inferior a
qualquer excelentíssimo capitão: todavia, a sua feroz crueldade e
desumanidade, com infinitos crimes, não permitiram que fosse celebrado
entre os excelentíssimos homens (\emph{O Príncipe}, cap. \versal{VIII}, 10-11).
\end{quote}

Para Maquiavel, a \emph{virtù} de Agátocles não é plenamente
\emph{virtù}, falta-lhe a glória, ou seja, falta-lhe a admiração que o
governante deve produzir de sua condição de liderança política. Ainda
que haja respeito por parte dos comandados, ausenta-se, no caso desse
governante, a boa imagem que deveria se projetar do seu governo. Donde
essa \emph{virtù} não ser plenamente \emph{virtù} e de Agátocles não ser
o melhor exemplo de príncipe novo, justamente por esse uso desmedido e
``intolerável'' da violência.

Enfim, pelos usos do termo violência, ela não se insere no repertório
indicado das ações para o príncipe praticar, seja na conquista do
principado, seja na conservação. Porém, ainda resta o caso da crueldade,
como os assassinatos e as execuções relatadas ao longo do livro, que não
geraram nem perda da condição de comando político daqueles que as
praticou, muito menos a sua condenação.

A resposta neste caso nos remete a um ponto delicado e inovador do
pensamento político maquiaveliano acerca do uso da crueldade nas ações
de governo. Mesmo tendo criticado Agátocles, Oliverotto da Fermo e até
mesmo César Borgia pelo uso da crueldade na conquista e conservação do
principado, ao final do capítulo \versal{VIII} e depois com mais atenção no
capítulo \versal{XVII}, Maquiavel afirma que a crueldade não é todo má para o
exercício do governo, ao contrário, podendo ser muito útil para gerar
temor e respeito. Mas em todos os casos citados e explicitamente no
capítulo \versal{XVII}, a crueldade produz o efeito de exemplo para disciplinar
as condutas e não como prática louvável em si. A crueldade é tratada
mais no efeito que gera sobre os cidadãos, tornando-os temerosos e
obedientes, do que pela força em si mesma da ação cruel. Como diz
Maquiavel ao final do capítulo \versal{VIII}:

\begin{quote}
{[}27{]} Donde é de se notar que, ao pilhar um governo, deve o
invasor fazer todas aquelas afrontas que são necessárias, e fazê-las de
uma só vez, para não ter de renovar tudo e para poder, não as renovando,
tranquilizar os homens e ganhá-los ao beneficiá- los. {[}28{]} Quem faz
de outro modo, ou por timidez ou por mau conselho, sempre precisa ter a
faca na mão; também não pode nunca se apoiar nos seus súditos, nem podem
estes, pelas injúrias recentes e contínuas, jamais confiar nele.
{[}29{]} Por isso, as injúrias devem ser feitas todas de uma só vez, a
fim de que se saboreiem menos e afrontem menos; os benefícios se devem
fazer pouco a pouco, afim de serem melhor saboreados (\emph{O Príncipe},
cap. \versal{VIII}, 27-29).
\end{quote}

Enfim, a crueldade pode ser uma prática de governo a ser adotada, na
medida em que é exemplar, mas não como rotina, porque neste caso ela
desperta o ódio, que é diferente do temor e mais perigoso para o
príncipe, visto que gera opositores aguerridos. Raciocínio muito
semelhante ocorre com os usos do termo ``força'' (\emph{forza}) e seus
correlatos: \emph{forzare} e \emph{forzati}. Das treze ocorrências no
livro\footnote{São elas: cap. \versal{II}, 3; \versal{III}, 50; \versal{VI}, 16, 20, 21, 22; \versal{VII}, 43; \versal{VIII}, 30;
  \versal{XI}, 17; \versal{XVI}, 11; \versal{XVIII}, 2; \versal{XIX}, 37 e\versal{XX}, 20.}, em cinco Maquiavel se refere aos que foram forçados, ou
seja, impelidos, sendo pois passivos diante da força de outrem. Em
outros cinco casos, ele se refere à necessidade da força e nos três
restantes, a noção de forçar, como o sentido de ter que se utilizar da
força para conquistar, ou seja, num sentido positivo de tomar a
iniciativa de forçar. Considerando esses últimos oito casos no qual a
ação de força parte do príncipe ou de seu governo, em apenas três casos
ele tece considerações sobre o uso desta força.

Na primeira ocorrência, cap. \versal{III} (50), é dito que os governantes que
fundam principados para outros, tornando esses poderosos, podem gerar a
sua própria ruína, pois esses novos principados, fundados ou pela força
ou pela astúcia, nunca são confiáveis. Maquiavel está criticando a
transferência de força de um principado para outro, como prejuízo para
aquele que doa esse poder. Para o principado novo, a força recebida é,
evidentemente, um fator positivo, ainda que ele não emita qualquer juízo
a respeito, mas cuja conclusão é óbvia: força política não se transfere.

A segunda ocorrência do termo força, cap. \versal{VI} (22), está inserida em uma
argumentação capital para o que estamos tratando aqui, a saber, sobre a
necessidade da força no exercício do governo. O capítulo tem como tema a
conquista de principados por meio das armas próprias e da \emph{virtù},
momento no qual Maquiavel mobiliza quatro exemplos históricos de
príncipes conquistadores: Moisés, Teseu, Ciro e Rômulo.

A dificuldade que nasce na metade da exposição refere-se à necessidade
desse novo príncipe ter que constituir novos ordenamentos políticos,
fonte de conflitos e oposições. O argumento é evidente: ao ter que
instituir novos ordenamentos políticos, ou seja, reordenar o governo da
cidade, modificando funções, criando novas atribuições, extinguindo-o
outras, retirando algumas pessoas de certos cargos, colocando outras,
nessa ação o príncipe ganha novos inimigos e verifica, segundo
Maquiavel, que os seus amigos não são árduos defensores de seu governo.
É nesse momento de crise, no qual se aumenta a oposição e não se tem uma
base de apoio político confiável que o príncipe deve se valer das armas
para dar sustentação aos novos ordenamentos nos quais está fundando. As
armas se fazem necessárias justamente para forçar a aceitação desse novo
ordenamento político. O fundo do argumento de Maquiavel, também exposto
nos \emph{Discursos}, se apoia no fato de que os homens são volúveis e é
preciso medidas de força, e não apenas o convencimento ou o exemplo da
\emph{virtù}, para que esse novo governo consiga se instalar na cidade.
Nos \emph{Discursos,} ele também se vale da religião como forma de
persuasão para o povo, sem contar que estamos em um contexto republicano
na exposição dessa obra. N'\emph{O Príncipe}, então, a necessidade do
uso da força para a implantação de novos ordenamentos está diretamente
ligada à necessidade de exércitos próprios, o que confere novos
contornos a essa noção. Numa primeira leitura, pareceria que já teríamos
em Maquiavel a apresentação da necessidade da força na figura do
príncipe para o exercício do governo, até mesmo para sua constituição.
Todavia, ainda que seja necessário usar de força, esse uso somente terá
um valor positivo para o governo se ele for acompanhado de \emph{virtù}
e armas próprias.

Ora, conforme ele demonstrará nos capítulos 12, 13 e 14, de \emph{O
Príncipe}, das forças militares a disposição de um príncipe, a mais
adequada e aconselhável são as armas próprias, em outros termos, o
príncipe deve se valer, fundamentalmente, dos exércitos compostos por
cidadãos da cidade em sua maioria. O motivo deste exércitos serem os
mais adequados está na sua composição: são os próprios cidadãos que se
tornam soldados, criando desse modo um vínculo de fidelidade à cidade, à
pátria, que não se encontra com mesma intensidade nas outras formas de
forças militares.

Assim, mesmo no caso do príncipe que precise forçar, sua força deve
estar fundada em armas próprias, ou seja, ele deve possuir uma condição
política de líder dos seus concidadãos, proeminência política essa
apoiada também numa força militar formada por cidadãos.

Não vamos explorar aqui esse aspecto, mas ainda que brevemente, importa
lembrar a importância da dimensão política dos exércitos para Maquiavel,
que foi objeto de sua análise na \emph{Arte da Guerra}, notadamente no
livro \versal{I}. Ele não somente defende enfaticamente essa condição do soldado
ser um cidadão, como também mostra os bons efeitos e a necessidade desse
elementos para a vida da cidade. Na verdade, Maquiavel está retomando um
\emph{tópica} clássica do `cidadão soldado' que remonta a Cícero e a
Vergério, e que teve inúmeros outros defensores entre os pensadores
latinos.

Enfim, conforme fica claro no capítulo 14 do \emph{Príncipe}, o exército
próprio não é somente um instrumento de defesa indispensável, mas ele
também produz um engajamento político muito saudável na cidade. No caso
desse príncipe novo que deseja assumir o governo, se ele tiver essa
força militar para lhe apoiar, isso robustecerá sua condição de líder
político e facilitará seu governo.

Entretanto, essa introdução forçada de novos ordenamentos se apresenta
sempre como um problema, como declara na linha 17 desse capítulo \versal{VI}:
``\emph{E deve-se considerar que não há coisa mais difícil de tratar,
nem mais duvidosa em obter, nem mais perigosa em manejar, do que
fazer-se chefe para introduzir} {[}forçar{]} \emph{novos
ordenamentos}''. Maquiavel não rejeita a força, utilizar-se dela para
introduzir algum ordenamento não é tarefa isenta de riscos e
dificuldades, o que poderia soar como uma contradição, visto que o seu
uso parece redundar no engrandecimento da figura política do governante.
Contudo, esse não é o caso, a força por si só não é benéfica, dependerá
dos seus efeitos ou resultados para que se reconheça seu real lugar para
o governo do príncipe novo.

Ora, se articularmos essa exigência de que os exércitos sejam formados
pelos próprios cidadãos, com o que é apresentado ao final do capítulo
\versal{VI}, verifica-se que essa necessidade de introduzir (forçar) novos
ordenamentos apoiados nos exércitos próprios na fase inicial da fundação
do governo, tudo isso esvazia a ideia de que o príncipe novo detém em si
uma força ou domínio que impõe aos seus súditos. Ainda que Maquiavel
diga explicitamente que o príncipe precise introduzir novos
ordenamentos, tal procedimento se faz apoiado nos exércitos formados a
partir dos cidadãos (e por que não considerá-los como seus
partidários?), para que, uma vez modificados os ``costumes políticos'',
esse príncipe novo seja, então, admirado e consolide o seu apoio. O
texto é claro nesse sentido:

\begin{quote}
É necessário, portanto, querendo discorrer bem sobre esta parte,
examinar se estas inovações se sustentam por si mesmas ou se dependem de
outros, isto é, se para conduzir a sua obra, precisa rezar ou pode
forçar. {[}21{]} No primeiro caso, sempre entendem mal e não leva a
coisa alguma, mas, quando dependem de si próprios e podem forçar, então
é que raras vezes correm perigo. Daqui nasce que todos os profetas
armados venceram e os desarmados se arruinaram. {[}22{]} Porque, além
das outras coisas ditas, a natureza dos povos é variada e é fácil
persuadi-los em uma coisa, mas é difícil sustentá-los nesta persuasão.
Porém, convém ser ordenado de modo que, quando não crêem mais, pode-se
fazer crerem pela força. {[}23{]} Moisés, Ciro, Teseu e Rômulo não
teriam podido fazer observar sua constituição longamente caso estivessem
desarmados, como no nosso tempo sucedeu com o frei Jerônimo Savonarola,
o qual arruinou os seus novos ordenamentos, quando a multidão começou a
não acreditar nele, e ele não tinha como manter firmes aqueles que
haviam acreditado nele, nem fazer crer os descrentes. {[}24{]} Porém,
estes tem grande dificuldade no conduzir, e todos os seus perigos estão
no seu caminho, e convém que os superem com a \emph{virtù}. {[}25{]}
Mas, uma vez superadas essas adversidades, começam a ser venerados,
tendo perdido aquela sua qualidade que lhe tinham invejado, permanecendo
fortes, seguros, honrados e felizes (\emph{O Príncipe}, cap. \versal{VI}, 20-25).
\end{quote}

Note-se que a força se coloca tão somente no momento da fundação
política e não como prática constante de governo, menos ainda como um
atributo que emana da pessoa do príncipe. Nem estamos aqui levando em
conta se essa força se realiza com ou sem crueldade, mas apenas como ela
se apresenta no texto. É evidente, portanto, que essa noção de força
possui um estatuto mais fraco do que terá nas definições de soberania em
Jean Bodin ou mesmo nas definições de governo em Thomas Hobbes. Tal
força do príncipe, conforme demonstrado, não se aproxima também da noção
de \emph{dominium}, ao contrário, visto que nos exemplos expostos o
príncipe tem que levar em conta o jogo das forças contrárias dos humores
para poder obter êxito nessa disputa e assim conquistar ou conservar o
poder. Ou, por outro viés, o modo como a força e a crueldade são
apresentadas no texto, levando sempre em conta a sua real necessidade e
o efeito que geram, revelam que ambas, tomadas em si mesmas, não
perfazem em elementos essenciais e permanentes do exercício do governo.

Enfim, conforme extraímos do texto, as noções de força e crueldade,
principalmente do cidadão comum (\emph{privato}) que se torna príncipe
novo, não podem se apresentar como elementos permanentes e constitutivos
do governo, mas como estratagemas para a fundação e, de modo inusual no
caso da crueldade, para conservação do governo. Donde ser possível
afirmar que, no mínimo, esse príncipe novo de Maquiavel aproxima-se
muito à tradição antiga, na medida em que é o exercício do governo que
faz deste indivíduo príncipe.

Todavia, muitos podem argumentar em contrário, mostrando que justamente
por essas referências à força, à crueldade e às armas poder-se-ia
sustentar que em Maquiavel temos os elementos constitutivos
característicos do pensamento político moderno\footnote{Como já
  mencionado, essa discussão está bem apresentada no artigo de Ames
  (2011).}. Mas ainda assim, tais argumentos, em sua maioria, forçam uma
interpretação moderna do pensamento político maquiaveliano, na medida em
que renegam justamente essas heranças do pensamento político latino
sobre o Florentino.

Finalmente, verifica-se que \emph{O Príncipe,} de Maquiavel conserva
elementos importantes da tradição dos `espelhos de príncipes', que
iluminam e fornecem novos contornos para o que se está sendo exposto.
Por essa nova perspectiva, a obra ganha um contorno menos moderno, se
poderíamos dizer assim, na medida em que esse príncipe é mais próximo de
um cidadão comum que assume a regência e a condução, do que a figura
política do monarca moderno que encarna em si a natureza do Estado.

Essa análise sobre a noção de principado nos conduz a uma dúvida sobre
os tipos de principados, que se dividem em: hereditário, mistos, novos e
eclesiásticos. Haveria ainda o principado absoluto, no qual podem todos
eles podem se transformar. Contudo, destes todos, Maquiavel concentra
mais sua atenção ao principado civil, que é uma forma de principado
novo, não somente na primeira parte do livro, como na segunda, quando
trata da figura do príncipe, considerado sempre como príncipe novo.

Sobre o principado civil, analisado no capítulo \versal{IX}, sabemos que algumas
de suas características já vinham sendo apresentadas antes e o seu
modelo de governante, o príncipe novo, é doravante o personagem político
principal da obra. Caso essa atenção dada a esse tipo de regime político
não fosse o bastante para concentrarmos nossas atenções, o lugar
conceitual desse principado no pensamento político maquiaveliano é
certamente um ponto nevrálgico. No limite, esse modelo de principado
apresenta elementos teóricos que ao mesmo tempo que rompem com os
modelos tradicionais de principados e dos ``espelhos de príncipes'',
obrigam o leitor a considerar melhor qual a verdadeira relação deste
regime com o republicanismo defendido por Maquiavel nas suas outras
obras. Essa dificuldade já foi explorada por vários comentadores que não
é conveniente aqui retomar o debate nos seus detalhes\footnote{A lista é
  de fato extensiva, contudo apresentamos algumas indicações para
  orientar o leitor: Sasso, Cadoni, Martins\ldots{}}. Para nosso
interesse, pretendemos explorar alguns aspectos dessa noção de
principado civil que corroboram a nossa tese de que temos em \emph{O
Príncipe}, de Maquiavel, um texto que se coaduna com o seu pensamento
republicano. Isso não deve implicar em dizer que o principado civil é
uma república, pois são coisas diversas e o próprio Maquiavel
informa-nos disso. Porém, encontramos nesse tipo de principado elementos
de uma dinâmica política que, por um lado, pressupõe que esse regime
nasceu de um governo republicano e que conserva inúmeros traços dessa
forma de governo, podendo (e aqui convém insistir no caráter hipotético
do termo) fazer com que a cidade volte ao regime republicano, embora
isso não seja uma necessidade ou destino, mas uma possibilidade conforme
o rumo dos acontecimentos.

Em uma leitura superficial deste capítulo \versal{IX}, alguns aspectos já chamam
a atenção: a sua denominação como civil -- e o que entende-se aqui por
civil --; o seu fundador é um cidadão comum (\emph{privato ciptadino});
esse principado não é fundado pela violência e nem por crime, mas pelo
consenso; esse príncipe é escolhido (alguns comentadores chegam a
declarar que ele é eleito); é neste principado que Maquiavel relata que
há dois humores antagônicos em disputa; há uma apresentação das formas
de governo (principado, liberdade e licença); há uma referência ao final
à transformação em principado absoluto (que não é analisado na obra, mas
que podemos deduzir o que seja pela arquitetônica do argumento); é o
capítulo mais conceitual e com menos exemplos históricos (apenas um).
Todos esses elementos indicam que se trata de um dos momentos mais
conceituais da obra, de maior elaboração teórica. Passar por todos esses
aspectos seria tema de uma tese, e, novamente, lembrando nossa intenção
inicial, desejamos chamar a atenção para alguns aspectos tão somente.

Inicialmente a própria denominação de principado civil é digna de nota,
visto que civil é um termo que remete à condição de civilidade, por
oposição ao súdito. Como é notório ao longo do capítulo, esse principado
conserva uma dinâmica política na qual os membros da cidade estão
envolvidos com os rumos da cidade, tomando partido na determinação do
governante ou fazendo oposição contra esse, ação política essa
engendrada seja pelos grandes, seja pelo povo. Ora, tal quadro revela
que estamos tratando de um contexto político de cidadãos e não súditos
que se inserem na vida política da cidade, algo totalmente distante de
uma monarquia ou regime autocrático e muito próximo da dinâmica política
republicana. Neste sentido, o termo ``civil'' nos mostra a presença da
civilidade, ou seja, embora seja um principado, ele não anula ou
extingue a iniciativa política dos cidadãos, ao contrário, o governo é
expressão dessa luta, visto ser o príncipe alguém que é escolhido.

Como já dito, esse príncipe aqui é o cidadão comum (\emph{privato
ciptadino}) que torna-se príncipe, o que pressupõe sua condição de
liderança política num contexto institucional de igualdade de condições
políticas entre os cidadãos. Mais ainda, conforme enfatizado por
Maquiavel já na primeira linha, esse cidadão comum se torna príncipe sem
o uso da força, da violência, da crueldade, ou seja, sem o recurso às
armas, logo, utilizando-se de meios pacíficos e dentro da normalidade
institucional para alcançar o governo da cidade. O que não significa que
esse cidadão não seja um comandante militar ou tenha uma força armada
que esteja na retaguarda, mas que ela não é usada e não é o sustento de
sua conquista. Esse cidadão peculiar chegará ao comando da cidade,
principalmente, fundado nos apoios políticos que angariou em seu
processo de ascensão ao governo. Enfim, esse cidadão se torna príncipe
dentro das normas de civilidade, ele conquista o governo de forma
cívica.

Esse aspecto não pode ser menosprezado. Ao longo da história da
humanidade como um todo, mas particularmente considerando a história
política dos romanos e dos povos latinos que de Roma descenderam, ou
restringindo mais ainda, tendo em consideração apenas as cidades da
península itálica dos séculos anteriores a Maquiavel, conquistar o
governo da cidade de forma cívica, sem o uso das armas, da violência era
uma fato, no mínimo, inusual. Nos próprios exemplos arrolados no livro,
verifica-se quantos foram os casos de tomada do governo por meio de
assassinatos, mortes, crimes, golpes de estado e quão poucos os casos de
conquista por vias pacíficas e institucionais. Falar da conquista do
governo de um principado sem derramamento de sangue, calcado na escolha,
é algo muito peculiar.

A forma da instalação do governo é um outro dado chamativo, pois diz
Maquiavel que esse principado é originado dos grandes ou do povo. Mais
específico ainda, de um lado, ``\emph{porque, vendo os grandes que não
podem resistir ao povo, começam a aumentar a reputação e o prestígio de
um dos seus e fazem-no príncipe para poderem, sob sua proteção,
desafogar o seu apetite.}'' (cap. \versal{IX}, linha 3). De outro, temos o povo,
que, não resistindo a esse desejo dos grandes, adere a alguém que venha
defendê-los contra as ânsias dos grandes. O governo, portanto, é
originado, causado pelos grupos políticos, tem nesses sua fundação,
tanto no sentido temporal -- originando-se deles --, quanto no sentido
de sustentação. Nesse aspecto é um governo conduzido por um só, mas que
não tem neste personagem nem sua origem (convém insistir neste aspecto,
visto que Maquiavel usa o passivo ``\emph{è causato}'', é originado por
outro e não por aquele que vai governar) e nem sua fundamentação. Alguns
comentadores (Larivaille, 1997, Frosini, 2005) chegam a dizer que esse
príncipe é eleito pelos seus concidadãos. Apesar da ideia não ser
estranha ao modo como o argumento é montado, Maquiavel não utiliza nem o
termo ``eleição'' e nem ``escolha'', o que nos parece uma extrapolação,
pois nos ritos políticos da Florença do tempo de Maquiavel, o que havia
eram regras muito rígidas sobre aqueles que seriam aptos aos cargos
políticos (as magistraturas) que eram sorteados a partir de nomes
colocados em uma bolsa. Totalmente estranho a esse contexto florentino
seria um processo de eleição, muito menos de votação, donde nosso
incômodo com essa terminologia. O que se verifica pela economia do
argumento é que um cidadão é alçado à condição de príncipe com o
sustento político de uma das partes, num claro sentido de escolha ou
predileção de um grupo que conduz esse personagem ao governo. Tão
importante quanto esse dado é a informação, não explícita, que há um
modo de institucionalização do governo que se funda em uma disputa
cívica e pacífica. O povo não coloca alguém no governo por ter armas ou
por meio da força militar e nem os grandes por algum golpe. Ora, então
somos obrigados a admitir que esse cidadão comum chega ao principado
após uma disputa política institucionalizada, no qual as partes se
apresentam na cena pública e sustentam seus prediletos. Há, pois, um
palco de disputa política no qual as partes podem tomar parte, os grupos
podem defender os seus interesses de modo pacífico e institucionalizado,
enfim, há uma dinâmica política.

A argumentação maquiaveliana nos remete ao fim para o seu personagem
político principal: o príncipe. Mas o que é, ao fim, ao cabo, o
\emph{príncipe} de Maquiavel?

\subsection{O príncipe ciceroniano e o príncipe na república}

A noção de príncipe, conforme visto, teve um largo emprego entre os
pensadores desde o período romano, de modo que podemos encontrar várias
acepções e usos nos autores latinos durante vários séculos. Todavia, uma
questão pertinente seria verificar nessa história do conceito uma
acepção do termo \emph{príncipe} que não estivesse ligada
necessariamente ao regime monárquico ou, por outro lado e mais
importante ainda, se haveria algum emprego do termo, anterior a
Maquiavel, que o diferenciasse dessa acepção mais próxima ao
\emph{vivere civile} e distante do \emph{principado absoluto}. Como
estamos argumentando, tendo em vista o quadro conceitual do pensamento
político maquiaveliano como um todo, com suas nítidas posições
republicanas -- que se corroboram, como visto, pelo papel central do
conceito de cidadão comum (\emph{privato ciptadino}) na economia do
argumento de \emph{O Príncipe} --, cumpre encontrar uma possível fonte
da concepção de \emph{príncipe} que divergiria da acepção corrente de
governante em regime monárquico e o aproximaria de noções republicanas,
para que possamos mensurar o grau de inovação ou recuperação conceitual
operado por Maquiavel.

Uma primeira origem, como já mostramos, está nas noções herdadas da
medievalidade latina do \emph{regere}, no qual o governante tem essa
condição em função do exercício da sua ação de governo, não exercendo o
poder político de forma absoluta. Uma outra fonte que vem se associar a
esta é o uso que o termo \emph{príncipe} apresenta no pensador romano
Cícero e como essa noção pode ter servido de fundamento para a reflexão
maquiaveliana. Ora, não somente essa origem do termo \emph{princeps}
definido por Cícero, mas as aproximações que se podem fazer dessa
acepção nos \emph{Discursos} e em \emph{O Príncipe} corroboram a nossa
hipótese de que Maquiavel forja um conceito muito particular de príncipe
como governante, não necessariamente monárquico e compatível com um
regime republicano, ou um governo a ``meio caminho'' entre a monarquia e
a república.

\subsubsection{O \emph{princeps} ciceroniano}

Ettore Lepore (1954), em seu estudo sobre a noção de \emph{princeps} no
pensamento político ciceroniano, ao recuperar as origens do termo, nos
revela semelhanças conceituais entre esse conceito romano, da tarda
república, com o \emph{príncipe} em Maquiavel, que permitem uma
interpretação diversa da noção de príncipe como uma figura tipicamente
monárquica.

Ao analisar nas obras de Cícero o uso e os contextos nos quais é
mobilizado o termo \emph{princeps} e, até mesmo, quem é esse
\emph{princeps}, Lepore apresenta um conceito pertencente ao ordenamento
republicano romano, notadamente, ao contexto da tarda república (séculos
\versal{II} e \versal{I} a.C.). Dado esse que já nos mostra que, durante a república
romana, fonte de inspiração dos republicanismos posteriores, inclusive o
florentino, havia esse personagem político. Donde se constata que, para
Cícero, há um \emph{princeps} no interior do ordenamento político
republicano, que ocupa um papel destacado na estruturação do regime.

A análise de Lepore tem, também, a preocupação de reconhecer se o termo
é meramente um nome diverso -- talvez numa acepção mais literária -- ou
se ele é um conceito, de acordo com o seu emprego nos escritos políticos
ciceronianos. Seus estudos nos permitem dizer, pois, que esse conceito
de \emph{princeps} se revela em seus contornos definitivos nas três
obras políticas principais de Cícero -- \emph{De officis, De Leggibus} e
\emph{De republica} --, momento esse de maturidade intelectual do
pensador romano e de seu distanciamento dos ideais aristocráticos que
marcaram seus primeiros escritos, como diz: ``\emph{o termo `princeps'
está presente em todo o desenvolvimento do pensamento político
ciceroniano como perfeito equivalente aos outros termos com o qual
designa o homem político}'' (\versal{LEPORE}, 1954, p. 34).

Concentrando suas atenções, primeiramente, às ocorrências do termo
\emph{princeps}, Lepore verifica que Cícero se vale de dois termos muito
próximos: \emph{princeps e principes}. O primeiro é usado de vários
modos: \emph{princeps-rector, gubernator, moderator, tutor, procurator,
conservator} etc. (\versal{LEPORE}, 1954, p. 34-35), acepções estas que revelam a
associação do termo a uma função política de comando ou a um cargo ou
magistratura de relevo no ordenamento republicano romano. Em todas essas
ocorrências, mostra-se ainda o predomínio de duas compreensões do
\emph{princeps}: o primeiro, em ordem cronológica, é o melhor em
comparação a um grupo. Ao lado dessa dupla acepção, encontra-se uma
terceira, pois se verifica que o \emph{princeps} é também aquele que
toma a iniciativa da ação política, aquele que lidera e está na
vanguarda, o responsável pelo princípio da investida política. Tais usos
indicam que, para Cícero, o \emph{princeps} é uma figura política de
proa, que possui uma \emph{virtus} política destacada e, por isso, se
põe à frente na ação política. Esse \emph{princeps} é, portanto, o
primeiro a agir, o primeiro ou líder em uma iniciativa política.

Constatado esse primeiro bloco de acepções correntes do termo
\emph{princeps} nos textos ciceronianos, Lepore parte para uma outra
vertente de investigação, no afã de descobrir as fontes teóricas do
termo. Tendo em vista a tradição filosófica grega, da qual Cícero é
herdeiro, e dos usos do termo, o \emph{princeps} se aproxima em muito ao
\emph{politikós} grego. \emph{Politikós} esse que não é tanto o rei, o
\emph{basileu}, mas o homem que tem sua natureza conformada pela
\emph{polis}, ou seja, o \emph{princeps} tem as mesmas funções e as
mesmas incumbências políticas do \emph{politikós}, ou seja, ele é alguém
que deve partilhar as magistraturas na \emph{polis}. Um exemplo notório
é que os generais, os juristas, os filósofos, os oradores, enfim, homens
que não eram necessariamente governantes e não tinham cargos públicos de
destaque são designados como \emph{princeps} (Lepore, 1954, p. 48-49).
Então, assim como o \emph{politikós} era concebido como o político por
excelência numa comunidade de políticos -- sem necessariamente ser o
governante --, conforme definido por Aristóteles na \emph{Política}
(\versal{III}, 2, 1275b19), do mesmo modo o \emph{princeps} é um cidadão dotado
de virtude política que assume a liderança de uma ação política entre
iguais. Tal uso equivalente dos termos mostram que, ``\emph{mesmo
mudando o âmbito da linguagem filosófica grega,} {[}o emprego
ciceroniano{]} \emph{é assimilado perfeitamente pela experiência
lingüística romana}'' (Lepore, 1954, p. 45). Ao \emph{princeps} se
associa, então, um ideal de homem político que é, de um lado, herdeiro
da tradição grega do \emph{politikós}, e, por outro, agrega as
qualidades ou \emph{virtus} própria do cidadão romano.

Em contraposição, o \emph{principes}, diferentemente do \emph{princeps},
era uma conceito de homem político ligado aos antigos valores
aristocráticos, de um contexto próprio da \emph{concordia ordinum},
quando se buscava uma conciliação entre as ordens patrícias ou
senatoriais, como no governo da \emph{nobilitas}, ou seja, como o ideal
de governo da aristocracia do início da república romana. Nesta acepção,
esse \emph{principes} é um típico aristocrata que se coloca entre
aristocratas, num círculo político seleto e restrito. Ora, quando não
mais se está colocada a questão em termos de ordenamentos
aristocráticos, mas num quadro de intensas disputas e dissensões
políticas, há uma mudança nessa conceituação de \emph{principes} para um
novo modelo político, o \emph{princeps.} A mudança no quadro político,
de um contexto de \emph{concordia ordinum} para a preocupação com o
\emph{consensus}, evidenciada depois de 60 a.C., leva Cícero a rever o
modo de conceber o seu ideal de homem político. A distinção de
\emph{principes} e \emph{princeps,} sendo o primeiro um conceito
tradicional, ligado à aristocracia, e o segundo como a prefiguração de
um \emph{novus hominus}, um novo político, é fruto das novas exigências
políticas após 60 a.C. Avançando ainda mais, o novo \emph{princeps} é
associado ao \emph{popularis,} ao \emph{sapienter popularis} (\emph{De
Republica,} \versal{II}, 54), ou seja, ao cidadão dotado de prudência, tal qual
se diz de Péricles (\emph{De Oratore,} \versal{III}, 138), a clássica figura
grega que encarna a prudência política do governante. \emph{Popularis}
esse que, desde um uso anterior a Cícero, era sinônimo de \emph{civis},
donde constata Lepore (1954, p. 216):

\begin{quote}
O vocábulo popularis assumiu na tradição retórica e naquela mais
antiga e redescoberta, aquilo que o faz simplesmente equivalente de
\emph{civis}, tendo valor estritamente técnico, como o encontramos no
âmbito filosófico, para exprimir o complexo de valores inclusos no
grego \emph{politikós}.
\end{quote}

Segundo Lepore (1954, p. 230), tal mudança se deve a uma percepção mais
isocrática da política romana por parte de Cícero, o que na verdade era
antes uma crença para a superação das antigas \emph{ordines}
aristocráticas, por regimes mais moderados ou mistos. Na verdade, depois
das sucessivas crises políticas da república romana, Cícero, ao
contrário da política defendida pelos patrícios, se posiciona em favor
de mudanças no regime republicano no sentido de diminuição da hegemonia
patrícia e das diversas ordens aristocráticas, para a incorporação de
novos atores políticos vinculados à plebe. Ora, a concepção política
aristocrática, que pressupunha uma ordem estática e não dinâmica do
campo político, estava em cheque após o advento dos conflitos políticos
que resultaram nas várias guerras civis da tarda república romana. Em
face da guerra civil, Cícero passa a admitir uma dinâmica do mundo
político, aceitando o conflito como um fato próprio da vida republicana,
mas propondo uma \emph{contentio sapiens}, que discipline essa luta
política contra a possibilidade da sedição. É sob tal ótica que se deve
ler o livro \versal{VI} do \emph{De Republica}, tendo em vista o dissenso
político que exigem um \emph{princeps moderator} e \emph{prudens,} ou
seja, um novo homem político que reconheça as mudanças e não esteja mais
preso aos modelos estáticos e conservadores próprios do patriciado. As
mudanças contra as quais ele deve reagir são aquelas que ameacem a
destruição dos ordenamentos, ``\emph{frutos dos egoísmos e das paixões
dos homens indignos de serem aceitos nos círculos dos princeps''}
(Lepore, 1954, 251). Nota-se, pois, a adoção de uma nova visão, mais
dinâmica e orgânica da vida política, dos seus elementos e dos seus
contrastes como característica fundamental do pensamento ciceroniano,
particularmente no momento de composição do \emph{De republica,} no
período de seu exílio após o Consulado, na década de 50 a.C.

Então, essas considerações sobre o conceito de \emph{princeps}
ciceroniano impedem qualquer associação desse com um ideal político de
tipo monárquico. Nos inúmeros trechos das obras ciceronianas citadas por
Lepore, fica muito difícil, para não dizer impossível, que esse conceito
fizesse remissão ao governo de um só, ao chefe de um corpo político que
concentra em si os poderes decisórios. Ao contrário, a noção de
\emph{princeps} se associou cada vez mais ao conceito de
\emph{politikós} grego ou de \emph{civis} romano, ao cidadão que toma
parte na vida política da cidade e que por vezes lidera ou dá a
iniciativa da ação política, sem ser o ponto de concentração do poder
político. Como diz: ``\emph{Todos os termos até aqui usados não permitem
identificar o ideal de} princeps \emph{com um poder monárquico ou de
qualquer modo a um singular}'' (Lepore, 1954, 71).

Esse retorno às origens do termo \emph{princeps} em Cícero mostra que
não somente é inadequado caracterizá-lo como uma designação de monarca
ou congênere, mas, ao contrário, seu uso, principalmente nas obras de
maturidade, revela um emprego terminológico muito próximo do cidadão em
um contexto republicano. Figura essa inserida num mundo político não
mais caracterizado pela ordenação estática aristocrática, mas regido
pela dinâmica das disputas políticas, que deve lutar contra as sedições
e consequente dissolução do ordenamento político da cidade.

Portanto, no pensamento político ciceroniano, particularmente nos textos
políticos de maturidade, a noção de \emph{princeps} remete de modo
direto e inequívoco a um contexto republicano e não monárquico. Ainda
dentro desse contexto republicano, como visto, o \emph{princeps} se
identifica a um modelo de novo homem político em um contexto não mais
dominado pela lógica aristocrática do \emph{consensus ordinum}, mas
inserido numa configuração política não dominada pela elite senatorial,
e sim pelas lutas e tensões políticas entre os diversos grupos políticos
da tarda república romana. Ora, esse quadro de significações do termo
\emph{princeps} na tradição política ciceroniana influencia o pensamento
político latino posterior e, certamente, Maquiavel. Desse modo, cumpre
entender as possíveis aproximações dessa terminologia ciceroniana com as
noções de príncipe mobilizadas nos capítulos \versal{VIII} e \versal{IX} de \emph{O
Príncipe}, bem como com os \emph{Discursos}, capítulos \versal{IX} e \versal{X}, no qual
ele também faz um largo uso da noção de \emph{príncipe}.

\subsubsection{O príncipe dos \emph{Discursos}}

Comecemos pelos \emph{Discursos,} onde, nos capítulos \versal{IX} e \versal{X} do
livro, Maquiavel aborda o papel que devem desempenhar os
ordenadores de reinos e repúblicas, mobilizando, para isso, uma série de
exemplos de governantes romanos que tiveram êxito ou fracassaram nesse
trabalho. Tema esse que já havia sido tratado em parte no capítulo \versal{II}
deste mesmo livro \versal{I}, quando da análise da fundação das cidades. É
particularmente nesses capítulos que o governante único ou a figura do
príncipe é citada várias vezes como o responsável pela fundação ou
reordenação política da cidade. Assim, antes de entrar na análise desses
capítulos em questão, faz-se necessário resgatar o contexto
argumentativo no qual eles se inserem.

No ``\emph{Pequeno tratado sobre as repúblicas''}, como já indicado, o
centro da reflexão maquiaveliana é, pois, apresentar os fundamentos das
repúblicas, sua estruturação, para, em seguida, ou seja, tendo como
referência tais concepções, interpretar a história romana. Então, tanto
o capítulo \versal{II} quanto os capítulos \versal{IX} e \versal{X} do livro dos \emph{Discursos}
inserem-se nesse itinerário argumentativo que busca determinar os
fundamentos da república, não somente romana -- seu caso exemplar --,
mas da noção de república de modo geral.

No capítulo \versal{II}, depois de ter analisado a fundação da cidade por meio de
um ordenador (no caso, a Esparta de Licurgo), Maquiavel passa a tratar
do caso romano, no qual lamenta o fato desta não ter tido a mesma sorte
daquela. Ao contrário de Esparta, os ordenamentos romanos nascem dos
conflitos entre os nobres e a plebe, sendo isso caracterizado como acaso
ou como acidentes. Ao lado dessa contraposição, uma outra mais
significativa tem lugar, pois, diz ele:

\begin{quote}
Porque Rômulo e todos os outros reis fizeram muitas e boas leis, ainda
em conformidade com a vida livre: mas, como sua finalidade foi fundar um
reino, e não uma república, quando aquela cidade se tornou livre,
faltavam-lhe muitas coisas que cumpria ordenar em favor da liberdade,
coisas que não haviam sido ordenadas por aqueles reis. E, se bem que
aqueles seus reis perdessem o poder pelas razões e nos modos narrados,
aqueles que os depuseram, ao constituírem {[}ordinandovi{]}
imediatamente dois cônsules para içarem no lugar dos reis, na verdade
depuseram em Roma o nome, mas não o poder régio: de tal forma que, como
só tivesse cônsules e senado, aquela república vinha a ser mescla de
duas qualidades das três acima citadas, ou seja, o principado e
optimates. Faltava-lhes apenas dar lugar ao governo popular: motivo
porque, tornando-se a nobreza romana insolente pelas razões que abaixo
se descreverão, o povo sublevou-se contra ela; e, assim, para não perder
tudo, ela foi obrigada a conceder ao povo a sua parte, e, por outro
lado, o senado e os cônsules ficaram com tanta autoridade que puderam
manter suas respectivas posições naquela república (\emph{Discursos},
\versal{I}, 2, 18-19).
\end{quote}

Como destacam Sasso (1987) e Reale (1985), chama atenção a rápida
passagem da forma monárquica para a forma republicana. Passagem esta que
estabelece a república como o lugar das análises que se seguiram nos
demais capítulos. Mas, de qualquer modo, o Rômulo citado no capítulo \versal{II}
é uma figura monárquica que deu os fundamentos para a constituição de
uma república ou de um ordenamento político conforme o \emph{vivere
civile}. Por outro modo, caso se queira considerar que Maquiavel tecia
suas considerações levando em conta uma monarquia, estas se encerram
neste momento final do capítulo \versal{II}.

Entretanto, tendo em vista isso que foi dito no capítulo \versal{II}, a
mobilização de Rômulo no capítulo \versal{IX} ganha novos contornos e
dificuldades. Neste capítulo dos \emph{Discursos}, ao retomar o papel de
Rômulo na dinâmica política romana, a perspectiva de análise é outra,
equiparada a de um fundador em um contexto de disputa política, com
vistas à república. A questão central está em como entender esse
governante único, esse primeiro monarca romano no meio de uma análise
voltada para o estabelecimento dos fundamentos da república. Ora, também
no capítulo \versal{IX}, em seu início, Rômulo é apresentado como um fundador de
cidades, não se levando em conta a dificuldade que perpassa as
considerações sobre o nascimento da república romana.

Porém, tanto no capítulo \versal{II} quanto no \versal{IX}, essa transição em si não é
problematizada, não é analisada a fundo. Concomitantemente a esse pouco
falar ou mesmo não falar da transição constitucional, o regime
republicano se apresenta como um \emph{telos}, uma finalidade à qual
Roma parecia destinada. Apesar da imprevisibilidade sobre o que seria no
futuro, Roma apresenta-se destinada a se transformar numa república,
como se ela estivesse orientada para tanto desde seus momentos
primordiais. Segundo Sasso: ``Como se fosse, na realidade, o
\emph{telos} a constituir, além do fim e ao fim do processo, também o
seu critério, a sua origem, a sua razão de ser, o seu impulso condutor''
(\versal{SASSO}, 1987, p. 128). Assim, à débil análise da transformação
política se contrapõe a profundidade de uma necessidade teleológica de
Roma se tornar uma república. A busca dessa condição política
republicana instala-se na reflexão maquiaveliana e passa a conduzir o
seu raciocínio.

Os argumentos mobilizados, então, têm como objetivo a transformação de
uma monarquia, que nasce tumultuada pelo assassinato de Tito Tazio e
pautada por tumultos, numa república completa em meio às vicissitudes. A
questão que nasce da análise do capítulo \versal{II} e da complementação do \versal{IX}
era que Roma tinha como destino não a instauração de uma monarquia
perfeita, mas de uma república. No quadro apresentado por Maquiavel,
desde o seu nascedouro, Roma estava destinada a se transformar numa
república, pois os eventos convergiam para esse fim. Como ele repete ao
longo desses capítulos, quando se olha para o \emph{fim} e não para o
ato em si, a instauração de um \emph{vivere libero} esteve sempre no
horizonte. Esta era uma motivação encontrada já nos primeiros reis, ou
seja, os ordenamentos políticos iniciais tinham como força indutora a
instalação de um \emph{vivere libero}, forma essa que se completará ou
se realizará perfeitamente no modelo republicano:

\begin{quote}
Mas voltemos a Roma. Embora Roma não tivesse um Licurgo que no princípio
a ordenasse de tal modo que lhe permitisse viver livre por longo tempo,
foram tantos os acontecimentos que nela surgiram, devido à desunião que
havia entre a plebe e o senado, que aquilo que não fora feito por um
ordenador foi feito pelo acaso. Porque, se Roma não teve a primeira
fortuna, teve a segunda; pois se seus primeiros ordenamentos foram
insuficientes, nem por isso o desviaram do bom caminho que a pudesse
levar a perfeição (\emph{Discursos}, \versal{I}, \versal{II}, 30-33).
\end{quote}

Estabelece-se, assim, o quadro conceitual das afirmações feitas por
Maquiavel nas primeiras linhas do capítulo \versal{IX}, nas quais se insere a
análise dos fundadores na sequência da exposição sobre as instituições
políticas, aparentemente não apresentando nenhuma relação com os temas
tratados anteriormente. O que se mostrava inicialmente no capítulo \versal{II}
era uma reflexão sobre a fundação por meio do legislador e suas
possíveis implicações sobre a história romana. No capítulo \versal{IX}, apesar de
retomar a temática da origem constitucional, a chave de leitura não é a
fundação, mas a ordenação, que, do ponto de vista da compreensão da
estrutura política romana, apresenta uma outra perspectiva, visto que já
se considera o substrato material do conflito político.

Tributária dessa compreensão é a figura de Rômulo, que, no capítulo \versal{II},
se assemelha à figura de Licurgo. Nessa tentativa de traçar um paralelo
entre as duas personalidades, Maquiavel ressaltava as carências do rei
romano em comparação com o legislador espartano. No capítulo \versal{IX}, ao
contrário, o que nasce é a figura de um outro Rômulo, não mais a versão
romana e imperfeita de legislador conforme descrição anterior, mas o
responsável pela instalação de um processo de ordenamento constitucional
que fará de Roma uma república poderosa.

Conforme Reale (1985, p. 45), o que seria uma aparente questão retórica,
ganha os contornos de uma questão real, se levarmos em conta o fato de
Maquiavel ter tratado apenas dos fundadores de cidades e não dos
reformadores, o que, no limite, impõe a questão da fundação e da reforma
ou reordenação da cidade. O problema parece não estar restrito à
temática da fundação das cidades, muito menos a uma retomada do papel do
legislador nesse momento inaugural. A afirmação maquiaveliana evidencia
uma sutileza terminológica que se configura como um problema de fundo.
Maquiavel fala em termos de \emph{ordenadores} e não de
\emph{fundadores} ou \emph{legisladores}, que, em um primeiro momento,
poderiam ser compreendidos como sinônimos ou como possuidores de função
igual na origem das cidades. O início do capítulo vem aprimorar a
compreensão do papel do ordenador, que passa a se diferenciar do
legislador, como foi Licurgo (\versal{REALE}, 1985, p. 46). O exemplo seria
Rômulo que, ao ser identificado como o ordenador de Roma, ao mesmo tempo
se diferencia dos fundadores, expostos de início. A figura do fundador
foi apresentada como aquele que concebia a cidade e suas instituições
por critérios racionais, dotando este universo político de mecanismos
estáveis e seguros em função de sua própria racionalidade. Essa
racionalidade, marca distintiva da ordenação designada pelo legislador,
contrapõe-se à ordenação segundo o acaso. Levando-se em conta o
legislador helênico, vê-se que ele disporá a cidade segundo um
``critério geométrico, segundo uma regra racional, em um verdadeiro
cosmo de leis'' (\versal{SASSO}, 1987, p. 121)\footnote{Essa imagem do ordenador
  helênico concebendo a cidades segundos critérios racionais e
  geométricos é um dos pontos centrais da argumentação de Jean-Pierre
  Vernant. Em sua explicação, a racionalização da vida e a
  racionalização do campo político estão extremamente imbricadas no
  mundo grego, cujo melhor exemplo seria Hipodamo de Mileto, contratado
  para reconstruir a sua cidade, o fazendo de modo geométrico, donde
  tudo ser ordenado a partir do centro que é a \emph{ágora}. Como nos
  diz Vernant: ``Ele a reconstrói segundo um plano de conjunto que marca
  uma vontade de racionalizar o espaço urbano.'' Concluindo: ``Ora,
  deve-se constatar que o domínio político aparece tão solidário de uma
  representação do espaço que acentua, de maneira deliberada, o círculo
  e o centro, dando-lhe um significado muito definido. {[}\ldots{}{]} A esse
  respeito, pode-se dizer que o advento da Cidade manifesta-se de início
  por uma transformação do espaço urbano, isto é, do plano das cidades.
  É no mundo grego, sem dúvida, primeiro nas colônias, que aparece um
  \emph{plano novo} {[}grifo nosso{]} de cidade em que todas as
  construções urbanas são centradas ao redor de uma praça que se chama
  ágora. {[}\ldots{}{]} Para que exista uma ágora é preciso um sistema social
  de vida implicando, para todos os negócios comuns, um debate político.
  A existência da ágora é a marca do advento das instituições políticas
  da cidade'' (\versal{VERNANT}, 1995, p. 245).}. Isso permite afirmar que
haveria, de um lado, uma ordenação conforme o \emph{logos} e, de outro,
uma ordenação mediante a fortuna, o que não significa que o \emph{logos}
esteja excluído da fundação das cidades que não tiveram a sua origem
pela mão do legislador, mas apenas que não é esse seu critério
prioritário. Ao final do capítulo \versal{II}, Maquiavel se refere a Rômulo,
mesmo não podendo equipará-lo a Licurgo, como aquele que fez ``muitas e
boas leis, conforme ao \emph{vivere libero}'' (\emph{Discursos}, \versal{I}, \versal{II},
32), ou seja, o primeiro rei romano teve também a intenção de bem
conformar a cidade. Do ponto de vista do projeto, Rômulo também figura
entre os fundadores da cidade, embora nesse momento do texto não fosse
ainda possível perceber se Maquiavel falava de um legislador ou de um
ordenador.

Entretanto, paralelamente a essa fundação conforme o \emph{logos},
Maquiavel chama a atenção para um outro modo de ordenação política da
cidade feita pelos acidentes. O termo ``acidente'' sugere várias
acepções, entre elas imprevistos ou acontecimentos não regulares que
alteram o curso político, bem como, acidente como oposto à essência,
retomando um vocabulário aristotélico. O desenvolvimento do texto tende
a reforçar o primeiro aspecto, haja vista o papel dos tumultos políticos
para a instauração do ordenamento constitucional. Todavia, levando-se em
conta que a fundação pelo legislador é conforme o \emph{logos}, sendo
isto uma busca de modelação da natureza do corpo político conforme a
razão, em tais condições, os tumultos, os acidentes também podem ser
compreendidos como algo que se insere no corpo e o modifica, à
semelhança de uma forma acidental ou causa acidental. Associação que não
deveria ser absurda, pois a tradição aristotélica medieval desenvolveu
tanto o conceito de forma acidental como o de forma substancial, que não
está presente na \emph{Metafísica} de Aristóteles, fazendo deles
conceitos-chave para os sistemas metafísicos medievais, principalmente
depois de Averróis e Tomás de Aquino. Ora, mesmo sabendo não ser muito
adequado utilizar um jargão tributário desses sistemas metafísicos nos
textos políticos maquiavelianos (tendo em vista a ausência de uma
reflexão metafísica por parte de Maquiavel), essa acepção de acidente no
sentido de causa acidental pode ser aceitável se considerarmos as
implicações de uma fundação conforme o \emph{logos}. Pensando na
contraposição evidente entre \emph{logos} e acidente, esses fatos que
alteram a ordenação da cidade e inserem algo de novo em sua natureza --
como é o caso dos tribunos da plebe -- podem também ser compreendidos
como formas acidentais. Independentemente da interpretação que se queira
adotar para a compreensão do termo ``acidente'', o que essa retomada do
papel do ordenador no capítulo \versal{IX} vem problematizar é seu estatuto para
a compreensão da formação das instituições. Conhecido desde o início do
livro o papel do legislador e do \emph{logos}, bem como os problemas
decorrentes de uma tal fundação, as atenções se voltam para a
necessidade de se entender os acidentes na fundação de uma cidade.
Ordenação por acidentes que, ao ser pensada no confronto com a fundação
racional do legislador, adquire novos contornos.

O trabalho realizado pelo legislador de conformar as instituições
políticas segundo um critério racional tem, como uma de suas
consequências, a perenidade dessa instituição. Nesse sentido, a
conformação segundo o \emph{logos} pretende retirar da esfera temporal
as constituições políticas (\versal{SASSO}, 1987, p. 120-132). De fato, a
constituição perfeita, o governo misto, coloca-se fora da circularidade
temporal, perfazendo uma linearidade. Por ser um ordenamento político
acabado, pode-se chamá-lo de perfeito, entendendo-o não como o melhor
dos regimes, mas como aquele que não carece de nada, conforme a
conceituação clássica grega. Tal é a constituição política que nasce do
trabalho do legislador.

Já quanto à fundação ordenada pelo acaso, diferentemente da perfeita,
ela se define por não estar acabada e, consequentemente, por estar
submetida às vicissitudes do tempo, à fortuna. Como demonstra Maquiavel
no capítulo \versal{II} dos \emph{Discursos}, teríamos uma gradação de fundações
em três níveis: a perfeita, a ``menos perfeita'' (mas com possibilidade
de reforma), e uma terceira classe de fundações imperfeitas, sem nenhuma
possibilidade de reforma. Roma encontrar-se-ia nesse segundo grupo,
sendo uma constituição a princípio imperfeita, mas que ao longo do tempo
foi se aperfeiçoando, ou seja, foi se reordenando. Os acidentes, nesse
contexto, são os atributos que aperfeiçoam o regime, que o conduzem à
perfeição; são os agregados que se unem a um corpo político
pré-existente e o modificam. Assim, a perfeição (como ``acabamento'' e
não como ``ausência de defeitos'') pode ser possível para esse segundo
grupo, que não fica refém de um determinismo naturalista que impede que
uma constituição imperfeita se transforme numa perfeita. A
perfectibilidade não é um dado inserido apenas no momento de criação do
regime, mas pode ser, para Maquiavel, uma possibilidade ao longo da
existência, sujeita ao acaso.

A garantia dessa perfectibilidade joga essas constituições inacabadas
para a esfera do tempo, submetidos que estão à fortuna. Embora seja
possível alcançar a perfeição, ela se realiza num quadro de dependência
relacionado à esfera temporal, às subidas e descidas, sem
previsibilidade. Se a fundação segundo o \emph{logos} retira o regime
perfeito das variações temporais, a ordenação segundo os acidentes
insere totalmente o corpo político na História, no tempo. É na História,
no interior do tempo, que essa constituição se perfaz, se aperfeiçoa,
agrega a si aquilo de que carece, no caso, as instituições que a tornam
perfeitas. Essa constituição imperfeita está sujeita, também, à dinâmica
do tempo, à variação dos fatos, dos acidentes e, por isso, deve estar
aberta às mudanças, disposta a incorporar aquilo que o tempo lhe traz
como novidade.

Assim, não é meramente uma questão terminológica a distinção entre a
fundação de uma cidade pelo legislador e uma ordenação segundo os
acidentes. Ao expressar que Roma teve uma ordenação e não uma fundação,
Maquiavel demarca o campo teórico no qual deve ser pensada a
constituição romana, que de nenhum modo pode ser equiparada às
repúblicas conformadas por um legislador.

Retomando a questão de como entender esse príncipe dos capítulos \versal{IX} e \versal{X}
do livro \versal{I} dos \emph{Discursos}, temos agora a constituição de uma outra
imagem que não somente a do governante único. Conforme se depreende da
análise da figura de Rômulo, o governante único desses capítulos está
inserido em um contexto republicano, marcado por lutas e dissensões
políticas entre os dois grupos principais. Importa frisar: nas duas
referências ao rei romano, Maquiavel parece ignorar a passagem da
monarquia para a república, e passa a tratar de Roma num contexto
republicano. Roma, nesses capítulos, é considerada em sua fase
republicana, que, na verdade, esteve no meio de duas formas de governos
centralizados: a monarquia anterior à fundação republicana e o império,
posterior a essa fase. Mesmo quando se faz referência a César no
capítulo \versal{X}, onde Maquiavel expõe toda a sua crítica a ele, o contexto
político é republicano e as críticas se devem em grande parte por ter
César contribuído para a destruição da república. Esse governante único
que foi César, ao invés de recuperar o \emph{vivere civile}, aprofundou
a dominação e retirou o pouco que restava de liberdade da república.

Então, os governantes desses capítulos, tendo em vista as circunstâncias
políticas nas quais estão inseridos e o papel político que devem
desempenhar -- o de ordenar ou reordenar um regime --, são considerados
em termos de governantes executivos em condições republicanas. Logo, não
parece existir a possibilidade de qualificá-los como típicos governantes
monárquicos, que centralizam o poder político na figura do chefe de
governo. Mesmo quando considerados unicamente como reordenadores,
Maquiavel enfatiza que eles devem agir em vista do restabelecimento dos
ordenamentos republicanos. Com efeito, ao declarar que ``um ordenador
prudente e virtuoso não deve deixar por herança a autoridade que tomou''
(\emph{Discursos}, \versal{I}, \versal{IX}, linha 8), que remeteria à importância
de um governante único, ele destaca que a \emph{herança} não pode ser
essa autoridade excepcional, mas cuidar para deixá-la nas mãos de
muitos. Ou seja, mesmo que haja uma reordenação por meio de um só, o
resultado deve ser a instalação do governo de muitos. Neste sentido, o
fim de toda a ordenação visada nessas passagens dos \emph{Discursos} é
um regime republicano e não a perpetuação de uma dinastia.

Essa preocupação com o \emph{vivere civile} é tão importante que
Maquiavel insiste neste mesmo capítulo \versal{IX} e no \versal{X} sobre a ameaça de
instalação de um governo absoluto ou tirânico. Nesses é que se encontra
o grande temor: que, após a reordenação de um regime, o poder fique nas
mãos de um só homem ambicioso que usaria mal aquilo que virtuosamente
foi conquistado (\emph{Discursos}, \versal{I}, \versal{X}, linha 8-10). O mesmo se aplica
a César, que não foi o reordenador da república, mas o seu destruidor,
ao contrário de Rômulo (\emph{Discursos}, \versal{I}, \versal{X}, linha 30).

Então, agora pode ser possível entender porque esse príncipe dos
capítulos \versal{IX} e \versal{X} não é um típico monarca, mas, quando muito, um
reordenador de um regime com vistas à recuperação dos ordenamentos
republicanos. Ele é antes de tudo um líder, um precursor de um processo
político, muito próximo ao ideal de \emph{princeps} de Cícero. Como ele
mesmo diz: ``Aquele que se tornou príncipe nalguma república deve
considerar que, depois de Roma tornar-se Império, mais merecem louvores
os imperadores que viveram de acordo com as leis e como príncipes bons,
do que aqueles viveram de modo oposto'' (\emph{Discursos}, \versal{I}, \versal{X}, linha
16).

Enfim, Maquiavel se vale, nesses capítulos, de uma figura de príncipe em
contexto republicano, o príncipe como um reformador de regime, que não
parece se identificar de nenhum modo com a imagem tradicional de monarca
soberano que centraliza o poder político, muito menos com um governante
tirânico, que é a figura contrária desse príncipe.

Enfim, tanto os paralelos suscitados pela noção de \emph{princeps}
ciceroniano, quanto esse uso da noção de príncipe nos primeiros
capítulos dos \emph{Discursos} nos mostram que o termo se distancia
muito de uma acepção autocrática ou monárquica e se aproxima da figura
política do líder que conduz a cidade, seja reordenando as instituições,
seja fundando novos ordenamentos, estabelecendo uma nova dinâmica
política.

Retornando a \emph{O Príncipe}, quando consideramos aquilo que
Maquiavel atribui ao príncipe civil do capítulo \versal{IX}, conforme já
expusemos (p. 73), mas que convém retomar:

\begin{itemize}
\item
  \begin{quote}
  em quase todos os casos apresentados, à exceção do príncipe herdeiro,
  o príncipe de Maquiavel tem o sua condição de governante fundada antes
  no seu exercício político do que no território ou reino;
  \end{quote}
\item
  \begin{quote}
  o príncipe deve, em função de sua condição e daquilo que o legitima no
  governo, reger, conduzir e liderar a cidade -- algo que ganha
  contornos claros e dramáticos no capítulo final do livro;
  \end{quote}
\item
  \begin{quote}
  No caso do uso da violência, apesar de ser apresentado como
  estratagema para a conquista da condição de príncipe, Maquiavel deixa
  claro que ela não é \emph{virtù,} e sua mobilização no texto não tem a
  mesma justificação que terá posteriormente o argumento do uso legítimo
  da violência pelo Estado. Esse, na verdade, é um dos pontos pantanosos
  de \emph{O Príncipe}, pois, ainda que não tenhamos a mesma
  caracterização da violência estatal da modernidade, ela se faz
  presente como elemento da ação política do príncipe.
  \end{quote}
\end{itemize}

Ora, evidencia-se, enfim, que, nos usos e acepções dados particularmente
ao príncipe civil e nos atributos necessários a este (conforme é exposto
nos capítulos seguintes até o fim da obra), essa personagem não se
indentifica com o monarca, mas com a figura de um lider político em
contexto de disputa política que, se não é de fato republicano, é um
palco no qual o conflito político e a disputa entre as forças
antagônicas se fazem presentes.

Mais ainda, recuperando a noção política do \emph{regere} e como
Maquiavel apresenta a figura do príncipe civil como uma conquista do
cidadão comum (\emph{privato ciptadino}), tudo isso configura de outro
modo esse personagem político central, não somente em \emph{O Príncipe},
mas para o pensamento maquiaveliano. Por todos esses elementos expostos
aqui, verifica-se que o príncipe é uma figura política muito próxima,
senão identificada, ao cidadão (\emph{politikós} ou \emph{civis}), que
assume a liderança política da cidade, tendo como desafio principal
fundar ou reordená-la institucionalmente.

Contudo, convém destacar que Maquiavel está o tempo todo tratando de um
tipo de príncipe, o príncipe civil, que não é o único exemplo e nem um
modelo de conduta política que pode ser universalizado. Nos vários
escritos políticos, ele deixa claro e cita que um príncipe pode se
tornar num tirano e fundar um regime centralizado nele. Enfim, no
pensamento político maquiaveliano não há só esse príncipe civil sendo
retratado, mas é este certamente a figura política central para sua
obra.

\subsubsection{O príncipe civil entre república e principado}

Neste ponto, poderíamos dar por concluída nossa exposição, porém, na
análise do príncipe civil nasce uma última dificuldade. Em todas as
circunstâncias apresentadas aqui desse cidadão que se torna príncipe, há
um pano político de fundo de uma república (donde tratarmos de
``cidadão'' e não de ``súdito''), que parece passar por crises, gerando
a necessidade desse cidadão dotado de \emph{virtù} para liderar a cidade
nessa tarefa de fundação ou reordenação política. Esse é o quadro
político, seja nos \emph{Discursos,} livro \versal{I}, capítulos \versal{IX-X} e
\versal{XVI-XVIII}, seja em \emph{O Príncipe}, na primeira parte do livro,
particularmente, nos capítulos \versal{VIII} e \versal{IX}. No limite, a dificuldade pode
ser enunciada nos seguintes termos: quais são as condições políticas que
geram a necessidade da fundação do principado por esse \emph{princeps}
que é um cidadão dotado de \emph{virtù}? Isso nos remete a recuperar a
questão de como pensar a relação entre a república, modelo político do
qual nasce essa figura do príncipe, e o principado? Não somente como de
uma se passa a outra, no caso da república ao principado, mas, também,
se seria possível que um principado pudesse se tornar uma república.

Associar os capítulos dos \emph{Discursos} que tratam da corrupção
(capítulos \versal{XVI}, \versal{XVII} e \versal{XVIII}) com \emph{O Príncipe},
particularmente os capítulos \versal{VIII} e \versal{IX}, é uma hipótese que pode apontar
para a compreensão da solução daquilo que Maquiavel define como a cidade
corrompidíssima. Conforme diz Sasso:

\begin{quote}
Entre o nono capítulo do \emph{Príncipe} e os capítulos dezesseis,
dezessete e dezoito do primeiro livro dos \emph{Discursos}, existe uma
relação sutil e complexa {[}\ldots{}{]} No nono {[}capítulo{]} do
\emph{Príncipe}, a premissa do raciocínio e da análise teórica, é
fornecida por uma forma republicana que, em vista do `excessivo'
conflito dos `humores', o prevalecer dos `grandes' e, paralelamente, o
desencadear-se das paixões populares estão, depois de te-las restituídas
numa `igual' desigualdade, sempre `aprofundando' mais na corrupção
(\versal{SASSO}, 1987, p. 396-397).
\end{quote}

Seja numa relação de gênese, seja numa relação de consequência, as
imbricações entre essas obras necessitam de maiores considerações, a fim
de que se chegue aos limites reais dessa relação. Tanto na perspectiva
de causa quanto na perspectiva de consequência, o importante é entender,
primeiramente, os termos da relação entre república corrompida e
principado civil. Como também diz Sasso, mais importante que o ângulo
visual, deve-se destacar o contexto da cidade corrompidíssima dos
\emph{Discursos} que precede ou está num

\begin{quote}
{[}\ldots{}{]} tempo anterior àquele considerado no \emph{Príncipe}, no
qual a passagem à forma monárquica ou principesca é já, por assim dizer,
considerada atual e imanente ao consenso que os ``populares'' e os
``grandes'' concederam à iniciativa do ``privado'' na cidade
corrompidíssima (\versal{SASSO}, 1987, p. 398).
\end{quote}

Com efeito, conforme apontamos antes sobre as características do
principado civil, as condições que permitem a um cidadão tornar-se
príncipe num principado civil estão já presentes na cidade republicana
dos \emph{Discursos}. A presença dos dois humores, seus desejos
antagônicos, o conflito político entre ambas, a necessidade do
governante em não se apoiar nos grandes, mas saber controlar os seus
desejos, enfim, todas essas condições que pautam as circunstâncias da
existência do principado civil já estavam presentes na república que se
encaminha para a corrupção.

Retornando ao capítulo \versal{XVIII} dos \emph{Discursos}, ao levarmos em conta
o projeto que Maquiavel tem em vista -- reordenar a cidade
corrompidíssima --, o indivíduo que assume esse empreendimento assume
também para si uma autoridade que compete ao príncipe. Seja ele um
ditador ou um \emph{gonfaloniere}, seja ele o primeiro cidadão dotado de
\emph{virtù}, o ponto central é que a esse indivíduo deve-se atribuir um
tal poder \emph{extraordinário,} estranho ao ordenamento republicano e
muito próximo ao príncipe que assume um principado civil. Ora, se a
solução é extraordinária, a questão se desloca para a transição dos
regimes, ou, em outras palavras, a dificuldade está em como pensar essa
passagem de uma república corrompida para o principado civil.

Antes de tratar da passagem, convém destacar o tipo de principado que se
tem em vista. Estamos tratando do principado civil como o regime
político proposto como solução, mas não seria essa a única opção, pois,
por outro lado, nada impediria o estabelecimento de uma tirania ou de
algum tipo de governo despótico e autoritário. Os regimes tirânicos (ou
nem tão tirânicos, como o governo do Turco, que era um governo
centralizado, mas não necessariamente tirânico ou absoluto) não são
sugeridos em nenhum momento como a solução mais adequada ou a mais
consequente para as condições republicanas. O problema parece ser que
esses principados, que podem ser identificados como absolutos, negam ou
anulam os conflitos políticos ocasionados pelos humores presentes na
cidade. Ao concentrarem todo o fundamento da ação política no
governante, impedem o ``natural'' funcionamento da vida política e, por
consequência, impedem que os grupos ou os humores manifestem seus
desejos pelo meio natural de luta política dentro do corpo político,
aquilo que caracteriza a civilidade política. O principado civil,
conforme visto, é aquele regime que mais assegura o \emph{vivere
libero}, que respeita e garante os conflitos, pois os assume como
inerentes à vida política no principado. Esse será o ponto central: ao
contrário do principado de tipo absoluto, o principado civil conserva os
aspectos básicos da vida política numa república, não anula por completo
o \emph{vivere libero}, a civilidade, o jogo político e os conflitos que
lhe são inerentes; antes os reconhece e os assume como dados essenciais
do principado. O maior problema em se considerar a transição de uma
república corrompida para um principado de tipo absoluto é a
possibilidade de subtração completa das características presentes na
primeira, não mais reconhecidas e existentes nesse tipo de regime. Logo,
o principado de tipo absoluto, apesar de ser uma solução possível, não
pode ser compreendido como a mais adequada para uma cidade que necessita
ou \emph{conservar-se como republica ou criá-la de novo}, que é o
problema central do capítulo \versal{XVIII} dos \emph{Discursos}.

Neste sentido, convém retomar o que Maquiavel expõe no capítulo \versal{X} do
livro \versal{I} dos \emph{Discursos}, quando trata dos reformadores de Roma.
Paralelamente a sua crítica a César, visto como um dos principais
destruidores da república romana, a análise que se segue dos imperadores
romanos visa ressaltar, fundamentalmente, que: aqueles que alcançaram o
império por herança foram maus, ao contrário daqueles que o assumiram
com o apoio dos seus concidadãos (\emph{Discursos}, \versal{I}, \versal{X}, linha 20); que
os \emph{príncipes} (termo do próprio Maquiavel para se referir aos
imperadores que se seguiram a César) que procuraram reordenar o reino e
fazer com que as instituições funcionassem conforme a sua finalidade,
foram mais bem sucedidos em relação àqueles que procuraram, por meio
delas, conquistar glória para si. Donde conclui:

\begin{quote}
E o príncipe que realmente buscar a glória mundana deverá desejar ter
nas mãos uma cidade corrompida, não para destruí-la de todo, como César,
mas para reordená-la, como Rômulo. E realmente, os céus não podem dar
aos homens maior ocasião de glória, nem os homens podem desejar glória
maior. E, se, para bem ordenar uma cidade, houvesse necessidade de depor
o principado, mereceria alguma desculpa quem não a ordenasse para não
cair de tal posição, mas, em sendo possível manter o principado e
ordena-lo, não merece desculpa algum quem não o faça (\emph{Discursos},
\versal{I}, \versal{X}, 30-32).
\end{quote}

Não está descartada, seja nos \emph{Discursos}, seja em \emph{O
Príncipe}, a possibilidade das repúblicas se transformarem em tiranias
ou que os principados tornem-se regimes despóticos. Essas
degenerescências políticas, se pudermos chamarmos assim esses regimes
autocráticos, não estão nas atenções de Maquiavel. Ele está preocupado
com a solução para a recuperação da civilidade política, donde ser o
principado civil o regime que melhor se adequa a esse fim. O modo como é
ordenado o principado civil permite, pois, considerá-lo como o regime
que melhor prepara o povo ou o universal (também denominado como matéria
em vista de uma forma) para o \emph{vivere libero} e o \emph{vivere
civile.} A instalação desse novo governo principesco confirma a hipótese
de que, apesar da derrocada do governo anterior, a cidade na qual ocorre
essa transição política conserva os aspectos essenciais de legalidade
política, de civilidade, de um respeito, ainda que mínimo -- poder-se-ia
conjecturar --, aos ordenamentos políticos, aos valores cívicos.

Aqui cabe ressaltar o porquê de Maquiavel nomear esse tipo de principado
como ``civil''. Evidentemente, é em função dele se instalar reconhecendo
a dinâmica dos conflitos e respeitando esse dado essencial durante o
governo, sem descambar para um governo absoluto ou aristocrático, bem
como numa espécie de populismo ou, anacronicamente falando, numa
``ditadura do proletariado''. O principado civil é civil por instaurar e
estimular a dinâmica política fundamental para a vitalidade da cidade,
por recuperar e conservar a civilidade.

Retomando a análise, verifica-se ainda que o quadro no qual Maquiavel
descreve a origem do principado civil é muito semelhante a uma
república. Tendo em vista a origem não dinástica do príncipe, podemos
questionar até se o regime político anterior ao principado civil era uma
monarquia ou não. A transição que não deve ser calcada na violência, mas
no consenso, a presença dos humores, que tensionam o governo, enfim,
todos esses aspectos elencados em \emph{O Príncipe} remetem à dinâmica
política retratada nos dezoito primeiros capítulos dos \emph{Discursos}.

Por isso, se pretendemos pensar numa gênese do principado civil, devemos
concordar que uma hipótese, talvez a mais provável, é a da cidade
republicana que atinge um certo grau de corrupção e não consegue, por si
só, retomar o seu ordenamento político inicial. O desenrolar do capítulo
\versal{IX} de \emph{O Príncipe} comprova ainda mais essa constatação inicial,
pois Maquiavel mostra como o príncipe novo, que chega ao poder nessas
condições, deve se comportar diante do jogo de interesses e de poder que
permanece após a sua instalação no comando do principado. Pelo controle
dos humores e dos desejos, deve-se tomar todo o cuidado para não ficar
refém dos interesses dos grandes, mas manter um certo equilíbrio entre
os dois grupos principais (grandes e povo) e, quando isso não for
possível, apoiar-se totalmente no povo, ainda que isso implique certos
constrangimentos às suas decisões políticas.

Portanto, a descrição que emerge nesse capítulo \versal{IX} de \emph{O
Príncipe} sobre o principado civil coloca-o muito próximo do
ordenamento da cidade republicana e permite pensar que a transição de um
regime a outro não é uma inferência inadequada. Ao contrário, entre os
modelos de regimes que figuram no horizonte do possível nas descrições
maquiavelianas, o principado civil é o mais adequado às necessidades de
um governo forte exigidas ao final do capítulo dezoito do livro \versal{I} dos
\emph{Discursos}. Por conservar os elementos fundamentais da república
e, também, por manter a presença do essencial da vida política, com seus
humores e os conflitos entre eles, esse regime vem totalmente ao
encontro das exigências que a cidade corrompidíssima solicita para o seu
reordenamento. Seja ao tratar da corrupção nos capítulos \versal{XVI}, \versal{XVII} e
\versal{XVIII} do livro \versal{I} dos \emph{Discursos}, seja nesse capítulo \versal{IX} de
\emph{O Príncipe}, seja, ainda, no capitulo \versal{LV} do mesmo livro \versal{I}
dos \emph{Discursos}, entre as principais causas da corrupção está a
ambição dos grandes em tomar o poder. Em todos esses capítulos, bem como
em inúmeras outras partes, o desejo dos aristocratas em assumir o
comando do poder para si ou instalar um governante que lhe seja
favorável está sempre presente. Ora, mais do que pensar numa corrupção
endêmica e generalizada pela cidade, ao considerar-se a corrupção do
povo (a corrupção da matéria), encontrar-se-á mais um desejo de
usurpação dos grandes e menos uma desobediência às leis por parte do
povo em geral. Quando, pois, numa república dominada pelos grandes, não
se encontram mais meios de impedir esse avanço da aristocracia sobre o
poder, não há outro remédio senão instalar um governo forte, \emph{quase
régio}, sob a forma do principado civil:

\begin{quote}
Razão por que nessas províncias não surgiu nenhuma república nem nenhum
tipo de vida política; porque tais tipos de homens são totalmente
inimigos da civilidade. E não seria possível introduzir uma república em
províncias assim constituídas, mas, para reordena-las -- caso a alguém
coubesse tal arbítrio --, não haveria outro caminho senão constituir um
reino. A razão é que, onde a matéria está tão corrompida, não bastam
leis para contê-la, e é preciso ordenar junto com elas maior força, que
é a mão régia, que com poder absoluto e excessivo, ponha freio à
excessiva ambição e corrupção dos poderosos (\emph{Discursos}, \versal{I}, \versal{LV},
21-23).
\end{quote}

Nessa passagem, como em outras, repetem-se as mesmas exigências
apresentadas para a instalação de um principado civil em substituição à
república corrompida: excessivo poder da aristocracia, ineficácia das
leis e das instituições, um governo forte, mas que se instale sem
violência, a existência de uma parcela, ainda que mínima, de civilidade.
A necessidade de um governo forte não implica necessariamente a fundação
deste com o uso da violência: força e violência não se seguem.

Sobre a passagem do principado civil para uma república não há uma
exposição detalhada de Maquiavel em \emph{O} \emph{Príncipe} e nem na
parte inicial dos \emph{Discursos}. Talvez as considerações dos
capítulos \versal{IX} e \versal{X} do livro \versal{I} dos \emph{Discursos} pudessem fazer uma
alusão a isso, embora saibamos que Maquiavel está tratando da transição
da monarquia romana para uma república. Caso pensemos que esse príncipe
evocado nesses capítulos dos \emph{Discursos} seja um típico príncipe
civil -- hipótese essa não absurda, visto que o objetivo desse príncipe
é fundar ou reordenar as instituições com vista a um regime republicano
e o príncipe novo do principado civil, se não tem esse objetivo, deve ao
menos conservar o mínimo de civilidade que resta à cidade --, então
teríamos sim uma exposição da passagem do principado civil para uma
república.

Junto a essa possibilidade interpretativa, outro argumento que também
poderia auxiliar nessa compreensão da possível passagem do principado
civil para a república, está nos \emph{Discursus rerum Florentinarum},
escrito em 1520 sob encomenda do papa Leão \versal{X} em função da morte de seu
sobrinho e governante de Florença, Lorenzo de Medici. Neste opúsculo,
Maquiavel disserta sobre qual seria a melhor forma de governo a se
instalar na cidade naquele momento de crise do regime comandado pela
família Médici, que, segundo ele, é uma crise no ordenamento político da
cidade de longa data. Após demonstrar que Florença nunca teve de fato
nem uma república e nem um principado, ele sugere dois modos de
reordenar a cidade: uma ordenação verdadeiramente republicana ou uma
reordenação verdadeiramente principesca. Desenvolvendo um pouco mais,
diz ele que, caso esse ordenamento não seja assim, as duas formas de
regime entrarão em dissolução. Donde conclui:

\begin{quote}
{[}\ldots{}{]} digo que não se pode ordenar nenhum regime estável que não
seja um verdadeiro principado ou uma verdadeira república, pois todos os
regimes postos entre estes dois são defeituosos. A razão disso é
claríssima: se o principado tem apenas uma via para sua dissolução, que
é se tornar uma república, e da mesma maneira a república tem uma única
via para se dissolver, que é se tornar um principado, {[}\ldots{}{]}
(\emph{Discursus rerum florentinarum}, § 11).
\end{quote}

Para além de analisar os meandros argumentativos do opúsculo
maquiaveliano, o trecho citado corrobora as duas partes da nossa
hipótese. Já nos era claro que a república corrompida tem a
possibilidade de se transformar em um principado civil. Agora vemos o
próprio Maquiavel enunciar a possibilidade de um principado se
transformar em uma república. Portanto, conforme o pensamento
maquiaveliano, a transição de um principado para uma república é uma
possibilidade.

Acerca dessas transições, há duas coisas a ressaltar: em primeiro lugar,
em ambos os casos estamos tratando de uma possibilidade e não de uma
certeza de mudança de regime. Nesse sentido, é sempre bom lembrar que
não há nenhum caráter determinista ou finalista no pensamento político
maquiaveliano (no que diz respeito às mudanças de regime, isso
implicaria uma associação com os ciclos políticos de Políbio e sua
inexorabilidade). Um segundo aspecto é que Maquiavel não declara, no seu
opúsculo, que esse principado que se transforma em república é um
principado civil. Na verdade, ele se furta em fazer essa análise sobre o
principado, coisa que não ocorrerá com a república. Sobre o principado,
ele apenas tece algumas considerações sobre a condição do povo, se há
igualdade política ou desigualdade política.

Contudo, a mensagem reiterada ao papa é de que, mesmo que se instale um
principado verdadeiro, este deve ter em vista a inserção dos diversos
grupos políticos no governo da cidade, criando uma espécie de ``governo
\emph{largo}'', ou seja, um principado composto de várias magistraturas,
com a participação dos vários extratos políticos. Tal ordenação
prepararia o povo (preparar a matéria) com vistas à fundação de fato de
uma república na cidade, cuja medida principal consistiria numa
redistribuição dos cargos políticos entre os diversos grupos (diversos
humores) e a reabertura do Conselho Maior, que era o espaço político
onde todos podiam tomar parte e que foi fechado pelo governo Medici
instaurado em 1513. Note-se que Maquiavel não defende de imediato a
instalação de uma república, mas um principado que prepare o povo para o
regime republicano. Preparação essa que passa fundamentalmente pela
recuperação dos valores cívicos, o \emph{vivere civile}, que culminará
na fundação de uma república. Ora, isso é quase o que literalmente ele
apresenta nos \emph{Discursos}, no capítulo \versal{IX} e \versal{X} do livro \versal{I}, conforme
foi visto. O príncipe invocado no texto republicano comparece no
opúsculo de igual modo; em ambos os casos, as ações do príncipe
governante direcionam-se no sentido de recuperar a civilidade degradada
ou mesmo perdida. Modos e procedimentos esses que ficam muito claros em
\emph{O Príncipe}.

Portanto, é muito plausível pensar que a república corrompidíssima
encontra sua melhor solução de recuperação no principado civil,
evitando, ainda que precariamente, a instalação de uma tirania ou de um
regime absoluto. Principado esse que deve ter por finalidade a
preparação do povo para a volta ao regime republicano, desde que, e
devemos enfatizar esse aspecto, o príncipe aja com vistas a recuperar a
dinâmica dos conflitos políticos que expressam a saúde e a vitalidade da
cidade. Então, não é sempre que uma república se transforma em um
principado civil e não é necessário que um principado se transforme em
uma república. Em todos os casos há sempre a possibilidade dessas
mudanças, mas não como necessidade, e sim conforme as ações dos atores
políticos, no caso o príncipe, o povo e os grandes.

Finalmente, ressalte-se a coerência da reflexão política maquiaveliana,
possível de ser percebida tanto quando tomamos seus diferentes textos
como quando levamos em conta apenas um deles. A percepção de tal
coerência permite compreender quais são os pontos fundamentais da sua
reflexão política e as suas convicções teóricas mais caras. Unidade que
supera as possíveis objeções advindas de problemas relacionados à
cronologia das obras, e que passa a ser o dado mais relevante para a
afirmação de que, independentemente de qual obra tenha sido escrita
antes ou depois, a reflexão maquiaveliana se mostra coerente e
articulada. Enfim, pelo exposto, parece ser de todo evidente que \emph{O
Príncipe} de Nicolau Maquiavel não é um texto em defesa da monarquia,
mas uma obra que busca compreender as dinâmicas políticas, em perfeita
consonância com sua reflexão republicana.

\chapter*{}
\addcontentsline{toc}{chapter}{O príncipe}
\begin{center}
\begin{vplace}[0.3]
\Large
O príncipe
\end{vplace}
\end{center}
\thispagestyle{empty}

\begin{Parallel}[p]{}{} 
\ParallelLText{\selectlanguage{italian} \section{MAGNIFICO LAURENTIO MEDICI SALUTEM}

{[}1{]} Sogliono el più delle volte coloro che desiderando acquistar
grazia appresso uno principe farsegli incontro con quelle cose che infra
le loro abbino più care o delle quali vegghino lui più dilettarsi; donde
si vede molte volte essere loro presentati cavagli, arme, drappi d'oro,
pietre preziose e simili ornamenti degni della grandezza di quelli.
{[}2{]} Desiderando io adunque offerirmi alla vostra Magnificenzia con
qualche testimone della servitù mia verso di quella, non ho trovando,
intra la mia supellettile, cosa quale io abbia più cara o tanto existimi
quanto la cognizione delle actioni delli uomini grandi, imparata da me
con una lunga experienza delle cose moderne et una continua lectione
delle antiche; le quali avendo io con gran diligenzia lungamente
excogitate et examinate, et ora in uno piccolo volume ridotte, mando
alla Magnificenzia vostra. {[}3{]} E benché io iudichi questa opera
indegna della presenza di quella, tamen confido assai che per sua
umanità gli debba essere accepta, considerato come da me non gli possa
essere fatto maggiore dono che darle facultà a potere in brevissimo
tempo intendere tutto quello che io, in tanti anni e con tanti mia
disagi e pericoli, ho conosciuto et inteso. {[}4{]} La qual opera io non
ho ornata né ripiena di clausule ample o di parole ampullose e
magnifiche o di qualunque altro lenocinio o ornamento extrinseco, con li
quali molti sogliono le loro cose descrivere et ornare, perché io ho
voluto o che veruna cosa la onori o che solamente la varietà della
materia e la gravitá del subietto la facci grata. {[}5{]} Né voglio sia
imputata prosumptione se uno uomo di basso et infimo stato ardisce
discorrere e regolare e governi de' principi; perché, così come coloro
che disegnano e paesi si pongono bassi nel piano a considerare la natura
de' monti e de' luoghi alti e, per considerare quella de' luoghi bassi,
si pongono alto sopra' monti, similmente, a conoscere bene la natura de'
populi, bisogna essere principe et, a conoscere bene quella de'
principi, conviene essere populare.

{[}6{]} Pigli adunque vostra Magnificenzia questo piccolo dono con
quello animo che io 'l mando; il quale se da quella fia diligentemente
considerato e letto, vi conoscerà dentro uno extremo mio desiderio che
lei pervenga a quella grandezza che la fortuna e l'altre sua qualità le
promettano. {[}7{]} E se vostra Magnificenzia dallo apice della sua
altezza qualche volta volgerà li occhi in questi luoghi bassi, conoscerà
quanto io indegnamente sopporti una grande e continua malignità di
fortuna.

\setcounter{secnumdepth}{2}

\quebra\section{QUOT SINT GENERA PRINCIPATUUM ET QUIBUS MODIS ACQUIRANTUR
{[}Di quante ragioni sieno e'principati e in che modo si acquistino.{]}}

{[}1{]} Tutti gli stati, tutti e' dominii che hanno avuto et hanno
imperio sopra gli uomini, sono stati e sono o republiche o principati.
{[}2{]} E principati sono o ereditarii, de' quali el sangue del loro
signore ne sia suto lungo tempo principe, o sono nuovi. {[}3{]} E nuovi,
o e' sono nuovi tutti, come fu Milano a Francesco Sforza, o sono come
membri aggiunti allo stato ereditario del principe che gli acquista,
come è el regno di Napoli a re di Spagna. {[}4{]} Sono questi dominii
così acquistati o consueti a vivere sotto uno principe o usi ad essere
liberi; et acquistonsi o con l'arme d'altri o con le proprie, o per
fortuna o per virtù.

\quebra\section{DE PRINCIPATIBUS HEREDITARIIS
{[}De' principati ereditari.{]}}

{[}1{]} Io lascerò indrieto il ragionare delle repubbliche, perché altra
volta ne ragionai a lungo. {[}2{]} Volterommi solo al principato et
andrò ritexendo gli orditi soprascripti, e disputerò come questi
principati si possino governare e mantenere.

{[}3{]} Dico adunque che, nelli stati ereditari et assuefatti al sangue
del loro principe, sono assai minore difficultà a mantenergli che ne'
nuovi, perché basta solo non preterire gli ordini de' sua antinati e
dipoi temporeggiare con gli accidenti; in modo che, se tale principe è
di ordinaria industria, sempre si manterrà nel suo stato, se non è una
extraordinaria et excessiva forza che ne lo privi: e privato che ne fia,
quantunque di sinistro abbi l'occupatore, lo riacquista.

{[}4{]} Noi abbiamo in Italia, in exemplis, el duca di Ferrara, il quale
non ha retto alli assalti de' Viniziani nell' ottantaquatro, né a quelli
di papa Iulio nel dieci, per altre

cagioni che per essere antiquato in quello dominio. {[}5{]} Perché el
principe naturale ha minori cagioni e minore necessità di offendere,
donde conviene che sia più amato; e se extraordinarii vizii non lo fanno
odiare, è ragionevole che naturalmente sia benevoluto dalli sua. {[}6{]}
E nella antiquità e continuazione del dominio sono spente le memorie e
le cagioni delle innovazioni: perché sempre una mutazione lascia lo
adentellato per la edificazione dell'altra.

\quebra\section{DE PRINCIPATIBUS MIXTIS
{[}De' principati misti{]}}

{[}1{]} Ma nel principato nuovo consistono le difficultà. E prima, -- se
non è tutto nuovo, ma come membro: che si può chiamare tutto insieme
quasi mixto,-- le variazioni sue nascono im prima da una naturale
difficultà, quale è in tutti li principati nuovi: le quali sono che li
uomini mutano volentieri signore, credendo migliorare, e questa credenza
li fa pigliare l'arme contro a quello: di che s'ingannano, perché
veggono poi per experienzia avere piggiorato. {[}2{]} Il che depende da
un'altra necessità naturale et ordinaria, quale fa che sempre bisogni
offendere quegli di chi si diventa nuovo principe e con gente d'arme e
con infinite altre ingiurie che si tira drietro il nuovo acquisto:
{[}3{]} di modo che tu hai nimici tutti quegli che hai offesi in
occupare quello principato, e non ti puoi mantenere amici quelli che vi
ti hanno messo, per non gli potere satisfare in quel modo che si erano
presupposti e per non potere tu usare contro di loro medicine forte,
sendo loro obligato; perché sempre, ancora che uno sia fortissimo in
sulli exerciti, ha bisogno del favore de' provinciali ad entrare in una
provincia. {[}4{]} Per queste ragioni Luigi XII re di Francia occupò
subito Milano e subito lo perdé; e bastò a torgnene, la prima volta, le
forze proprie di Ludovico: perché quegli populi che gli avevano aperte
le porte, trovandosi ingannati della opinione loro e di quello futuro
bene che si avevano presupposto, non potevano sopportare e fastidi del
nuovo principe.

{[}5{]} Bene è vero che, acquistandosi poi la seconda volta, e paesi
ribellati si perdono con più difficultà: perché el signore, presa
occasione dalla ribellione, è meno respettivo ad assicurarsi con punire
e delinquenti, chiarire e sospetti, provedersi nelle parte più deboli.
{[}6{]} In modo che, se a fare perdere Milano a Francia bastò la prima
volta uno duca Ludovico che rumoreggiassi in su' confini, a farlo dipoi
perdere la seconda gli bisognò avere contro tutto il mondo e che gli
exerciti sua fussino spenti o fugati di Italia: il che nacque dalle
cagioni sopraddette. {[}7{]} Nondimanco, e la prima e la seconda volta
gli fu tolto: le cagioni universali della prima si sono discorse; resta
ora a dire quelle della seconda e vedere che rimedi lui ci aveva e quali
ci può avere uno che fussi nelli termini sua, per potere meglio
mantenersi nello acquisto che non fece Francia.

{[}8{]} Dico pertanto che questi stati, quali acquistandosi si
aggiungano a uno stato antico di quello che acquista, o ei sono della
medesima provincia e della medesima lingua, o non sono. {[}9{]} Quando
sieno, è facilità grande a tenerli, maxime quando non sieno usi a vivere
liberi: et a possederli sicuramente basta avere spenta la linea del
principe che gli dominava, perché, nelle altre cose mantenendosi loro le
condizioni vecchie e non vi essendo disformità di costumi, gli uomini si
vivono quietamente; come si è visto che ha fatto la Borgogna, la
Brettagna, la Guascogna e la Normandia, che tanto tempo sono state con
Francia: e benché vi sia qualche disformità di lingua, nondimeno li
costumi sono simili e possonsi infra loro facilmente comportare.
{[}10{]} E chi le acquista, volendole tenere, debba avere dua respetti:
l'uno, che el sangue del loro principe antico si spenga; l'altro, di non
alterare né loro legge né loro dazii: talmente che in brevissimo tempo
diventa con il loro principato antiquo tutto uno corpo.

{[}11{]} Ma quando si acquista stati in una provincia disforme di
lingua, di costumi e di ordini, qui sono le difficultà e qui bisogna
avere gran fortuna e grande industria a tenerli. {[}12{]} Et uno de'
maggiori remedii e più vivi sarebbe che la persona di chi acquista vi
andassi ad abitare; questo farebbe più sicura e più durabile quella
possessione, come ha fatto il Turco di Grecia: il quale, con tutti li
altri ordini observati da lui per tenere quello stato, se non vi fussi
ito ad abitare non era possibile che lo tenessi. {[}13{]} Perché
standovi si veggono nascere e disordini e presto vi puo' rimediare: non
vi stando, s'intendono quando sono grandi e che non vi è più rimedio;
non è oltre a questo la provincia spogliata da' tua offiziali;
satisfannosi e subditi del ricorso propinquo al principe, donde hanno
più cagione di amarlo, volendo essere buoni, e, volendo essere
altrimenti, di temerlo; chi delli externi volessi assaltare quello
stato, vi ha più respecto; tanto che, abitandovi, lo può con grandissima
difficultà perdere.

{[}14{]} L'altro migliore remedio è mandare colonie in uno o in dua
luoghi, che sieno quasi compedes di quello stato: perché è necessario o
fare questo o tenervi assai gente d'arme e fanti. {[}15{]} Nelle colonie
non si spende molto; e sanza sua spesa, o poca, ve le manda e tiene, e
solamente offende coloro a chi toglie e campi e le case per darle a'
nuovi abitatori, che sono una minima parte di quello stato; {[}16{]} e
quegli che gli offende, rimanendo dispersi e poveri, non gli possono mai
nuocere; e tutti li altri rimangono da uno canto inoffesi, -- e per
questo doverrebbono quietarsi, -- dall'altro paurosi di non errare, per
timore che non intervenissi a loro come a quelli che sono stati
spogliati. {[}17{]} Concludo che queste colonie non costono, sono più
fedeli, offendono meno, e li offesi non possono nuocere, sendo poveri e
dispersi, come è detto. {[}18{]} Per che si ha a notare che gli uomini
si debbono o vezzeggiare o spegnere; perché si vendicano delle leggieri
offese, delle gravi non possono: sì che la offesa che si fa' l'uomo
debbe essere in modo che la non tema la vendetta. {[}19{]} Ma tenendovi,
in cambio di colonie, gente d'arme, spende più assai, avendo a consumare
nella guardia tutte le intrate di quello stato, in modo che l'acquisto
gli torna perdita; e offende molto più, perché nuoce a tutto quello
stato, tramutando con li alloggiamenti il suo exercito; del quale
disagio ognuno ne sente e ciascuno gli diventa nimico: e sono nimici che
gli possono nuocere, rimanendo battuti in casa loro. {[}20{]} Da ogni
parte dunque questa guardia è inutile, come quella delle colonie è
utile.

{[}21{]} Debbe ancora chi è in una provincia disforme, come è detto,
farsi capo e defensore de' vicini minori potenti, et ingegnarsi di
indebolire e potenti di quella, e guardarsi che per accidente alcuno non
vi entri uno forestiere potente quanto lui: e sempre interverrà ch'e' vi
sarà messo da coloro che saranno in quella malcontenti o per troppa
ambizione o per paura, come si vidde già che gli Etoli missono e Romani
in Grecia, et, in ogni altra provincia che gli entrorno, vi furono messi
da' provinciali. {[}22{]} E l'ordine delle cose è che, subito che uno
forestiere potente entra in una provincia, tutti quelli che sono in espa
meno potenti gli aderiscano, mossi da una invidia hanno contro a chi è
suto potente sopra di loro: tanto che, respetto a questi minori potenti,
lui non ha a durare fatica alcuna a guadagnarli, perché subito tutti
insieme volentieri fanno uno globo col suo stato che lui vi ha
acquistato. {[}23{]} Ha solamente a pensare che non piglino troppe forze
e troppa autorità, e facilmente può con le forze sua e col favore loro
sbassare quelli che sono potenti, per rimanere in tutto arbitro di
quella provincia; e chi non governerà bene questa parte, perderà presto
quello che arà acquistato e, mentre lo terrà, vi arà dentro infinite
difficultà e fastidii.

{[}24{]} E Romani, nelle provincie che pigliorno, observorno bene queste
parte: e' mandorno le colonie, intrattennono e meno potenti sanza
crescere loro potenza, abbassorno e potenti, e non vi lasciorno prendere
riputazione a' potenti forestieri. {[}25{]} E voglio mi basti solo la
provincia di Grecia per esemplo: furono intrattenuti da lloro gli Achei
e gli Etoli, fu abbassato il regno de' Macedoni, funne cacciato Antioco;
né mai e'meriti degli Achei o delli Etoli feciono che permettessino loro
accrescere alcuno stato, né le persuasioni di Filippo gl'indussono mai
ad essergli amici sanza sbassarlo, né la potenzia di Antioco possé fare
gli consentissino che tenessi in quella provincia alcuno stato. {[}26{]}
Perché e' romani feciono in questi casi quello che tutti e' principi
savi debbono fare: e' quali non solamente hanno ad avere riguardo alli
scandoli presenti, ma a' futuri, e a quelli con ogni industria ovviare;
perché, prevedendosi discosto, vi si rimediare facilmente, ma,
aspettando che ti si appressino, la medicina non è a tempo, perché la
malattia è diventata incurabile; {[}27{]} e interviene di questa, come
dicono e' fisici dello etico, che nel principio del suo male è facile a
curare e difficile a conoscere: ma nel progresso del tempo, non la
avendo nel principio conosciuta né medicata, diventa facile a conoscere
e difficile a curare. {[}28{]} Così interviene nelle cose di stato:
perché conoscendo discosto, il che non è dato se non a uno prudente, e'
mali che nascono in quello si guariscono presto; ma quando, per non gli
avere conosciuti, si lasciono crescere in modo che ognuno gli conosce,
non vi è più rimedio.

{[}29{]} Però e romani, veggendo discosto gl'inconvenienti, vi
rimediorno sempre, e non gli lasciorno mai seguire per fuggire una
guerra, perché sapevano che la guerra non si lieva, ma si differisce a
vantaggio di altri: però vollono fare con Filippo e Antioco guerra in
Grecia, per non la avere a fare con loro in Italia; e potevono per
allora fuggire l'una e l'altra: il che non vollono. {[}30{]} Né piacque
mai loro quello che è tutto dí in bocca de' savi de' nostri tempi, di
godere il benefizio del tempo, ma sì bene quello della virtù e prudenza
loro: perché il tempo si caccia innanzi ogni cosa, e può condurre seco
bene come male e male come bene.

{[}31{]} Ma torniamo a Francia ed esaminiamo se delle cose dette e' ne
ha fatte alcuna: e parlerò di Luigi, e non di Carlo, come di colui che,
per aver tenuta più lunga possessione in Italia, si sono meglio visti e'
sua progressi: e vedrete come egli ha fatto il contrario di quelle cose
che si debbono fare per tenere uno stato in una provincia disforme.
{[}32{]} El re Luigi fu messo in Italia da la ambizione de' viniziani,
che vollono guadagnarsi mezzo lo stato di Lombardia per quella venuta.
{[}33{]} Io non voglio biasimare questo partito preso dal re: perché,
volendo cominciare a mettere uno piè in Italia e non avendo in questa
provincia amici, anzi sendogli per li portamenti del re Carlo serrate
tutte le porte, fu necessitato prendere quelle amicizie che poteva; e
sarebbegli riuscito el partito bene preso, quando nelli altri maneggi
non avessi fatto alcuno errore. {[}34{]} Acquistata adunque el re la
Lombardia, subito si riguadagnò quella reputazione che gli aveva tolta
Carlo: Genova cedé; fiorentini gli diventorono amici; marchese di
Mantova, duca di Ferrara, Bentivogli, Madonna di Furlí, signore di
Faenza, di Rimini, di Pesero , di Camerino, di Piombino, lucchesi,
pisani, sanesi, ognuno se gli fece incontro per essere suo amico.
{[}35{]} E allora poterno considerare e' viniziani la temerità del
partito preso da loro, e' quali, per acquistare dua terre in Lombardia,
feciono signore el re de' dua terzi di Italia.

{[}36{]} Consideri ora uno con quanta poca difficultà poteva el re
tenere in Italia la sua reputazione, se lui avessi osservate le regule
soprascritte e tenuti sicuri e difesi tutti quelli sua amici, e' quali,
per essere gran numero e deboli e paurosi chi della Chiesa chi de'
viniziani, erano sempre necessitati a stare seco; e per il mezzo loro
poteva facilmente assicurarsi di chi ci restava grande. {[}37{]} Ma lui
non prima fu in Milano che fece il contrario, dando aiuto a papa
Alessandro perché egli occupassi la Romagna; né si accorse, con questa
deliberazione, che faceva sé debole, togliendosi gli amici e quegli che
se gli erano gittati in grembo, e la Chiesa grande, aggiugnendo allo
spirituale, che le dà tanta autorità, tanto temporale. {[}38{]} E fatto
un primo errore fu constretto a seguitare: in tanto che, per porre
termine alla ambizione di Alessandro e perché e' non divenissi signore
di Toscana, e' fu constretto venire in Italia.

{[}39{]} Non gli bastò avere fatto grande la Chiesa e toltosi gli amici:
che, per volere il regno di Napoli, lo divise con il re di Spagna; e
dove egli era prima arbitro di Italia, vi misse uno compagno, acciò che
gli ambiziosi di quella provincia e malcontenti di lui avessino dove
ricorrere; e dove posseva lasciare in quel regno uno re suo pensionario,
e' ne lo trasse per mettervi uno che potessi cacciarne lui. {[}40{]} È
cosa veramente molto naturale e ordinaria desiderare di acquistare: e
sempre, quando li uomini lo fanno, che possano, saranno laudati o non
biasimati; ma quando e' non possono, e vogliono farlo in ogni modo, qui
è lo errore et il biasimo. {[}41{]} Se Francia adunque posseva con le
sue forze sua assaltare Napoli, doveva farlo: s'e' non poteva, non
doveva dividerlo; e se la divisione fece co' viniziani di Lombardia
meritò scusa, per avere con quella messo el piè in Italia, questa merita
biasimo, per non essere scusata da quella necessità.

{[}42{]} Aveva dunque fatto Luigi questi cinque errori: spenti e minori
potenti; accresciuto in Italia potenza a uno potente; messo in quella
uno forestiere potentissimo; non venuto ad abitarvi; non vi messo
colonie. {[}43{]} E' quali errori ancora, vivendo lui, potevono non lo
offendere, se non avessi fatto il sesto, di tòrre lo stato a' viniziani.
{[}44{]} Perché, quando e' non avessi fatto grande la Chiesa né messo in
Italia Spagna, era bene ragionevole e necessario abbassargli; ma avendo
preso quegli primi partiti, non doveva mai consentire alla ruina loro:
perché, sendo quegli potenti, sempre arebbono tenuti gli altri discosto
da la impresa di Lombardia, sì perché e' viniziani non vi arebbono
consentito sanza diventarne signori loro, sì perché li altri non
arebbono voluto torla a Francia per darla a loro: e andare a urtarli
tutti a dua non arebbono avuto animo.

{[}45{]} E se alcuno dicessi: el re Luigi cedé ad Alessandro la Romagna
e a Spagna el Regno per fuggire una guerra; rispondo con le ragioni
dette di sopra, che non si debbe mai lasciare seguire uno disordine per
fuggire una guerra: perché la non si fugge, ma si differisce a tuo
disavvantaggio. {[}46{]} E se alcuni altri allegassino la fede che il re
aveva data al papa, di fare per lui quella impresa per la resoluzione
del suo matrimonio e il cappello di Roano, rispondo con quello che per
me di sotto si dirà circa alla fede de' principi e come la si debbe
osservare.

{[}47{]} Ha perduto adunque el re Luigi la Lombardia per non avere
osservato alcuno di quelli termini osservati da altri che hanno preso
provincie e volutole tenere; né è miraculo alcuno questo, ma molto
ordinario e ragionevole. {[}48{]} E di questa materia parlai a Nantes
con Roano, quando el Valentino, -- che così era chiamato popularmente
Cesare Borgia, figliuolo di papa Alessandro, -- occupava la Romagna;
perché, dicendomi el cardinale di Roano che gli italiani non si
intendevano della guerra, io gli risposi che ' franzesi non si
intendevano dello stato: perché, s'e' se ne 'ntendessino, non
lascerebbono venire in tanta grandezza la Chiesa. {[}49{]} E per
esperienza si è visto che la grandezza in Italia di quella e di Spagna è
stata causata da Francia, e la ruina sua è suta causata da loro.
{[}50{]} Di che si trae una regula generale, la quale mai o raro falla,
che chi è cagione che uno diventi potente, ruina: perché quella potenza
è causata da colui o con industria o con forza, e l'una e l'altra di
queste dua è sospetta a chi è divenuto potente.

\quebra\section{CUR DARII REGNUM, QUOD ALEXANDER OCCUPAVERAT, A SUCCESSORIBUS SUIS POST
ALEXANDRI MORTEM NON DEFECIT
{[}Per quale cagione el regno di Dario, il quale da Alessandro fu
occupato, non si rebellò da' sua successori dopo la morte di
Alessandro{]}}

{[}1{]} Considerate le difficultà le quali s'hanno a tenere uno stato
occupato di nuovo, potrebbe alcuno maravigliarsi donde nacque che
Alessandro Magno diventò signore della Asia in pochi anni e, non la
avendo appena occupata, morí: donde pareva ragionevole che tutto quello
stato si ribellassi; nondimeno e' successori di Alessandro se lo
mantennono e non ebbono, a tenerlo, altra difficultà che quella che
infra loro medesimi per propria ambizione nacque. {[}2{]} Rispondo come
e' principati de' quali si ha memoria si truovono governati in dua modi
diversi: o per uno principe e tutti li altri servi, e' quali come
ministri, per grazia e concessione sua, aiutono governare quello regno;
o per uno principe e per baroni e' quali non per grazia del signore, ma
per antichità di sangue, tengono quel grado. {[}3{]} Questi tali baroni
hanno stati e sudditi propri, e' quali gli ricognoscono per signori et
hanno in loro naturale affezione. {[}4{]} Quelli stati che si governano
per uno principe e per servi hanno el loro principe con più autorità,
perché in tutta la sua provincia non è uomo che riconosca alcuno per
superiore se non lui; e se ubbidiscono alcuno altro, lo fanno come
ministro e offiziale; e a lui portano particulare amore.

{[}5{]} Li esempli di queste dua diversità di governi sono, ne' nostri
tempi, el Turco et il re di Francia. {[}6{]} Tutta la monarchia del
Turco è governata da uno signore: li altri sono sua servi; e
distinguendo il suo regno in sangiacchie, vi manda diversi
amministratori e gli muta e varia come pare a lui. {[}7{]} Ma il re di
Francia è posto in mezzo di una moltitudine antiquata di signori, in
quello stato riconosciuti da' loro sudditi e amati da quegli: hanno le
loro preminenze, non le può il re tòrre loro sanza suo periculo. {[}8{]}
Chi considera adunque l'uno e l'altro di questi stati, troverrà
difficultà nell'acquistare lo stato del Turco, ma, vinto che fia,
facilità grande a tenerlo. {[}9{]} Così per avverso troverrà per qualche
respetto più facilità a potere occupure il regno di Francia, ma
difficultà grande a tenerlo.

{[}10{]} Le cagioni delle difficultà, in potere occupare il regno del
Turco, sono per non potere essere chiamato da' principi di quell regno,
né sperare, con la rebellione di quegli che gli ha d'intorno, potere
facilitare la tua impresa; il che nasce da le ragioni sopraddette:
perché, sendogli tutti stiavi e obligati, si possono con più difficultà
corrompere e, quando bene si corrompessino, se ne può sperare poco
utile, non potendo quelli tirarsi drieto e' populi per le ragioni
assegnate. {[}11{]}Onde a chi assalta el Turco è necessario pensare di
averlo a trovare unito, e gli conviene sperare più nelle forze proprie
che ne' disordini di altri. {[}12{]} Ma vinto ch'e' fussi, e rotto alla
campagna in modo che non possa rifare exerciti, non si ha a dubitare di
altro che del sangue del principe: el quale spento, non resta alcuno di
chi si abbia a temere, non avendo gli altri credito con e' populi; e
come el vincitore avanti la vittoria non poteva sperare in loro, così
non debba dopo quella temere di loro.

{[}13{]} Al contrario interviene ne' regni governati come quello di
Francia: perché con facilità tu puoi entrarvi guadagnandoti alcuno
barone del regno, perché sempre si truova de' mali contenti e di quegli
che desiderano innovare. {[}14{]} Costoro per le ragioni dette ti
possono aprire la via a quello stato e facilitarti la vittoria: la quale
di poi, a volerti mantenere, si tira drieto infinite difficultà e con
quelli che ti hanno aiutato e con quelli che tu hai oppressi. {[}15{]}
Né ti basta spegnere el sangue del principe, perché vi rimangono quelli
signori, che si fanno capi delle nuove alterazioni: e non gli potendo né
contentare né spegnere, perdi quello stato qualunque volta la occasione
venga.

{[}16{]} Ora, se voi considerrete di qual natura di governi era quello
di Dario, lo troverrete simile al regno del Turco: e però ad Alessandro
fu necessario prima urtarlo tutto e tòrgli la campagna. {[}17{]} Dopo la
qual vittoria, sendo Dario morto, rimase ad Alessandro quello stato
sicuro per le ragioni di sopra discorse; ed e' sua successori, se
fussino stati uniti, se lo potevano godere oziosi: né in quello regno
nacquono altri tumulti che quegli che loro propri sucitorno. {[}18{]} Ma
gli stati ordinati come quello di Francia è impossibile possederli con
tanta quiete. {[}19{]} Di qui nacquono le spesse ribellioni di Spagna,
di Francia e di Grecia da' Romani, per gli spessi principati che erano
in quelli stati: de' quali mentre durò la memoria, sempre fu Roma
incerta di quella possessione. {[}20{]} Ma spenta la memoria di quelli,
con la potenza e diuturnità dello imperio, ne diventorno sicuri
possessori: e poterno anche quelli di poi, combattendo infra loro,
ciascuno tirarsi dreto parte di quelle provincie secondo l'autorità vi
aveva presa dentro; e quelle, per essere e' sangui de' loro antiqui
signori spenti, non riconoscevano se non e' romani. {[}21{]} Considerato
adunque tutte queste cose, non si maraviglierà alcuno della facilità
ebbe Alessandro a tenere lo stato di Asia, e delle difficultà che hanno
avuto gli altri a conservare lo acquistato, come Pirro e molti: il che
non è nato da la poca o da la molta virtù del vincitore, ma da la
disformità del subietto.

\quebra\section{QUOMODO ADMINISTRANDAE SUNT CIVITATES VEL PRINCIPATUS, QUI, ANTEQUAM
OCCUPARENTUR SUIS LEGIBUS VIVEBANT
{[}In che modo si debbino governare le città o principati li quali,
innanzi fussino occupati, si vivevano con le loro legge.{]}}

{[}1{]} Quando quelli stati, che si acquistano come è detto, sono
consueti a vivere con le loro leggi e in libertà, a volergli tenere ci
sono tre modi: {[}2{]} il primo, ruinarle; l'altro, andarvi ad abitare
personalmente; il terzo, lasciàgli vivere con le sua legge, traendone
una pensione e creandovi dentro uno stato di pochi, che te le conservino
amico: {[}3{]} perché, sendo quello stato creato da quello principe, sa
che non può stare sanza l'amicizia e potenza sua e ha a fare tutto per
mantenerlo; e più facilmente si tiene una città usa a vivere libera con
il mezzo de' sua cittadini che in alcuno altro modo, volendola
perservare.

{[}4{]} In exemplis ci sono gli spartani ed e' romani. Gli spartani
tennono Atene e Tebe creandovi uno stato di pochi, tamen le riperderno.
{[}5{]} E' romani, per tenere Capua Cartagine e Numanzia, le disfeciono,
e non le perderno; vollono tenere la Grecia quasi come tennono gli
spartani, faccendola libera e lasciandole le sua legge, e non successe
loro: tale che furono constretti disfare di molte città di quella
provincia per tenerla. {[}6{]} Perché in verità non ci è modo sicuro a
possederle altro che la ruina; e chi diviene patrone di una città
consueta a vivere libera, e non la disfaccia, aspetti di essere disfatto
da quella: perché sempre ha per refugio, nella rebellione el nome della
libertà e gli ordini antiqui sua, e' quali né per la lunghezza di tempo
né per benefizi mai si dimenticano. {[}7{]} E per cosa che si faccia o
si provegga, se non si disuniscano o dissipano gli abitatori non
dimenticano quello nome né quegli ordini, e subito in ogni accidente vi
ricorrono; come fe' Pisa dopo cento anni che la era suta posta in
servitù da' fiorentini. {[}8{]} Ma quando le città o le provincie sono
use a vivere sotto uno principe e quello sangue sia spento, sendo da uno
canto usi a ubbidire, da l'altro non avendo il principe vecchio, farne
uno infra loro non si accordano, vivere liberi non sanno: di modo che
sono più tardi a pigliare l'arme e con più facilità se gli può uno
principe guadagnare et assicurarsi di loro.

{[}9{]} Ma nelle repubbliche è maggiore vita, maggiore odio, più
desiderio di vendetta: né gli lascia, né può lasciare, riposare la
memoria della antiqua libertà; tale che la più sicura via è spegnerle, o
abitarvi.

\quebra\section{DE PRINCIPATIBUS NOVIS QUI ARMIS PROPRIIS ET VIRTUTE ACQUIRUNTUR.
{[}De' Principati nuovi che s'acquistano con l'arme proprie e virtuosamente{]}}

{[}1{]} Non si maravigli alcuno se, nel parlare che io farò de'
principati al tutto nuovi e di principe e di stato, io addurrò
grandissimi esempli. {[}2{]} Perché, camminando gli uomini sempre per le
vie battute da altri e procedendo nelle azioni loro con le imitazioni,
né si potendo le vie d'altri al tutto tenere né alla virtù di quegli che
tu imiti aggiugnere, debbe uno uomo prudente intrare sempre per vie
battute da uomini grandi, e quegli che sono stati eccellentissimi
imitare: acciò che, se la sua virtù non vi arriva, almeno ne renda
qualche odore; {[}3{]} e fare come gli arcieri prudenti, a' quali
parendo el luogo dove desegnano ferire troppo lontano, e conoscendo fino
a quanto va la virtù del loro arco, pongono la mira assai più alta che
il luogo destinato, non per aggiugnere con la loro freccia a tanta
altezza, ma per potere con lo aiuto di sì alta mira pervenire al disegno
loro.

{[}4{]} Dico adunque che ne' principati tutti nuovi, dove sia uno nuovo
principe, si truova a mantenergli più o meno difficultà secondo che più
o meno è virtuoso colui che gli acquista. {[}5{]} E perché questo
evento, di diventare di privato principe, presuppone o virtù o fortuna,
pare che l'una o l'altra di queste dua cose mitighino in parte molte
difficultà; nondimanco, colui che è stato meno in su la fortuna si è
mantenuto più. {[}6{]} Genera ancora facilità essere el principe
constretto, per non avere altri stati, venire personalmente ad abitarvi.

{[}7{]} Ma per venire a quegli che per propria virtù e non per fortuna
sono diventati principi, dico che e' più eccellenti sono Moisè, Ciro,
Romulo, Teseo e simili. {[}8{]} E benché di Moisè non si debba
ragionare, sendo suto uno mero esecutore delle cose che gli erano
ordinate da Dio, tamen debbe essere ammirato, solum per quella grazia
che lo faceva degno di parlare con Dio. {[}9{]} Ma consideriamo Ciro e
li altri che hanno acquistato o fondati regni, gli troverrete tutti
mirabili; e se si considerranno le azioni et ordini loro particulari,
parranno non discrepanti da quegli di Moisè, che ebbe sì grande
precettore. {[}10{]} Ed esaminando le azioni e vita loro non si vede che
quelli avessino altro da la fortuna che la occasione, la quale dette
loro materia a potere introdurvi dentro quella forma che parse loro: e
sanza quella occasione la virtù dello animo loro si sarebbe spenta, e
sanza quella virtù la occasione sarebbe venuta invano.

{[}11{]} Era adunque necessario a Moisè trovare el populo d'Israel in
Egitto stiavo et oppresso dalli egizi, acciò che quegli, per uscire di
servitú, si disponessino a seguirlo. {[}12{]} Conveniva che Romulo non
capessi in Alba, fussi stato esposto al nascere, a volere che diventassi
re di Roma e fondatore di quella patria. {[}13{]} Bisognava che Ciro
trovassi e' persi malcontenti dello imperio de' medi, ed e' medi molli
et effeminati per la lunga pace. {[}14{]} Non poteva Teseo dimostrare la
sua virtù, se non trovava gli ateniesi dispersi. {[}15{]} Queste
occasioni per tanto feciono questi uomini felici e la eccellente virtù
loro fé quella occasione essere conosciuta: donde la loro patria ne fu
nobilitata e diventò felicissima.

{[}16{]} Quelli e' quali per vie virtuose, simili a costoro, diventono
principi, acquistono el principato con difficultà, ma con facilità lo
tengano; e le difficultà che gli hanno nello acquistare el principato
nascono in parte da' nuovi ordini e modi che sono forzati introdurre per
fondare lo stato loro e la loro sicurtà. {[}17{]} E debbesi considerare
come e' non è cosa più difficile a trattare, né più dubia a riuscire, né
più pericolosa a maneggiare, che farsi capo di introdurre nuovi ordini.
{[}18{]} Perché lo introduttore ha per nimico tutti quegli che degli
ordini vecchi fanno bene, e ha tiepidi defensori tutti quelli che delli
ordini nuovi farebbono bene: la quale tepidezza nasce parte per paura
delli avversari, che hanno le leggi dal canto loro, parte da la
incredulità degli uomini, e' quali non credano in verità le cose nuove,
se non ne veggono nata una ferma esperienza. {[}19{]} Donde nasce che,
qualunque volta quelli che sono nimici hanno occasione di assaltare, lo
fanno partigianamente, e quelli altri difendano tiepidamente: in modo
che insieme con loro si periclita.

{[}20{]} È necessario pertanto, volendo discorrere bene questa parte,
esaminare se questi innovatori stanno per loro medesimi o se dependano
da altri: cioè, se per condurre l'opera loro bisogna che preghino, o
vero possono forzare. {[}21{]} Nel primo caso, sempre capitano male e
non conducono cosa alcuna; ma quando dependono da loro propri e possano
forzare, allora è che rare volte periclitano: di qui nacque che tutti e'
profeti armati vinsono ed e' disarmati ruinorno. {[}22{]} Perché, oltre
alle cose dette, la natura de' populi è varia ed è facile a persuadere
loro una cosa, ma è difficile fermargli in quella persuasione: e però
conviene essere ordinato in modo che, quando non credono più, si possa
fare loro credere per forza. {[}23{]} Moisè, Ciro, Teseo e Romulo non
arebbono potuto fare osservare loro lungamente le loro constituzioni, se
fussino stati disarmati; come ne' nostri tempi intervenne a fra Ieronimo
Savonerola, il quale ruinò ne' sua ordini nuovi, come la moltitudine
cominciò a non credergli, e lui non aveva modo a tenere fermi quelli che
avevano creduto né a fare credere e' discredenti. {[}24{]} Però questi
tali hanno nel condursi grande difficultà, e tutti e' loro pericoli sono
fra via e conviene che con la virtù gli superino. {[}25{]} Ma superati
che gli hanno, e che cominciano a essere in venerazione, avendo spenti
quegli che di sua qualità gli avevano invidia, rimangono potenti,
sicuri, onorati e felici.

{[}26{]} A sì alti esempli io voglio aggiugnere uno exemplo minore; ma
bene arà qualche proporzione con quegli, e voglio mi basti per tutti gli
altri simili: e questo è Ierone siracusano. {[}27{]} Costui di privato
diventò principe di Siracusa; né ancora lui conobbe altro da la fortuna
che la occasione; perché, sendo e' siracusani oppressi, lo elessono per
loro capitano; donde meritò di essere fatto loro principe. {[}28{]} E fu
di tanta virtù, etiam in privata fortuna, che chi ne scrive dice
\emph{quod nihil illi deerat ad regnandum praeter regnum.} {[}29{]}
Costui spense la milizia vecchia, ordinò della nuova; lasciò le amicizie
antiche, prese delle nuove; e come ebbe amicizie e soldati che fussino
sua, possé in su tale fondamento edificare ogni edifizio, tanto che lui
durò assai fatica in acquistare e poca in mantenere.

\quebra\section{DE PRINCIPATIBUS NOVIS QUI ALIENIS ARMIS ET FORTUNA ACQUIRUNTUR.
{[}De' principati nuovi che s'acquistano con le armi e fortuna di altri{]}}

{[}1{]} Coloro e' quali solamente per fortuna diventano di privati
principi, con poca fatica diventono, ma con assai si mantengono; e non
hanno alcuna difficultà fra via, perché vi volano: ma tutte le
difficultà nascono quando e' sono posti. {[}2{]} E questi tali sono
quando è concesso ad alcuno uno stato o per danari o per grazia di chi
lo concede: come intervenne a molti in Grecia nelle città di Ionia e di
Ellesponto, dove furno fatti principi da Dario, acciò le tenessino per
sua sicurtà e gloria; come erano fatti ancora quelli imperatori che di
privati, per corruzione de' soldati, pervenivano allo imperio.

{[}3{]} Questi stanno semplicemente in su la volontà e fortuna di chi lo
ha concesso loro, che sono dua cose volubilissime e instabili, e non
sanno e non possano tenere quello grado: non sanno, perché s'e' non è
uomo di grande ingegno e virtù, non è ragionevole che, sendo vissuto
sempre in privata fortuna, sappia comandare; non possono, perché non
hanno forze che gli possino essere amiche e fedele. {[}4{]} Di poi gli
stati che vengano subito, come tutte l'altre cose della natura che
nascono e crescono presto, non possono avere le barbe e correspondenzie
loro in modo che il primo tempo avverso non le spenga, se già quelli
tali -- come è detto -- che sì de repente sono diventati principi non
sono di tanta virtù che quello che la fortuna ha messo loro in grembo e'
sappino subito prepararsi a conservarlo, e quelli fondamenti, che gli
altri hanno fatti avanti che diventino principi, gli faccino poi.

{[}5{]} Io voglio all'uno e l'altro di questi modi detti, circa il
diventare principe per virtù o per fortuna, addurre dua esempli stati
ne' dí della memoria nostra: e questi sono Francesco Sforza e Cesare
Borgia. {[}6{]} Francesco, per li debiti mezzi e con una grande sua
virtù, di privato diventò duca di Milano; e quello che con mille affanni
aveva acquistato, con poca fatica mantenne. {[}7{]} Da l'altra parte,
Cesare Borgia, chiamato dal vulgo duca Valentino, acquistò lo stato con
la fortuna del padre e con quella lo perdé, non obstante che per lui si
usassi ogni opera e facessinsi tutte quelle cose che per uno prudente e
virtuoso uomo si doveva fare per mettere le barbe sua in quelli stati
che l'arme e fortuna di altri li aveva concessi. {[}8{]} Perché, come di
sopra si disse, chi non fa e' fondamenti prima, gli potrebbe con una
grande virtù farli poi, ancora che si faccino con disagio dello
architettore e periculo dello edifizio. {[}9{]} Se adunque si considerrà
tutti e' progressi del duca, si vedrà lui aversi fatti grandi fondamenti
alla futura potenza; e' quali non iudico superfluo discorrere perché io
non saprei quali precetti mi dare migliori, a uno principe nuovo, che lo
esemplo delle azioni sue: e se gli ordini sua non gli profittorno, non
fu sua colpa, perché nacque da una extraordinaria ed estrema malignità
di fortuna.

{[}10{]} Aveva Alexandro VI, nel volere fare grande il duca suo
figliuolo, assai difficultà presenti e future. {[}11{]} Prima, e' non
vedeva via di poterlo fare signore di alcuno stato che non fussi stato
di Chiesa: e, volgendosi a tòrre quello della Chiesa, sapeva che il duca
di Milano e' viniziani non gliene consentirebbono, perché Faenza e
Rimino erano di già sotto la protezione de' viniziani. {[}12{]} Vedeva
oltre a questo l'arme di Italia, e quelle in spezie di chi si fussi
potuto servire, essere nelle mani di coloro che dovevano temere la
grandezza del papa, -- e però non se ne poteva fidare, -- sendo tutte
nelli Orsini e Colonnesi e loro complici. {[}13{]} Era adunque
necessario si turbassino quelli ordini e disordinare gli stati di
Italia, per potersi insignorire sicuramente di parte di quelli. {[}14{]}
Il che gli fu facile, perché e' trovò e' viniziani che, mossi da altre
cagioni, si erano volti a fare ripassare e' franzesi in Italia: il che
non solamente non contradisse, ma lo fe' più facile con la resoluzione
del matrimonio antico del re Luigi.

{[}15{]} Passò adunque il re in Italia con lo aiuto de' vineziani e
consenso di Alessandro: né prima fu in Milano che il papa ebbe da lui
gente per la impresa di Romagna, la quale gli fu acconsentita per la
reputazione del re. {[}16{]} Acquistata adunque il duca la Romagna e
sbattuti e' Colonnesi, volendo mantenere quella e procedere più avanti,
lo impedivano dua cose: l'una, le arme sua che non gli parevano fedeli;
l'altra, la volontà di Francia; cioè che l'arme Orsine, delle quali si
era valuto, gli mancassino sotto, e non solamente gl'impedissino lo
acquistare ma gli togliessino lo acquistato, e che il re ancora non li
facessi il simile. {[}17{]} Delli Orsini ne ebbe uno riscontro quando
dopo la espugnazione di Faenza, assaltò Bologna, che gli vidde andare
freddi in quello assalto; e circa el re conobbe lo animo suo quando,
preso el ducato d'Urbino, assaltò la Toscana: da la quale impresa il re
lo fece desistere.

{[}18{]} Onde che il duca deliberò di non dependere più da le arme e
fortuna di altri; e, la prima cosa, indebolì le parte Orsine e Colonnese
in Roma: perché tutti gli aderenti loro, che fussino gentili uomini, se
gli guadagnò, faccendoli suoi gentili uomini e dando loro grande
provisioni, et onorogli, secondo le loro qualità, di condotte e di
governi: in modo che in pochi mesi negli animi loro l'affectione delle
parti si spense e tutta si volse nel Duca. {[}19{]} Dopo questo, aspettò
la occasione di spegnere e capi Orsini, avendo dispersi quelli di casa
Colonna: la quale gli venne bene, e lui la usò meglio. {[}20{]} Perché,
advedutosi gli Orsini tardi che la grandezza del Duca e della Chiesa era
la loro ruina, feciono una dieta alla Magione nel Perugino; da quella
nacque la ribellione di Urbino, e tumulti di Romagna et infiniti
periculi del Duca, e quali tutti superò con lo aiuto delli Franzesi.
{[}21{]} E ritornatoli la reputazione, né si fidando di Francia né di
altre forze externe, per non le avere a cimentare si volse alli inganni;
e seppe tanto dissimulare l'animo suo che li Orsini, mediante il signore
Paulo, si riconciliorono seco, -- con il quale il Duca non mancò d'ogni
ragione di offizio per assicurarlo, dandoli danari veste e cavalli --
tanto che la simplicità loro gli condusse a Sinigaglia nelle sua mane.

{[}22{]} Spenti adunque questi capi e ridotti li partigiani loro sua
amici, aveva il Duca gittati assai buoni fondamenti alla potenza sua,
avendo tutta la Romagna col ducato di Urbino, parendoli maxime aversi
acquistata amica la Romagna e guadagnatosi quelli popoli per avere
cominciato a gustare il bene essere loro. {[}23{]} E perché questa parte
è degna di notizia e da essere da altri imitata, non la voglio lasciare
indietro. {[}24{]} Preso che ebbe il Duca la Romagna e trovandola suta
comandata da Signori impotenti, -- li quali più presto avevano spogliato
e loro subditi che corretti, e dato loro materia di disunione, non di
unione, -- tanto che quella provincia era tutta piena di latrocinii, di
brighe e d'ogni altra ragione di insolenzia, iudicò fussi necessario, a
volerla ridurre pacifica et ubbidiente al braccio regio, dargli buono
governo; e però vi prepose messer Remirro de Orco, uomo crudele et
expedito, al quale dette plenissima potestà. {[}25{]} Costui im poco
tempo la ridusse pacifica et unita, con grandissima reputazione.
{[}26{]} Dipoi iudicò il duca non essere necessario sì excessiva
autorità perché dubitava non divenissi odiosa, e preposevi uno iudizio
civile nel mezzo della provincia, con uno presidente excellentissimo,
dove ogni città vi aveva lo advocato suo. {[}27{]} E perché conosceva le
rigorosità passate avergli generato qualche odio, per purgare li animi
di quegli populi e guadagnarseli in tutto, volse monstrare che, se
crudeltà alcuna era seguita, non era causata da lui ma dalla acerba
natura del ministro. {[}28{]} E presa sopra a questo occasione, lo fece,
a Cesena, una mattina mettere in dua pezzi in sulla piazza, con uno
pezzo di legne et uno coltello sanguinoso accanto; la ferocità del quale
spettaculo fece quegli popoli in uno tempo rimanere satisfatti e
stupidi.

{[}29{]} Ma torniamo donde noi partimmo. Dico che, trovandosi il Duca
assai potente et in parte assicurato de' presenti periculi, per essersi
armato a suo modo et avere in buona parte spente quelle arme che,
vicine, lo potevano offendere, gli restava, volendo procedere collo
acquisto, el respecto del re di Francia: perché conosceva come dal Re,
il quale tardi s'era accorto dello error suo, non gli sarebbe
sopportato. {[}30{]} E cominciò per questo a cercare di amicizie nuove e
vacillare con Francia, nella venuta che li Franzesi feciono verso el
regno di Napoli contro alli Spagnuoli che assediavono Gaeta; e lo animo
suo era assicurarsi di loro: il che gli sarebbe presto riuscito, se
Alessandro viveva. {[}31{]} E questi furono e governi sua, quanto alle
cose presente.

{[}32{]} Ma, quanto alle future, lui aveva a dubitare im prima che uno
nuovo successore alla Chiesa non gli fussi amico e cercassi torgli
quello che Alessandro li aveva dato. {[}33{]} Di che pensò assicurarsi
in quattro modi: prima, di spegnere tutti e sangui di quelli Signori che
lui aveva spogliati, per tòrre al Papa quella occasione; secondo, di
guadagnarsi tutti e gentili uomini di Roma, come è detto, per potere con
quelli tenere il Papa in freno; terzio, ridurre il Collegio più suo che
poteva; quarto, acquistare tanto imperio, avanti che il papa morissi,
che potessi per sé medesimo resistere a uno primo impeto. {[}34{]} Di
queste quattro cose alla morte di Alessandro ne aveva condotte tre, la
quarta aveva quasi per condotta: perché de' Signori spogliati ne ammazzò
quanti ne poté aggiugnere, e pochissimi si salvorono; e gentili uomini
romani si aveva guadagnati; e nel Collegio aveva grandissima parte; e
quanto al nuovo acquisto, aveva disegnato diventare signore di Toscana,
e possedeva di già Perugia e Piombino, e di Pisa aveva presa la
protectione. {[}35{]} E come non avessi avuto ad avere rispetto a
Francia, -- ché non gliene aveva ad avere più, per essere di già e
Franzesi spogliati del Regno dalli Spagnuoli: di qualità che ciascuno di
loro era necessitato comperare l'amicizia sua, -- egli saltava in Pisa.
{[}36{]} Dopo questo, Lucca e Siena cedeva subito, parte per invidia de'
Fiorentini, parte per paura: Fiorentini non avevano rimedio. {[}37{]} Il
che se gli fusse riuscito, -- che gli riusciva l'anno medesimo che
Alessandro morì, -- si acquistava tante forze e tanta reputazione che
per sé stesso si sarebbe retto e non sarebbe più dependuto dalla fortuna
e forze di altri, ma dalla potenza e virtù sua.

{[}38{]} Ma Alessandro morì dopo cinque anni che egli aveva cominciato a
trarre fuora la spada; lasciollo con lo stato di Romagna solamente
assolidato, con tutti li altri in aria, infra dua potentissimi exerciti
inimici, e malato a morte. {[}39{]} Et era nel Duca tanta ferocità e
tanta virtù, e sì bene conosceva come li uomini si hanno a guadagnare o
perdere, e tanto erano validi e fondamenti che in sì poco tempo si aveva
fatti, che, s'e' non avessi avuto quelli exerciti adosso, o lui fussi
stato sano, arebbe retto ad ogni difficultà.

{[}40{]} E che e fondamenti sua fussino buoni, si vidde: ché la Romagna
lo aspettò più d'uno mese; in Roma, ancora che mezzo vivo, stette
sicuro, e, benché Baglioni Vitelli et Orsini venissino in Roma, non
ebbono seguito contro di lui; poté fare, se non chi e' volle papa,
almeno che non fussi chi egli non voleva. {[}41{]} Ma se nella morte di
Alessandro fussi stato sano, ogni cosa gli era facile: e lui mi disse,
ne' dí che fu creato Iulio II, che aveva pensato a ciò che potessi
nascere morendo el padre, et a tutto aveva trovato remedio, excepto che
non pensò mai, in sulla sua morte, di stare ancora lui per morire.

{[}42{]} Raccolte io adunque tutte le azioni del Duca, non saprei
riprenderlo: anzi mi pare, come io ho fatto, di preporlo imitabile a
tutti coloro che per fortuna e con l'arme di altri sono ascesi allo
imperio; perché lui, avendo l'animo grande e la sua intenzione alta, non
si poteva governare altrimenti, e solo si oppose alli sua disegni la
brevità della vita di Alessandro e la sua malattia. {[}43{]} Chi adunque
iudica necessario nel suo principato nuovo assicurarsi delli inimici,
guadagnarsi delli amici; vincere o per forza o per fraude; farsi amare e
temere da' populi, seguire e reverire da' soldati; spegnere quelli che
ti possono o debbono offendere; innovare con nuovi modi gli ordini
antiqui; essere severo e grato, magnanimo e liberale; spegnere la
milizia infidele, creare della nuova; mantenere l'amicizie de' re e de'
principi in modo che ti abbino a beneficare con grazia o offendere con
respecto, non può trovare e più freschi exempli che le actioni di
costui.

{[}44{]} Solamente si può accusarlo nella creazione di Iulio pontefice,
nella quale il Duca ebbe mala electione. {[}45{]} Perché, come è detto,
non possendo fare uno papa a suo modo, poteva tenere che uno non fussi
papa; e non doveva mai consentire al papato di quelli cardinali che lui
avessi offesi o che, diventati papa, avessino ad aver paura di lui:
perché gli uomini offendono o per paura o per odio. {[}46{]} Quelli che
lui aveva offeso erano, infra li altri, Sancto Pietro ad Vincula,
Colonna, San Giorgio, Ascanio; tutti li altri avevano, divenuti papi, a
temerlo, eccepto Roano e gli Spagnuoli: questi per coniunzione et
obligo, quello per potenza, avendo coniunto seco el regno di Francia.

{[}47{]} Pertanto el Duca innanzi ad ogni cosa doveva creare papa uno
spagnolo; e, non potendo, doveva consentire a Roano, non a San Piero ad
vincula. {[}48{]} E chi crede che nelli personaggi grandi e benefizii
nuovi faccino dimenticare le iniurie vecchie, s'inganna. {[}49{]} Errò
adunque el Duca in questa electione; e fu cagione dell'ultima ruina sua.

\quebra\section{DE HIS QUI PER SCELERA AD PRINCIPATUM PERVENERE
{[}Di quelli che per scelleratezze sono venuti al principato{]}}

{[}1{]} Ma perché di privato si diventa principe ancora in dua modi, il
che non si può al tutto o alla fortuna o alla virtù attribuire, non mi
pare da lasciarli indrieto, ancora che dell'uno si possa più
diffusamente ragionare dove si trattassi delle repubbliche. {[}2{]}
Questi sono quando o per qualche via scellerata e nefaria si ascende al
principato, o quando uno privato ciptadino con el favore degli altri sua
ciptadini diventa principe della sua patria. {[}3{]} E parlando del
primo modo si mosterrà con dua exempli, uno antico, l'altro moderno,
sanza entrare altrimenti ne' meriti di questa parte: perché io iudico
che bastino a chi fussi necessitato imitargli.

{[}4{]} Agatocle siciliano, non solo di privata ma d'infima et abietta
fortuna, divenne re di Siracusa. {[}5{]} Costui, nato d'uno figulo,
tenne sempre, per i gradi della sua età, vita scellerata: nondimanco
accompagnò le sua scelleratezze con tanta virtù di animo e di corpo che,
voltosi alla milizia, per li gradi di quella pervenne ad essere pretore
di Siracusa. {[}6{]} Nel qual grado sendo constituito, et avendo
deliberato diventare principe e tenere con violenzia e sanza obligo di
altri quello che d'accordo gli era suto concesso, et avuto di questo suo
disegno intelligenzia con Amilcare cartaginese, il quale con li exerciti
militava in Sicilia, raunò una mattina il populo et il senato di
Siracusa, come se egli avessi avuto a deliberare cose pertinenti alla
repubblica. {[}7{]} Et ad uno cenno ordinato fece da' sua soldati
uccidere tutti e senatori e li più ricchi del popolo; e quali morti,
occupò e tenne el principato di quella città sanza alcuna controversia
civile. {[}8{]} E benché da' Cartaginesi fussi dua volte rotto e demum
assediato, non solum poté difendere la sua città, ma, lasciato parte
delle sue genti alla difesa della obsidione, con le altre assaltò
l'Affrica et in breve tempo liberò Siracusa dallo assedio e condusse
Cartaginesi in extrema necessità; e furono necessitati accordarsi con
quello, essere contenti della possessione di Affrica, et ad Agatocle
lasciare la Sicilia.

{[}8{]} Chi considerassi adunque le actioni e virtù di costui, non vedrà
cose, o poche, le quali possa attribuire alla fortuna, con ciò sia cosa,
come di sopra è detto, che non per favore di alcuno, ma per li gradi
della milizia, e quali con mille disagi e pericoli si aveva guadagnati,
pervenissi al principato, e quello dipoi con tanti partiti animosi e
periculosissimi mantenessi. {[}10{]} Non si può ancora chiamare virtù
ammazzare e suoi ciptadini, tradire gli amici, essere sanza fede, sanza
pietà, sanza relligione: e quali modi possono fare acquistare imperio ma
non gloria. {[}11{]} Perché, se si considerassi la virtù di Agatocle
nello entrare e nello uscire de' periculi e la grandezza dello animo suo
nel sopportare e superare le cose adverse, non si vede perché egli abbia
ad essere iudicato inferiore a qualunque excellentissimo capitano:
nondimanco la sua efferata crudeltà et inumanità, con infinite
scelleratezze, non consentono che sia infra gli excellentissimi uomini
celebrato. {[}12{]} Non si può adunque attribuire alla fortuna o alla
virtù quello che sanza l'una e l'altra fu da lui conseguito.

{[}13{]} Ne' tempi nostri, regnante Alessandro VI, Liverotto firmano,
sendo più anni innanzi rimaso piccolo, fu da uno suo zio materno,
chiamato Giovanni Fogliani, allevato, e ne' primi tempi della sua
gioventù dato a militare sotto Paulo Vitegli, acciò che, ripieno di
quella disciplina, pervenissi a qualche excellente grado di milizia.
{[}14{]} Morto dipoi Paulo, militò sotto Vitellozzo, suo fratello, et in
brevissimo tempo, per essere ingegnoso e della persona e dello animo
gagliardo, diventò el primo uomo della sua milizia. {[}15{]} Ma,
parendogli cosa servile lo stare con altri, pensò, con lo aiuto di
alcuno ciptadino firmano, alli quali era più cara la servitù che la
libertà della loro patria, e con il favore vitellesco, occupare Fermo.
{[}16{]} E scripse a Giovanni Fogliani come, sendo stato più tempo fuora
di casa, voleva venire a vedere lui e la sua città, e riconoscere in
qualche parte el suo patrimonio; e perché non si era affaticato per
altro che per acquistare onore, acciò che li suoi ciptadini vedessino
come non aveva speso il tempo invano, voleva venire onorevole et
accompagnato da cento cavagli di sua amici e servidori; e pregavalo
fussi contento ordinare che da' Firmiani fussi ricevuto onorevolmente:
il che non solamente tornava onore a se proprio, ma a lui, sendo suo
alunno.

{[}17{]} Non mancò pertanto Giovanni di alcuno offizio debito verso el
nipote, e, fattolo ricevere da' Firmiani onoratamente, si alloggiò nelle
case sua; dove, passato alcuno giorno et atteso ad ordinare segretamente
quello che alla sua futura sceleratezza era necessario, fece uno convito
solennissimo, dove invitò Giovanni Fogliani e tutti li primi uomini di
Fermo. {[}18{]} E consumate che furono le vivande e tutti gli altri
intrattenimenti che in simili conviti si usano, Liverotto ad arte mosse
certi ragionamenti di cose gravi, parlando della grandezza di papa
Alessandro e di Cesare suo figliuolo e delle imprese loro: alli quali
ragionamenti rispondendo Giovanni e gli altri, lui ad uno tratto si
rizzò, dicendo quelle essere cose da ragionarme in luogo più secreto; e
ritirossi in una camera, dove Giovanni e tutti gli altri ciptadini gli
andorono drieto. {[}19{]} Né prima furono posti a sedere che, delli
lochi segreti di quella, uscirono soldati che ammazzorono Giovanni e
tutti gli altri. {[}20{]} Dopo il quale omicidio montò Liverotto a
cavallo e corse la terra et assediò nel palazzo el supremo magistrato:
tanto che per paura furono constretti ubbidirlo e formare uno governo
del quale si fece principe; e morti tutti quelli che per essere
malcontenti lo potevono offendere, si corroborò con nuovi ordini civili
e militari: in modo che, in spazio di uno anno che tenne el principato,
non solamente lui era sicuro nella città di Fermo, ma era diventato
pauroso a tutti e sua vicini. {[}21{]} E sarebbe suta la sua
expugnazione difficile come quella di Agatocle, se non si fussi lasciato
ingannare da Cesare Borgia, quando a Sinigaglia, come di sopra si disse,
prese gli Orsini e Vitelli: dove, preso ancora lui, in uno anno dopo il
commisso parricidio fu insieme con Vitellozzo, il quale aveva avuto
maestro delle virtù e delle scelleratezze sue, strangolato.

{[}22{]} Potrebbe alcuno dubitare donde nascesse che Agatocle et alcuno
simile, dopo infiniti tradimenti e crudeltà, possé vivere lungamente
sicuro nella sua patria e difendersi dalli inimici externi, e dalli suoi
ciptadini non gli fu mai conspirato contro: con ciò sia che molti altri
mediante la crudeltà non abbino, etiam ne' tempi pacifici, potuto
mantenere lo stato, non che ne' tempi dubiosi di guerra. {[}23{]} Credo
che questo advenga dalle crudeltà male usate o bene usate. {[}24{]} Bene
usate si possono chiamare quelle, -- se del male è lecito dire bene, --
che si fanno ad uno tratto per la necessità dello assicurarsi: e dipoi
non vi si insiste dentro, ma si convertono in più utilità de' sudditi
che si può. {[}25{]} Male usate sono quelle le quali, ancora che nel
principio sieno poche, più tosto col tempo crescano che le si spenghino.
{[}26{]} Coloro che observano el primo modo, possono con Dio e con li
uomini avere allo stato loro qualche rimedio, come ebbe Agatocle; quelli
altri è impossibile si mantenghino.

{[}27{]} Onde è da notare che, nel pigliare uno stato, debbe lo
occupatore di epso discorrere tutte quelle offese che gli è necessario
fare, e tutte farle a un tratto, per non le avere a rinnovare ogni dì e
potere, non le innovando, assicurare li uomini e guadagnarseli con
beneficarli. {[}28{]} Chi fa altrimenti, o per timidità o per mal
consiglio, è sempre necessitato tenere il coltello in mano; né mai può
fondarsi sopra e sua subditi, non si potendo quegli, per le fresche e
continue iniurie, mai assicurare di lui. {[}29{]} Per che le iniurie si
debbono fare tutte insieme, acciò che, assaporandosi meno, offendino
meno; e benefizii si debbono fare a poco a poco, acciò si assaporino
meglio. {[}30{]} E debba sopratutto uno principe vivere in modo, con li
suoi subditi, che veruno accidente o di male o di bene lo abbia a fare
variare: perché, venendo per li tempi adversi le necessità, tu non se' a
tempo al male, et il bene che tu fai non ti giova perché è iudicato
forzato, e non te n'è saputo grado alcuno.

\quebra\section{DE PRINCIPATU CIVILI
{[}Del Principato Civile{]}}

{[}1{]} Ma venendo all'altra parte, quando uno privato ciptadino, non
per scelleratezza o altra intollerabile violenzia, ma con il favore
delli altri sua ciptadini diventa principe della sua patria, -- il quale
si può chiamare principato civile: né a pervenirvi è necessario o tutta
virtù o tutta fortuna, ma più presto una astuzia fortunata, -- dico che si ascende a questo principato o con il favore del populo o con de' grandi. {[}2{]} Perché in ogni città si truovano questi dua umori diversi: e nasce, da questo, che il populo desidera non essere comandato et oppresso da' grandi e li grandi desiderano comandare et opprimere el populo; e da questi dua appetiti diversi nasce nelle città uno de' tre effetti: o principato o libertà o licenza. {[}3{]} El principato è causato o dal populo o da' grandi, secondo che l'una o l'altra di queste parti ne ha l'occasione: perché, vedendo e grandi non potere resistere al populo, cominciano a voltare la reputazione ad uno di loro e fannolo principe per potere sotto la sua ombra sfogare il loro appetito; il populo ancora, vedendo non potere resistere a' grandi, volta la eputazione ad uno e lo fa principe per essere con la sua autorità difeso.

{[}4{]} Colui che viene al principato con lo aiuto de' grandi, si
mantiene con più difficultà che quello che diventa con lo aiuto del
populo, perché si truova principe con dimolti intorno che gli paiano
essere sua equali, e per questo non gli può né comandare né maneggiare a
suo modo. {[}5{]} Ma colui che arriva al principato con il favore
popolare, vi si truova solo et ha dintorno o nessuno o pochissimi che
non sieno parati a ubbidire.

{[}6{]} Oltre a questo non si può con onestà satisfare a' grandi, e
sanza iniuria di altri, ma sì bene al populo: perché quello del populo è
più onesto fine che quello de' grandi, volendo questi opprimere e quello
non essere oppresso. {[}7{]} Preterea, del populo inimico uno principe
non si può mai assicurare, per essere troppi: de' grandi si può
assicurare, per essere pochi. {[}8{]} Il peggio che possa aspettare uno
principe dal populo inimico, è lo essere abbandonato da lui; ma da'
grandi, inimici, non solo debba temere di essere abbandonato, ma etiam
che loro gli venghino contro: perché, essendo in quelli più vedere e più
astuzia, avanzono sempre tempo per salvarsi e cercono gradi con quelli
che sperano che vinca. {[}9{]} È necessitato ancora el principe vivere
sempre con quello medesimo populo, ma può ben fare sanza quelli medesimi
grandi, potendo farne e disfarne ogni dí e tòrre e dare a sua posta
reputazione loro.

{[}10{]} E per chiarire meglio questa parte, dico come e grandi si
debbono considerare in dua modi principalmente: o si governano in modo
col procedere loro che si obligano in tutto alla tua fortuna, o no.
{[}11{]} Quegli che si obligano, e non sieno rapaci, si debbono onorare
et amare. {[}12{]} Quelli che non si obligano, si hanno ad examinare in
dua modi: o e' fanno questo per pusillanimità e difetto naturale
d'animo; allora tu te ne debbi servire, maxime di quelli che sono di
buono consiglio, perché nelle prosperità te ne onori e non hai nelle
avversità a temere di loro. {[}13{]} Ma quando e' non si obligano per
arte e per cagione ambiziosa, è segno come pensano più a sé che a te: e
da quelli si debbe el principe guardare, e temergli come se fussino
scoperti nimici, perché sempre nelle adversità aiuteranno ruinarlo.

{[}14{]} Debba pertanto uno, che diventi principe mediante el favore del
populo, mantenerselo amico: il che gli fia facile, non domandando lui se
non di non essere oppresso. {[}15{]} Ma uno che, contro al populo,
diventi principe con il favore de' grandi, debbe innanzi ad ogni altra
cosa cercare di guadagnarsi el populo: il che gli fia facile, quando
pigli la protectione sua. {[}16{]} E perché li uomini, quando hanno bene
da chi credevano avere male, si obligano più al beneficatore loro,
diventa el populo subito più suo benivolo che s'e' si fussi condotto al
principato con favori sua. {[}17{]} E puosselo guadagnare el principe in
molti modi: e quali perché variano secondo el subiecto, non se ne può
dare certa regula, e però si lasceranno indrieto. {[}18{]} Concluderò
solo che a uno principe è necessario avere il populo amico, altrimenti
non ha nelle avversità remedio. {[}19{]} Nabide principe delli Spartani
sostenne la obsidione di tutta Grecia e di uno exercito romano
vittoriosissimo, e difese contro a quelli la patria sua et il suo stato;
e gli bastò solo, sopravvenente il periculo, assicurarsi di pochi: ché,
se egli avessi avuto el populo inimico, questo non li bastava. {[}20{]}
E non sia alcuno che repugni a questa mia opinione con quello proverbio
trito, che chi fonda in sul populo fonda in sul fango: perché quello è
vero quando uno ciptadino privato vi fa su fondamento e dassi ad
intendere che il populo lo liberi, quando fussi oppresso dalli nimici o
da' magistrati. {[}21{]} In questo caso si potrebbe trovare spesso
ingannato, come a Roma e Gracchi et a Firenze messer Giorgio Scali.
{[}22{]} Ma essendo uno principe che vi fondi su, che possa comandare e
sia uomo di core, né si sbigottisca nelle adversità, e non manchi delle
altre preparazione e tenga con lo animo et ordini suoi animato
l'universale, mai si troverrà ingannato da lui e gli parrà avere fatto
li suo fondamenti buoni. {[}23{]} Sogliono questi principati periclitare
quando sono per salire dallo ordine civile allo assoluto. {[}24{]}
Perché questi principi o comandano per loro medesimi o per mezzo delli
magistrati: nello ultimo caso è più debole e più periculoso lo stato
loro, perché gli stanno al tutto con la volontà di quelli ciptadini che
a' magistrati sono preposti: e quali, maxime ne' tempi adversi, gli
possono torre con facilità grande lo stato, o con abbandonarlo o con
fargli contro. {[}25{]} Et il principe non è a tempo ne' pericoli, a
pigliare l'autorità assoluta, perché e ciptadini e subditi, che sogliono
avere e' comandamenti da' magistrati, non sono in quelli frangenti per
ubbidire a' suoi. {[}26{]} Et arà sempre ne' tempi dubii penuria di chi
si possa fidare; perché simile principe non può fondarsi sopra quello
che vede ne' tempi quieti, quando e' ciptadini hanno bisogno dello
stato: perché allora ognuno corre, ognuno promette, e ciascuno vuole
morire per lui, quando la morte è discosto; ma ne' tempi adversi, quando
lo stato ha bisogno de' ciptadini, allora se ne truova pochi. {[}27{]} E
tanto più è questa experienzia periculosa, quanto la non si può fare se
non una volta: però uno principe savio debbe pensare uno modo per il
quale li sua ciptadini, sempre et in ogni qualità di tempo, abbino
bisogno dello stato e di lui; e sempre dipoi gli saranno fedeli.

\quebra\section{QUOMODO OMNIUM PRINCIPATUUM VIRES PERPENDI DEBEANT
{[}In che modo si debbino misurare le forze di tutti i principati{]}}

{[}1{]} Conviene avere, nello examinare le qualità di questi principati,
un'altra considerazione: cioè se uno principe ha tanto stato che possa,
bisognando, per sé medesimo reggersi, o vero se ha sempre necessità
della defensione d'altri. {[}2{]} E per chiarire meglio questa parte,
dico come io iudico coloro potersi reggere per sé medesimi che possono,
o per abbondanzia di uomini o di denari, mettere insieme un exercito
iusto e fare una giornata con qualunque lo viene ad assaltare. {[}3{]} E
così iudico coloro avere sempre necessità di altri, che non possono
comparire contro al nimico in campagna, ma sono necessitati rifuggirsi
dentro alle mura e guardare quelle.

{[}4{]} Nel primo caso, si è discorso e per lo advenire direno quello ne
occorre. {[}5{]} Nel secondo caso, non si può dire altro salvo che
confortare tali principi a fortificare e munire la terra propria e del
paese non tenere alcuno conto. {[}6{]} E qualunque arà bene fortificata
la sua terra e, circa alli altri governi, con li subditi si sarà
maneggiato come di sopra è detto e di sotto si dirà, sarà sempre con
gran respetto assaltato; perché li uomini sono nimici delle imprese dove
si vegga difficultà: né si può vedere facilità assaltando uno che abbia
la suo terra gagliarda e non sia odiato dal populo.

{[}7{]} Le città della Magna sono liberissime, hanno poco contado et
obbediscano allo Imperatore quando le vogliono, e non temono né quello
né alcuno altro potente che le abbino intorno. {[}8{]} Perché le sono in
modo affortificate che ciascuno pensa la expugnazione di epse dovere
essere tediosa e difficile: perché tutte hanno fossi e mura conveniente;
hanno artiglieria a sufficienzia; tengono sempre nelle canove publiche
da bere e da mangiare e da ardere per uno anno; {[}9{]} et oltre a
questo, per potere tenere la plebe pasciuta e sanza perdita del publico,
hanno sempre in comune da potere per uno anno dare loro da lavorare
loro, in quelli exercizii che sieno el nervo e la vita di quella città e
delle industrie de' quali la plebe pasca; tengono ancora li exercizii
militari in reputazione, e sopra questo hanno molti ordini a
mantenergli.

{[}10{]} Uno principe adunque, che abbia una città così ordinata e non
si facci odiare, non può essere assaltato; e, se pure fussi chi lo
assaltassi, se ne partire', con vergogna: perché le cose del mondo sono
sì varie che egli è impossibile che uno potessi con li exerciti stare
uno anno ocioso a campeggiarlo. {[}11{]} E chi replicassi: se il populo
arà le sua possessioni fuora e veggale ardere, non ci arà pazienza, et
il lungo assedio e la carità propria gli farà sdimenticare lo amore del
principe; rispondo che uno principe prudente et animoso supererà sempre
tutte quelle difficultà, dando a' subditi ora speranza che il male non
fia lungo, ora timore della crudeltà del nimico, ora assicurandosi con
destrezza di quegli che gli paressino troppo arditi. {[}12{]} Oltre a
questo, el nimico ragionevolmente debba ardere e ruinare el paese in su
la sua giunta e nelli tempi quando gli animi degli uomini sono ancora
caldi e volenterosi alla difesa: e però tanto meno el principe debba
dubitare, perché dopo qualche giorno, che gli animi sono raffreddi, sono
di già fatti e danni, sono ricevuti e mali, non vi è più remedio.
{[}13{]} Et allora tanto più si vengono ad unire con il loro principe,
parendo che lui abbia con loro obligo, sendo loro sute arse le case,
ruinate le possessioni per la difesa sua: e la natura delli uomini è
così obligarsi per li benefizii che si fanno, come per quelli che si
ricevano. {[}14{]} Onde, se si considerrà bene tutto, non fia difficile
ad uno principe prudente tenere, prima e poi, fermi gli animi de' sua
ciptadini nella obsidione, quando non vi manchi né da vivere né da
difendersi.

\quebra\section{DE PRINCIPATIBUS ECCLESIASTICIS.
{[}De' principati ecclesiastici{]}}

{[}1{]} Restaci solamente al presente a ragionare de' principati
ecclesiastici, circa quali tutte le difficultà sono avanti che si
possegghino; perché si acquistano o per virtù o per fortuna, e sanza
l'una e l'altra si mantengono: perché sono subtentati dalli ordini
antiquati nella religione, quali sono stati tanto potenti e di qualità
che tengono e loro principi in stato in qualunque modo si procedino e
vivino. {[}2{]} Costoro soli hanno stati e non gli difendano; hanno
subditi, e non li governano. {[}3{]} E gli stati, per essere indifesi,
non sono loro tolti; et e subditi, per non essere governati, non se ne
curano, né pensano, né possono alienarsi da loro. {[}4{]} Solo adunque
questi principati sono sicuri e felici; ma, essendo quelli retti da
cagione superiori, alle quali mente umana non aggiugne, lascerò il
parlarne: perché, essendo exaltati e mantenuti da Dio, sarebbe officio
di uomo presumptuoso e temerario discorrerne. {[}5{]} Nondimanco, se
alcuno mi ricercassi donde viene che la Chiesa nel temporale sia venuta
a tanta grandezza, -- con ciò sia che da Alessandro indrieto e potentati
italiani, e non solum quelli che si chiamavono e potentati ma ogni
barone e signore benché minimo, quanto al temporale, la extimava poco,
et ora uno re di Francia ne trema, e lo ha possuto cavare di Italia e
ruinare Viniziani, -- la qual cosa, ancora che sia nota, non mi pare
superfluo ridurla in buona parte alla memoria.

{[}6{]} Avanti che Carlo re di Francia passassi in Italia, era questa
provincia sotto lo imperio del Papa, Viniziani, re di Napoli, duca di
Milano e Fiorentini. {[}7{]} Questi potentati avevano ad avere dua cure
principali: l'una, che uno forestiero non entrassi in Italia con le
arme; l'altra, che veruno di loro occupassi più stato. {[}8{]} Quegli a
chi si aveva più cura erano Papa e Viniziani; et a tenere indrieto e
Viniziani, bisognava la unione di tutti li altri, come fu nella difesa
di Ferrara; et a tenere basso il Papa, si servivono de' baroni di Roma,
li quali sendo divisi in due factioni, Orsine e Colonnese, sempre vi era
cagione di scandolo fra loro; e, stando con le arme in mano in su li
occhi al Pontefice, tenevano il pontificato debole et infermo. {[}9{]}
E, benché surgessi qualche volta alcuno papa animoso, come fu Sixto,
tamen la fortuna o il sapere non lo poté mai disobligare da queste
incommodità. {[}10{]} E la brevità della vita loro n'era cagione; perché
in dieci anni che, raguagliato, uno papa viveva, a fatica ch'é potessi
abassare una delle factioni; e se, verbi gratia, l'uno aveva quasi
spenti Colonnesi, surgeva un altro inimico agli Orsini, che gli faceva
risurgere e li Orsini non era a tempo a spegnere. {[}11{]} Questo faceva
che le forze temporali del Papa erano poco stimate in Italia.

{[}12{]} Surse dipoi Alessandro VI, il quale, di tutt'i pontefici che
sono mai stati, mostrò quanto uno papa e con danaio e con le forze si
poteva prevalere; e fece, con lo instrumento del duca Valentino e con la
occasione della passata de' Franzesi, tutte quelle cose che io discorro
di sopra nelle acyioni del Duca. {[}13{]} E benché la intenzione suo non
fussi fare grande la Chiesa, ma il duca, nondimeno ciò che fece tornò a
grandezza della Chiesa; la quale dopo la sua morte, spento il Duca, fu
erede delle sua fatiche.

{[}14{]} Venne dipoi papa Iulio e trovò la Chiesa grande, avendo tutta
la Romagna e essendo spenti e baroni di Roma e, per le battiture di
Alessandro, annullate quelle fazioni; e trovò ancora la via aperta al
modo dello accumulare danari, non mai più usitato da Alessandro
indrieto. {[}15{]} Le quali cose Iulio non solum seguitò, ma accrebbe, e
pensò a guadagnarsi Bologna e spegnere e Viniziani et a cacciare
Franzesi di Italia: e tutte queste imprese gli riuscirono, e con tanta
più sua laude, quanto lui fece ogni cosa per adcrescere la Chiesa e non
alcuno privato. {[}16{]} Mantenne ancora le parte Orsine e Colonnese in
quelli termini le trovò. {[}17{]} E benché fra loro fussi qualche capo
da fare alterazione, tamen dua cose gli ha tenuti fermi: l'una, la
grandezza della Chiesa, che gli sbigottisce; l'altra, il non avere loro
cardinali, i quali sono origine delli tumulti intra loro; né mai
staranno quiete, qualunque volta queste parti abbino cardinali, perché
questi nutriscono, in Roma e fuori, le parte, e quelli baroni sono
forzati a difenderle; e così, dalla ambizione de' prelati nascono le
discordie e li tumulti intra baroni. {[}18{]} Ha trovato adunque la
Sanctità di papa Leone questo pontificato potentissimo: il quale si
spera, se quegli lo feciono grande con le arme, questo, con la bontà et
infinite altre sue virtù lo farà grandissimo e venerando.

\quebra\section{QUOT SUNT GENERA MILITIAE ET DE MERCENNARIIS MILITIBUS
{[}Di quante ragioni sia la milizia, e de' soldati mercennarii{]}}

{[}1{]} Avendo discorso particularmente tutte le qualità di quelli
principati de' quali nel principio proposi di ragionare, e considerato
in qualche parte le cagioni del bene e del male essere loro, e mostro e
modi con li quali molti hanno cerco di acquistargli e tenergli, mi resta
ora a discorrere generalmente le offese e difese che in ciascuno de'
prenominati possono accadere.

{[}2{]} Noi abbiamo detto di sopra come a uno principe è necessario
avere e sua fondamenti buoni, altrimenti di necessità conviene che
ruini. {[}3{]} E principali fondamenti che abbino tutti li stati, così
nuovi come vecchi o mixti, sono le buone legge e le buone arme: e perché
non può essere buone legge dove non sono buone arme, e dove sono buone
arme conviene sieno buone legge, io lascerò indrieto el ragionare delle
legge e parlerò delle arme.

{[}4{]} Dico adunque che l'arme con le quali uno principe difende el suo
stato o le sono proprie, o le sono mercennarie o auxiliare o mixte.
{[}5{]} Le mercennarie et auxiliarie sono inutile e periculose; e se uno
tiene lo stato suo fondato in su l'arme mercennarie, non starà mai fermo
né sicuro, perché le sono disunite, ambiziose, sanza disciplina,
infedele, gagliarde infra gli amici, infra ` nimici vile: non timore di
Dio, non fede con li uomini; e tanto si differisce la ruina, quanto si
differisce lo assalto; e nella pace se' spogliato da loro, nella guerra
dagli inimici. {[}6{]} La cagione di questo è che le non hanno altro
amore né altra cagione che le tenga in campo che uno poco di stipendio,
il quale non è suffiziente a fare che voglino morire per te. {[}7{]}
Vogliono bene essere tua soldati mentre che tu non fai guerra; ma, come
la guerra viene, o fuggirsi o andarsene. {[}8{]} La qual cosa doverrei
durare poca fatica a persuadere, perché ora la ruina di Italia non è
causata da altro che per essersi per spazio di molti anni riposata tutta
in sulle arme mercennarie. {[}9{]} Le quali feciono già per alcuno
qualche progresso, e parevano gagliarde infra loro; ma come venne el
forestiero le mostrorono quello che elle erano: onde che a Carlo re di
Francia fu lecito pigliare la Italia col gesso; e chi diceva come
n'erano cagione e peccati nostri, diceva il vero; ma non erano già
quegli che credeva, ma questi che io ho narrati; e perché gli erano
peccati di principi, ne hanno patito le pene ancora loro.

{[}10{]} Io voglio dimonstrare meglio la infelicità di queste arme. E
capitani mercennarii o e' sono uomini excellenti, o no; se sono, non te
ne puoi fidare, perché sempre aspireranno alla grandezza propria, o con
lo opprimere te, che gli se' patrone, o con opprimere altri fuora della
tua intenzione; ma se il capitano non è virtuoso, ti rovina per
l'ordinario. {[}11{]} E se si responde che qualunque arà le arme in mano
farà questo, o mercennario o no, replicherrei come l'arme hanno ad
essere operate o da uno principe o da una repubblica: el principe debbe
andare in persona, e fare lui l'offizio del capitano; la repubblica ha a
mandare sua ciptadini; e, quando ne manda uno che non riesca valente
uomo, debbe cambiarlo; e, quando sia, tenerlo con le leggi che non passi
el segno. {[}12{]} E per experienzia si vede alli principi soli e
republiche armate fare progressi grandissimi, et alle arme mercennarie
non fare mai se non danno; e con più difficultà viene alla obbedienza di
uno suo cittadino una repubblica armata di arme proprie, che una armata
di armi externe.

{[}13{]} Stettono Roma e Sparta molti seculi armate e libere. Svizzeri
sono armatissimi e liberissimi. {[}14{]} Delle arme mercennarie antiche
sono in exemplis e Cartaginesi, li quali furono per essere oppressi da'
loro soldati mercennarii, finita la loro prima guerra con i Romani,
ancora che li Cartaginesi avessino, per capitani loro proprii ciptadini
{[}15{]} Filippo Macedone fu fatto da' Tebani, dopo la morte di
Epaminunda, capitano di loro genti: e tolse loro, dopo la vittoria, loro
la libertà

{[}16{]} Milanesi, morto il duca Filippo, soldorono Francesco Sforza
contro a' Viniziani: il quale, superati gli inimici a Caravaggio, si
coniunse con loro per opprimere e Milanesi suoi patroni. {[}17{]} Sforza
suo padre, essendo soldato della regina Giovanna di Napoli, la lasciò in
un tratto disarmata: onde lei, per non perdere el regno, fu constretta
gittarsi in grembo al re di Aragonia. {[}18{]} E se Viniziani e
Fiorentini hanno per lo adrieto accresciuto lo imperio loro con queste
arme, e li loro capitani non se ne sono però fatti principi ma gli hanno
difesi, rispondo che e Fiorentini in questo caso sono suti favoriti
dalla sorte: perché, de' capitani virtuosi de' quali potevano temere,
alcuni non hanno vinto, alcuni hanno avuto opposizione, alcuni altri
hanno volto l'ambizione loro altrove.

{[}19{]} Quello che non vinse fu Giovanni Aucut, del quale, non
vincendo, non si poteva conoscere la fede: ma ognuno confesserà che,
vincendo, stavano e Fiorentini a sua discrezione. {[}20{]} Sforzo ebbe
sempre e Bracceschi contrarii, che guardorono l'uno l'altro. {[}21{]}
Francesco volse l'ambizione sua in Lombardia; Braccio, contro alla
Chiesa et il regno di Napoli.

{[}22{]} Ma veniamo a quello che è seguito poco tempo fa Feciono e
Fiorentini Paulo Vitelli loro capitano, uomo prudentissimo e che di
privata fortuna aveva presa grandissima reputazione; se costui expugnava
Pisa, veruno fia che neghi come conveniva a' Fiorentini stare, seco:
perché, se fussi diventato soldato de loro nimici, non avevano remedio;
e, se e' Fiorentini lo tenevano, aveano ad ubbidirlo.

{[}23{]} E Viniziani, se si considerrà e progressi loro, si vedrà quegli
avere sicuramente e gloriosamente operato mentre feciono la guerra loro
proprii, -- che fu avanti che si volgessino con le loro imprese loro in
terra, -- dove co' gentili uomini e con la plebe armata operorono
virtuosissimamente; ma, come cominciorono a combattere in terra,
lasciorono questa virtù e seguitorono e costumi delle guerre di Italia.
{[}24{]} E nel principio dello augumento loro in terra, per non vi avere
molto stato e per essere in grande reputazione, non avevono da temere
molto de' loro capitani. {[}25{]} Ma, come eglino ampliorono, che fu
sotto el Carmignola, ebbono uno saggio di questo errore: perché,
vedutolo virtuosissimo, battuto che loro ebbono sotto il suo governo il
duca di Milano, e conoscendo dall'altra parte come egli era raffreddo
nella guerra, iudicorono non potere con lui più vincere, perché non
voleva; né potere licenziarlo, per non riperdere ciò che aveano
acquistato; onde che furono necessitati, per assicurarsene, ammazzarlo.
{[}26{]} Hanno dipoi avuto per loro capitani Bartolomeo da Bergomo,
Ruberto da Sancto Severino, conte di Pitigliano, e simili, con li quali
avevano a temere della perdita, non del guadagno loro: come intervenne
dipoi a Vailà, dove in una giornata, perderono cio che in ottocento anni
con tanta fatica, avevono acquistato: perché da queste arme nascono solo
e lenti, tardi e deboli acquisti e le subite e miracolose perdite.

{[}27{]} E perché io sono venuto con questi exempli in Italia, la quale
è stata molti anni governata dalle arme mercennarie, io le voglio
discorrere più da alto acciò che, veduta l'origine e progressi di esse,
si possa meglio correggerle. {[}28{]} Avete adunque ad intendere come,
tosto che in questi ultimi tempi lo Imperio cominciò ad essere ributtato
di Italia e che il papa nel temporale vi prese più reputazione, si
divise la Italia in più stati: perché molte delle città grosse presono
l'arme contra a' loro nobili, e quali prima, favoriti dallo Imperatore,
le tennono oppresse, e la Chiesa le favoriva per darsi reputazione nel
temporale; di molte altre e loro ciptadini ne diventorono principi.
{[}29{]} Onde che, essendo venuta l'Italia quasi che nelle mani della
Chiesa e di qualche republica, et essendo quelli preti e quelli altri
ciptadini usi a non conoscere arme, cominciorono a soldare forestieri.
{[}30{]} El primo che dette reputazione a questa milizia fu Alberigo da
Conio, romagnuolo: dalla disciplina di costui discese intra gli altri
Braccio e Sforza, che ne' loro tempi furono arbitri di Italia. {[}31{]}
Dopo questa vennono tutti li altri che in fino alli nostri tempi hanno
governato queste arme: e'l fine della loro virtù è stato che Italia è
stata corsa da Carlo, predata da Luigi, sforzata da Fernando e
vituperata da' Svizzeri.

{[}32{]} L'ordine che gli hanno tenuto è stato prima, per dare
reputazione a loro proprii, avere tolto reputazione alle fanterie:
Feciono questo perché, sendo sanza stato et in sulla industria, e pochi
fanti non davano loro reputazione e gli assai non potevano nutrire; e
però si redussono a' cavagli, dove con numero sopportabile erano nutriti
et onorati: et erono ridotte le cose in termine che in uno exercito di
XX mila soldati non si trovava dumila fanti. {[}33{]} Avevano oltre a
questo usato ogni industria per levare a sé et a' soldati la paura e la
fatica, non si ammazzando nelle zuffe, ma pigliandosi prigioni e sanza
taglia; non traevano la notte alle terre; quegli della terre non
traevano alle tende; non facevano intorno al campo né steccato né fossa;
non campeggiavano el verno. {[}34{]} E tutte queste cose erano permesse
nelli loro ordini militari e trovate da loro per fuggire, come è detto,
la fatica e li pericoli: tanto che gli hanno condotta la Italia stiava e
vituperata.

\quebra\section{DE MILITIBUS AUXILIARIIS, MIXTIS ET PROPRIIS
{[}De' soldati aussiliari, delli aussiliari e propri insieme, e de' propri soli{]}}

{[}1{]} L'armi auxiliare, che sono l'altre arme inutili, sono quando si
chiama uno potente che con le sua arme ti venga a difendere, come fece
nelli proximi tempi papa Iulio: il quale, avendo visto nella impresa di
Ferrara la trista pruova delle sue arme mercennarie, si volse alle
auxiliare, e convenne con Ferrando re di Spagna che con le sua gente et
exerciti dovesse aiutarlo. {[}2{]} Queste arme possono essere buone e
utile per loro medesime, ma sono, per chi le chiama, quasi sempre
dannose: perché, perdendo rimani disfatto; vincendo, resti loro
prigione. {[}3{]} Et ancora che di questi exempli ne sieno piene le
antiche storie, nondimando io non mi voglio partire da questo exemplo
fresco di Iulio II: el partito del quale non poté essere meno
considerato, per voler Ferrara, cacciarsi tutto nelle mani d'uno
forestieri. {[}4{]} Ma la sua buona fortuna fece nascere una terza cosa,
acciò non cogliessi el frutto della sua mala electione: perché, sendo
gli auxiliari suoi rotti a Ravenna, e surgendo e Svizzeri che cacciorono
e vincitori fuora di ogni opinione e sua e d'altri, venne a non rimanere
prigione delli inimici, sendo fugati, né delli auxiliarii sua, avendo
vinto con altre arme che con le loro. {[}5{]} Fiorentini, sendo al tutto
disarmati, condussono diecimila Franzesi a Pisa per expugnarla: per il
quale partito portorono più pericolo che in qualunque tempo de' travagli
loro. {[}6{]} Lo imperatore di Constantinopoli, per opporsi alli suoi
vicini, misse in Grecia diecimila Turchi, li quali finita la guerra non
se ne volsono partire: il che fu il principio della servitù di Grecia
con gli infedeli.

{[}7{]} Colui adunque che vuole non potere vincere, si vaglia di queste
arme, perché sono molto più pericolose che le mercennarie. {[}8{]}
Perché in queste è la coniura fatta: sono tutte unite, tutte volte alla
obbedienza d'altri; ma nelle mercennarie ad offenderti, vinto che
l'hanno, bisogna maggiore occasione più tempo, non sendo tutte uno corpo
et essendo trovate e pagate da te: nelle quale un terzo che tu facci
capo non può pigliare subitamente tanta autorità che ti offenda. {[}9{]}
Insomma nelle mercennarie è più pericolosa la ignavia, nelli auxiliare
la virtù. {[}10{]} Uno principe pertanto savio sempre ha fuggito queste
arme e voltosi alle proprie: et ha voluto più tosto perdere con li suoi
che vincere con li altri, iudicando non vera vittoria quella che con le
armi aliene si acquistassi.

{[}11{]} Io non dubiterò mai di allegare Cesare Borgia e le sue actioni
Questo duca intrò in Romagna con le armi auxiliare, conducendovi tutte
gente franzese, e con quelle prese Imola e Furlí, ma non gli parendo poi
tale arme sicure, si volse alle mercennarie, iudicando in quelle meno
pericolo, e soldò gli Orsini e Vitelli; le qual dipoi trovando,
maneggiare nel dubbie infideli e pericolose, le spense e volsesi alle
proprie. {[}12{]} E puossi facilmente vedere che differenzia è fra l'una
e l'altra di queste arme, considerato che differenzia fu dalla
reputazione del Duca quando aveva Franzesi soli, a quando aveva gli
Orsini e Vitelli, a quando e' rimase con li soldati sua e sopra se
stesso: e sempre si troverrà accresciuta, né mai fu stimato assai se non
quando ciascuno vidde come lui era intero possessore delle sua arme.

{[}13{]} Io non mi volevo partire dalli exempli italiani e freschi:
tamen non voglio lasciare indrieto Ierone siracusano, sendo uno delli
sopra nominati da me. {[}14{]} Costui, come io dixi, fatto dalli
Siracusani capo degli exerciti, conobbe subito quella milizia
mercennaria non essere utile, per essere condottieri fatti come li
nostri italiani; e parendoli non gli potere tenere né lasciare, gli fece
tutti tagliare a pezzi, e dipoi fece guerra con le arme sua e non con le
aliene. {[}15{]} Voglio ancora ridurre a memoria una figura del
Testamento vecchio, fatta a questo proposito. {[}16{]} Offerendosi Davit
a Saul d'andare a combattere con Golia provocatore filisteo, Saul per
dargli animo l'armò dell'arme sua: le quali Davit come l'ebbe indosso,
recusò, dicendo con quelle non si potere bene valere di sé stesso; e
però voleva trovare el nimico con la sua fromba e con il suo coltello.
{[}17{]} Infine, le arme di altri o le ti caggiono di dosso o le ti
pesano o le ti stringano.

{[}18{]} Carlo VII, padre del re Luigi XI, avendo con la sua fortuna e
virtù libera la Francia dagli Inghilesi, conobbe questa necessità di
armarsi di arme proprie et ordinò nel suo regno l'ordinanza delle genti
d'arme e delle fanterie. {[}19{]} Dipoi el re Luigi suo figliuolo spense
quella de' fanti e cominciò a soldare Svizzeri: il quale errore
seguitato dalli altri è, come si vede ora in fatto, cagione de' pericoli
di quello regno. {[}20{]} Perché, avendo dato reputazione a' Svizzeri,
ha invilito tutte le arme sua; perché le fanterie ha spente in tutto e
le sua gente d'arme ha obligate alla virtù di altri: perché, sendo
assuefatte a militare con Svizzeri, non pare loro potere vincere sanza
epsi. {[}21{]} Di qui nasce che li Franzesi contro a Svizzeri non
bastano e sanza Svizzeri, contro ad altri, non pruovano. {[}22{]} Sono
dunque stati gli exerciti di Francia mixti, parte mercennarii e parte
proprii: le quali arme tutte insieme sono molto migliori che le semplice
auxiliare o semplice mercennarie, e molto inferiore alle proprie.
{[}23{]} E basti lo exemplo detto; perché el regno di Francia sarebbe
insuperabile, se l'ordine di Carlo era adcresciuto o preservato; ma la
poca prudenza delli uomini comincia una cosa che, per sapere allora di
buono, non si accorge del veleno che vi è sotto, come io dissi di sopra
delle febbre etiche. {[}24{]} Pertanto colui che in uno principato non
conosce e mali quando nascono, non è veramente savio: e questo è dato a
pochi. {[}25{]} E, se si considerassi la prima cagione della ruina dello
imperio romano, si troverrà essere suto solo cominciare a soldare e
Gotti: perché da quello principio cominciorono ad enervare le forze
dello imperio; e tutta quella virtù, che si levava da lui, si dava a
loro.

{[}26{]} Concludo adunque che, sanza avere arme proprie, nessuno
principato è sicuro, anzi è tutto obligato alla fortuna, non avendo
virtù che nelle adversità con fede lo difenda: e fu sempre opinione e
sentenza delli uomini savi, quod nihil sit tam infirmum aut instabile
quam fama potentiae non sua vi nixa. {[}27{]} E l'arme proprie sono
quelle che sono composte o di subditi o di ciptadini o di creati tua:
tutte l'altre sono o mercennarie o auxiliare et il modo ad ordinare
l'arme proprie sarà facile a trovare, se si discorrerà gli ordini de'
quattro sopra nominati da me, e se si vedrà come Filippo, padre di
Alessandro Magno, e come molte repubbliche e principi si sono armati et
ordinati: a' quali ordini io al tutto mi rimetto.

\quebra\section{QUOD PRINCIPEM DECEAT CIRCA MILITIAM
{[}Quello che s'appartenga a uno principe circa la milizia{]}}

{[}1{]} Debbe dunque uno principe non avere altro obietto né altro
pensiero né prendere cosa alcuna per sua arte, fuora della guerra et
ordini e disciplina di epsa: perché quella è sola arte che si espetta a
chi comanda, et è di tanta virtù che non solamente mantiene quelli che
sono nati principi, ma molte volte fa gli uomini di privata fortuna
salire a quello grado. {[}2{]} E per adverso si vede che, quando e
principi hanno pensato più alle delicatezze che alle arme, hanno perso
lo stato loro: e la prima cagione che ti fa perdere quello è negligere
questa arte, e la cagione che te lo fa acquistare è lo essere professo
di questa arte. {[}3{]} Francesco Sforza, per essere armato, di privato
diventò duca di Milano; e figliuoli, per fuggire e disagi delle arme, di
duchi diventorono privati. {[}4{]} Perché, intra le altre cagioni che ti
arreca di male, lo essere disarmato ti fa contennendo, la quale è una di
quelle infamie delle quali el principe si debbe guardare, come di sotto
si dirà. {[}6{]} Perché da uno armato a uno disarmato non è proporzione
alcuna, e non è ragionevole che chi è armato ubbedisca volentieri a chi
è disarmato, e che el disarmato stia sicuro intra servitori armati:
perché, sendo nell'uno sdegno e nell'altro sospetto, non è possibile
operino bene insieme. {[}6{]} E però uno principe che della milizia non
si intenda, oltre alle altre infelicità, come è detto, non può essere
stimato dalli suoi soldati né fidarsi di loro.

{[}7{]} Debbe pertanto mai levare il pensiero da questo exercizio della
guerra; e nella pace vi si debbe più exercitare che nella guerra; il che
può fare in dua modi: l'uno con le opere; l'altro, con la mente. {[}8{]}
E quanto alle opere, oltre al tenere bene ordinati et exercitati i suoi,
debba stare sempre in sulle cacce: e mediante quelle assuefare il corpo
a' disagi, e parte imparare la natura de' siti, e conoscere come surgono
e' monti, come imboccano le valle, come iacciono i piani, et intendere
la natura de' fiumi e de' paduli, et in questo porre grandissima cura.
{[}9{]} La quale cognizione è utile in dua modi: prima, s'impara a
conoscere el suo paese, e può meglio intendere le difese di epso; dipoi,
mediante la cognizione e pratica di quegli siti, con facilità
comprendere ogni altro sito che di nuovo gli sia necessario speculare:
perché li poggi, le valle, e piani, e fiumi, e paduli che sono, verbi
gratia, in Toscana hanno con quelli dell'altre provincie certa
similitudine, tale che dalla cognizione del sito di una provincia si può
facilmente venire alla cognizione dell'altre. {[}10{]} E quel principe
che manca di questa perizie, manca della prima parte che vuole avere uno
capitano: perché questa t'insegna trovare el nimico, pigliare gli
alloggiamenti, condurre gli exerciti, ordinare le giornate, campeggiare
le terre con tuo vantaggio.

{[}11{]} Filopemene, principe delli Achei, intra le altre laude che
dagli scriptori gli sono date, è che ne' tempi della pace non pensava
mai se non a' modi della guerra; e quando era in campagna con gli amici
spesso si fermava e ragionava con quelli: {[}12{]} Se li inimici fussino
in su quel colle e noi ci trovassimo qui col nostro exercito, chi arebbe
dinoi vantaggio? Come si potrebbe ire, servando l'ordini, a trovarli? se
noi volessimo ritirarci, come aremo a fare? se loro si ritirassino, come
aremo a seguirli? {[}13{]} E proponeva loro, andando, tutti e casi che
in uno exercito possono occorrere: intendeva la opinione loro, diceva la
sua, corroboravala con le ragioni: tale che, per queste continue
cogitazioni, non poteva mai, guidando gli exerciti, nascere accidente
alcuno che lui non vi avessi el remedio.

{[}14{]} Ma quanto allo exercizio della mente, debbe el principe leggere
le storie et in quelle considerare le actioni delli uomini excellenti,
vedere come si sono governati nelle guerre, examinare le cagioni della
vittoria e perdite loro, per potere queste fuggire e quelle imitare; e
sopratutto fare come ha fatto per lo adrieto qualche uomo excellente che
ha preso ad imitare se alcuno, innanzi a lui, è stato laudato e
gloriato, e di quello ha tenuto sempre e gesti et actioni appresso di
sé: come si dice che Alessandro Magno imitava Achille; Cesare,
Alessandro; Scipione, Ciro. {[}15{]} E qualunque legge la vita di Ciro
scritta da Xenofonte, riconosce dipoi nella vita di Scipione quanto
quella imitactione gli fu a gloria, e quanto, nella castità affabilità
umanità liberalità, Scipione si conformassi con quelle cose che di Ciro
da Xenofonte sono sute scripte.

{[}16{]} Questi simili modi debba observare uno principe savio; e mai
nelli tempi pacifici stare ozioso, ma con industria farne capitale per
potersene valere nelle adversità, acciò che la fortuna, quando si muta,
lo truovi parato a resisterle

\quebra\section{DE HIS REBUS QUIBUS HOMINES ET PRAESERTIM PRINCIPES LAUDANTUR AUT VITUPERANTUR
{[}Di quelle cose che li òmini e spezialmente e' principi sono laudati o biasimati{]}}

{[}1{]} Resta ora a vedere quali debbino essere e modi e governi di uno
principe o con subditi o con li amici. {[}2{]} E perché io so che molti
di questo hanno scripto, dubito, scrivendone ancora io, non essere
tenuto prosumptuoso, partendomi maxime, nel disputare questa materia,
dalli ordini delli altri. {[}3{]} Ma sendo l'intenzione mia stata
scrivere cosa che sia utile a chi la intende, mi è parso più conveniente
andare drieto alla verità effettuale della cosa che alla immaginazione
di epsa. {[}4{]} E molti si sono immaginati repupbliche e principati che
non si sono mai visti né conosciuti essere in vero. {[}5{]} Perché gli è
tanto discosto da come si vive a come si doverrebbe vivere, che colui
che lascia quello che si fa, per quello che si doverrebbe fare, impara
più presto la ruina che la perservazione sua: perché uno uomo che voglia
fare in tutte le parte professione di buono, conviene che ruini infra
tanti che non sono buoni. {[}6{]} Onde è necessario, volendosi a uno
principe mantenere, imparare a potere essere non buono et usarlo e non
usare secondo la necessità.

{[}7{]} Lasciando adunque adrieto le cose circa uno principe immaginate,
e discorrendo quelle che sono vere, dico che tutti li uomini, quando se
ne parla, e maxime e principi, per essere posti più alti, sono notati di
alcune di queste qualità che arrecano loro o biasimo o laude. {[}8{]} E
questo è che alcuno è tenuto liberale, alcuno misero; usando uno termine
toscano, perché avaro in nostra lingua è ancora colui che per rapina
desidera di avere: misero chiamiamo noi quello che si astiene troppo di
usare il suo; -- alcuno è tenuto donatore, alcuno rapace; alcuno
crudele, alcuno pietoso; {[}9{]} l'uno fedifrago, l'altro fedele; l'uno
effeminato e pusillanime, l'altro feroce et animoso; l'uno umano,
l'altro superbo; l'uno lascivo, l'altro casto; l'uno intero, l'altro
astuto; l'uno duro, l'altro facile; l'uno grave, l'altro leggieri; l'uno
religioso, l'altro incredulo, e simili. {[}10{]} Et io so che ciascuno
confesserà che sarebbe laudabilissima cosa uno principe trovarsi di
tutte le soprascritte qualità, quelle che sono tenute buone. {[}11{]} Ma
perché le non si possono avere tutte né interamente observare, per le
condizioni umane che non lo consentono, è necessario essere tanto
prudente che sappi fuggire l'infamia di quelle vizzi che gli torrebbono
lo stato; e da quegli che non gliene tolgano guardarsi, se gli è
possibile; ma, non possendo, vi si può con meno respetto lasciare
andare. {[}12{]} Et etiam non si curi di incorrere nella infamia di
quelli vizii, sanza e quali possa difficilmente salvare lo stato;
perché, se si considerrà bene tutto, si troverrà qualche cosa che parrà
virtù, e seguendola sarebbe la ruina sua: e qualcuna altra che parrà
vizio, e seguendola ne riesce la sicurtà et il bene essere suo.

\quebra\section{DE LIBERALITATE ET PARSIMONIA
{[}Della liberalità e della parsimonia{]}}

{[}1{]} Cominciandomi adunque alle prime soprascritte qualità dico come
sarebbe bene essere tenuto liberale. {[}2{]} Nondimanco, la liberalità,
usata in modo che tu sia tenuto, ti offende: perché, se ella si usa
virtuosamente e come ella si debbe usare, la non fia conosciuta e non ti
cascherà l'infamia del suo contrario; e però, a volersi mantenere infra
li uomini el nome di liberale, è necessario non lasciare indrieto alcuna
qualità di sumptuosità; talmente che sempre uno principe così fatto
consumerà in simili opere tutte le sue facultà; {[}3{]} e sarà
necessitato alla fine, se si vorrà mantenere el nome del liberale,
gravare li populi extraordinariamente et essere fiscale, e fare tutte
quelle cose che si possono fare per avere danari; il che comincerà a
farlo odioso a' subditi, o poco stimare da ciascuno divenendo povero.
{[}4{]} In modo che, con questa sua liberalità avendo offeso gli assai e
premiato e pochi, sente ogni primo disagio e periclita in qualunque
primo periculo: il che conoscendo lui e volendosene ritrarre, incorre
subito nella infamia del misero. {[}5{]} Uno principe adunque, non
potendo usare questa virtù del liberale, sanza suo danno, in modo che la
sia conosciuta, debba, se gli è prudente, non si curare del nome del
misero: perché col tempo sarà tenuto sempre più liberale veggendo che,
con la sua parsimonia, le sua entrate gli bastano, può difendersi da chi
gli fa guerra, può fare imprese sanza gravare i populi. {[}6{]} Talmente
che viene ad usare liberalità a tutti quelli a chi gli non toglie, che
sono infiniti, e miseria a tutti coloro a chi non dà, che sono pochi.

{[}7{]} Nelli nostri tempi noi non abbiamo veduto fare gran cose se non
a quelli che sono tenuti miseri; li altri, essere spenti. {[}8{]} Papa
Iulio II, come si fu servito del nome del liberale per aggiugnere al
papato, non pensò poi a mantenerselo, per poter fare guerra. {[}9{]} El
re di Francia presente ha fatto tante guerre sanza porre uno dazio
extraordinario a' sua, solum perché alle superflue spese ha
subministrato la lunga parsimonia sua. {[}10{]} El re di Spagna
presente, se fussi tenuto liberale, non arebbe ne fatto né vinto tante
imprese. {[}11{]} Pertanto uno principe debbe existimare poco, -- per
non avere a rubare e subditi, per potere difendersi, per non diventare
povero e contennendo, per non essere forzato di diventare rapace, -- di
incorrere nel nome del misero: perché questo è uno di quelli vizii che
lo fanno regnare. {[}12{]} E se alcuno dicessi: Cesare con la liberalità
pervenne allo imperio, e molti altri, per essere stati et essere tenuti
liberali, sono venuti a gradi grandissimi; rispondo: o tu se' principe
fatto o tu se' in via di acquistarlo. {[}13{]} Nel primo caso questa
liberalità è dannosa. Nel secondo, è bene necessario essere tenuto
liberale; e Cesare era uno di quelli che voleva pervenire al principato
di Roma: ma se, poi che vi fu per venuto, fussi sopravvissuto e non si
fussi temperato da quelle spese, arebbe destrutto quello imperio.

{[}14{]} E se alcuno replicassi: molti sono stati principi e con li
exerciti hanno fatto gran cose, che sono stati tenuti liberalissimi; ti
respondo: o el principe spende del suo e de' sua subditi, o di quello
d'altri. {[}15{]} Nel primo caso, debbe essere parco. Nell'altro, non
debbe lasciare indrieto parte alcuna di liberalità. {[}16{]} E quel
principe che va con li exerciti, che si pasce di prede, di sacchi e di
taglie, maneggia quello di altri, gli è necessaria questa liberalità
altrimenti non sarebbe seguíto da' soldati. {[}17{]} E di quello che non
è tuo o de' subditi tuoi si può essere più largo donatore, come fu Ciro,
Cesare et Alessandro: perché lo spendere quel d'altri non ti toglie
reputazione, ma te ne aggiugne; solamente lo spendere el tuo è quello
che ti nuoce. {[}18{]} E non ci è cosa che consumi sé stessa quanto la
liberalità, la quale mentre che tu usi, perdi la facultà di usarla; e
diventi, o povero e contennendo, o, per fuggire la povertà, rapace et
odioso. {[}19{]} Et intra tutte le cose di che uno principe si debbe
guardare è lo essere contennendo et odioso: e la liberalità all'una e
l'altra cosa ti conduce. {[}20{]} Pertanto è più sapienza tenersi el
nome del misero, che partorisce una infamia sanza odio, che, per volere
el nome del liberale, essere necessitato incorrere nel nome del rapace,
che partorisce una infamia con odio.

\quebra\section{DE CRUDELITATE ET PIETATE; ET AN SIT MELIUS AMARI QUAM TIMERI, VEL E CONTRA
{[}Della crudeltà e pietà, e s'elli è meglio esser amato che temuto, o
più tosto temuto che amato{]}}

{[}1{]} Scendendo appresso alle altre qualità preallegate, dico che
ciascuno principe debbe desiderare di essere tenuto pietoso e non
crudele: nondimanco debbe avvertire di non usare male questa pietà.
{[}2{]} Era tenuto Cesare Borgia crudele; nondimanco quella sua crudeltà
aveva racconcia la Romagna, unitola, ridottola in pace et in fede.
{[}3{]} Il che se si considera bene, si vedrà quello essere stato molto
più pietoso che il populo fiorentino, il quale, per fuggire il nome del
crudele, lasciò distruggere Pistoia. {[}4{]} Debbe pertanto uno principe
non si curare della infamia del crudele per tenere e subditi sua uniti
et in fede: perché con pochissimi exempli sarà più pietoso che quelli e
quali per troppa pietà lasciono seguire e disordini, di che ne nasca
uccisioni o rapine; perché queste sogliono offendere una universalità
intera, e quelle execuzioni che vengano dal principe offendono uno
particulare. {[}5{]} Et intra tutti e principi al principe nuovo è
impossibile fuggire il nome di crudele, per essere gli stati nuovi pieni
di pericoli {[}6{]} E Vergilio nella bocca di Didone dice: \emph{Res
dura et regni novitas me talia cogunt moliri et late fines custode
tueri.} {[}7{]} Nondimanco debbe essere grave al credere et al muoversi,
né si fare paura da sé stesso, e procedere in modo, temperato con
prudenza et umanità, che la troppa confidenzia non lo facci incauto e la
troppa diffidenzia non lo renda intollerabile.

{[}8{]} Nasce da questo una disputa, s'egli è meglio essere amato che
temuto o e converso. {[}9{]} Rispondesi che si vorrebbe essere l'uno e
l'altro; ma perché egli è difficile accozzarli insieme, è molto più
sicuro essere temuto che amato, quando si abbi a mancare dell'uno delli
duoi. {[}10{]} Perché degli uomini si può dire questo generalmente, che
sieno ingrati, volubili, simulatori e dissimulatori, fuggitori de'
pericoli, cupidi del guadagno; e mentre fai loro bene sono tutti tua,
offeronti el sangue, la roba, la vita, e figliuoli, come di sopra dixi,
quando el bisogno è discosto: ma quando ti si appressa, si rivoltono, e
quello principe che si è tutto fondato in su le parole loro, trovandosi
nudo di altre preparazioni, ruina. {[}11{]} Perché le amicizie che si
acquistano col prezzo, e non con grandezza e nobiltà di animo, si
meritano, ma elle non si hanno, et alli tempi non si possano spendere; e
li uomini hanno meno rispetto a offendere uno che si facci amare, che
uno che si facci temere: perché lo amore è tenuto da uno vinculo di
obligo, il quale, per essere gl'uomini tristi, da ogni occasione di
propria utilità è rotto, ma il timore è tenuto da una paura di pena che
non abbandona mai.

{[}12{]} Debbe nondimanco el principe farsi temere in modo che, se non
acquista lo amore, che fugga l'odio; perché può molto bene stare insieme
essere temuto e non odiato. {[}13{]} Il che farà sempre, quando si
abstenga dalla roba de' sua ciptadini e delli sua subditi e dalle donne
loro. E quando pure gli bisognassi procedere contro al sangue di alcuno,
farlo quando vi sia iustificazione conveniente e causa manifesta.
{[}14{]} Ma sopratutto abstenersi dalla roba di altri, perché li uomini
sdimenticano più presto la morte del padre che la perdita del
patrimonio; dipoi, le cagioni del tòrre la roba non mancono mai, e
sempre, colui che comincia a vivere con rapina, truova cagione di
occupare quello di altri: e per avverso contro al sangue sono più rare e
mancono più presto.

{[}15{]} Ma quando el principe è con li exerciti et ha in governo
moltitudine di soldati, allora al tutto è necessario non si curare del
nome del crudele: perché sanza questo nome non si tenne mai exercito
unito né disposto ad alcuna fazione {[}16{]} Intra le mirabili actioni
di Annibale si connumera questa, che, avendo uno exercito grossissimo,
mixto di infinite generazioni di uomini, condotto a militare in terre
aliena, non vi surgessi mai alcuna dissensione, né infra loro, né contro
al principe, così nella captiva come nella sua buona fortuna. {[}17{]}
Il che non possé nascere da altro che da quella sua inumana crudeltà: la
quale, insieme con infinite sua virtù, lo fece sempre nel conspetto de'
sua soldati venerando e terribile. {[}18{]} E sanza quella, a fare
quello effetto, l'altre sua virtù non li bastavano: e li scriptori, in
questo poco considerati dall'una parte admirano questa sua actione,
dall'altra dannano la principale cagione di epsa.

{[}19{]} E che sia vero che le altre sua virtù non sarebbano bastate, si
può considerare in Scipione, rarissimo non solamente ne' tempi sua ma in
tutta la memoria delle cose che si sanno, dal quale li exerciti sua in
Ispagna si ribellorono; è che non nacque da altro che dalla troppa sua
pietà, la quale aveva data alli suoi soldati più licenza che alla
disciplina militare non si conveniva. {[}20{]} La qual cosa gli fu da
Fabio Maximo in Senato rimproverata e chiamato da lui corruptore della
romana milizia. {[}21{]} E Locrensi, essendo suti da uno legato di
Scipione destrutti, non furono vendicati, né fu da lui la insolenza di
quello legato corretta tutto nascendo da quella sua natura facile;
talmente che, volendolo alcuno in excusare Senato, dixe come gli erano
di molti uomini che sapevano meglio non errare che correggere gli
errori. {[}22{]} La qual natura arebbe col tempo violato la fama e la
gloria di Scipione, se gli avessi con epsa perseverato nello imperio:
ma, vivendo sotto il governo del Senato, questa sua qualità dannosa non
solum si nascose, ma gli fu a gloria.

{[}22{]} Concludo adunque, tornando allo essere temuto et amato, che,
amando li uomini a posta loro e temendo a posta del principe, debbe uno
principe savio fondarsi in su quello che è suo, non in su quello che è
d'altri; debbe solamente ingegnarsi di fuggire lo odio, come è detto.

\quebra\section{QUOMODO FIDES A PRINCIPIBUS SIT SERVANDA
{[}In che modo e' principi abbino a mantenere la fede{]}}

{[}1{]} Quanto sia laudabile in uno principe il mantenere la fede e
vivere con integrità e non con astuzia, ciascuno lo intende; nondimanco
si vede per esperienza nelli nostri tempi quelli principi avere fatto
gran cose, che della fede hanno tenuto poco conto e che hanno saputo con
l'astuzia aggirare e' cervelli delli uomini; et alla fine hanno superato
quelli che si sono fondati in sulla realtà.

{[}2{]} Dovete adunque sapere come e' sono dua generazione di
combattere: l'uno, con le leggi; l'altro, con la forza. {[}3{]} Quel
primo è proprio dello uomo; quel secondo delle bestie. {[}4{]} Ma perché
el primo molte volte non basta, conviene ricorrere al secondo: pertanto
ad uno principe è necessario sapere bene usare la bestia e lo uomo.
{[}5{]} Questa parte è suta insegnata alli principi copertamente dalli
antichi scriptori, li quali scrivono come Achille e molti altri di
quelli principi antichi, furono dati a nutrire a Chirone centauro, che
sotto la sua disciplina li custodissi. {[}6{]} Il che non vuole dire
altro, avere per preceptore uno mezzo bestia e mezzo uomo, se non che
bisogna ad uno principe sapere usare l'una e l'altra natura; e l'una
sanza l'altra non è durabile.

{[}7{]} Sendo dunque necessitato uno principe sapere bene usare la
bestia, debbe di quelle pigliare la volpe e il lione; perché el lione
non si defende da' lacci, la volpe non si difende da' lupi; bisogna
adunque essere volpe a conoscere e lacci, e lione a sbigottire e lupi:
coloro che stanno semplicemente in sul lione, non se ne intendono.
{[}8{]} Non può pertanto uno signore prudente, ne debbe, osservare la
fede quando tale observanzia gli torni contro e che sono spente le
cagioni che la feciono promettere. {[}9{]} E se li uomini fussino tutti
buoni, questo precetto non sarebbe buono; ma perché sono tristi e non la
observarebbono a te, tu etiam non l'hai ad observare a loro; né mai ad
uno principe mancorono cagioni legittime di colorire la inobservanzia.
{[}10{]} Di questo se ne potrebbe dare infiniti exempli moderni e
monstrare quante pace, quante promesse sono state fatte irrite e vane
per la infedelità de' principi: e quello che ha saputo meglio usare la
volpe, è meglio capitato. {[}11{]} Ma è necessario questa natura saperla
bene colorire et essere gran simulatore e dissimulatore: e sono tanto
semplici gli uomini, e tanto ubbediscano alle necessità presenti, che
colui che inganna troverrà sempre chi si lascerà ingannare.

{[}12{]} Io non voglio delli esempli freschi tacerne uno. Alessandro VI
non fece mai altro, non pensò mai ad altro che ad ingannare uomini; e
sempre trovò subietto da poterlo fare: e non fu mai uomo che avessi
maggiore efficacia in asseverare, e con maggiori iuramenti affermassi
una cosa, che l'observassi meno; nondimeno sempre gli succederono
gl'inganni ad votum, perché conosceva bene questa parte del mondo.

{[}13{]} A uno principe adunque non è necessario avere in fatto tutte le
soprascritte qualità, ma è ben necessario parere di averle; anzi ardirò
di dire questo: che avendole et observandole sempre, sono dannose, e,
parendo di averle, sono utili; come parere piatoso, fedele, umano,
intero, relligioso, et essere: ma stare in modo edificato con lo animo
che, bisognando non essere, tu possa e sappia diventare il contrario.
{[}14{]} Et hassi ad intendere questo, che uno principe e maxime uno
principe nuovo non può observare tutte quelle cose per le quali gli
uomini sono chiamati buoni, sendo spesso necessitato, per mantenere lo
stato, operare contro alla fede, contro alla carità, contro alla
umanità, contro alla religione. {[}15{]} E però bisogna che egli abbia
uno animo disposto a volgersi secondo che e venti della fortuna e la
variazioni delle cose gli comandano; e, come di sopra dixi, non partirsi
dal bene, potendo, ma sapere entrare nel male, necessitato.

{[}16{]} Debba adunque uno principe avere gran cura che non gli esca mai
di bocca cosa che non sia piena delle soprascritte cinque qualità; e
paia, ad udirlo et vederlo, tutto pietà, tutto fede, tutto integrità,
tutto relligione e non è cosa più necessaria, a parere di avere, che
questa ultima qualità. {[}17{]} E li uomini in universali iudicano più
alli occhi che alle mani; perché tocca a vedere ad ognuno, a sentire a
pochi: ognuno vede quello che tu pari, pochi sentono quello che tu se';
e quelli pochi non ardiscano opporsi alla opinione di molti che abbino
la maestà dello stato che li difenda: e nelle actione di tutti li
uomini, e maxime de' principi, dove non è iudizio a chi reclamare, si
guarda al fine.

{[}18{]} Facci dunque uno principe di vincere e mantenere lo stato: e
mezzi sempre fieno iudicati onorevoli e da ciascuno saranno laudati;
perché el vulgo ne va preso con quello che pare e con lo evento della
cosa: e nel mondo non è se non vulgo, e' pochi ci hanno luogo quando gli
assai hanno dove appoggiarsi. {[}19{]} Alcuno principe de' presenti
tempi, il quale non e bene nominare, non predica mai altro che pace e
fede, e dell'una e dell'altra è inimicissimo; e l'una e l'altra, quando
egli l'avessi osservata, egli arebbe più volte tolto e la reputazione e
lo stato.

\quebra\section{DE CONTEMPTU ET ODIO FUGIENDO
{[}In che modo si abbia a fuggire lo essere sprezzato e odiato{]}}

{[}1{]} Ma perché, circa le qualità di che di sopra si fa menzione, io
ho parlato delle più importanti, l'altre voglio discorrere brevemente
sotto queste generalità: che el principe pensi, come in parte di sopra è
detto, di fuggire quelle cose che lo faccino odioso o contennendo; e
qualunque volta egli fuggirà questo, arà adempiuto le parte sua e non
troverrà nelle altre infamie periculo alcuno. {[}2{]} Odioso sopratutto
lo fa, come io dissi, essere rapace et usurpatore della roba e delle
donne de' subditi: da che si debba abstenere. {[}3{]} E qualunque volta
alle universalità delli uomini non si toglie né onore né roba, vivono
contenti: e solo si ha a combattere con la ambizione de pochi, la quale
in molti modi e con facilità si raffrena. {[}4{]} Contennendo lo fa
essere tenuto vario, leggieri, effeminato, pusillanime, irresoluto: da
che uno principe si debbe guardare come di uno scoglio, et ingegnarsi
che nelle actioni sua si riconosca grandezza, animosità, gravità,
fortezza; e circa a' maneggi privati trà subditi volere che la sua
sentenza sia inrevocabile; e si mantenga in tale opinione che alcuno non
pensi né ad ingannarlo né ad aggirarlo.

{[}5{]} Quel principe che dà di sé questa opinione è reputato assai, e
contro a chi è reputato con difficultà si congiura, con difficultà è
assaltato, purché s'intenda che sia excellente e che sia reverito da'
sua. {[}6{]} Perché uno principe debba avere dua paure: una dentro, per
conto de' subditi; l'altra di fuori, per conto de' potentati externi.
{[}7{]} Da questa si difende con le buone arme e con li buoni amici: e
sempre, se arà buone arme, arà buoni amici. {[}8{]} E sempre staranno
ferme le cose di dentro, quando stieno ferme quelle di fuora, se già le
non fussino perturbate da una congiura: e quando pure quelle di fuora
movessino, s'egli è ordinato e vissuto come ho detto, quando egli non si
abbandoni, sosterrà sempre ogni impeto, come io dixi che fece Nabide
spartano.

{[}9{]} Ma circa subditi, quando le cose di fuora non muovino, si ha a
temere che non coniurino secretamente; di che el principe si assicura
assai fuggendo lo essere odiato o disprezzato, e tenendosi el populo
satisfatto di lui: il che è necessario conseguire, come di sopra a lungo
si disse. {[}10{]} Et uno de' più potenti remedii che abbia uno principe
contro alle congiure, è non essere odiato dallo universale: perché
sempre chi coniura crede con la morte del principe satisfare al populo,
ma quando creda offenderlo non piglia animo a prendere simile partito.
{[}11{]} Perché le difficultà che sono dalla parte de' congiuranti sono
infinite, e per esperienza si vede molte essere state le congiure e
poche avere avuto buono fine. {[}12{]} Perché chi congiura non può
essere solo, né può prendere compagnia se non di quelli che creda essere
malcontenti: e subito che a uno malcontento tu hai scoperto lo animo
tuo, gli dai materia a contentarsi, perché manifestamente lui ne può
sperare ogni commodità; talmente che, veggendo il guadagno sicuro da
questa parte, e dall'altra veggendolo dubio e pieno di periculo,
conviene bene o che sia raro amico o che sia al tutto ostinato inimico
del principe, ad observarti la fede

{[}13{]} E per ridurre la cosa in brevi termini, dico che dalla parte
del coniurante non è se non paura, gelosia e sospecto di pena che lo
sbigottisce: ma dalla parte del principe è la maestà del principato, le
legge, le difese delli amici e dello stato che lo difendono. {[}14{]}
Talmente che, adgiunto a tutte queste cose la benivolenzia populare, è
impossibile che alcuno sia sì temerario che congiuri: perché, dove per
lo ordinario, uno coniurante ha a temere innanzi alla execuzione del
male, in questo caso debbe temere ancora poi, avendo per nimico el
populo, seguito lo excesso, né potendo per questo sperare refugio
alcuno.

{[}15{]} Di questa materia se ne potria dare infiniti esempli, ma voglio
solo essere contento di uno seguito a' tempo de' padri nostri. {[}16{]}
Messere Annibale Bentivogli, avolo del presente messer Annibale, che era
principe di Bologna, sendo da' Canneschi, che gli coniurorono contro,
ammazzato, né rimanendo di lui altri che messere Giovanni, quale era in
fasce, subito dopo tale omicidio si levò el populo et ammazzò tutti e
Canneschi. {[}17{]} Il che nacque dalla benivolenzia populare che la
Casa de' Bentivogli aveva in quelli tempi: la quale fu tanta che, non
restando di quella alcuno, in Bologna, che potessi, morto Annibale,
reggere lo stato, et avendo indizio come in Firenze era uno nato de'
Bentivogli, che si teneva fino allora figliuolo di uno fabbro, vennono e
Bolognesi per quello in Firenze e gli dettono il governo di quella
città; la quale fu governata da lui fino a tanto che messer Giovanni
pervenissi in età conveniente al governo.

{[}18{]} Concludo pertanto che uno principe debbe tenere delle congiure
poco conto, quando il popolo gli sia benivolo: ma quando gli sia nimico
et abbilo in odio, debbe temere d'ogni cosa e d'ognuno. {[}19{]} E gli
stati bene ordinati e li principi savi hanno con ogni diligenzia pensato
di non disperare e grandi e satisfare al populo e tenerlo contento:
perché questa è una delle più importanti materie che abbi uno principe.

{[}20{]} Intra e regni bene ordinati e governati a' tempi nostri è
quello di Francia, et in epso si truovano infinite constituzioni buone
donde depende la libertà e sicurtà del re: delle quali la prima è il
Parlamento e la sua autorità. {[}21{]} Perché quello che ordinò quello
regno, conoscendo l'ambizione de' potenti e la insolenzia loro, e
iudicando essere loro necessario uno freno in bocca che gli correggessi;
e da altra parte conoscendo l'odio dello universale contro a' grandi
fondato in su la paura, e volendo assicurargli, -- non volle che questa
fussi particulare cura del re, per tòrgli quel lo carico che potessi
avere con li grandi favorendo e populari, e co' populari favorendo e
grandi.

{[}22{]} E però constituí uno iudice terzo, che fussi quello che sanza
carico del re battessi e grandi e favorissi e minori: né poté essere
questo ordine migliore né più prudente, né che sia maggiore cagione
della sicurtà del re e del regno. {[}23{]} Di che si può trarre un altro
notabile: che e principi le cose di carico debbono fare subministrare ad
altri, quelle di grazia a loro medesimi. {[}24{]} Ed nuovo concludo che
uno principe debbe stimare e grandi, ma non si fare odiare dal populo.

{[}25{]} Parrebbe forse a molti, considerato la vita e morte di alcuno
imperatore romano, che fussino exempli contrarii a questa mia opinione,
trovando alcuno essere vissuto sempre egregiamente e mostro gran virtù
d'animo: nondimeno avere perso lo imperio, o vero essere stato morto da'
sua che gli hanno congiurato contro. {[}26{]} Volendo pertanto
rispondere a queste obiectioni, discorrerò le qualità di alcuni
imperatori, mostrando le cagioni della loro ruina non disforme da quello
che da me si è addutto; e parte metterò in considerazione quelle cose
che sono notabili a chi legge le actioni di quelli tempi. {[}27{]} E
voglio mi basti pigliare tutti quelli imperatori che succederono allo
imperio da Marco filosofo a Maximino, li quali furono: Marco, Commodo
suo figliuolo, Pertinace, Iuliano, Severo, Antonino Caracalla suo
figliuolo, Macrino, Eliogabal, Alessandro e Maximino. {[}28{]} Et è
prima da notare che, dove nelli altri principati si ha solo a contendere
con la ambizione de' grandi et insolenzia de' populi, gl'imperatori
romani avevano una terza difficultà, di avere a sopportare la crudeltà
et avarizia de' soldati. {[}29{]} La quale cosa era sì difficile che la
fu cagione della ruina di molti, sendo difficile satisfare a' soldati et
a' populi; perché e populi amavano la quiete, e per questo e principi
modesti erano loro grati e li soldati amavono el principe d'animo
militare e che fussi crudele insolente, e rapace: le quali cose volevano
che lui exercitassi ne' populi, per potere avere duplicato stipendio e
sfogare la loro avarizia e crudeltà. {[}30{]} Le quali cose feciono che
quelli imperatori che per natura o per arte non avevano una gran
reputazione, tale che con quella e' tenessino l'uno e l'altro in freno,
sempre ruinavono. {[}31{]} E li più di loro, maxime di quegli che come
uomini nuovi venivono al principato, conosciuta la difficultà di questi
due diversi umori, si volgevano a satisfare a' soldati, stimando poco lo
iniuriare el populo. {[}32{]} Il quale partito era necessario: perché,
non potendo e principi mancare di non essere odiati da qualcuno, si
debbano sforzare prima di non essere odiati dalla università, e quando
non possono conseguire questo, debbono fuggire con ogni industria l'odio
di quelle università che sono più potenti. {[}33{]} E però quelli
imperatori che per novità avevano bisogno di favori extraordinarii, si
aderivano a' soldati più tosto che a' populi: il che tornava nondimeno
loro utile, o no, secondo che quel principe si sapeva mantenere reputato
con epso loro.

{[}34{]} Da queste cagioni sopradette nacque che Marco, Pertinace et
Alessandro, sendo tutti di modesta vita, amatori della iustizia, inimici
della crudeltà, umani, benigni, ebbono tutti, da Marco in fuora, tristo
fine. {[}35{]} Marco solo visse e morí onoratissimo, perché lui succedé
allo imperio iure hereditario e non aveva a riconoscere quello né da'
soldati né da' populi; dipoi, essendo accompagnato da molte virtù che lo
facevano venerando, tenne sempre, mentre che visse, l'uno e l'altro
ordine intra e termini suoi, e non fu mai né odiato né disprezzato.
{[}36{]} Ma Pertinace, creato imperatore contro alla voglia de' soldati,
li quali essendo usi a vivere licenziosamente sotto Commodo non poterono
sopportare quella vita onesta alla quale Pertinace gli voleva ridurre;
onde avendosi creato odio et a questo odio aggiunto el disprezzo sendo
vecchio ruinò ne' primi principii della sua administrazione. {[}37{]} E
qui si debbe notare che l'odio si acquista così mediante le buone opere,
come le triste: e però, come io dixi di sopra, uno principe, volendo
mantenere lo stato, è spesso forzato a non essere buono. {[}38{]}
Perché, quando quella università, o populo o soldati o grandi che sieno,
della qual tu iudichi avere per mantenerti, piu bisogno è corrotta, ti
conviene seguire l'umore suo per satisfarlo: et allora le buone opere ti
sono nimiche.

{[}39{]} Ma vegnamo ad Alessandro: il quale fu di tanta bontà che, intra
le altre laude che gli sono attribuite, è questa, che in 14 anni che
tenne l'imperio non fu mai morto da lui alcuno iniudicato: nondimanco,
sendo tenuto effeminato et uomo che si lasciassi governare alla madre, e
per questo venuto in disprezzo, conspirò in lui l'exercito et
ammazzollo.

{[}40{]} Discorrendo ora per opposito le qualità di Commodo, di Severo,
di Antonino Caracalla e Maximino, gli troverrete crudelissimi e
rapacissimi: li quali, per satisfare a' soldati, non perdonorono ad
alcuna qualità di iniuria che ne' populi si potessi commettere. {[}41{]}
E tutti excetto Severo ebbono triste fine; perché in Severo fu tanta
virtù che, mantenendosi e soldati amici, ancora che populi fussino da
lui gravati, poté sempre regnare felicemente: perché quelle sua virtù lo
facevano nel conspetto de' soldati e delli populi sì mirabile che questi
rimanevano quodammodo stupidi e attoniti, e quelli altri reverenti e
satisfatti. {[}42{]} E perché le actioni di costui furono grande e
notabile in uno principe nuovo, io voglio brevemente monstrare quanto é
seppe bene usare la persona del lione e della volpe, le quali nature io
dico di sopra essere necessario imitare a uno principe.

{[}43{]} Conosciuto Severo la ignavia di Iuliano imperatore, persuase al
suo exercito, del quale era in Stiavonia capitano, che egli era bene
andare a Roma a vendicare la morte di Pertinace, il quale da' soldati
pretoriani era suto morto. {[}44{]} E sotto questo colore, sanza
monstrare di aspirare allo imperio, mosse lo exercito contro a Roma e fu
prima in Italia che si sapessi la sua partita. {[}45{]} Arrivato a Roma,
fu dal Senato per timore eletto imperatore e morto Iuliano. {[}46{]}
Restava dopo questo principio a Severo dua difficultà, volendosi
insignorire di tutto lo stato: l'una in Asia, dove Nigro, capo delli
exerciti asiatici, si era fatto chiamare imperatore; e l'altra in
Ponente, dove era Albino quale ancora lui aspirava allo imperio.
{[}47{]} E perché iudicava periculoso scoprirsi inimico a tutti a dua,
deliberò di assaltare Nigro et ingannare Albino: al quale scripse come,
sendo dal Senato electo imperatore, voleva partecipare quella dignità
con lui; e mandogli il titulo di Cesare e per diliberazione del Senato
se lo aggiunse conlega: le quali cose furno da Albino acceptate per
vere. {[}48{]} Ma poi ché Severo ebbe vinto e morto Nigro e pacate le
cose orientali, ritornatosi a Roma, si querelò in Senato come Albino,
poco conoscente de' benefizii ricevuti da lui, aveva dolosamente cerco
di ammazzarlo: e per questo era necessitato di andare a punire la sua
ingratitudine; dipoi andò a trovare infrancia, e gli tolse lo stato e la
vita. {[}49{]} E chi examinerà tritamente le actione di costui, lo
troverrà uno ferocissimo lione et una astutissima golpe, e vedrà quello
temuto e reverito da ciascuno e dalli exerciti non odiato; e non si
maraviglierà se lui, uomo nuovo, arà potuto tenere tanto imperio, perché
la sua grandissima reputazione lo difese sempre da quello odio che li
populi per le sue rapine avevano potuto concipere.

{[}50{]} Ma Antonino suo figliuolo fu ancora lui uomo che aveva parte
excellentissime e che lo facevano maraviglioso nel conspetto de' populi
e grato a' soldati, perché lui era uomo militare, sopportantissimo
d'ogni fatica, disprezzatore d'ogni cibo dilicato e d'ogni altra
mollizie: la qual cosa lo faceva amare da tutti li exerciti. {[}51{]}
Nondimanco la sua ferocia e crudeltà fu tanta e sì inaudita, per avere
dopo infinite occisioni particulari morto gran parte del populo di Roma
e tutto quello di Alessandria, che diventò odiosissimo a tutto il mondo
e cominciò ad essere temuto etiam da quelli che lui aveva dintorno: in
modo che fu ammazzato da uno centurione in mezzo del suo exercito.
{[}52{]} Dove è da notare che queste simili morte, le quali seguano per
diliberazione di uno animo obstinato, sono da' principi inevitabili,
perché ciascuno che non si curi di morire lo può offendere: ma debba
bene el principe temerne meno, perché le sono rarissime. {[}53{]} Debba
solo guardarsi di non fare grave ingiuria ad alcuno di coloro di chi si
serve e che egli ha dintorno al servizio del suo principato; come aveva
fatto Antonino, il quale aveva morto contumeliosamente uno fratello di
quello centurione e lui ogni giorno minacciava, tamen lo teneva a
guardia del corpo suo: il che era partito temerario e da ruinarvi, come
gl'intervenne.

{[}54{]} Ma vegnamo a Commodo, al quale era facilità grande tenere
l'imperio per averlo iure hereditario, sendo figliuolo di Marco: e solo
gli bastava seguire le vestigie del padre, et a' soldati et a' populi
arebbe satisfatto. {[}55{]} Ma essendo di animo crudele e bestiale, per
potere usare la sua rapacità ne' populi, si volse ad intrattenere li
exerciti e fargli licenziosi: dall'altra parte, non tenendo la sua
dignità, discendendo spesso ne' teatri a combattere con gladiatori e
facendo altre cose vilissime e poco degne della maestà imperiale,
diventò contennendo nel conspetto de' soldati. {[}56{]} Et essendo
odiato da l'una parte e disprezzato dall'altra, fu conspirato in lui e
morto.

{[}57{]} Restaci a narrare le qualità di Maximino. Costui fu uomo
bellicosissimo, et essendo gli exerciti infastiditi della mollizie di
Alessandro, del quale ho di sopra discorso, morto lui lo elessono allo
imperio; il quale non molto tempo possedé, perché dua cose lo feciono
odioso e contennendo. {[}58{]} L'una, essere vilissimo per avere già
guardato le pecore in Tracia: la qual cosa era per tutto notissima, li
chi faceva una grande dedignazione nel conspetto di qualunque. {[}59{]}
L'altra, perché, avendo nello ingresso del suo principato differito lo
andare a Roma et intrare nella possessione della sedia imperiale, aveva
dato di sé opinione di crudelissimo, avendo per li suoi prefetti in Roma
e in qualunque luogo dello imperio exercitato molte crudeltà. {[}60{]}
Talmente che, commosso tutto il mondo dallo sdegno per la viltà del suo
sangue e dall'odio per la paura della sua ferocia, si ribellò prima
Affrica, dipoi el Senato, con tutto el populo di Roma e tutta la Italia,
gli conspirò contro; a che si aggiunse el suo proprio exercito, quale,
campeggiando Aquileia e trovando difficultà nella expugnazione,
infastidito dalla crudeltà sua e, per vedergli tanti nimici, temendolo
meno, lo ammazzò.

{[}61{]} Io non voglio ragionare né di Eliogabalo né di Macrino né di
Iuliano, e quali per essere al tutto contennendi si spensono subito, ma
verrò alla conclusione di questo discorso; e dico che li principi de'
nostri tempi hanno meno questa difficultà di satisfare
extraordinariamente a' soldati ne' governi loro: perché, non obstante
che si abbia ad avere a quegli qualche considerazione, tamen si resolve
presto per non avere, alcuno di questi principi exerciti insieme che
sieno inveterati con li governi et administrazione delle provincie, come
erano gli exerciti dello imperio romano. {[}62{]} E però, se allora era
necessario satisfare più alli soldati che a' populi, perché e soldati
potevano più che e populi, ora è più necessario a tutti e principi,
excepto che al Turco et al Soldano, satisfare a' populi che a' soldati,
perché e' populi possono più di quelli. {[}63{]} Di che io ne exceptuo
el Turco, tenendo quello continuamente insieme intorno a sé XII mila
fanti e 15 mila cavagli, da' quali dipende la securtà e la fortezza del
suo regno: et è necessario che, posposto ogni altro respetto, quel
signore se li mantenga amici. {[}63{]} Similmente el regno del Soldano
sendo tutto in nelle mani de' soldati, conviene che ancora lui sanza
respetto de' populi se li mantenga amici. {[}65{]} Et avete a notare che
questo stato del Soldano è disforme a tutti li altri principati, perché
egli è simile al pontificato cristiano, il quale non si può chiamare né
principato ereditario né principato nuovo: perché non e figliuoli del
principe vecchio sono eredi e rimangono signori, ma colui che è eletto a
quello grado da quegli che ne hanno autorità; {[}66{]} et essendo questo
ordine antiquato, non si può chiamare principato nuovo; per che in
quello non sono alcune di quelle difficultà che sono ne' nuovi; perché,
se bene el principe è nuovo, gli ordini di quello stato sono vecchi et
ordinati a riceverlo come se fussi loro signore ereditario.

{[}67{]} Ma torniamo alla materia nostra. Dico che qualunque considerrà
el soprascritto discorso, vedrà o l'odio o il disprezzo essere suti
cagione della ruina di quelli imperatori prenominati; e conoscerà ancora
donde nacque che, parte di loro procedendo in uno modo e parte al
contrario, in qualunque di quegli uno di loro ebbe felice e gli altri
infelice fine. {[}68{]} Perché a Pertinace et Alessandro, per essere
principi nuovi, fu inutile e dannoso volere imitare Marco, che era nel
principato iure hereditario; e similmente a Caracalla, Commodo e
Maximino essere stata cosa perniziosa imitare Severo, per non avere
avuta tanta virtù che bastassi a seguitare le vestigie sua. {[}69{]}
Pertanto uno principe nuovo in uno principato nuovo non può imitare le
actioni di Marco, né ancora è necessario seguitare quelle di Severo: ma
debbe pigliare da Severo quelle parte che per fondare el suo stato sono
necessarie, e da Marco quelle che sono convenienti e gloriose a
conservare uno stato che sia già stabilito e fermo.

\quebra\section{AN ARCES ET MULTA ALIA, QUAE QUOTTIDIE A PRINCIPIBUS FIUNT, UTILIA AN INUTILLIA SINT
{[}Se le fortezze e molte altre cose che ogni giono si fanno da'
principi, per conservazione del loro stato, sono utili o no{]}}

{[}1{]} Alcuni principi per tenere sicuramente lo stato hanno disarmati
e loro subditi; alcuni hanno tenuto divise le terre subiette. {[}2{]}
Alcuni hanno nutrito inimicizie contro a sé medesimi; alcuni altri si
sono volti a guadagnarsi quelli che li erano suspetti nel principio del
suo stato. {[}3{]} Alcuni hanno edificato fortezze; alcuni le hanno
ruinate e destrutte. {[}4{]} E benché di tutte queste cose non si possa
dare determinata sentenza, se non si viene a' particulari di quegli
stati dove si avessi a pigliare alcuna simile deliberazione, nondimanco
io parlerò in quell modo largo che la materia per sé medesima sopporta.

{[}5{]} Non fu mai adunque che uno principe nuovo disarmassi li suoi
subditi: anzi, quando gli ha trovati disarmati, sempre gli ha armati;
perché, armandosi, quelle arme diventano tua, diventano fedeli quelli
che ti sono sospetti, e quelli che erano fedeli si mantengono, e di
subditi si fanno tua partigiani. {[}6{]} E perché tutti li subditi non
si possono armare, quando si beneficano quegli che tu armi, con gli
altri si può fare più a sicurtà: e quella diversità del procedere, che
conoscono in loro, gli fa tua obligati; quelli altri ti scusano,
iudicando essere necessario quegli avere più merito che hanno più
periculo e più obligo. {[}7{]} Ma quando tu gli disarmi, tu cominci ad
offendergli: monstri che tu abbi in loro diffidenzia, o per viltà o per
poca fede, e l'una e l'altra di queste opinioni concepe odio contro di
te; e perché tu non puoi stare disarmato, conviene ti volti alla milizia
mercennaria, la quale è di quella qualità che di sopra è detto: e quando
la fussi buona, non può essere tanta che ti difenda da inimici potenti e
da'subditi sospecti. {[}8{]} Però, come io ho detto, uno principe nuovo,
in uno principato nuovo, sempre vi ha ordinato l'arme: di questi exempli
sono piene le storie. {[}9{]} Ma quando uno principe acquista uno stato
nuovo, che come membro si aggiunga al suo vecchio, allora è necessario
disarmare quello stato, excepto quegli che nello acquistarlo sono suti
tua partigiani: e quegli ancora col tempo e con le occasioni è
necessario renderli molli et effeminati, et ordinarsi in modo che solo
le arme di tutto il tuo stato sieno in quelli tuoi soldati proprii che
nello stato tuo antico vivevano appresso di te.

{[}10{]} Solevano li antiqui nostri, e quelli che erano stimati savi,
dire come era necessario tenere Pistoia con le parti e Pisa con le
fortezze; e per questo nutrivano in qualche terra loro subdita le
differenzie, per possederle più facilmente. {[}11{]}Questo, in quelli
tempi che Italia era in uno certo modo bilanciata, doveva essere bene
fatto: ma non credo già che si possa dare oggi per precepto; perché io
non credo che le divisioni facessino mai bene alcuno: anzi è necessario,
quando el nimico si accosta, che le città divise si perdino subito,
perché sempre la parte più debile si aderirà alle forze externe e
l'altra non potrà reggere.

{[}12{]} Viniziani, mossi come io credo dalle ragioni soprascripte,
nutrivano le sette guelfe e ghibelline nelle città loro subdite; e
benché non li lasciassino mai venire al sangue, tamen nutrivano tra loro
questi dispareri acciò che, occupati quelli cittadini in quelle loro
differenzie, non si unissino contro di loro. {[}13{]} Il che, come si
vide, non tornò loro poi a proposito: perché, sendo rotti a Vailà,
subito una parte di quelle prese ardire e tolsono loro tutto lo stato.
{[}14{]} Arguiscano pertanto simili modi debolezza del principe, perché
in uno principato gagliardo mai si permetteranno simili divisioni:
perché le fanno solo profitto a tempo di pace, potendosi mediante quelle
più facilmente maneggiare e subditi, ma, venendo la guerra, monstra
simile ordine la fallacia sua.

{[}15{]} Senza dubbio e principi diventano grandi quando superano le
difficultà e le opposizioni che sono fatte loro; e però la fortuna,
maxime quando vuole fare grande uno principe nuovo, il quale ha maggiore
necessità di acquistare reputazione che uno ereditario, gli fa nascere
de' nimici e fagli fare delle imprese contro, acciò che quello abbi
cagione di superarle e, su per quella scala che gli hanno pòrta li
inimici suoi, salire più alto. {[}16{]} Però molti iudicano che uno
principe savio debbe, quando egli ne abbia la occasione, nutrirsi con
astuzia qualche inimicizia, acciò che, oppresso quella, ne seguiti
maggior sua grandezza

{[}17{]} Hanno e principi, e praesertim quegli che sono nuovi, trovato
più fede e più utilità in quelli uomini che nel principio del loro stato
sono suti tenuti sospetti, che in quelli che erano nel principio
confidenti. {[}18{]} Pandolfo Petrucci, principe di Siena, reggeva lo
stato suo più con quelli che gli furono sospetti che con li altri.
{[}19{]} Ma di questa cosa non si può parlare largamente, perché la
varia secondo el subietto; solo dirò questo, che quelli uomini che nel
principio di uno principato sono stati inimici, che sono di qualità che
a mantenersi abbino bisogno di appoggiarsi, sempre el principe con
facilità grandissima se gli potrà guadagnare: e loro maggiormente sono
forzati a servirlo con fede, quanto conoscano esseree loro più
necessario cancellare con le opere quella opinione sinistra che si aveva
di loro. {[}20{]} E così el principe ne trae sempre più utilità, che di
coloro che, servendolo con troppa sicurtà, straccurono le cose sua.

{[}21{]} E poiché la materia lo ricerca, non voglio lasciare indrieto
ricordare alli principi che hanno preso uno stato di nuovo, mediante e
favori intrinsichi di quello, che considerino bene qual cagione abbi
mosso quegli che lo hanno favorito, a favorirlo. {[}22{]} E se ella non
è affectione naturale verso di loro, ma fussi solo perché quelli non si
contentavano di quello stato, con fatica e difficultà grande se gli
potrà mantenere amici: perché fia impossibile che lui possa
contentargli. {[}23{]} E discorrendo bene, con quelli exempli che dalle
cose antiche e moderne si traggono, la cagione di questo, vedrà essergli
molto più facile guadagnarsi amici quegli uomini che dello stato innanzi
si contentavono, e però erano sua inimici, che quegli che, per non se ne
contentare, gli diventorono amici e favorironlo ad occuparlo.

{[}24{]} È suta consuetudine de' principi, per potere tenere più
sicuramente lo stato loro, edificare fortezze che sieno la briglia e il
freno di quelli che disegnassino fare loro contro, et avere uno refugio
sicuro da uno subito impeto. {[}25{]} Io laudo questo modo perché egli è
usitato ab antiquo: nondimanco messer Niccolò Vitelli, ne' tempi nostri,
si è visto disfare dua fortezze in Città di Castello per tenere quello
stato; Guido Ubaldo, duca di Urbino, ritornato nella sua dominazione
donde da Cesare Borgia era suto cacciato, ruinò funditus tutte le
fortezze di quella sua provincia e iudicò sanza quelle più difficilmente
riperdere quello stato; Bentivogli, ritornati in Bologna, usorono simili
termini. {[}26{]} Sono dunque le fortezze utili, o no, secondo e tempi:
e se le ti fanno bene in una parte, ti offendano in un'altra. {[}27{]} E
puossi discorrere questa parte così: quel principe che ha più paura de'
populi che de' forestieri, debbe fare le fortezze; ma quello che ha più
paura de' forestieri che de' populi, debbe lasciarle indietro. {[}28{]}
Alla Casa Sforzesca ha fatto e farà più guerra el castello di Milano,
che vi edificò Francesco Sforza, che veruno altro disordine di quello
stato. {[}29{]} Però la migliore fortezza che sia, è non essere odiato
dal populo; perché, ancora che tu abbi le fortezze et il populo ti abbia
in odio, le non ti salvono: perché non mancano mai a' populi, preso che
gli hanno l'arme, forestieri che gli soccorrino. {[}30{]} Nelli tempi
nostri non si vede che quelle abbino profittato ad alcuno principe, se
non alla contessa di Furlí, quando fu morto il conte Ieronimo suo
consorte: perché mediante quella possé fuggire l'impeto populare et
aspettare el soccorso da Milano e recuperare lo stato; e li tempi
stavano allora in modo che il forestiere non posseva soccorrere il
populo. {[}31{]} Ma dipoi valsono ancora a lei poco le fortezze, quando
Cesare Borgia l'assaltò e che il populo, suo inimico, si congiunse col
forestieri. {[}32{]} Pertanto allora e prima sarebbe suto più sicuro a
lei non essere odiata dal populo, che avere le fortezze. {[}33{]}
Considerato adunque tutte queste cose, io lauderò chi farà le fortezze e
chi non le farà; e biasimerò qualunque, fidandosi delle fortezze,
stimerà poco essere odiato da' populi.

\quebra\section{QUOD PRINCIPEM DECEAT UT EGREGIUS HABEATUR
{[}Quello che s' appartenga fare a uno principe per esserre stimato e
reputato{]}}

{[}1{]} Nessuna cosa fa tanto stimare uno principe, quanto fanno le
grande imprese e dare di sé rari exempli. {[}2{]} Noi abbiamo nelli
nostri tempi Ferrando di Aragona, presente re di Spagna; costui si può
chiamare quasi principe nuovo, perché d'uno re debole è diventato per
fama e per gloria el primo re de' Cristiani; e se considerrete le
actioni sua, le troverrete tutte grandissime e qualcuna extraordinaria.
{[}3{]} Lui nel principio del suo regno assaltò la Granata, e quella
impresa fu il fondamento dello stato suo. {[}4{]} Prima, egli la fece
ozioso e sanza sospetto di essere impedito; tenne occupati in quella gli
animi di quelli baroni di Castiglia, e quali, pensando a quella guerra,
non pensavano ad innovazione: e lui acquistava in quel mezzo reputazione
et imperio sopra di loro, che non se ne accorgevano; possé nutrire, con
danari della Chiesa e de' populi, exerciti, e fare uno fondamento, con
quella guerra lunga, alla milizia sua, la quale lo ha di poi onorato.
{[}5{]} Oltre a questo, per potere intraprendere maggiori imprese,
servendosi sempre della religione, si volse ad una pietosa crudeltà,
cacciando e spogliando el suo regno de' Marrani: né può essere questo
exemplo più miserabile né più raro. {[}6{]} Assaltò, sotto questo
medesimo mantello, l'Affrica. Fece l'impresa di Italia. Ha ultimamente
assaltato la Francia. {[}7{]} E così sempre ha fatte et ordite cose
grandi, le quali hanno sempre tenuti sospesi et ammirati gli animi de'
subditi, et occupati nello evento di epse. {[}8{]} E sono nate queste
sua actioni in modo l'una da l'altra, che non ha dato mai infra l'una e
l'altra spazio alli uomini di potere quietamente operarli contro.

{[}9{]} Giova ancora assai ad uno principe dare di sé exempli rari circa
a' governi di dentro, -- simili a quegli che si narrano di messer
Bernabò da Milano, -- quando si ha l'occasione di qualcuno che operi
alcuna cosa extraordinaria, o in bene o in male, nella vita civile: e
pigliare uno modo, circa premiarlo o punirlo, di che si abbia a parlare
assai. {[}10{]} E sopratutto uno principe si debba ingegnare dare di sé
in ogni sua actione fama di uomo grande e di ingegno excellente.
{[}11{]} È ancora stimato uno principe, quando egli è vero amico e vero
inimico: cioè quando senza alcuno respecto egli si scuopre in favore di
alcuno contro ad un altro. {[}12{]} El quale partito fia sempre più
utile che stare neutrale: perché, se dua potenti tua vicini vengono alle
mane, o e' sono di qualità che, vincendo uno di quegli, tu abbia a
temere del vincitore, o no. {[}13{]} In qualunque di questi dua casi ti
sarà sempre più utile lo scoprirti e fare buona guerra; perché, nel
primo caso, se tu non ti scuopri sarai sempre preda di chi vince, con
piacere e satisfazione di colui che è stato vinto; e non hai ragione né
cosa alcuna che ti defenda, né chi ti riceva: perché chi vince non vuole
amici sospetti e che non lo aiutino nelle avversità; chi perde, non ti
riceve per non avere tu voluto con le arme in mano correre la fortuna
sua.

{[}14{]} Era passato in Grecia Antioco, messovi dagli Etoli per
cacciarne Romani; mandò Antioco oratori alli Achei, che erano amici de'
Romani, a confortargli a stare di mezzo: e dalla altra parte e Romani
gli persuadevano a pigliare le arme per loro. {[}15{]} Venne questa
materia a deliberarsi nel concilio delli Achei, dove el legato di
Antioco gli persuadeva a stare neutrali; a che il legato romano rispose:
«Quod autem isti dicunt, non interponendi vos bello, nihil magis alienum
rebus vestris est: sine gratia, sine dignitate praemium victoris
eritis.» {[}16{]} E sempre interverrà che colui che non è amico ti
ricercherà della neutralità, e quello che ti è amico ti richiederà che
ti scuopra con le arme. {[}17{]} Et e principi male resoluti, per
fuggire e presenti periculi, seguono el più delle volte quella via
neutrale, et il più delle volte rovinano.

{[}18{]} Ma quando el principe si scuopre gagliardamente in favore di
una parte, se colui con chi tu ti aderisci vince, ancora che sia potente
e che tu rimanga a sua discrezione, egli ha teco obligo, e' vi è
contratto l'amore: e gli uomini non sono mai sì disonesti, che con tanto
exemplo di ingratitudine e' ti opprimessino; dipoi le vittorie non sono
mai sì stiette che el vincitore non abbia ad avere qualche respetto, e
maxime alla iustizia. {[}19{]} Ma se quello con il quale tu ti aderisci
perde, tu sei ricevuto da lui, e mentre che può ti aiuta, e diventi
compagno di una fortuna che può resurgere.

{[}20{]} Nel secondo caso, quando quelli che combattono insieme sono di
qualità che tu non abbi da temere, di quello che vince,, tanto è
maggiore prudenzia lo aderirsi, perché tu vai alla ruina d'uno con lo
aiuto di chi lo doverrebbe salvare, se fussi savio; e vincendo rimane a
tua discrezione, et è impossibile, con lo aiuto tuo, che non vinca.
{[}21{]} E qui è da notare che uno principe debba advertire di non fare
mai compagnia con uno più potente di sé per offendere altri, se non
quando la necessità ti constringe, come di sopra si dice; perché,
vincendo, rimani suo prigione: e li principi debbono fuggire, quanto
possono, lo stare a discrezione di altri. {[}22{]} E Viniziani si
accompagnorono con Francia contro al duca di Milano, e potevano fuggire
di non fare quella compagnia: di che ne resultò la ruina loro. {[}23{]}
Ma quando e' non si può fuggirla, -- come intervenne a' Fiorentini,
quando el papa e Spagna andorono con li exerciti ad assaltare la
Lombardia, -- allora si debba el principe aderire per le ragioni sopradette. {[}24{]} Né creda mai alcuno stato potere pigliare partiti sicuri, anzi pensi di avere a prenderli tutti dubii; perché si trova questo, nell'ordine delle cose, che mai si cerca fuggire uno inconveniente che non si incorra in uno altro; ma la prudenza consiste in sapere conoscere le qualità delli inconvenienti e pigliare el men tristo per buono.

{[}25{]} Debbe ancora uno principe monstrarsi amatore delle virtù, dando
ricapito alli uomini et onorando gli excellenti in una arte. {[}26{]}
Appresso debbea animare e sua ciptadini di potere quietamente exercitare
li exercizii loro, e nella mercanzia e nella agricultura et in ogni
altro exercizio delli uomini; e che quello non tema di ornare le sua
possessione per timore che le gli sia tolta, e quello altro di aprire
uno traffico per paura delle taglie. {[}27{]} Ma debbe preparare premii
a chi vuole fare queste cose et a qualunque pensa in qualunque modo
ampliare o la sua città o il suo stato. {[}28{]} Debba oltre a questo,
ne' tempi convenienti dello anno, tenere occupati e populi con feste e
spettaculi; e perché ogni città è divisa in arte o in tribù, tenere
conto di quelle università, raunarsi con loro qualche volta, dare di sé
exempli di umanità e di munificenzia, tenendo sempre ferma nondimanco la
maestà della dignità sua.

\quebra\section{DE HIS QUOS A SECRETIS PRINCIPES HABENT
{[}De' secretari ch'e' principi hanno apresso di loro{]}}

{[}1{]} Non è di poca importanza a uno principe la electione de'
ministri, e quali sono buoni, o no, secondo la prudenzia del principe.
{[}2{]} E la prima coniettura che si fa del cervello d'uno Signore, è
vedere li uomini che lui ha dintorno: e quando sono suffizienti e
fedeli, sempre si può reputarlo savio, perché ha saputo conoscerli
suffizienti e sa mantenerli fideli; ma quando sieno altrimenti, sempre
si può fare non buono iudizio di lui: perché el primo errore che fa, lo
fa in questa electione.

{[}3{]} Non era alcuno che conoscessi messer Antonio da Venafro, per
ministro di Pandolfo Petrucci, principe di Siena, che non iudicasse
Pandolfo essere valentissimo uomo, avendo quello per suo ministro.
{[}4{]} E perché sono di tre generazione cervelli, -- l'uno intende da
sé, l'altro discerne quello che altri intende, el terzo non intende né
sé né altri: quel primo è excellentissimo, el secondo excellente, el
terzo inutile, -- conveniva pertanto di necessità che, se Pandolfo non
era nel primo grado, che fussi nel secondo.

{[}5{]} Perché ogni volta che uno ha iudizio di conoscere il bene o il
male che uno fa o dice, ancora che da sé non abbia invenzione, conosce
le opere buone e le triste del ministro e quelle exalta e le altre
corregge: et il ministro non può sperare di ingannarlo e mantiensi
buono.

{[}6{]} Ma come uno principe possa conoscere el ministro, ci è questo
modo che non falla mai: quando tu vedi el ministro pensare più a sé che
a te, e che in tutte le actioni vi ricerca dentro l'utile suo, questo
tale così fatto mai fia buono ministro, mai te ne potrai fidare. {[}7{]}
Perché quello che ha lo stato di uno in mano, non debbe pensare mai a
sé, ma sempre al principe, e non gli ricordare mai cosa che non
appartenga a lui; e dall'altro canto el principe per mantenerlo buono,
debba pensare al ministro, onorandolo, facendolo ricco, obligandoselo,
participandogli gli onori e carichi: acciò che veggia che non può stare
sanza lui, e che gli assai onori non li faccino desiderare più onori, le
assai ricchezze non gli faccino desiderare più ricchezze, li assai
carichi gli faccino temere le mutazioni. {[}8{]} Quando adunque li
ministri, e li principi circa e ministri, sono così fatti, possono
confidare l'uno dell'altro: quando altrimenti, sempre el fine fia
dannoso o per l'uno o per l'altro.

\quebra\section{QUOMODO ADULATORES SINT FUGIENDI
{[}In che modo si abbino a fuggire li adulatori{]}}

{[}1{]} Non voglio lasciare indrieto uno capo importante et uno errore
dal quale e principi con difficultà si difendano, se non sono
prudentissimi o se non hanno buona electione. {[}2{]} E questi sono gli
adulatori, delli quali le corte sono piene: perché li uomini si
compiacciono tanto nelle cose loro proprie, et in modo vi si ingannono,
che con difficultà si difendano da questa peste. {[}3{]} Et a volersene
difendere si porta periculo di non diventare contennendo; perché non ci
è altro modo a guardarsi dalle adulazioni, se non che gli uomini
intendino che non ti offendino a dirti el vero; ma quando ciascuno ti
può dire il vero, ti manca la reverenza. {[}4{]} Pertanto uno principe
prudente debba tenere uno terzo modo, eleggendo nel suo stato uomini
savii, e solo a quelli eletti dare libero arbitrio a parlargli la
verità, e di quelle cose sole che lui gli domanda e non d'altro, -- ma
debbe domandargli d'ogni cosa,-- e le opinioni loro udire: dipoi
deliberare da sé a suo modo; {[}5{]} et in questi consigli e con
ciascuno di loro portarsi in modo che ognuno cognosca che, quanto più
liberamente si parlerà più gli fia accepto: fuora di quelli, non volere
udire alcuno, andare dietro alla cosa deliberata et essere obstinato
nelle deliberazioni sua. {[}6{]} Chi fa altrimenti, o precipita per li
adulatori o si muta spesso per la variazione de' pareri: di che ne nasce
la poca existimazione sua.

{[}7{]} Io voglio a questo proposito addurre uno exemplo moderno. Pre'
Luca, uomo di Maximiliano presente imperatore, parlando di Sua Maestà
dixe come egli non si consigliava con persona e non faceva mai di cosa
alcuna a suo modo. {[}8{]} Il che nasceva dal tenere contrario termine
al sopradetto; perché lo Imperatore è uomo secreto, non comunica e sua
disegni, non ne piglia parere: ma come nel metterli in atto si
cominciano a conoscere e scoprire, gli cominciono ad essere contradetti
da coloro che lui ha dintorno, e quello, come facile, se ne stoglie; di
qui nasce che quelle cose che fa uno giorno, destrugge l'altro, e che
non si intenda mai quello si voglia o che disegni fare, e che non si può
sopra le sua deliberazioni fondarsi.

{[}9{]} Uno principe, pertanto debba consigliarsi sempre, ma quando lui
vuole e non quando altri vuole: anzi debba torre animo a ciascuno di
consigliarlo d'alcuna cosa, se non gliene domanda; ma lui debbe bene
essere largo domandatore, e dipoi circa alle cose domandate, paziente
auditore del vero: anzi, intendendo che alcuno per alcuno rispetto non
gliene dica, turbarsene. {[}10{]} E perché molti existimano che alcuno
principe, il quale dà di sé opinione di prudente, sia così tenuto non
per sua natura, ma per li buoni consigli che lui ha dintorno, sanza
dubio s'ingannano. {[}11{]} Perché questa è una regola generale che non
falla mai: che uno principe, il quale non sia savio per sé stesso, non
può essere consigliato bene, se già a sorte non si rimettessi in uno
solo che al tutto lo governassi, che fussi uomo prudentissimo. {[}12{]}
In questo caso potrebbe bene essere, ma durerebbe poco: perché quel
governatore in breve tempo gli torrebbe lo stato. {[}13{]} Ma
consigliandosi con più d'uno, uno principe che non sia savio non arà mai
e consigli uniti; non saprà per sé stesso unirgli; de' consiglieri,
ciascuno penserà alla proprietà sua; lui non gli saperrà nè correggere
né conoscere: e non si possono trovare altrimenti, perché gl'uomini
sempre ti riusciranno tristi, se da una necessità non sono fatti buoni.
{[}14{]} Però si conclude che li buoni consigli, da qualunque venghino, conviene naschino dalla~prudenza del principe, e non la prudenza del principe da' buoni consigli.

\quebra\section{CUR ITALIAE PRINCIPES REGNUM AMISERUNT
{[}Per qual cagione li principi di Italia hanno perso li stati loro{]}}

{[}1{]} Le cose soprascripte, osservate prudentemente, fanno parere
antico uno principe nuovo, e lo rendono subito più sicuro e più fermo
nello stato, che s'e' vi fussi antiquato dentro. {[}2{]} Perché uno
principe nuovo è molto più osservato nelle sue actioni che uno
ereditario: e quando le sono conosciute virtuose, pigliono molto più
gl'uomini e molto più gli obligano che el sangue antico. {[}3{]} Perché
gli uomini sono molto più presi dalle cose presenti che dalle passate;
e, quando nelle presenti truovono il bene, vi si godono e non cercano
altro; anzi, piglieranno ogni difesa per lui, quando non manchi
nell'altre cose a sé medesimo. {[}4{]} E così arà duplicata gloria, di
avere dato principio a uno principato et ornatolo e corroboratolo di
buone legge, di buone arme e di buoni amici e di buoni exempli; come
quello ha duplicata vergogna che, nato principe, per sua poca prudenzia
lo ha perduto

{[}5{]} E se si considerà quelli signori che in Italia hanno perduto lo
stato ne' nostri tempi, come il re di Napoli, duca di Milano et altri,
si troverà in loro, prima, uno comune difetto quanto alle arme, per le
cagioni che di sopra si sono discorse; dipoi si vedrà alcuni di loro o
che arà avuto inimici e populi, o, se arà avuto il popolo amico, non si
sarà saputo assicurare de' grandi. {[}6{]} Perché sanza questi difetti
non si perdono gli stati che abbino tanto nervo che possino tenere uno
exercito alla campagna. {[}7{]} Filippo Macedone, non il padre di
Alessandro, ma quello che fu da Tito Quinto vinto, aveva non molto stato
rispetto alla grandezza de' Romani e di Grecia, che lo assaltò:
nondimanco, per esser uomo militare e che sapeva intrattenere il populo
et assicurarsi de' grandi, sostenne più anni la guerra contro a quelli;
e se alla fine perdé el dominio di qualche città, gli rimase nondimanco
el regno.

{[}8{]} Pertanto questi nostri principi, e queli erano stati molti anni
nel loro principato, per averlo dipoi perso non accusino la fortuna, ma
la ignavia loro: perché, non avendo mai ne' tempi quieti pensato ch'e'
possono mutarsi, -- il che è comune difetto degli uomini, non fare conto
nella bonaccia della tempesta, -- quando poi vennono e tempi adversi,
pensorono a fuggirsi non a defendersi; e sperorono che e populi,
infastiditi per la insolenzia de' vincitori, gli richiamassino. {[}9{]}
Il quale partito, quando mancano gli altri, è buono, ma è ben male avere
lasciati li altri remedii per quello: perché non si vorrebbe mai cadere
per credere di trovare chi ti ricolga. {[}10{]} Il che o non adviene o,
se li adviene, non è con tua sicurtà, per essere quella difesa suta vile
e non dependere da te: e quelle difese solamente sono buone, sono certe,
sono durabili, che dependono da te proprio e dalla virtù tua.

\quebra\section{QUANTUM FORTUNA IN REBUS HUMANIS POSSIT ET QUOMODO ILLI SIT OCCURRENDUM
{[}Quanto possa la fortuna nelle cose umane e in che modo se li abbia a resistere{]}}

{[}1{]} E' non mi è incognito come molti hanno avuto et hanno opinione
che le cose del mondo sieno in modo governate, dalla fortuna e da Dio,
che li uomini con la prudenzia loro non possino correggerle, anzi non vi
abbino remedio alcuno; e per questo potrebbono iudicare che non fussi da
insudare molto nelle cose, ma lasciarsi governare alla sorte. {[}2{]}
Questa opinione è suta più creduta nelli nostri tempi per le variazione
grande delle cose che si sono viste e veggonsi ogni dí, fuora d'ogni
umana coniettura. {[}3{]} A che pensando io qualche volta, mi sono in
qualche parte inclinato nella opinione loro. {[}4{]} Nondimanco, perché
il nostro libero arbitrio non sia spento, iudico potere essere vero che
la fortuna sia arbitra della metà delle actioni nostre, ma che etiam lei
ne lasci governare l'altra metà, o presso, a noi. {[}5{]} Et assimiglio
quella a uno di questi fiumi rovinosi che quando si adirano allagano e
piani, rovinano li albori e li edifizii, lievono da questa parte
terreno, pongono da quell'altra: ciascuno fugge loro dinanzi, ognuno
cede allo impeto loro, sanza potervi in alcuna parte obstare. {[}6{]} E
benché sieno così fatti, non resta però che gli uomini, quando sono
tempi quieti, non vi potessino fare provedimenti e con ripari e con
argini: in modo che, crescendo poi, o eglino andrebbono per uno canale o
l'impeto loro non sarebbe né si dannoso né si licenzioso. {[}7{]}
Similmente interviene della fortuna, la quale dimonstra la sua potenza,
dove non è ordinata virtù a resisterle: e quivi volta li sua impeti,
dove ella sa che non sono fatti gli argini nelli ripari a tenerla.
{[}8{]} E se voi considerrete la Italia, che è la sedia di queste
variazioni e quella che ha dato loro il moto, vedrete essere una
campagna sanza argini e sanza alcuno riparo: ché, s'ella fussi riparata
da conveniente virtù, come é la Magna la Spagna e la Francia, o questa
piena non arebbe fatto le variazioni grande che la ha, o ella non ci
sarebbe venuta. {[}9{]} E questo voglio basti avere detto quanto allo
opporsi alla fortuna, in universali.

{[}10{]} Ma ristringendomi più a' particulari, dico come si vede oggi
questo principe felicitare e domani ruinare, sanza avergli veduto mutare
natura o qualità alcuna; il che credo che nasca, prima, dalle cagioni
che si sono lungamente per lo adrieto discorse: cioè che quel principe,
che si appoggia tutto in sulla fortuna, rovina come quella varia.

{[}11{]}Credo ancora che sia felice quello che riscontra il modo del procedere suo con le qualità de' tempi: e similmente sia infelice quello che con il procedere suo si discordano e tempi. {[}12{]} Perché si vede gli uomini, nelle cose che gli conducano al fine quale ciascuno ha innanzi, cioè glorie e ricchezze, procedervi variamente: l'uno con respetto, l'altro con impeto; l'uno per violenzia, l'altro con arte; l'uno per pazienza, l'altro con suo contrario; e ciascuno con questi diversi modi vi può pervenire. {[}13{]} E vedesi ancora dua respettivi, l'uno pervenire al suo disegno, l'altro no; e similmente dua egualmente felicitare con diversi studii, sendo l'uno rispectivo e l'altro impetuoso: il che non nasce da altro, se non da la qualità de' tempi che si conformano, o no, col procedere loro. {[}14{]} Di qui nasce quello ho detto, che dua, diversamente operando, sortiscano el medesimo effetto: e dua, egualmente operando, l'uno si conduce al suo fine e l'altro no. {[}15{]} Da questo ancora depende la variazione del bene; perché se uno, che si governa con rispetti e pazienzia, e tempi e le cose girono in modo che il governo suo sia buono, e' viene felicitando: ma se e tempi e le cose si mutano, rovina, perché non muta modo di procedere. {[}16{]} Né si truova uomo sì prudente che si sappi accomodare a questo: sì perché non si può deviare da quello a che la natura lo inclina, sì etiam perché, avendo sempre uno prosperato camminando per una via, non si può persuadere che sia bene partirsi da quella. {[}17{]} E però lo uomo respettivo, quando egli è tempo di venire allo impeto, non lo sa fare: donde rovina; ché se si mutassi di natura con li tempi e con le cose, non si muterebbe fortuna. {[}18{]} Papa Iulio II procedé in ogni sua actione impetuosamente; e trovò tanto e tempi e le cose conforme a quello suo modo di procedere, che sempre sortì felice fine. {[}19{]} Considerate la prima impresa che fe' di Bologna, vivendo ancora messer Giovanni Bentivogli. {[}20{]} Viniziani non se ne contentavono; el re di Spagna, quel medesimo; con Francia aveva ragionamenti di tale impresa. E lui nondimanco con la sua ferocia et impeto si mosse personalmente a quella expedizione. {[}21{]} La quale mossa fece stare sospesi e fermi Spagna e Viniziani, quegli per paura e quell'altro per il desiderio aveva di recuperare tutto el regno di Napoli; e dall'altro canto si tirò drieto il re di Francia perché, vedutolo quel re mosso e  desiderando farselo amico per abbassare Viniziani, iudicò non poterli negare gli exerciti sua anza iniuriarlo manifestamente. {[}22{]} Condusse adunque Iulio con la sua mossa impetuosa quello che mai altro pontefice, con tutta la umana prudenza, arebbe condotto. {[}23{]} Perché, se egli aspettava di partirsi da Roma con le conclusioni ferme e tutte le cose ordinate, come qualunque altro pontefice arebbe fatto, mai gli riusciva; perché il re di Francia arebbe avuto mille scuse, e li altri li arebbono messo mille paure. {[}24{]} Io voglio lasciare stare l'altre sua actioni, che tutte sono state simili e tutte gli sono successe bene: e la brevità della vita non li ha lasciato sentire il contrario; perché, se fussino sopravvenuti tempi che fussi bisognato procedere con respetti, ne seguiva la sua ruina: né mai arebbe deviato da quegli modi alli quali la natura lo inclinava.

{[}25{]} Concludo adunque che, variando la fortuna i tempi e stando li
uomini nelli loro modi obstinati, sono felici mentre concordano insieme
e, come discordano, infelici. {[}26{]} Io iudico bene questo, che sia
meglio essere impetuoso che respettivo: perché la fortuna è donna, et è
necessario, volendola tenere sotto, batterla et urtarla. {[}27{]} E si
vede che la si lascia più vincere da questi, che da quegli che
freddamente procedano: e però sempre, come donna, è amica de' giovani,
perché sono meno respettivi, più feroci e con più audacia la comandano.

\quebra\section{EXHORTATIO AD CAPESSENDAM ITALIAM IN LIBERTATEMQUE A BARBARIS VINDICANDAM
{[}Exortazione a pigliar la difesa di Italia e liberarla dalle mani de' barbari{]}}

{[}1{]} Considerato adunque tutte le cose di sopra discorse, e pensando meco medesimo se al presente in Italia correvano tempi da onorare uno nuovo principe, e se ci era materia che dessi occasione a uno prudente e virtuoso di introdurvi forma che facessi onore a lui e bene alla università delli uomini di quella, mi pare concorrino tante cose in benefizio di uno principe nuovo, che io non so qual mai tempo fussi più atto a questo. {[}2{]} E se, come io dixi, era necessario, volendo vedere la virtù di Moisè, che il populo d'Isdrael fussi schiavo in Egitto; et a conoscere la grandezza dello animo di Ciro, ch'e' Persi fussino oppressati da' Medi; e la excellenzia di Teseo, che li Ateniensi fussino dispersi; {[}3{]} così al presente, volendo conoscere la virtù di uno spirito italiano, era necessario che la Italia si riducessi nel termine presenti, e che ella fussi più stiava che li Ebrei, più serva ch'e' Persi, più dispersa che gli Ateniensi: sanza capo, sanza ordine, battuta, spogliata, lacera, corsa, et avessi sopportato d'ogni sorte ruina.

{[}4{]} E benché insino a qui si sia mostro qualche spiraculo in
qualcuno, da potere iudicare che fussi ordinato da Dio per sua
redemptione, tamen si è visto come dipoi, nel più alto corso delle
actioni sua, è stato dalla fortuna reprobato. {[}5{]} In modo che,
rimasa sanza vita, aspetta quale possa essere quello che sani le sue
ferite e ponga fine a' sacchi di Lombardia, alle taglie del Reame e di
Toscana, e la guarisca da quelle sue piaghe già per lungo tempo
infistolite. {[}6{]} Vedesi come la priega Iddio che li mandi qualcuno
che la redima da queste crudeltà et insolenzie barbare. {[}7{]} Vedesi
ancora tutta pronta e disposta a seguire una bandiera, pur che ci sia
uno che la pigli. {[}8{]} Né ci si vede al presente in quale lei possa
più sperare che nella illustre Casa vostra, la quale con la sua fortuna
e virtù, favorita da Dio e dalla Chiesa, della quale è ora principe,
possa farsi capo di questa redemptione. {[}9{]} Il che non fia molto
difficile, se Vi recherete innanzi le actioni e vita dei sopra nominati;
e benché quelli uomini sieno rari e maravigliosi, nondimeno furono
uomini, et ebbe ciascuno di loro minore occasione che la presente:
perché la impresa loro non fu più iusta di questa, né più facile, né fu
Dio più amico loro che a Voi. {[}10{]} Qui è iustizia grande: «iustum
enim est bellum quibus necessarium et pia arma ubi nulla nisi in armis
spes est». {[}11{]} Qui è disposizione grandissima: né può essere, dove
è grande disposizione, grande difficultà, pure che quella pigli delli
ordini di coloro che io ho proposti per mira. {[}12{]} Oltre a di
questo, qui si veggono extraordinarii senza exemplo condotti da Dio: el
mare si è aperto; una nube vi ha scòrto il cammino; la pietra ha versato
acque: qui è piovuto la manna; ogni cosa è concorsa nella Vostra
grandezza. {[}13{]} El rimanente dovete fare Voi: Dio non vuole fare
ogni cosa, per non ci tòrre el libero arbitrio e parte di quella gloria
che tocca a noi.

{[}14{]} E non è maraviglia se alcuno de' prenominati Italiani non ha
possuto fare quello che si può sperare facci la illustre Casa vostra, e
se, in tante revoluzioni di Italia et in tanti maneggi di guerra, e'
pare sempre che in Italia la virtù militare sia spenta; perché questo
nasce che gli ordini antichi di quella non erano buoni, e non ci è suto
alcuno che abbia saputo trovare de' nuovi. {[}15{]} E veruna cosa fa
tanto onore a uno uomo che di nuovo surga, quanto fa le nuove legge e li
nuovi ordini trovati da lui; queste cose, quando sono bene fondate e
abbino in loro grandezza, lo fanno reverendo e mirabile. {[}16{]} Et in
Italia non manca materia da introdurvi ogni forma: qui è virtù grande
nelle membra, quando non la mancassi ne' capi. {[}17{]} Specchiatevi ne'
duelli e ne' congressi de' pochi, quanto li Italiani sieno superiori con
le forze, con la destrezza, con lo ingegno; ma, come si viene alli
exerciti, non compariscono. {[}18{]} E tutto procede dalla debolezza de'
capi; perché quelli che sanno non sono obediti, et a ciascuno pare
sapere, non ci essendo insino a qui suto alcuno che si sia saputo
rilevato tanto, e per virtù e per fortuna, che li altri cedino.

{[}19{]} Di qui nasce che in tanto tempo in tante guerre fatte nelli passati XX anni, quando gli è stato uno exercito tutto italiano, sempre ha fatto mala pruova; di che è testimone prima el Taro, dipoi Alexandria, Capua, Genova, Vailà, Bologna, Mestri.


{[}20{]}  Volendo adunque la illustre Casa vostra seguitare quelli excellenti
uomini che redimirno le provincie loro, è necessario innanzi a tutte
le altre cose, come vero fondamento d'ogni impresa, provedersi d'arme
proprie, perché non si può avere né più fidi, né più veri, né migliori
soldati: e benché ciascuno di epsi sia buono, tutti insieme
diventeranno migliori quando si vedessino comandare dal loro principe,
e da quello onorare et intrattenere. {[}21{]} È necessario pertanto
prepararsi a queste arme, per potere con la virtù italica defendersi
dalli externi. {[}22{]} E benché la fanteria svizzera e spagnola sia
esistimata terribile, nondimanco in ambedua è difetto per il quale uno
ordine terzo potrebbe non solamente opporsi loro ma, confidare di
superargli. {[}23{]} Perché gli Spagnuoli non possono sostenere e
cavagli, e li Svizzeri hanno ad avere paura de' fanti quando gli
riscontrino nel combattere ostinati come loro: donde si è veduto e
vedrassi, per experienza, li Spagnuoli non potere sostenere una
cavalleria franzese e li Svizzeri essere rovinati da una fanteria
spagnuola. {[}24{]} E benché di questo ultimo non se ne sia visto
intera esperienzia, tamen se ne è veduto uno saggio nella giornata di
Ravenna, quando le fanterie spagnuole si affrontorono con le battaglie
tedesche, le quali servano el medesimo ordine che le svizzere: dove li
Spagnuoli, con la agilità del corpo et aiuto delli loro brocchieri,
erano entrati tra le picche loro sotto, e stavano securi ad offenderli
sanza che Tedeschi vi avessino remedio; e se non fussi la cavalleria
che gli urtò, gli arebbano consumati tutti. {[}25{]} Puossi adunque
conosciuto il difetto dell'una e dell'altra di queste fanterie,
ordinarne una di nuovo, la quale resista a' cavalli e non abbia paura
de' fanti: il che farà la generazione delle arme e la variazione delli
ordini; e queste sono di quelle cose che, di nuovo ordinate, dànno
reputazione e grandezza a uno principe nuovo.


{[}26{]}Non si debba adunque lasciare passare questa occasione, acciò
che la Italia vegga dopo tanto tempo apparie uno suo redemptore.
{[}27{]} Né posso exprimere con quale amore egli fussi ricevuto in tutte
quelle provincie che hanno patito per queste illuvioni externe, con che
sete di vendetta, con che ostinata fede, con che pietà, con che lacrime.
{[}28{]} Quali porte se li serrerebbono? Quali populi gli negherebbano
la obedienza? Quale invidia se li opporrebbe? Quale Italiano gli
negherebbe l'ossequio? Ad ognuno puzza questo barbaro dominio. {[}29{]}
Pigli adunque la illustre Casa vostra questo absumpto, con quello animo
e con quella speranza che si pigliano le imprese iuste; acciò che, sotto
la sua insegna, e questa patria ne sia nobilitata, e, sotto li sua
auspizi, si verifichi quel detto del Petrarca, quando dixe:

Virtù contro a furore

prenderà l'armi, e fia el combatter corto,

che l'antico valore

nelli italici cor non è ancor morto.
} 
\ParallelRText{\setcounter{section}{0} \section*{NICOLAUS~MACLAVELLUS \break{}CARTA~DEDICATÓRIA}

{[}1{]} Aqueles que desejam conquistar as graças de um príncipe
costumam, muitas vezes, presenteá"-lo com aquelas coisas, entre as suas,
que têm como as mais caras ou das quais o vejam deleitar"-se. Donde se
vê, muitas vezes, os príncipes serem presenteados com cavalos, armas,
tecidos bordados de ouro, pedras preciosas e semelhantes ornamentos
dignos da grandeza deles. {[}2{]} Desejando, pois, oferecer"-me à Vossa
Magnificência com algum testemunho de minha servidão para convosco, não
encontrei entre os meus bens coisa que eu tenha mais cara ou tanto
estime quanto o conhecimento das ações dos grandes homens, apreendidas
por mim mediante uma longa experiência das coisas modernas e uma
contínua lição das coisas antigas, sobre as quais tenho com grande
diligência longamente refletido e examinado, e reunido agora em um
pequeno volume, que envio à Vossa Magnificência. {[}3{]} Embora julgue
esta obra indigna de estar à Vossa presença, muito confio, todavia, que
pela sua benevolência, ela deva ser aceita, pois considero que não vos
possa ser oferecido maior presente do que a faculdade para poder, em
brevíssimo tempo, entender tudo aquilo que eu, em tantos anos e com
tantos incômodos e perigos, conheci e entendi\footnote{O binômio citado
  revela o modo como Maquiavel entende a aquisição do conhecimento do
  mundo político. O conhecer diz respeito à experiência concreta da vida
  política, mas que não completa todo o, digamos, ``processo do
  conhecimento político''. Faz"-se necessário refletir sobre esse
  material bruto da experiência política, principalmente cotejando com o
  conhecimento acumulado da História. Portanto, a expressão ``conhecer e
  entender'' dá a medida de como se realiza o conhecimento político em
  sua plenitude, ou seja, experiência associada à reflexão iluminada
  pela História.}. {[}4{]} Obra à qual eu não adornei e nem preenchi com
períodos longos ou com palavras empoladas e magníficas, ou com qualquer
outro enfeite ou ornamento extrínseco, com os quais comumente muitos
descrevem e adornam as suas coisas, porque desejei ou que nenhuma coisa
a honre ou que somente a variedade da matéria e a importância do assunto
a torne agradável. {[}5{]} Tampouco quero que seja considerado como
presunção que um homem de baixa e ínfima condição ouse discorrer e
regular os governos dos príncipes, porque, assim como aqueles que
desenham os territórios se põem no plano baixo para analisar a natureza
dos montes e dos lugares altos, e, para considerar aquelas coisas de
baixo, põem"-se sobre o alto dos montes, igualmente, para conhecer bem a
natureza dos povos, necessita ser príncipe e, para conhecer bem a
natureza dos príncipes, convém ser homem do povo.

{[}6{]} Receba, portanto, Vossa Magnificência este pequeno presente
segundo aquele ânimo que eu vos mando. Obra a qual, se diligentemente
considerada e lida, vos fará conhecer um grande desejo que está no meu
interior: que seja alcançcada aquela grandeza que a fortuna e outras
qualidades vossas prometem. {[}7{]} E se Vossa Magnificência, do ápice
de vossa grandeza, alguma vez voltar os olhos para estes lugares baixos,
entenderá como eu suporto indignamente uma grande e contínua maldade da
fortuna.

\setcounter{secnumdepth}{2}

\quebra\section{\emph{QUOT SINT GENERA PRINCIPATUUM ET~QUIBUS~MODIS~ACQUIRANTUR} {[}Quantos~são~os~gêneros~dos~principados~e~de~que~modos~são~adquiridos{]}}


{[}1{]} Todos os governos\footnote{O termo \emph{stato} no
  \emph{Príncipe} possui ao menos duas acepções diferentes: pode
  significar a condição política, o ``estado'' político do governante,
  em suma seu \emph{status}; pode significar, ainda, o governo adotado
  por uma província, como sinonimo de regime político. Tendo em vista
  essa dualidade de acepções, adotaremos o uso de ``estado'' (com o
  \emph{e} em minusculo), apenas quando esse indicar o ente político ou
  o governo. Quando \emph{stati} fizer referência à condição política
  adotaremos ``status'', pois se trata da condição política de um
  indivíduo. Por fim, utilizaremos ``governo'' para traduzir
  \emph{stato} como regime político ou governo. Sobre esse tema e uma
  análise da bibliografia a respeito cf. Ames, José Luiz. ``A formação
  do conceito moderno de estado: a contribuição de Maquiavel'' in
  \emph{Revista Discurso}, São Paulo, 41, 2011 (293-328).}, todos os
domínios\footnote{Na edição Martelli (2006, p. 63-64, nota 3 e 5) consta
  que \emph{stati\ldots{} domini} referem"-se respectivamente às repúblicas e
  aos principados do final do período, haja vista que eles estão em
  paralelo pela construção empregada.} que tiveram e têm
autoridade\footnote{Optamos por traduzir \emph{imperio} por
  ``autoridade'' e ``comando militar'' e não ``poder'', justamente para
  não incorrermos na acepção moderna de ``poder'' que pressupõe a noção
  de soberania (Duso, 2005, cap. 1), algo que não se pode afirmar que
  esteja presente em início do século \versal{XVI}, haja vista a elaboração ainda
  do conceito de soberania entre os pensadores políticos. Por outro
  lado, \emph{imperio} em italiano descende diretamente do latim
  \emph{imperium}, termo esse associado à \emph{autorictas}. Nessa
  acepção, a opção de tradução conserva uma ligação terminológica da
  tradição romana e evita uma aproximação conceitual não presente no
  pensamento maquiaveliano de início do século \versal{XVI}. Na edição francesa
  de Fournel e Zancarini, eles traduzem \emph{dominii} por
  \emph{seigneuries} (senhorias) e \emph{imperio} por
  \emph{commandement} (comando).} sobre os homens, foram e são ou
repúblicas ou principados\footnote{Maquiavel abre a obra com uma
  definição geral sobre o campo do político: ele é uma autoridade, um
  \emph{imperio} sobre os homens. Depois dessa curta definição faz"-se
  uma primeira tipificação dessa autoridade. Como é evidente, tal
  definição é precária, necessitando de uma maior elaboração, algo que
  será feito, no caso dos principados, nessa obra para as repúblicas em
  outra obra. Importa destacar também que o poder político se faz sobre
  homens e não sobre territórios, ou seja, os estados e domínios são
  relação de autoridade sobre pessoas e não territórios definidos, em
  mais uma clara evidência de que não estamos tratando de uma noção
  política que pressupõe soberania típica.}. {[}2{]} Os principados ou
são hereditários, nos quais o sangue do seu senhor foi por longo tempo
príncipe, ou são novos. {[}3{]} Os novos ou são inteiramente novos, como
foi Milão para Francesco Sforza\footnote{Francesco Sforza, 1401-1466,
  conhecido pela sua grande habilidade militar, casado com Bianca Maria,
  filha do Duque de Milão (Filipo Maria Visconti), herdou o governo da
  cidade depois da morte do duque em 1447, que não deixou descendentes,
  e dos conflitos travados contra os partidários da República
  Ambrosiana, assumindo o governo definitivamente em 1448.}, ou são como
membros acrescentados ao Estado hereditário do príncipe que os
conquista, como é o reino de Nápoles para o rei de Espanha\footnote{O
  reino de Nápoles foi conquistado em 1452 pelo rei espanhol Afonso \versal{V} de
  Aragão. Com a morte de Afonso, em 1496, Nápoles passou ao seu filho
  bastardo Ferdinando. Um tratado secreto de 11 de novembro de 1500
  assegurou a Luiz \versal{XII}, rei francês, o título de rei de Nápoles. Em
  julho de 1501 o reino foi atacado pelo norte e pelo sul pelos
  franceses: Ferdinando \versal{I} de Aragão renunciou de imediato a luta e se
  pôs nas mãos de Luiz. Mas, em julho de 1502, o conflito entre
  espanhóis e franceses desembocou numa guerra aberta. No final de 1503
  os franceses foram derrotados, ficando o reino sob o controle
  espanhol.}. {[}4{]} Estes domínios assim conquistados ou estão
habituados a viver sob um príncipe ou estão acostumados a ser livres; e
são conquistados ou com as armas de outros ou com as próprias, ou pela
fortuna ou pela \emph{virtù}\footnote{Após ter dado de início uma
  definição geral sobre o campo do político, Maquiavel passa, no
  restante do capítulo, a fazer uma classificação ou tipificação de cada
  item, à maneira dos tratados medievais e bem ao sabor do estilo da
  ciência aristotélica. Verifica"-se, pois, a seguinte divisão:
 \resizebox{8.7cm}{!}{
\begin{tabular}{|ll|l|l|l|}
\hline
\multicolumn{1}{|c|}{\multirow{2}{*}{}} & \multicolumn{4}{c|}{República} \\ \cline{2-5} 
\multicolumn{1}{|c|}{} & \multirow{3}{*}{Principados} & \multicolumn{3}{l|}{Hereditários {[}2{]}} \\ \cline{3-5} 
\multicolumn{1}{|l|}{\begin{tabular}[c]{@{}l@{}}Governos e\\ Domínios\end{tabular}} &  & \multirow{7}{*}{Novos} & \multirow{2}{*}{Mistos} & \begin{tabular}[c]{@{}l@{}}Acrescentados\\ a um principado {[}3{]}\end{tabular} \\ \cline{5-5} 
\multicolumn{1}{|l|}{} &  &  &  & \begin{tabular}[c]{@{}l@{}}Habituados a viver\\  sobre o comando\\ de um príncipe {[}3-4{]}\end{tabular} \\ \cline{1-2} \cline{4-5} 
 &  &  & \multirow{5}{*}{\begin{tabular}[c]{@{}l@{}}Inteiramente\\ novos\end{tabular}} & \begin{tabular}[c]{@{}l@{}}Habituados a viver\\ livremente {[}5{]}\end{tabular} \\ \cline{5-5} 
 &  &  &  & \begin{tabular}[c]{@{}l@{}}Conquistados com\\ armas próprias {[}6{]}\end{tabular} \\ \cline{5-5} 
 &  &  &  & \begin{tabular}[c]{@{}l@{}}Conquistados com\\ armas alheias {[}7{]}\end{tabular} \\ \cline{5-5} 
 &  &  &  & \begin{tabular}[c]{@{}l@{}}Conquistados pela\\ fortuna {[}7{]}\end{tabular} \\ \cline{5-5} 
 &  &  &  & \begin{tabular}[c]{@{}l@{}}Conquistados pela\\ \emph{virtù} {[}8-9{]}\end{tabular} \\ \hline
\end{tabular}
}
 Outro aspecto que se nota é que essa divisão dos itens obedece quase
  que integralmente à sequência das temáticas dos capítulos, conforme
  indicado nos parênteses. Isso revela que Maquiavel pretende, neste
  primeiro capítulo, expor a sequência argumentativa da obra, como que
  fazendo um sumário do que conterá a primeira parte do livro, fórmula
  essa também característica dos tratados medievais. Enfim, por tais
  elementos constitutivos da obra, pode"-se afirmar com segurança a
  intenção do autor em apresentar um tratado sobre a forma de governo
  principesca, haja vista ainda que o livro se nomeia como \emph{De
  Principatibus}, que, numa tradução literal, deveria ser \emph{Sobre os
  principados}.}.
%\end{table}  %\begin{longtable}[]{@{}lll@{}}
  %\toprule
  %Governos e Domínios & República\tabularnewline
  %\midrule
  %\endhead
  %\begin{minipage}[t]{0.32\columnwidth}\raggedright\strut
  %\strut
  %\end{minipage} &
  %\begin{minipage}[t]{0.32\columnwidth}\raggedright\strut
  %Principados\strut
  %\end{minipage} &
  %\begin{minipage}[t]{0.32\columnwidth}\raggedright\strut
  %Hereditários
%
  %{[}2{]}\strut
  %\end{minipage}\tabularnewline
  %& & Novos\tabularnewline
  %Membros acrescentados & Habituados a viver sobre o\tabularnewline
  %ou conquistados & comando de um príncipe {[}3-4{]}\tabularnewline
  %& Habituados a viverem livremente\tabularnewline
  %& {[}5{]}\tabularnewline
  %& Conquistados com armas alheias\tabularnewline
  %& {[}7{]}\tabularnewline
  %& Conquistados com armas\tabularnewline
  %& próprias {[}6{]}\tabularnewline
  %& Conquistados pela fortuna {[}7{]}\tabularnewline
  %& Conquistados pela \emph{virtù} {[}8-9{]}\tabularnewline
  %\bottomrule
  %\end{longtable}

\quebra\section{\emph{DE PRINCIPATIBUS HEREDITARIIS}\break {[}Dos principados hereditários{]}}

{[}1{]} Deixarei de lado a discussão sobre as repúblicas, porque, em
outro momento, dissertei longamente sobre elas\footnote{Essa frase tem
  sido fonte de grandes discussões entre os especialistas,
  principalmente no tocante a qual obra Maquiavel faz remissão. Os
  comentadores levantam três hipóteses: a) que haveria um livro sobre as
  repúblicas escrito antes do \emph{Príncipe}; b) que esta obra seria os
  \emph{Discursos sobre a primeira década de Tito Lívio}, embora a data
  mais plausível sobre a composição desse texto seja posterior ao
  \emph{Príncipe}; c) que uma primeira parte dos \emph{Discursos} foi
  composta antes do \emph{Príncipe}. Apesar de não haver um consenso, a
  hipótese mais provável é a terceira conforme demonstra Felix Gilbert
  (1977, p. 225-253) e também Gennaro Sasso (1993, p. 349-359; p.
  563-568).}. {[}2{]} Ocupar"-me"-ei somente do principado, e retecerei as
urdiduras\footnote{``Retecer as urdiduras'' (\emph{ritexendo} ou
  \emph{ritessendo gli orditi}) remete ao primeiro momento do trabalho
  do tecelão que concebe e estabelece a trama dos fios no tear antes de
  começar a confeccionar o tecido. Maquiavel estabelece aqui uma
  analogia entre o trabalho do pensador político e o do tecelão, pois,
  assim como o tecelão, o seu trabalho consiste num movimento de
  construção e desconstrução das tramas das estruturas políticas.
  Confira também uma metáfora semelhante em Dante: \emph{``}{[}\ldots{}{]}
  \emph{si mostrò spedita / l'anima santa di metter la trama / in quella
  tela ch'io le porsi ordita''.} (\emph{Divina Comédia, Paraíso}, \versal{XVII},
  vv. 100-102) e nos seus \emph{Discursos sobre a Primeira década de
  Tito Lívio}, ``\emph{gli uomini possono secondare la fortuna e non
  opporsegli, possono tessere gli orditi suoi e non rompergli}''. (livro
  \versal{I}, cap. 2, linha 24). Nunca é demais lembrar que a Florença dos
  séculos \versal{XV} e \versal{XVI} era famosa por saber trabalhar com tecidos,
  particularmente com a lã, donde essa analogia aproximar o texto do
  contexto de seu tempo.} acima descritas, e demonstrarei como estes
principados podem ser governados e conservados\footnote{Esse parágrafo
  inicial tem uma dupla função na economia do texto: por um lado
  responder a demanda apresentada no início do cap. \versal{I}, mostrar porque
  não tratará de repúblicas, visto que estas também são governos
  (\emph{stati}), mas somente dos principados. No primeiro período,
  Maquiavel já responde àquela demanda deixada de início e parte para um
  dos temas centrais que percorrerá a sua obra: a conquista, o governo e
  a conservação dos principados. Contudo, parece que temos uma
  redundância ao final, pois governar se confunde muitas vezes com o
  conservar. De fato, ao longo da obra, esse binômio parece se
  confundir, mas convém, desde já, lembrar que governar é estar na
  condição de príncipe, o que implica também em conquista. A conservação
  restringe"-se às ações de manutenção da configuração política. No
  limite, temos duas ações necessárias para aqueles que se põem na
  condição de liderar o processo político como será o caso dessa
  personagem também política do príncipe que se apresentará ao longo da
  obra.}.

{[}3{]} Digo, portanto, que nos governos hereditários e acostumados à
dinastia do seu príncipe são muito menores as dificuldades para
conservá"-los do que nos novos, porque basta não preterir os
ordenamentos\footnote{O termo \emph{ordine}, aqui empregado no seu
  plural \emph{ordini}, pode ser traduzido por ``ordem'',
  ``ordenamentos'' ou mesmo, no caso do mundo político, por
  ``instituições''. Seu uso remete à disposição das diversas partes
  políticas que compõem um corpo político. Como bem lembram Fournel e
  Zancarini, no \emph{Príncipe,} Maquiavel mobiliza uma série de termos
  derivados de \emph{ordini}, como os verbos \emph{riordinare,
  disordinare, ordinare} e os substantivos \emph{ordinaria},
  \emph{extraordinario} etc., de modo que a tradução de \emph{ordine}
  por ``instituições'' impediria a conservação desse sentido original,
  embora ela seja mais próxima da nossa compreensão. Todavia, ao optar
  em traduzir \emph{ordine} por ``ordenação'', não somente recuperamos
  uma compreensão também presente em nossa língua -- à semelhança do
  emprego de ``ordenamento jurídico'' --, como abrimos a possibilidade
  de conservar, nos demais empregos, o mesmo radical: para
  \emph{riordinare,} reordenar; para \emph{disordinare,} desordenar etc.
  (Cf. \emph{Le Prince,} org. Fournel e Zancarini, 2002, p. 573-578).}
de seus antecessores e posteriormente contemporizar com os
acidentes\footnote{Aqui a expressão tem o sentido de ``governar com as
  circunstâncias''. O termo ``acidente'' é empregado como um imprevisto,
  um acaso, e não numa acepção própria da metafísica, por oposição à
  essência.}, de modo que, se tal príncipe tiver uma indústria
ordinária, sempre conservará o seu estado, a não ser que uma força
extraordinária e excessiva o prive dele\footnote{Convém destacar, aqui,
  o jogo terminológico entre a indústria \emph{ordinária} (uma
  engenhosidade comum, uma criatividade mediana) e a força
  \emph{extraordinária} e excessiva (uma ação acima da média). Ou seja,
  para o governante herdeiro, em seguindo o normal dos acontecimentos,
  sem nada de extraordinário, fora do comum, não há com o que se
  preocupar. Neste caso, a conservação é ordinária e a conquista deve
  ser extraordinária.}. E tendo sido dela privado, reconquista tal
condição na medida em que o conquistador enfrentar alguma
adversidade\footnote{O argumento maquiaveliano neste período destaca a
  pouca dificuldade em governar um regime político herdado: basta ter
  uma indústria ordinária, ou seja, basta não modificar muito, manter os
  ordenamentos como estão. No limite, esse herdeiro não requer muitas
  qualidades políticas, mas uma condição mediana já é o suficiente. A
  força e o prestígio herdados são garantias suficientes para conservar
  sua condição de governante.}.

{[}4{]} Temos na Itália, por exemplo, o duque de Ferrara\footnote{A
  família ducal dos Estensi, em particular Hercules \versal{I} (duque de 1471 a
  1505) e Afonso \versal{I} (duque de 1505 a 1534).}, que não resistiu aos
assaltos dos venezianos\footnote{Os venezianos iniciaram uma guerra
  contra a cidade de Ferrara em maio de 1482 e os seus sucessos
  obrigaram os reinos da Itália a formar uma coalizão para a defesa do
  governo da família Estensi.} em 1484, nem aqueles do papa
Júlio\footnote{Giuliano della Rovere, 1443-1513, foi eleito papa em 1503
  sob o nome de Júlio \versal{II}. Maquiavel se refere aqui à guerra de Julio \versal{II},
  aliando"-se aos venezianos e aos suíços, contra Ferrara em 15 de
  fevereiro de 1510.} em 1510, por outras razões senão por ser antigo
naquele domínio. {[}5{]} Porque o príncipe natural tem menores razões e
menor necessidade de ofender, donde se segue que seja mais amado. E se
vícios extraordinários não o fizer odiado, é razoável que seja
naturalmente benquisto pelos seus. {[}6{]} E na Antiguidade e
continuação do domínio são extintas a memória e os motivos das
inovações: porque uma mudança sempre deixa o fundamento\footnote{O termo
  \emph{adentellato} tem aqui o sentido de introduzir, entrar, colocar
  algo para dentro, que neste caso vem de encontro da noção de fundação
  política.} para a edificação de outra.


\quebra\section{\emph{DE PRINCIPATIBUS MIXTIS}\break {[}Dos~principados~mistos{]}}

{[}1{]} Todavia, é no principado novo que residem as dificuldades. Em
primeiro lugar, se o principado não é totalmente novo, mas é como um
membro acrescido\footnote{Maquiavel concebe o principado e as repúblicas
  como corpos políticos, que assim como os corpos naturais, possuem
  membros ou partes. Neste caso, trata"-se de um acréscimo ao corpo
  político pré existente, donde a noção seguinte de principado misto ou
  misturado ao corpo político já existente. Martelli (2006), na edição
  comentada, apresenta a hipótese, a partir de uma imprecisão na
  concordância verbal do texto original italiano (\emph{quale è le quali
  sono}) que Maquiavel operou aqui uma fusão de capítulos, no caso, dois
  capítulos que seriam separados: um dedicado aos principados novos e um
  dedicado aos principados mistos ou conquistados. De fato, conforme
  enunciado no capítulo \versal{I}, um dos objetivos da obra é tratar dos
  principados novos (um ausente capítulo \emph{De principatibus nouis})
  e em seguida dos principados conquistados ou mistos. Contudo, o que se
  verifica nesse início é uma junção, na medida em que Maquiavel
  equipara os desafios dos principados novos com o dos mistos já nesse
  período introdutório. Com efeito, esse capítulo, juntamente com os
  capítulos \versal{VII} e \versal{XIX}, perfazem um dos maiores do livro e apresentam
  dois momentos da argumentação: quando ele trata dos principados
  conquistados, como no caso grego, esse um exemplo de principado
  inteiramente novo {[}da linha 01 a 30{]} e, no segundo exemplo {[}da
  linha 31 a 50{]}, o caso da França com as conquistas do rei Luiz, esse
  um exemplo de principados mistos. Note"-se adiante que a linha 31
  inicia"-se com ``Mas voltemos à França'', o que evidencia claramente o
  movimento argumentativo. No limite, a hipótese de Martelli nos é útil
  para perceber esses dois momentos argumentativos dos capítulos e como
  eles correspondem àquilo que foi indicado no capítulo \versal{I}. Cf. Martelli
  (2006), \emph{Nota al texto}, ed. Comentada, p. 428-443.} (e o
conjunto destes principados pode ser chamado de misto), as suas
diferenciações nascem, primeiramente, de uma dificuldade natural
presente em todos os principados novos, a saber: eles são como homens
que voluntariamente mudam de senhor, acreditando melhorar, e esta crença
os faz pegar em armas contra este, no que se enganam, porque vêem
posteriormente pela experiência que pioraram. {[}2{]} O que depende de
uma outra necessidade natural e ordinária, a qual faz com que sempre
precise importunar -- com gente armada e outras infinitas injúrias que a
nova conquista traz consigo -- aqueles de quem se tornou novo príncipe.
{[}3{]} De modo que terá como inimigos todos aqueles que tiver
importunado na ocupação\footnote{O verbo \emph{occupare} empregado
  apresenta um sentido mais fraco politicamente que o termo
  \emph{conquista}, o que nos remete ao caráter provisório dessa
  condição política.} daquele principado e não poderá conservar como
seus amigos aqueles que nele o colocaram, por não poder satisfazer"-lhes
naquilo que pressupunham e por não poder usar contra eles os remédios
fortes, uma vez que você tem obrigação para com eles. Porque, ainda que
se tenha um fortíssimo exército seu, sempre se precisa da ajuda dos
provincianos para entrar em uma província\footnote{A nomeação dos
  territórios no Renascimento italiano apresenta uma peculiaridade: uma
  pequena vila de casas, que para nós seria um distrito, bairro ou
  cidade pequena é nomeada como \emph{paese}, que pertenciam, do ponto
  de vista político"-administrativo, a uma cidade. As cidades, neste caso
  dos territórios italianos desse período, eram repúblicas autônomas ou
  lutavam pela sua autonomia diante das potências políticas, sejam
  outras repúblicas conquistadoras, seja o Papado, sejam os demais
  Impérios de então: o Francês, o Otomano, o Espanhol etc. ``Província''
  era ora empregada como uma parte ou fração do território, que poderia
  incluir várias cidades, como no caso da Toscana, Lombardia e Veneto,
  embora isso não significasse união política ou administrativa, ora
  como uma designação geográfica, sem qualquer caráter político.}.
{[}4{]} Por estes motivos Luís \versal{XII}\footnote{Luiz \versal{XII} ocupou Milão em
  setembro 1499 e a perdeu em fevereiro de 1500.}, rei da França,
rapidamente ocupou Milão e rapidamente a perdeu; para perdê"-la pela
primeira vez, bastaram as forças próprias de Ludovico: porque aquele
povo que lhe tinha aberto as portas, encontrando"-se enganados pela
opinião dele e desiludidos daquele futuro que tinham pressuposto, não
podiam suportar os aborrecimentos de um novo príncipe\footnote{Em abril
  de 1500, as forças de Ludovico foram definitivamente expulsas de
  Milão.}.

{[}5{]} É bem verdade que, conquistando"-se pela segunda vez os países
rebelados, eles se perdem com mais dificuldade, porque o senhor,
aproveitando"-se da ocasião da rebelião -- para assegurar"-se -- tem menos
escrúpulos em punir os delinquentes, identificar os suspeitos,
precaver"-se nos seus pontos mais fracos. {[}6{]} De modo que, na
primeira vez, para fazer perder Milão para França, bastou um duque
Ludovico amotinar"-se em seus domínios, depois, para fazê"-lo perder na
segunda vez, precisou ter todo mundo contra\footnote{``Todo mundo'', no
  caso, a Igreja, a República de Veneza e a Espanha que expulsaram os
  franceses da Itália, entre outubro de 1511 e abril de 1512.} e que os
seus exércitos fossem eliminados ou expulsos da Itália: o que decorre
das razões sobreditas. {[}7{]} Não obstante, tanto na primeira como na
segunda vez lhe foi tirado o ducado. As razões gerais da primeira já
foram discutidas, resta agora falar sobre a segunda e ver que remédios
ele\footnote{O rei francês.} tinha e quais poderia ter alguém nas suas
condições para poder melhor conservar a sua conquista, coisa que não fez
a França.

{[}8{]} Digo, portanto, que estes estados -- que, ao serem conquistados,
agregam"-se a um estado mais antigo do que aquele que os conquistou -- ou
são da mesma província e da mesma língua, ou não o são. {[}9{]} Quando o
são, grande é a facilidade em tê"-los, mais ainda quando não estão
habituados a viver livremente: e para possuí"-los com segurança basta
extinguir a dinastia do príncipe que os dominava, porque, nas outras
coisas conservando"-se"-lhes as velhas condições e não havendo diferença
de costumes, os homens vivem sossegadamente, como se viu ocorrer na
Borgonha, na Bretanha, na Gasconha e na Normandia, que durante tanto
tempo foram da França; e ainda que haja alguma diferença na língua, os
costumes são, todavia, similares e eles podem facilmente conviver entre
si. {[}10{]} E quem os conquista, querendo tê"-los, deve ter dois
cuidados: um, que seja extinta a dinastia do seu antigo príncipe, outro,
de não alterar nem as suas leis nem os seus impostos, de tal modo que, em brevíssimo tempo,
juntamente com seu principado antigo, faça tudo um só corpo.

{[}11{]} Contudo, quando se conquista estados em uma província de
língua, costumes e ordenações diferentes, aqui se encontram as
dificuldades e aqui é preciso ter grande fortuna e grande indústria para
mantê"-los. {[}12{]} E um dos maiores e mais eficazes remédios seria que
a pessoa que o conquista vá habitar pessoalmente o lugar; isto tornaria
mais segura e mais durável aquela posse, tal como fez o Turco\footnote{Como
  era chamado o imperador turco Maomé \versal{II}.} na Grécia: ao qual, com todas
as outras medidas\footnote{\emph{Ordini} tem aqui o sentido de medidas,
  meios, modos de proceder e não de ordenamentos.} observadas por ele
para manter aquele estado, não seria possível mantê"-lo se não fosse lá
habitar. {[}13{]} Porque, estando ali, vê"-se nascer as desordens e
rapidamente se lhes pode dar remédio; não estando ali, só as percebe
quando já forem grandes e não houver mais remédio. Além disso, a
província não é espoliada por seus oficiais, os súditos se satisfazem
com a possibilidade de recorrer direta e facilmente ao príncipe, donde
têm mais razões de amá"-lo, quando querem ser bons, e de temê"-lo, quando
querem ser o contrário; e quem do exterior desejasse assaltar aquele
estado, terá mais respeito, pois se o príncipe habitar o lugar, com
grandíssima dificuldade pode perdê"-lo.

{[}14{]} Outro remédio melhor é fundar colônias em um ou dois lugares
para que sejam como pontos de apoio para aquele estado conquistador,
pois é necessário ou fazer isto ou manter na província conquistada muita
gente armada e muitos soldados. {[}15{]} Com as colônias não se gasta
muito e, sem suas despesas ou com poucas, pode"-se funda"-las e mantê"-las,
e somente ofende aqueles de quem tira os campos e as casas para dá"-las
aos novos habitantes, que são uma parte mínima daquele governo; {[}16{]}
e aqueles que ele (o conquistador) importuna, permanecendo dispersos e
pobres, jamais lhe podem prejudicar; e todos os outros permanecem por um
lado sem serem importunados -- e por isto deveriam acalmar"-se -- e, por
outro lado, preocupados em não errar, por temor que aconteça a eles
aquilo que ocorreu com os espoliados. {[}17{]} Concluo que estas
colônias não são dispendiosas, são mais fiéis, importunam menos, e que
os importunados não podem prejudicar, pois são pobres e dispersos, como
já foi dito. {[}18{]} Pelo que há de se notar que os homens devem ser ou
acalentados ou eliminados: porque caso se vinguem das afrontas leves,
não podem se vingar das afrontas graves, pois a afrontas que se faz a um
homem deve ser de tal modo que não tenha de temer a vingança. {[}19{]}
Mas mantendo no território conquistado, em vez das colônias, gente
armada, gasta"-se muito mais, tendo de consumir com a guarda toda a renda
daquela condição, de modo que a conquista se torna perda, e afronta
muito mais, porque causa dano a todo aquela condição, transferindo de um
lado para outro alojamentos e exércitos. Todos se ressentem de tal
incômodo e todos se tornam inimigos do conquistador, e são os inimigos
que lhe podem prejudicar permanecendo vencidos em suas próprias casas.
{[}20{]} De qualquer modo, esta proteção é inútil, assim como aquela das
colônias é útil.

{[}21{]} Quem está em uma província diferente deve ainda, como foi dito,
fazer"-se chefe e defensor dos vizinhos de menor poder, empenhar"-se em
enfraquecer os poderosos daquele lugar e precaver"-se de que por um
acidente não entre aí algum forasteiro tão poderoso quanto ele. E sempre
acontecerá que ser introduzido na província por aqueles que aí estão
descontentes, ou por desmedida ambição ou por medo, como já se viu com
os etólios que introduziram os romanos na Grécia, e, em todas as outras
províncias em que os romanos entraram, foram eles introduzidos pelos
próprios provincianos\footnote{Referência aos conflitos entre os romanos
  e a liga Aquéia na passagem do século \versal{II} para o século \versal{I} a.C.}.
{[}22{]} E é esta a ordem das coisas: assim que um forasteiro poderoso
entra em uma província, todos aqueles que nela são menos poderosos
aderem a ele, movidos pela inveja que possuem contra quem tem o poder
sobre eles, tanto que, no que diz respeito a estes menos poderosos, ele
não se fadigará em nada para se proteger, porque de imediato todos se
unem voluntariamente, perfazendo um todo com a condição que ele aí
conquistou. {[}23{]} Há somente que tomar cuidado para que eles não
ganhem muita força e muita autoridade, pois facilmente pode, com a sua
força e com o favor deles, diminuir os que são poderosos, para manter o
controle total daquela província. E quem não governar bem esta parte,
perderá rapidamente aquilo que conquistou, e, enquanto a mantiver,
sofrerá infinitas dificuldades e aborrecimentos.

{[}24{]} Os romanos, nas províncias que pilharam, observaram bem estas
normas: fundaram colônias, detiveram os menos poderosos, sem deixar
crescer o poder deles, enfraqueceram os poderosos e não deixaram que os
forasteiros poderosos alcançassem reputação nela. {[}25{]} Como exemplo,
basta"-me apenas a província da Grécia: nela os romanos detiveram os
aqueus e os etólios, enfraqueceram o reino dos macedônios e expulsaram
Antíoco da Grécia. Nem os méritos dos aqueus ou dos etólios impediram
que os romanos conquistassem algum governo, nem as tentativas de
persuasão de Felipe os induziram a serem amigos dos macedônios sem
enfraquecê"-los, tampouco o poder de Antíoco pôde fazê"-los consentir que
ele mantivesse naquela província algum status\footnote{Conforme indicado
  no período anterior, o exemplo histórico expressa a divisão dos papéis
  políticos, pois: os aqueus e os etólios são os menos poderosos, Felipe
  é o poderoso da província e Antíoco é o forasteiro poderoso,
  respectivamente Felipe \versal{V} da Macedônia (237-179 a.C.) e Antíoco \versal{III} da
  Síria (242- 187 a.C.), imperador selêucida. Em 192 Antíoco tentou se
  inserir nos conflitos entre os romanos e os etólios, mas foi
  repetidamente derrotado. Esse fato é apresentado por Tito Lívio em sua
  \emph{História de Roma}, livro \versal{XXXVI}.}. {[}26{]} Porque os romanos
fizeram, nestes casos, aquilo que todos os príncipes sábios devem fazer:
eles devem não somente se resguardar dos escândalos presentes, mas
também dos futuros, prevenir"-se destes com toda indústria, porque,
prevendo"-se com antecedência, remedia"-se facilmente. Contudo, esperando
que se aproximem, os medicamentos não vêm a tempo, porque a doença
tornou"-se incurável. {[}27{]} E ocorre aqui aquilo que diz o
médico\footnote{Ou seja, o médico naturalista.} do tísico: que no
princípio de sua doença é fácil curar e difícil reconhecer, mas, com o
passar do tempo, não sendo reconhecida no princípio nem medicada a
tempo, torna"-se fácil reconhecê"-la e difícil curá"-la. {[}28{]} Assim
ocorre nas coisas do estado: porque, reconhecendo com antecedência -- o
que não é um atributo senão de um homem prudente -- as doenças que
nascem ele rapidamente as cura, mas, caso ele não as reconheça, deixando
que cresçam, de modo que todos passam a reconhecê"-las, não tem mais
remédio.

{[}29{]} Os romanos, todavia, vendo com antecedência os inconvenientes,
remediaram"-nos sempre e jamais deixaram que progredissem para evitar uma
guerra, porque sabiam que a guerra não se evita, mas se adia à vantagem
do outro. Porém entraram em guerra na Grécia com Felipe e Antíoco para
não ter que travá"-la com eles na Itália, e podiam naquele momento evitar
uma e outra, o que não quiseram. {[}30{]} Jamais lhes agradou aquilo que
está na boca de todos os sábios de nossos tempos, isto é, gozar o
benefício do tempo, mas sim os benefícios da sua \emph{virtù} e da sua
prudência: porque o tempo traz todas as coisas, e pode conduzir consigo
o bem como mal, e o mal como bem.

{[}31{]} Mas voltemos à França e examinemos se das coisas ditas ela fez
alguma. Falarei de Luiz\footnote{Luiz \versal{XII}, rei francês de 1498 a 1515.},
e não de Carlos\footnote{Carlos \versal{VIII}, rei francês de 1470 a 1498.}, como
daquele de quem, por ter mantido por mais tempo seu domínio na Itália,
os seus progressos foram melhor vistos: e verá como ele fez o contrário
daquelas coisas que devem ser feitas para se manter um estado com outra
conformação. {[}32{]} O rei Luiz foi introduzido na Itália pela ambição
dos venezianos, quedesejavam ganhar metade do estado da Lombardia com a
sua vinda. {[}33{]} Não quero criticar esta decisão tomada pelo rei,
porque, desejando começar a por um pé na Itália e não tendo amigos nesta
província, estando"-lhe, aliás, fechadas todas as portas em função das
atitudes do rei Carlos, foi forçado a aceitar aquelas amizades que
podia, e teria sido para ele uma decisão bem tomada, se nas outras
manobras não cometesse nenhum erro. {[}34{]} Conquistada, portanto, a
Lombardia, o rei recuperou imediatamente aquela reputação que Carlos lhe
tinha tirado: Genova cedeu; os Florentinos tornaram"-se amigos; o marquês
de Mântua\footnote{Francisco Gonzaga, marido de Isabel d'Este.}, o duque
de Ferrara\footnote{Hercules \versal{I} d'Este.}, os Bentivogli\footnote{Giovanni
  Bentivoglio, senhor de Bologna.}, a senhora de Furlí\footnote{Catarina
  Sforza Riario, senhora de Imola e Forli.}, o senhor de
Faenza\footnote{Astorre Manfredi.}, de Pesaro\footnote{Giovanni di
  Costanzo Sforza.}, de Rimini\footnote{Pandolfo Maltesta.}, de
Camerino\footnote{Giulio Cesare da Varano.}, de Piombino\footnote{Iacopo
  degli Appiani.}, os de Lucca, de Pisa e de Siena, todos lhe foram ao
encontro para serem seus amigos. {[}35{]} E então os venezianos puderam
considerar a temeridade da decisão tomada, e, para conquistar metade da
Lombardia, fizeram do rei francês senhor de dois terços da Itália.

{[}36{]} Considere"-se, agora, com quão pouca dificuldade poderia o rei
manter a sua reputação na Itália, caso ele tivesse observado as regras
acima descritas e mantivesse seguros e defendesse todos aqueles que eram
seus amigos -- os quais eram, em grande número, fracos e medrosos --
quer fossem da Igreja, quer fossem os venezianos, sempre necessitavam
estar com ele, e por meio deles poderia facilmente proteger"-se daqueles
que ainda eram poderosos. {[}37{]} Mas assim que chegou a Milão, fez o
contrário, auxiliou o papa Alexandre\footnote{O papa Alexandre \versal{VI},
  nascido Rodrigo Borgia, governou a Igreja de 10 de Agosto de 1492 a 18
  de agosto de 1503. Era pai de César Borgia, conhecido como o Duque
  Valentino.} para que ele ocupasse a Romanha. Não percebeu que se
enfraquecia com esta deliberação, privando"-se dos amigos e daqueles que
tinham recorrido à sua proteção, e tornava grande a Igreja,
acrescentando ao poder espiritual, que lhe dava tanta autoridade, muito
poder temporal. {[}38{]} E depois de cometer um primeiro erro, foi
obrigado a prosseguir, na medida em que, para pôr fim às ambições de
Alexandre e para que este não se tornasse senhor da Toscana, foi forçado
a vir à Itália.

{[}39{]} Não bastou ao rei francês ter feito grande a Igreja e perder os
amigos, pois, por desejar o reino de Nápoles, dividiu"-o com o rei de
Espanha; e, enquanto antes ele era árbitro da Itália, colocou aí um
sócio, a fim de que os ambiciosos daquela província e os descontentes
com ele tivessem a quem recorrer; e enquanto podia deixar naquele reino
um rei pensionário seu, retirou"-o, para colocar um que pudesse
expulsar"-lhe da Itália. {[}40{]} É verdadeiramente coisa muito natural e
ordinária desejar conquistar: e sempre quando os homens o fazem e podem
serão louvados ou não serão censurados; porém, quando eles não podem, e
desejam fazê"-lo de todo modo, aqui está o erro e a censura. {[}41{]}
Portanto, se a França pudesse com as suas forças assaltar Nápoles,
deveria fazê"-lo; se não podia, não deveria dividi"-la; e se fez a divisão
da Lombardia com os venezianos, mereceu desculpas por, com eles, ter
colocado os pés na Itália; merece censura por não ter a desculpa daquela
necessidade\footnote{No caso, os venezianos ajudaram os franceses a
  entrar na península itálica, mas eles (os franceses) não tinham a
  necessidade de aliar"-se aos venezianos e dividir a Lombardia.}.

{[}42{]} Luiz cometeu, pois, estes cinco erros: eliminou os poderosos
menores, na Itália acrescentou força a um poderoso, nela colocou um
forasteiro poderosíssimo, não foi habitá"-la, não pôs nela colônias.
{[}43{]} Erros que, ainda assim, estando ele vivo, não poderiam
prejudicá"-lo, se não tivesse cometido o sexto {[}erro{]}: de tirar o
estado dos venezianos. {[}44{]} Porque, se ele não tivesse aumentado o
poder da Igreja e nem posto a Espanha na Itália, era bem razoável e
necessário enfraquecer os venezianos; mas, ao tomar aquelas primeiras
decisões, não devia nunca consentir na ruína deles, porque, sendo os
venezianos poderosos, teriam sempre mantido os outros distantes da sua
façanha na Lombardia, seja porque os venezianos não lhes consentiriam
sem tornarem"-se eles seus senhores da Lombardia, seja porque os outros
não desejariam retirá"-la da França para dá"-la aos venezianos; e não
teriam coragem de afrontar os dois.

{[}45{]} E se alguém dissesse: o rei Luiz cedeu a Alexandre a Romanha e
à Espanha o reino {[}de Nápoles{]} para evitar uma guerra, respondo, com
os argumentos ditos acima, que não se deve nunca deixar progredir uma
desordem para evitar uma guerra, porque não se a evita, mas se a adia a
sua desvantagem. {[}46{]} E se alguns outros alegassem a promessa que o
rei fez ao Papa, de executar por ele aquele feito para obter a
dissolução do seu matrimônio e a nomeação do cardeal de Ruão, respondo
com aquilo que direi acerca da promessa do príncipe e de como se deve
observá"-la\footnote{Vide cap. 18.}.

{[}47{]} Por conseguinte, o rei Luiz perdeu a Lombardia por não ter
observado alguns daqueles preceitos observados por outros que
conquistaram províncias e desejaram mantê"-las, não há nisso milagre
algum, mas é muito ordinário e razoável. {[}48{]} Sobre este assunto
falei em Nantes com o cardeal de Ruão, quando o Valentino como era
popularmente chamado César Borgia, filho do papa Alexandre\footnote{César
  Borgia (1475-1507), filho do Papa Alexandre \versal{VI} e Vanossa Catanei,
  tornado cardeal aos 17 anos, recebeu de seu pai a administração de
  territórios da Igreja.} -- ocupava a Romanha. Porque, ao dizer"-me o
cardeal de Ruão que os italianos não entendiam de guerra, respondi"-lhe
que os franceses não entendiam de estado, porque, se entendessem, não
deixariam que a Igreja viesse a ter tanta grandeza. {[}49{]} E pela
experiência se viu que a grandeza dela e da Espanha, na Itália, foi
causada pela França, e que a sua ruína foi causada por eles. {[}50{]} Do
que se tira uma regra geral, a qual nunca ou raramente falha: que quem
faz alguém poderoso, causa a sua ruína, porque aquele poder é criado por
ele ou com astúcia ou com força, e uma e outra destas duas é suspeita
para quem se torna poderoso.

\quebra\section{\emph{CUR DARII REGNUM QUOD ALEXANDER OCCUPAVERAT A SUCCESSORIBUS SUIS
POST ALEXANDRI MORTEM NON DEFECIT}\break {[}Por quê razão o reino de Dário, que~tinha~sido~ocupado~Alexandre, não~se~rebelou~contra~os~seus sucessores após a sua morte{]}}

{[}1{]} Consideradas as dificuldades que se tem em manter um estado
recém"-ocupado, poderia alguém supreender"-se com o fato de Alexandre
Magno\footnote{Alexandre da Macedônia ou Alexandre Magno, conquistou a
  Ásia entre 334 e 327 a.C.} tornar"-se senhor da Ásia em poucos anos, e,
mal tendo"-a ocupado, ter morrido: donde pareceria razoável que todos
aqueles estados se rebelassem. Todavia, os seus sucessores
mantiveram"-nos e não encontraram, para mantê"-los, outra dificuldade
senão aquela que surgiu entre eles mesmos, pelas suas próprias ambições.
{[}2{]} Respondo que os principados, dos quais se tem memória,
encontram"-se governados por dois modos diversos: ou por um príncipe e
todos os outros lhe servem, os quais, como ministros, por sua graça e
concessão, ajudam a governar aquele reino; ou por um príncipe e por
barões, os quais possuem esta posição não pela graça do senhor, mas pela
antiguidade da dinastia. {[}3{]} Estes barões têm estados e súditos
próprios, os quais os reconhecem como senhores e são naturalmente
afeiçoados a eles. {[}4{]} Aqueles estados governados por um príncipe e
por servos têm o seu príncipe com mais autoridade, porque em toda a sua
província não há ninguém que se reconheça por superior senão ele; e se
obedecem a algum outro, o fazem como a um ministro e a um oficial, e a
ele não nutrem nenhum amor em particular.

{[}5{]} Exemplos destas duas diversidades de governos são, em nossa
época, o Turco\footnote{No caso, o governante do Império Otomano.} e o
rei de França. {[}6{]} Toda a monarquia do Turco é governada por um só
senhor: os outros são seus servos. Ele divide o seu reino em
\emph{sandjacs}\footnote{\emph{Sangiacchie} é a italianização do termo
  político turco \emph{sangiak,} que constituíam as subdivisões do
  Império Otomano. Cada uma dessas partes era governada por um
  administrador, também conhecido como Paxá.}, aos quais envia diversos
administradores, mudando"-os e transferindo"-os como lhe convém. {[}7{]}
Mas o rei de França se encontra no meio de uma antiga multidão de
senhores, os quais são reconhecidos em seus estados por seus próprios
súditos e amados por eles: eles têm a sua preeminência, e o rei não pode
tirá"-los sem perigo para si. {[}8{]} Quem considerar, pois, um e outro
destes estados, encontrará dificuldade para conquistar o estado do
governo turco; todavia, uma vez tendo"-o vencido, terá grande facilidade
para mantê"-lo. {[}9{]} Ao contrário, encontrará em alguns aspectos mais
facilidade em ocupar o reino de França, porém, grande dificuldade para
mantê"-lo.

{[}10{]} As razões destas dificuldades em poder ocupar o reino do Turco
estão em não poder ser convocado pelos príncipes\footnote{O plural de
  \emph{príncipe} empregada por Maquiavel, os \emph{principi,} demonstra
  um dos fundamentos desse conceito, a saber: o príncipe, como a própria
  etimologia da palavra revela, são os primeiros, aqueles que estão na
  vanguarda. Neste caso do regime turco, trata"-se dos administradores
  das \emph{sandjacs}, que são lideres ou as principais figuras
  políticas naquele território. Contudo, sua condição de funcionários e
  a sua pouca ou inexistente liderança política os faz pouco confiáveis
  na conquista do governo e não são eles capazes de liderar o povo em
  uma rebelião política.} daquele reino, nem esperar, mediante a
rebelião daqueles que estão ao redor do Turco, facilidades em sua
empresa, o que tem origem nas razões sobreditas: porque, sendo todos
escravos seus e obrigados a ele, com mais dificuldade podem ser
corrompidos e, quando se corrompem, pode"-se esperar pouca ajuda deles,
pois, pelas razões assinaladas, eles não podem trazer consigo o povo.
{[}11{]} Donde ser necessário para quem assalta o Império Turco pensar
que haverá de encontrá"-lo unido, e convém a ele confiar mais na própria
força do que na desordem dos outros. {[}12{]} Mas, uma vez que o tenha
vencido e derrotado em campanha, de tal modo que não possa recompor os
exércitos, não tem do que recear senão da dinastia do príncipe e, uma
vez que ela esteja extinta, não resta ninguém a temer, pois os outros
não têm crédito junto ao povo; e assim como o vencedor não pode, antes
da vitória, confiar nele, assim também não deve, depois dela, temê"-lo.

{[}13{]} O contrário ocorre nos reinos governados, como aquele de
França, porque você pode entrar com facilidade, ganhando para si algum
barão do reino, e porque sempre se encontra alguém descontente e aqueles
desejosos de mudanças\footnote{Um relato semelhante encontramos no texto
  maquiaveliano \emph{Retratos das coisas da França} (linhas de 4 a 6),
  escrito entre 1510 e 1511, no qual Maquiavel faz uma análise da
  dinâmica política do regime francês. Esse exemplo, como outros que se
  seguiram ao longo de \emph{O Príncipe}, revela como Maquiavel emprega
  todos os conhecimentos adquiridos nas duas missões diplomáticas a
  serviço do governo de Florença, tendo como uma de suas incumbências, a
  análise e a descrição atenta da dinâmica política dos governos
  visitados. Neste caso, a análise \emph{in loco} do regime francês
  possibilitou"-lhe uma análise aguda da dinâmica política de um regime
  com divisões de poder muito acentuadas.}. {[}14{]} Esses, pelas razões
ditas, podem abrir"-lhe o caminho naquele estado e lhe facilitar a
vitória, a qual, depois, se quiser conservá"-la, traz consigo infinitas
dificuldades tanto com aqueles que o ajudaram, como com aqueles que
oprimiu. {[}15{]} Não é suficiente para você eliminar a dinastia do
príncipe, porque permanecerão aqueles senhores que se farão líderes de
novas sedições: e, não podendo contentar"-lhes nem eliminá"-los, perderá
aquele estado tão logo haja ocasião.

{[}16{]} Ora, se levarem em conta a natureza do governo de
Dario\footnote{Dario \versal{III}, rei da Pérsia entre 337 a 330 a.C., derrotado
  e morto por Alexandre Magno.}, vocês o considerarão semelhante ao
reino do Turco. Para Alexandre, porém, foi necessário primeiro
combatê"-lo e derrotá"-lo completamente em campanha. {[}17{]} Depois de
tal vitória, estando Dario morto, permaneceu seguro aquele estado para
Alexandre pelas razões acima expostas, e os seus sucessores, se fossem
unidos, poderiam usufruí"-lo com ócio, pois naquele reino não surgiram
outros tumultos senão aqueles que eles próprios suscitaram\footnote{Após
  a morte de Alexandre, seu império foi dividido entre os seus generais:
  Antígono, Antipatro, Lisimaco, Perdicca, Seleuco e Ptolomeo.}.
{[}18{]} Mas é impossível possuir com tamanha tranquilidade os estados
ordenados à semelhança daquele de França. {[}19{]} Disso se originaram
as frequentes rebeliões na Espanha, na França e na Grécia dominadas
pelos romanos, por serem muitos os principados nesses estados\footnote{``\emph{Muitos
  principados nesses estados''}, com isso Maquiavel quer destacar os
  inúmeros governos que havia nesses territórios antes da dominação
  romana e, mesmo após essa, havia nesses povos a lembrança desses
  governos próprios, o que dificultou o domínio imposto por Roma nessas
  localidades.} e, enquanto durou a memória deles, os romanos sempre
estiveram incertos daqueles domínios. {[}20{]} Mas, extinta a memória
desses senhores, com a força e a duração do Império, os romanos ficaram
seguros em seu domínio. Combatendo depois entre eles\footnote{No caso,
  as guerras civis entre os próprios romanos (Mario e Sila, César e
  Pompeu, Otaviano e Antônio), que enfraqueceram o controle de Roma
  sobre esses territórios.}, cada um daqueles principados pode também
retomar parte daquelas províncias, conforme a autoridade que haviam
obtido ali, e, aquelas, por terem extinta a dinastia dos seus antigos
senhores, não reconheciam senão os romanos como senhores. {[}21{]}
Considerando, portanto, todas estas coisas, ninguém se surpreenderá com
a facilidade com que Alexandre obteve o estado da Ásia e da dificuldade
que outros tiveram para conservar o conquistado, como Pirro\footnote{Rei
  do Épiro (atual Albânia) de 307 a 303 e de 297 a 272 a.C. Conquistou a
  Macedônia, mas, em função de uma insurreição, perdeu"-a em poucos meses.} e muitos outros,
o que não tem origem na muita ou pouca \emph{virtù} do vencedor, mas na
diversidade das matérias\footnote{O termo italiano \emph{subietto} é
  tradução do latim \emph{subiectum}, que por sua vez, é a tradução do
  termo grego \emph{hypokeímenon}, que é empregado por Aristóteles como
  o substrato do ser. Tal substrato pode ser entendido de vários modos
  (por exemplo, \emph{Metafísica}, \versal{Z}, \versal{III}), mas a interpretação mais
  adotada pelos leitores dos textos aristotélicos foi a de associar o
  \emph{hypokeímenon} à matéria, tornado"-se o substrato sobre o qual se
  agregam os acidentes. Assim, quando Maquiavel emprega o termo
  \emph{subietto}, como neste caso, está pensando nesse substrato
  material, o povo, que é o fundamento material da cidade, conservando,
  portanto, sua significação grega original. Cf. também Aristóteles,
  \emph{Política}, \versal{III}, cap. 1, 1274b30-1276a5.}.

\quebra\section{\emph{QUOMODO ADMINISTRANDAE SUNT CIVITATES VEL PRINCIPATUS, QUI,
ANTEQUAM OCCUPARENTUR SUIS LEGIBUS VIVEBANT}\break {[}De que modo se deve administrar as cidades ou principados que, antes de serem ocupados, viviam segundo as suas próprias leis{]}}

{[}1{]} Quando aqueles estados que se conquistam, como foi dito, estão
habituados a viver segundo as suas próprias leis e em liberdade, para
querer mantê"-los são três os modos: {[}2{]} o primeiro, arruiná"-los; o
outro, ir habitá"-los pessoalmente; o terceiro, deixá"-los viver segundo
as suas próprias leis, cobrando um tributo e criando dentro deles um
estado de poucos\footnote{Estado de poucos é uma outra forma de designar
  o regime aristocrático.}, que os conserve seus amigos. {[}3{]} Porque,
sendo esse governo {[}aristocrático{]} criado por aquele príncipe, sabe
que não pode ficar sem a sua amizade e a sua força e há de fazer tudo
para conservá"-lo. E, desejando preservá"-la, mais facilmente se mantém
uma cidade acostumada a viver livremente por meio de seus
cidadãos\footnote{``\emph{Uma cidade acostumada a viver livremente por
  meio de seus cidadãos}'' é uma expressão típica para caracterizar um
  regime republicano, ou seja, uma cidade que não é dominada ou
  comandada por um outro poder político, e na qual a condução do governo
  está nas mãos dos cidadãos. Convém destacar o uso do vocábulo
  \emph{ciptadino} (cidadãos), o que revela o estatuto político dos
  habitantes, por oposição aos \emph{súditos} de um reino. Como é
  evidente, a liberdade política é um atributo fundamental da cidadania
  para Maquiavel.}, do que de qualquer outro modo.

{[}4{]} Como exemplo temos os espartanos e os romanos. Os espartanos, ao
ocuparem Atenas e Tebas, criaram nelas um estado de poucos\footnote{Como
  já mencionado, trata"-se de uma aristocracia (regime virtuoso) ou
  oligarquia (regime desviado), conforme a definição clássica.}, contudo
voltaram a perdê"-las. {[}5{]} Os romanos, para manter Cápua, Cartago e
Numância, destruíram"-nas e não as perderam; quiseram manter a Grécia,
assim como a mantiveram os espartanos, tornando"-a livre e deixaram"-na
com suas leis, e não obtiveram sucesso, de modo que, para mantê"-la,
foram obrigados a destruir muitas cidades daquela província. {[}6{]}
Porque, em verdade, não havia um modo seguro para apossar"-se dela, senão
arruinando"-a; e quem se torna senhor de uma cidade acostumada a viver
livremente e não a destrói, pode esperar ser destruído por ela, pois a
cidade sempre encontra guarida na rebelião, em nome da
liberdade\footnote{Aqui ``nome'' com sentido de valor ou ideal, que
  jamais abandona os povos que experimentaram a liberdade política.} e
nas suas antigas ordenações, as quais jamais são esquecidas, nem pela
duração do tempo, nem pelos benefícios realizados. {[}7{]} E qualquer
coisa que se faça ou se proveja, se não se dividirem ou se dispersarem
os habitantes, e se não se esquecerem daquele nome nem daquelas
ordenações, logo, em qualquer imprevisto, acontecerá como fez Pisa cem
anos depois de ter sido posta sob o domínio dos florentinos\footnote{A
  cidade de Pisa foi ocupada pelos florentinos em 1405 e se rebelou em
  1494, com a chegada na Itália de Carlos \versal{VIII}, rei francês. Foi
  novamente ocupada em 1509 pelos florentinos.}. {[}8{]} Mas, quando as
cidades ou as províncias estão habituadas a viver sob um príncipe e
extingue"-se essa dinastia -- estando de um lado habituadas a obedecer e
de outro não tendo o velho príncipe -- não entram em acordo em si para
ter um outro príncipe, nem sabem viver livremente, de modo que demoram
mais para pegar em armas e com mais facilidade um apríncipe pode
conquistá"-las e estar seguro com eles.

{[}9{]} Mas, nas repúblicas, há mais vida, mais ódio, mais desejo de
vingança; nem os deixa\footnote{No caso, os cidadãos, que não abandonam
  a memória da liberdade.}, nem pode deixar"-se aquietar a memória da
antiga liberdade, de modo que o caminho mais seguro é extingui"-las ou
habitá"-las\footnote{O período final sintetiza a idéia que percorre e
  comanda o capítulo: a liberdade, característica própria das
  repúblicas, é o valor maior de uma cidade. Logo, no ato da conquista,
  o príncipe tem que saber lidar com esse dado, pois a memória da
  liberdade política dos cidadãos sempre será um empecilho para o seu
  governo. Donde a necessidade de se tomar medidas drásticas, como
  destruir todas as antigas instituições, para que o conquistador
  consiga impor o seu domínio sobre aquele povo. De outro lado, para uma
  república, acostumada com a liberdade, isso implica também lutas e
  conflitos, pois esse é um dado inerente da vida política: o conflito
  político permanente. Tal noção de conflito político, que também é
  expressão da liberdade política, se apresentará melhor nos capítulos
  \versal{VIII} e \versal{IX} de \emph{O Príncipe}, mas com mais força e vigor nos
  \emph{Discursos sobre a primeira década de Tito Lívio},
  particularmente nos capítulos \versal{III} e \versal{IV} do livro \versal{I}.}.

\quebra\section{\emph{DE PRINCIPATIBUS NOVIS QUI ARMIS PROPRIIS~ET~VIRTUTE~ACQUIRUNTUR}\break
{[}Dos principados novos que se conquistam com armas próprias e virtude{]}}

{[}1{]} Ninguém se surpreenda se, na exposição que farei dos principados
completamente novos, dos príncipes e dos estados, eu apresentar
grandíssimos exemplos. {[}2{]} Porque, caminhando os homens sempre pelos
caminhos percorridos por outros e procedendo por imitação nas suas
ações, não podendo em tudo seguir nos caminhos alheios, nem adquirir a
\emph{virtù} daqueles que você imita, deve um homem prudente seguir
sempre pelas estradas percorridas por grandes homens, e imitar aqueles
que foram excelentíssimos, a fim de que, se a sua \emph{virtù} não os
alcançar, ao menos receba deles algum aroma; {[}3{]} e fazer como os
arqueiros prudentes, os quais, parecendo muito distante o lugar que
desejam alvejar e conhecendo bem até que ponto vai a \emph{virtù} do seu
arco, põem a mira muito mais alta que o lugar mirado, não para atingir
tão alto com a sua flecha, mas para poder, com a ajuda de sua mira alta,
alcançar o alvo desejado.

{[}4{]} Digo, pois, que nos principados inteiramente novos, em que haja
um príncipe novo, encontra"-se menor ou maior dificuldade para
conservá"-los, segundo seja mais ou menos virtuoso\footnote{Eis aqui um
  bom exemplo da distinção operada por Maquiavel entre a sua noção de
  \emph{virtù} e a noção de virtude. Neste caso Maquiavel utiliza
  ``virtuoso'' quando poderia muito bem utilizar \emph{virtù}, contudo,
  em não o fazendo, pretende ressaltar justamente a diferença entre ter
  virtudes e ter \emph{virtú}. Aquele que conquista até pode ter alguma
  ``virtude'', ser virtuoso, mas isso não é o mesmo que ter
  \emph{virtù}, necessária à conquista política.} aquele que os
conquista. {[}5{]} E porque este evento, de passar de cidadão
comum\footnote{O termo \emph{privato,} utilizado aqui por Maquiavel, é
  de difícil tradução para o nosso contexto discursivo, pois
  literalmente \emph{privato} deve ser traduzido por \emph{privado} em
  português. No capítulo \versal{IX}, ele usará a expressão \emph{privato
  ciptadino} (1) e \emph{ciptadino privato} (20), que poderia ser
  traduzida literalmente por \emph{cidadão particular}. Entretanto, o
  autor está se referindo, aqui, bem como nas demais ocorrências a
  seguir {[}capítulos \versal{VI} (27), \versal{VII} (1, 2, 6), \versal{VIII} (1, 4), \versal{IX} (1, 20),
  \versal{XI} (15), \versal{XIV} (3){]}, ao cidadão comum, não pertencente à família do
  governante, que se torna príncipe de uma cidade. Esse personagem
  político, mais do que os outros, é o caso típico do estatuto político
  de \emph{príncipe} defendido ao longo de toda obra: alguém que se
  torna o líder, o primeiro cidadão que conduz os demais politicamente.
  Tendo em vista essa interpretação, defendida na \emph{Introdução}
  desta edição, entendemos que a melhor tradução para esse termo
  \emph{privato}, bem como para \emph{ciptadino privato}, seja mais
  adequado o termo cidadão comum, que expressa esse personagem político
  de proa que assume o comando da cidade calcado na sua \emph{virtù}.} a
príncipe, pressupõe ou \emph{virtù} ou fortuna, parece que uma ou outra
destas duas coisas mitiga, em parte, muitas dificuldades. Todavia,
aquele que menos se apoiou na fortuna, manteve"-se mais. {[}6{]} Gera
ainda mais facilidade o príncipe ser obrigado, por não ter outro estado,
a vir habitá"-lo pessoalmente.

{[}7{]} Mas, tratando daqueles que, pela própria \emph{virtù} e não pela
fortuna, tornaram"-se príncipes, digo que os mais excelentes são Moisés,
Ciro, Rômulo, Teseu\footnote{Moisés: profeta bíblico, fundador da Israel
  do Antigo Testamento, provavelmente no séc. \versal{XIII} a.C.; Ciro: fundador
  do Império Persa no séc. \versal{VI} a.C.; Teseo: unificador da península
  Ática, provavelmente no séc. \versal{XII} a.C.; Rômulo: fundador de Roma,
  séc.\versal{VIII} a.C.} e outros semelhantes. {[}8{]} E ainda que não se deva
discutir sobre Moisés, tendo sido ele um mero executor das coisas que
lhe eram ordenadas por Deus, todavia deve ser admirado tão somente por
aquela graça que o tornava digno de falar com Deus. {[}9{]} Mas,
considerando Ciro e os outros que conquistaram ou fundaram reinos,
achá"-los"-eis todos admiráveis; e se considerarmos as suas ações e os
seus ordenamentos particulares, não pareceram discrepantes em relação
aos de Moisés, que teve tão grande preceptor\footnote{No caso, Maquiavel
  está se referindo ao próprio Deus, que, conforme a narração bíblica,
  ordenou a Moisés conduzir os hebreus à ``terra prometida'' e fundar um
  novo reino.}. {[}10{]} E examinando as suas ações e as suas vidas,
vê"-se que não tiveram nada da fortuna senão a ocasião, a qual deu"-lhes a
matéria para que uma forma semelhante possa ser introduzida dentro
dela\footnote{Expressão típica da metafísica de origem aristotélica, mas
  de uso corrente no Renascimento italiano, no qual se concebe que a
  forma é inserida na matéria correspondente para formar uma substância,
  um algo. Aqui, a ocasião, o momento, correspondem à matéria e a
  \emph{virtù} à forma, donde o príncipe novo tomar a ocasião adequada
  para inserir, com sua \emph{virtù,} a ordenação política na cidade, o
  que resulta no regime político ou governo.}; e, sem essa ocasião, a
\emph{virtù} do seu ânimo teria sido extinta, e sem essa \emph{virtù} a
ocasião teria vindo em vão.

{[}11{]} Era, portanto, necessário a Moisés encontrar o povo de Israel
no Egito, escravizado e oprimido pelos egípcios, a fim de que ele (o
povo de Israel), para sair da servidão, se dispusesse a segui"-lo.
{[}12{]} Convinha que Rômulo não pudesse permanecer em Alba e fosse
abandonado ao nascer, para querer tornar"-se rei de Roma e fundador
daquela pátria. {[}13{]} Era necessário que Ciro encontrasse os persas
descontentes com o Império dos medos, e os medos amolecidos e efeminados
pela longa paz. {[}14{]} Não poderia Teseu demonstrar a sua
\emph{virtù}, se não encontrasse os atenienses dispersos. {[}15{]} Estas
ocasiões, portanto, fizeram estes homens felizes e a sua excelente
\emph{virtù} fez com que aquela ocasião se tornasse conhecida, donde a
sua pátria foi dignificada e tornou"-se felicíssima.

{[}16{]} Aqueles os quais, por caminho virtuoso, à semelhança destes,
tornaram"-se príncipes, conquistam o principado com dificuldade, mas com
facilidade o mantêm; e as dificuldades que eles têm ao conquistar o
principado, nascem, em parte, dos novos ordenamentos e costumes que são
forçados a introduzir para fundar o seu estado e a sua segurança.
{[}17{]} E deve"-se considerar que não há coisa mais difícil de tratar,
nem mais duvidosa em obter, nem mais perigosa em manejar, do que
fazer"-se chefe para introduzir novos ordenamentos. {[}18{]} Porque o
introdutor tem por inimigos todos aqueles que se beneficiaram dos velhos
ordenamentos, e tem por defensores tíbios todos aqueles que dos novos
ordenamentos se beneficiarão. Tal tibieza nasce em parte por medo dos
adversários, que têm as leis do seu lado, em parte da incredulidade dos
homens, os quais não acreditam na verdade das coisas novas, senão quando
vêem nascida de uma firme experiência. {[}19{]} Donde nasce que algumas
vezes aqueles que são inimigos tem ocasião para assaltar, o fazem
apaixonadamente, e aqueles outros defendem tibiamente, de modo que
juntos com esses se corre perigo.

{[}20{]} É necessário, portanto, querendo discorrer bem sobre esta
parte, examinar se estas inovações se sustentam por si mesmas ou se
dependem de outros, isto é, se para conduzir a sua obra, precisa rezar
ou pode forçar. {[}21{]} No primeiro caso, sempre entendem mal e não
leva a coisa alguma, mas, quando dependem de si próprios e podem forçar,
então é que raras vezes correm perigo. Daqui nasce que todos os profetas
armados venceram e os desarmados se arruinaram. {[}22{]} Porque, além
das outras coisas ditas, a natureza dos povos é variada e é fácil
persuadi"-los em uma coisa, mas é difícil sustentá"-los nesta persuasão.
Porém, convém ser ordenado de modo que, quando não crêem mais, pode"-se
fazer crerem pela força. {[}23{]} Moisés, Ciro, Teseu e Rômulo não
teriam podido fazer observar sua constituição longamente caso estivessem
desarmados, como no nosso tempo sucedeu com o frei Jerônimo
Savonarola\footnote{Jerônimo Savonarola (1452-1498) foi um frei
  dominicano, que ganhou notoriedade em Florença, onde foi enviado para
  pregar em 1475. Ao longo do ano de 1494, seus sermões contribuem para
  a queda do governo da família Médici e ele assume a liderança do novo
  governo em dezembro deste ano, constituindo um governo republicano de
  caráter popular. Seu governo entra no crise em início de 1498, sendo
  ele deposto, julgado pela Inquisição católica e condenado a morrer na
  fogueira neste mesmo ano. Apesar de liderar um governo republicano, o
  regime de Savonarola e seus partidários era adversário do grupo
  político ao qual Maquiavel pertencia, grupo político esse que assume o
  governo em 1498, sob o comando de Pier Soderini e que perdura até
  1512.}, o qual arruinou os seus novos ordenamentos, quando a multidão
começou a não acreditar nele, e ele não tinha como manter firmes aqueles
que haviam acreditado nele, nem fazer crer os descrentes. {[}24{]}
Porém, estes tem grande dificuldade no conduzir, e todos os seus perigos
estão no seu caminho, e convém que os superem com a \emph{virtù}.
{[}25{]} Mas, uma vez superadas essas adversidades, começam a ser
venerados, tendo perdido aquela sua qualidade que lhe tinham invejado,
permanecendo fortes, seguros, honrados e felizes.

{[}26{]} A estes altos exemplos, desejo acrescentar um exemplo menor,
mas terá alguma similitude com aqueles, e desejo que me seja suficiente
para todos os outros similares: e este é Hierão de Siracusa\footnote{Hierão \versal{II}, estrategista de Siracusa a partir de 275 a.C., depois rei de Siracusa de 265 a 215 a.C., quando morre. A fonte provável de Maquiavel é Políbio (\emph{Histórias}, livro \versal{I}, cap. 8-9 e 16).}. {[}27{]} Este, de cidadão, tornou"-se príncipe de
Siracusa; nem lhe conhecem outra ocasião senão a fortuna, porque,
estando os siracusanos oprimidos, elegeram"-no para seu capitão, donde
mereceu ser feito príncipe deles. {[}28{]} E foi de tanta \emph{virtù},
ainda que privado pela fortuna, que quem nos escreve sobre ele diz:
\emph{quod nihil illi deerat ad reganandum praeter regnum}\footnote{``Que
  nada lhe faltava para reinar exceto um reino''. Citação de memória do
  texto de Justiniano, \versal{XXIII}, 4. O texto exato é: \emph{ut nihil ei
  regium deesse, praeter regnum videretur}.}. {[}29{]} Este extinguiu a
velha milícia, ordenou uma nova; deixou as antigas amizades, fazendo
novas; e como teve amizades e soldados, pode sobre tais fundamentos
edificar todo o edifício, tanto que lhe deu muito cansaço a conquista e
pouco a conservação.

\quebra\section{\emph{DE PRINCIPATIBUS NOVIS QUI ALIENIS ARMIS~ET~FORTUNA~ACQUIRUNTUR}\break
{[}Dos principados novos que são conquistados com armas e fortuna alheias{]}}

{[}1{]} Aqueles que somente pela fortuna de cidadão se tornam príncipes,
com pouco esforço conseguem sê"-lo e com muito se mantêm. E não têm
nenhuma dificuldade neste caminho, porque voam\footnote{Muito
  ilustrativo e importante aqui o emprego do termo \emph{volare} (voar),
  pois é esta a trajetória daqueles que chegam à condição de príncipe
  apoiados na fortuna. Eles, em função da fortuna, saltam ou ``passam
  por cima'' literalmente dos problemas inerentes à conquista. Enquanto
  \emph{virtù} esta pressupõe esforço e labuta, o recurso à fortuna
  configura"-se como um salto dessas etapas, mas que geram, por
  consequência, maiores problemas na conservação, como destacará a
  sequência da exposição.} para esta condição, mas todas as dificuldades
surgem quando a ela chegam. {[}2{]} Estes são aqueles aos quais é
concedido um estado ou por dinheiro ou pela graça de quem o concede,
como sucedeu a muitos na Grécia, nas cidades da Jônia e do
Helesponto\footnote{Cidades gregas da antiga Ásia menor. Helesponto é o
  antigo nome da região dos Dardanelos, atual Turquia.}, onde foram feitos príncipes por Dario\footnote{Segundo
  Heródoto, o rei persa Dario \versal{I}, após conquistar parte da península
  grega, foi derrotado na batalha de Maratona (490 a.C.).}, para que as
conservassem para a sua segurança e glória; como foram feitos ainda
aqueles imperadores que, de cidadãos comuns -- mediante a corrupção dos
soldados --, tomaram o Império.

{[}3{]} Estes estão fundados unicamente na vontade e na fortuna de quem
lhes concedeu tal status, duas muito volúveis e instáveis, e não sabem e
não podem se manter naquele posto: não sabem, porque não sendo\footnote{Como
  destaca Inglese, aqui se encontra um exemplo de uma brusca passagem
  explicativa do plural para o singular.} um homem de grande engenho e
\emph{virtù}, não é razoável que, sempre vivendo como homens de condição
particular\footnote{O texto refere"-se aqui, aos homens que vivem em
  \emph{privata fortuna} (fortuna particular), que diz respeito a pouca
  ou nenhuma inserção política ou pública desses indivíduos. No caso
  ``\emph{privata}'' se apresenta como oposto ao ``público''. A
  \emph{fortuna} é aqui um qualificativo desta condição não pública, ou
  seja, reforça o caráter voltado para os interesses particulares desses
  indivíduos. Enfim, mais do que ressaltar o aspecto econômico,
  Maquiavel enfatiza a condição desses homens, que, ao não terem
  experiências de comando político (de vivência pública), não sabem
  comandar, contam apenas com a fortuna para conquistar a condição de
  príncipe. Importa destacar neste argumento a utilização do termo
  \emph{comandar}, uma das incumbências daquele que alcança a condição
  de príncipe.}, saibam comandar; não podem, porque não têm forças que
lhes possam ser amigas e fiéis. {[}4{]} Disso decorre que, os estados
que surgem subitamente, assim como todas as outras coisas da natureza
que nascem e crescem depressa, não podem ter raízes e
ramificações\footnote{\emph{Barbe e correspondenzie} são as raízes e as
  demais partes da planta que estão ligadas à terra, no caso, as demais
  ramificações. Conforme Inglese e Martelli, o termo
  \emph{correspondenzie} é de compreensão exata, pois poderia remeter
  ainda ao tronco, às folhas, ao caule etc. O termo ramificações aqui, a
  nosso ver, expressa melhor a alegoria descrita por Maquiavel.}, de
modo que o primeiro período de adversidade os extingue; exceto no caso
daqueles que, como foi dito, de repente se tornaram príncipes e tenham
tanta \emph{virtù} que saibam rapidamente organizar e conservar aquilo
que a fortuna lhes colocou no colo, e aqueles fundamentos, que os outros
construíram antes de se tornarem príncipes, eles construam depois.

{[}5{]} Desejo aduzir a um e a outro destes modos, de se tornar
príncipe, pela \emph{virtù} ou pela fortuna, com dois exemplos presentes
em nossa memória: Francisco Sforza\footnote{Francisco Sforza
  (1401-1466), conhecido pela sua grande habilidade militar. Casou"-se
  com Bianca Maria, filha do Duque de Milão (Filippo Maria Visconti), herdando o
  governo da cidade depois da morte do duque em 1447 e dos conflitos
  travados contra os partidários da República Ambrosina. Assumiu
  definitivamente o governo em 1448.} e César Bórgia. {[}6{]} Francisco,
com os meios adequados e com a sua grande \emph{virtù}, passou de
cidadão a duque de Milão, e aquilo que com mil percalços havia
conquistado, com pouco esforço conservou. {[}7{]} Por outro lado, César
Bórgia, chamado pelo povo de duque Valentino, conquistou o governo com a
fortuna do pai\footnote{Como já mencionado (nota 46) o pai de César era
  o papa Alexandre \versal{VI}.} e com a mesma o perdeu, apesar de ter ele usado
de todos os recursos e ter feito todas aquelas coisas que um homem
prudente e virtuoso deveria fazer para deitar suas raízes naqueles
governos, que as armas e a fortuna de outros lhe haviam concedido.
{[}8{]} Porque, como se disse anteriormente, aquele que não constrói
primeiro os fundamentos, poderia, com uma grande \emph{virtù},
construí"-los depois, ainda que se façam com incômodo para o arquiteto e
perigo para o edifício. {[}9{]} Se, então, considerarmos todos os
progressos do duque, veremos que ele construiu grandes fundamentos para
um poder futuro, sobre os quais não julgo supérfluo discorrer, porque
não saberia quais preceitos melhores dar a um príncipe novo, senão o
exemplo de suas ações; e se seus modos de proceder não lhe forem
proveitosos, não será por culpa sua, porque nasce de uma extraordinária
e extrema malignidade\footnote{A expressão ``extraordinária e extrema
  malignidade'' visa ressaltar os excessos da fortuna.} da fortuna.

{[}10{]} Alexandre \versal{VI}, no desejo de fazer grande o duque, seu filho,
tinha muitas dificuldades presentes e futuras. {[}11{]} Primeiro, ele
não via meios de poder fazer dele senhor de algum estado que não fosse
um estado da Igreja; e, apressando"-se em tomar algum estado da Igreja,
sabia que o duque de Milão\footnote{Ludovico Sforza (1452-1508), último
  filho de Francesco Visconti e Bianca Maria, assumiu o título ducal e o
  governo da cidade em 1494.} e os venezianos não lhe consentiriam,
porque Faenza e Rimini já estavam sob a proteção dos venezianos.
{[}12{]} Via, além disso, que as armas da Itália, e em particular
aquelas das quais seria possível se servir, estavam nas mãos daqueles
que deviam temer a grandeza do papa, e por isso não poderia confiar
nelas -- sendo todas elas dos Orsini e dos Colonna\footnote{Os Orsini e
  os Colonna eram as duas famílias mais poderosas de Roma no início do
  século \versal{XVI}, de
  modo que toda a eleição papal sofria forte influência dos interesses
  delas.} e dos cúmplices destes. {[}13{]} Era, portanto, necessário
romper aqueles equilíbrios e desordenar os seus estados da Itália, para
poder assenhorear"-se seguramente de parte deles. {[}14{]} O que lhe foi
fácil, porque encontrou os venezianos, que tinham decidido deixar os
franceses reentrar na Itália, rebelados por outros motivos\footnote{Para
  maiores informações sobre a disputa com os milaneses, confirir cap.
  \versal{III}, 32.}, o que não somente não foi dificultado, mas tornou"-se mais
fácil com a dissolução do antigo matrimônio do rei Luiz\footnote{Luiz
  queria se divorciar de sua primeira mulher, Joana, irmã de Carlos
  \versal{VIII}, para casar com a viúva deste, Ana da Bretanha. O papa Alexandre
  \versal{VI} concedeu o divórcio em outubro de 1498.}.

{[}15{]} O rei invadiu, pois, a Itália com a ajuda dos venezianos e o
consentimento de Alexandre; não foi primeiro em Milão que o papa obteve
dele gente para a sua empresa na Romanha\footnote{Esse trecho retoma
  aquilo que havia sido dito no cap. \versal{III} (37): ``mas assim que chegou a
  Milão, fez o contrário, auxiliou o papa Alexandre para que ele
  ocupasse a Romanha. Não percebeu que se enfraquecia com esta
  deliberação, privando"-se dos amigos e daqueles que tinham recorrido à
  sua proteção, e tornava grande a Igreja, acrescentando ao poder
  espiritual, que lhe dava tanta autoridade, muito poder temporal''.},
apoio que lhe foi concedido pela reputação do rei. {[}16{]} Conquistada
então a Romanha\footnote{Romanha é uma região ou província da Itália do
  centro"-norte, que teve várias cidades conquistadas pelas forças papais
  neste período: Imola (27 de novembro de 1499), Forlì (19 de dezembro
  de 1499), Cesena (2 de agosto de 1500), Rimini (10 de outubro de
  1500), Pesaro (21 de outubro 1500), Faenza (25 de abril de 1501).}
pelo duque\footnote{César Borgia, o duque Valentino.} e batidos os
Colonna, desejando o papa mantê"-la e prosseguir adiante, duas coisas o
impediam: uma, as suas armas, que não lhe pareciam fiéis; a outra, a
vontade da França. Caso as armas Orsini\footnote{As milícias comandadas
  por Paolo Orsini.}, das quais se tinha valido, viessem a lhe faltar,
isso não somente o impediria na conquista, mas lhe tiraria o que tinha
conquistado; e, ainda, que o rei não lhe fizesse algo semelhante.
{[}17{]} Dos Orsini teve a confirmação quando, depois da tomada de
Faenza, assaltou Bolonha, porque os viu frívolos neste assalto; acerca
do rei, conheceu o seu ânimo quando, conquistado o ducado de Urbino,
assaltou a Toscana: empresa da qual o rei fê"-lo desistir\footnote{Verifica"-se
  nas linhas 17 e 18 uma construção em paralelo das duas coisas que
  obstruiam o intento papal, a saber: a) a infidelidade dos exércitos
  que combatiam pelo papado, b) a falta de apoio do rei. Em seguida,
  para \emph{a}, corresponde o pouco empenho dos exércitos comandados
  por Paolo Orsini e, para \emph{b}, o não apoio do rei a sua gestão
  para que o papa não atacasse a região da Toscana.}.

{[}18{]} Assim que o duque deliberou não depender mais das armas e da
fortuna dos outros, a primeira iniciativa que tomou enfraqueceu os
partidários dos Orsini e dos Colonna em Roma, porque ganhou para si
todos os gentis"-homens\footnote{Expressão equivalente a aristocratas,
  nobres.} que os apoiavam, tornando"-os gentis"-homens dele, dando"-lhes
grandes pagamentos e honrando"-os\footnote{Honrar significa aqui designar
  para um cargo público. Desde a época da Roma antiga, é uma honraria
  ser nomeado para um cargo público.}, segundo a sua qualidade, com
cargos militares e de governo\footnote{\emph{Condotte} remete ao comando
  das forças militares das cidades.}, de modo que, em poucos meses,
perdeu"-se, nos seus ânimos, a afeição ao partido, e toda ela se voltou
ao duque. {[}19{]} Depois disto, esperou a ocasião para eliminar os
chefes dos Orsini, tendo já dispersado os da casa dos Colonna: o que lhe
foi conveniente e ele a usou melhor. {[}20{]} Porque os Orsini, tendo
percebido tardiamente que a grandeza do duque e da Igreja era a sua
ruína, fizeram uma reunião\footnote{Reunião aqui traduz \emph{dieta},
  termo que designa a reunião dos príncipes alemães para tomar alguma
  deliberação sobre o Estado. O termo é empregado aqui por Maquiavel com
  a mesma conotação, importada provavelmente por ocasião de sua visita
  às terras alemãs quando funcionário da Chancelaria.} em Magione, na
Perugia; dessa nasceu a rebelião de Urbino e os tumultos da Romanha e os
infinitos perigos para o duque, aos quais superou a todos eles com a
ajuda dos franceses. {[}21{]} E, tendo"-lhe retornado\footnote{Este é um
  particípio que não concorda em gênero com o substantivo
  (\emph{reputazione}).} a reputação, não confiando nem na França nem em
outras forças externas, para não ter que experimentá"-las contra si,
recorreu às artimanhas; e soube tanto dissimular a sua intenção, que os
Orsini, por intermédio do senhor Paulo, reconciliaram"-se com ele -- de
modo que o duque não deixou de se valer de todos os meios de ofício para
assegurar"-se disso, dando"-lhes dinheiro, roupas e cavalos --, tanto que
a simplicidade\footnote{Mais do que simplicidade, a \emph{simplicità}
  aqui mencionada é uma falta de perspicácia dos partidários de Paolo
  Orsini, que foram ingênuos em aderir a alguém que haviam traído antes.}
deles conduziu Sinigalia às suas mãos.

{[}22{]} Extintos, portanto, esses chefes e convertidos os partidários
deles em seus amigos, tinha o duque lançado muitos bons fundamentos para
o seu poder, tendo toda a Romanha com o ducado de Urbino, parecendo"-lhe,
sobretudo, que tivesse conquistado a amizade da Romanha e ganho todos
aqueles povos, porque começavam a provar o seu bem"-estar. {[}23{]} E
porque esta parte é digna de nota e de ser imitada por outros, não quero
deixá"-la para trás. {[}24{]} Tendo o duque tomado a Romanha e
encontrando"-a comandada por senhores impotentes -- os quais tinham mais
rapidamente espoliado os seus súditos do que os corrigido, dando"-lhes
motivo para a desunião e não para a união --, tanto que aquela província
estava cheia de latrocínios, de brigas e de todas as outras causas de
insolência, julgou ser necessário, por querer torná"-la pacífica e
obediente ao braço régio, dar"-lhe bom governo. Porém, designou
Senhor\footnote{Maquiavel utiliza o termo \emph{messere} que era
  utilizado como pronome de tratamento para os detentores de título
  honorífico dado aos juristas ou todo aquele que trabalhavam nos
  negócios jurídicos, tal como utilizado o título de doutor no Brasil
  para os profissionais que trabalham no mundo jurídico (advogados,
  delegados, juízes e promotores). Contudo, a origem do termo
  \emph{messere} é a contração de \emph{mio sire}, sendo \emph{sire} um
  termo antigo equivalente a \emph{signore}, donde \emph{messere}
  significar \emph{mio signore} (meu senhor ou monsenhor). Optamos aqui
  pela utilização de Senhor ao invés de doutor e monsenhor para traduzir
  \emph{messere}.} Remirro de Orco\footnote{Ramiro de Lorqua, mordomo de
  César Borgia, tornou"-se governador da Romanha em 1501.}, homem cruel e
diligente, ao qual deu pleníssimos poderes. {[}25{]} Este em pouco tempo
a tornou pacífica e unida, com grandíssima reputação. {[}26{]} Depois
julgou o duque não ser necessária tão excessiva autoridade, porque
receava torná"-la odiosa, e instituiu um tribunal civil no centro da
província, com um presidente excelentíssimo, no qual toda cidade tinha
seu advogado. {[}27{]} E porque sabia que a severidade do passado havia
gerado neles algum ódio, queria, para purgar os ânimos daqueles povos e
ganhá"-los totalmente, mostrar que, se alguma crueldade havia sido
cometida, não nasceu dele, contudo da natureza cruel do ministro.
{[}28{]} E aproveitou esta ocasião para colocá"-lo numa manhã na praça,
em Cesena, dividido em duas partes: com um pau e uma faca ensanguentada
do lado. A ferocidade daquele espetáculo fez aquele povo ficar ao mesmo
tempo satisfeito e estupefato.

{[}29{]} Mas retornemos ao ponto de onde partimos. Digo que,
encontrando"-se o duque muito forte e em parte assegurado contra os
perigos presentes, por ter"-se armado a seu modo e ter, em boa parte,
eliminado aquelas armas que, próximas, poderiam prejudicá"-lo, querendo
prosseguir com a conquista, restava"-lhe o respeito do rei de França,
porque sabia o rei, o qual tardiamente tinha se dado conta dos seus
erros, que não continuaria a suportá"-lo. {[}30{]} Por isso, começou a
procurar novas amizades e a falhar com a França, na incursão que os
franceses fizeram ao reino de Nápoles contra os espanhóis que assediavam
Gaeta\footnote{Os franceses assediaram Gaeta, cidade napolitana, em
  junho de 1503 e deixaram"-na em 1º de janeiro de 1504.}; sua intenção
era assegurar"-se contra eles, o que ele teria conseguido rapidamente se
Alexandre estivesse vivo\footnote{O papa Alexandre \versal{VI} morre em 18 de
  agosto de 1503, enquanto durava o assédio à cidade de Gaeta. Com a
  convocação do Conclave (reunião dos cardeais para a escolha de um novo
  papa) antes da consolidação de suas conquistas, César Borgia não teve
  as condições adequadas para manter o seu poder.}. {[}31{]} Estas foram
as suas ações quanto às coisas presentes.

{[}32{]} Mas quanto às coisas futuras, ele tinha de desconfiar,
primeiro, que um novo sucessor da Igreja não lhe fosse amigo e
procurasse tirar"-lhe aquilo que Alexandre tinha lhe dado. {[}33{]}
Contra o que pensou em assegurar"-se de quatro modos: primeiro, eliminar
toda a linhagem daqueles senhores que ele tinha espoliado, para retirar
do papa a ocasião de restituir"-lhes os domínios; segundo, ganhar para si
todos os gentis"-homens de Roma, como foi dito, para poder
com eles ter o papa sob controle; terceiro, tornar o Colégio
Cardinalício\footnote{Isto é, o Colégio dos Cardeais ou Colégio,
  responsável pela eleição do papa.} tão seu quanto possível; quarto,
antes que o papa morresse, conquistar tanta força, que pudesse, por si
mesmo, resistir a um primeiro ataque do novo papa. {[}34{]} Destas
quatro coisas, à época da morte de Alexandre, ele já havia conseguido
três, a quarta quase conseguira, porque dos senhores espoliados matou
tantos quanto pode alcançar e pouquíssimos se salvaram; e ganhou os
gentis"-homens romanos; e no Colégio tinha o apoio de uma grandíssima
facção. Quanto às novas conquistas, havia planejado tornar"-se senhor da
Toscana e já possuía Perugia e Piombino, e de Pisa tinha conquistado o
apoio. {[}35{]} E como não tivesse mais de ter cuidados com a França --
que não lhe preocupava mais, pois os franceses já tinham sido espoliados
do reino de Nápoles pelos espanhóis, de sorte que cada um deles tinha a
necessidade de comprar sua amizade -- ele teria avançado sobre Pisa.
{[}36{]} Depois disto, Lucca e Siena cederiam seu apoio a ele
rapidamente, em parte pela inveja dos florentinos, em parte por medo; e
os florentinos não teriam remédio. {[}37{]} Se César Borgia o tivesse
conseguido -- e teria conseguido no mesmo ano que Alexandre morreu --,
teria conquistado tanta força e tanta reputação que por si mesmo
ter"-se"-ia mantido e não mais seria dependente da fortuna e força dos
outros, mas da sua potência e da sua \emph{virtù}\footnote{Para
  Maquiavel o ano de 1503 foi central na sorte política de César Borgia.
  Pelo raciocínio apresentado, se o papa tivesse sobrevivido mais 6
  meses César teria conquistado todos os seus objetivos políticos e
  consolidado seus domínios na Itália. Neste caso, a fortuna lhe faltou
  apesar de toda a sua \emph{virtù.}}.

{[}38{]} Mas Alexandre morreu cinco anos depois que ele começou a usar a
espada: deixando"-o somente com o estado da Romanha consolidado e todos
os demais no ar, entre dois poderosos exércitos inimigos e doente de
morte\footnote{No mesmo período da morte do papa Alexandre \versal{VI}, César
  Borgia também fica gravemente doente e, segundo Biaggio Buonacorsi,
  também diplomata em Florença e amigo da Maquiavel, suspeitava"-se que
  ambos, o papa e César, haviam sido envenenados.}. {[}39{]} E havia no
duque tanta ferocidade e tanta \emph{virtù}, tão bem se conhece como os
homens podem conquistar ou perder, e tão válidos foram os fundamentos
que tinha lançado em tão pouco tempo, que, se não tivesse tido aqueles
exércitos contra si, ou estivesse são, teria resistido a todas as
dificuldades.

{[}40{]} E viu"-se que os seus fundamentos eram bons: porque a Romanha o
esperou por mais de um mês; em Roma, ainda que semi"-vivo, esteve seguro,
e, embora os Ballioni, os Vitelli e os Orsini retornassem a Roma, não
obtiveram seguidores contra ele; e ele conseguiu, se não fazer papa quem
ele queria, ao menos que não se tornasse papa quem ele não desejava.
{[}41{]} Mas se na morte de Alexandre ele estivesse são, todas as coisas
lhe teriam sido fáceis: e ele me disse, nos dias em que Júlio \versal{II} foi
eleito papa, que pensou no que poderia ocorrer quando seu pai tivesse
morrido, e para tudo havia encontrado remédio, mas nunca pensara que, na
morte de seu pai, também ele estaria morrendo.

{[}42{]} Colhendo, pois, todas as ações do duque, eu não saberia
repreendê"-lo, mas antes me parece o caso de, como tenho feito, propô"-lo
imitável a todos aqueles que, pela fortuna e com as armas alheias,
ascenderam ao poder, porque ele, tendo grande ânimo e intenção elevada,
não poderia governar de outro modo, e somente se opuseram aos seus
desígnios a brevidade da vida de Alexandre e a sua doença. {[}43{]}
Quem, portanto, julga necessário no seu principado novo assegurar"-se
contra os inimigos, ganha para si amigos; vencer ou pela força ou pela
fraude; ser amado e temido pelo povo, seguido e reverenciado pelos
soldados; eliminar aqueles que podem ou devem prejudica"-lo; inovar com
novos costumes os antigos ordenamentos; ser severo e grato, magnânimo e
liberal, acabar com a milícia infiel, criar uma nova; manter a amizade
dos reis e dos príncipes de modo que te beneficiem"-no com a sua graça ou
ofendam"-no com cautela; não se pode encontrar exemplos mais recentes do
que as ações deste\footnote{Ou seja, César Borgia.}.

{[}44{]} Somente se pode acusá"-lo na eleição do pontífice Júlio, na qual
o duque fez uma má escolha. {[}45{]} Porque, como foi dito, não podendo
fazer um papa a seu modo, ele poderia impedir que alguém o fosse; e não
deveria nunca consentir o papado àqueles cardeais que ofendeu ou que,
tornando"-se papa, tivessem de ter medo dele, porque os homens ofendem ou
por medo ou por ódio. {[}46{]} Aqueles que ele ofendeu eram, entre
outros, os cardeais de São Pedro em Víncula\footnote{Giuliano Della
  Rovere.}, Colonna\footnote{Giovanni Colonna.}, São Jorge\footnote{Rafaello
  Riario.}, Ascânio\footnote{Ascanio Sforza.}; todos os outros, tornados
papa, tinham de temê"-lo, exceto o de Ruão\footnote{George d'Ambroise,
  arcebispo de Ruão.} e os espanhóis: estes por parentesco e obrigação,
aquele pelo poder, pois tinha ao seu favor o reino de França.

\quebra

{[}47{]}
Portanto, o duque, antes de tudo, deveria fazer papa um espanhol, e, não
podendo, deveria consentir que o fosse o cardeal de Ruão e não o de São
Pedro em Víncula. {[}48{]} E quem crê que nos grandes personagens os 
benefícios novos fazem esquecer as antigas injúrias, se engana. {[}49{]}
Errou, pois, o duque nesta eleição; e foi a razão de sua ruína final.

\quebra\section{\emph{DE HIS QUI PER SCELERA AD~PRINCIPATUM~PERVENERE}\break
{[}Daqueles que por atos criminosos conquistaram principados{]}}

{[}1{]} Mas porque há ainda dois modos de se passar de cidadão comum a
príncipe\footnote{Note"-se que a frase que abre o capítulo apresenta a
  mesma temática do capítulo precedente: analisar aqueles que passam da
  condição de cidadãos para a condição de príncipe {[}\emph{diventa di
  privato in príncipe}{]}. Esse é, pois, um dos temas centrais do livro,
  conforme exposto na \emph{Introdução.}}, o que não se pode atribuir de
todo ou à fortuna ou à \emph{virtù}, não me parece que deva deixá"-las de
lado, ainda que sobre uma delas se possa discorrer mais amplamente em se
tratando de repúblicas\footnote{No caso das repúblicas, trata"-se de um
  cidadão que obtém a condição de príncipe em função da sua \emph{virtù}
  com o favor de seus concidadãos.}. {[}2{]} Estes modos são: ou quando
por algum meio criminoso e nefasto alguém ascende ao principado, ou
quando um cidadão comum, com o favor de outros cidadãos, torna"-se
príncipe da sua pátria. {[}3{]} E falando do primeiro modo, mostrar"-se"-á
com dois exemplos, um antigo e outro moderno, sem entrar nos méritos
desta parte, porque eu julgo que basta, a quem for necessário,
imitá"-los.

{[}4{]} Agátocles\footnote{Soberano de Siracusa de 316 a 289 a.C.}
siciliano, não só de condição particular\footnote{Essa é a mesma
  expressão do capítulo precedente, cf. nota 66.}, mas também ínfima e
abjeta, tornou"-se rei de Siracusa. {[}5{]} Este homem, nascido de um
oleiro, sempre teve, nas diferentes fases da sua vida, uma conduta
celerada; não obstante, associou a seus crimes tanta \emph{virtù} de
alma e corpo, que, ingressando na milícia\footnote{Milicias eram
  formações armadas, contratadas por cidades ou aristocratas para ações
  militares específicas, se distinguindo, dos exércitos regulares. Esse
  tipo de formação era muito utilizado pelas cidades italianas na
  Renascença, haja vista, que muitas delas não possuíam exércitos
  regulares.}, pelos seus diversos graus, chegou a ser pretor\footnote{Cargo
  romano responsável pelo supremo comando militar ou pela magistratura
  suprema numa determinada localidade.} de Siracusa. {[}6{]} Ao ser
investido em tal posto, decidiu tornar"-se príncipe e manter com
violência e sem obrigação a outrem aquilo que lhe tinha sido concedido
por um acordo. Tendo acordado este seu plano com o cartaginês
Amílcar\footnote{Conforme explica Inglese, a história é mais complexa do
  que esse rápido relato de Maquiavel. Diz Inglese: ``Os episódios
  narrados por Justino são mais complexos: Agátocles tentou tomar o
  poder, mas foi mandado ao exílio; tomou em vão as armas contra a
  pátria e, então, rogou ao cartaginês Amílcar (não confundí"-lo com o
  homônimo Amílcar Barca, general cartaginês) de fazer a pacificação
  entre ele e os siracusanos. Em consequência disto obteve o título de
  pretor de Siracusa e, por fim, com o apoio de cinquenta mil homens de
  Amilcar, tomou o governo da cidade.'' {[}Inglese, 1995, p. 55, nota
  3{]}.} -- o qual militava com os seus exércitos na Sicília --, reuniu
certa manhã o povo e o senado de Siracusa, como se fosse deliberar algo
pertinente à república. {[}7{]} E com um aceno combinado, ordenou aos
seus soldados matarem todos os senadores e os mais ricos do povo. Mortos
estes, ocupou e manteve o principado daquela cidade sem nenhuma
controvérsia civil. {[}8{]} E embora fosse duas vezes derrotado pelos
cartagineses e, por fim, assediado por eles, não somente pode defender a
sua cidade (porém, deixando parte de seu exército na defesa dela), como
com a outra parte assaltou a África e em pouco tempo libertou Siracusa
do assédio e conduziu os cartagineses a um extremo perigo. Em seguida os
obrigou a um acordo com ele: os cartagineses ficaram com a possessão da
África e deixaram a Sicília para Agátocles.

{[}9{]} Quem considerar, portanto, as ações e a vida deste homem verá
pouca ou nenhuma coisa que se possa atribuir à fortuna, porque, como se
disse acima, não pelo favor de alguém, mas pelos postos da milícia -- os
quais conquistou com mil incômodos e perigos -- alcançou o principado; e
este, posteriormente, conservou com muitas resoluções corajosas e
perigosas. {[}10{]} Não se pode também chamar de \emph{virtù} matar os
seus cidadãos, trair os amigos, agir de má"-fé, sem piedade, sem
religião: meios estes que permitem conquistar poder, mas não
glória\footnote{Maquiavel toca aqui num ponto central de sua teoria
  política, a necessidade de glória para que haja de fato \emph{virtù}.
  Glória entendida como admiração e louvor perante os outros. No caso de
  Agátocles, ele tinha respeito dos siracusanos, mas não era digno de
  admiração e louvor, donde a falta de glória.}. {[}11{]} Porque, se se
considera a \emph{virtù} de Agátocles ao entrar e ao sair dos perigos, e
a grandeza do seu ânimo ao suportar e superar as coisas adversas, não se
vê porque ele haveria de ser julgado inferior a qualquer excelentíssimo
capitão: todavia, a sua feroz crueldade e desumanidade, com infinitos
crimes, não permitiram que fosse celebrado entre os excelentíssimos
homens. {[}12{]} Não se pode, portanto, atribuir à fortuna ou à
\emph{virtù} aquilo que ele conseguiu sem uma e sem a outra.

{[}13{]} No nosso tempo, reinando Alexandre \versal{VI}, Oliverotto de
Fermo\footnote{Oliverotto Euffreducci da Fermo nasceu em 1475 e morreu
  estrangulado a mando de César Bórgia em 31 de dezembro de 1502, visto
  que era um dos conspiradores da reunião em Perúgica, conforme narrado
  no capítulo \versal{VII} {[}20{]}. Ele pertencia a uma família nobre da cidade
  e, conforme Guicciardini era um valente soldado e foi um dos
  auxiliares dos comandantes militares Vitellozzo Vitelli e Paolo
  Orsini.}, tendo ficado, quando era ainda pequeno, sem seus pais, foi
criado por um tio materno, chamado Giovanni Fogliani, e nos primeiros
anos de sua juventude foi destinado a servir sob o comando de Paulo
Vitelli\footnote{Paolo Vitelli, da cidade de Castelo, foi comandante
  militar (\emph{condottiero}) de grande fama, tendo papel de destaque
  em junho de 1498, quando comandou os soldados florentinos na
  reconquista de Pisa.}, a fim de que, pleno daquela disciplina,
alcançasse excelente posto na milícia. {[}14{]} Morto depois Paulo,
serviu sob o comando de Vitellozzo, irmão de Paulo, e, em brevíssimo
tempo, por ser engenhoso e corajoso de corpo e de alma, tornou"-se o
primeiro homem\footnote{\emph{Primo uomo} que bem pode ser compreendido
  aqui como aquele que lidera, que está na vanguarda, que principia, ou
  seja, um \emph{príncipe}. Nos capítulos adiante Maquiavel insistirá na
  necessidade do príncipe ser também um comandante militar.} da sua
milícia. {[}15{]} Contudo, parecendo coisa servil estar sob as ordens de
outro, pensou em ocupar Fermo com a ajuda de alguns cidadãos desta
cidade -- aos quais era mais cara a servidão do que a liberdade da sua
pátria -- e com o patrocínio de Vitellozzo. {[}16{]} E escreveu a
Giovanni Fogliani que, por ter estado muitos anos fora de casa, desejava
revê"-lo e rever sua cidade, e inspecionar algumas partes do seu
patrimônio. E porque não se tinha esforçado senão em conquistar
honrarias, a fim de que os seus cidadãos vissem como ele não perdera o
seu tempo em vão, queria vir à cidade de forma honrada e acompanhado por
cem cavaleiros de sua amizade e seus auxiliares. Pediu, pois, a seu tio
que fizesse com que eles fossem recebidos honrosamente pelos cidadãos de
Fermo, o que não traria honra apenas a ele, seu tio, mas também ao
próprio Oliverotto, seu discípulo.

{[}17{]} Não faltou Giovanni, portanto, com nenhum de seus deveres em
relação a seu sobrinho: fez os firmianos receberem"-no honradamente e se
alojou\footnote{Aqui uma brusca mudança de sujeito; o sujeito é
  Oliverotto, que se aloja em sua própria casa em Fermo.} em sua casa,
onde, passando alguns dias empenhado em tramar secretamente aquilo que
era necessário a seus futuros crimes, fez um convite soleníssimo, com o
qual convidou Giovanni Fogliani e todos os principais homens de Fermo.
{[}18{]} E uma vez consumidas as iguarias e todos os outros
entretenimentos empregados comumente nos banquetes, Oliverotto,
artificialmente, teceu certos argumentos graves, falando da grandeza do
papa Alexandre e de César, seu filho, e dos seus feitos. A tais
raciocínios responderam Giovanni e os outros. De repente Oliverotto se
levantou, dizendo que aquilo era coisa para falar em lugar mais secreto
e retirou"-se para uma câmara, onde Giovanni e todos os outros cidadãos
lhe seguiram. {[}19{]} Nem bem tinham se sentado quando, de lugares
secretos daquela câmara, saíram soldados que mataram Giovanni e todos os
outros. {[}20{]} Depois de tais homicídios, Oliverotto montou a cavalo e
atravessou correndo a cidade, e assediou no palácio o supremo
magistrado\footnote{A Magistratura Suprema era composta pelos membros do
  Conselho Geral, a maior instância deliberativa da cidade. Em geral,
  esses conselhos eram dominados pelas aristocracias locais.}, tanto
que, por medo, os firmianos foram constrangidos a obedecê"-lo e a formar
um governo do qual se fez príncipe. Mortos todos aqueles, que
descontentes, podiam prejudicá"-lo, fortaleceu"-se com novos ordenamentos
civis e militares, de modo que, no espaço de um ano depois de ter tomado
o principado, ele não somente estava seguro na cidade de Fermo, mas se
tornou temido por todos os seus vizinhos. {[}21{]} E seria a sua
expulsão difícil, como aquela de Agátocles, se não se tivesse deixado
enganar por César Bórgia, quando em Sinigallia, como acima se
disse\footnote{Capítulos \versal{VII} {[}20-21{]}.}, foram presos os Orsini e os
Vitelli, e onde foi preso ele também. Um ano depois de cometer o
parricídio\footnote{Isto é, o tio Giovanni, que fez as vezes de pai para
  Oliverotto.}, foi estrangulado, junto com Vitellozzo, que foi mestre
da sua \emph{virtù} e de seus crimes.

{[}22{]} Poderia alguém se perguntar como foi possível que Agátocles e
alguns semelhantes, depois de infinitas traições e crueldades, pudessem
viver longamente seguros na sua pátria e se defenderem dos inimigos
externos, sem que seus cidadãos jamais tivessem conspirado contra eles:
e isto, apesar de muitos outros não terem podido, por meio da crueldade,
conservar o governo nem nos tempos de paz, nem nos duvidosos tempos de
guerra. {[}23{]} Creio que isto advenha da crueldade mal usada ou bem
usada. {[}24{]} ``Bem usadas'' se podem chamar aquelas -- se é lícito
falar bem do mal\footnote{Aqui Maquiavel indica um tema que será melhor
  analisado nos capítulos \versal{XV} e \versal{XVII}.} -- que se fazem de uma só vez pela
necessidade de assegurar"-se, e depois não se insiste mais nelas, mas se
convertem na maior quantidade possível de benefícios para os súditos.
{[}25{]} ``Mal usadas'' são aquelas as quais, ainda que no princípio
sejam poucas, rapidamente crescem com o tempo, em vez de se extinguirem.
{[}26{]} Aqueles que observam o primeiro modo, podem com Deus e com os
homens ter algum remédio para o seu governo, como teve Agátocles; os que
empregam mal a crueldade não conseguem se manter.

{[}27{]} Donde é de se notar que, ao pilhar um governo, deve o invasor
fazer todas aquelas afrontas que são necessárias, e fazê"-las de uma só
vez, para não ter de renovar tudo e para poder, não as renovando,
tranquilizar os homens e ganhá"-los ao beneficiá"-los. {[}28{]} Quem faz
de outro modo, ou por timidez ou por mau conselho, sempre precisa ter a
faca na mão; também não pode nunca se apoiar nos seus súditos, nem podem
estes, pelas injúrias recentes e contínuas, jamais confiar nele.
{[}29{]} Por isso, as injúrias devem ser feitas todas de uma só vez, a
fim de que se saboreiem menos e afrontem menos; os benefícios se devem
fazer pouco a pouco, afim de serem melhor saboreados. {[}30{]} E um
príncipe deve, sobretudo, viver com os seus súditos de modo que nenhum
acidente, mau ou bom, obrigue"-o a mudar, porque, advindo as necessidades
em tempos adversos, você não terá tempo para o mal, e o bem que fizer
não lhe serve, porque é julgado forçado, e não terá com ele nenhum
reconhecimento.

\quebra\section{\emph{DE PRINCIPATU CIVILI}\break
{[}Do principado civil{]}}

{[}1{]} Voltando à outra parte\footnote{No início do cap. \versal{VIII} (2),
  Maquiavel havia apresentado dois modos de se ascender à condição de
  príncipe (não calcado totalmente na \emph{virtù} e nem na fortuna), a
  saber: por meio de atos criminosos -- conforme exposto no capítulo
  \versal{VIII}, e quando um cidadão, com auxílio de seus concidadãos, torna"-se
  príncipe e lidera a cidade. É deste segundo ponto que Maquiavel
  tratará aqui. A conjunção italiana \emph{ma} não possui, portanto,
  valor adversativo e sim consecutivo, donde não ser indicado inseri"-la
  na tradução para que não haja uma confusão na interpretação.}, quando
um cidadão comum\footnote{Conforme explicado na nota 66, traduzimos aqui
  \emph{privato ciptadino} por ``cidadão comum''. Entretanto, devemos
  ter claro, não se trata de um cidadão qualquer, mas daquele que assume
  o comando ou a liderança das ações políticas. Neste caso, ao usar o
  termo \emph{privato}, Maquiavel parece querer reforçar essa
  possibilidade de transformação do \emph{status} do homem, que pode
  sair de sua condição de privada, particular, alheia às demandas
  públicas e se engajar na esfera pública, seja ocupando cargos, seja
  liderando politicamente a cidade. A condição particular do cidadão,
  pela expressão construída, enfatiza, neste caso, a antítese da
  condição deste indivíduo em relação à esfera pública. Maquiavel dá a
  entender que escolheu o indivíduo mais avesso ao mundo da política
  para transformá"-lo em modelo de ação política, deixando claro que a
  todos está aberta a possibilidade de se inserir na esfera pública,
  desde que busque agir com \emph{virtù} e saiba lidar com os desafios
  da fortuna.}, não por meio de crimes ou outra violência intolerável,
mas com o favor dos outros cidadãos, torna"-se príncipe da sua pátria --
que poderia ser chamada de principado civil\footnote{Destaque"-se que é a
  primeira vez que Maquiavel se refere a um principado no singular,
  visto que todos os outros exemplos são dados no plural. Trata"-se,
  pois, de um caso especial no qual a grande maioria dos comentadores do
  pensamento político de Maquiavel entende ser um modelo de regime mais
  indicado para a crise das repúblicas, crises estas constantes em
  Florença, bem como nas demais repúblicas de seu tempo.}: e para sê"-lo
não é necessário toda \emph{virtù} ou toda fortuna, mas, antes, uma
astúcia afortunada\footnote{A \emph{astuzia} (astúcia) se não é
  entendida como sinônimo da \emph{virtù} é certamente uma de suas
  qualidades principais, conforme se verá no cap. \versal{XVIII}. Essa expressão
  sintetiza a condição principal de um cidadão que deseje tornar"-se
  príncipe: ter astúcia e \emph{virtù}, mas que elas venham acompanhadas
  da fortuna, donde a \emph{astuzia afortunada}, ou seja, astúcia e
  \emph{virtù} com fortuna.} --, digo que se ascende a este principado
ou com o favor do povo\footnote{Povo (\emph{populo}) não deve ser
  entendido aqui como sendo somente os pobres, o oposto dos ricos ou dos
  grandes, mas conforme explicou Tafuro, trata"-se, para Maquiavel, de
  uma denominação que inclui vários setores sociais, principalmente as
  parcelas medianas, no caso, algo que seria para nós a classe média
  composta de artesãos, comerciantes, pequenos proprietários. Cf.
  Tafuro, A. \emph{La formazione di Niccolò Machiavelli.} Napoli: Dante
  \& Descates, 2003, {[}parte \versal{I}, 1.2 e 1.3{]}.} ou com o favor dos
grandes\footnote{O termo \emph{grande} é sinônimo também de nobre ou
  aristocrata.}. {[}2{]} Porque em toda cidade se encontram estes dois
humores\footnote{Maquiavel recupera aqui uma imagem clássica da
  medicina, mostrando que entende a cidade como um corpo político, tal
  qual o pensamento político clássico sempre a entendeu. Os humores na
  medicina antiga eram quatro: o sanguínio, o fleumático, a bile negra e
  a bile amarela. Da combinação desses humores nascem os diversos
  temperamentos dos corpos. Aqui trata"-se de forças políticas que atuam
  na cidade e, à semelhança do que ocorre nos corpos, na predominância
  de um desses humores, a cidade apresentará um certo ``temperamento''
  ou tendência. Essa imagem está também nos \emph{Discursos sobre a
  primeira década de Tito Lívio,} livro \versal{I}, cap. 4 (5): ``e' sono in ogni
  republica due umori diversi, quello del popolo e quello de' grandi.''}
diversos e nasce, disto, que o povo deseja não ser nem comandado nem
oprimido pelos grandes e os grandes desejam comandar e oprimir o povo.
Destes dois apetites diversos nasce na cidade um destes três efeitos: ou
o principado, ou a liberdade ou a licença\footnote{Nota sobre licença
  como regime político.}. {[}3{]} O principado origina"-se do povo ou dos
grandes, segundo que uma ou outra destas partes tenha a ocasião, porque,
vendo os grandes que não podem resistir ao povo, começam a aumentar a
reputação e o prestígio de um dos seus e fazem"-no príncipe para poderem,
sob sua proteção, desafogar o seu apetite. O povo, também, vendo que não
pode resistir aos grandes, aumenta a reputação de um e o faz
príncipe\footnote{Neste caso, para aquele que se torna príncipe com o
  favor do povo, não se trata de alguém do próprio povo. Maquiavel não
  apresenta esta restrição, que se verifica no caso dos grandes que
  escolhem alguém de seu grupo social. Para o caso do povo pode ocorrer
  que seja até um grande o escolhido pelo povo, desde que ele defenda o
  povo do desejo de dominação dos grandes.}, para serem defendidos pela
sua autoridade.

{[}4{]} Aquele que chega ao principado com a ajuda dos grandes,
conserva"-se com mais dificuldade do que aquele que chega com a ajuda do
povo, porque, como príncipe, encontra"-se com muitos ao entorno que se
lhe equiparam como seus iguais e, por isto, não lhes pode nem comandar
nem guiá"-los a seu modo. {[}5{]} Mas, aquele que chega ao principado
pelo favor popular, encontra"-se sozinho e tem entorno a si ou nenhum ou
pouquíssimos que não estejam dispostos a lhe obedecer. {[}6{]} Além
disto, não se pode, com honestidade, satisfazer os grandes sem injuriar
outros, mas ao povo sim, porque o fim do povo é mais honesto que o dos
grandes, querendo esses oprimir e aqueles não ser oprimidos\footnote{Essa
  é uma máxima de Maquiavel para caracterizar os desejos ou humores
  opostos dos grandes e do povo: os grandes querem oprimir e o povo não
  quer ser oprimido. Essa caracterização é fundamental para a reflexão
  política maquiaveliana, pois dela é que derivam várias consequências
  argumentativas.}. {[}7{]} Além disso, um príncipe inimigo do povo não
pode nunca estar seguro, por serem esses muitos; contra os grandes pode
estar seguro, por serem poucos. {[}8{]} O pior que pode acontecer a um
príncipe inimigo do povo é ser abandonado por ele, mas dos grandes, que
lhe são inimigos, não somente deve temer ser abandonado, mas ainda mais
eles lhe venham contra, porque, tendo estes mais visão e mais astúcia,
sempre antecipam o tempo para se salvarem e buscarem as graças de quem
esperam que vença. {[}9{]} É necessário, também, ao príncipe, viver
sempre com o mesmo povo, contudo, pode muito bem dispensar os mesmos
grandes, podendo fazer e se desfazer deles todo dia, e tirar"-lhes e lhes
dar, a seu bel prazer, a sua reputação.

{[}10{]} Para esclarecer melhor esta parte, digo que se devem considerar
os grandes de dois modos principais: ou se governam de maneira que, com
o seu proceder, estejam totalmente vinculados a sua fortuna, ou não.
{[}11{]} Aqueles que se vinculam a você, e não são rapaces, devem"-se
honrar e amar. {[}12{]} Aqueles que não se vinculam a você devem ser
examinados de dois modos: ou eles fazem isto por pusilanimidade e
defeito natural de ânimo\footnote{\emph{Difetto naturalle d'animo}
  (defeito natural de ânimo) trata"-se de falta de coragem, completando e
  explicando a condição do pusilânime (volúvel).} -- então tu deves
servir"-te mais ainda daqueles que são bons conselheiros, porque na
prosperidade te honram e na adversidade não terás que temê"-los --,
{[}13{]} ou quando eles não se vinculam a você por artifício e por
motivos ambiciosos, é sinal de que pensam mais em si mesmos que em você.
Deve o príncipe, pois, proteger"-se e temê"-los como se fossem inimigos
declarados, porque sempre, nas adversidades, colaboram para arruiná"-lo.

{[}14{]} Deve, portanto, alguém que se torna príncipe mediante o favor
do povo, conservar"-se amigo dele, o que faz facilmente, pois não desejam
eles senão não ser oprimidos. {[}15{]} Mas alguém que, contra o povo,
torna"-se príncipe com o favor dos grandes, deve, antes de qualquer outra
coisa, procurar ganhar o povo para si, o que ele faz facilmente, quando
assume a proteção do povo. {[}16{]} E porque os homens ficam mais
ligados a seu benfeitor quando recebem o bem de quem acreditavam receber
o mal, torna"-se o povo imediatamente mais benévolo para com ele do que
se tivesse ele chegado ao principado com os favores do povo. {[}17{]} E
pode o príncipe ganhá"-los de muitos modos: aos quais, porque variam
segundo a matéria\footnote{Isto é, conforme a lugar, conforme a cidade.},
não se pode dar uma regra certa e constante, e, por isso, deixaremos de
lado essa questão. {[}18{]} Concluo, somente que, a um príncipe, é
necessário ter o povo como amigo, pois, de outro modo, não tem remédio
nas adversidades. {[}19{]} Nabis\footnote{Nabis, tirano de Esparta de
  206 a 192 a.C.}, príncipe dos espartanos, suportou o assédio de toda
Grécia e de um exército romano vitoriosíssimo e defendeu contra eles a
sua pátria e o seu status. Bastou"-lhe somente, quando sobreveio o
perigo, assegurar"-se contra poucos, o que, caso ele fosse inimigo do
povo, não lhe teria sido suficiente.

{[}20{]} E não me venha alguém refutar esta minha opinião com aquele
provérbio trivial, de que quem se apóia sobre o povo, apóia"-se sobre a
lama: porque, com efeito, isto é verdadeiro quando um cidadão comum faz
do povo o seu fundamento, e ilude"-se que o povo o libertará quando ele
for oprimido pelos inimigos ou pelos magistrados\footnote{Esse
  raciocínio complementa aquilo que foi dito no início do capítulo.
  Maquiavel não afirmou na linha 3 que o povo escolhe um dos seus para
  príncipe, mas alguém que o defenda da vontade opressora dos grandes,
  sem distinção aqui de grupo social. Agora ele explicita essa ideia ao
  declarar que o cidadão comum (no caso um \emph{privato ciptadino}, que
  assumiu a condição de príncipe) que se apóia tão somente no povo, não
  tem respaldo o bastante para manter"-se na condição de comando. Em
  suma, faz"-se necessário a este \emph{privato ciptadino} buscar alguma
  forma de apoio dos grandes. No limite, o príncipe tem sempre que
  buscar o apoio do outro grupo político: se ele for membro dos grandes,
  necessita de apoio do povo, se ele for um \emph{privato ciptadino},
  tem que buscar apoio dos grandes.}. {[}21{]} Neste caso se poderia
frequentemente enganar, como em Roma os Gracos\footnote{Ou seja, os
  irmãos Tibério Sempronio e Caio Sempronio, líderes da revolução
  popular de Roma e Tribunos da Plebe, sendo o primeiro assassinado em
  133 a.C., e o segundo em 121 a.C.} e em Florença o senhor Giorgio
Scali\footnote{Giorgio Scali foi eleito membro da \emph{Signoria} em 1
  de setembro de 1378 após os conflitos do \emph{Ciompi}. Por três anos
  foi um dos líderes da cidade, com o apoio das \emph{Artes Menores}, ou
  seja, os artesãos em oposição aos grandes comerciantes, mas caiu em
  desgraça e foi decapitado, por vontade da \emph{Signoria}, em 17 de
  janeiro de 1382.}. {[}22{]} Mas, sendo um príncipe que se funda sobre
o povo, sendo que ele pode comandar e é homem de coragem, que não se
amedronta nas adversidades, nem lhe falta outros preparos e mantém com a
sua coragem e seus ordenamentos todos animados, nunca se encontrará
enganado pelo povo e lhe parecerá ter bem feito os seus fundamentos.
{[}23{]} Costumam estes principados correr perigo quando passam de
principado civil a principado absoluto. {[}24{]} Porque estes príncipes
ou comandam por si mesmo ou por meio dos magistrados: no último caso é
mais débil e mais perigosa a situação deles, porque estão em tudo
submetidos à vontade daqueles cidadãos que são prepostos como
magistrados, os quais, principalmente nos tempos adversos, lhe podem
retirar o estado com grande facilidade ou abandonando"-o\footnote{Existe
  uma divergência nas edições de Giorgio Inglese e Mario Martelli: o
  primeiro apresenta a expressão \emph{o con abbandonarlo} e o segundo
  \emph{o con non lo obedire}. Contudo, como explica Martelli em sua
  nota (nota 39, p. 170) ``basta ter em conta o fato que abandoná"-lo e
  obedecê"-lo são paleograficamente conversíveis um pelo outro''.
  Portanto, seja qual for a expressão utilizada, o sentido
  conservar"-se"-ia equivalente.}, ou não o obedecendo, ou agindo
ativamente contra o príncipe. {[}25{]} E o príncipe não tem tempo, nos
momentos perigosos, em retomar a autoridade absoluta, porque os cidadãos
e os súditos, que estavam habituados aos comandos dos magistrados, nesta
conjuntura, não são obedientes às suas ordens. {[}26{]} E sempre terá,
nos tempos de penúria, poucos em quem possa confiar, porque um príncipe
assim não pode fundar"-se sobre aquilo que vê nos tempos calmos, quando
os cidadãos têm necessidade do governo, porque quando a morte é
distante, todos acorrem, todos prometem e todos desejam morrer por ele,
mas nos tempos adversos, quando o governo necessita dos cidadãos, nesta
hora se encontram poucos. {[}27{]} E, tanto mais é esta experiência
perigosa quanto não se pode tê"-la senão uma vez. Porém, um príncipe
sábio deve pensar em um modo pelo qual os seus cidadãos, sempre e em
todas as ocasiões, precisem do governo e dele, e assim sempre lhe serão
fiéis.

\quebra\section{\emph{QUOMODO OMNIUM PRINCIPATUUM VIRES~PERPENDI DEBEANT}\break
{[}De que modo se devem considerar as forças de todos os principados{]}}

{[}1{]} Convém fazer, ao examinar as qualidades destes
principados\footnote{Martelli sustenta, com alguma razão, que esse
  capítulo é uma continuação da argumentação do capítulo \versal{IX}, sendo que
  naquele Maquiavel tratou das coisas internas do principado civil e
  neste ele tratará das coisas externas. Cf. Martelli, M. \emph{Il
  Príncipe}. Roma: Salerno, 2006, p. 171, n. 2.}, uma outra
consideração: isto é, se um príncipe se vê em uma tal condição na qual
possa, se necessitar, regê"-la\footnote{O vocábulo \emph{reggere} aqui
  demonstra como Maquiavel concebe a condição do príncipe como alguém
  que ``rege'' e não alguém que ``reina'', donde se comprova mais uma
  vez que não temos presente ainda a noção do príncipe soberano próprio
  do pensamento político da Modernidade. Trata"-se, antes, da clássica
  imagem do maestro que rege o coro, alguém que conduz e guia outros,
  mas não se impõe pelo seu poder. Sobre essa distinção no pensamento
  político anterior a Maquiavel, cf. Senellart, M. \emph{As artes de
  governar}. São Paulo: ed. 34, 2006.} por si mesmo, ou, se de fato, tem
sempre necessidade de ser defendido por outros. {[}2{]} E, para
esclarecer melhor esta parte, digo como eu julgo aqueles que
podem\footnote{Como nos lembra Inglese, note"-se o paralelismo deste
  \emph{``coloro potersi regere\ldots{} che possono''} com aquilo que se
  afirma na linha seguinte, ``\emph{coloro avere\ldots{} che non possono''},
  ou seja, Maquiavel está tratando inicialmente ``daqueles que podem''
  e, em seguida, ``daqueles que não podem''. Cf. Inglese, G. \emph{Il
  Príncipe}. Torino: Einaudi, 1995, p. 69, cap. \versal{X}, n. 2.} reger por si
mesmos, ou por abundância de homens ou de dinheiro, reunir um exército
suficiente para uma batalha campal contra qualquer um que venha
assaltá"-los. {[}3{]} E assim julgo que têm sempre necessidade de outros
aqueles que não podem defrontar"-se com o inimigo em campo aberto, mas
necessitam refugiar"-se dentro dos muros e protegê"-los. {[}4{]} Acerca do
primeiro caso, já se discutiu\footnote{No capítulo \versal{IX}.} e, em seguida,
diremos algo mais a esse respeito. {[}5{]} Acerca do segundo caso, não
se pode dizer outra coisa senão exortar tais príncipes a fortificar e
municiar a própria cidade, e não levar em conta os demais territórios.
{[}6{]} E qualquer um que tiver bem fortificada a sua cidade e, acerca
dos outros governos, tenha"-se comportado com os súditos como acima se
disse e adiante se dirá, será sempre atacado com grande respeito, porque
os homens são sempre inimigos das empresas em que vêem dificuldade: e
tampouco podem ver facilidade em assaltar alguém que tenha a sua cidade
fortificada e não seja odiado pelo povo.

{[}7{]} As cidades da Alemanha\footnote{Maquiavel esteve em missão
  diplomática no sul da Alemanha e na Suiça entre 1507 e 1508, momento
  esse em que também escreveu alguns opúsculos com análises das
  condições que encontrou nesses territórios. Nesta parte do capítulo
  encontramos um resumo dessas análises sobre a Alemanha, cuja exposição
  mais ampla se encontra em \emph{Discorso sopra le cose della Magna e
  sopra l'Imperatore}, \emph{Ritratto delle cose della Magna} e
  \emph{Rapporto di cose della Magna}. Cf. Machiavelli, N. \emph{L'Arte
  della guerra/ Scritti politici minori}. Edizione Nazionale delle Opere
  -- \versal{I}/3. Roma: Salerno, 2001.} são libérrimas, têm poucos territórios e
obedecem ao imperador quando desejam, e não temem nem este nem aquele
poderoso que habita no seu entorno. {[}8{]} Porque elas são de tal modo
fortificadas que qualquer um pensa que sua conquista deve ser tediosa e
difícil, pois todas têm fosso e muros convenientes; têm artilharia
suficiente; têm sempre nos celeiros públicos bebidas, mantimentos e
combustíveis para um ano; {[}9{]} e, além disto, para poder manter a
plebe alimentada e sem prejuízo para o poder público, têm sempre na
cidade trabalhos para dar"-lhe por um ano naquelas atividades que são o
nervo e a vida daquela cidade e naquelas atividades com as quais a plebe
se sustenta; têm, ainda, em grande conta os exercícios militares, e
nisto tomam as medidas necessárias para mantê"-los.

{[}10{]} Um príncipe, portanto, que tenha uma cidade assim ordenada e
não se faz odiado, não pode ser tomado de assalto, e, ainda que houvesse
quem o tomasse de assalto, partiria com vergonha, porque as coisas do
mundo são tão variadas que é quase impossível que alguém pudesse ficar
com os exércitos ociosos por um ano a sitiá"-lo. {[}11{]} E quem a isto
replicasse que o povo tem suas posses fora dos muros, e que ao vê"-las
arder, não terá paciência, e o longo assédio e o amor a si próprio o
fará esquecer o amor ao príncipe, respondo que um príncipe prudente e
animoso superará sempre todas aquelas dificuldades, dando aos súditos
ora esperança de que o mal não será longo, ora incutindo o temor da
crueldade do inimigo, ora precavendo"-se com astúcia daqueles que lhe
parecem muito ousados. {[}12{]} Além disto, é razoável pensar que o
inimigo deva queimar e arruinar o país em sua investida e nos momentos
em que os ânimos dos homens estão ainda quentes e voluntariosos para a
defesa. Porém, por isso deve o príncipe recear menos ainda, porque
depois de alguns dias, quando as forças estão arrefecidas, os danos já
foram feitos, os males já aceitos, e não há mais remédio. {[}13{]} E
então tanto mais se unirão a seu príncipe, parecendo"-lhes que este tem
obrigação para com eles, tendo sido as suas casas queimadas e arruinadas
as suas posses para a defesa dele. E é da natureza dos homens obrigar"-se
tanto pelos benefícios que se fazem, como por aqueles que se recebem.
{[}14{]} Donde, se se considera tudo bem, não é difícil a um príncipe
prudente manter, antes e depois, firmes os ânimos dos seus cidadãos na
defesa da cidade, enquanto não lhes faltar nem o necessário para viver,
nem o necessário para se defender.

\quebra\section{\emph{DE PRINCIPATIBUS ECCLESIASTICIS}\break
{[}Dos principados eclesiásticos{]}}

{[}1{]} Resta somente discorrer, agora, sobre os principados
eclesiásticos, acerca dos quais todas as dificuldades estão antes de se
possuí"-los, porque se conquistam ou pela \emph{virtù} ou pela fortuna, e
sem uma e outra se conservam, pois são apoiados pelas antigas ordens da
religião, as quais têm sido tão poderosas e de uma tal qualidade que
conservam os seus príncipes no poder, não importando o modo como
procedam e vivam. {[}2{]} Somente eles têm estados e não os defendem;
têm súditos e não os governam. {[}3{]} Os estados, por serem indefesos,
não lhes são retirados, e os súditos, por não serem governados, não se
preocupam com isso, nem pensam e nem podem rebelar"-se contra eles.
{[}4{]} Logo, apenas estes principados são seguros e felizes, mas, sendo
eles regidos por razões superiores, as quais a mente humana não alcança,
deixarei de falar deles, porque sendo engrandecidos e mantidos por Deus,
seria ofício de um homem presunçoso e temerário discorrer sobre eles.
{[}5{]} Todavia\footnote{Neste momento da exposição evidencia"-se uma das
  marcas do estilo retórico de Maquiavel, que indica de início que os
  principados eclesiásticos são diferentes dos demais principados,
  contudo, apresenta a seguir a dinâmica política inerente a estes
  principados, igualando"-os aos outros. No limite, como declara o
  próprio texto, na esfera temporal e na dinâmica das lutas políticas,
  os principados eclesiásticos são iguais aos demais principados.}, se
alguém me perguntasse por onde a Igreja chegou a tamanha grandeza na
esfera temporal -- uma vez que, antes de Alexandre\footnote{O papa
  Alexandre \versal{VI}.}, os poderosos italianos, e não somente aqueles que se
chamavam poderosos, mas todos os barões e senhores, ainda que pouco
poderosos no plano temporal, a prezavam pouco, e agora um rei como o de
França a teme, e ela pôde expulsá"-lo da Itália e arruinar os venezianos
--, coisa que, ainda que seja conhecida, não me parece supérfluo trazer
parte dela de novo à memória.

{[}6{]} Antes que Carlos, rei de França, invadisse a Itália\footnote{Em
  1494.}, estava esta província sob o império do papa, dos venezianos,
do rei de Nápoles, do duque de Milão e dos florentinos. {[}7{]} Estes
poderosos tinham duas preocupações principais: uma, que um forasteiro
não entrasse na Itália com suas armas; a outra, que nenhum dentre eles
ocupasse mais poder do que os outros. {[}8{]} Aqueles que suscitavam
mais preocupação eram o papa e os venezianos. Para deter os venezianos,
era necessária a união de todos os outros, como aconteceu na defesa de
Ferrara; para limitar o poder do papa, serviram"-se dos barões de Roma,
os quais, estando divididos em duas facções -- os Orsini e os
Colonna\footnote{No início do século \versal{XVI} as famílias Orsini e Colonna
  exerciam grande influência sobre os rumos políticos do papado e de
  Roma, de modo que era muito difícil governar a cidade sem um acordo
  político com alguma das partes.} --, sempre havia motivo de discórdia
entre eles, e, estando com as armas em mãos e com os olhos sobre o
pontífice, conservaram o pontificado débil e enfermo. {[}9{]} E ainda
que surgisse alguma vez um papa corajoso, como foi Sixto\footnote{Sixto
  \versal{IV}, nascido Francesco Maria Della Rovere, foi papa de 1471 a 1484.}, a
fortuna ou o conhecimento, todavia, não puderam nunca livrá"-lo destes
incômodos. {[}10{]} E a brevidade da vida deles era a razão disso,
porque em dez anos que, em média, vivia um papa, mal podia diminuir o
poder de uma das facções. Caso, por exemplo, um deles houvesse quase
eliminado os Colonna, surgia um outro inimigo dos Orsini, que fazia
ressurgir aqueles e os Orsini não tinham tempo de extingui"-los. {[}11{]}
Isto fazia com que as forças temporais do papa fossem pouco respeitadas
na Itália.

{[}12{]} Surge depois Alexandre \versal{VI}, o qual, de todos os pontífices que
já existiram, mostrou quanto um papa, com o dinheiro e com a força,
poderia prevalecer; e fez, usando o duque Valentino como seu instrumento
e pela ocasião da invasão dos franceses, todas aquelas coisas sobre as
quais eu discorri acima\footnote{No cap. \versal{VII}.} ao tratar das ações do
duque. {[}13{]} E, embora a sua intenção não fosse engrandecer a Igreja,
mas o duque, contudo, o que ele fez trouxe grandeza à Igreja, a qual,
depois de sua morte, desaparecido o duque, foi herdeira de seus
esforços.

{[}14{]} Veio depois o papa Júlio, que encontrou a Igreja grande,
dominando toda a Romanha, eliminados os barões de Roma e, pelas
investidas de Alexandre, anuladas aquelas facções. Ele encontrou o
caminho ainda mais aberto para acumular dinheiro, jamais usado antes de
Alexandre. {[}15{]} Coisas que Júlio não apenas continuou, mas aumentou,
pois pensou em conquistar Bolonha, eliminar os venezianos e expulsar os
franceses da Itália: em todas estas empresas obteve êxito, e com tanto
mais honra para si, quanto fez todas essas coisas para o engrandecimento
da Igreja e não para o de algum cidadão. {[}16{]} Conservou, ainda, os
partidos dos Orsini e Colonna nos limites que os encontrou. {[}17{]} E,
embora entre eles algum chefe fosse propenso a suscitar discórdias, duas
coisas, todavia, detiveram"-nos: uma, a grandeza da Igreja, que lhes
amedrontava; outra, não ter eles cardeais, os quais são as origens dos
tumultos entre eles: e nunca ficarão tranquilos estes partidos enquanto
tiverem cardeais, porque estes alimentam, em Roma e fora dela, os partidos e os barões para
defendê"-los; e, assim, da ambição dos prelados nascem as discórdias e os
tumultos entre os barões.

{[}18{]} Sua Santidade, o papa Leão\footnote{Leão \versal{X}, nascido Giovanni di
  Medice, foi eleito papa em 21 de fevereiro de 1513.}, encontrou,
portanto, este\footnote{Este período final é um excelente indicativo
  temporal do momento da escrita de \emph{O Príncipe}, ou seja, a obra
  foi escrita durante o papado de Leão \versal{X}, provavelmente no início do
  pontificado e não foi revisada posteriormente, pois nesse caso
  Maquiavel corrigiria essa informação de época. Tudo leva a crer que,
  ao menos essa primeira parte que trata dos principados, foi escrita no
  ano de 1513.} pontificado poderosíssimo, do qual se espera, se aqueles
o fizeram grande com as armas, que este, com a sua bondade e as suas
outras infinitas \emph{virtù}, torne"-o grandíssimo e venerável.

\quebra\section{\emph{QUOT SUNT GENERA MILITIAE ET DE~MERCENNARIIS~MILITIBUS}\break
{[}De quantos são os gêneros de milícias e de soldados mercenários{]}}

{[}1{]} Tendo discorrido detalhadamente sobre todas as qualidades
daqueles principados que no princípio nos propuzemos a pensar,
considerando em alguns pontos as razões do bem e mal"-estar deles, e
expostos os modos pelos quais muitos têm procurado conquistá"-los e
conservá"-los, resta"-me agora discorrer, de modo geral\footnote{Maquiavel
  já havia escrito, durante o seu período de trabalho na Chancelaria de
  Florença, alguns opúsculos sobre questões militares. Posteriormente
  ele escreverá uma obra somente sobre esse tema, \emph{A arte da
  guerra} (entre 1516 e 1520), o que comprova os vários testemunhos que
  indicam que Maquiavel era reputado como um grande conhecedor de
  questões militares.}, sobre os ataques e as defesas que podem ocorrer
em cada um dos principados citados anteriormente\footnote{O período
  inicial revela a conclusão daquilo que foi proposto no capítulo \versal{I}, no
  caso, a exposição sobre os principados, que ocupou a primeira parte do
  livro. Ciente disso, Maquiavel apresenta em seguida a justificativa
  que articula os três capítulos seguintes (\versal{XII}, \versal{XIII} e \versal{XIV}) com o que
  havia sido exposto, eliminando com isso uma quebra na argumentação.}.

{[}2{]} Dissemos acima\footnote{Cap. \versal{VII}, 4.} quão necessário é a um
príncipe ter os seus fundamentos bons, pois, de outro modo, é forçoso
que se arruíne. {[}3{]} Os principais fundamentos comuns a todos os
estados, tantos os novos como os velhos e os mistos, são as boas leis e
as boas armas: porque não podem ser boas as leis onde não há boas armas,
e onde há boas armas convém que haja boas leis. Deixarei de lado o
raciocínio relativo às leis e falarei das armas\footnote{A noção de arma
  remete à força militar de uma cidade. A importância do fator militar
  na política é um \emph{topos} recorrente no pensamento político de
  Maquiavel, presente em outros escritos políticos, seja nas grandes
  obras, como os \emph{Discursos sobre a primeira década de Tito Lívio},
  seja em opúsculos menores, e principalmente na \emph{Arte da Guerra},
  que, como dito, é dedicado inteiramente ao tema.}.

{[}4{]} Digo, portanto, que as armas com as quais um príncipe defende o
seu estado ou são próprias ou são mercenárias, ou auxiliares ou mistas.
{[}5{]} As mercenárias e auxiliares são inúteis e perigosas. Se alguém
tem o seu estado fundado sobre as armas mercenárias, nunca estará nem
firme nem seguro, porque elas são desunidas, ambiciosas, sem disciplina,
infiéis, valorosas entre os amigos, vis entre os inimigos: não temerosas
a Deus, não confiáveis para com os homens; com elas se adia a ruína
enquanto se adia o ataque; na paz se é espoliado por elas, na guerra
pelos inimigos. {[}6{]} A razão disto é que elas não têm outro amor nem
outra razão que as conserve em campo, senão um pouco de soldo, o qual
não é suficiente para fazer com que queiram morrer por você. {[}7{]}
Querem bem ser seus soldados enquanto você não for à guerra, mas, quando
a guerra vem, ou fogem ou se vão. {[}8{]} E disto é fácil de se
persuadir, porque a atual ruína da Itália não é causada por outra coisa
senão por ter"-se, pelo espaço de muitos anos, apoiado inteiramente sobre
as armas mercenárias. {[}9{]} Tais armas já fizeram algumas conquistas
sob o comando de alguns\footnote{Ou seja, ``pela mão de alguém''.} e
pareciam valorosas quando combatiam entre si, mas, quando veio o
forasteiro\footnote{Carlos \versal{V}, rei da França.}, elas mostraram aquilo que
eram. Por isso para Carlos, rei de França\footnote{Maquiavel já havia
  feito referência a este fato no cap. \versal{VII}, 18.}, foi lícito riscar a
Itália com giz\footnote{Maquiavel se vale aqui de uma metáfora para
  mostrar a força e o poder do rei francês Carlos \versal{V}, que numa investida
  rápida conquistou toda a península itálica, como se passasse um giz
  sobre uma lousa. Essa expressão é atribuída a Alexandre \versal{VI} pelo
  historiador francês Philippe de Commynes, nas suas \emph{Memóires}
  (\versal{VII}, 14): ``\emph{les Françoys y sont alléz avecques des esperons de
  boys et de la craye en la main des fourriers pour marcher leurs logis,
  sans aultre peyne}''. {[}``Os franceses partiram com esporas de
  madeiras e com giz na mão dos entendentes para marcar as suas
  habitações, sem qualquer outro castigo''{]}. Com o giz eram marcados
  os edifícios das várias cidades da Itália destinadas ao alojamento dos
  franceses. Este teria sido o único trabalho dos franceses na
  conquistada Itália. Confira, Borsellino, N. \emph{Niccolò
  Machiavelli}, in \emph{Letteratura Italiana}. Bari: Laterza, 1973.
  Vol. 4. t.1, pp. 35-180.}; e quem dizia\footnote{Alusão ao frei
  Jerônimo Savonarola que pregou sobre a ruína de Florença no ano de
  1494 como sendo causada pela fraqueza dos próprios florentinos.} que a
causa disto eram os nossos erros dizia a verdade, embora não se tratasse
daqueles que acreditavam ser, mas destes que eu narrei; e porque eram
erros dos príncipes, eles também sofreram as suas penas.

{[}10{]} Vou demonstrar melhor a infelicidade destas armas. Os capitães
mercenários ou são homens excelentes ou não; se o são, não se pode
confiar neles, porque ou sempre aspirarão à própria grandeza ou vão
oprimi"-lo, você que é seu patrão, ou oprimirão outros para além de sua
intenção. Porém, se o capitão não é virtuoso, normalmente o arruína.
{[}11{]} E se alguém disser que qualquer um fará isso, tendo as armas em
mão, seja ele mercenário ou não, replicarei que as armas devem ser
manejadas ou por um príncipe ou por uma república: o príncipe deve ir
pessoalmente e fazer ele mesmo o ofício de capitão; a república tem de
mandar um cidadão seu e quando manda alguém que não seja um homem
valente, deve trocá"-lo; quando for o caso, detê"-lo com as leis, para que
não ultrapasse o prescrito. {[}12{]} E por experiência se vê apenas os
príncipes e as repúblicas armadas fazerem grandíssimos progressos, e as
armas mercenárias não causarem senão danos; e com muito mais dificuldade
se submete à obediência de um só cidadão uma república armada de armas
próprias, do que uma república armada de armas externas\footnote{Digna
  de nota a remissão, neste trecho, às repúblicas e à necessidade que
  tenham forças militares próprias e não mercenárias. O que evidencia a
  necessidade da força militar para a constituição de qualquer poder
  político que se pretenda autônomo.}.

{[}13{]} Estiveram Roma e Esparta por muitos séculos armadas e livres.
Os suíços são armadíssimos e libérrimos. {[}14{]} Das armas mercenárias
antigas são exemplo os cartagineses, que foram oprimidos por seus
soldados mercenários, terminada a primeira guerra com os
romanos\footnote{Primeira Guerra Púnica (241 a 237 a.C.).}, embora os
cartagineses tivessem por chefe seus próprios cidadãos. {[}15{]} Felipe
da Macedônia\footnote{Rei da Macedônia de 359 a 336 a.C., pai de
  Alexandre, o grande.} foi feito pelos tebanos capitão de seus
exércitos, depois da morte de Epaminondas, e tirou deles, depois da
vitória, a liberdade.

{[}16{]} Os milaneses, morto o duque Felipe, assoldadaram\footnote{Apesar
  de ser pouco utilizado, o verbo \emph{assoldadar} confere uma
  significação precisa ao texto. Assoldadar significa contratar alguém
  por meio de \emph{soldo}, cujo sujeito que recebe é o \emph{soldado.}
  A família terminológica aqui sugerida está, justamente, no centro da
  argumentação maquiaveliana, já que, o que se discute é o grau de
  fidelidade ou infidelidade das armas mercenárias, ou o seu grau de
  adesão, de solda, de união ao governo pagante.} Francisco Sforza
contra os venezianos\footnote{Batalha de Caravaggio de setembro de 1448.},
o qual, superados os inimigos em Caravaggio, se associou a eles para
oprimir os milaneses, seus patrões. {[}17{]} Sforza, seu pai\footnote{Muzio
  Attendolo Sforza (1369-1424), grande comandante militar italiano que
  esteve a serviço da rainha Giovanna de Napoli ou Giovanna \versal{II},
  rebelando"-se em 1420 contra ela e se aliando a Luiz \versal{III} D'Anjou. Esse
  episódio também é narrado na \emph{Arte da Guerra,} livro \versal{I}, e nas
  \emph{História de Florença,} livro \versal{I}, cap. 38.}, sendo soldado da
rainha Giovanna de Nápoles, deixou"-a, subitamente, desarmada; e ela,
para não perder o reino, foi constrangida a se atirar no colo do rei de
Aragão. {[}18{]} E se venezianos e florentinos, entretanto, acresceram
os seus impérios com estas armas, e se os seus capitães, porém, não
foram feitos príncipes, mas os defenderam, respondo que os florentinos,
neste caso, foram favorecidos pela sorte\footnote{Um dos poucos casos em
  que Maquiavel usa o termo \emph{sorte} e não \emph{fortuna,} como
  seria conveniente.}: porque, dos capitães virtuosos que podiam temer,
alguns não venceram, alguns enfrentaram oposição, outros voltaram as
suas ambições para outro lugar. {[}19{]} Aquele que não venceu foi
Giovanni Aucut\footnote{John Hawkwood, comandante inglês que de 1390 a
  1392 comandou as tropas florentinas contra as tropas milanesas de Gian
  Galeazzo Visconti.}, de quem, não vencendo, não se podia conhecer a
lealdade, embora qualquer um concorde que, se ele vencesse, estavam os
florentinos à sua mercê. {[}20{]} Sforza teve sempre os
Brancceshi\footnote{Como era conhecida a Companhia militar de Andréa
  Fortebracci, dito Braccio da Montone (1368- 1424), que esteve
  diretamente envolvido nos acontecimentos relativos a Nápoles.} por
antagonistas, que vigiavam um ao outro. {[}21{]} Francisco voltou a sua
ambição para a Lombardia; Braccio\footnote{No caso Braccio da Montone,
  que assumiu o lugar de Muzio Sforza e esteve a serviço de Giovanna \versal{II}
  e Alfonso de Aragão.} contra a Igreja e o reino de Nápoles.


{[}22{]}
Mas vejamos o que aconteceu há pouco tempo. Fizeram os florentinos Paulo
Vitelli\footnote{Confira cap. \versal{VIII}, 13.} seu capitão, homem
prudentíssimo e que, a partir da fortuna privada, havia conquistado
grande reputação. Se ele tomasse Pisa, ninguém negaria que conviria aos
florentinos tê"-lo consigo, porque, se ele se convertesse em soldado dos
inimigos dos florentinos, não teriam remédio; se os florentinos o
mantivessem ao seu lado, teriam de obedecê"-lo.

{[}23{]} Os venezianos, se se considera os seus progressos, ver"-se"-á que
agiram seguramente e gloriosamente enquanto fizeram a guerra com suas
próprias armas -- o que ocorreu antes de empreenderem suas conquistas em
terra --, quando, com os gentis homens e com a plebe armada, agiram de
maneira muito virtuosa. Contudo, quando começaram a combater em terra,
deixaram esta \emph{virtù} e seguiram os costumes das guerras da
Itália\footnote{Entre os séculos \versal{IX} e \versal{XIII} os venezianos fizeram uma
  grande expansão marítima, conquistando vários territórios ao longo do
  mar Adriático e Egeu, tornando"-se uma grande potência econômica e
  militar. Todavia, ao tentar fazer uma expansão sobre os territórios do
  norte da Itália, principalmente sobre a região do Veneto, essa força
  conquistadora não se verificou, pois, conforme denunciado por
  Maquiavel, não se valeram de seus próprios exércitos, mas de milícias
  mercenárias contratadas.}. {[}24{]} E no princípio da sua expansão por
terra, por não ter nela muitos poderosos e por terem grande reputação,
não tinham muito do que temer os seus capitães. {[}25{]} Mas, quando
eles aumentaram seu poder em terra, o que ocorreu sob o governo de
Carminhola\footnote{Francesco Bussone, conde de Castelnuovo Scrivia
  (1340 a 1432), comandante militar que no século \versal{XV} obteve várias
  conquistas em favor dos venezianos.}, tiveram uma lição deste erro:
porque, vendo"-o poderosíssimo, após ter batido, sob seu comando, o duque
de Milão, e vendo, por outro lado, como ele estava arrefecendo na
guerra, julgaram não poder mais vencer com ele, porque não desejavam;
nem podiam liberá"-lo, para não perderem novamente aquilo que haviam
conquistado; foi"-lhes necessário, para sua segurança, matá"-lo. {[}26{]}
Tiveram, depois, como seus capitães Bartolomeu de Bergamo\footnote{Bartolomeu
  Colleoni (1400-1475), derrotado na Batalha de Caravaggio.}, Ruberto de
San Severino\footnote{Roberto de San Severino (1418-1487), comandante
  veneziano na guerra contra Ferrara (1484).} o Conde de
Pitiglino\footnote{Nicollò Orsini (1442-1510), comandante na batalha de
  Vailate (maio de 1509), na qual os venezianos foram derrotados.} e
semelhantes, com os quais temiam a derrota e não os seus ganhos, como
ocorreu depois em Vailá\footnote{Ou Vailate.}, onde, em uma jornada,
perderam aquilo que em oitocentos\footnote{Aqui Maquiavel faz alusão a
  um período de tempo que não é exato, mas que compõe aquilo que ficou
  conhecido como o ``mito de Veneza'', principalmente em sua semelhança
  com Esparta. Afirmava"-se no Renascimento italiano que a república
  espartana havia durado 800 anos, sendo a mais longeva da história. As
  mesmas qualidades de Esparta eram atribuídas a Veneza, e, neste caso,
  Maquiavel comete o equívoco de atribuir uma temporalidade que não é de
  Veneza, mas da mitologia sobre Esparta, o que evidencia o recurso
  retórico de falsa exaltação, pois pretende"-se ressaltar o seu
  fracasso.} anos haviam conquistado com tanto esforço: porque destas
armas nascem somente as conquistas lentas, tardias e débeis e as
derrotas rapidíssimas e extraordinárias.

{[}27{]} E porque eu vim parar com estes exemplos na Itália, que foi
governada por muitos anos pelas armas mercenárias, gostaria de discorrer
sobre elas ainda um pouco mais, para que vendo a origem e o progresso
delas, se possa melhor corrigi"-las. {[}28{]} Você tem de entender,
portanto, como o Império\footnote{Isto é, o Sacro"-Império Romano
  Germânico.} rapidamente começou, nestes últimos tempos, a ser expulso
da Itália e como o papa ganhou mais poder temporal, e a Itália se
dividiu em vários estados porque muitas das cidades grandes pegaram em
armas contra os seus nobres -- os quais, primeiramente favorecidos pelo
imperador, as oprimira"-nas --, e a Igreja as favoreceu para dar a si
poder no plano temporal; em muitas outras cidades seus cidadãos se
tornaram príncipes\footnote{Conforme se disse, no cap. \versal{IX}, dos cidadãos
  comuns que se tornaram príncipes.}. {[}29{]} Assim, tendo a Itália
quase caído nas mãos da Igreja e de algumas repúblicas, e sendo aqueles
padres e aqueles outros cidadãos não habituados ao uso das armas,
começaram a pagar forasteiros. {[}30{]} O primeiro que deu força a este
tipo de milícia foi Alberigo di Conio\footnote{Alberico da Barbiano,
  conde de Cunio, foi o primeiro a constituir uma companhia militar.},
da Romanha: da escola deste descende, entre outros, Braccio e Sforza,
que nos seus tempos foram árbitros da Itália. {[}31{]} Depois destes
vieram todos os outros, que até os nossos tempos têm comandado estas
armas, e o fim da sua \emph{virtù} foi a Itália ter sido devastada por
Carlos, saqueada por Luiz, subjugada por Fernando e vituperada pelos
suíços.

{[}32{]} As táticas que eles seguiram serviram, primeiramente, para dar
reputação a si próprios e tirar a reputação da infantaria; fizeram isto
porque, não tendo um status e vivendo do seu trabalho, poucos infantes
não lhes davam reputação e não poderiam alimentar a muitos; porém, por
isso se limitaram à cavalaria, com a qual, com um número suportável,
estavam bem armados e eram honrados; e as coisas eram reduzidas a tal
ponto que, em um exército de vinte mil soldados, não se encontravam dois
mil infantes\footnote{O destaque para a infantaria nos exércitos faz
  parte de um pressuposto central do pensamento político e militar de
  Maquiavel que entende que o soldado, mais especificamente o
  cidadão"-soldado, é o fundamento primeiro da força militar. Esses
  comandantes militares, ao optarem por exércitos predominantemente
  compostos por cavalarianos, enfraqueciam esse elemento político
  central. Há que se destacar, ainda, que apesar de nesse momento já se
  conhecer e usar a pólvora, Maquiavel não faz referência ou atribui
  qualquer importância à artilharia em seus escritos. Esse elemento
  revolucionário das técnicas militares do século \versal{XVI} não é destacado
  por ele, que pensa as forças militares no interior do quadro da
  dinâmica política.}. {[}33{]} Além disso, haviam usado todos os meios
para isentarem a si e aos seus soldados do cansaço e do medo, não se
matavam nos combates, os prisioneiros eram libertados sem resgate, não
atacavam à noite as cidades amuralhadas\footnote{\emph{Terre} são
  cidades contornadas por uma muralha, como se vê, ainda hoje, em várias
  cidades da Itália, por exemplo Lucca, na Toscana.}, os sitiados não
saiam das muralhas para atacar os assediantes; não faziam, no entorno
dos campos, nem as paliçadas nem as fossas; não guerreavam durante
inverno. {[}34{]} E todas estas coisas eram permitidas nas suas táticas
militares e forjadas por eles para fugir, como foi dito, da fadiga e dos
perigos: tanto que foram eles que tornaram a Itália escrava e vituperada.

\quebra\section{\emph{DE MILITIBUS AUXILIARIIS, MIXTIS~ET~PROPRIIS}\break
{[}Das milícias auxiliares, mistas~e~próprias{]}}

{[}1{]} As armas auxiliares, que são outras armas inúteis, são as que se
têm quando se chama um poderoso que, com as suas armas, vem defende"-lo,
como fez em tempos recentes o papa Júlio: o qual, tendo visto na empresa
de Ferrara a triste prova das suas armas mercenárias, voltou"-se às
auxiliares e fez um acordo com Fernando\footnote{Fernando, fazendo parte
  da ``Santa Liga'' (Espanha, Veneza e Igreja), ajudou o papa a
  conquistar Ferrara em 11 outubro de 1511.}, rei de Espanha, para que
ele, com a sua gente e os seus exércitos\footnote{Convém notar que
  Maquiavel dispõe de um termo próprio para os exércitos
  (\emph{eserciti}), mas não o usacom frequência, preferindo o termo
  \emph{arma}, o que revela uma compreensão mais restrita do primeiro e
  mais ampla do segundo, uma vez que este inclui exércitos, milícias e
  outros tipos de forças armadas.}, tivesse de ajudá"-lo. {[}2{]} Estas
armas podem ser úteis e boas para si mesmas, mas são, para quem as
chama, quase sempre danosas, porque, perdendo, você permanece derrotado,
vencendo, fica prisioneiro delas. {[}3{]} Embora as histórias antigas
estejam repletas destes exemplos, não desejo, todavia, afastar"-me do
exemplo recente do papa Júlio \versal{II}, cuja decisão não poderia ser mais
insensata, por querer Ferrara, ficou completamente nas mãos de um
forasteiro. {[}4{]} Mas a sua boa fortuna fez nascer um terceiro fator,
a fim de que não colhesse o fruto de sua má escolha, porque, sendo os
seus auxiliares derrotados em Ravena\footnote{Batalha de Ravena, 11 de
  abril de 1512, onde a ``Santa Liga'' é derrotada pelos exércitos
  franceses.} e aparecendo os suíços que expulsaram os vencedores,
contra toda expectativa sua e de outros, não permaneceu prisioneiro nem
dos inimigos, postos em fuga, nem dos seus auxiliares, tendo vencido com
armas de outros e não com as suas. {[}5{]} Os florentinos, estando
completamente desarmados, conduziram dez mil franceses a Pisa\footnote{Retomada
  de Pisa em maio de 1500.} para tomá"-la de assalto, decisão que
comportou mais perigos do que em qualquer outro momento de suas
dificuldades. {[}6{]} O imperador de Constantinopla\footnote{Giovanni \versal{VI}
  Cantacuzeno reinou de 1347 a 1355.}, para se opor aos seus vizinhos,
colocou na Grécia\footnote{Fazer nota sobre o que era a Grécia antes.}
dez mil turcos, os quais, terminada a guerra, não quiseram partir, o que
foi o princípio da servidão da Grécia\footnote{Grécia, aqui, entendida
  como a região da Península Ática, que compreende mais territórios que
  a Grécia do século \versal{XXI}.} aos infiéis.

{[}7{]} Aquele, portanto, que não deseja vencer, que se valha destas
armas, porque são muito mais perigosas que as mercenárias. {[}8{]}
Porque, com as armas auxiliares, a ruína é certa: são todas unidas,
voltadas à obediência dos outros; mas as mercenárias, uma vez que tenham
vencido, para prejudicarem"-no precisam de mais tempo e de melhores
ocasiões, porque não sendo um só corpo e sendo contratadas e pagas por
você, um terceiro que você eleve ao comando, não pode ganhar rapidamente
tanta autoridade que o prejudique. {[}9{]} Em suma, nas mercenárias, é
mais perigosa a indolência, nas auxiliares, a \emph{virtù}. {[}10{]} Um
príncipe sábio, portanto, sempre fugiu destas armas e voltou"-se às
próprias, e preferiu, antes, perder com as suas do que vencer com as
outras, julgando não verdadeira a vitória que se conquista com as armas
alheias.

{[}11{]} Não hesitarei nunca de dar como exemplo César Borgia e suas
ações. Este duque entrou na Romanha com as armas auxiliares, conduzindo
todas as tropas francesas, e com elas tomou Imola e Furli. Porém, não
lhe parecendo, pois, tais armas seguras, voltou"-se às mercenárias, e
pagou os Orsini e os Vitelli, cujas armas, depois, ao manobrá"-las,
julgou dúbias, infiéis e perigosas, eliminou"-as e voltou"-se para as
armas próprias. {[}12{]} E pode"-se facilmente ver que diferença há entre
uma e outra dessas armas, considerando a diferença de reputação do duque
quando apenas tinha os franceses, quando tinha os Orsini e os Vitelli e
quando ficou com os seus soldados e consigo mesmo: sempre se encontrará
seu poder engrandecido, e nem foi muito estimado, senão quando todos
viram que ele controlava completamente suas armas.

{[}13{]} Eu não quero me distanciar dos exemplos italianos e recentes,
como não deixarei para trás Hierão de Siracusa, sendo um dos acima
nomeados por mim\footnote{Confira cap. \versal{VI}, 26-28.}. {[}14{]} Este, como
eu disse, feito chefe dos exércitos dos siracusanos, percebeu
imediatamente que aquela milícia mercenária não era útil, por serem os
chefes militares como os nossos italianos. Parecendo"-lhe não poder nem
conservar tais armas e nem deixá"-las, ele fez com que ela fossem
destroçadas e depois fez guerra com as suas armas e não com as alheias.
{[}15{]} Quero ainda trazer à memória uma figura do Velho Testamento,
que vem a propósito. {[}16{]} Oferecendo"-se Davi a Saul para ir combater
com Golias, provocador filisteu\footnote{Cf. \versal{I} Samuel, 17.}, Saul, para
lhe dar coragem, o armou com suas próprias armas. Assim que Davi as
vestiu, recusou"-as, dizendo que com aquelas não se podia valer"-se bem de
suas próprias qualidades, e por isso desejava enfrentar o inimigo com a
sua funda e a sua faca. {[}17{]} Enfim, as armas dos outros ou lhe caem
por terra, ou lhe pesam ou lhe apertam.

{[}18{]} Carlos \versal{VII}\footnote{Reinou de 1422 a 1461.}, pai do rei Luiz
\versal{XI}, tendo com a sua fortuna e \emph{virtù} libertado a França dos
ingleses\footnote{Com a ``Guerra dos Cem Anos'', que terminou em 1452.},
reconheceu a necessidade de se armar de armas próprias e ordenou no seu
reino o recrutamento da cavalaria e da infantaria. {[}19{]} Depois o rei
Luiz\footnote{Luiz \versal{XI}, rei de 1461 a 1483.}, seu filho, extinguiu o
recrutamento da infantaria e contratou os soldados suíços, erro que,
seguido por outros, é, como de fato se vê agora, a razão dos perigos
daquele reino. {[}20{]} Porque, tendo dado reputação aos suíços, aviltou
todas as suas armas, tendo em vista que eliminou toda a infantaria e
tornou sua cavalaria dependente da \emph{virtù} de outros, porque, sendo
acostumadas a combaterem com os suíços, não lhe parecia poder vencer sem
eles. {[}21{]} É por isso que os franceses não bastam contra os suíços e
não se colocam à prova contra os outros sem os suíços. {[}22{]} Foram,
portanto, exércitos mistos os da França, em parte mercenário e em parte
próprio: estas armas, todas juntas, são muito melhores do que as
auxiliares sozinhas ou as mercenárias sozinhas, e muito inferior às
próprias. {[}23{]} E que basta o exemplo dado, porque o reino de França
seria insuperável se o recrutamento de Carlos fosse aumentado ou
preservado; mas a pouca prudência dos homens começa uma coisa que, por
parecer então boa, não se percebe o veneno que tem debaixo, assim como
eu disse acerca da febre tísica\footnote{Ou tuberculose, confira cap.
  \versal{III}, 27.}. {[}24{]} Portanto, aquele que em um principado não conhece
os males quando nascem, não é verdadeiramente sábio, e isso é atributo
de poucos. {[}25{]} E, caso se considerasse a principal causa da ruína
do Império Romano, encontrar"-se"-ia ter sido somente o começar a pagar os
mercenários godos\footnote{Trata"-se da incorporação de 40.000 visigodos
  ao exército romano realizada, uma primeira vez pelo imperador Valente
  em 376 d.C. e, uma segunda vez, pelo imperador Teodósio em 382 d.C.},
porque daquele princípio as forças do império romano começaram a se
debilitar, e toda aquela \emph{virtù} que se extraía dele dava"-se aos
godos.

{[}26{]} Concluo, portanto, que sem ter armas próprias, nenhum
principado está seguro. Aliás, fica completamente dependente da fortuna,
não tendo \emph{virtù} que com confiança o defenda na adversidade. E foi
sempre a opinião e a sentença dos homens sábios, \emph{quod nihil sit
tam infirmum aut instabile quam fama potentiae non sua vi
nixa}\footnote{``Não há nada de mais instável e frágil do que a fama de
  uma potência que não se apóia na própria força''. Tácito, Anais, \versal{XIII},
  19. Citado de memória por Maquiavel, mas com o mesmo sentido do texto
  original. O texto original é: ``Nihil rerum mortalium tam instabile ac
  fluxum est quam fama potentiae non sua vi nixae.''}. {[}27{]} As armas
próprias são aquelas compostas ou pelos súditos, ou pelos cidadãos, ou
pelos seus criados: todas as outras são ou mercenárias ou auxiliares; e
o modo para ordenar as armas próprias será fácil de se encontrar se
examinar os ordenamentos dos quatro acima nomeados\footnote{Há nesta
  passagem uma discordância entre os comentadores sobre quais seriam
  esses quatro exemplos, pois duas hipóteses se apresentam e ambas
  excluem os romanos, que foram sempre elogiados por Maquiavel por
  possuírem um exército próprio. A primeira hipótese nos remete ao
  capítulo \versal{VI}, quando ele fala de Moisés, Rômulo, Ciro e Teseu. A
  segunda hipótese, recuperada a partir dos exemplos citados neste
  capítulo, seriam Cesar Bórgia, Hierão, Carlos \versal{VII} e Davi. O problema
  dessa última hipótese é que Davi não é apresentado aqui como detentor
  de um exército próprio, mas apenas de armamento próprio. Martelli
  formula uma outra hipótese mais controversa, sugerindo que Maquiavel
  havia escrito uma outra obra, na qual cita explicitamente quatro
  personagens ou quatro povos. (Martelli, \emph{edição comentada}, p.
  209, nota 48). Seja como for, não é possível saber exatamente quem são
  esses quatro exemplos mencionados por Maquiavel.} por mim, e se verá
como Felipe, pai de Alexandre Magno, e muitas repúblicas e príncipes se
armaram e ordenaram, cujos ordenamentos eu me remeto em tudo.

\quebra\section{\emph{QUOD PRINCIPEM DECEAT CIRCA MILITIAM}\break
{[}O que compete a um príncipe no que~diz~respeito~às milícias{]}}

{[}1{]} Deve, pois, um príncipe não ter outro objetivo nem outro
pensamento, nem tomar coisa alguma como arte sua que não seja a guerra,
a organização e a disciplina desta, porque apenas ela é a arte que se
espera de quem comanda e tem tamanha \emph{virtù,} que não somente
conserva aqueles que nasceram príncipes, mas, muitas vezes, faz com que
os homens de fortuna privada alcancem aquele posto. {[}2{]} E, ao
contrário, se vê que, quando os príncipes pensaram mais nas delicadezas
do que nas armas, perderam o seu status, e a principal razão que faz
você perdê"-lo é negligenciar esta arte; e a razão que faz você
conquistá"-lo é ser professor nesta arte. {[}3{]} Francisco Sforza, por
estar armado, de pessoa comum tornou"-se duque de Milão; e os seus
filhos\footnote{Francisco Sforza teve três filhos: Galeazzo Maria,
  Ludovico Moro e Ascânio. O primeiro foi morto numa conjuração em 1476,
  o segundo perde o ducado em 1499-1500 e Ascânio tornou"-se cardeal e
  morreu em 1505. Apesar desses fatos, em 1513, quando Maquiavel está
  escrevendo \emph{O Príncipe}, o ducado de Milão está ainda nas mãos da
  família Sforza, com Ercole Massimiliano, filho de Ludovido Moro e,
  portanto, neto de Francisco Sforza. Ora, esses fatos parecem indicar
  um equívoco de Maquiavel. Contudo, desde 1924, Federico Chabod, e,
  depois, toda uma série de outros comentadores que se seguiram, são
  concordes em interpretar essa passagem como sendo uma referência a
  Ludovico Moro principalmente -- este sim que perdeu o governo da cidade
  de Milão por não ter armas próprias e nem saber guerrear --, apesar do
  plural empregado remeter aos outros filhos. Enfim, seja principalmente
  Ludovico, mas também Galeazzo e Ascânio, o fato é que os filhos de
  Francisco Sforza não conservaram o governo da cidade, apesar de seu
  neto retomá"-lo depois. De qualquer modo, Maquiavel não foi muito
  preciso em sua referência histórica.}, para fugirem dos incômodos das
armas, de duques tornaram"-se pessoas comuns. {[}4{]} Porque, entre
outros males que lhe acarretam, estar desarmado torná"-o desprezível, e
isso é uma daquelas infâmias das quais o príncipe deve se guardar, como
adiante se dirá\footnote{Sobretudo no cap. \versal{XIX}.}. {[}5{]} Porque não há
proporção alguma entre um príncipe armado e um desarmado, e não é
razoável que quem está armado obedeça voluntariamente a quem está
desarmado, e que o desarmado esteja seguro entre serviçais armados,
pois, havendo em um o desprezo e noutro a suspeita, não é possível
agirem bem juntos. {[}6{]} Por isso que um príncipe que não entenda de
milícia, além das outras infelicidades, como foi dito, não pode ser
estimado pelos seus soldados nem fiar"-se neles.

{[}7{]} Portanto, nunca deve desviar o pensamento destes exercícios da
guerra, e na paz deve mais exercitá"-los que na guerra, o que pode fazer
de dois modos: um com o agir, outro com a mente. {[}8{]} E, quanto ao
agir, além de ter bem organizados e exercitados os seus, deve sempre ir
às caçadas, e, mediante estas, acostumar o corpo aos incômodos e,
paralelamente, apreender a natureza do lugar, conhecer como se erguem os
montes, como se abrem os vales, como se estendem as planícies, entender
a natureza dos rios e dos pântanos, e nisso tudo por grandíssima
atenção. {[}9{]} Tal conhecimento é útil de dois modos: primeiro,
aprende"-se a conhecer seu próprio território, e se pode melhor entender
a defesa dele; depois, mediante o conhecimento e a prática nesses
lugares, compreende"-se com facilidade todos os outros lugares que
novamente será necessário estudar, porque as colinas, os vales, as
planícies, os rios, os pântanos, que estão, por exemplo, na Toscana, têm
certa semelhança com os de outras províncias, de tal modo que, do
conhecimento do lugar de uma província, pode"-se facilmente chegar ao
conhecimento de outra. {[}10{]} E aquele príncipe que é desprovido desta
perícia é desprovido da primeira qualidade que um capitão deve possuir
porque esta lhe ensina a encontrar o inimigo, escolher o melhor lugar
para os alojamentos, conduzir os exércitos, preparar o plano de batalha,
assediar as cidades com vantagem para você.

{[}11{]} Filopomene\footnote{Filopemene de Megalópolis (252 a 184 a.C),
  estrategista da Liga Aqueia.}, príncipe dos Aqueus, entre as outras
glórias que lhe foram dadas pelos escritores\footnote{Tito Lívio, livro
  \versal{XXXV}, cap. 28.}, há aquela que nos tempos de paz só pensava nos modos
de fazer guerra; e quando estava em campanha, com frequência se detinha
com os amigos e raciocinava com eles: {[}12{]} ``Se os inimigos
estivessem sobre aquelas colinas, e nós nos encontrássemos aqui com os
nossos exércitos, quem de nós teria vantagem? Como se poderia, mantendo
as formações, ir encontrá"-las? Se nós quiséssemos bater em retirada,
como faríamos? Se eles batessem em retirada, como iríamos
segui"-los?''\footnote{Essa é uma citação da \emph{História de Roma}, de
  Tito Lívio, indicada acima.} {[}13{]} E, assim, propunha a eles todos
os casos que podem ocorrer a um exército, ouvia a opinião deles, dizia a
sua, corroborava"-as com os argumentos; tanto que, por estes contínuos
raciocínios, não podia nunca, guiando os exércitos, surgir algum
acidente para o qual ele não tivesse solução.

{[}14{]} Mas, quanto ao exercício da mente, deve o príncipe ler as
histórias\footnote{Aqui se evidencia dois requisitos que perfazem a
  metodologia proposta por Maquiavel desde o início da obra: experiência
  no agir e reflexão ou estudo da História. Esses são os dois
  fundamentos da reflexão política, que não se pode valer apenas da
  experiência e nem dos estudos somente. Esse pressuposto metodológico
  será reiterado no capítulo seguinte.} e nessas considerar as ações dos
homens excelentes, ver como se governaram nas guerras, examinar as
causas das suas vitórias e das suas derrotas, para poder fugir dessas e
imitar aquelas, e, sobretudo, fazer como fizeram antes alguns daqueles
homens excelentes, que tentaram imitar alguém que, antes deles, foi
louvado e glorificado, e cujos feitos e ações sempre mantiveram junto a
si: como se disse que Alexandre Magno imitava Aquiles; César, Alexandre;
Cipião, Ciro\footnote{Esses exemplos encontram"-se em: Plutarco
  (\emph{Vida de Alexandre}, 8), Suetônio (\emph{Julio César}, 7) e
  Cícero (\emph{Ad Quintum fratem}, \versal{I}, 23).}. {[}15{]} E alguém que leia
a vida de Ciro escrita por Xenofonte\footnote{A \emph{Ciropédia}.}
reconhece depois, na vida de Cipião\footnote{Trata"-se aqui de Publio
  Cornélio Cipião Africano (235-183 a.C.), general romano que venceu o
  cartaginês Aníbal, na Segunda Guerra Púnica, na batalha de Zama, em 19
  de outurbro 202 a.C.}, o quanto aquela imitação lhe foi gloriosa, e
quanto, na castidade, na afabilidade, na humanidade, na liberalidade,
Cipião se conformou àquelas coisas que Xenofonte escreveu de Ciro.

{[}16{]} Modos semelhantes deve observar um príncipe sábio e nunca ficar
ocioso nos tempos pacíficos, mas, com habilidade, reunir fundos para
poder se valer deles nas adversidades, de modo que, quando a fortuna
muda, encontrar"-lo"-á pronto para resistir a ela.

\quebra\section{\emph{DE HIS REBUS QUIBUS HOMINES ET PRAESERTIM PRINCIPES LAUDANTUR AUT~
VITUPERANTUR}\break
{[}Das coisas pelas quais os homens, e~especialmente~os~príncipes,~são
louvados ou vituperados{]}}

{[}1{]} Resta agora ver quais devem ser os modos e os atos de governo de
um príncipe para com os súditos ou para com os amigos. {[}2{]} E, porque
sei que muitos escreveram sobre isto\footnote{Esta é uma referência
  explícita de Maquiavel à tradição literária dos ``espelhos de
  príncipe'', que, em linhas gerais, concentrava"-se em apresentar regras
  e máximas de condutas para o príncipe conduzir o povo, principalmente
  no que diz respeito às virtudes que ele deveria possuir ou adquirir.
  Nota"-se, em seguida, que ele não se manterá nessa tradição, pois
  formulará uma análise diferente sobre as qualidades políticas do
  príncipe que não se encaixaram dentro do quadro tradicional da virtude
  cristã preconizada.}, temo, escrevendo eu também, ser considerado
presunçoso, sobretudo porque, ao debater esta matéria, afasto"-me do modo
de raciocinar dos outros. {[}3{]} Mas, sendo a minha intenção escrever
coisa útil a quem a entenda, pareceu"-me mais convincente ir direto à
verdade efetiva da coisa do que à imaginação dessa\footnote{A expressão
  \emph{verità effetuale della cosa} indica um dos fundamentos da
  reflexão política maquiaveliana, que não pretende se pautar pela
  imaginação ou idealização da política, mas se apóia diretamente na
  realidade da vida política.}. {[}4{]} E muitos imaginaram
repúblicas\footnote{Aqui uma referência direta à \emph{República}, de
  Platão.} e principados que nunca foram vistos, nem conhecidos de
verdade. {[}5{]} Porque há tanta diferença entre como se vive e como se
deveria viver, que quem deixa aquilo que se faz por aquilo que se
deveria fazer apreende mais rapidamente a sua ruína que a sua
preservação, porque um homem que deseja ser bom em todas as situações, é
inevitável que se destrua entre tantos que não são bons. {[}6{]} Assim,
é necessário a um príncipe que deseja conservar"-se no poder, apreender a
não ser bom, e sê"-lo e não sê"-lo conforme a necessidade.

{[}7{]} Deixando, portanto, para trás as coisas imaginadas sobre um
príncipe e discorrendo sobre aquelas que são verdadeiras, digo que todos
os homens, quando falam deles, e mais ainda os príncipes, por estarem em
posição mais elevada, são tachados de algumas destas qualidades que
causam ou a sua ruína ou o seu louvor. {[}8{]} E assim é que um é tido
por liberal, outro por miserável -- usando um termo toscano, porque
``avaro'', em nossa língua, é também aquele que por roubo deseja ter e
``miserável'' chamamos àquele que se abstém muito de usar o que é seu
--; um é considerado pródigo, outro rapace; um cruel, outro piedoso;
{[}9{]} um não confiável, outro fiel; um efeminado e pusilânime, outro
feroz e animoso; um humano, outro soberbo; um lascivo, outro casto; um
íntegro, outro astuto; um rijo, outro fácil; um sério, outro leviano; um
religioso, outro incrédulo e assim por diante. {[}10{]} Sei que todos
afirmaram que seria coisa louvabilíssima encontrar"-se em um príncipe,
de todas as sobreditas qualidades, aquelas que são consideradas boas.
{[}11{]} Mas, porque não se podem ter todas, nem observá"-las
inteiramente, por causa das condições humanas que não o consentem, é
necessário ser tão prudente que saiba evitar a infâmia\footnote{Neste
  capítulo Maquiavel anuncia um \emph{topus} central da sua exposição
  doravante, a saber: o modo como um príncipe deve lidar com a fama e a
  infâmia. Para isso, convém ter em mente que a fama e a infâmia são as
  percepções que os indivíduos têm de seu governante em função do efeito
  dos seus atos e não necessariamente dos atos em si praticados pelo
  príncipe. Essa dimensão dos efeitos da ação e da não ação é
  propriamente o ponto central da argumentação maquiaveliana, que é um
  dos primeiros a chamar a atenção para esse aspecto do agir político: o
  seu efeito sobre os outros. Não devemos nos esquecer que Maquiavel era
  um teatrólogo; na verdade, sua fama entre os florentinos era mais
  resultado do sucesso de suas peças de teatro do que propriamente dos
  seus escritos políticos. Sobre essa temática, cf. Adverse, Helton.
  \emph{Politica e Retórica em Maquivel}. Belo Horizonte: ed. \versal{UFMG},
  2009.} daqueles vícios que lhe tirariam o status, e se guardar, se lhe
é possível, daqueles que não lhe fariam perdê"-lo. Mas não podendo, com
menos escrúpulo, pode"-se deixar ir. {[}12{]} E ainda não se preocupe em
incorrer na infâmia daqueles vícios sem os quais dificilmente poderia
manter o status, porque, caso se considere tudo muito bem,
encontrar"-se"-á algo que parece \emph{virtù}, e que, seguindo"-a, seria a
sua ruína, e alguma outra que parece vício, e seguindo"-a consegue a
segurança e o seu bem"-estar.

\quebra\section{\emph{DE LIBERALITATE ET PARSIMONIA}\break
{[}Da liberalidade e da parcimônia{]}}

{[}1{]} Começando, então, pelas primeiras sobreditas qualidades, digo
que seria bom ser considerado liberal. {[}2{]} Contudo, a liberalidade
praticada de modo que você sejas reconhecido como tal, prejudica"-o,
porque, se ela é praticada virtuosamente e como se deve praticá"-la, não
lhe será atribuida e não lhe recairá a infâmia do seu contrário; porém,
querendo conservar entre os homens a fama de liberal, é necessário não
deixar de lado nenhuma manifestação de sua suntuosidade, de tal modo que
um príncipe liberal sempre consumirá em semelhantes obras todas as suas
riquezas; {[}3{]} e será necessário, por fim, se quiser conservar a fama
de liberal, tributar o povo extraordinariamente, cobrar impostos e fazer
todas aquelas coisas que se fazem para obter dinheiro, o que começará a
fazê"-lo odioso aos súditos ou pouco estimado por todos, pois ficarão
pobres. {[}4{]} De modo que, com esta sua liberalidade, tendo ofendido
muitos e premiado poucos, sente logo qualquer incipiente contrariedade,
e periclita logo no primeiro perigo: percebendo isso e desejando voltar
atrás, incorre, imediatamente, na infâmia de miserável. {[}5{]}
Portanto, se um príncipe não puder usar desta \emph{virtù} de liberal
sem dano para si, para que ela lhe seja reconhecida, deve, se é
prudente, não se importar com a fama de miserável, porque, com o tempo,
será considerado cada vez mais liberal, ao verem que, com a sua
parcimônia, suas receitas lhe bastam, podendo defender"-se daqueles que
lhe fazem guerra, podendo fazer obras sem tributar o povo. {[}6{]} De
tal modo que se torna liberal para todos aqueles dos quais não subtrai,
que são infinitos, e miserável para todos aqueles a quem não dá, que são
poucos.

{[}7{]} No nosso tempo não vimos fazer grandes coisas senão aqueles que
foram considerados miseráveis, os outros foram extintos. {[}8{]} O papa
Júlio \versal{II}, ao se servir da fama de liberal para conseguir o papado, não
pensou, pois, em conservar essa fama, para poder guerrear. {[}9{]} O
atual rei de França\footnote{Luiz \versal{XII}. Essa expressão é um dado
  fundamental para determinar o limite temporal de quando foi escrito
  \emph{O Príncipe}, no caso, antes de 31 de dezembro de 1514, data da
  morte do rei. Ou seja, o texto estava concluído antes desta data,
  senão Maquiavel teria retirado a expressão ``o atual rei''.} fez
tantas guerras sem impor um imposto extraordinário aos seus, somente
porque administrou as despesas supérfluas com sua grande parcimônia.
{[}10{]} O atual rei de Espanha\footnote{Fernando, o católico.}, se
fosse tomado por liberal, não teria nem vencido nem feito tantas obras.
{[}11{]} Portanto, um príncipe deve preocupar"-se pouco -- para não ter
de roubar os súditos, para poder defender"-se, para não se tornar pobre e
desprezível, para não ser forçado a tornar"-se rapace --, de ter a fama
de miserável, porque este é um daqueles vícios que o faz reinar.
{[}12{]} E se alguém dissesse: César\footnote{Caio Júlio César (100 a 44
  a.C).} com a liberalidade ganhou o império, e muitos outros, por terem
sido e serem considerados liberais, ganharam postos altíssimos,
respondo: ou você já é príncipe ou você está em vias de conquistar o
principado. {[}13{]} No primeiro caso, essa liberalidade é danosa. No
segundo, é bem necessário ser considerado liberal; e César era um dos
que queriam obter o principado de Roma, mas, depois que o alcançou, se
tivesse sobrevivido e não fosse moderado naquelas despesas, teria
destruído aquele império\footnote{Maquiavel faz remissão aqui aos fatos
  ocorridos no ano 44 a.C., quando Julio César entra com suas tropas em
  Roma e exige ser coroado como príncipe dos romanos e sumo pontífice,
  mas é assassinado em 15 de março no senado com punhaladas desferidas
  pelos senadores romanos.}.

{[}14{]} E se alguém replicasse: muitos foram os príncipes considerados
liberalíssimos, que com o exército fizeram grandes coisas, respondo: ou
o príncipe gasta do seu e dos seus súditos, ou de outros. {[}15{]} No
primeiro caso, deve ser comedida. No segundo, não deve deixar de lado
nenhuma manifestação de liberalidade. {[}16{]} E aquele príncipe que vai
com os exércitos, que se alimenta das pilhagens, dos saques e dos
roubos, lida com o que é dos outros; a ele é necessária essa
liberalidade, pois, de outro modo, não seria seguido pelos soldados.
{[}17{]} E, daquilo que não é seu ou dos seus súditos, se pode ser maior
doador, como o foi Ciro, César e Alexandre: porque gastar o que é dos
outros não tira sua reputação, ao contrário, agrega"-lhe mais; somente
gastar do que é seu é que o prejudica. {[}18{]} E não há nada que
consuma tanto a si mesma quanto a liberalidade, enquanto você a usa,
perde a faculdade de usá"-la e se torna ou pobre ou desprezível, ou, para
fugir da pobreza, rapace e odioso. {[}19{]} E, entre todas as coisas de
que um príncipe se deve guardar, está o ser temido e odiado, e a
liberalidade o conduz a uma coisa e outra. {[}20{]} Portanto, é mais
sábio ter a fama de miserável, que gera uma infâmia sem ódio, do que,
por querer a fama de liberal, ser necessário ter a fama de ladrão, que
gera uma infâmia com ódio.

\quebra\section{\emph{DE CRUDELITATE ET PIETATE; ET AN SIT MELIUS AMARI QUAM TIMERI, VEL
E CONTRA}\break
{[}Da crueldade e da piedade; e se é melhor ser amado do que temido ou~o~contrário{]}}

{[}1{]} Examinando as outras qualidades apresentadas acima, digo que
todo príncipe deve desejar ser tido por piedoso e não cruel, todavia,
deve cuidar em não usar mal esta piedade. {[}2{]} César Borgia era tido
por cruel, não obstante, aquela sua crueldade pacificou a Romanha,
uniu"-a, reconduzindo"-a à paz e à confiança. {[}3{]} Se se considerar
bem, se verá que ele foi muito mais piedoso que o povo florentino, o
qual, para fugir da fama de cruel, deixou que destruíssem
Pistóia\footnote{A cidade de Pistóia foi controlada pelos florentinos
  desde 1328 e, ao longo do século \versal{XV}, ficou dividida em dois partidos:
  um de apoiadores da família Médici e outro contrário, controlado pelo
  partido dos Panciatici. Em 1500 e 1501, diante da ameaça de César
  Bórgia em conquistar a cidade, os florentinos deixaram que ela caísse
  em uma guerra civil com a derrota dos Panciatici. Esses fatos, também
  analisados por Maquiavel em \emph{De rebus pistoriensibus} (de março
  de 1502), mostram como os florentinos foram inábeis no controle da
  cidade e permitiram que uma parte exercesse sua crueldade e força
  sobre a outra, sem que isso implicasse ao final na pacificação e
  unidade entre eles.}. {[}4{]} Um príncipe deve, portanto, não se
importar com a má fama de cruel para manter os seus súditos unidos e
confiantes, porque, com pouquíssimas punições exemplares, será mais
piedoso do que aqueles que, por excessiva piedade, deixam que avancem as
desordens, das quais nascem os assassínios ou os roubos; porque estes
costumam ofender uma comunidade inteira, e aquelas execuções, que partem
do príncipe, prejudicam a um particular. {[}5{]} E, entre todos os
príncipes, ao príncipe novo é impossível fugir da fama de cruel, por
serem os estados novos repletos de perigos. {[}6{]} E Virgílio, pela
boca de Dido, disse: \emph{Res dura, et regni novitas me talia cogunt
Moliri, et late fines custode tueri}\footnote{``A difícil circunstância
  e a novidade do meu reino me constrange a usar de tais modos e vigiar
  todas as partes até os confins.'' Eneida, \versal{I}, 563-564.}. {[}7{]}
Contudo, deve ser prudente no crer e no se mover, não criando medos para
si mesmo, e proceder, de modo temperado, com prudência e humanidade, de
modo que a muita confiança não o faça incauto e a muita desconfiança não
o torne intolerável.

{[}8{]} Nasce disto uma discussão: se é melhor ser amado que temido, ou
o contrário. {[}9{]} Responde"-se que se gostaria de ser um e outro, mas,
porque é difícil tê"-los juntos, é muito mais seguro ser temido que
amado, quando se deve ser desprovido de um dos dois. {[}10{]} Porque,
dos homens, pode"-se dizer isto: que geralmente são ingratos, volúveis,
simuladores e dissimuladores, esquivos aos perigos, cobiçosos de ganho;
e enquanto os beneficia são todos seus: oferecem o sangue, os bens, a
vida, os filhos, como acima se disse\footnote{Confira cap. \versal{IX}, 26.},
quando a necessidade está longe; mas, quando ela se avizinha de você,
revoltam"-se, e aquele príncipe que está todo fundado em suas palavras,
encontrando"-se despido de outra predisposição, arruína"-se. {[}11{]}
Porque as amizades que se conquistam com dinheiro, e não com grandeza e
nobreza de alma, se compram, mas não se possuem e, quando necessárias,
não se podem gastá"-las; e os homens têm menos respeito para ofender a
alguém que se faz amar, do que alguém que se faz temer, porque o amor é
mantido por um vínculo de obrigação, o qual, por serem os homens
decepcionantes\footnote{Uma das poucas ocasiões que Maquiavel emite um
  juízo sobre a natureza humana, coisa que ele fará também nos
  \emph{Discursos} (\versal{I}, 3), o que não nos permite, contudo, afirmar que
  haja uma antropologia em seus escritos. Cf. Bignotto, N.
  \emph{Antropologia negativa em Maquiavel} in \versal{ANALYTICA}, Rio de
  Janeiro, vol 12 nº 2, 2008, p. 77-100.}, é rompido em todas as
ocasiões úteis a si próprios, mas o temor mantém"-se pelo medo da punição
que não o abandona nunca.

{[}12{]} Contudo, deve o príncipe fazer"-se temer de tal modo que, se não
conquista o amor, evite o ódio, porque pode muito bem estarem juntos o
ser temido e o não ser odiado. {[}13{]} O que sempre acontecerá,
enquanto o príncipe se abstiver dos bens dos seus cidadãos, dos seus
súditos e das suas mulheres. E quando lhe for necessário atentar contra
a família de algum deles, fazê"-lo desde que haja justificação
conveniente e causa manifesta. {[}14{]} Mas, sobretudo, deve abster"-se
dos bens dos outros, porque os homens esquecem mais rápido a morte do
pai do que a perda do patrimônio; depois, nunca faltam motivos para
confiscar bens, e aqueles que começam a viver do roubo sempre encontram
motivos para tomar aquilo que é dos outros, e, inversamente, os motivos
para atentar contra a vida são mais raros e acabam mais depressa.

{[}15{]} Mas, quando o príncipe está com os exércitos e tem sob seu
comando uma multidão de soldados, então é totalmente necessário não se
importar com a fama de cruel, porque sem esta não houve nunca exército
unido nem disposto à ação alguma. {[}16{]} Entre as maravilhosas ações
de Aníbal\footnote{Comandante militar de Cartago, que guerreou contra os
  romanos de 221 a 202 a.C., chegando com os seus exércitos até os muros
  de Roma, mas que teve sua grande derrota na batalha de Zama, com os
  exércitos romanos comandados por Cipião {[}cf. nota 206{]}. Essas
  informações sobre Aníbal, Maquiavel retira da \emph{História de Roma}
  de Tito Lívio (\versal{XXI} 4, 9 e \versal{XXVIII} 12, 2-5).} enumera"-se esta, na qual,
tendo um exército grandíssimo, composto por inúmeras raças de homens,
levou"-os para guerrear em terras alheias, e não surgiu nunca nenhuma
distensão, nem entre eles, nem contra o príncipe, tanto na má como na
boa fortuna. {[}17{]} O que não pode nascer de outra coisa senão daquela
sua crueldade desumana, a qual, juntamente com as suas infinitas
\emph{virtù}, o fez sempre venerável e temível para os seus soldados.
{[}18{]} E sem aquela crueldade, que produziu este efeito, as suas
outras \emph{virtù} não lhe bastariam, e os escritores neste ponto são
pouco perspicazes: de uma parte admiram esta sua ação, de outra condenam
a principal razão desta.

{[}19{]} E que é verdade que as suas outras \emph{virtù} não teriam
bastado, pode"-se comprovar por Cipião, exemplo raro não somente nos seus
tempos, mas em toda a memória dos fatos que se sabem, cujos exércitos na
Espanha se rebelaram, o que não aconte ceu por outro motivo senão pela
sua excessiva piedade, a qual tinha dado a seus soldados mais liberdade
do que convinha à disciplina militar. {[}20{]} Coisa que lhe foi
repreendida por Fábio Maximo\footnote{Quinto Fabio Maximo (275-203
  a.C.), general romano.} no Senado, sendo chamado por ele de corruptor
da milícia romana. {[}21{]} Os Locrenses\footnote{Os habitantes de Locri
  Epizefrii, na Calábria, sul da Itália.}, tendo sido arruinados por um
legado de Cipião, não foram por ele vingados, nem a insolência daquele
legado corrigida, nascendo tudo daquela sua natureza permissiva, tanto
que, querendo alguém desculpá"-lo no Senado, disse que como ele havia
muitos homens que mais sabiam não errar do que corrigir os erros.
{[}22{]} Esta característica teria com o tempo desonrado a fama e a
glória de Cipião, se ele tivesse perseverado junto com ela no comando
militar\footnote{``Comando Militar'' aqui traduz ``\emph{imperio}" que
  em seu significado original denota, principalmente, o comando militar
  que se possui.}, mas, vivendo sob governo do Senado, esta sua
qualidade danosa não somente ficou escondida, mas foi sua glória.

{[}23{]} Concluo, portanto, voltando ao ser temido e amado -- os homens
amam quando lhes convém e temem o príncipe pelas escolhas deste -- que
um príncipe sábio deve fundar"-se sobre o que é seu e não sobre o que é
dos outros; deve somente aplicar"-se em evitar o ódio, como foi dito.

\quebra\section{\emph{QUOMODO FIDES A PRINCIPIBUS SIT~SERVANDA}\break
{[}De que modo os príncipes devem~conservar~a~fé\protect\footnote{\uppercase{M}aquiavel está
  utilizando aqui o termo \emph{fede} que é traduzido pelo português
  ``fé''. Contudo, não se trata somente da fé como crença em valores
  religiosos, mas da confiança que se deposita na palavra dada a alguém
  ou na confiança depositada em outra pessoa.}{]}}

{[}1{]} Que seja louvável em um príncipe conservar a fé e viver com
integridade e não com astúcia, todos entendem; não obstante se vê, por
experiência em nossos tempos, que aqueles príncipes que fizeram grandes
coisas, tiveram a fé em pouca conta e souberam com a astúcia enganar o
juízo dos homens, e, ao final, superaram os que fundaram suas ações na
lealdade.

{[}2{]} Deveis, portanto, saber que são dois os gêneros de combate: um
com as leis, outro com a força\footnote{Esta ideia já se encontra
  expressa no início do cap. 12.}. {[}3{]} O primeiro é próprio do homem
e o segundo dos animais. {[}4{]} Mas porque o primeiro muitas vezes não
basta, convém recorrer ao segundo: portanto, a um príncipe é necessário
saber bem usar o animal e o homem. {[}5{]} Esta parte foi ensinada pelos
escritores antigos aos príncipes secretamente\footnote{Maquiavel faz
  referência aqui a uma tradição literária denominada de \emph{Arcana
  Imperii} ou os segredos da arte de governar, do qual o gênero
  literário dos ``espelhos de príncipes'' faziam parte na medida em que
  revelavam informações secretas sobre como conduzir os governos. Cf.
  Senellart, M. \emph{As artes de governar.} São Paulo: ed. 34, 2006,
  parte \versal{III}, cap. 2.}, os quais escrevem como Aquiles, e muitos outros
daqueles príncipes antigos, foram alimentados pelo centauro
Quiron\footnote{Centauro é um monstro fabuloso, metade homem e metade
  cavalo. Quíron era um centauro sábio, filho de Filira e do deus
  Cronos, tutor de Aquiles.}, para que sob a sua disciplina os educasse.
{[}6{]} O que não quer dizer outra coisa ter por predecessor um meio
animal e um meio homem senão a necessidade que um príncipe tem de saber
usar uma e outra natureza, e que uma sem a outra não é durável\footnote{No
  caso, o governo do príncipe, que se perde sem uma dessas naturezas.}.

{[}7{]} Sendo, pois, necessário a um príncipe saber bem usar o animal,
deve destes tomar por modelos a raposa e o leão: porque o leão não se
defende das armadilhas e a raposa não se defende dos lobos. Necessita,
pois, ser raposa para conhecer as armadilhas e leão para amedrontar os
lobos: aqueles que são somente leão não entendem nada de governo.
{[}8{]} Não pode, e nem deve, portanto, um senhor prudente conservar a
fé quando tal observância se lhe volta contra e quando tiverem
desaparecido os motivos que levaram"-no a fazer suas promessas. {[}9{]} E
se os homens fossem todos bons, este preceito não seria bom, mas porque
são decepcionantes e não a observariam consigo, você, então, não tem que
observar a fé com eles; nem faltarão ocasiões legítimas para um príncipe
poder colorir a inobservância. {[}10{]} Disto poder"-se"-ia dar infinitos
exemplos modernos e mostrar quantos tratados de paz, quantas promessas
tornaram"-se inválidas e vãs pela infidelidade dos príncipes; e aquele
que soube melhor usar a raposa, teve melhor resultado. {[}11{]} Mas é
necessário saber mascarar esta natureza e ser grande simulador e
dissimulador: e são tão ingênuos os homens, e tanto se sujeitam às
necessidades presentes, que aquele que engana encontrará sempre quem se
deixará enganar.

{[}12{]} Eu não quero, dos exemplos recentes, silenciar um. Alexandre \versal{VI}
não fez nunca outra coisa, não pensou nunca em outra coisa, senão em
enganar os homens, e sempre encontrou matéria para poder fazê"-lo: e
nunca houve homem que tivesse maior eficácia em assegurar, e com os
maiores juramentos afirmasse uma coisa e que os observasse menos.
Todavia, sempre os enganos tiveram os resultados que ele desejava,
porque conhecia bem esta parte do mundo.

{[}13{]} A um príncipe, portanto, não é necessário ter de fato todas as
sobreditas qualidades, mas é muito necessário parecer tê"-las. Assim,
ousarei dizer isto: que, tendo"-as e observando"-as sempre são danosas, e
parecendo tê"-las são úteis; como parecer piedoso, fiel, humano, íntegro,
religioso, e sê"-lo: mas estar com o ânimo predisposto, para que,
necessitando não sê"-lo, você possa e saiba ser o contrário. {[}14{]} E
deve"-se entender isto: que um príncipe, e muito mais um príncipe novo,
não pode observar todas aquelas coisas pelas quais os homens são
considerados bons, sendo frequentemente necessário, para conservar o
governo, agir contra a fé, contra a caridade, contra a humanidade,
contra a religião. {[}15{]} E, porém, é necessário que ele tenha um
ânimo disposto a mudar segundo o que lhe ordenem os ventos da fortuna e
as variações das coisas exigirem; e, como acima se disse, podendo não
separar"-se do bem, mas, se necessário, saber praticar mal.

{[}16{]} Deve, pois, um príncipe ter grande cuidado para que não lhe
saia jamais da boca uma coisa que não seja plena das cinco sobreditas
qualidades, e que pareça, ao vê"-lo e escutá"-lo, todo piedade, todo
fiel, todo íntegro, todo humano, todo religioso; e não há coisa mais
necessária de se parecer ter do que esta última qualidade. {[}17{]} Os
homens, em geral, julgam mais com os olhos do que com as mãos, mais
pelas aparências, porque se vêem todos e se conhecem poucos; todos vêem
aquilo que você pareces ser, poucos conhecem aquilo que você é. E
aqueles poucos não se atrevem a opor"-se à opinião dos muitos que têm a
majestade do Estado para defende"-los; e nas ações de todos o homens,
sobretudo na dos príncipes, quando não há juiz para quem reclamar,
olha"-se para os fins.

{[}18{]} Faça, portanto, um príncipe tudo para vencer e conservar o
governo: os meios serão sempre julgados honrosos e por todos serão
louvados, porque o vulgo se deixa levar por aquilo que parece e pelo
resultado das coisas, e no mundo não há senão o vulgo, e os poucos não
têm lugar quando os muitos têm onde se apoiarem. {[}19{]} Certo
príncipe\footnote{Alguns comentadores reconheceram aqui uma alusão ao
  rei espanhol Fernando, o católico.} dos tempos presentes, o qual não é
bom nomear, não prega nunca outra coisa senão a paz e a fé, e de uma e
de outra é inimicíssimo; e uma e outra, se ele as tivesse observado,
ter"-lhe"-iam muitas vezes tirado a reputação e o governo.

\quebra\section{\emph{DE CONTEMPTU ET ODIO FUGIENDO}\break
{[}Como se deve evitar ser desprezado~e~odiado{]}}

{[}1{]} Uma vez que, acerca das qualidades de que acima\footnote{Confira
  cap. \versal{XV}.} se fez menção, eu falei das mais importantes, quero
discorrer brevemente sobre as outras a partir destes princípios gerais:
que o príncipe pense, como acima foi dito em parte\footnote{Confira cap.
  \versal{XVI}, 3 e 18; cap. \versal{XVII}, 12.}, acerca de como evitar aquelas coisas que
o fazem odioso e desprezível; e sempre que ele as evitar, terá cumprido
a sua parte e não encontrará nas outras infâmias qualquer perigo.
{[}2{]} O que o faz odioso, sobretudo, como eu disse\footnote{Confira
  cap. \versal{XVII}, 13.}, ser rapace e usurpador dos bens e das mulheres dos
súditos, do que se deve abster. {[}3{]} E toda vez que não se tira da
totalidade\footnote{Algumas vezes Maquiavel se refere à
  \emph{universalità delli uomini}, o que poderia ser traduzido como
  universalidade ou totalidade, em ambos os casos conservando a ideia de
  congregação de todos os homens. Neste capítulo, mais do que nos
  outros, ao mobilizar mais vezes o termo \emph{universalle} e
  \emph{universalità,} nota"-se claramente que ele pretende apresentar
  normas gerais para a condução do governo, num esforço em constituir
  uma teoria geral da ação política.} dos homens nem os bens e nem a
honra, eles vivem contentes: e somente se tem que combater a ambição de
poucos, a qual se contém de muitos modos e com facilidade. {[}4{]} O que
o faz desprezado é ser considerado volúvel, leviano, efeminado,
pusilânime, irresoluto: do que um príncipe deve se guardar do mesmo modo
que se evita um obstáculo perigoso, e planejar que, nas suas ações, se
reconheça grandeza, animosidade, gravidade, firmeza; e acerca da
administração privada dos súditos, querer que a sua sentença seja
irrevogável, e se conserve em tal opinião, que ninguém pense nem em
enganá"-lo e nem em iludi"-lo.

{[}5{]} O príncipe que cria em torno de si esta opinião é sempre
estimado, e contra quem é estimado com dificuldade se conjura\footnote{Conjurar
  remete à conspiração feita entre um grupo, o que implica numa
  pactuação e associação entre os membros desses grupos.}, com
dificuldade é assaltado, desde que se compreenda que é excelente e
reverenciado pelos seus. {[}6{]} Porque um príncipe deve ter dois medos:
um de dentro, por conta dos súditos; o outro de fora, por conta dos
poderosos externos. {[}7{]} Deste se defende com as boas armas e com os
bons amigos: e sempre, se tiver boas armas, terá bons amigos. {[}8{]} E
sempre estará firme a situação interna, quando estiver segura aquela
externa, se não estiver já conturbada por uma conjuração; e mesmo que
externamente se movessem contra ele, se ele se organizou e viveu
conforme eu disse, se ele não se amedrontar, sempre suportará todo
ímpeto, como eu disse que fez o espartano Nábis\footnote{Confira cap.
  \versal{IX}, 19.}.

{[}9{]} Mas, sobre os súditos, quando as coisas externas não os movem,
deve"-se temer que não conjurem secretamente, o que o príncipe se
assegura evitando sempre ser odiado ou desprezado, e mantendo o povo
satisfeito com ele: o que é necessário conseguir, como acima se disse
longamente. {[}10{]} E um dos mais poderosos remédios que tem um
príncipe contra as conjurações é não ser odiado pela totalidade dos
homens: porque sempre quem conjura crê com a morte do príncipe
satisfazer o povo, mas quando crê prejudica"-lo, não tem coragem para
tomar semelhante decisão. {[}11{]} Porque são infinitas as dificuldades
dos conjurados, e, por experiência, se vê que muitas foram as
conjurações, e poucas chegaram a um bom fim. {[}12{]} Porque quem
conjura não pode estar sozinho, nem pode aceitar a companhia senão
daqueles que crê ser descontentes; e logo que você revele a sua
disposição a um descontente, lhe dá motivos para se contentar, porque
denunciando"-o, ele pode tirar vantagem, de tal modo que, vendo a
vantagem assegurada desta parte, e da outra, da conjuração, vendo"-a
hesitante e repleta de riscos, acaba acontecendo que raramente o
conjurado seja amigo ou que seja tão obstinado inimigo do príncipe, ao
ponto de se conservar fiel.

{[}13{]} E, para reduzir tudo isso em breves
palavras, digo que, da parte dos conjurados, não há senão medo, inveja,
o pressentimento da pena que os amedronta; mas, da parte do príncipe,
está a majestade do principado, as leis, a defesa dos amigos e do estado
que o protegem. {[}14{]} De tal modo que, unida a todas estas coisas a
benevolência popular, é impossível que alguém seja tão imprudente que
conjure: porque, usualmente, um conjurado tem que temer antes da
execução da conjura, e, neste caso, deve temer ainda mais depois, uma
vez executada, tendo por inimigo o povo, nem pode, por isso, esperar
refúgio algum\footnote{Período intricado, no qual Maquiavel equipara os
  perigos anteriores e posteriores a uma conjuração.}.

{[}15{]} Desta
matéria se poderiam dar infinitos exemplos, mas quero apenas limitar"-me
a um que ocorreu nos tempos de nossos pais. {[}16{]} Senhor Aníbal
Bentivoglio\footnote{Aníbal \versal{I} (1413-1445), senhor de Bologna de 1443 a
  1445.}, avo do atual Senhor Aníbal\footnote{Aníbal \versal{II} (1469-1540),
  filho de Giovanni \versal{II} e neto de Aníbal \versal{I}, foi \emph{condottiero}
  (comandante militar) e responsável pela restauração do governo dos
  Bentivogli na cidade de Bolonha de maio de 1511 a junho de 1512.}, que
era príncipe em Bolonha, foi assassinado pelos Canneschi\footnote{Canneschi,
  os partidários dos Canetoli, que assassinaram Aníbal \versal{I} em 24 de junho
  de 1445. Estes fatos são narrados mais extensamente na \emph{História
  Florentina,} livro \versal{VI}, cap. 9.}, que conjuraram contra ele, não
restando outro descendente dele senão Senhor Giovanni\footnote{Giovanni
  \versal{II} (1443-1508), filho de Aníbal \versal{I}, que na época do assassinato tinha
  apenas 2 anos. Governou posteriormente Bolonha de 1463 a 1506.}, que
estava nas fraldas; imediatamente se levantou o povo, após tal
homicídio, e matou todos os Canneschi. {[}17{]} O que aconteceu pela
benevolência popular da qual a casa dos Bentivogli gozava naqueles
tempos, a qual foi tanta, que, morto Aníbal, não restando daquela casa
ninguém em Bolonha que pudesse reger o estado, e tendo indícios de que
em Florença havia alguém nascido dos Bentivogli\footnote{Sante
  (1424-1463), primo em primeiro grau de Aníbal \versal{I}, governou Bolonha de
  1443-1463.} que tinha sido considerado, até então, filho de um
ferreiro, vieram por isso os bolonheses à Florença e lhe deram o governo
daquela cidade, a qual foi governada por ele até que Senhor Giovanni
chegasse à idade conveniente para o governo\footnote{Sante morre em
  1463, passando o governo a Giovanni \versal{II}, que tinha na época 20 anos.}.

{[}18{]} Concluo, portanto, afirmando que um príncipe deve dar pouca
atenção às conjurações, quando o povo lhe é benevolente, mas, quando
este lhe é inimigo e odeia"-o, deve temer todas as coisas e a todos.
{[}19{]} E os estados bem ordenados e os príncipes sábios pensaram com
toda diligência em não desprezar os grandes e em satisfazer o povo e
mantê"-lo contente, porque esta é uma das matérias mais importantes que
tem um príncipe.

{[}20{]} Entre os reinos bem ordenados e governados de nosso tempo, está
o da França, e neste se encontram uma infinidade de boas instituições,
das quais depende a liberdade e a segurança do rei, das quais a primeira
é o parlamento e a sua autoridade. {[}21{]} Porque aquele que ordenou
aquele reino, conhecendo a ambição dos poderosos e a sua insolência, e
julgando ser"-lhes necessário um freio na boca que os corrigisse -- e, de
outra parte, conhecendo o ódio do povo\footnote{Aqui o termo
  originalmente utilizado ``\emph{universale}'' diz respeito ao povo,
  grupo político antagônico dos grandes conforme apresentado no cap. \versal{IX},
  2.} contra os grandes, nascido do medo, e querendo protegê"-los --, não
quis que este fosse um remédio particular do rei, para tirar"-lhe as
dificuldades que poderia ter com os grandes, ao favorecer a população, e
com a população, ao favorecer os grandes.

{[}22{]} Por isso constitui um
terceiro em juiz, que seria aquele que, sem o encargo do rei, submetesse
os grandes e favorecesse os menores: nem poderia ser esta ordenação
melhor nem mais prudente, nem há maior causa para a segurança do rei e
do reino. {[}23{]} Do que se pode tirar um outro princípio digno de
nota: que os príncipes devem atribuir aos outros as tarefas
desagradáveis, e a si mesmos aquelas que lhes geram graças. {[}24{]} E
de novo concluo que um príncipe deve estimar os grandes, mas não se
fazer odiar pelo povo.

{[}25{]} Pareceria a muitos, considerada a vida e a morte de alguns
imperadores romanos, que fossem exemplos contrários a esta minha
opinião, encontrando algum que tenha sempre vivido distintamente e
demonstrado grande \emph{virtù} de alma, e, não obstante, ter perdido o
império, ou mesmo ter sido assassinado pelos seus que conjuraram contra
ele. {[}26{]} Querendo, portanto, responder a esta objeção, discorrerei
sobre as qualidades de alguns imperadores, mostrando as razões das suas
ruínas, não diferentes daquelas que foram antes apresentadas para mim;
e, por outro lado, examinarei aquelas coisas que são notáveis a quem lê
as ações daqueles tempos. {[}27{]} E quero restringir"-me a considerar
todos aqueles imperadores que sucederam ao império de Marco, o filósofo,
até Maxímino, os quais foram: Marco, seu filho Cômodo, Pertinax,
Juliano, Severo, Antonino, seu filho Caracala, Macrino, Eliogábalo,
Alexandre e Maxímino\footnote{Pela ordem: Marco Aurélio Antonino
  (161-180 d.C.), Aurélio Cômodo Antonino (180-192 d.C.), Públio Elvio
  Pertinax (janeiro a março de 193 d.C.), Marco Dídio Juliano (março a
  junho de 193 d.C.), Sétimo Severo (193-211 d.C.), Antonino Caracala
  (211-217 d.C.), Opílio Macrino (217-18 d.C.), Eliogábalo (218-222
  d.C.), Severo Alexandre (222-235 d.C.) e Maximino, o Traço (235-238
  d.C.).}. {[}28{]} E é notável, primeiramente, que, enquanto nos outros
principados se deve combater apenas a ambição dos grandes e a insolência
do povo, os imperadores romanos tinham uma terceira dificuldade: ter que
suportar a crueldade e a avareza dos soldados\footnote{Aqui Maquiavel
  apresenta uma novidade conceitual na teoria dos humores políticos:
  agora já basta controlar os desejos antagônicos dos grandes e do povo,
  mas é necessário também lidar com uma outra força política, a saber: a
  crueldade e avareza dos soldados. De tal modo que há, portanto, não
  apenas duas forças políticas concorrendo sobre o governo do príncipe,
  mas também uma terceira força, os militares, que se constituem agora
  como força política relevante.}. {[}29{]} Coisa que, de tão difícil,
foi a ruína de muitos, sendo trabalhoso satisfazer aos soldados e ao
povo, porque o povo amava a quietude, e por isto eram gratos aos
príncipes modestos, e os soldados amavam os príncipes de alma militar e
que fossem cruéis, insolentes e rapaces, qualidades que desejavam que
ele exercesse sobre povo, para poder ter o soldo duplicado e desafogar a
sua avareza e crueldade. {[}30{]} Coisas que fizeram com que aqueles
imperadores, que por natureza ou por arte\footnote{Por artifício, por
  aprendizado.} não tinham uma grande reputação, com a qual refreavam
uns e outros, sempre se arruinavam. {[}31{]} E muitos deles, mais ainda
aqueles que como homens novos chegavam ao principado, quando percebiam a
dificuldade destas duas diversidades de humores\footnote{Note"-se que
  agora os humores antagônicos são entre o povo e os militares e não
  mais, como no cap. \versal{IX}, entre os grandes e o povo, o que demonstra
  claramente a existência de três humores conflitantes na dinâmica
  política apresentada aqui em \emph{O Príncipe.}}, se ocupavam em
satisfazer os soldados, pouco se preocupando em injuriar\footnote{Convém
  lembrar que o termo italinao \emph{iniuriare} advém do latim
  \emph{iniuria}, que é ir contra a lei, violar o direito, a justiça.
  Logo, ``injuriar o povo'' é atacar o seu direito.} o povo. {[}32{]}
Tal decisão era necessária, porque, não podendo os príncipes evitar de
serem odiado por alguns, devem esforçar"-se primeiro em não ser odiado
pela totalidade\footnote{Totalidade, aqui, o universo dos cidadãos,
  independente do seu estrato social, contudo, não engloba o conjunto
  dos habitantes.} dos cidadãos, e quando não pode conseguir isto,
devem"-se empenhar, com toda astúcia, para evitar o ódio daquela
totalidade que é mais poderosos. {[}33{]} Porém, aqueles imperadores
que, por serem novos, tinham necessitado de favores extraordinários,
procuravam o apoio dos soldados, mais do que do povo, o que, no entanto,
era útil ou não, conforme o príncipe soubesse conservar aquela reputação
entre eles.

{[}34{]} Foi por estas razões sobreditas que Marco, Pertinax e
Alexandre, sendo todos de vida modesta, amantes da justiça, inimigos da
crueldade, humanos e benignos, tiveram todos, com exceção de Marco,
triste fim. {[}35{]} Marco só viveu e morreu honradíssimo porque ele
chegou ao poder por direito hereditário, e não tinha que ser reconhecido
nem pelos soldados nem pelo povo; depois, possuía muitas \emph{virtù},
que o faziam venerando, tendo sempre, enquanto viveu, uma e outra
ordenação\footnote{Aqui Maquiavel se refere ao povo e aos exércitos como
  ``ordenação'', visto serem os fundamentos do governo do príncipe.
  Terminologia esta totalmente coerente com a economia do texto, haja
  vista que os dois grupos apresentam"-se como sustentáculos do governo,
  logo, como ordenações políticas, tal qual ele mobiliza o termo em
  todos os seus escritos políticos.} nos seus limites, e não foi nunca
nem odiado nem desprezado. {[}36{]} Mas Pertinax tornou"-se imperador
contra a vontade dos soldados -- os quais estavam habituados a viver
licenciosamente sob Cômodo, e não puderam suportar aquela vida honesta à
qual Pertinax queria submetê"-los --, daí nasceu o ódio e, a este ódio,
juntou"-se o desprezo, por ser ele velho, e arruinou"-se logo no início de
sua administração. {[}37{]} E aqui se deve notar que o ódio se conquista
mediante as boas obras, bem como pelas más; porém, como eu disse acima,
um príncipe, querendo conservar o estado, é frequentemente forçado a não
ser bom. {[}38{]} Porque, quando aquela totalidade, ou o povo, ou os
soldados, ou que sejam os grandes, da qual você julga ter mais
necessidade para conservar"-se, é corrompida, convém a você seguir"-lhe o
humor para satisfazê"-la, e então as boas obras lhe são inimigas.

{[}39{]} Mas consideremos Alexandre, o qual foi de tanta bondade que,
entre as outras coisas louváveis que lhe são atribuídas, há esta: que em
quatorze anos que teve o comando militar, ninguém foi morto por ele sem
ser julgado; não obstante, sendo considerado efeminado e homem que se
deixasse governar pela mãe\footnote{Severo Alexandre (222-235 d.C.) foi
  adotado pelo imperador Eliogabalo (218-222 d.C.), sendo sua mãe Giulia
  Mamea. Ambos foram assassinados por ordem de Maximino (235-238 d.C.).},
por isto caiu em desprezo, o exército conspirou contra ele e o
assassinou.

{[}40{]} Discorrendo agora, pelo contrário, acerca das qualidades de
Cômodo, de Severo, Antonino Caracala e Maximino, vemo"-nos crudelíssimos
e rapacíssimos, os quais, para satisfazer os soldados, não abstiveram de
nenhum tipo de injúria que pudessem cometer contra o povo. {[}41{]} E
todos, exceto Severo, tiveram triste fim, porque Severo teve tanta
\emph{virtù} que, conservando os soldados amigos, ainda que o povo fosse
oprimido por ele, pode sempre reinar felizmente, porque aquelas suas
\emph{virtù} faziam"-no admirável no conceito dos soldados e do povo, já
que este permanecia de certo modo atônito e atordoado, e aqueles outros
reverentes e satisfeitos. {[}42{]} E porque as ações dele foram grandes
e notáveis para um príncipe novo, eu quero mostrar brevemente o quanto
ele soube bem usar a natureza da raposa e do leão, cujas naturezas eu
disse acima\footnote{Confira cap. \versal{XVIII}, 7-12.} serem necessárias a um
príncipe imitar.

{[}43{]} Tendo Severo conhecido a indolência do imperador Juliano,
persuadiu o seu exército, do qual era capitão na Eslavônia\footnote{Territórios
  eslavos a leste do Mar Adriático, Ilíria para os romanos, atual
  Bálcãs.}, de que era bom ele ir a Roma e vingar a morte de Pertinax, o
qual tinha sido morto pelos soldados pretorianos. {[}44{]} E com este
pretexto, sem mostrar que aspirava ao comando militar, moveu o exército
contra Roma e chegou na Itália antes que soubessem da sua partida.
{[}45{]} Chegando em Roma e morto Juliano, foi por temor eleito
imperador pelo Senado. {[}46{]} Restavam a Severo, depois deste
princípio, querendo assenhorear"-se totalmente do estado, duas
dificuldades: uma na Ásia, onde Nigro\footnote{Gaio Pescennio Nigro,
  comandante das legiões na Síria, foi vencido por Sétimo Severo em
  Cizico, Nicéia e Isso em 194, e depois foi morto.}, chefe dos
exércitos asiáticos, se fez nomear imperador; a outra no ocidente, onde
estava Albino\footnote{Décimo Clódio Albino, comandante das legiões na
  Britânia, foi chamado por Severo a Roma para dividir o império, quando
  foi assassinado em Lyon em 197.}, o qual ainda aspirava ao poder.
{[}47{]} E porque julgava perigoso mostrar"-se inimigo de todos os dois,
deliberou atacar Nigro e enganar Albino: ao qual escreveu que, tendo
sido eleito imperador pelo Senado, desejava partilhar aquela dignidade
com ele; e mandou"-lhe o título de César e, por deliberação do Senado,
associou"-o a ele como colega\footnote{Aqui o termo italiano
  \emph{collega} tem o sentido de posto ou grau equivalente, no sentido
  de pares num mesmo ofício.}, coisas que foram aceitas como verdadeiras
por Albino. {[}48{]} Mas, depois que Severo venceu e matou Nigro e
acalmadas as coisas no Oriente, voltando a Roma, lamentou"-se, no Senado,
de como Albino, pouco reconhecido dos benefícios por ele recebidos,
havia dolosamente procurado matá"-lo; por isto lhe era necessário punir a
sua ingratidão; foi encontrá"-lo depois na França e lhe tirou o status e
a vida. {[}49{]} Portanto, quem examinar atentamente as ações deste,
encontrará nele um ferocíssimo leão e uma astuciosíssima raposa, e o
verá temido e reverenciado por todos, e não odiado pelos exércitos; e
não se surpreenderá se ele, homem novo, pode manter tanto poder, porque
a sua grandíssima reputação sempre o defendeu daquele ódio que podia
nascer no povo por causa de seus roubos.

{[}50{]} Mas Antonino, seu filho, foi homem que também tinha qualidades
excelentíssimas e que o tornavam admirável no conceito do povo e grato
para os soldados, porque era homem de guerra, suportava todas as
fadigas, desprezava todo alimento delicado e qualquer outras molezas, o
que o fazia amado por todos os exércitos. {[}51{]} Todavia, a sua
ferocidade e a sua crueldade foram tantas e tão inacreditáveis que, por
ter, depois de infinitos assassinatos individuais, matado grande parte
do povo de Roma e todo aquele de Alexandria, tornou"-se muito odiado por
todo mundo e começou a ser temido também por aqueles que estavam perto
dele, de modo que foi assassinado por um centurião no meio do seu
exército. {[}52{]} Deve"-se notar aqui que semelhantes mortes, que
decorrem da deliberação de um ânimo obstinado, são, para os príncipes,
inevitáveis, porque qualquer um que não se importe em morrer pode
prejudicá"-lo, mas o príncipe deve temê"-los bem menos, porque esses são
raríssimos. {[}53{]} Deve apenas guardar"-se de não cometer grave injúria
contra algum daqueles dos quais se serve e que ele tem em torno de si a
serviço do seu principado, como fez Antonino, que matou injuriosamente
um irmão daquele centurião que o ameaçava todo dia, e ele, no entanto,
ainda o mantinha na sua guarda pessoal: o que era uma decisão temerária
e ruinosa, como veio a lhe acontecer\footnote{Este centurião que matou
  Antonino se chamava Marcial.}.

{[}54{]} Mas voltemos a Cômodo, para o qual foi muito fácil manter o
poder, por tê"-lo por direito hereditário, sendo filho de Marco, e apenas
lhe bastava seguir as pegadas do pai, que os soldados e o povo estariam
satisfeitos. {[}55{]} Mas, sendo de ânimo cruel e bestial, para poder
usar a sua rapacidade sobre o povo, dedicou"-se a entreter os exércitos e
torná"-los licenciosos; de outra parte, não conservando a sua dignidade,
apresentando"-se frequentemente nos teatros para combater com os
gladiadores, e fazendo outras coisas vis e pouco dignas da majestade
imperial, tornou"-se desprezível no conceito dos soldados. {[}56{]} E
sendo odiado de um lado e, desprezado de outro, conspiraram contra ele e
o mataram.

{[}57{]} Falta narrar as qualidades de Maximino. Ele foi homem muito
belicoso e, estando os exércitos insatisfeitos com a moleza de
Alexandre, sobre a qual se discorreu acima, assassinado este,
elegeram"-no para o Império, o qual não manteve por muito tempo, porque
duas coisas fizeram"-no odioso e desprezível. {[}58{]} Uma, ser vilíssimo
por já ter sido pastor de ovelhas na Trácia, o que era muito bem sabido
por todos e lhe trazia grande descrédito no conceito de todos. {[}59{]}
A outra, porque, tendo, no início do seu principado, adiado a ida a Roma
e a tomada da posse da sede imperial, criou para si a imagem de
crudelíssimo, tendo por meio dos seus prefeitos\footnote{Ou ministros
  imperiais, responsáveis pela administração de várias províncias.}, em
Roma e em toda parte do império, praticado muitas crueldades. {[}60{]}
Tanto que, impressionando a todos pelo desdém e pela vileza da sua
dinastia e pelo ódio causado pelo medo da sua ferocidade, primeiro se
rebelou a África, depois o Senado com todo o povo de Roma e toda a
Itália conspiraram contra ele; ao que se acrescentou o seu próprio
exército, o qual, assediando Aquiléia e encontrando dificuldade no
assalto, insatisfeito com a sua crueldade e por ver"-lhe tantos inimigos,
temendo"-o menos, assassinaram"-no.

{[}61{]} Eu não quero discorrer nem sobre Eliogábalo, nem sobre Macrino,
nem sobre Juliano, os quais, por serem em tudo desprezíveis, se
extinguiram de imediato, mas passarei à conclusão deste discurso, e digo
que os príncipes de nosso tempo tem esta dificuldade a menos, de
satisfazer com meios extraordinários os soldados ao lidar com eles,
porque, ainda que se deva ter para com eles alguma consideração, é coisa
que se resolve logo, por não ter nenhum destes príncipes exércitos que
estejam estacionados com o governo e a administração das províncias,
como eram os exércitos do Império Romano. {[}62{]} Outrossim, se então
era necessário satisfazer mais aos soldados que ao povo, era porque os
soldados podiam mais que o povo; agora, é mais necessário a todos os
príncipes, exceto ao Turco\footnote{O imperador otomano Selim \versal{I}, que
  governou de 1512 a 1520.} e ao Sultão\footnote{O Sultão do Egito,
  Tuman Bey, que foi assassinado pelos turcos em 1517, anexando o Egito
  ao Império Turco.}, satisfazer mais ao povo que aos soldados, porque o
povo pode mais do que aqueles. {[}63{]} Do que eu excluo o Turco, tendo
aquele sempre em torno de si doze mil soldados de infantaria e quinze
mil cavaleiros, dos quais depende a segurança e a força do seu reino; e
é necessário que, adiada qualquer outra consideração, que o Senhor os
conserve amigos. {[}64{]} Igualmente para o reino do Sultão, estando
todo nas mãos dos soldados, convém a ele também que, sem o respeito do
povo, mantenha"-os seus amigos. {[}65{]} E deve"-se notar que este estado
do Sultão é diferente de todos os outros principados, porque ele é
semelhante ao pontificado cristão, o qual não se pode chamar nem de
principado hereditário nem de principado novo, porque os filhos do
príncipe velho não são os seus herdeiros e não se tornam senhores, mas
aquele que é eleito para o posto o é por aqueles que têm autoridade para
isto; {[}66{]} e sendo este ordenamento antigo, não se pode chamar de
principado novo, pois naquele não há nenhuma daquelas dificuldades que
existem nos novos, porque, se bem que o príncipe seja novo, as
ordenações daquele estado são velhas e ordenadas para recebê"-lo como se
fosse seu senhor hereditário.

{[}67{]} Mas retornando à nossa matéria. Digo que qualquer um que
considerar o sobredito discurso verá ou o ódio ou o desprezo ser a
ocasião da ruína daqueles imperadores nomeados anteriormente, e saberá,
ainda, porque parte deles procedendo de um modo e parte de outro modo,
em ambas as situações alguns tiveram um final feliz e outros um triste
fim. {[}68{]} Porque, para Pertinax e Alexandre, por serem príncipes
novos, foi inútil e danoso imitar Marco, que estava no principado por
direito hereditário; e igualmente para Caracala, Cômodo e Maximino foi
coisa perniciosa imitar Severo, por não terem tido \emph{virtù}
suficiente para seguir as suas pegadas. {[}69{]} Portanto, um príncipe
novo em um principado novo não pode imitar as ações de Marco, nem por
isso é necessário seguir aquelas de Severo, mas deve tomar de Severo
aquelas qualidades que são necessárias para fundar o seu governo, e de
Marco aquelas que são convenientes e gloriosas para conservar um governo
que já esteja estabelecido e firme.

\quebra\section{\emph{AN ARCES ET MULTA ALIA QUAE QUOTIDIE A PRINCIPIBUS FIUNT UTILIA AN
INUTILIA SINT}
{[}Se as fortalezas e muitas outras coisas que cotidianamente os
príncipes fazem são úteis ou inúteis{]}}

{[}1{]} Alguns príncipes, para manter em segurança o governo, desarmaram
os seus súditos; alguns outros mantiveram divididas as terras
subordinadas. {[}2{]} Alguns alimentaram inimizades contra si mesmos;
alguns outros tentaram ganhar a confiança daqueles de quem ele
desconfiava no princípio do seu governo. {[}3{]} Alguns edificaram
fortalezas, alguns arruinaram"-nas e destruíram"-nas. {[}4{]} E ainda que
não se possa dar determinada sentença, senão se conhece as
particularidades daqueles governos nos quais se tivesse que tomar alguma
semelhante deliberação, todavia, eu falarei daquele modo geral que a
matéria por si mesma permite.

{[}5{]} Ora, jamais ocorreu que um príncipe novo desarmasse os seus
súditos, pelo contrário, quando os encontrou desarmados sempre os armou,
porque, quando você os armanda, aquelas armas tornam"-se suas, tornando
fiéis aqueles dos quais você suspeitava, e aqueles que eram fiéis
mantêm"-se assim, e de súditos, tornam"-se seus partidários. {[}6{]} E,
porque não se podem armar todos os súditos, enquanto aqueles que você
arma se beneficiam, em relação aos outros pode proceder sem maiores
considerações: e essa diversidade no proceder para com aqueles, que eles
reconhecem, torna"-os gratos a você; estes outros o desculpam, julgando
ser necessário ter mais méritos aqueles que estão sujeitos a maiores
perigos e obrigações. {[}7{]} Mas, quando você os desarma, começa a
contrariá"-los, mostra que não confia neles, seja por covardia, seja por
desconfiança, e uma e outra destas opiniões suscita o ódio contra você.
E como você não pode ficar desarmado, convém lhe recorrer à milícia
mercenária, que tem aquelas características das quais acima se
falou\footnote{Confira cap. \versal{XIII}, 5-6.}; e, ainda que esta fosse boa,
não pode ser tão boa que o defenda dos inimigos poderosos e dos súditos
suspeitos. {[}8{]} Por isso, como eu disse, um príncipe novo em um
principado novo sempre ordenou as suas armas e destes exemplos as
histórias estão repletas. {[}9{]} Mas, quando um príncipe conquista um
novo governo, que se agrega como membro ao seu antigo governo, então é
necessário desarmar este governo, exceto aqueles que, na sua conquista,
foram seus partidários; e, ainda aqueles, com o tempo e com as ocasiões,
é necessário torná"-los moles e efeminados, e ordená"-los de modo que, em
todo seu governo, as únicas armas que si mantenham sejam aquelas dos
seus próprios soldados, que, no seu governo antigo, vivem próximos de
ti.

{[}10{]} Costumavam os nossos antepassados\footnote{Os líderes políticos
  florentinos.}, e aqueles que eram estimados como sábios, dizer como
era necessário manter Pistóia com os partidários e Pisa com as
fortalezas, e por isto alimentavam, por toda a cidade que lhe estava
submetida, as diferenças entre os seus súditos, para possuí"-las mais
facilmente. {[}11{]} Isto, naqueles tempos em que a Itália estava de
certo modo equilibrada\footnote{Alude aos tempos de Lorenzo, O
  Magnífico, quando um delicado e eficaz equilíbrio político"-diplomático garantia a estabilidade da Itália, entre 1454 e 1492.},
era bem feito, mas não creio que se possa dar hoje este preceito, pois
não creio que as divisões nunca façam bem algum; antes, é forçoso,
quando o inimigo se aproxima, que as cidades divididas se percam logo,
porque sempre a parte mais débil aderirá às forças externas e a outra
parte não poderá governar. {[}12{]} Os venezianos, impelidos, como
creio, pelas razões supracitadas, alimentavam as facções
guélfas\footnote{Grupo político favorável ao Papa.} e
guibelinas\footnote{Grupo político favorável à república ou em algumas
  cidades favorável ao Império. As disputas entre os guelfos e os
  guibelinos marcaram os séculos \versal{XII} a \versal{XIV}, nas diversas cidades do
  norte da Itália.} nas cidades dominadas por eles; e ainda que não os
deixassem nunca chegar ao derramamento de sangue, todavia alimentavam
entre eles essas divergências, a fim de que, ocupados aqueles cidadãos
com seus conflitos, não se unissem contra eles. {[}13{]} O que, como se
vê, não correspondeu aos seus propósitos, porque, sendo derrotados em
Vailá\footnote{Cf. cap. \versal{XII}, 26.}, imediatamente um destes
partidos\footnote{No caso, os guibelinos que eram favoráveis ao império.}
começou a ousar e tirou deles todo o governo. {[}14{]} Provam, portanto,
tais costumes a fraqueza do príncipe, porque, em um principado forte,
nunca se permitirão semelhantes divisões, porque trazem vantagens apenas
em tempos de paz, podendo"-se mediante aquelas divisões mais facilmente
manipular os súditos, mas, chegando a guerra, semelhante ordenação
mostra a sua falácia\footnote{Modo de construção ou figura de lógica que
  redunda em falsidade.}.

{[}15{]} Sem dúvida, os príncipes tornam"-se grandes quando superam as
dificuldades e as oposições que lhe são feitas, e, por isso, a fortuna,
sobretudo quando deseja fazer grande um príncipe novo -- o qual tem
maior necessidade de conquistar a reputação do que um príncipe
hereditário --, lhe dá inimigos e os faz agir contra ele, a fim de que o
príncipe tenha a oportunidade de vencê"-los e, com aquela escada que os
seus inimigos lhe trouxeram, ele ascenda mais alto. {[}16{]} Porém,
muitos julgam que um príncipe sábio deve, quando tem a ocasião, com
astúcia alimentar algumas inimizades, a fim de que, oprimido por essas,
seja maior a sua grandeza.

{[}17{]} Têm os príncipes, e, sobretudo, aqueles que são novos,
encontrado mais fé e mais utilidade naqueles homens que, no princípio do
seu governo, eram considerados suspeitos, do que naqueles que no
princípio eram confiáveis. {[}18{]} Pandolfo Petrucci\footnote{Pandolfo
  Petrucci (1450-1512) apoderou"-se gradualmente de Siena a partir de
  1487, governando"-a com plenos poderes até a sua morte, em 1512.},
príncipe de Siena, governava mais com aqueles que lhe foram suspeitos do
que com os outros. {[}19{]} Mas, destas coisas, não se pode falar
longamente, porque elas variam segundo a matéria. Apenas direi isto, que
aqueles homens que no início de um principado tinham sido inimigos, e
que estão em uma condição que, para conservar"-se, precisam do apoio do
príncipe, este sempre com grandíssima facilidade poderá conquistá"-los. E
são eles mais ainda forçados a servi"-lo com fidelidade, quando sabem ser
mais necessário apagar com obras a má opinião que se tinha sobre eles.
{[}20{]} E assim o príncipe sempre obtém mais utilidade destes do que
daqueles que, servindo"-o com pouca consideração, descuidam das coisas do
príncipe.

{[}21{]} E porque o assunto\footnote{Na linha 19 acima, Maquiavel
  utilizou o termo \emph{subietto}, que nós traduzimos por matéria,
  tendo em vista sua vinculação a noção metafísica de substrato (cf.
  nota 58). Agora ele utiliza o termo \emph{materia}, que não possui o
  mesmo significado do nosso termo português `matéria', mas refere"-se ao
  assunto, ao tema a ser tratado.} exige, não quero deixar de lado a
lembrança dos príncipes que conquistaram um novo governo por meio do
favor dos cidadãos, que se considere bem que causa moveu aqueles que
favoreceram esse príncipe a favorecê"-los\footnote{Como destacam as
  edições Crítica e Comentada, neste período Maquiavel reitera a ideia
  de \emph{favor} que há entre as partes por meio da repetida utilização
  dos termos \emph{favori, favorito a favorirlo}. Tal redundância tem a
  função estilística de destacar como inicialmente se estabelece as
  relações de trocas políticas entre o príncipe novo e os seus novos
  dirigidos.}. {[}22{]} E se esta causa não é uma afeição natural para
com eles, mas somente porque não se contentavam com aquele governo, com
esforço e grande dificuldade se poderá conservá"-los amigos, porque é
impossível que se possa contentá"-los. {[}23{]} E discorrendo bem acerca
das razões disto, com aqueles exemplos que se encontram nas coisas
antigas e modernas, ver"-se"-á que é muito mais fácil conquistar como
amigos aqueles homens que se contentavam com o governo anterior, porém
eram seus inimigos\footnote{Inglese sugere que neste caso Maquiavel está
  citando o seu próprio exemplo em relação à família Médici, do qual foi
  inimigo político durante o governo de Soderini. Cf. Inglese, p. 143,
  nota 3.}, do que aqueles que, por não se contentarem com ele,
tornaram"-se amigos e favoreceram o príncipe para ocupá"-lo.

{[}24{]} É costume dos príncipes, para poder manter com mais segurança o
seu governo, edificar fortalezas que sejam o arreio e os freios daqueles
que desejassem atacá"-lo, e ter um refúgio seguro para um assalto
imprevisto. {[}25{]} Eu louvo este costume porque ele é comum desde a
Antiguidade, todavia, nos nossos tempos, viu"-se o Senhor Niccolò
Vitelli\footnote{Niccolò Vitelli (1414-1486), comandante militar da
  cidade de Castelo; apoderou"-se de sua cidade em 1462, estando longe do
  poder entre 1474-1482, quando retorna ao governo e permanece até a sua
  morte. A destruição da fortaleza de Castelo é de 1482.} derrubar duas
fortalezas na cidade de Castello para conservar aquele governo; Guido
Ubaldo\footnote{Guido Ubaldo de Montefeltro governou a cidade de Urbino
  de 1482 a 1508.}, duque de Urbino, retornando aos seus domínios, donde
foi expulso por César Borgia\footnote{Entre junho de 1502 a agosto de
  1503.}, destruiu por completo todas as fortalezas da sua província e
julgou que, sem aquelas, muito dificilmente\footnote{Maquiavel usa aqui
  \emph{piú difficilmente}, que traduzindo literalmente seria ``mais
  dificilmente'', ou seja, um superlativo de um advérbio de intensidade,
  que não possui nenhum correlativo em português. Aqui inserimos dois
  advérbios para conservar a mesma ideia do texto.} perderia de novo
aquele governo. O Bentivogli, de volta à Bolonha, usaram meios
semelhantes. {[}26{]} São, pois, as fortalezas úteis ou não, conforme os
tempos: e se fazem lhe bem em uma parte, prejudicam"-no em outra.
{[}27{]} E pode"-se assim discorrer sobre esta parte: aquele príncipe que
tem mais medo do povo que dos forasteiros deve fazer as fortalezas.
Contudo, aquele que tem mais medo dos forasteiros que do povo, deve
deixá"-las de lado. {[}28{]} À casa dos Sforza trouxe e trará mais
guerras\footnote{Não hesitando em vexar o povo, que não o apoiou na luta
  contra os franceses.} ao castelo de Milão, que Francisco Sforza aí
edificou\footnote{A construção do castelo durou de 1450 a 1572.}, que
qualquer outra desordem daquele governo. {[}29{]} Por isso, a melhor
fortaleza que existe é não ser odiado pelo povo, porque, ainda que você
tenhas as fortalezas, mas o povo o odeie, elas não o salvaram, porque
não falta nunca ao povo, uma vez que pegue em armas, forasteiros que lhe
ajudem. {[}30{]} Nos nossos tempos não se vê serem as fortalezas de
algum aproveito para príncipe algum, exceto para a condessa de
Forlí\footnote{Catarina Sforza (1463-1509), filha de Galeazzo Maria
  Sforza, casou"-se com Girolamo Riario, senhor de Imola e Forlí. Quando
  este foi assassinado em 1488, ela reagiu com grande coragem e resistiu
  ao ataques até a chegada de um exército milanês.}, quando morreu o
conde Jerônimo, seu marido, porque, por meio delas, ela pode fugir da
fúria popular, esperar o socorro de Milão e recuperar o governo. E os
tempos eram tais que os forasteiros não podiam socorrer o povo. {[}31{]}
Mas, depois pouco lhe valeram as fortalezas, quando César Borgia a
assaltou e o povo, inimigo dela, aliou"-se ao forasteiro. {[}32{]}
Portanto, nesta ocasião e antes, seria mais seguro a ela não ser odiada
pelo povo do que ter as fortalezas. {[}33{]} Considerando, portanto,
todas estas coisas, eu louvarei quem construir as fortalezas e quem não
as constrói; e censurarei todo aquele que, confiando nas fortalezas, dê
pouca atenção em ser odiado pelo povo.

\quebra\section{\emph{QUOD PRINCIPEM DECEAT UT EGREGIUS HABEATUR}
{[}O que convém a um príncipe para que seja estimado{]}}

{[}1{]} Nada faz um príncipe ser tão estimado quanto realizar grandes
feitos e dar raros exemplos de si. {[}2{]} Nós temos em nossos tempos
Fernando de Aragão\footnote{Fernando de Aragão, o católico (1452-1516),
  unificador dos reinos espanhóis no novo estado moderno Espanhol que
  veio a se transformar nos séculos seguintes em império colonial.},
atual rei de Espanha. A este se pode chamar quase um príncipe novo
porque, de um rei débil que era, tornou"-se, pela fama e pela glória, o
primeiro rei dos cristãos; e se se considerar as suas ações, as
encontraram todas grandíssimas e algumas extraordinárias. {[}3{]} Ele,
no princípio do seu reinado, tomou Granada\footnote{Conquista de
  Granada, 1481-1492.}, e aquela empresa foi o fundamento do seu
governo. {[}4{]} Primeiro, ele o fez tranquilamente e sem levantar
suspeitas para que não fosse impedido, tendo ocupado com isto os ânimos
daqueles barões de Castela, os quais, pensando naquela guerra, não
pensavam em inovações\footnote{Trata"-se aqui de mudanças políticas que
  pudessem ocupar a atenção de todos. Maquiavel também utiliza o termo
  com o sentido de fundação política, para se referir as novidades ou
  modificações no
  ordenamento político de uma cidade.}: e ele conquistava, por este
meio, reputação e poder sobre eles, que não se apercebiam; pode
sustentar o exército com o dinheiro da Igreja e do povo e, com aquela
longa guerra, dar o fundamento às suas milícias, as quais depois o
honraram. {[}5{]} Além disto, para poder empreender maiores feitos,
servindo"-se sempre da religião, voltou"-se a uma piedosa crueldade,
expulsando e espoliando, em seu reino, os mouros\footnote{Modo usual de
  se referir às populações de cultura árabe, tanto para aqueles que
  habitavam a Península Ibérica, quanto para os demais povos do norte da
  África e Península Arábica.}: não pode ser este exemplo mais miserável
e nem mais raro. {[}6{]} Assaltou, com este mesmo pretexto, a
África\footnote{Guerras feitas pelo rei Fernando no norte da África:
  conquista de Orano (1509), Bugia (1510) e Trípoli (1511).}; fez o
mesmo na Itália\footnote{Como relatado nos caps. \versal{III}, 39 e \versal{XII}, 1-4,
  Fernando foi protagonista na divisão do reino napolitano.};
ultimamente assaltou a França\footnote{Ataques ao território francês
  entre maio e dezembro de 1512 e conquista de Navarra. Essas afirmações
  são um forte indicativo da época de composição d'\emph{O Príncipe},
  entre os anos de 1513-1514.}. {[}7{]} E assim sempre fez e urdiu
grandes coisas, as quais sempre mantiveram elevada e cheio de admiração
o animo dos súditos, e ocupados no desenrolar destes eventos. {[}8{]} E
de tal modo estas ações nasceram uma das outras, que ele não deu nunca,
entre uma e outra, espaço aos homens para poderem calmamente agir contra
ele.

{[}9{]} Ademais, auxilia muito a um príncipe dar de si exemplos raros
quanto ao governo interno -- semelhante àqueles que se narram a respeito
do Senhor Bernabò de Milão\footnote{Bernabo Visconti (1323-1385).}--,
quando se tem a ocasião de encontrar alguém que faça alguma coisa
extraordinária na vida civil, ou para o bem ou para o mal, e escolher um
modo de como premiá"-lo ou puni"-lo, sobre o que há muito o que dizer.
{[}10{]} E, sobretudo, um príncipe deve esforçar"-se para obter para si,
em todas as suas ações, a fama de grande homem e de excelente
engenhosidade.

{[}11{]} Um príncipe é ainda estimado quando é verdadeiro amigo e
verdadeiro inimigo, isto é, quando, sem hesitação alguma, ele se revela
em favor de alguém contra outro. {[}12{]} Posição essa que é sempre mais
útil do que ficar neutro, porque, se dois poderosos vizinhos seus
começam a combater, ou eles são de tal qualidade que, vencendo um deles,
você tenha que temer o vencedor, ou não. {[}13{]} Em qualquer destes
dois casos, lhe será sempre mais útil decidir"-se por um dos lados e
declarar guerra aberta, porque, no primeiro caso, se não se decide, será
sempre presa daquele que vence, com prazer e satisfação daquele que foi
vencido; e não terá razão nem coisa alguma que o defenda nem que o
acolha, porque quem vence não deseja as amizades suspeitas e que não o
ajudaram na adversidade; quem perde, não o acolhe por não ter você, com
as armas em mãos, desejado compartilhar a sua fortuna.

{[}14{]} Tinha chegado à Grécia Antíoco\footnote{Cf. nota 30.}, movido
pelos etólios para expulsar os romanos; mandou ele embaixadores aos
aqueus, que eram amigos dos romanos, para exortá"-los a permanecerem
neutros: e de outra parte os romanos os persuadiram a pegar em armas por
eles. {[}15{]} Veio esta matéria para deliberação no concílio dos
aqueus, onde o legado de Antíoco os persuadiu a ficarem neutros, ao que
o legado romano respondeu: «Quod autem isti dicunt non interponendi vos
bello, nihil magis alienum rebus vestris est; sine gratia, sine
dignitate, praemium victoris eritis»\footnote{``Ora, isto que eles vos
  dizem -- de não vos interpor nesta guerra -- não pode ser mais
  contrário a vossos interesses: sem benevolência, sem dignidade, vós
  seríeis a recompensa do vencedor'', Tito Lívio, \emph{História de
  Roma}, l. \versal{XXXV}, cap. 49, 13. Citado de memória por Maquiavel, já que
  no original é: ``\emph{Nam quod optimum esse dicunt, non interponi vos
  bello, nihil immo tam alienum rebus vestris est: quippe'sine gratia,
  sine dignitate praemium victoris eritis.}''}. {[}16{]} E sempre
ocorrerá que aquele que não é amigo lhe pedirá a neutralidade, e aquele
que é seu amigo lhe pedirá para que se apresente com as armas. {[}17{]}
E os príncipes hesitantes, para fugir dos perigos presentes, seguem
muitas vezes aquela via da neutralidade, e muitas vezes arruínam"-se.

{[}18{]} Mas, quando o príncipe se revela bravamente em favor de uma
parte, se aquele a quem você se associas vence, ainda que ele seja
poderoso e que você permaneças à mercê dele, ele tem, em relação a você,
uma obrigação, estabelece"-se um contrato de confiança: e os homens nunca
são tão desonestos, com tamanho exemplo de ingratidão, a ponto de
oprimi"-lo; depois, as vitórias não são nunca assim tão completas que o
vencedor não tenha que ter alguma consideração, sobretudo em relação à
justiça. {[}19{]} Mas, se aquele com o qual você se associa perde, você
é acolhido por ele, e, enquanto puder, ele ajudará você, e você se torna
companheiro de uma fortuna que pode ressurgir.

{[}20{]} No segundo caso\footnote{Confira linha 12 deste capítulo.},
quando aqueles que combatem um contra o outro são de tal qualidade que
você não tem que temer aquele que vence, é tanto maior a prudência em
associar"-se a ele, porque você ajuda na ruína de alguém com a ajuda de
quem deveria salvá"-lo, se fosse sábio; vencendo, este permanece à sua
mercê, e é impossível que, com a sua ajuda, não vença. {[}21{]} Deve"-se
aqui notar que um príncipe deve cuidar para não fazer nunca campanha com
um mais poderoso do que si mesmo para prejudicar a outros, a não ser
quando a necessidade o constranja, como acima se disse, porque,
vencendo, fica prisioneiro dele, e os príncipes devem evitar o quanto
podem ficar ao capricho de outros. {[}22{]} Os venezianos se associaram
aos franceses contra o duque de Milão\footnote{Em 1499. Cf. cap. \versal{III},
  32.}, e podiam evitar fazer aquela campanha, da qual resultou a sua
ruína. {[}23{]} Mas quando não se pode evitá"-la -- como ocorreu aos
florentinos\footnote{Em 1511-1512.}, quando o papa e a Espanha, com seus
exércitos, foram assaltar a Lombardia --, então deve o príncipe aliar"-se
pelas razões sobreditas. {[}24{]} Nem creia nunca que estado algum possa
sempre tomar decisões seguras, ao contrário, pense que terá de tomar
sempre decisões duvidosas, porque isto está na ordem das coisas, que não
se pode nunca evitar um inconveniente sem incorrer em um outro, mas a
prudência consiste em saber conhecer os tipos de inconvenientes, e tomar
os menos ruins como bons.

{[}25{]} Deve, ainda, um príncipe mostrar"-se amante da \emph{virtù},
dando hospitalidade aos homens virtuosos e honrando aqueles que são
excelentes em uma arte. {[}26{]} Deve ainda animar os seus cidadãos a
poder exercitar calmamente as suas atividades: nos negócios, na
agricultura e em todos os outros afazeres dos homens; e que
este\footnote{Apesar de estar no singular, o pronome refere"-se aos
  homens do período anterior.} não tema em melhorar as suas propriedades
por temor de que lhe seja tirado, e aquele em abrir um comércio por medo
dos impostos. {[}27{]} Mas deve prever prêmios a quem deseje fazer estas
coisas e a qualquer um que pense, de algum modo, ampliar a sua cidade ou
o seu status. {[}28{]} Deve, além disto, nos tempos convenientes do ano,
manter o povo ocupado com as festas e os espetáculos, e, porque toda
cidade é dividida em artes ou em tribos\footnote{Tribos significa
  quarteirão ou bairros. As cidades romanas eram divididas em
  \emph{tribus}. Essa expressão se encontra também em Dante e na
  \emph{Arte da guerra} de Maquiavel.}, deve ter em conta essas
comunidades, reunir"-se com elas algumas vezes, dar de si exemplo de
humanidade e de magnanimidade; todavia, mantendo sempre firme a
majestade de sua dignidade, porque esta não pode faltar em parte
alguma\footnote{A frase final ``\emph{perchè questo non vuol mancare in
  cosa alcuna}'' não se faz presente na Edição Crítica
  de Inglese, embora ele faça referência a esta na nota (cf. p. 153, n.
  8). Contudo, conforme ele mesmo nos informa, essa frase está presente
  num número considerável de manuscritos -- os reunidos no grupo
  \emph{y} -- que, ainda que não componham o escopo de manuscritos
  principais, aqueles que congrega o grupo G e \emph{y}, entende"-se que
  tal sentença estava presente no autógrafo de \emph{O Príncipe}. Na
  Edição Comentada de Martelli essa frase está presente. Tendo em vista
  a presença da sentença em vários manuscritos e a sua completa
  adequação ao texto, decidimos inseri"-la aqui, ainda que ela não esteja
  na Edição Crítica, por nós adotada como edição de trabalho.}.

\quebra\section{\emph{DE HIS QUOS A SECRETIS PRINCIPES HABENT}
{[}Dos secretários que os príncipes têm{]}}

{[}1{]} Não é de pouca importância para um príncipe a escolha dos
ministros, os quais são bons ou não, segundo a prudência do príncipe.
{[}2{]} E a primeira conjectura que se faz da inteligência de um senhor
é ver os homens que ele tem em torno de si: quando eles são aptos e
fiéis, sempre se pode reputá"-lo como sábio, porque soube conhecê"-los
suficientemente e sabe conservá"-los fiéis. Mas, quando são de outro
modo, não se pode sempre fazer bom juízo dele, porque, ao primeiro erro
que o príncipe faz, o faz nesta escolha.

{[}3{]} Não havia ninguém que conhecesse o Senhor Antonio de
Venafro\footnote{Antonio da Venafro (1459-1530), professor do Studio
  (Colégio) de Siena, depois juiz, ministro e braço direito de Pandolfo
  Petrucci.}, ministro de Pandolfo Petrucci, príncipe de Siena, que não
julgasse Pandolfo homem valentíssimo, por aquele seu ministro. {[}4{]} E
como são de três gêneros de intelectos -- um entende por si, outro
discerne aquilo que outro entendeu, e o terceiro não entende nem por si
nem por outro, do qual o primeiro é excelentíssimo, o segundo excelente
e o terceiro inútil --, portanto, convinha pela necessidade que, se
Pandolfo não era do primeiro grau, que fosse do segundo. {[}5{]} Porque
toda vez que alguém tem discernimento suficiente para conhecer o bem ou
o mal que outro alguém faz ou diz, ainda que por si não tenha um
intelecto inventivo, conhece as obras más e boas do ministro, a estas
exalta e àquelas corrige, e o ministro não pode esperar enganá"-lo e
manter"-se bem.

{[}6{]} Mas, para que um príncipe possa conhecer o ministro, ele deve
fazer isto, que não falha nunca: quando você vê o ministro pensar mais
em si que em ti, e procura em todas as suas ações aquilo que lhe é mais
útil, este, que assim é, nunca será um bom ministro e nunca poderás
confiar nele. {[}7{]} Porque aquele que tem o governo de alguém nas
mãos, não deve pensar nunca em si, mas sempre no príncipe, e nunca
lembrar de coisas que não pertençam ao príncipe: e, de outro lado, o
príncipe, para bem conservá"-lo, deve pensar no ministro, honrando"-lhe,
fazendo"-o rico, garantindo"-lhe o cargo, atribuindo"-lhe as honras e os
cargos\footnote{Cargos não somente no sentido de posto, mas de
  responsabilidades na condução do governo.}: a fim de que veja que não
pode estar sem ele, e que as muitas honras não lhe façam desejar mais
honras, as muitas riquezas não lhe façam desejar mais riquezas, os
muitos cargos lhe façam temer as mudanças. {[}8{]} Quando, pois, os
ministros, e os príncipes em relação aos ministros, são deste feitio,
podem confiar um no outro: quando são de outro modo, sempre o fim será
danoso ou para um ou para outro.

\quebra\section{\emph{QUOMODO ADULATORES SINT FUGIENDI}
{[}De que modo se deve evitar os aduladores{]}}

{[}1{]} Não quero deixar de lado um assunto importante e um erro do qual
os príncipes com dificuldade se defendem, se não são prudentíssimos ou
se não fazem boa escolha. {[}2{]} E estes são os aduladores, dos quais
as cortes estão cheias, porque os homens se comprazem tanto nas coisas
deles próprios, e de tal modo aí se enganam, que com dificuldade se
defendem desta peste. {[}3{]} E, para querer se defender, correm o risco
de tornarem"-se desprezíveis, porque não há outro modo de proteger"-se das
adulações que não seja os homens entenderem que não o ofendem quando
dizem"-lhe a verdade; mas, quando todo mundo pode dizer"-lhe a verdade,
falta"-lhe a reverência. {[}4{]} Portanto, um príncipe prudente deve ter
um terceiro modo, escolhendo no seu governo homens sábios e apenas a
estes deve dar liberdade para dizer"-lhe a verdade, e apenas sobre
aquelas coisas que ele lhes perguntar e não de outras -- mas deve
perguntar"-lhes sobre todas as coisas --, e escutar as suas opiniões,
depois deliberar por si, a seu modo. {[}5{]} Com estes conselhos e, com
cada um deles, portar"-se de modo que todos saibam que, quanto mais
livremente se falar, mais será aceito; fora aqueles sábios, não queira
ouvir ninguém, siga a deliberação tomada e seja obstinado nas suas
deliberações. {[}6{]} Quem faz de outro modo, ou cai por causa dos
aduladores ou frequentemente muda sua decisão pelas variações dos
pareceres, o que leva a ser pouco estimado.

{[}7{]} Eu desejo a este propósito apresentar um exemplo moderno. O
padre Luca\footnote{Luca Rinaldi, bispo de Trieste, foi em várias
  ocasiões embaixador do imperador alemão Maximiliano.}, homem do atual
imperador Maximiliano\footnote{Massimiliano d'Asburgo (1459-1519),
  eleito rei do Sacro Império Germanico em 1486, assumiu plenamente o
  título imperial em 1508.}, falando de Sua Majestade, disse como ele
não se aconselhava com as pessoas e não fazia nunca coisa alguma a seu
modo. {[}8{]} O que resultava de ter contrariado os conselhos
supracitados, porque o imperador é homem reservado, não comunica os seus
planos, não pede parecer: mas, quando os coloca em ação, começa"-se a
conhecê"-los e a descobri"-los, começam a ser contestados por aqueles que
ele tem em entorno de si, e o imperador, que facilmente muda de opinião,
afasta"-se de seus planos; daqui decorre que aquelas coisas que faz num
dia, destrói no outro, e que não seja nunca compreendido naquilo que se
deseja ou planeja fazer, e que não se pode fundar sobre as suas
deliberações.

{[}9{]} Um príncipe, portanto, deve aconselhar"-se sempre, mas quando ele
deseje e não quando o outro deseja; antes, deve desencorajar qualquer um
que queira aconselhá"-lo sobre alguma coisa caso ele não lhe tenha
perguntado, mas ele deve ser assíduo inquiridor, e depois, sobre as
coisas perguntadas, pacientemente ouvir a verdade: aliás, penso que se
alguém, por algum respeito não lhe fale a verdade, ele deve zangar"-se.
{[}10{]} E os muitos que avaliam que certo príncipe, o qual cria para si
a imagem de prudente, seja assim considerado não por sua natureza, mas
pelos bons conselhos que tem em torno de si sem dúvida se enganam.
{[}11{]} Porque esta é uma regra geral que não falha nunca: que um
príncipe, o qual não seja sábio por si mesmo, não pode ser bem
aconselhado, a não ser que tenha tido a sorte de poder confiar"-se a um
homem prudentíssimo, que pudesse a tudo governar. {[}12{]} Neste caso,
poderia ser bem aconselhado, mas duraria pouco, porque esse governador
em breve tempo lhe tomaria o governo. {[}13{]} Mas, aconselhando"-se com
mais de um, um príncipe que não seja sábio não terá nunca os conselhos
concordes, não saberá por si mesmo uni"-los, cada um dos conselheiros
pensará no seu interesse; ele não saberá corrigi"-los nem distingui"-los e
não se podem encontrar de outro modo, porque os homens sempre se mostram
maus, se por uma necessidade não se tornam bons. {[}14{]} Por isso,
conclui se que os bons conselhos, não importa de quem provenham, convém
que nasçam da prudência do príncipe e não a prudência do príncipe dos
bons conselhos.

\quebra\section{\emph{CUR ITALIAE PRINCIPES REGNUN AMISERUNT}
{[}Por que os príncipes da Itália perderam os seus reinos{]}}

{[}1{]} Observadas prudentemente, as sobreditas coisas fazem parecer
antigo\footnote{Aqui antigo com o sentido de hereditário, como se verá
  adiante.} um príncipe novo e o tornam imediatamente mais seguro e mais
firme no governo do que se nele fosse antigo. {[}2{]} Porque um príncipe
novo é muito mais observado nas suas ações do que um hereditário, e
quando são conhecidas as suas virtudes, conquista mais ainda os homens e
muito mais se obrigam a ele do que a uma antiga dinastia. {[}3{]} Porque
os homens são muito mais presos às coisas presentes do que às coisas
passadas, e, quando nas presentes encontram o bem, se regozijam e não
procuram outra, antes, o defenderão de toda maneira, quando nas outras
coisas o príncipe não falta a si mesmo. {[}4{]} E assim terá dupla
glória, de ter dado início a um principado, tê"-lo honrado e fortalecido
com boas leis, com boas armas e com bons exemplos, assim como tem dupla
vergonha aquele que, tendo nascido príncipe, perdeu o governo por sua
pouca prudência.

{[}5{]} E se se considera aqueles senhores que, na Itália, perderam o
governo em nossos tempos, como o rei de Nápoles\footnote{Confira cap.
  \versal{III}, 39.}, o duque de Milão\footnote{Confira cap. \versal{III}, 4.} e outros,
se encontrarão neles, primeiro, um defeito comum quanto às armas, pelas
razões que longamente foram discutidas acima; depois, se verá que alguns
deles ou tiveram por inimigo o povo ou, se tiveram o povo como amigo,
não souberam assegurar o apoio dos grandes. {[}6{]} Porque, sem estes
defeitos, não se perdem os governos que têm tanta força que podem manter
um exército em campanha. {[}7{]} Felipe da Macedônia\footnote{Confira
  cap. \versal{III}, 21.}, não o pai de Alexandre, mas aquele que foi vencido por
Tito Quinto, não tinha um governo muito forte em relação à grandeza dos
romanos e dos gregos que o assaltaram; todavia, por ser homem de guerra
e que sabia manter o povo consigo e assegurar"-se dos grandes, manteve
por anos a guerra contra aqueles, e, se ao final perdeu o domínio de
algumas cidades, restou"-lhe, contudo, o reino.

{[}8{]} Portanto, que estes nossos príncipes que estavam há muitos anos
nos seus principados, por tê"-los perdido depois, não acusem a fortuna,
mas a sua ignávia, porque, não tendo nunca, nos tempos de paz, pensado
que poderiam mudar -- o que é um defeito comum dos homens, não levar em
conta, na bonança, a tempestade --, quando depois vieram os tempos
adversos, pensaram em fugir e não em se defender, e esperaram que o
povo, insatisfeito com a insolência dos vencedores, os chamasse de
volta. {[}9{]} Esta decisão, quando não há outras, é boa, mas é muito
ruim ter deixado os outros remédios por este, porque nunca se deve
desejar cair, por acreditar que encontrará quem o acolha. {[}10{]} O que
ou não acontece, ou, se acontece, não é seguro para você, por ser esta
defesa vil e não depender de você. E somente aquelas defesas que
dependem de você e de sua própria \emph{virtù} são boas, são certas e
são duráveis.

\quebra\section{\emph{QUANTUM FORTUNA IN REBUS HUMANIS POSSIT ET QUOMODO ILLI SIT OCCURRENDU}
{[}Quanto pode a fortuna nas coisas humanas e de que modo se deve resistir a ela{]}}

{[}1{]} E não desconheço como muitos\footnote{Muitos aqui, conforme os
  editores (Inglese e Martelli), podem ser deste os antigos -- Cícero,
  Teofrasto, Salustio -- como os italianos do renascimento --
  particularmente Dante e Boccacio. De toda forma, é consenso que se
  tratava antes de um provérbio popular ao tempo de Maquiavel,
  largamente difundido entre os florentinos, donde ser essa a origem
  principal.} tiveram e têm opiniões que as coisas do mundo são, de
certo modo, governadas pela fortuna e por Deus; que os homens com a sua
prudência não podem corrigi"-las, não havendo, então, remédio algum; e,
por isto, poderiam julgar que não seria necessário cansar"-se muito
nessas coisas, mas deixar"-se governar pela sorte\footnote{Importa
  recordar que fortuna não tem o mesmo signficado de sorte, como para
  nossa sociedade
  lusófona. Como apresentará Maquiavel, a fortuna está ligada a uma
  concepção cosmológica que ultrapassa a ideia de mero acaso central no
  conceito de sorte, donde o equívoco de se traduzir \emph{fortuna} (em
  italiano) por \emph{sorte} e vice"-versa. Convém ainda destacar que
  esse é o capítulo do texto com maiores referências à noções
  cosmológicas e metafísicas do livro, haja vista que o autor mobiliza
  conceitos como \emph{natureza, livre arbítrio, tempo, providência}
  etc.}. {[}2{]} Esta opinião tem muito crédito nos nossos tempos por
causa da grande mudança nas situações que foram vistas e se vêem todos
os dias, que estão além de toda conjectura humana. {[}3{]} Diante do
que, pensando eu algumas vezes, inclinei"-me de certo modo pela opinião
deles\footnote{Acredita"-se que Maquiavel esteja fazendo remissão aos
  seus escritos \emph{Ghiribizi al Soderini} e \emph{Capitolo}
  \emph{di Fortuna}, ambos escritos em 1506.}. {[}4{]} Todavia, porque o
nosso livre arbítrio não esteja extinto, julgo ser verdadeiro que a
fortuna seja árbitra de metade das nossas ações, mas que ela ainda nos
deixa governar a outra metade, ou quase. {[}5{]} E comparo a fortuna a
um desses rios ruinosos que, quando se iram, alagam as planícies,
arruínam as árvores e os edifícios, levam terra desta parte, põem"-na em
outro lugar: qualquer um foge em sua presença, todos cedem ao seu ímpeto
sem poder impedi"-lo de modo algum. {[}6{]} E, ainda que sejam assim, aos
homens nada impede que, quando os tempos estão calmos, tomem
providências, com proteções e com diques: de modo que, ao se avolumarem
depois, ou iriam por um canal ou o seu ímpeto não seria nem tão violento
nem tão danoso. {[}7{]} Ocorre o mesmo à fortuna, a qual demonstra o seu
poder onde a \emph{virtù} não é ordenada para resisti"-la, e então volta
o seu ímpeto para onde ela sabe que não se fizeram os diques e as
proteções para contê"-la. {[}8{]} E se vocês considerem a Itália, que é o
palco\footnote{Maquiavel usa aqui o termo \emph{sedia}, com o mesmo
  sentido de lugar da cena teatral. Não devemos nos esquecer que em vida
  ele foi mais reconhecido como escritor de peças teatrais de sucesso,
  como \emph{Mandragora}. Essa dimensão artística percorre quase todos
  os seus escritos em vários momentos.} destas mudanças e que lhes deu o
movimento inicial, veriam ser ela uma planície sem diques e sem nenhuma
proteção, que, se ela fosse reparada pela conveniente \emph{virtù}, como
é a Alemanha, a Espanha e a França, ou estas planícies não teriam
sofrido as grandes variações que sofreram, ou elas não teriam lá
chegado. {[}9{]} E creio que basta ter dito isto, de modo geral, quanto
ao opor"-se à fortuna.

{[}10{]} Mas, restringindo"-me mais ao particular, digo como se vê hoje
um príncipe prosperar e amanhã arruinar"-se, sem tê"-lo visto mudar alguma
natureza ou atributo, o que, creio, decorra primeiro, das razões que
foram longamente discutidas antes\footnote{Confira cap. \versal{VII}.}, isto é,
que aquele príncipe que se apóia totalmente na sua fortuna, cai quando
ela muda. {[}11{]} Creio, ainda, que seja feliz aquele que conforma o
seu modo de proceder com os atributos do tempo, do mesmo modo creio
infeliz aquele que cujo proceder diverge do tempo. {[}12{]} Porque se vê
os homens, nas coisas que lhes conduzem ao fim ao qual se propunham,
isto é glórias e riquezas, procederem diferentemente: um com respeito,
outro com ímpeto; um com violência, outro com arte\footnote{Engenhosidade,
  criatividade.}; um pela paciência, outro com o seu contrário; e cada
um pode com estes diversos modos alcançar o que deseja. {[}13{]} Vê"-se
ainda dois homens prudentes, um chegar a seu intento, o outro não; e
semelhantemente dois igualmente prosperarem com diferentes modos de
agir, sendo um respeitoso e outro impetuoso, o que não decorre de outra
coisa, senão, de um atributo dos tempos que se conformam, ou não, ao seu
proceder. {[}14{]} Daqui advém aquilo que eu disse, que dois, agindo
diferentemente, chegam ao mesmo efeito; e dois igualmente agindo, um
alcança seu fim e o outro não. {[}15{]} Disto dependem ainda as
variações do que se considera bem, porque, se para alguém que governa
com respeito e paciência, os tempos e as coisas giram de modo que o seu
governo seja bom, ele prospera, mas, se o tempo e as coisas mudam, ele
se arruína, porque não muda o seu modo de proceder. {[]}16{]} Nem se encontra homem assim tão prudente que saiba se acomodar a isto: seja porque não pode desviar"-se daquilo que a natureza o inclina, seja ainda porque, tendo alguém sempre prosperado, caminhando por uma estrada, não se pode persuadi"-lo a deixá"-la. {[}17{]} E, assim, o homem ponderado, quando é tempo de agir com ímpeto, não sabe fazê"-lo, por isso se arruína; pois, se se mudasse a natureza com os tempos e as coisas, não se mudaria a fortuna.

{[}18{]} O papa Júlio \versal{II} procedeu em todas as suas coisas impetuosamente
e encontrou tanto o tempo como as coisas de acordo com aquele seu modo
de proceder, que sempre teve bom êxito. {[}19{]} Observe"-se primeiro a
ação que empreendeu em Bolonha\footnote{No verão de 1506 o papa atacou
  Perugia, e como essa cidade era um território bolonhês, logo ele
  atacou Bolonha. Em 11 de novembro deste ano, ele fez sua entrada
  vitoriosa na cidade.}, vivendo ainda o Senhor Giovanni Bentivoglio.
{[}20{]} Os venezianos não o viam com bons olhos; o rei de Espanha, o
mesmo; com a França, estava em tratativas sobre tal ação. Todavia, ele,
com a sua ferocidade e ímpeto, moveu pessoalmente aquela expedição.
{[}21{]} Movimento que deixou perplexos e imóveis a Espanha e os
venezianos, estes por medo, aquela pela vontade que tinha de recuperar
todo o reino de Nápoles; e, por outro lado, o Papa arrastou consigo o
rei de França, porque, vendo este aquele rei em marcha e desejando
fazê"-lo amigo para subjugar os venezianos, julgou não poder lhe negar os
seus exércitos sem injuriá"-lo abertamente. {[}22{]} Fez, portanto,
Júlio, com seu movimento impetuoso, aquilo que nunca outro pontífice,
com toda prudência humana, teria feito. {[}23{]} Porque, se ele tivesse
esperado para partir de Roma com os acordos firmados e todas as coisas
ordenadas, como qualquer outro pontífice teria feito, nunca teria
conseguido, porque o rei de França teria tido mil desculpas e os outros
incutido"-lhe mil temores. {[}24{]} Eu não vou tratar das outras ações
suas, que foram todas semelhantes e todas tiveram bons resultados, e a
brevidade da vida não lhe deixou sentir o contrário, porque, se
sobreviessem tempos nos quais fosse necessário proceder com prudência,
daí seria a sua ruína, pois nunca se teria desviado daqueles costumes
aos quais a natureza o inclinava.

{[}25{]} Concluo, portanto, que, mudando a fortuna, e permanecendo os
homens obstinados nos seus costumes, são felizes enquanto a fortuna e os
costumes concordam e, quando discordam, são infelizes. {[}26{]} Acredito
que seja melhor ser impetuoso que ponderado, porque a fortuna é mulher e
é necessário, se se a quer, subjugá"-la, submetê"-la e bater
nela\footnote{Conforme Inglese (p. 167, nota 2), a expressão
  ``\emph{batterla e urtala''} equivale a \emph{``te"-la submetida}'' ou
  ``\emph{possuí"-la carnalmente}'', que se adéqua bem melhor ao
  raciocínio. Essa metáfora de possessão carnal da mulher é mais
  conforme ao argumento, tendo em vista que no imaginário feminino
  sugerido por Maquiavel aqui, a fortuna como mulher deseja ser possuída
  carnalmente pelo jovem impetuoso, mais do que ser agredida por ele tão
  somente.}. {[}27{]} E se vê que ela se deixa mais vencer pelos
impetuosos do que por aqueles que friamente procedem, e, por isso, como
é mulher, sempre é amiga dos jovens, porque são menos prudentes, mais
ferozes e comandam"-na com mais audácia.

\quebra\section{\emph{EXORTATIO AD CAPESSENDAM ITALIAM IN LIBERTATEMQUE A BARBARIS VINDICANDAM}
{[}Exortação para tomar a defesa da Itália e libertá"-la das mãos dos
bárbaros\protect\footnote{\uppercase{C}onvém lembrar que o termo bárbaro possuia a acepção
  de povo estrangeiro invasor que não possuía os costumes e,
  principalmente, a língua. No caso da Itália do Renascimento, haja
  vista o histórico de invasões de exércitos estrangeiros, conforme
  relatado aqui no \emph{Príncipe}, o título propõe diretamente o fim
  dessas invasões e a constituição de um estado italiano dominado a
  península itálica.}{]}}

{[}1{]} Consideradas, pois, todas as coisas discutidas acima, e pensando
comigo mesmo se na Itália do presente são tempos propícios para honrar
um novo príncipe, e se houvesse matéria que desse ocasião para que
alguém prudente e virtuoso pudesse aí introduzir a forma\footnote{``Introduzir
  a forma'' é uma expressão oriunda das metafísicas clássicas,
  principalmente o aristotelismo, cuja ideia principal remonta a noção
  de que todas as coisas (substâncias) são possuem uma matéria e uma
  forma que se unem para formar o ente ou o composto substancial. No
  caso, o povo é a matéria (\emph{hipokeimenon}) na qual o regime
  confere a forma, aqui em particular, essa é a tarefa do príncipe,
  conferir um regime ou uma forma ao povo. Maquiavel está retomando o
  que foi dito no cap. \versal{VI}, 10; cf. também nota 69.}, e que o honrasse e
fizesse o bem a todos os homens da Itália, me parece que tantas coisas
concorrem para benefício de um príncipe novo, que eu não saberia qual
tempo seria mais apto para isto. {[}2{]} E se, como eu disse, era
necessário, para se ver a \emph{virtù} de Moisés, que o povo de Israel
fosse escravo no Egito; e para se conhecer a grandeza de ânimos de Ciro,
que os persas fossem oprimidos pelos medos; e para se conhecer a
excelência de Teseu, que os atenienses estivesse dispersos; {[}3{]}
assim no presente, desejando"-se conhecer a \emph{virtù} de um espírito
italiano, era necessário que a Itália se reduzisse às condições
presentes, e que ela fosse mais escrava que os hebreus, mais serva que
os persas, mais dispersa que os atenienses: sem chefe, sem ordem,
abatida, espoliada, dilacerada, invadida e tivesse suportado toda sorte
de ruína.

{[}4{]} E se bem que até aqui já tenha aparecido, em alguns, sinal que
pudesse levar a julgar que alguém fosse ordenado por Deus para a sua
redenção, todavia se viu como depois foi reprovado pela fortuna, no
momento mais alto das suas ações\footnote{A referência aqui é para César
  Bórgia, que morreu sem cumprir seu desejo maior de governar toda a
  Itália.}. {[}5{]} De modo que a Itália jaz como sem vida, espera
aquele que possa curar as suas feridas e ponha fim aos saques da
Lombardia, aos impostos no reino de Nápoles e na Toscana, e a cure
daquelas suas chagas, já por um longo tempo supuradas. {[}6{]} Vê"-se
como ela roga a Deus que lhe mande alguém que a redima desta crueldade e
insolência bárbaras. {[}7{]} Vê"-se ainda toda pronta e disposta a seguir
uma bandeira, desde que haja alguém que a empunhe. {[}8{]} Nem se vê, no
presente, em que ela possa mais esperar senão na sua ilustre
Casa\footnote{Referência direta a família Médici.}, a qual -- com a sua
fortuna e \emph{virtù}, favorecida por Deus e pela Igreja, da qual é
agora o príncipe\footnote{No caso, o papa Leão \versal{X} (Giovanni di Médici).}
-- pode tornar"-se chefe desta redenção. {[}9{]} O que não seria muito
difícil, se mantiver à vista diante de você as ações e a vida dos acima
nomeados\footnote{Moisés, Ciro e Teseu.}, e ainda que aqueles homens
sejam raros e maravilhosos, todavia foram homens, e tiveram cada um
deles menores ocasiões do que a presente, porque o feito deles não foi
mais justo do que este, nem mais fácil, nem foi Deus mais amigo deles do
que do senhor. {[}10{]} Aqui é grande justiça: «iustum enim est bellum
quibus necessarium, et pia arma ubi nulla nisi in armis spes
est».\footnote{``Justas, pois, são as guerras necessárias, e piedosas
  são as armas quando só nelas há esperança.'' Tito Lívio,
  \emph{História}, \versal{L. IX}, cap. 1, citado de memória por Maquiavel, já
  que o correto é ``\emph{Iustum est bellum, Samnites, quibus
  necessarium, et pia arma, quibus nulla nisi in armis rilinquitur
  spes}.''} {[}11{]} Aqui há grandíssima disposição, e não pode pode
haver, onde há grande disposição, grande dificuldade, desde que esta
casa siga os ordenamentos daqueles que eu tenho proposto como exemplo.
{[}12{]} Além disto, no exemplo de Moisés se vêem exemplos
extraordinários, conduzidos por Deus: o mar que se abriu; uma nuvem
escoltou"-o pelo caminho; da pedra jorrou água; aqui choveu o maná. Todas
as coisas concorreram para sua grandeza. {[}13{]} O restante o senhor
deve fazer, Deus não deseja fazer tudo, para não tirar o livre arbítrio
e parte daquela glória que nos cabe.

{[}14{]} E não é surpreendente se nenhum dos pré"-nomeados italianos
puder fazer aquilo que se pode esperar que faça a sua ilustre Casa, e
se, em tantas revoluções na Itália e em tantas manobras de guerra,
pareça sempre que, na Itália, a \emph{virtù} militar foi extinta, isto
ocorre as porque suas antigas ordens não eram boas, e não houve ninguém
que tivesse sabido encontrar novas. {[}15{]} E nenhuma coisa dá tanta
honra a um homem novo\footnote{No caso aqui significa o príncipe novo
  que chega ao poder.}\textsuperscript{348} que ascende quanto fazer
novas leis e novas ordens: estas coisas, quando são bem fundadas e há
nelas grandeza, tornam"-no reverenciado e admirado. {[}16{]} E na Itália
não falta matéria para introduzir qualquer forma: aqui é grande a
\emph{virtù} nos membros, quando ela não falta nos chefes. {[}17{]}
Observe os duelos e os combates de poucos, quanto os italianos são
superiores na força, na destreza e na astúcia; mas, quando se trata de
exércitos, não têm sucesso. {[}18{]} E tudo se origina da debilidade dos
chefes: porque aqueles que sabem não são obedecidos e qualquer um parece
saber, não havendo por aqui ninguém que tenha sabido sobressair"-se pela
\emph{virtù} e pela fortuna, ao qual os outros cedam.

{[}19{]} Daqui advém que, em tanto tempo, com tantas guerras feitas nos
últimos vinte anos, quando houve um exército todo italiano, ele sempre
foi mal, do que é testemunha primeiro o Taro\footnote{Batalha de Fornovo
  sobre Taro (6 de julho de 1495), entre os franceses e a liga de
  estados italianos.}, depois Alexandria\footnote{Batalha de Alexandria
  (28 de agosto de 1499), novamente entre os franceses e os italianos.},
Cápua\footnote{Saque de Cápua pelos franceses (24 de julho de 1501).},
Gênova\footnote{Sufocamento da rebelião de Gênova contra os franceses
  (28 de abril de 1507).}, Vailá\footnote{Confira cap. \versal{XII}, 26.},
Bolonha\footnote{Abandono da defesa de Bolonha (20 de maio de 1511).},
Mestre\footnote{Incêndio da cidade de Mestre e batalha de Vicenza (7 de
  outubro de 1513).}.

{[}20{]} Querendo, portanto, a sua ilustre Casa seguir aqueles
excelentes homens que redimiram as suas províncias, é necessário, antes
de todas as outras coisas, como verdadeiro fundamento de qualquer
empresa, prover"-se de armas próprias, porque não se pode ter nem mais
fiéis, nem mais verdadeiros, nem melhores soldados: e se bem que cada um
destes soldados seja bom, todos juntos tornam"-se melhores quando se vêem
comandar por seu príncipe e por ele serem honrados e bem tratados.
{[}21{]} É necessário, portanto, preparar"-se para estas armas, para
poder com a \emph{virtù} itálica defender"-se dos estrangeiros. {[}22{]}
E ainda que a infantaria suíça e a espanhola sejam consideradas
terríveis, contudo, em ambas há defeitos para o qual uma terceira forma
de organização militar poderia não somente opor"-se a elas, mas,
confiante, superá"-las. {[}23{]} Porque os espanhóis não podem resistir a
uma carga de cavalaria e os suíços têm medo dos infantes quando os
encontram no combate, obstinados com eles: donde se vê e verá, na
prática, os espanhóis não poderem suportar um ataque da cavalaria
francesa e os suíços serem derrotados pela infantaria espanhola.
{[}24{]} E, ainda que desta última não se tenha tido uma prova completa,
todavia se viu um ensaio na batalha de Ravena\footnote{Confira cap. \versal{III},
  6.}, quando as infantarias espanholas se defrontaram com os batalhões
alemães, os quais se utilizaram das mesmas ordenações dos suíços, e na
qual os espanhóis, com a agilidade do corpo e a ajuda de seus
escudos\footnote{No original, \emph{brocchieri,} pequenos escudos
  redondos munidos no centro de uma ponta grossa que servia como arma de
  defesa e ataque.}, entraram sozinhos entre os lanceiros, e estavam
seguros para atacá"-los sem que os alemães tivessem como se defender; e
se não fosse a cavalaria, que os atacou, os espanhóis teriam feridos e
matado todos. {[}25{]} Pode"-se, portanto, conhecendo o defeito de uma e
de outra destas infantarias, criar uma nova, que resista à cavalaria e
não tenha medo dos infantes: e que será fruto das armas e das mudanças
das ordenações; e estas são daquelas coisas que, novamente ordenadas,
dão reputação e grandeza a um príncipe novo.

{[}26{]} Não se deve, pois, deixar passar esta ocasião, a fim de que a
Itália, depois de tanto tempo, veja aparecer um seu redentor. {[}27{]}
Nem posso exprimir com qual amor ele seria recebido em todas as
províncias que têm sofrido por causa destas invasões estrangeiras, com
que sede de vingança, com que obstinada confiança, com que piedade, com
que lágrimas. {[}28{]} Quais portas se lhe fechariam? Quais povos lhe
negariam obediência? Que invejas se lhe oporiam? Que italiano lhe
negaria o serviço? A todos fede este bárbaro domínio. {[}29{]} Tome,
portanto, a sua ilustre Casa este assunto com aquele ânimo e aquela
esperança com que se tomam as façanhas justas, a fim de que, sob o seu
estandarte, esta pátria seja enobrecida e, sob os seus auspícios, se
realize aquele dito de Petrarca:

\begin{verse}
\emph{Virtù} contra o furor\\
Tomará as armas, e que seja breve o combate\\
Que o antigo valor\\
Nos corações italianos não está ainda morto.\\
(Italia mia, canzoniere, \versal{CXXVIII}, 93-96)
\end{verse}} %


\renewcommand{\ParallelAtEnd}{\noindent{}\vspace{1cm}\Large{Notas}}
\end{Parallel} 

\chapter*{Bibliografia}

\section{Obras de Maquiavel}

MACHIAVELLI, Niccolò. \emph{De Principatibus}. Testo critico a cura di
G. Inglese. Roma: Istituto Storico Italiano per il Medio Evo, 1994;

MACHIAVELLI, Niccolò. \emph{Il Principi.} Opere. V. 1. A cura di Corrado
Vivanti. Torino: Einaud/Gallimard, 1997.

MACHIAVELLI, Niccolò. \emph{Il Principi}. Edizione Nazionale Delle Opere
-- I/1, a cura de Mario Martelli. Roma: Salerno Editrice, 2006;

MACHIAVELLI, Niccolò. \emph{Discorsi sopra la prima deca di Tito Lívio.}
Introduzione di Gennaro Sasso, premessa al testo e note di Giorgio
Inglese. Milano: Rizzoli, 1984;

MACHIAVELLI, Niccolò. \emph{Discursus Florentinarum rerum post mortem
iunioris Laurentii Medices}. Edizione Nazionale Delle Opere -- I/3, a
cura de Jean-Jacques Marchand, Denis Fachard e Giorgio Masi. Roma:
Salerno Editrice, 2001;

MACHIAVELLI, Nicolau. \emph{Discursos sobre as formas de governo de
Florença}. Introdução, tradução e notas de Gabriel Pancera. Belo
Horizonte: ed. UFMG, 2010.

MACHIAVELLI, Nicolau. \emph{Discursos sobre a primeira década de Tito
Lívio.} Trad. Martins Fontes. São Paulo: Martins Fontes, 2007;

MACHIAVELLI, Nicolau. \emph{História de Florença}. Trad. e notas de
Nelson Canabarro. São Paulo: Musa, 1994;

\section{Fontes primárias}

ARISTÓTELES. \emph{Política}. {[}edição bilíngüe{]} Trad. Antônio C.
Amaral e Carlos C. Gomes. Lisboa: Vega, 1998;

CICERO, Marco Tulio. \emph{I Doveri}. {[}edição bilingue{]}. Traduzione
di Anna Resta Berrile. Milano: BUR, 2004;

CICERO, Marco Tulio. \emph{La Repubblica.} {[}edição bilingue{]}. A cura
de Francesca Nenci. Milano: BUR, 2008;

DANTE Aliguieri. \emph{A Divina Comédia.} Paraíso. {[}Edição
bilingue{]}. Tradução e notas de Italo Eugenio Moro. São Paulo: Ed. 34
Letras, 1998.

POLÍBIO. \emph{Storie}. Libri V-VI, {[}edição bilingue{]}. A cura di
Domenico Musti, traduzione di Manuela Mari, note di John Thornton. Vol.
III. Milano: BUR, 2002;

TITO Lívio. \emph{História de Roma.} Trad. Paulo Matos Peixoto. São
Paulo: Ed. Paumape, 1989. {[}6 vol.{]}

\section{Fontes secundárias}

ADVERSE, Helton. \emph{Maquiavel, a República e o Desejo de Liberdade.}
Marília: Revista Trans/Form/Ação, v. 30, p. 33-52, 2007;

%\href{http://lattes.cnpq.br/3325441375860351}{ADVERSE, Helton}.
\emph{Maquiavel. Política e Retórica}. Belo Horizonte: UFMG, 2009.

AMES, José Luiz. ``A formação do conceito moderno de estado: a
contribuição de Maquiavel'' in \emph{Revista Discurso}, São Paulo, 41,
2011 (293-328);

AMES, José Luiz. \emph{Conflito e Liberdade.} A vida política para
Maquiavel. Curitiba: CRV, 2017.

AMES, José Luiz. Liberdade e conflito. O confronto dos desejos como
fundamento da ideia de liberdade em Maquiavel\emph{. Kriterion}, Belo
Horizonte, n. 119, p. 179-196, jul. 2009.

AMES, José Luiz. \emph{Maquiavel: a lógica da ação política.} Cascavel:
Edunioeste, 2002;

AMES, José Luiz. Transformação do significado de conflito na
\emph{História de Florença} de Maquiavel\emph{. Kriterion}, Belo
Horizonte, n. 129, p. 265-286, jun. 2014.

BARON, Hans. \emph{The crisis of the early Italian Renaissance}.
Princenton: Princenton University Press, 1989;

BAUSI, Francesco. \emph{I Discorsi di Niccolò Machiavelli}. \emph{Genesi
e structure}. Firenze: Sansoni, 1985;

BAUSI, Francesco. ``Il problema dei `Discorsi''', \emph{Interpress},
XIX, (2000) {[}p. 249-261{]};

BENETTI, Fabiana. O conceito de Stato em Maquaivel. Dissertação de
mestrado. Toledo: Programa de Pós-graduação em Filosofia -- Unioeste,
2010;

BERTELLONI, Francisco. Quando a política começa a ser ciência
(antecedentes históricos e requisitos científicos da teoria política nos
séculos XIII e XIV), Revista Analytica, v. 09, n. 1, 2005 {[}p.
13-38{]};

BIGNOTTO, Newton. \emph{Maquiavel Republicano}. São Paulo: Loyola, 1991;

BIGNOTTO, Newton. \emph{Origens do republicanismo moderno.} Belo
Horizonte: ed. UFMG, 2001;

BIGNOTTO, Newton. \emph{O aprendizado da força} in Adverse, Helton.
\emph{Reflexões sobre Maquiavel.} 500 anos de O Príncipe. Belo
Horizonte: ed. UFMG, 2015 {[}p. 87-107{]}

BIGNOTTO, N. \emph{Antropologia negativa em Maquiavel} in ANALYTICA, Rio
de Janeiro, vol 12 nº 2, 2008 {[}p. 7-100{]}.

BORSELLINO, N. \emph{Niccolò Machiavelli}, in \emph{Letteratura
Italiana}. Bari: Laterza, 1973. Vol. 4. t.1, {[}p. 35-180{]}.

CADONI, Giorgio. \emph{Machiavelli. Regno di Francia e `principato
civile'}. Roma: Bulzoni Editore, 1974;

CARDOSO, Sérgio. Em direção ao núcleo da `obra Maquiavel': sobre a
divisão civil e suas interpretações. \emph{Discurso}, São Paulo, n.
45/2, p. 207-247, 2015.

CARDOSO, Sérgio. Maquiavel: lições das Histórias Florentinas.
\emph{Discurso}, São Paulo, v. 48, n. 1, p. 121-154, 2018.

CHABOD, Federico. \emph{Scritti su Machiavelli}. Torino: Einaudi, 1993;

CIZEK, Eugen. \emph{Mentalités et institutions politiques romaines.}
Paris: Fayard, 1990;

CUTINELLI-RÈNDINA, Emanuele. \emph{Chiesa e Religione in Machiavelli}.
Pisa-Roma: Istituto Editoriali e Poligrafici Internazionali, 1998;

DUSO, Giuseppe. \emph{O Poder. História da filosofia política moderna.}
Trad. de Andrea Ciacchi, Líssia da Cruz e Silva e Giuseppe Tosi. São
Paulo: Ed. Vozes, 2005

ERCOLE, F. \emph{Lo stato nel pensiero politico di Niccolo'
Machiavelli}, \emph{La politica di N. Machiavelli}, Anonima Romana
Editoriale, Roma, 1926.

FOURNEL, Jean-Louis. \emph{Ritorno su una vecchia questione : la
traduzione della parola 'stato' nel 'Principe' di Machiavelli} in
Chroniques italiennes, janvier 2008, série Web n° 13, 1/2008;

FROSINI, Fabio. \emph{L'aporia del `principato civil': il problema
politico del `forzare' in Principe IX. Filosofia Politica, nao 19, n.2,
p. 199-218,} 2005.

FUBINI, Riccardo. \emph{Italia quattrocentesca.} Milano: Franco Angeli,
1994.

GARIN, Eugênio. \emph{Ciência e vida civil no renascimento Italiano}.
Trad. Cecília Prada. São Paulo: Ed. Unesp, 1996;

GILBERT, Felix. \emph{Machiavel et Guichardin, Politique et histoire à
Florence au XVI siècle}. Paris: Seuil, 1996;

GILBERT, Felix. \emph{Machiavelli il suo tempo}. Bologna: Il Mulino,
1977;

GUIDI, Andrea. \emph{Un Segretario militante.} Politica, diplomazia e
armi nel Cancelliere Machiavelli. Bologne: Il Mulino, 2009.

HALE, John R. \emph{A Europa durante o Renascimento (1480-1520)}.
Lisboa: Editorial Presença, 1971;

INGLESE, Giorgio. Introduzione in Machiavelli, Nicollò. \emph{De
Principatibus.} . Testo critico a cura di G. Inglese. Roma: Istituto
Storico Italiano per il Medio Evo, 1994.

LARAVAILLE, Paul. \emph{La pensée politique de Machiavel}, \emph{Les
``Discours sur la Premiére Décade de Tite-Live}. Nancy: PUN, 1982;

LARIVAILLE, Paul. \emph{Introduction} in Machiavel, \emph{Il Principe}
et Agostino Nifo, \emph{De regnandi Peritia}. Introduction, traduction
et notes de Paul Larivaille. Paris: Les Belles Lettres, 2008.

LARIVAILLE, Paul. \emph{Il capítul IX del `Principe' e la crise del
`principato civile'.} In \emph{Cultura e Scritura di Machiavelli}. Roma:
Salerno Editrice, 1997 {[}p. 221-239{]}

LASLETT, P. \emph{Introdução} in \emph{Segundo Tratado sobre o governo
civil}. São Paulo: Martins Fontes, 1998

LEFORT, Claude. \emph{Le travail de l´ouvre Machiavel}. Paris:
Gallimard, 1972;

LEPORE, Ettore. \emph{Il princeps ciceroniano e gli ideali politici
della tarda republica}. Napoli: Istituto Italiano per gli Studi Storici,
1954;

MARTELLI, Mario. \emph{Saggio sul Principe.} Roma: Salerno Editrice,
1999.

MARTELLI, Mario. \emph{Introduzione.} In: \emph{Il Principi}. Edizione
Nazionale Delle Opere -- I/1, a cura de Mario Martelli. Roma: Salerno
Editrice, 2006.

MARTINS, José Antônio. \emph{Os fundamentos da república e sua corrupção
nos Discursos de Maquiavel}. Tese de doutorado. São Paulo: FFLCH/USP,
2007.

MARTINS, José Antônio. \emph{Sobre as origens do vocabulário político
medieval} in Trans/Form/Ação, vol. 34, n. 3, 2011 {[}p. 51-68{]}

MARTINS, José Antônio. \emph{Sobre o príncipe civil e a soberania em} O
Príncipe \emph{de Maquiavel} in Adverse, Helton. \emph{Reflexões sobre
Maquiavel.} 500 anos de O Príncipe. Belo Horizonte: ed. UFMG, 2015 {[}p.
127-152{]}

NICODIMOV, Marie Gaille. \emph{Conflit civil et liberté}. Le politique
machiavélienne entre histoire et medicine. Paris: Honoré Champion, 2004.

NICOLET, Claude. \emph{Les idées politiques à Rome sous la République}.
Paris: Colin, 1964.

NICOLET, Claude. \emph{Polybe et les institutions romanes} in
\emph{Entretiens}, tomo XX. Geneve: Vandouvres, 1973. {[}p. 222-231{]};

OSTROGORSK, Georg. \emph{Storia dell'impero bizantino.} Torino: Einaudi,
1968

POCOCK, J. G. A. \emph{Il momento machiavelliano}. Bologna: Società ed.
Il Mulino, 1980;

PROCACCI, Giuliano. \emph{Machiavelli nella cultura europea dell'eta
Moderna}. Bari: Laterza, 1995;

PROCACCI, Guiliano. \emph{Studi sulla fortuna del Machiavelli}. Roma:
Istituto Storico Italiano. 1965;

REALE, Mario. \emph{Machiavelli, la poltica e il problema del tempo. Un
doppio cominciamente della storia romana? A proposito di Romolo in}
Discorsi \emph{I, 9}. in \emph{La Cultura, XXIII, nº 1,} 1985 {[}p.
45-123{]};

RIDOLFI, Roberto. \emph{Biografia de Nicolau Maquaivel}. Trad. Nelson
Canabarro, São Paulo: Musa, 2005;

RUBINSTEIN, Nicolai. \emph{Le allegorie di Ambrogio Lorenzetti nella
Sala della Pace e il pensiero político del suo tempo} (1997) in
\emph{Studies in Italian history in teh middle ages and the
renaissance}. Roma: Edizioni di Stori e Letteratura, 2004.

RUBINSTEIN, Nicolai. \emph{An unknown version of Machiavelli's Ritratto
delle cose della Magna} (1998) in \emph{Studies in Italian History in
the Middle Ages and Renaissance.} Vol. III. Roma: Edizioni di Storia e
Letteratura, 2012.

RUBINSTEIN, Nicolai. \emph{Studies in Italian History in the Middle Ages
and Renaissance.} Vol. II. Politics, Diplomacy and the Constitution in
Florence and Italy. Roma: Edizioni di Storia e Letteratura, 2011.

RUBINSTEIN, Nicolai. ``The history of the word \emph{politicus} in early
modern Europe'' (1987) in \emph{Studies in Italian history in teh middle
ages and the renaissance}. Roma: Edizioni di Stori e Letteratura, 2004.

RUBINSTEIN, Nicolau. \emph{Il governo di Firenze sotto i Médici}.
Firenze: Nuova Itália, 1999;

SASSO, Gennaro. ``Intorno allá composizione dei Discorsi di Niccolò
Machiavelli'', \emph{Giornale Storico della Letteratura Italiana},
CXXXIV (1957), {[}p. 482ss{]} e CXXXV (1958), {[}p. 215ss{]};

SASSO, Gennaro. ``Note machiavelliane, I (Príncipe, IX)'', in \emph{La
Cultura,} Vol. XII, 1972 {[}p. 123-142{]};

SASSO, Gennaro. \emph{Machiavelli e gli antichi} \emph{e altri saggi}.
Tomo I e II Milano: Riccardo Ricciardi editore, 1987;

SASSO, Gennaro. \emph{Niccolo Machiavelli}, \emph{storia del suo
pensiero politico}. Bologna: Il Mulino, 1980;

SASSO, Gennaro. \emph{Principato civile e tirannide} e
\emph{Paralipomeni al ``principato civile''} in \emph{Machiavelli e gli
Antichi e altre saggi.} Napoli: Riccardo Ricciardi Editore, 1988;

SASSO, Gennaro. \emph{Studi su Machiavelli}. Napoli: Morano, 1967;

SENNELART, Michel. \emph{As artes de governar.} Trad. Paulo Neves. São
Paulo: Ed. 34 Letras, 2006.

SKINNER, Quentin. \emph{As fundações do Pensamento Político Moderno}.
Trad. Renato Janine Ribeiro. São Paulo: Cia das Letras, 2000;

SKINNER, Quentin. \emph{Virtù rinascimentali}. Bologna: Il Mulino, 2006

TAFURO, A. \emph{La formazione di Niccolò Machiavelli.} Napoli: Dante \&
Descates, 2003, {[}parte I, 1.2 e 1.3{]}

TAFURO, Antonio. \emph{Il reggimento di Firenze secondo Francesco
Guicciardini}. Napoli: Libreria Dante \& Descartes, 2005;

TAFURO, Antonio. \emph{La formazione di Niccolò Machiavelli. Ambiente
fiorentino, esperienza política, vicenda umana}. Napoli: Libreria Dante
\& Descartes, 2004;

TENENTI, Alberto. ```Civilità' e civiltà in Machiavelli'' in
\emph{Credenze, ideologie, libertinismi tra Medioevo ed Età moderna}.
Bologna: Il Mulino, 1978 {[}p. 155- 173{]};

VERNANT, Jean-Pierre. \emph{Mito e Pensamento entre gregos.} Rio de
Janeiro: Paz e Terra, 1995;

VIROLI, Maurízio. \emph{O sorriso de Nicolau}. Trad. Valéria Pereira da
Silva. São Paulo: Estação Liberdade, 2002.

 %\pagebreak
 %\def\contentsname{Índice}
 %\setcounter{tocdepth}{1}
 %\tableofcontents

%\pagebreak
%\addcontentsline{toc}{chapter}{Notas}
%\theendnotes

% Minitoc
 

 \end{document}
