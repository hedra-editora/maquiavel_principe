\section*{NICOLAUS~MACLAVELLUS MAGNIFICO~LAURENTIO~MEDICI~SALUTEM}

{[}1{]} Sogliono el più delle volte coloro che desiderando acquistar
grazia appresso uno principe farsegli incontro con quelle cose che infra
le loro abbino più care o delle quali vegghino lui più dilettarsi; donde
si vede molte volte essere loro presentati cavagli, arme, drappi d'oro,
pietre preziose e simili ornamenti degni della grandezza di quelli.
{[}2{]} Desiderando io adunque offerirmi alla vostra Magnificenzia con
qualche testimone della servitù mia verso di quella, non ho trovando,
intra la mia supellettile, cosa quale io abbia più cara o tanto existimi
quanto la cognizione delle actioni delli uomini grandi, imparata da me
con una lunga experienza delle cose moderne et una continua lectione
delle antiche; le quali avendo io con gran diligenzia lungamente
excogitate et examinate, et ora in uno piccolo volume ridotte, mando
alla Magnificenzia vostra. {[}3{]} E benché io iudichi questa opera
indegna della presenza di quella, tamen confido assai che per sua
umanità gli debba essere accepta, considerato come da me non gli possa
essere fatto maggiore dono che darle facultà a potere in brevissimo
tempo intendere tutto quello che io, in tanti anni e con tanti mia
disagi e pericoli, ho conosciuto et inteso. {[}4{]} La qual opera io non
ho ornata né ripiena di clausule ample o di parole ampullose e
magnifiche o di qualunque altro lenocinio o ornamento extrinseco, con li
quali molti sogliono le loro cose descrivere et ornare, perché io ho
voluto o che veruna cosa la onori o che solamente la varietà della
materia e la gravitá del subietto la facci grata. {[}5{]} Né voglio sia
imputata prosumptione se uno uomo di basso et infimo stato ardisce
discorrere e regolare e governi de' principi; perché, così come coloro
che disegnano e paesi si pongono bassi nel piano a considerare la natura
de' monti e de' luoghi alti e, per considerare quella de' luoghi bassi,
si pongono alto sopra' monti, similmente, a conoscere bene la natura de'
populi, bisogna essere principe et, a conoscere bene quella de'
principi, conviene essere populare.

{[}6{]} Pigli adunque vostra Magnificenzia questo piccolo dono con
quello animo che io 'l mando; il quale se da quella fia diligentemente
considerato e letto, vi conoscerà dentro uno extremo mio desiderio che
lei pervenga a quella grandezza che la fortuna e l'altre sua qualità le
promettano. {[}7{]} E se vostra Magnificenzia dallo apice della sua
altezza qualche volta volgerà li occhi in questi luoghi bassi, conoscerà
quanto io indegnamente sopporti una grande e continua malignità di
fortuna.

\setcounter{secnumdepth}{2}

\quebra\section{QUOT SINT GENERA PRINCIPATUUM ET~QUIBUS~MODIS~ACQUIRANTUR\break
{[}Di quante ragioni sieno e'principati e~in~che~modo~si~acquistino{]}}

{[}1{]} Tutti gli stati, tutti e' dominii che hanno avuto et hanno
imperio sopra gli uomini, sono stati e sono o republiche o principati.
{[}2{]} E principati sono o ereditarii, de' quali el sangue del loro
signore ne sia suto lungo tempo principe, o sono nuovi. {[}3{]} E nuovi,
o e' sono nuovi tutti, come fu Milano a Francesco Sforza, o sono come
membri aggiunti allo stato ereditario del principe che gli acquista,
come è el regno di Napoli a re di Spagna. {[}4{]} Sono questi dominii
così acquistati o consueti a vivere sotto uno principe o usi ad essere
liberi; et acquistonsi o con l'arme d'altri o con le proprie, o per
fortuna o per virtù.

\quebra\section{DE PRINCIPATIBUS HEREDITARIIS\break
{[}De' principati ereditari{]}}

{[}1{]} Io lascerò indrieto il ragionare delle repubbliche, perché altra
volta ne ragionai a lungo. {[}2{]} Volterommi solo al principato et
andrò ritexendo gli orditi soprascripti, e disputerò come questi
principati si possino governare e mantenere.

{[}3{]} Dico adunque che, nelli stati ereditari et assuefatti al sangue
del loro principe, sono assai minore difficultà a mantenergli che ne'
nuovi, perché basta solo non preterire gli ordini de' sua antinati e
dipoi temporeggiare con gli accidenti; in modo che, se tale principe è
di ordinaria industria, sempre si manterrà nel suo stato, se non è una
extraordinaria et excessiva forza che ne lo privi: e privato che ne fia,
quantunque di sinistro abbi l'occupatore, lo riacquista.

{[}4{]} Noi abbiamo in Italia, in exemplis, el duca di Ferrara, il quale
non ha retto alli assalti de' Viniziani nell' ottantaquatro, né a quelli
di papa Iulio nel dieci, per altre

cagioni che per essere antiquato in quello dominio. {[}5{]} Perché el
principe naturale ha minori cagioni e minore necessità di offendere,
donde conviene che sia più amato; e se extraordinarii vizii non lo fanno
odiare, è ragionevole che naturalmente sia benevoluto dalli sua. {[}6{]}
E nella antiquità e continuazione del dominio sono spente le memorie e
le cagioni delle innovazioni: perché sempre una mutazione lascia lo
adentellato per la edificazione dell'altra.

\quebra\section{DE PRINCIPATIBUS MIXTIS\break
{[}De' principati misti{]}}

{[}1{]} Ma nel principato nuovo consistono le difficultà. E prima, -- se
non è tutto nuovo, ma come membro: che si può chiamare tutto insieme
quasi mixto,-- le variazioni sue nascono im prima da una naturale
difficultà, quale è in tutti li principati nuovi: le quali sono che li
uomini mutano volentieri signore, credendo migliorare, e questa credenza
li fa pigliare l'arme contro a quello: di che s'ingannano, perché
veggono poi per experienzia avere piggiorato. {[}2{]} Il che depende da
un'altra necessità naturale et ordinaria, quale fa che sempre bisogni
offendere quegli di chi si diventa nuovo principe e con gente d'arme e
con infinite altre ingiurie che si tira drietro il nuovo acquisto:
{[}3{]} di modo che tu hai nimici tutti quegli che hai offesi in
occupare quello principato, e non ti puoi mantenere amici quelli che vi
ti hanno messo, per non gli potere satisfare in quel modo che si erano
presupposti e per non potere tu usare contro di loro medicine forte,
sendo loro obligato; perché sempre, ancora che uno sia fortissimo in
sulli exerciti, ha bisogno del favore de' provinciali ad entrare in una
provincia. {[}4{]} Per queste ragioni Luigi XII re di Francia occupò
subito Milano e subito lo perdé; e bastò a torgnene, la prima volta, le
forze proprie di Ludovico: perché quegli populi che gli avevano aperte
le porte, trovandosi ingannati della opinio-

\pagebreak

\noindent{}ne loro e di quello futuro
bene che si avevano presupposto, non potevano sopportare e fastidi del
nuovo principe.

{[}5{]} Bene è vero che, acquistandosi poi la seconda volta, e paesi
ribellati si perdono con più difficultà: perché el signore, presa
occasione dalla ribellione, è meno respettivo ad assicurarsi con punire
e delinquenti, chiarire e sospetti, provedersi nelle parte più deboli.
{[}6{]} In modo che, se a fare perdere Milano a Francia bastò la prima
volta uno duca Ludovico che rumoreggiassi in su' confini, a farlo dipoi
perdere la seconda gli bisognò avere contro tutto il mondo e che gli
exerciti sua fussino spenti o fugati di Italia: il che nacque dalle
cagioni sopraddette. {[}7{]} Nondimanco, e la prima e la seconda volta
gli fu tolto: le cagioni universali della prima si sono discorse; resta
ora a dire quelle della seconda e vedere che rimedi lui ci aveva e quali
ci può avere uno che fussi nelli termini sua, per potere meglio
mantenersi nello acquisto che non fece Francia.

{[}8{]} Dico pertanto che questi stati, quali acquistandosi si
aggiungano a uno stato antico di quello che acquista, o ei sono della
medesima provincia e della medesima lingua, o non sono. {[}9{]} Quando
sieno, è facilità grande a tenerli, maxime quando non sieno usi a vivere
liberi: et a possederli sicuramente basta avere spenta la linea del
principe che gli dominava, perché, nelle altre cose mantenendosi loro le
condizioni vecchie e non vi essendo disformità di costumi,

\quebra

\noindent{}gli uomini si
vivono quietamente; come si è visto che ha fatto la Borgogna, la
Brettagna, la Guascogna e la Normandia, che tanto tempo sono state con
Francia: e benché vi sia qualche disformità di lingua, nondimeno li
costumi sono simili e possonsi infra loro facilmente comportare.
{[}10{]} E chi le acquista, volendole tenere, debba avere dua respetti:
l'uno, che el sangue del loro principe antico si spenga; l'altro, di non
alterare né loro legge né loro dazii: talmente che in brevissimo tempo
diventa con il loro principato antiquo tutto uno corpo.

{[}11{]} Ma quando si acquista stati in una provincia disforme di
lingua, di costumi e di ordini, qui sono le difficultà e qui bisogna
avere gran fortuna e grande industria a tenerli. {[}12{]} Et uno de'
maggiori remedii e più vivi sarebbe che la persona di chi acquista vi
andassi ad abitare; questo farebbe più sicura e più durabile quella
possessione, come ha fatto il Turco di Grecia: il quale, con tutti li
altri ordini observati da lui per tenere quello stato, se non vi fussi
ito ad abitare non era possibile che lo tenessi. {[}13{]} Perché
standovi si veggono nascere e disordini e presto vi puo' rimediare: non
vi stando, s'intendono quando sono grandi e che non vi è più rimedio;
non è oltre a questo la provincia spogliata da' tua offiziali;
satisfannosi e subditi del ricorso propinquo al principe, donde hanno
più cagione di amarlo, volendo essere buoni, e, volendo essere
altrimenti, di temerlo; chi delli externi\mbox{  } volessi assaltare\mbox{  } quello
\mbox{  }stato, vi ha\mbox{  } più respecto;

\quebra

\noindent{}tanto che, abitandovi, lo può con grandissima difficultà perdere.

{[}14{]} L'altro migliore remedio è mandare colonie in uno o in dua
luoghi, che sieno quasi compedes di quello stato: perché è necessario o
fare questo o tenervi assai gente d'arme e fanti. {[}15{]} Nelle colonie
non si spende molto; e sanza sua spesa, o poca, ve le manda e tiene, e
solamente offende coloro a chi toglie e campi e le case per darle a'
nuovi abitatori, che sono una minima parte di quello stato; {[}16{]} e
quegli che gli offende, rimanendo dispersi e poveri, non gli possono mai
nuocere; e tutti li altri rimangono da uno canto inoffesi, -- e per
questo doverrebbono quietarsi, -- dall'altro paurosi di non errare, per
timore che non intervenissi a loro come a quelli che sono stati
spogliati. {[}17{]} Concludo che queste colonie non costono, sono più
fedeli, offendono meno, e li offesi non possono nuocere, sendo poveri e
dispersi, come è detto. {[}18{]} Per che si ha a notare che gli uomini
si debbono o vezzeggiare o spegnere; perché si vendicano delle leggieri
offese, delle gravi non possono: sì che la offesa che si fa' l'uomo
debbe essere in modo che la non tema la vendetta. {[}19{]} Ma tenendovi,
in cambio di colonie, gente d'arme, spende più assai, avendo a consumare
nella guardia tutte le intrate di quello stato, in modo che l'acquisto
gli torna\mbox{  } perdita; e offende\mbox{  } molto\mbox{  } più, perché\mbox{  } nuoce a tutto

\quebra

\noindent{}quello
stato, tramutando con li alloggiamenti il suo exercito; del quale
disagio ognuno ne sente e ciascuno gli diventa nimico: e sono nimici che
gli possono nuocere, rimanendo battuti in casa loro. {[}20{]} Da ogni
parte dunque questa guardia è inutile, come quella delle colonie è
utile.

{[}21{]} Debbe ancora chi è in una provincia disforme, come è detto,
farsi capo e defensore de' vicini minori potenti, et ingegnarsi di
indebolire e potenti di quella, e guardarsi che per accidente alcuno non
vi entri uno forestiere potente quanto lui: e sempre interverrà ch'e' vi
sarà messo da coloro che saranno in quella malcontenti o per troppa
ambizione o per paura, come si vidde già che gli Etoli missono e Romani
in Grecia, et, in ogni altra provincia che gli entrorno, vi furono messi
da' provinciali. {[}22{]} E l'ordine delle cose è che, subito che uno
forestiere potente entra in una provincia, tutti quelli che sono in espa
meno potenti gli aderiscano, mossi da una invidia hanno contro a chi è
suto potente sopra di loro: tanto che, respetto a questi minori potenti,
lui non ha a durare fatica alcuna a guadagnarli, perché subito tutti
insieme volentieri fanno uno globo col suo stato che lui vi ha
acquistato. {[}23{]} Ha solamente a pensare che non piglino troppe forze
e troppa autorità, e facilmente può con le forze sua e col favore loro
sbassare quelli che sono potenti, per ri-

\quebra

\noindent{}manere in tutto arbitro di
quella provincia; e chi non governerà bene questa parte, perderà presto
quello che arà acquistato e, mentre lo terrà, vi arà dentro infinite
difficultà e fastidii.

{[}24{]} E Romani, nelle provincie che pigliorno, observorno bene queste
parte: e' mandorno le colonie, intrattennono e meno potenti sanza
crescere loro potenza, abbassorno e potenti, e non vi lasciorno prendere
riputazione a' potenti forestieri. {[}25{]} E voglio mi basti solo la
provincia di Grecia per esemplo: furono intrattenuti da lloro gli Achei
e gli Etoli, fu abbassato il regno de' Macedoni, funne cacciato Antioco;
né mai e'meriti degli Achei o delli Etoli feciono che permettessino loro
accrescere alcuno stato, né le persuasioni di Filippo gl'indussono mai
ad essergli amici sanza sbassarlo, né la potenzia di Antioco possé fare
gli consentissino che tenessi in quella provincia alcuno stato. {[}26{]}
Perché e' romani feciono in questi casi quello che tutti e' principi
savi debbono fare: e' quali non solamente hanno ad avere riguardo alli
scandoli presenti, ma a' futuri, e a quelli con ogni industria ovviare;
perché, prevedendosi discosto, vi si rimediare facilmente, ma,
aspettando che ti si appressino, la medicina non è a tempo, perché la
malattia è diventata incurabile; {[}27{]} e interviene di questa, come
dicono e' fisici dello etico, che nel principio del suo male è facile a
curare e difficile a conoscere: ma nel progresso del tempo, non la
avendo nel principio conosciuta né medicata,

\quebra

\noindent{}diventa facile a conoscere
e difficile a curare. {[}28{]} Così interviene nelle cose di stato:
perché conoscendo discosto, il che non è dato se non a uno prudente, e'
mali che nascono in quello si guariscono presto; ma quando, per non gli
avere conosciuti, si lasciono crescere in modo che ognuno gli conosce,
non vi è più rimedio.

{[}29{]} Però e romani, veggendo discosto gl'inconvenienti, vi
rimediorno sempre, e non gli lasciorno mai seguire per fuggire una
guerra, perché sapevano che la guerra non si lieva, ma si differisce a
vantaggio di altri: però vollono fare con Filippo e Antioco guerra in
Grecia, per non la avere a fare con loro in Italia; e potevono per
allora fuggire l'una e l'altra: il che non vollono. {[}30{]} Né piacque
mai loro quello che è tutto dí in bocca de' savi de' nostri tempi, di
godere il benefizio del tempo, ma sì bene quello della virtù e prudenza
loro: perché il tempo si caccia innanzi ogni cosa, e può condurre seco
bene come male e male come bene.

{[}31{]} Ma torniamo a Francia ed esaminiamo se delle cose dette e' ne
ha fatte alcuna: e parlerò di Luigi, e non di Carlo, come di colui che,
per aver tenuta più lunga possessione in Italia, si sono meglio visti e'
sua progressi: e vedrete come egli ha fatto il contrario di quelle cose
che si debbono fare per tenere uno stato in una provincia disforme.
{[}32{]} El re Luigi fu messo in Italia da la ambizione de' viniziani,
che vollono guadagnarsi mezzo lo stato di Lombardia per quella venuta.
{[}33{]} Io non voglio biasimare questo partito preso dal re: perché,
volendo cominciare a mettere uno piè in Italia

\quebra

\noindent{}e non avendo in questa
provincia amici, anzi sendogli per li portamenti del re Carlo serrate
tutte le porte, fu necessitato prendere quelle amicizie che poteva; e
sarebbegli riuscito el partito bene preso, quando nelli altri maneggi
non avessi fatto alcuno errore. {[}34{]} Acquistata adunque el re la
Lombardia, subito si riguadagnò quella reputazione che gli aveva tolta
Carlo: Genova cedé; fiorentini gli diventorono amici; marchese di
Mantova, duca di Ferrara, Bentivogli, Madonna di Furlí, signore di
Faenza, di Rimini, di Pesero , di Camerino, di Piombino, lucchesi,
pisani, sanesi, ognuno se gli fece incontro per essere suo amico.
{[}35{]} E allora poterno considerare e' viniziani la temerità del
partito preso da loro, e' quali, per acquistare dua terre in Lombardia,
feciono signore el re de' dua terzi di Italia.

{[}36{]} Consideri ora uno con quanta poca difficultà poteva el re
tenere in Italia la sua reputazione, se lui avessi osservate le regule
soprascritte e tenuti sicuri e difesi tutti quelli sua amici, e' quali,
per essere gran numero e deboli e paurosi chi della Chiesa chi de'
viniziani, erano sempre necessitati a stare seco; e per il mezzo loro
poteva facilmente assicurarsi di chi ci restava grande. {[}37{]} Ma lui
non prima fu in Milano che fece il contrario, dando aiuto a papa
Alessandro perché egli occupassi la Romagna; né si accorse, con questa
deliberazione, che faceva sé debole, togliendosi gli amici e quegli che
se gli erano gittati in grembo, e la Chiesa grande, aggiugnendo allo
spirituale, che le dà tanta autorità,

\quebra

\noindent{}tanto temporale. {[}38{]} E fatto
un primo errore fu constretto a seguitare: in tanto che, per porre
termine alla ambizione di Alessandro e perché e' non divenissi signore
di Toscana, e' fu constretto venire in Italia.

{[}39{]} Non gli bastò avere fatto grande la Chiesa e toltosi gli amici:
che, per volere il regno di Napoli, lo divise con il re di Spagna; e
dove egli era prima arbitro di Italia, vi misse uno compagno, acciò che
gli ambiziosi di quella provincia e malcontenti di lui avessino dove
ricorrere; e dove posseva lasciare in quel regno uno re suo pensionario,
e' ne lo trasse per mettervi uno che potessi cacciarne lui. {[}40{]} È
cosa veramente molto naturale e ordinaria desiderare di acquistare: e
sempre, quando li uomini lo fanno, che possano, saranno laudati o non
biasimati; ma quando e' non possono, e vogliono farlo in ogni modo, qui
è lo errore et il biasimo. {[}41{]} Se Francia adunque posseva con le
sue forze sua assaltare Napoli, doveva farlo: s'e' non poteva, non
doveva dividerlo; e se la divisione fece co' viniziani di Lombardia
meritò scusa, per avere con quella messo el piè in Italia, questa merita
biasimo, per non essere scusata da quella necessità.

{[}42{]} Aveva dunque fatto Luigi questi cinque errori: spenti e minori
potenti; accresciuto in Italia potenza a uno potente; messo in quella
uno forestiere potentissimo; non venuto ad abitarvi; non vi messo
colonie. {[}43{]} E' quali errori ancora, vivendo lui, potevono non lo
offendere, se non avessi fatto il sesto, di tòrre lo stato a' viniziani.

\quebra

\noindent{}{[}44{]} Perché, quando e' non avessi fatto grande la Chiesa né messo in
Italia Spagna, era bene ragionevole e necessario abbassargli; ma avendo
preso quegli primi partiti, non doveva mai consentire alla ruina loro:
perché, sendo quegli potenti, sempre arebbono tenuti gli altri discosto
da la impresa di Lombardia, sì perché e' viniziani non vi arebbono
consentito sanza diventarne signori loro, sì perché li altri non
arebbono voluto torla a Francia per darla a loro: e andare a urtarli
tutti a dua non arebbono avuto animo.

{[}45{]} E se alcuno dicessi: el re Luigi cedé ad Alessandro la Romagna
e a Spagna el Regno per fuggire una guerra; rispondo con le ragioni
dette di sopra, che non si debbe mai lasciare seguire uno disordine per
fuggire una guerra: perché la non si fugge, ma si differisce a tuo
disavvantaggio. {[}46{]} E se alcuni altri allegassino la fede che il re
aveva data al papa, di fare per lui quella impresa per la resoluzione
del suo matrimonio e il cappello di Roano, rispondo con quello che per
me di sotto si dirà circa alla fede de' principi e come la si debbe
osservare.

{[}47{]} Ha perduto adunque el re Luigi la Lombardia per non avere
osservato alcuno di quelli termini osservati da altri che hanno preso
provincie e volutole tenere; né è miraculo alcuno questo, ma molto
ordinario e ragionevole. {[}48{]} E di questa materia parlai a Nantes
con Roano, quando el Valentino, -- che così era chiamato popularmente
Cesare Borgia, figliuolo di papa Alessandro, -- occupava la Romagna;
perché,\mbox{  } dicendomi el\mbox{  } cardinale di\mbox{  } Roano che gli\mbox{  } italiani

\quebra

\noindent{}non si
intendevano della guerra, io gli risposi che ' franzesi non si
intendevano dello stato: perché, s'e' se ne 'ntendessino, non
lascerebbono venire in tanta grandezza la Chiesa. {[}49{]} E per
esperienza si è visto che la grandezza in Italia di quella e di Spagna è
stata causata da Francia, e la ruina sua è suta causata da loro.
{[}50{]} Di che si trae una regula generale, la quale mai o raro falla,
che chi è cagione che uno diventi potente, ruina: perché quella potenza
è causata da colui o con industria o con forza, e l'una e l'altra di
queste dua è sospetta a chi è divenuto potente.

\quebra\section{CUR DARII REGNUM, QUOD ALEXANDER OCCUPAVERAT, A SUCCESSORIBUS SUIS POST
ALEXANDRI MORTEM NON DEFECIT\break
{[}Per quale cagione el regno di Dario, il~quale~da~Alessandro~fu
occupato, non~si~rebellò~da'~sua successori dopo~la morte di
Alessandro{]}}

{[}1{]} Considerate le difficultà le quali s'hanno a tenere uno stato
occupato di nuovo, potrebbe alcuno maravigliarsi donde nacque che
Alessandro Magno diventò signore della Asia in pochi anni e, non la
avendo appena occupata, morí: donde pareva ragionevole che tutto quello
stato si ribellassi; nondimeno e' successori di Alessandro se lo
mantennono e non ebbono, a tenerlo, altra difficultà che quella che
infra loro medesimi per propria ambizione nacque. {[}2{]} Rispondo come
e' principati de' quali si ha memoria si truovono governati in dua modi
diversi: o per uno principe e tutti li altri servi, e' quali come
ministri, per grazia e concessione sua, aiutono governare quello regno;
o per uno principe e per baroni e' quali non per grazia del signore, ma
per antichità di sangue, tengono quel grado. {[}3{]} Questi tali baroni
hanno stati e sudditi propri, e' quali gli ricognoscono per signori et
hanno in loro naturale affezione. {[}4{]} Quelli stati che si governano
per uno principe e per servi hanno el loro principe con più autorità,
perché in tutta la sua provincia non è uomo\mbox{  } che riconosca alcuno per
superiore se non lui;\mbox{  } e se

\quebra

\noindent{}ubbidiscono alcuno altro, lo fanno come
ministro e offiziale; e a lui portano particulare amore.

{[}5{]} Li esempli di queste dua diversità di governi sono, ne' nostri
tempi, el Turco et il re di Francia. {[}6{]} Tutta la monarchia del
Turco è governata da uno signore: li altri sono sua servi; e
distinguendo il suo regno in sangiacchie, vi manda diversi
amministratori e gli muta e varia come pare a lui. {[}7{]} Ma il re di
Francia è posto in mezzo di una moltitudine antiquata di signori, in
quello stato riconosciuti da' loro sudditi e amati da quegli: hanno le
loro preminenze, non le può il re tòrre loro sanza suo periculo. {[}8{]}
Chi considera adunque l'uno e l'altro di questi stati, troverrà
difficultà nell'acquistare lo stato del Turco, ma, vinto che fia,
facilità grande a tenerlo. {[}9{]} Così per avverso troverrà per qualche
respetto più facilità a potere occupure il regno di Francia, ma
difficultà grande a tenerlo.

{[}10{]} Le cagioni delle difficultà, in potere occupare il regno del
Turco, sono per non potere essere chiamato da' principi di quell regno,
né sperare, con la rebellione di quegli che gli ha d'intorno, potere
facilitare la tua impresa; il che nasce da le ragioni sopraddette:
perché, sendogli tutti stiavi e obligati, si possono con più difficultà
corrompere e, quando bene si corrompessino, se ne può sperare poco
utile, non potendo quelli tirarsi drieto e' populi per le ragioni
assegnate. {[}11{]}Onde a chi assalta el Turco è necessario pen-

\quebra

\noindent{}sare di
averlo a trovare unito, e gli conviene sperare più nelle forze proprie
che ne' disordini di altri. {[}12{]} Ma vinto ch'e' fussi, e rotto alla
campagna in modo che non possa rifare exerciti, non si ha a dubitare di
altro che del sangue del principe: el quale spento, non resta alcuno di
chi si abbia a temere, non avendo gli altri credito con e' populi; e
come el vincitore avanti la vittoria non poteva sperare in loro, così
non debba dopo quella temere di loro.

{[}13{]} Al contrario interviene ne' regni governati come quello di
Francia: perché con facilità tu puoi entrarvi guadagnandoti alcuno
barone del regno, perché sempre si truova de' mali contenti e di quegli
che desiderano innovare. {[}14{]} Costoro per le ragioni dette ti
possono aprire la via a quello stato e facilitarti la vittoria: la quale
di poi, a volerti mantenere, si tira drieto infinite difficultà e con
quelli che ti hanno aiutato e con quelli che tu hai oppressi. {[}15{]}
Né ti basta spegnere el sangue del principe, perché vi rimangono quelli
signori, che si fanno capi delle nuove alterazioni: e non gli potendo né
contentare né spegnere, perdi quello stato qualunque volta la occasione
venga.

{[}16{]} Ora, se voi considerrete di qual natura di governi era quello
di Dario, lo troverrete simile al regno del Turco: e però ad Alessandro
fu necessario prima urtarlo tutto e tòrgli la campagna. {[}17{]} Dopo la
qual vittoria, sendo Dario morto, rimase ad Alessandro quello stato
sicuro per le ragioni di sopra discorse; ed e' sua successori, se
fussino stati uniti, se lo potevano godere oziosi: né in quello regno
nacquono altri tumulti che quegli che loro propri sucitorno. {[}18{]} Ma
gli stati ordinati come quello di Francia è impossibile possederli con
tanta quiete. {[}19{]} Di qui nacquono le spesse ribellioni di Spagna,
di Francia e di Grecia da' Romani, per gli spessi principati che erano
in quelli stati: de' quali mentre durò la memoria, sempre fu Roma
incerta di quella possessione. {[}20{]} Ma spenta la memoria di quelli,
con la potenza e diuturnità dello imperio, ne diventorno sicuri
possessori: e poterno anche quelli di poi, combattendo infra loro,
ciascuno tirarsi dreto parte di quelle provincie secondo l'autorità vi
aveva presa dentro; e quelle, per essere e' sangui de' loro antiqui
signori spenti, non riconoscevano se non e' romani. {[}21{]} Considerato
adunque tutte queste cose, non si maraviglierà alcuno della facilità
ebbe Alessandro a tenere lo stato di Asia, e delle difficultà che hanno
avuto gli altri a conservare lo acquistato, come Pirro e molti: il che
non è nato da la poca o da la molta virtù del vincitore, ma da la
disformità del subietto.

\quebra\section{QUOMODO ADMINISTRANDAE SUNT CIVITATES VEL PRINCIPATUS, QUI, ANTEQUAM
OCCUPARENTUR SUIS LEGIBUS VIVEBANT\break
{[}In che modo si debbino governare le città o principati li quali,
innanzi fussino occupati, si vivevano con le loro legge{]}}

{[}1{]} Quando quelli stati, che si acquistano come è detto, sono
consueti a vivere con le loro leggi e in libertà, a volergli tenere ci
sono tre modi: {[}2{]} il primo, ruinarle; l'altro, andarvi ad abitare
personalmente; il terzo, lasciàgli vivere con le sua legge, traendone
una pensione e creandovi dentro uno stato di pochi, che te le conservino
amico: {[}3{]} perché, sendo quello stato creato da quello principe, sa
che non può stare sanza l'amicizia e potenza sua e ha a fare tutto per
mantenerlo; e più facilmente si tiene una città usa a vivere libera con
il mezzo de' sua cittadini che in alcuno altro modo, volendola
perservare.

{[}4{]} In exemplis ci sono gli spartani ed e' romani. Gli spartani
tennono Atene e Tebe creandovi uno stato di pochi, tamen le riperderno.
{[}5{]} E' romani, per tenere Capua Cartagine e Numanzia, le disfeciono,
e non le perderno; vollono tenere la Grecia quasi come tennono gli
spartani, faccendola libera e lasciandole le sua legge, e non successe
loro: tale che furono constretti disfare di molte città di quella
provincia per tenerla. {[}6{]} Perché in verità non ci è modo sicuro a
possederle altro che la ruina; e chi diviene patrone di una città
consueta a vivere libera, e non la disfaccia, aspetti di essere disfatto
da quella: perché sempre ha per refugio, nella rebellione el nome della
libertà e gli ordini antiqui sua, e' quali né per la lunghezza di tempo
né per benefizi mai si dimenticano. {[}7{]} E per cosa che si faccia o
si provegga, se non si disuniscano o dissipano gli abitatori non
dimenticano quello nome né quegli ordini, e subito in ogni accidente vi
ricorrono; come fe' Pisa dopo cento anni che la era suta posta in
servitù da' fiorentini. {[}8{]} Ma quando le città o le provincie sono
use a vivere sotto uno principe e quello sangue sia spento, sendo da uno
canto usi a ubbidire, da l'altro non avendo il principe vecchio, farne
uno infra loro non si accordano, vivere liberi non sanno: di modo che
sono più tardi a pigliare l'arme e con più facilità se gli può uno
principe guadagnare et assicurarsi di loro.

{[}9{]} Ma nelle repubbliche è maggiore vita, maggiore odio, più
desiderio di vendetta: né gli lascia, né può lasciare, riposare la
memoria della antiqua libertà; tale che la più sicura via è spegnerle, o
abitarvi.

\quebra\section{DE PRINCIPATIBUS NOVIS QUI ARMIS PROPRIIS ET VIRTUTE ACQUIRUNTUR.
{[}De' Principati nuovi che s'acquistano con l'arme proprie e virtuosamente{]}}

{[}1{]} Non si maravigli alcuno se, nel parlare che io farò de'
principati al tutto nuovi e di principe e di stato, io addurrò
grandissimi esempli. {[}2{]} Perché, camminando gli uomini sempre per le
vie battute da altri e procedendo nelle azioni loro con le imitazioni,
né si potendo le vie d'altri al tutto tenere né alla virtù di quegli che
tu imiti aggiugnere, debbe uno uomo prudente intrare sempre per vie
battute da uomini grandi, e quegli che sono stati eccellentissimi
imitare: acciò che, se la sua virtù non vi arriva, almeno ne renda
qualche odore; {[}3{]} e fare come gli arcieri prudenti, a' quali
parendo el luogo dove desegnano ferire troppo lontano, e conoscendo fino
a quanto va la virtù del loro arco, pongono la mira assai più alta che
il luogo destinato, non per aggiugnere con la loro freccia a tanta
altezza, ma per potere con lo aiuto di sì alta mira pervenire al disegno
loro.

{[}4{]} Dico adunque che ne' principati tutti nuovi, dove sia uno nuovo
principe, si truova a mantenergli più o meno difficultà secondo che più
o meno è virtuoso colui che gli acquista. {[}5{]} E perché questo
evento, di diventare di privato principe, presuppone o virtù o fortuna,
pare che l'una o l'altra di queste dua cose mitighino in parte molte
difficultà; nondimanco, colui che è stato meno in su la fortuna si è
mantenuto più. {[}6{]} Genera ancora facilità essere el principe
constretto, per non avere altri stati, venire personalmente ad abitarvi.

{[}7{]} Ma per venire a quegli che per propria virtù e non per fortuna
sono diventati principi, dico che e' più eccellenti sono Moisè, Ciro,
Romulo, Teseo e simili. {[}8{]} E benché di Moisè non si debba
ragionare, sendo suto uno mero esecutore delle cose che gli erano
ordinate da Dio, tamen debbe essere ammirato, solum per quella grazia
che lo faceva degno di parlare con Dio. {[}9{]} Ma consideriamo Ciro e
li altri che hanno acquistato o fondati regni, gli troverrete tutti
mirabili; e se si considerranno le azioni et ordini loro particulari,
parranno non discrepanti da quegli di Moisè, che ebbe sì grande
precettore. {[}10{]} Ed esaminando le azioni e vita loro non si vede che
quelli avessino altro da la fortuna che la occasione, la quale dette
loro materia a potere introdurvi dentro quella forma che parse loro: e
sanza quella occasione la virtù dello animo loro si sarebbe spenta, e
sanza quella virtù la occasione sarebbe venuta invano.

{[}11{]} Era adunque necessario a Moisè trovare el populo d'Israel in
Egitto stiavo et oppresso dalli egizi, acciò che quegli, per uscire di
servitú, si disponessino a seguirlo. {[}12{]} Conveniva che Romulo non
capessi in Alba, fussi stato esposto al nascere, a volere che diventassi
re di Roma e fondatore di quella patria. {[}13{]} Bisognava che Ciro
trovassi e' persi malcontenti dello imperio de' medi, ed e' medi molli
et effeminati per la lunga pace. {[}14{]} Non poteva Teseo dimostrare la
sua virtù, se non trovava gli ateniesi dispersi. {[}15{]} Queste
occasioni per tanto feciono questi uomini felici e la eccellente virtù
loro fé quella occasione essere conosciuta: donde la loro patria ne fu
nobilitata e diventò felicissima.

{[}16{]} Quelli e' quali per vie virtuose, simili a costoro, diventono
principi, acquistono el principato con difficultà, ma con facilità lo
tengano; e le difficultà che gli hanno nello acquistare el principato
nascono in parte da' nuovi ordini e modi che sono forzati introdurre per
fondare lo stato loro e la loro sicurtà. {[}17{]} E debbesi considerare
come e' non è cosa più difficile a trattare, né più dubia a riuscire, né
più pericolosa a maneggiare, che farsi capo di introdurre nuovi ordini.
{[}18{]} Perché lo introduttore ha per nimico tutti quegli che degli
ordini vecchi fanno bene, e ha tiepidi defensori tutti quelli che delli
ordini nuovi farebbono bene: la quale tepidezza nasce parte per paura
delli avversari, che hanno le leggi dal canto loro, parte da la
incredulità degli uomini, e' quali non credano in verità le cose nuove,
se non ne veggono nata una ferma esperienza. {[}19{]} Donde nasce che,
qualunque volta quelli che sono nimici hanno occasione di assaltare, lo
fanno partigianamente, e quelli altri difendano tiepidamente: in modo
che insieme con loro si periclita.

{[}20{]} È necessario pertanto, volendo discorrere bene questa parte,
esaminare se questi innovatori stanno per loro medesimi o se dependano
da altri: cioè, se per condurre l'opera loro bisogna che preghino, o
vero possono forzare. {[}21{]} Nel primo caso, sempre capitano male e
non conducono cosa alcuna; ma quando dependono da loro propri e possano
forzare, allora è che rare volte periclitano: di qui nacque che tutti e'
profeti armati vinsono ed e' disarmati ruinorno. {[}22{]} Perché, oltre
alle cose dette, la natura de' populi è varia ed è facile a persuadere
loro una cosa, ma è difficile fermargli in quella persuasione: e però
conviene essere ordinato in modo che, quando non credono più, si possa
fare loro credere per forza. {[}23{]} Moisè, Ciro, Teseo e Romulo non
arebbono potuto fare osservare loro lungamente le loro constituzioni, se
fussino stati disarmati; come ne' nostri tempi intervenne a fra Ieronimo
Savonerola, il quale ruinò ne' sua ordini nuovi, come la moltitudine
cominciò a non credergli, e lui non aveva modo a tenere fermi quelli che
avevano creduto né a fare credere e' discredenti. {[}24{]} Però questi
tali hanno nel condursi grande difficultà, e tutti e' loro pericoli sono
fra via e conviene che con la virtù gli superino. {[}25{]} Ma superati
che gli hanno, e che cominciano a essere in venerazione, avendo spenti
quegli che di sua qualità gli avevano invidia, rimangono potenti,
sicuri, onorati e felici.

{[}26{]} A sì alti esempli io voglio aggiugnere uno exemplo minore; ma
bene arà qualche proporzione con quegli, e voglio mi basti per tutti gli
altri simili: e questo è Ierone siracusano. {[}27{]} Costui di privato
diventò principe di Siracusa; né ancora lui conobbe altro da la fortuna
che la occasione; perché, sendo e' siracusani oppressi, lo elessono per
loro capitano; donde meritò di essere fatto loro principe. {[}28{]} E fu
di tanta virtù, etiam in privata fortuna, che chi ne scrive dice
\emph{quod nihil illi deerat ad regnandum praeter regnum.} {[}29{]}
Costui spense la milizia vecchia, ordinò della nuova; lasciò le amicizie
antiche, prese delle nuove; e come ebbe amicizie e soldati che fussino
sua, possé in su tale fondamento edificare ogni edifizio, tanto che lui
durò assai fatica in acquistare e poca in mantenere.

\quebra\section{DE PRINCIPATIBUS NOVIS QUI ALIENIS ARMIS ET FORTUNA ACQUIRUNTUR.
{[}De' principati nuovi che s'acquistano con le armi e fortuna di altri{]}}

{[}1{]} Coloro e' quali solamente per fortuna diventano di privati
principi, con poca fatica diventono, ma con assai si mantengono; e non
hanno alcuna difficultà fra via, perché vi volano: ma tutte le
difficultà nascono quando e' sono posti. {[}2{]} E questi tali sono
quando è concesso ad alcuno uno stato o per danari o per grazia di chi
lo concede: come intervenne a molti in Grecia nelle città di Ionia e di
Ellesponto, dove furno fatti principi da Dario, acciò le tenessino per
sua sicurtà e gloria; come erano fatti ancora quelli imperatori che di
privati, per corruzione de' soldati, pervenivano allo imperio.

{[}3{]} Questi stanno semplicemente in su la volontà e fortuna di chi lo
ha concesso loro, che sono dua cose volubilissime e instabili, e non
sanno e non possano tenere quello grado: non sanno, perché s'e' non è
uomo di grande ingegno e virtù, non è ragionevole che, sendo vissuto
sempre in privata fortuna, sappia comandare; non possono, perché non
hanno forze che gli possino essere amiche e fedele. {[}4{]} Di poi gli
stati che vengano subito, come tutte l'altre cose della natura che
nascono e crescono presto, non possono avere le barbe e correspondenzie
loro in modo che il primo tempo avverso non le spenga, se già quelli
tali -- come è detto -- che sì de repente sono diventati principi non
sono di tanta virtù che quello che la fortuna ha messo loro in grembo e'
sappino subito prepararsi a conservarlo, e quelli fondamenti, che gli
altri hanno fatti avanti che diventino principi, gli faccino poi.

{[}5{]} Io voglio all'uno e l'altro di questi modi detti, circa il
diventare principe per virtù o per fortuna, addurre dua esempli stati
ne' dí della memoria nostra: e questi sono Francesco Sforza e Cesare
Borgia. {[}6{]} Francesco, per li debiti mezzi e con una grande sua
virtù, di privato diventò duca di Milano; e quello che con mille affanni
aveva acquistato, con poca fatica mantenne. {[}7{]} Da l'altra parte,
Cesare Borgia, chiamato dal vulgo duca Valentino, acquistò lo stato con
la fortuna del padre e con quella lo perdé, non obstante che per lui si
usassi ogni opera e facessinsi tutte quelle cose che per uno prudente e
virtuoso uomo si doveva fare per mettere le barbe sua in quelli stati
che l'arme e fortuna di altri li aveva concessi. {[}8{]} Perché, come di
sopra si disse, chi non fa e' fondamenti prima, gli potrebbe con una
grande virtù farli poi, ancora che si faccino con disagio dello
architettore e periculo dello edifizio. {[}9{]} Se adunque si considerrà
tutti e' progressi del duca, si vedrà lui aversi fatti grandi fondamenti
alla futura potenza; e' quali non iudico superfluo discorrere perché io
non saprei quali precetti mi dare migliori, a uno principe nuovo, che lo
esemplo delle azioni sue: e se gli ordini sua non gli profittorno, non
fu sua colpa, perché nacque da una extraordinaria ed estrema malignità
di fortuna.

{[}10{]} Aveva Alexandro VI, nel volere fare grande il duca suo
figliuolo, assai difficultà presenti e future. {[}11{]} Prima, e' non
vedeva via di poterlo fare signore di alcuno stato che non fussi stato
di Chiesa: e, volgendosi a tòrre quello della Chiesa, sapeva che il duca
di Milano e' viniziani non gliene consentirebbono, perché Faenza e
Rimino erano di già sotto la protezione de' viniziani. {[}12{]} Vedeva
oltre a questo l'arme di Italia, e quelle in spezie di chi si fussi
potuto servire, essere nelle mani di coloro che dovevano temere la
grandezza del papa, -- e però non se ne poteva fidare, -- sendo tutte
nelli Orsini e Colonnesi e loro complici. {[}13{]} Era adunque
necessario si turbassino quelli ordini e disordinare gli stati di
Italia, per potersi insignorire sicuramente di parte di quelli. {[}14{]}
Il che gli fu facile, perché e' trovò e' viniziani che, mossi da altre
cagioni, si erano volti a fare ripassare e' franzesi in Italia: il che
non solamente non contradisse, ma lo fe' più facile con la resoluzione
del matrimonio antico del re Luigi.

{[}15{]} Passò adunque il re in Italia con lo aiuto de' vineziani e
consenso di Alessandro: né prima fu in Milano che il papa ebbe da lui
gente per la impresa di Romagna, la quale gli fu acconsentita per la
reputazione del re. {[}16{]} Acquistata adunque il duca la Romagna e
sbattuti e' Colonnesi, volendo mantenere quella e procedere più avanti,
lo impedivano dua cose: l'una, le arme sua che non gli parevano fedeli;
l'altra, la volontà di Francia; cioè che l'arme Orsine, delle quali si
era valuto, gli mancassino sotto, e non solamente gl'impedissino lo
acquistare ma gli togliessino lo acquistato, e che il re ancora non li
facessi il simile. {[}17{]} Delli Orsini ne ebbe uno riscontro quando
dopo la espugnazione di Faenza, assaltò Bologna, che gli vidde andare
freddi in quello assalto; e circa el re conobbe lo animo suo quando,
preso el ducato d'Urbino, assaltò la Toscana: da la quale impresa il re
lo fece desistere.

{[}18{]} Onde che il duca deliberò di non dependere più da le arme e
fortuna di altri; e, la prima cosa, indebolì le parte Orsine e Colonnese
in Roma: perché tutti gli aderenti loro, che fussino gentili uomini, se
gli guadagnò, faccendoli suoi gentili uomini e dando loro grande
provisioni, et onorogli, secondo le loro qualità, di condotte e di
governi: in modo che in pochi mesi negli animi loro l'affectione delle
parti si spense e tutta si volse nel Duca. {[}19{]} Dopo questo, aspettò
la occasione di spegnere e capi Orsini, avendo dispersi quelli di casa
Colonna: la quale gli venne bene, e lui la usò meglio. {[}20{]} Perché,
advedutosi gli Orsini tardi che la grandezza del Duca e della Chiesa era
la loro ruina, feciono una dieta alla Magione nel Perugino; da quella
nacque la ribellione di Urbino, e tumulti di Romagna et infiniti
periculi del Duca, e quali tutti superò con lo aiuto delli Franzesi.
{[}21{]} E ritornatoli la reputazione, né si fidando di Francia né di
altre forze externe, per non le avere a cimentare si volse alli inganni;
e seppe tanto dissimulare l'animo suo che li Orsini, mediante il signore
Paulo, si riconciliorono seco, -- con il quale il Duca non mancò d'ogni
ragione di offizio per assicurarlo, dandoli danari veste e cavalli --
tanto che la simplicità loro gli condusse a Sinigaglia nelle sua mane.

{[}22{]} Spenti adunque questi capi e ridotti li partigiani loro sua
amici, aveva il Duca gittati assai buoni fondamenti alla potenza sua,
avendo tutta la Romagna col ducato di Urbino, parendoli maxime aversi
acquistata amica la Romagna e guadagnatosi quelli popoli per avere
cominciato a gustare il bene essere loro. {[}23{]} E perché questa parte
è degna di notizia e da essere da altri imitata, non la voglio lasciare
indietro. {[}24{]} Preso che ebbe il Duca la Romagna e trovandola suta
comandata da Signori impotenti, -- li quali più presto avevano spogliato
e loro subditi che corretti, e dato loro materia di disunione, non di
unione, -- tanto che quella provincia era tutta piena di latrocinii, di
brighe e d'ogni altra ragione di insolenzia, iudicò fussi necessario, a
volerla ridurre pacifica et ubbidiente al braccio regio, dargli buono
governo; e però vi prepose messer Remirro de Orco, uomo crudele et
expedito, al quale dette plenissima potestà. {[}25{]} Costui im poco
tempo la ridusse pacifica et unita, con grandissima reputazione.
{[}26{]} Dipoi iudicò il duca non essere necessario sì excessiva
autorità perché dubitava non divenissi odiosa, e preposevi uno iudizio
civile nel mezzo della provincia, con uno presidente excellentissimo,
dove ogni città vi aveva lo advocato suo. {[}27{]} E perché conosceva le
rigorosità passate avergli generato qualche odio, per purgare li animi
di quegli populi e guadagnarseli in tutto, volse monstrare che, se
crudeltà alcuna era seguita, non era causata da lui ma dalla acerba
natura del ministro. {[}28{]} E presa sopra a questo occasione, lo fece,
a Cesena, una mattina mettere in dua pezzi in sulla piazza, con uno
pezzo di legne et uno coltello sanguinoso accanto; la ferocità del quale
spettaculo fece quegli popoli in uno tempo rimanere satisfatti e
stupidi.

{[}29{]} Ma torniamo donde noi partimmo. Dico che, trovandosi il Duca
assai potente et in parte assicurato de' presenti periculi, per essersi
armato a suo modo et avere in buona parte spente quelle arme che,
vicine, lo potevano offendere, gli restava, volendo procedere collo
acquisto, el respecto del re di Francia: perché conosceva come dal Re,
il quale tardi s'era accorto dello error suo, non gli sarebbe
sopportato. {[}30{]} E cominciò per questo a cercare di amicizie nuove e
vacillare con Francia, nella venuta che li Franzesi feciono verso el
regno di Napoli contro alli Spagnuoli che assediavono Gaeta; e lo animo
suo era assicurarsi di loro: il che gli sarebbe presto riuscito, se
Alessandro viveva. {[}31{]} E questi furono e governi sua, quanto alle
cose presente.

{[}32{]} Ma, quanto alle future, lui aveva a dubitare im prima che uno
nuovo successore alla Chiesa non gli fussi amico e cercassi torgli
quello che Alessandro li aveva dato. {[}33{]} Di che pensò assicurarsi
in quattro modi: prima, di spegnere tutti e sangui di quelli Signori che
lui aveva spogliati, per tòrre al Papa quella occasione; secondo, di
guadagnarsi tutti e gentili uomini di Roma, come è detto, per potere con
quelli tenere il Papa in freno; terzio, ridurre il Collegio più suo che
poteva; quarto, acquistare tanto imperio, avanti che il papa morissi,
che potessi per sé medesimo resistere a uno primo impeto. {[}34{]} Di
queste quattro cose alla morte di Alessandro ne aveva condotte tre, la
quarta aveva quasi per condotta: perché de' Signori spogliati ne ammazzò
quanti ne poté aggiugnere, e pochissimi si salvorono; e gentili uomini
romani si aveva guadagnati; e nel Collegio aveva grandissima parte; e
quanto al nuovo acquisto, aveva disegnato diventare signore di Toscana,
e possedeva di già Perugia e Piombino, e di Pisa aveva presa la
protectione. {[}35{]} E come non avessi avuto ad avere rispetto a
Francia, -- ché non gliene aveva ad avere più, per essere di già e
Franzesi spogliati del Regno dalli Spagnuoli: di qualità che ciascuno di
loro era necessitato comperare l'amicizia sua, -- egli saltava in Pisa.
{[}36{]} Dopo questo, Lucca e Siena cedeva subito, parte per invidia de'
Fiorentini, parte per paura: Fiorentini non avevano rimedio. {[}37{]} Il
che se gli fusse riuscito, -- che gli riusciva l'anno medesimo che
Alessandro morì, -- si acquistava tante forze e tanta reputazione che
per sé stesso si sarebbe retto e non sarebbe più dependuto dalla fortuna
e forze di altri, ma dalla potenza e virtù sua.

{[}38{]} Ma Alessandro morì dopo cinque anni che egli aveva cominciato a
trarre fuora la spada; lasciollo con lo stato di Romagna solamente
assolidato, con tutti li altri in aria, infra dua potentissimi exerciti
inimici, e malato a morte. {[}39{]} Et era nel Duca tanta ferocità e
tanta virtù, e sì bene conosceva come li uomini si hanno a guadagnare o
perdere, e tanto erano validi e fondamenti che in sì poco tempo si aveva
fatti, che, s'e' non avessi avuto quelli exerciti adosso, o lui fussi
stato sano, arebbe retto ad ogni difficultà.

{[}40{]} E che e fondamenti sua fussino buoni, si vidde: ché la Romagna
lo aspettò più d'uno mese; in Roma, ancora che mezzo vivo, stette
sicuro, e, benché Baglioni Vitelli et Orsini venissino in Roma, non
ebbono seguito contro di lui; poté fare, se non chi e' volle papa,
almeno che non fussi chi egli non voleva. {[}41{]} Ma se nella morte di
Alessandro fussi stato sano, ogni cosa gli era facile: e lui mi disse,
ne' dí che fu creato Iulio II, che aveva pensato a ciò che potessi
nascere morendo el padre, et a tutto aveva trovato remedio, excepto che
non pensò mai, in sulla sua morte, di stare ancora lui per morire.

{[}42{]} Raccolte io adunque tutte le azioni del Duca, non saprei
riprenderlo: anzi mi pare, come io ho fatto, di preporlo imitabile a
tutti coloro che per fortuna e con l'arme di altri sono ascesi allo
imperio; perché lui, avendo l'animo grande e la sua intenzione alta, non
si poteva governare altrimenti, e solo si oppose alli sua disegni la
brevità della vita di Alessandro e la sua malattia. {[}43{]} Chi adunque
iudica necessario nel suo principato nuovo assicurarsi delli inimici,
guadagnarsi delli amici; vincere o per forza o per fraude; farsi amare e
temere da' populi, seguire e reverire da' soldati; spegnere quelli che
ti possono o debbono offendere; innovare con nuovi modi gli ordini
antiqui; essere severo e grato, magnanimo e liberale; spegnere la
milizia infidele, creare della nuova; mantenere l'amicizie de' re e de'
principi in modo che ti abbino a beneficare con grazia o offendere con
respecto, non può trovare e più freschi exempli che le actioni di
costui.

{[}44{]} Solamente si può accusarlo nella creazione di Iulio pontefice,
nella quale il Duca ebbe mala electione. {[}45{]} Perché, come è detto,
non possendo fare uno papa a suo modo, poteva tenere che uno non fussi
papa; e non doveva mai consentire al papato di quelli cardinali che lui
avessi offesi o che, diventati papa, avessino ad aver paura di lui:
perché gli uomini offendono o per paura o per odio. {[}46{]} Quelli che
lui aveva offeso erano, infra li altri, Sancto Pietro ad Vincula,
Colonna, San Giorgio, Ascanio; tutti li altri avevano, divenuti papi, a
temerlo, eccepto Roano e gli Spagnuoli: questi per coniunzione et
obligo, quello per potenza, avendo coniunto seco el regno di Francia.

{[}47{]} Pertanto el Duca innanzi ad ogni cosa doveva creare papa uno
spagnolo; e, non potendo, doveva consentire a Roano, non a San Piero ad
vincula. {[}48{]} E chi crede che nelli personaggi grandi e benefizii
nuovi faccino dimenticare le iniurie vecchie, s'inganna. {[}49{]} Errò
adunque el Duca in questa electione; e fu cagione dell'ultima ruina sua.

\quebra\section{DE HIS QUI PER SCELERA AD PRINCIPATUM PERVENERE
{[}Di quelli che per scelleratezze sono venuti al principato{]}}

{[}1{]} Ma perché di privato si diventa principe ancora in dua modi, il
che non si può al tutto o alla fortuna o alla virtù attribuire, non mi
pare da lasciarli indrieto, ancora che dell'uno si possa più
diffusamente ragionare dove si trattassi delle repubbliche. {[}2{]}
Questi sono quando o per qualche via scellerata e nefaria si ascende al
principato, o quando uno privato ciptadino con el favore degli altri sua
ciptadini diventa principe della sua patria. {[}3{]} E parlando del
primo modo si mosterrà con dua exempli, uno antico, l'altro moderno,
sanza entrare altrimenti ne' meriti di questa parte: perché io iudico
che bastino a chi fussi necessitato imitargli.

{[}4{]} Agatocle siciliano, non solo di privata ma d'infima et abietta
fortuna, divenne re di Siracusa. {[}5{]} Costui, nato d'uno figulo,
tenne sempre, per i gradi della sua età, vita scellerata: nondimanco
accompagnò le sua scelleratezze con tanta virtù di animo e di corpo che,
voltosi alla milizia, per li gradi di quella pervenne ad essere pretore
di Siracusa. {[}6{]} Nel qual grado sendo constituito, et avendo
deliberato diventare principe e tenere con violenzia e sanza obligo di
altri quello che d'accordo gli era suto concesso, et avuto di questo suo
disegno intelligenzia con Amilcare cartaginese, il quale con li exerciti
militava in Sicilia, raunò una mattina il populo et il senato di
Siracusa, come se egli avessi avuto a deliberare cose pertinenti alla
repubblica. {[}7{]} Et ad uno cenno ordinato fece da' sua soldati
uccidere tutti e senatori e li più ricchi del popolo; e quali morti,
occupò e tenne el principato di quella città sanza alcuna controversia
civile. {[}8{]} E benché da' Cartaginesi fussi dua volte rotto e demum
assediato, non solum poté difendere la sua città, ma, lasciato parte
delle sue genti alla difesa della obsidione, con le altre assaltò
l'Affrica et in breve tempo liberò Siracusa dallo assedio e condusse
Cartaginesi in extrema necessità; e furono necessitati accordarsi con
quello, essere contenti della possessione di Affrica, et ad Agatocle
lasciare la Sicilia.

{[}8{]} Chi considerassi adunque le actioni e virtù di costui, non vedrà
cose, o poche, le quali possa attribuire alla fortuna, con ciò sia cosa,
come di sopra è detto, che non per favore di alcuno, ma per li gradi
della milizia, e quali con mille disagi e pericoli si aveva guadagnati,
pervenissi al principato, e quello dipoi con tanti partiti animosi e
periculosissimi mantenessi. {[}10{]} Non si può ancora chiamare virtù
ammazzare e suoi ciptadini, tradire gli amici, essere sanza fede, sanza
pietà, sanza relligione: e quali modi possono fare acquistare imperio ma
non gloria. {[}11{]} Perché, se si considerassi la virtù di Agatocle
nello entrare e nello uscire de' periculi e la grandezza dello animo suo
nel sopportare e superare le cose adverse, non si vede perché egli abbia
ad essere iudicato inferiore a qualunque excellentissimo capitano:
nondimanco la sua efferata crudeltà et inumanità, con infinite
scelleratezze, non consentono che sia infra gli excellentissimi uomini
celebrato. {[}12{]} Non si può adunque attribuire alla fortuna o alla
virtù quello che sanza l'una e l'altra fu da lui conseguito.

{[}13{]} Ne' tempi nostri, regnante Alessandro VI, Liverotto firmano,
sendo più anni innanzi rimaso piccolo, fu da uno suo zio materno,
chiamato Giovanni Fogliani, allevato, e ne' primi tempi della sua
gioventù dato a militare sotto Paulo Vitegli, acciò che, ripieno di
quella disciplina, pervenissi a qualche excellente grado di milizia.
{[}14{]} Morto dipoi Paulo, militò sotto Vitellozzo, suo fratello, et in
brevissimo tempo, per essere ingegnoso e della persona e dello animo
gagliardo, diventò el primo uomo della sua milizia. {[}15{]} Ma,
parendogli cosa servile lo stare con altri, pensò, con lo aiuto di
alcuno ciptadino firmano, alli quali era più cara la servitù che la
libertà della loro patria, e con il favore vitellesco, occupare Fermo.
{[}16{]} E scripse a Giovanni Fogliani come, sendo stato più tempo fuora
di casa, voleva venire a vedere lui e la sua città, e riconoscere in
qualche parte el suo patrimonio; e perché non si era affaticato per
altro che per acquistare onore, acciò che li suoi ciptadini vedessino
come non aveva speso il tempo invano, voleva venire onorevole et
accompagnato da cento cavagli di sua amici e servidori; e pregavalo
fussi contento ordinare che da' Firmiani fussi ricevuto onorevolmente:
il che non solamente tornava onore a se proprio, ma a lui, sendo suo
alunno.

{[}17{]} Non mancò pertanto Giovanni di alcuno offizio debito verso el
nipote, e, fattolo ricevere da' Firmiani onoratamente, si alloggiò nelle
case sua; dove, passato alcuno giorno et atteso ad ordinare segretamente
quello che alla sua futura sceleratezza era necessario, fece uno convito
solennissimo, dove invitò Giovanni Fogliani e tutti li primi uomini di
Fermo. {[}18{]} E consumate che furono le vivande e tutti gli altri
intrattenimenti che in simili conviti si usano, Liverotto ad arte mosse
certi ragionamenti di cose gravi, parlando della grandezza di papa
Alessandro e di Cesare suo figliuolo e delle imprese loro: alli quali
ragionamenti rispondendo Giovanni e gli altri, lui ad uno tratto si
rizzò, dicendo quelle essere cose da ragionarme in luogo più secreto; e
ritirossi in una camera, dove Giovanni e tutti gli altri ciptadini gli
andorono drieto. {[}19{]} Né prima furono posti a sedere che, delli
lochi segreti di quella, uscirono soldati che ammazzorono Giovanni e
tutti gli altri. {[}20{]} Dopo il quale omicidio montò Liverotto a
cavallo e corse la terra et assediò nel palazzo el supremo magistrato:
tanto che per paura furono constretti ubbidirlo e formare uno governo
del quale si fece principe; e morti tutti quelli che per essere
malcontenti lo potevono offendere, si corroborò con nuovi ordini civili
e militari: in modo che, in spazio di uno anno che tenne el principato,
non solamente lui era sicuro nella città di Fermo, ma era diventato
pauroso a tutti e sua vicini. {[}21{]} E sarebbe suta la sua
expugnazione difficile come quella di Agatocle, se non si fussi lasciato
ingannare da Cesare Borgia, quando a Sinigaglia, come di sopra si disse,
prese gli Orsini e Vitelli: dove, preso ancora lui, in uno anno dopo il
commisso parricidio fu insieme con Vitellozzo, il quale aveva avuto
maestro delle virtù e delle scelleratezze sue, strangolato.

{[}22{]} Potrebbe alcuno dubitare donde nascesse che Agatocle et alcuno
simile, dopo infiniti tradimenti e crudeltà, possé vivere lungamente
sicuro nella sua patria e difendersi dalli inimici externi, e dalli suoi
ciptadini non gli fu mai conspirato contro: con ciò sia che molti altri
mediante la crudeltà non abbino, etiam ne' tempi pacifici, potuto
mantenere lo stato, non che ne' tempi dubiosi di guerra. {[}23{]} Credo
che questo advenga dalle crudeltà male usate o bene usate. {[}24{]} Bene
usate si possono chiamare quelle, -- se del male è lecito dire bene, --
che si fanno ad uno tratto per la necessità dello assicurarsi: e dipoi
non vi si insiste dentro, ma si convertono in più utilità de' sudditi
che si può. {[}25{]} Male usate sono quelle le quali, ancora che nel
principio sieno poche, più tosto col tempo crescano che le si spenghino.
{[}26{]} Coloro che observano el primo modo, possono con Dio e con li
uomini avere allo stato loro qualche rimedio, come ebbe Agatocle; quelli
altri è impossibile si mantenghino.

{[}27{]} Onde è da notare che, nel pigliare uno stato, debbe lo
occupatore di epso discorrere tutte quelle offese che gli è necessario
fare, e tutte farle a un tratto, per non le avere a rinnovare ogni dì e
potere, non le innovando, assicurare li uomini e guadagnarseli con
beneficarli. {[}28{]} Chi fa altrimenti, o per timidità o per mal
consiglio, è sempre necessitato tenere il coltello in mano; né mai può
fondarsi sopra e sua subditi, non si potendo quegli, per le fresche e
continue iniurie, mai assicurare di lui. {[}29{]} Per che le iniurie si
debbono fare tutte insieme, acciò che, assaporandosi meno, offendino
meno; e benefizii si debbono fare a poco a poco, acciò si assaporino
meglio. {[}30{]} E debba sopratutto uno principe vivere in modo, con li
suoi subditi, che veruno accidente o di male o di bene lo abbia a fare
variare: perché, venendo per li tempi adversi le necessità, tu non se' a
tempo al male, et il bene che tu fai non ti giova perché è iudicato
forzato, e non te n'è saputo grado alcuno.

\quebra\section{DE PRINCIPATU CIVILI
{[}Del Principato Civile{]}}

{[}1{]} Ma venendo all'altra parte, quando uno privato ciptadino, non
per scelleratezza o altra intollerabile violenzia, ma con il favore
delli altri sua ciptadini diventa principe della sua patria, -- il quale
si può chiamare principato civile: né a pervenirvi è necessario o tutta
virtù o tutta fortuna, ma più presto una astuzia fortunata, -- dico che si ascende a questo principato o con il favore del populo o con de' grandi. {[}2{]} Perché in ogni città si truovano questi dua umori diversi: e nasce, da questo, che il populo desidera non essere comandato et oppresso da' grandi e li grandi desiderano comandare et opprimere el populo; e da questi dua appetiti diversi nasce nelle città uno de' tre effetti: o principato o libertà o licenza. {[}3{]} El principato è causato o dal populo o da' grandi, secondo che l'una o l'altra di queste parti ne ha l'occasione: perché, vedendo e grandi non potere resistere al populo, cominciano a voltare la reputazione ad uno di loro e fannolo principe per potere sotto la sua ombra sfogare il loro appetito; il populo ancora, vedendo non potere resistere a' grandi, volta la eputazione ad uno e lo fa principe per essere con la sua autorità difeso.

{[}4{]} Colui che viene al principato con lo aiuto de' grandi, si
mantiene con più difficultà che quello che diventa con lo aiuto del
populo, perché si truova principe con dimolti intorno che gli paiano
essere sua equali, e per questo non gli può né comandare né maneggiare a
suo modo. {[}5{]} Ma colui che arriva al principato con il favore
popolare, vi si truova solo et ha dintorno o nessuno o pochissimi che
non sieno parati a ubbidire.

{[}6{]} Oltre a questo non si può con onestà satisfare a' grandi, e
sanza iniuria di altri, ma sì bene al populo: perché quello del populo è
più onesto fine che quello de' grandi, volendo questi opprimere e quello
non essere oppresso. {[}7{]} Preterea, del populo inimico uno principe
non si può mai assicurare, per essere troppi: de' grandi si può
assicurare, per essere pochi. {[}8{]} Il peggio che possa aspettare uno
principe dal populo inimico, è lo essere abbandonato da lui; ma da'
grandi, inimici, non solo debba temere di essere abbandonato, ma etiam
che loro gli venghino contro: perché, essendo in quelli più vedere e più
astuzia, avanzono sempre tempo per salvarsi e cercono gradi con quelli
che sperano che vinca. {[}9{]} È necessitato ancora el principe vivere
sempre con quello medesimo populo, ma può ben fare sanza quelli medesimi
grandi, potendo farne e disfarne ogni dí e tòrre e dare a sua posta
reputazione loro.

{[}10{]} E per chiarire meglio questa parte, dico come e grandi si
debbono considerare in dua modi principalmente: o si governano in modo
col procedere loro che si obligano in tutto alla tua fortuna, o no.
{[}11{]} Quegli che si obligano, e non sieno rapaci, si debbono onorare
et amare. {[}12{]} Quelli che non si obligano, si hanno ad examinare in
dua modi: o e' fanno questo per pusillanimità e difetto naturale
d'animo; allora tu te ne debbi servire, maxime di quelli che sono di
buono consiglio, perché nelle prosperità te ne onori e non hai nelle
avversità a temere di loro. {[}13{]} Ma quando e' non si obligano per
arte e per cagione ambiziosa, è segno come pensano più a sé che a te: e
da quelli si debbe el principe guardare, e temergli come se fussino
scoperti nimici, perché sempre nelle adversità aiuteranno ruinarlo.

{[}14{]} Debba pertanto uno, che diventi principe mediante el favore del
populo, mantenerselo amico: il che gli fia facile, non domandando lui se
non di non essere oppresso. {[}15{]} Ma uno che, contro al populo,
diventi principe con il favore de' grandi, debbe innanzi ad ogni altra
cosa cercare di guadagnarsi el populo: il che gli fia facile, quando
pigli la protectione sua. {[}16{]} E perché li uomini, quando hanno bene
da chi credevano avere male, si obligano più al beneficatore loro,
diventa el populo subito più suo benivolo che s'e' si fussi condotto al
principato con favori sua. {[}17{]} E puosselo guadagnare el principe in
molti modi: e quali perché variano secondo el subiecto, non se ne può
dare certa regula, e però si lasceranno indrieto. {[}18{]} Concluderò
solo che a uno principe è necessario avere il populo amico, altrimenti
non ha nelle avversità remedio. {[}19{]} Nabide principe delli Spartani
sostenne la obsidione di tutta Grecia e di uno exercito romano
vittoriosissimo, e difese contro a quelli la patria sua et il suo stato;
e gli bastò solo, sopravvenente il periculo, assicurarsi di pochi: ché,
se egli avessi avuto el populo inimico, questo non li bastava. {[}20{]}
E non sia alcuno che repugni a questa mia opinione con quello proverbio
trito, che chi fonda in sul populo fonda in sul fango: perché quello è
vero quando uno ciptadino privato vi fa su fondamento e dassi ad
intendere che il populo lo liberi, quando fussi oppresso dalli nimici o
da' magistrati. {[}21{]} In questo caso si potrebbe trovare spesso
ingannato, come a Roma e Gracchi et a Firenze messer Giorgio Scali.
{[}22{]} Ma essendo uno principe che vi fondi su, che possa comandare e
sia uomo di core, né si sbigottisca nelle adversità, e non manchi delle
altre preparazione e tenga con lo animo et ordini suoi animato
l'universale, mai si troverrà ingannato da lui e gli parrà avere fatto
li suo fondamenti buoni. {[}23{]} Sogliono questi principati periclitare
quando sono per salire dallo ordine civile allo assoluto. {[}24{]}
Perché questi principi o comandano per loro medesimi o per mezzo delli
magistrati: nello ultimo caso è più debole e più periculoso lo stato
loro, perché gli stanno al tutto con la volontà di quelli ciptadini che
a' magistrati sono preposti: e quali, maxime ne' tempi adversi, gli
possono torre con facilità grande lo stato, o con abbandonarlo o con
fargli contro. {[}25{]} Et il principe non è a tempo ne' pericoli, a
pigliare l'autorità assoluta, perché e ciptadini e subditi, che sogliono
avere e' comandamenti da' magistrati, non sono in quelli frangenti per
ubbidire a' suoi. {[}26{]} Et arà sempre ne' tempi dubii penuria di chi
si possa fidare; perché simile principe non può fondarsi sopra quello
che vede ne' tempi quieti, quando e' ciptadini hanno bisogno dello
stato: perché allora ognuno corre, ognuno promette, e ciascuno vuole
morire per lui, quando la morte è discosto; ma ne' tempi adversi, quando
lo stato ha bisogno de' ciptadini, allora se ne truova pochi. {[}27{]} E
tanto più è questa experienzia periculosa, quanto la non si può fare se
non una volta: però uno principe savio debbe pensare uno modo per il
quale li sua ciptadini, sempre et in ogni qualità di tempo, abbino
bisogno dello stato e di lui; e sempre dipoi gli saranno fedeli.

\quebra\section{QUOMODO OMNIUM PRINCIPATUUM VIRES PERPENDI DEBEANT
{[}In che modo si debbino misurare le forze di tutti i principati{]}}

{[}1{]} Conviene avere, nello examinare le qualità di questi principati,
un'altra considerazione: cioè se uno principe ha tanto stato che possa,
bisognando, per sé medesimo reggersi, o vero se ha sempre necessità
della defensione d'altri. {[}2{]} E per chiarire meglio questa parte,
dico come io iudico coloro potersi reggere per sé medesimi che possono,
o per abbondanzia di uomini o di denari, mettere insieme un exercito
iusto e fare una giornata con qualunque lo viene ad assaltare. {[}3{]} E
così iudico coloro avere sempre necessità di altri, che non possono
comparire contro al nimico in campagna, ma sono necessitati rifuggirsi
dentro alle mura e guardare quelle.

{[}4{]} Nel primo caso, si è discorso e per lo advenire direno quello ne
occorre. {[}5{]} Nel secondo caso, non si può dire altro salvo che
confortare tali principi a fortificare e munire la terra propria e del
paese non tenere alcuno conto. {[}6{]} E qualunque arà bene fortificata
la sua terra e, circa alli altri governi, con li subditi si sarà
maneggiato come di sopra è detto e di sotto si dirà, sarà sempre con
gran respetto assaltato; perché li uomini sono nimici delle imprese dove
si vegga difficultà: né si può vedere facilità assaltando uno che abbia
la suo terra gagliarda e non sia odiato dal populo.

{[}7{]} Le città della Magna sono liberissime, hanno poco contado et
obbediscano allo Imperatore quando le vogliono, e non temono né quello
né alcuno altro potente che le abbino intorno. {[}8{]} Perché le sono in
modo affortificate che ciascuno pensa la expugnazione di epse dovere
essere tediosa e difficile: perché tutte hanno fossi e mura conveniente;
hanno artiglieria a sufficienzia; tengono sempre nelle canove publiche
da bere e da mangiare e da ardere per uno anno; {[}9{]} et oltre a
questo, per potere tenere la plebe pasciuta e sanza perdita del publico,
hanno sempre in comune da potere per uno anno dare loro da lavorare
loro, in quelli exercizii che sieno el nervo e la vita di quella città e
delle industrie de' quali la plebe pasca; tengono ancora li exercizii
militari in reputazione, e sopra questo hanno molti ordini a
mantenergli.

{[}10{]} Uno principe adunque, che abbia una città così ordinata e non
si facci odiare, non può essere assaltato; e, se pure fussi chi lo
assaltassi, se ne partire', con vergogna: perché le cose del mondo sono
sì varie che egli è impossibile che uno potessi con li exerciti stare
uno anno ocioso a campeggiarlo. {[}11{]} E chi replicassi: se il populo
arà le sua possessioni fuora e veggale ardere, non ci arà pazienza, et
il lungo assedio e la carità propria gli farà sdimenticare lo amore del
principe; rispondo che uno principe prudente et animoso supererà sempre
tutte quelle difficultà, dando a' subditi ora speranza che il male non
fia lungo, ora timore della crudeltà del nimico, ora assicurandosi con
destrezza di quegli che gli paressino troppo arditi. {[}12{]} Oltre a
questo, el nimico ragionevolmente debba ardere e ruinare el paese in su
la sua giunta e nelli tempi quando gli animi degli uomini sono ancora
caldi e volenterosi alla difesa: e però tanto meno el principe debba
dubitare, perché dopo qualche giorno, che gli animi sono raffreddi, sono
di già fatti e danni, sono ricevuti e mali, non vi è più remedio.
{[}13{]} Et allora tanto più si vengono ad unire con il loro principe,
parendo che lui abbia con loro obligo, sendo loro sute arse le case,
ruinate le possessioni per la difesa sua: e la natura delli uomini è
così obligarsi per li benefizii che si fanno, come per quelli che si
ricevano. {[}14{]} Onde, se si considerrà bene tutto, non fia difficile
ad uno principe prudente tenere, prima e poi, fermi gli animi de' sua
ciptadini nella obsidione, quando non vi manchi né da vivere né da
difendersi.

\quebra\section{DE PRINCIPATIBUS ECCLESIASTICIS.
{[}De' principati ecclesiastici{]}}

{[}1{]} Restaci solamente al presente a ragionare de' principati
ecclesiastici, circa quali tutte le difficultà sono avanti che si
possegghino; perché si acquistano o per virtù o per fortuna, e sanza
l'una e l'altra si mantengono: perché sono subtentati dalli ordini
antiquati nella religione, quali sono stati tanto potenti e di qualità
che tengono e loro principi in stato in qualunque modo si procedino e
vivino. {[}2{]} Costoro soli hanno stati e non gli difendano; hanno
subditi, e non li governano. {[}3{]} E gli stati, per essere indifesi,
non sono loro tolti; et e subditi, per non essere governati, non se ne
curano, né pensano, né possono alienarsi da loro. {[}4{]} Solo adunque
questi principati sono sicuri e felici; ma, essendo quelli retti da
cagione superiori, alle quali mente umana non aggiugne, lascerò il
parlarne: perché, essendo exaltati e mantenuti da Dio, sarebbe officio
di uomo presumptuoso e temerario discorrerne. {[}5{]} Nondimanco, se
alcuno mi ricercassi donde viene che la Chiesa nel temporale sia venuta
a tanta grandezza, -- con ciò sia che da Alessandro indrieto e potentati
italiani, e non solum quelli che si chiamavono e potentati ma ogni
barone e signore benché minimo, quanto al temporale, la extimava poco,
et ora uno re di Francia ne trema, e lo ha possuto cavare di Italia e
ruinare Viniziani, -- la qual cosa, ancora che sia nota, non mi pare
superfluo ridurla in buona parte alla memoria.

{[}6{]} Avanti che Carlo re di Francia passassi in Italia, era questa
provincia sotto lo imperio del Papa, Viniziani, re di Napoli, duca di
Milano e Fiorentini. {[}7{]} Questi potentati avevano ad avere dua cure
principali: l'una, che uno forestiero non entrassi in Italia con le
arme; l'altra, che veruno di loro occupassi più stato. {[}8{]} Quegli a
chi si aveva più cura erano Papa e Viniziani; et a tenere indrieto e
Viniziani, bisognava la unione di tutti li altri, come fu nella difesa
di Ferrara; et a tenere basso il Papa, si servivono de' baroni di Roma,
li quali sendo divisi in due factioni, Orsine e Colonnese, sempre vi era
cagione di scandolo fra loro; e, stando con le arme in mano in su li
occhi al Pontefice, tenevano il pontificato debole et infermo. {[}9{]}
E, benché surgessi qualche volta alcuno papa animoso, come fu Sixto,
tamen la fortuna o il sapere non lo poté mai disobligare da queste
incommodità. {[}10{]} E la brevità della vita loro n'era cagione; perché
in dieci anni che, raguagliato, uno papa viveva, a fatica ch'é potessi
abassare una delle factioni; e se, verbi gratia, l'uno aveva quasi
spenti Colonnesi, surgeva un altro inimico agli Orsini, che gli faceva
risurgere e li Orsini non era a tempo a spegnere. {[}11{]} Questo faceva
che le forze temporali del Papa erano poco stimate in Italia.

{[}12{]} Surse dipoi Alessandro VI, il quale, di tutt'i pontefici che
sono mai stati, mostrò quanto uno papa e con danaio e con le forze si
poteva prevalere; e fece, con lo instrumento del duca Valentino e con la
occasione della passata de' Franzesi, tutte quelle cose che io discorro
di sopra nelle acyioni del Duca. {[}13{]} E benché la intenzione suo non
fussi fare grande la Chiesa, ma il duca, nondimeno ciò che fece tornò a
grandezza della Chiesa; la quale dopo la sua morte, spento il Duca, fu
erede delle sua fatiche.

{[}14{]} Venne dipoi papa Iulio e trovò la Chiesa grande, avendo tutta
la Romagna e essendo spenti e baroni di Roma e, per le battiture di
Alessandro, annullate quelle fazioni; e trovò ancora la via aperta al
modo dello accumulare danari, non mai più usitato da Alessandro
indrieto. {[}15{]} Le quali cose Iulio non solum seguitò, ma accrebbe, e
pensò a guadagnarsi Bologna e spegnere e Viniziani et a cacciare
Franzesi di Italia: e tutte queste imprese gli riuscirono, e con tanta
più sua laude, quanto lui fece ogni cosa per adcrescere la Chiesa e non
alcuno privato. {[}16{]} Mantenne ancora le parte Orsine e Colonnese in
quelli termini le trovò. {[}17{]} E benché fra loro fussi qualche capo
da fare alterazione, tamen dua cose gli ha tenuti fermi: l'una, la
grandezza della Chiesa, che gli sbigottisce; l'altra, il non avere loro
cardinali, i quali sono origine delli tumulti intra loro; né mai
staranno quiete, qualunque volta queste parti abbino cardinali, perché
questi nutriscono, in Roma e fuori, le parte, e quelli baroni sono
forzati a difenderle; e così, dalla ambizione de' prelati nascono le
discordie e li tumulti intra baroni. {[}18{]} Ha trovato adunque la
Sanctità di papa Leone questo pontificato potentissimo: il quale si
spera, se quegli lo feciono grande con le arme, questo, con la bontà et
infinite altre sue virtù lo farà grandissimo e venerando.

\quebra\section{QUOT SUNT GENERA MILITIAE ET DE MERCENNARIIS MILITIBUS
{[}Di quante ragioni sia la milizia, e de' soldati mercennarii{]}}

{[}1{]} Avendo discorso particularmente tutte le qualità di quelli
principati de' quali nel principio proposi di ragionare, e considerato
in qualche parte le cagioni del bene e del male essere loro, e mostro e
modi con li quali molti hanno cerco di acquistargli e tenergli, mi resta
ora a discorrere generalmente le offese e difese che in ciascuno de'
prenominati possono accadere.

{[}2{]} Noi abbiamo detto di sopra come a uno principe è necessario
avere e sua fondamenti buoni, altrimenti di necessità conviene che
ruini. {[}3{]} E principali fondamenti che abbino tutti li stati, così
nuovi come vecchi o mixti, sono le buone legge e le buone arme: e perché
non può essere buone legge dove non sono buone arme, e dove sono buone
arme conviene sieno buone legge, io lascerò indrieto el ragionare delle
legge e parlerò delle arme.

{[}4{]} Dico adunque che l'arme con le quali uno principe difende el suo
stato o le sono proprie, o le sono mercennarie o auxiliare o mixte.
{[}5{]} Le mercennarie et auxiliarie sono inutile e periculose; e se uno
tiene lo stato suo fondato in su l'arme mercennarie, non starà mai fermo
né sicuro, perché le sono disunite, ambiziose, sanza disciplina,
infedele, gagliarde infra gli amici, infra ` nimici vile: non timore di
Dio, non fede con li uomini; e tanto si differisce la ruina, quanto si
differisce lo assalto; e nella pace se' spogliato da loro, nella guerra
dagli inimici. {[}6{]} La cagione di questo è che le non hanno altro
amore né altra cagione che le tenga in campo che uno poco di stipendio,
il quale non è suffiziente a fare che voglino morire per te. {[}7{]}
Vogliono bene essere tua soldati mentre che tu non fai guerra; ma, come
la guerra viene, o fuggirsi o andarsene. {[}8{]} La qual cosa doverrei
durare poca fatica a persuadere, perché ora la ruina di Italia non è
causata da altro che per essersi per spazio di molti anni riposata tutta
in sulle arme mercennarie. {[}9{]} Le quali feciono già per alcuno
qualche progresso, e parevano gagliarde infra loro; ma come venne el
forestiero le mostrorono quello che elle erano: onde che a Carlo re di
Francia fu lecito pigliare la Italia col gesso; e chi diceva come
n'erano cagione e peccati nostri, diceva il vero; ma non erano già
quegli che credeva, ma questi che io ho narrati; e perché gli erano
peccati di principi, ne hanno patito le pene ancora loro.

{[}10{]} Io voglio dimonstrare meglio la infelicità di queste arme. E
capitani mercennarii o e' sono uomini excellenti, o no; se sono, non te
ne puoi fidare, perché sempre aspireranno alla grandezza propria, o con
lo opprimere te, che gli se' patrone, o con opprimere altri fuora della
tua intenzione; ma se il capitano non è virtuoso, ti rovina per
l'ordinario. {[}11{]} E se si responde che qualunque arà le arme in mano
farà questo, o mercennario o no, replicherrei come l'arme hanno ad
essere operate o da uno principe o da una repubblica: el principe debbe
andare in persona, e fare lui l'offizio del capitano; la repubblica ha a
mandare sua ciptadini; e, quando ne manda uno che non riesca valente
uomo, debbe cambiarlo; e, quando sia, tenerlo con le leggi che non passi
el segno. {[}12{]} E per experienzia si vede alli principi soli e
republiche armate fare progressi grandissimi, et alle arme mercennarie
non fare mai se non danno; e con più difficultà viene alla obbedienza di
uno suo cittadino una repubblica armata di arme proprie, che una armata
di armi externe.

{[}13{]} Stettono Roma e Sparta molti seculi armate e libere. Svizzeri
sono armatissimi e liberissimi. {[}14{]} Delle arme mercennarie antiche
sono in exemplis e Cartaginesi, li quali furono per essere oppressi da'
loro soldati mercennarii, finita la loro prima guerra con i Romani,
ancora che li Cartaginesi avessino, per capitani loro proprii ciptadini
{[}15{]} Filippo Macedone fu fatto da' Tebani, dopo la morte di
Epaminunda, capitano di loro genti: e tolse loro, dopo la vittoria, loro
la libertà

{[}16{]} Milanesi, morto il duca Filippo, soldorono Francesco Sforza
contro a' Viniziani: il quale, superati gli inimici a Caravaggio, si
coniunse con loro per opprimere e Milanesi suoi patroni. {[}17{]} Sforza
suo padre, essendo soldato della regina Giovanna di Napoli, la lasciò in
un tratto disarmata: onde lei, per non perdere el regno, fu constretta
gittarsi in grembo al re di Aragonia. {[}18{]} E se Viniziani e
Fiorentini hanno per lo adrieto accresciuto lo imperio loro con queste
arme, e li loro capitani non se ne sono però fatti principi ma gli hanno
difesi, rispondo che e Fiorentini in questo caso sono suti favoriti
dalla sorte: perché, de' capitani virtuosi de' quali potevano temere,
alcuni non hanno vinto, alcuni hanno avuto opposizione, alcuni altri
hanno volto l'ambizione loro altrove.

{[}19{]} Quello che non vinse fu Giovanni Aucut, del quale, non
vincendo, non si poteva conoscere la fede: ma ognuno confesserà che,
vincendo, stavano e Fiorentini a sua discrezione. {[}20{]} Sforzo ebbe
sempre e Bracceschi contrarii, che guardorono l'uno l'altro. {[}21{]}
Francesco volse l'ambizione sua in Lombardia; Braccio, contro alla
Chiesa et il regno di Napoli.

{[}22{]} Ma veniamo a quello che è seguito poco tempo fa Feciono e
Fiorentini Paulo Vitelli loro capitano, uomo prudentissimo e che di
privata fortuna aveva presa grandissima reputazione; se costui expugnava
Pisa, veruno fia che neghi come conveniva a' Fiorentini stare, seco:
perché, se fussi diventato soldato de loro nimici, non avevano remedio;
e, se e' Fiorentini lo tenevano, aveano ad ubbidirlo.

{[}23{]} E Viniziani, se si considerrà e progressi loro, si vedrà quegli
avere sicuramente e gloriosamente operato mentre feciono la guerra loro
proprii, -- che fu avanti che si volgessino con le loro imprese loro in
terra, -- dove co' gentili uomini e con la plebe armata operorono
virtuosissimamente; ma, come cominciorono a combattere in terra,
lasciorono questa virtù e seguitorono e costumi delle guerre di Italia.
{[}24{]} E nel principio dello augumento loro in terra, per non vi avere
molto stato e per essere in grande reputazione, non avevono da temere
molto de' loro capitani. {[}25{]} Ma, come eglino ampliorono, che fu
sotto el Carmignola, ebbono uno saggio di questo errore: perché,
vedutolo virtuosissimo, battuto che loro ebbono sotto il suo governo il
duca di Milano, e conoscendo dall'altra parte come egli era raffreddo
nella guerra, iudicorono non potere con lui più vincere, perché non
voleva; né potere licenziarlo, per non riperdere ciò che aveano
acquistato; onde che furono necessitati, per assicurarsene, ammazzarlo.
{[}26{]} Hanno dipoi avuto per loro capitani Bartolomeo da Bergomo,
Ruberto da Sancto Severino, conte di Pitigliano, e simili, con li quali
avevano a temere della perdita, non del guadagno loro: come intervenne
dipoi a Vailà, dove in una giornata, perderono cio che in ottocento anni
con tanta fatica, avevono acquistato: perché da queste arme nascono solo
e lenti, tardi e deboli acquisti e le subite e miracolose perdite.

{[}27{]} E perché io sono venuto con questi exempli in Italia, la quale
è stata molti anni governata dalle arme mercennarie, io le voglio
discorrere più da alto acciò che, veduta l'origine e progressi di esse,
si possa meglio correggerle. {[}28{]} Avete adunque ad intendere come,
tosto che in questi ultimi tempi lo Imperio cominciò ad essere ributtato
di Italia e che il papa nel temporale vi prese più reputazione, si
divise la Italia in più stati: perché molte delle città grosse presono
l'arme contra a' loro nobili, e quali prima, favoriti dallo Imperatore,
le tennono oppresse, e la Chiesa le favoriva per darsi reputazione nel
temporale; di molte altre e loro ciptadini ne diventorono principi.
{[}29{]} Onde che, essendo venuta l'Italia quasi che nelle mani della
Chiesa e di qualche republica, et essendo quelli preti e quelli altri
ciptadini usi a non conoscere arme, cominciorono a soldare forestieri.
{[}30{]} El primo che dette reputazione a questa milizia fu Alberigo da
Conio, romagnuolo: dalla disciplina di costui discese intra gli altri
Braccio e Sforza, che ne' loro tempi furono arbitri di Italia. {[}31{]}
Dopo questa vennono tutti li altri che in fino alli nostri tempi hanno
governato queste arme: e'l fine della loro virtù è stato che Italia è
stata corsa da Carlo, predata da Luigi, sforzata da Fernando e
vituperata da' Svizzeri.

{[}32{]} L'ordine che gli hanno tenuto è stato prima, per dare
reputazione a loro proprii, avere tolto reputazione alle fanterie:
Feciono questo perché, sendo sanza stato et in sulla industria, e pochi
fanti non davano loro reputazione e gli assai non potevano nutrire; e
però si redussono a' cavagli, dove con numero sopportabile erano nutriti
et onorati: et erono ridotte le cose in termine che in uno exercito di
XX mila soldati non si trovava dumila fanti. {[}33{]} Avevano oltre a
questo usato ogni industria per levare a sé et a' soldati la paura e la
fatica, non si ammazzando nelle zuffe, ma pigliandosi prigioni e sanza
taglia; non traevano la notte alle terre; quegli della terre non
traevano alle tende; non facevano intorno al campo né steccato né fossa;
non campeggiavano el verno. {[}34{]} E tutte queste cose erano permesse
nelli loro ordini militari e trovate da loro per fuggire, come è detto,
la fatica e li pericoli: tanto che gli hanno condotta la Italia stiava e
vituperata.

\quebra\section{DE MILITIBUS AUXILIARIIS, MIXTIS ET PROPRIIS
{[}De' soldati aussiliari, delli aussiliari e propri insieme, e de' propri soli{]}}

{[}1{]} L'armi auxiliare, che sono l'altre arme inutili, sono quando si
chiama uno potente che con le sua arme ti venga a difendere, come fece
nelli proximi tempi papa Iulio: il quale, avendo visto nella impresa di
Ferrara la trista pruova delle sue arme mercennarie, si volse alle
auxiliare, e convenne con Ferrando re di Spagna che con le sua gente et
exerciti dovesse aiutarlo. {[}2{]} Queste arme possono essere buone e
utile per loro medesime, ma sono, per chi le chiama, quasi sempre
dannose: perché, perdendo rimani disfatto; vincendo, resti loro
prigione. {[}3{]} Et ancora che di questi exempli ne sieno piene le
antiche storie, nondimando io non mi voglio partire da questo exemplo
fresco di Iulio II: el partito del quale non poté essere meno
considerato, per voler Ferrara, cacciarsi tutto nelle mani d'uno
forestieri. {[}4{]} Ma la sua buona fortuna fece nascere una terza cosa,
acciò non cogliessi el frutto della sua mala electione: perché, sendo
gli auxiliari suoi rotti a Ravenna, e surgendo e Svizzeri che cacciorono
e vincitori fuora di ogni opinione e sua e d'altri, venne a non rimanere
prigione delli inimici, sendo fugati, né delli auxiliarii sua, avendo
vinto con altre arme che con le loro. {[}5{]} Fiorentini, sendo al tutto
disarmati, condussono diecimila Franzesi a Pisa per expugnarla: per il
quale partito portorono più pericolo che in qualunque tempo de' travagli
loro. {[}6{]} Lo imperatore di Constantinopoli, per opporsi alli suoi
vicini, misse in Grecia diecimila Turchi, li quali finita la guerra non
se ne volsono partire: il che fu il principio della servitù di Grecia
con gli infedeli.

{[}7{]} Colui adunque che vuole non potere vincere, si vaglia di queste
arme, perché sono molto più pericolose che le mercennarie. {[}8{]}
Perché in queste è la coniura fatta: sono tutte unite, tutte volte alla
obbedienza d'altri; ma nelle mercennarie ad offenderti, vinto che
l'hanno, bisogna maggiore occasione più tempo, non sendo tutte uno corpo
et essendo trovate e pagate da te: nelle quale un terzo che tu facci
capo non può pigliare subitamente tanta autorità che ti offenda. {[}9{]}
Insomma nelle mercennarie è più pericolosa la ignavia, nelli auxiliare
la virtù. {[}10{]} Uno principe pertanto savio sempre ha fuggito queste
arme e voltosi alle proprie: et ha voluto più tosto perdere con li suoi
che vincere con li altri, iudicando non vera vittoria quella che con le
armi aliene si acquistassi.

{[}11{]} Io non dubiterò mai di allegare Cesare Borgia e le sue actioni
Questo duca intrò in Romagna con le armi auxiliare, conducendovi tutte
gente franzese, e con quelle prese Imola e Furlí, ma non gli parendo poi
tale arme sicure, si volse alle mercennarie, iudicando in quelle meno
pericolo, e soldò gli Orsini e Vitelli; le qual dipoi trovando,
maneggiare nel dubbie infideli e pericolose, le spense e volsesi alle
proprie. {[}12{]} E puossi facilmente vedere che differenzia è fra l'una
e l'altra di queste arme, considerato che differenzia fu dalla
reputazione del Duca quando aveva Franzesi soli, a quando aveva gli
Orsini e Vitelli, a quando e' rimase con li soldati sua e sopra se
stesso: e sempre si troverrà accresciuta, né mai fu stimato assai se non
quando ciascuno vidde come lui era intero possessore delle sua arme.

{[}13{]} Io non mi volevo partire dalli exempli italiani e freschi:
tamen non voglio lasciare indrieto Ierone siracusano, sendo uno delli
sopra nominati da me. {[}14{]} Costui, come io dixi, fatto dalli
Siracusani capo degli exerciti, conobbe subito quella milizia
mercennaria non essere utile, per essere condottieri fatti come li
nostri italiani; e parendoli non gli potere tenere né lasciare, gli fece
tutti tagliare a pezzi, e dipoi fece guerra con le arme sua e non con le
aliene. {[}15{]} Voglio ancora ridurre a memoria una figura del
Testamento vecchio, fatta a questo proposito. {[}16{]} Offerendosi Davit
a Saul d'andare a combattere con Golia provocatore filisteo, Saul per
dargli animo l'armò dell'arme sua: le quali Davit come l'ebbe indosso,
recusò, dicendo con quelle non si potere bene valere di sé stesso; e
però voleva trovare el nimico con la sua fromba e con il suo coltello.
{[}17{]} Infine, le arme di altri o le ti caggiono di dosso o le ti
pesano o le ti stringano.

{[}18{]} Carlo VII, padre del re Luigi XI, avendo con la sua fortuna e
virtù libera la Francia dagli Inghilesi, conobbe questa necessità di
armarsi di arme proprie et ordinò nel suo regno l'ordinanza delle genti
d'arme e delle fanterie. {[}19{]} Dipoi el re Luigi suo figliuolo spense
quella de' fanti e cominciò a soldare Svizzeri: il quale errore
seguitato dalli altri è, come si vede ora in fatto, cagione de' pericoli
di quello regno. {[}20{]} Perché, avendo dato reputazione a' Svizzeri,
ha invilito tutte le arme sua; perché le fanterie ha spente in tutto e
le sua gente d'arme ha obligate alla virtù di altri: perché, sendo
assuefatte a militare con Svizzeri, non pare loro potere vincere sanza
epsi. {[}21{]} Di qui nasce che li Franzesi contro a Svizzeri non
bastano e sanza Svizzeri, contro ad altri, non pruovano. {[}22{]} Sono
dunque stati gli exerciti di Francia mixti, parte mercennarii e parte
proprii: le quali arme tutte insieme sono molto migliori che le semplice
auxiliare o semplice mercennarie, e molto inferiore alle proprie.
{[}23{]} E basti lo exemplo detto; perché el regno di Francia sarebbe
insuperabile, se l'ordine di Carlo era adcresciuto o preservato; ma la
poca prudenza delli uomini comincia una cosa che, per sapere allora di
buono, non si accorge del veleno che vi è sotto, come io dissi di sopra
delle febbre etiche. {[}24{]} Pertanto colui che in uno principato non
conosce e mali quando nascono, non è veramente savio: e questo è dato a
pochi. {[}25{]} E, se si considerassi la prima cagione della ruina dello
imperio romano, si troverrà essere suto solo cominciare a soldare e
Gotti: perché da quello principio cominciorono ad enervare le forze
dello imperio; e tutta quella virtù, che si levava da lui, si dava a
loro.

{[}26{]} Concludo adunque che, sanza avere arme proprie, nessuno
principato è sicuro, anzi è tutto obligato alla fortuna, non avendo
virtù che nelle adversità con fede lo difenda: e fu sempre opinione e
sentenza delli uomini savi, quod nihil sit tam infirmum aut instabile
quam fama potentiae non sua vi nixa. {[}27{]} E l'arme proprie sono
quelle che sono composte o di subditi o di ciptadini o di creati tua:
tutte l'altre sono o mercennarie o auxiliare et il modo ad ordinare
l'arme proprie sarà facile a trovare, se si discorrerà gli ordini de'
quattro sopra nominati da me, e se si vedrà come Filippo, padre di
Alessandro Magno, e come molte repubbliche e principi si sono armati et
ordinati: a' quali ordini io al tutto mi rimetto.

\quebra\section{QUOD PRINCIPEM DECEAT CIRCA MILITIAM
{[}Quello che s'appartenga a uno principe circa la milizia{]}}

{[}1{]} Debbe dunque uno principe non avere altro obietto né altro
pensiero né prendere cosa alcuna per sua arte, fuora della guerra et
ordini e disciplina di epsa: perché quella è sola arte che si espetta a
chi comanda, et è di tanta virtù che non solamente mantiene quelli che
sono nati principi, ma molte volte fa gli uomini di privata fortuna
salire a quello grado. {[}2{]} E per adverso si vede che, quando e
principi hanno pensato più alle delicatezze che alle arme, hanno perso
lo stato loro: e la prima cagione che ti fa perdere quello è negligere
questa arte, e la cagione che te lo fa acquistare è lo essere professo
di questa arte. {[}3{]} Francesco Sforza, per essere armato, di privato
diventò duca di Milano; e figliuoli, per fuggire e disagi delle arme, di
duchi diventorono privati. {[}4{]} Perché, intra le altre cagioni che ti
arreca di male, lo essere disarmato ti fa contennendo, la quale è una di
quelle infamie delle quali el principe si debbe guardare, come di sotto
si dirà. {[}6{]} Perché da uno armato a uno disarmato non è proporzione
alcuna, e non è ragionevole che chi è armato ubbedisca volentieri a chi
è disarmato, e che el disarmato stia sicuro intra servitori armati:
perché, sendo nell'uno sdegno e nell'altro sospetto, non è possibile
operino bene insieme. {[}6{]} E però uno principe che della milizia non
si intenda, oltre alle altre infelicità, come è detto, non può essere
stimato dalli suoi soldati né fidarsi di loro.

{[}7{]} Debbe pertanto mai levare il pensiero da questo exercizio della
guerra; e nella pace vi si debbe più exercitare che nella guerra; il che
può fare in dua modi: l'uno con le opere; l'altro, con la mente. {[}8{]}
E quanto alle opere, oltre al tenere bene ordinati et exercitati i suoi,
debba stare sempre in sulle cacce: e mediante quelle assuefare il corpo
a' disagi, e parte imparare la natura de' siti, e conoscere come surgono
e' monti, come imboccano le valle, come iacciono i piani, et intendere
la natura de' fiumi e de' paduli, et in questo porre grandissima cura.
{[}9{]} La quale cognizione è utile in dua modi: prima, s'impara a
conoscere el suo paese, e può meglio intendere le difese di epso; dipoi,
mediante la cognizione e pratica di quegli siti, con facilità
comprendere ogni altro sito che di nuovo gli sia necessario speculare:
perché li poggi, le valle, e piani, e fiumi, e paduli che sono, verbi
gratia, in Toscana hanno con quelli dell'altre provincie certa
similitudine, tale che dalla cognizione del sito di una provincia si può
facilmente venire alla cognizione dell'altre. {[}10{]} E quel principe
che manca di questa perizie, manca della prima parte che vuole avere uno
capitano: perché questa t'insegna trovare el nimico, pigliare gli
alloggiamenti, condurre gli exerciti, ordinare le giornate, campeggiare
le terre con tuo vantaggio.

{[}11{]} Filopemene, principe delli Achei, intra le altre laude che
dagli scriptori gli sono date, è che ne' tempi della pace non pensava
mai se non a' modi della guerra; e quando era in campagna con gli amici
spesso si fermava e ragionava con quelli: {[}12{]} Se li inimici fussino
in su quel colle e noi ci trovassimo qui col nostro exercito, chi arebbe
dinoi vantaggio? Come si potrebbe ire, servando l'ordini, a trovarli? se
noi volessimo ritirarci, come aremo a fare? se loro si ritirassino, come
aremo a seguirli? {[}13{]} E proponeva loro, andando, tutti e casi che
in uno exercito possono occorrere: intendeva la opinione loro, diceva la
sua, corroboravala con le ragioni: tale che, per queste continue
cogitazioni, non poteva mai, guidando gli exerciti, nascere accidente
alcuno che lui non vi avessi el remedio.

{[}14{]} Ma quanto allo exercizio della mente, debbe el principe leggere
le storie et in quelle considerare le actioni delli uomini excellenti,
vedere come si sono governati nelle guerre, examinare le cagioni della
vittoria e perdite loro, per potere queste fuggire e quelle imitare; e
sopratutto fare come ha fatto per lo adrieto qualche uomo excellente che
ha preso ad imitare se alcuno, innanzi a lui, è stato laudato e
gloriato, e di quello ha tenuto sempre e gesti et actioni appresso di
sé: come si dice che Alessandro Magno imitava Achille; Cesare,
Alessandro; Scipione, Ciro. {[}15{]} E qualunque legge la vita di Ciro
scritta da Xenofonte, riconosce dipoi nella vita di Scipione quanto
quella imitactione gli fu a gloria, e quanto, nella castità affabilità
umanità liberalità, Scipione si conformassi con quelle cose che di Ciro
da Xenofonte sono sute scripte.

{[}16{]} Questi simili modi debba observare uno principe savio; e mai
nelli tempi pacifici stare ozioso, ma con industria farne capitale per
potersene valere nelle adversità, acciò che la fortuna, quando si muta,
lo truovi parato a resisterle

\quebra\section{DE HIS REBUS QUIBUS HOMINES ET PRAESERTIM PRINCIPES LAUDANTUR AUT VITUPERANTUR
{[}Di quelle cose che li òmini e spezialmente e' principi sono laudati o biasimati{]}}

{[}1{]} Resta ora a vedere quali debbino essere e modi e governi di uno
principe o con subditi o con li amici. {[}2{]} E perché io so che molti
di questo hanno scripto, dubito, scrivendone ancora io, non essere
tenuto prosumptuoso, partendomi maxime, nel disputare questa materia,
dalli ordini delli altri. {[}3{]} Ma sendo l'intenzione mia stata
scrivere cosa che sia utile a chi la intende, mi è parso più conveniente
andare drieto alla verità effettuale della cosa che alla immaginazione
di epsa. {[}4{]} E molti si sono immaginati repupbliche e principati che
non si sono mai visti né conosciuti essere in vero. {[}5{]} Perché gli è
tanto discosto da come si vive a come si doverrebbe vivere, che colui
che lascia quello che si fa, per quello che si doverrebbe fare, impara
più presto la ruina che la perservazione sua: perché uno uomo che voglia
fare in tutte le parte professione di buono, conviene che ruini infra
tanti che non sono buoni. {[}6{]} Onde è necessario, volendosi a uno
principe mantenere, imparare a potere essere non buono et usarlo e non
usare secondo la necessità.

{[}7{]} Lasciando adunque adrieto le cose circa uno principe immaginate,
e discorrendo quelle che sono vere, dico che tutti li uomini, quando se
ne parla, e maxime e principi, per essere posti più alti, sono notati di
alcune di queste qualità che arrecano loro o biasimo o laude. {[}8{]} E
questo è che alcuno è tenuto liberale, alcuno misero; usando uno termine
toscano, perché avaro in nostra lingua è ancora colui che per rapina
desidera di avere: misero chiamiamo noi quello che si astiene troppo di
usare il suo; -- alcuno è tenuto donatore, alcuno rapace; alcuno
crudele, alcuno pietoso; {[}9{]} l'uno fedifrago, l'altro fedele; l'uno
effeminato e pusillanime, l'altro feroce et animoso; l'uno umano,
l'altro superbo; l'uno lascivo, l'altro casto; l'uno intero, l'altro
astuto; l'uno duro, l'altro facile; l'uno grave, l'altro leggieri; l'uno
religioso, l'altro incredulo, e simili. {[}10{]} Et io so che ciascuno
confesserà che sarebbe laudabilissima cosa uno principe trovarsi di
tutte le soprascritte qualità, quelle che sono tenute buone. {[}11{]} Ma
perché le non si possono avere tutte né interamente observare, per le
condizioni umane che non lo consentono, è necessario essere tanto
prudente che sappi fuggire l'infamia di quelle vizzi che gli torrebbono
lo stato; e da quegli che non gliene tolgano guardarsi, se gli è
possibile; ma, non possendo, vi si può con meno respetto lasciare
andare. {[}12{]} Et etiam non si curi di incorrere nella infamia di
quelli vizii, sanza e quali possa difficilmente salvare lo stato;
perché, se si considerrà bene tutto, si troverrà qualche cosa che parrà
virtù, e seguendola sarebbe la ruina sua: e qualcuna altra che parrà
vizio, e seguendola ne riesce la sicurtà et il bene essere suo.

\quebra\section{DE LIBERALITATE ET PARSIMONIA
{[}Della liberalità e della parsimonia{]}}

{[}1{]} Cominciandomi adunque alle prime soprascritte qualità dico come
sarebbe bene essere tenuto liberale. {[}2{]} Nondimanco, la liberalità,
usata in modo che tu sia tenuto, ti offende: perché, se ella si usa
virtuosamente e come ella si debbe usare, la non fia conosciuta e non ti
cascherà l'infamia del suo contrario; e però, a volersi mantenere infra
li uomini el nome di liberale, è necessario non lasciare indrieto alcuna
qualità di sumptuosità; talmente che sempre uno principe così fatto
consumerà in simili opere tutte le sue facultà; {[}3{]} e sarà
necessitato alla fine, se si vorrà mantenere el nome del liberale,
gravare li populi extraordinariamente et essere fiscale, e fare tutte
quelle cose che si possono fare per avere danari; il che comincerà a
farlo odioso a' subditi, o poco stimare da ciascuno divenendo povero.
{[}4{]} In modo che, con questa sua liberalità avendo offeso gli assai e
premiato e pochi, sente ogni primo disagio e periclita in qualunque
primo periculo: il che conoscendo lui e volendosene ritrarre, incorre
subito nella infamia del misero. {[}5{]} Uno principe adunque, non
potendo usare questa virtù del liberale, sanza suo danno, in modo che la
sia conosciuta, debba, se gli è prudente, non si curare del nome del
misero: perché col tempo sarà tenuto sempre più liberale veggendo che,
con la sua parsimonia, le sua entrate gli bastano, può difendersi da chi
gli fa guerra, può fare imprese sanza gravare i populi. {[}6{]} Talmente
che viene ad usare liberalità a tutti quelli a chi gli non toglie, che
sono infiniti, e miseria a tutti coloro a chi non dà, che sono pochi.

{[}7{]} Nelli nostri tempi noi non abbiamo veduto fare gran cose se non
a quelli che sono tenuti miseri; li altri, essere spenti. {[}8{]} Papa
Iulio II, come si fu servito del nome del liberale per aggiugnere al
papato, non pensò poi a mantenerselo, per poter fare guerra. {[}9{]} El
re di Francia presente ha fatto tante guerre sanza porre uno dazio
extraordinario a' sua, solum perché alle superflue spese ha
subministrato la lunga parsimonia sua. {[}10{]} El re di Spagna
presente, se fussi tenuto liberale, non arebbe ne fatto né vinto tante
imprese. {[}11{]} Pertanto uno principe debbe existimare poco, -- per
non avere a rubare e subditi, per potere difendersi, per non diventare
povero e contennendo, per non essere forzato di diventare rapace, -- di
incorrere nel nome del misero: perché questo è uno di quelli vizii che
lo fanno regnare. {[}12{]} E se alcuno dicessi: Cesare con la liberalità
pervenne allo imperio, e molti altri, per essere stati et essere tenuti
liberali, sono venuti a gradi grandissimi; rispondo: o tu se' principe
fatto o tu se' in via di acquistarlo. {[}13{]} Nel primo caso questa
liberalità è dannosa. Nel secondo, è bene necessario essere tenuto
liberale; e Cesare era uno di quelli che voleva pervenire al principato
di Roma: ma se, poi che vi fu per venuto, fussi sopravvissuto e non si
fussi temperato da quelle spese, arebbe destrutto quello imperio.

{[}14{]} E se alcuno replicassi: molti sono stati principi e con li
exerciti hanno fatto gran cose, che sono stati tenuti liberalissimi; ti
respondo: o el principe spende del suo e de' sua subditi, o di quello
d'altri. {[}15{]} Nel primo caso, debbe essere parco. Nell'altro, non
debbe lasciare indrieto parte alcuna di liberalità. {[}16{]} E quel
principe che va con li exerciti, che si pasce di prede, di sacchi e di
taglie, maneggia quello di altri, gli è necessaria questa liberalità
altrimenti non sarebbe seguíto da' soldati. {[}17{]} E di quello che non
è tuo o de' subditi tuoi si può essere più largo donatore, come fu Ciro,
Cesare et Alessandro: perché lo spendere quel d'altri non ti toglie
reputazione, ma te ne aggiugne; solamente lo spendere el tuo è quello
che ti nuoce. {[}18{]} E non ci è cosa che consumi sé stessa quanto la
liberalità, la quale mentre che tu usi, perdi la facultà di usarla; e
diventi, o povero e contennendo, o, per fuggire la povertà, rapace et
odioso. {[}19{]} Et intra tutte le cose di che uno principe si debbe
guardare è lo essere contennendo et odioso: e la liberalità all'una e
l'altra cosa ti conduce. {[}20{]} Pertanto è più sapienza tenersi el
nome del misero, che partorisce una infamia sanza odio, che, per volere
el nome del liberale, essere necessitato incorrere nel nome del rapace,
che partorisce una infamia con odio.

\quebra\section{DE CRUDELITATE ET PIETATE; ET AN SIT MELIUS AMARI QUAM TIMERI, VEL E CONTRA
{[}Della crudeltà e pietà, e s'elli è meglio esser amato che temuto, o
più tosto temuto che amato{]}}

{[}1{]} Scendendo appresso alle altre qualità preallegate, dico che
ciascuno principe debbe desiderare di essere tenuto pietoso e non
crudele: nondimanco debbe avvertire di non usare male questa pietà.
{[}2{]} Era tenuto Cesare Borgia crudele; nondimanco quella sua crudeltà
aveva racconcia la Romagna, unitola, ridottola in pace et in fede.
{[}3{]} Il che se si considera bene, si vedrà quello essere stato molto
più pietoso che il populo fiorentino, il quale, per fuggire il nome del
crudele, lasciò distruggere Pistoia. {[}4{]} Debbe pertanto uno principe
non si curare della infamia del crudele per tenere e subditi sua uniti
et in fede: perché con pochissimi exempli sarà più pietoso che quelli e
quali per troppa pietà lasciono seguire e disordini, di che ne nasca
uccisioni o rapine; perché queste sogliono offendere una universalità
intera, e quelle execuzioni che vengano dal principe offendono uno
particulare. {[}5{]} Et intra tutti e principi al principe nuovo è
impossibile fuggire il nome di crudele, per essere gli stati nuovi pieni
di pericoli {[}6{]} E Vergilio nella bocca di Didone dice: \emph{Res
dura et regni novitas me talia cogunt moliri et late fines custode
tueri.} {[}7{]} Nondimanco debbe essere grave al credere et al muoversi,
né si fare paura da sé stesso, e procedere in modo, temperato con
prudenza et umanità, che la troppa confidenzia non lo facci incauto e la
troppa diffidenzia non lo renda intollerabile.

{[}8{]} Nasce da questo una disputa, s'egli è meglio essere amato che
temuto o e converso. {[}9{]} Rispondesi che si vorrebbe essere l'uno e
l'altro; ma perché egli è difficile accozzarli insieme, è molto più
sicuro essere temuto che amato, quando si abbi a mancare dell'uno delli
duoi. {[}10{]} Perché degli uomini si può dire questo generalmente, che
sieno ingrati, volubili, simulatori e dissimulatori, fuggitori de'
pericoli, cupidi del guadagno; e mentre fai loro bene sono tutti tua,
offeronti el sangue, la roba, la vita, e figliuoli, come di sopra dixi,
quando el bisogno è discosto: ma quando ti si appressa, si rivoltono, e
quello principe che si è tutto fondato in su le parole loro, trovandosi
nudo di altre preparazioni, ruina. {[}11{]} Perché le amicizie che si
acquistano col prezzo, e non con grandezza e nobiltà di animo, si
meritano, ma elle non si hanno, et alli tempi non si possano spendere; e
li uomini hanno meno rispetto a offendere uno che si facci amare, che
uno che si facci temere: perché lo amore è tenuto da uno vinculo di
obligo, il quale, per essere gl'uomini tristi, da ogni occasione di
propria utilità è rotto, ma il timore è tenuto da una paura di pena che
non abbandona mai.

{[}12{]} Debbe nondimanco el principe farsi temere in modo che, se non
acquista lo amore, che fugga l'odio; perché può molto bene stare insieme
essere temuto e non odiato. {[}13{]} Il che farà sempre, quando si
abstenga dalla roba de' sua ciptadini e delli sua subditi e dalle donne
loro. E quando pure gli bisognassi procedere contro al sangue di alcuno,
farlo quando vi sia iustificazione conveniente e causa manifesta.
{[}14{]} Ma sopratutto abstenersi dalla roba di altri, perché li uomini
sdimenticano più presto la morte del padre che la perdita del
patrimonio; dipoi, le cagioni del tòrre la roba non mancono mai, e
sempre, colui che comincia a vivere con rapina, truova cagione di
occupare quello di altri: e per avverso contro al sangue sono più rare e
mancono più presto.

{[}15{]} Ma quando el principe è con li exerciti et ha in governo
moltitudine di soldati, allora al tutto è necessario non si curare del
nome del crudele: perché sanza questo nome non si tenne mai exercito
unito né disposto ad alcuna fazione {[}16{]} Intra le mirabili actioni
di Annibale si connumera questa, che, avendo uno exercito grossissimo,
mixto di infinite generazioni di uomini, condotto a militare in terre
aliena, non vi surgessi mai alcuna dissensione, né infra loro, né contro
al principe, così nella captiva come nella sua buona fortuna. {[}17{]}
Il che non possé nascere da altro che da quella sua inumana crudeltà: la
quale, insieme con infinite sua virtù, lo fece sempre nel conspetto de'
sua soldati venerando e terribile. {[}18{]} E sanza quella, a fare
quello effetto, l'altre sua virtù non li bastavano: e li scriptori, in
questo poco considerati dall'una parte admirano questa sua actione,
dall'altra dannano la principale cagione di epsa.

{[}19{]} E che sia vero che le altre sua virtù non sarebbano bastate, si
può considerare in Scipione, rarissimo non solamente ne' tempi sua ma in
tutta la memoria delle cose che si sanno, dal quale li exerciti sua in
Ispagna si ribellorono; è che non nacque da altro che dalla troppa sua
pietà, la quale aveva data alli suoi soldati più licenza che alla
disciplina militare non si conveniva. {[}20{]} La qual cosa gli fu da
Fabio Maximo in Senato rimproverata e chiamato da lui corruptore della
romana milizia. {[}21{]} E Locrensi, essendo suti da uno legato di
Scipione destrutti, non furono vendicati, né fu da lui la insolenza di
quello legato corretta tutto nascendo da quella sua natura facile;
talmente che, volendolo alcuno in excusare Senato, dixe come gli erano
di molti uomini che sapevano meglio non errare che correggere gli
errori. {[}22{]} La qual natura arebbe col tempo violato la fama e la
gloria di Scipione, se gli avessi con epsa perseverato nello imperio:
ma, vivendo sotto il governo del Senato, questa sua qualità dannosa non
solum si nascose, ma gli fu a gloria.

{[}22{]} Concludo adunque, tornando allo essere temuto et amato, che,
amando li uomini a posta loro e temendo a posta del principe, debbe uno
principe savio fondarsi in su quello che è suo, non in su quello che è
d'altri; debbe solamente ingegnarsi di fuggire lo odio, come è detto.

\quebra\section{QUOMODO FIDES A PRINCIPIBUS SIT SERVANDA
{[}In che modo e' principi abbino a mantenere la fede{]}}

{[}1{]} Quanto sia laudabile in uno principe il mantenere la fede e
vivere con integrità e non con astuzia, ciascuno lo intende; nondimanco
si vede per esperienza nelli nostri tempi quelli principi avere fatto
gran cose, che della fede hanno tenuto poco conto e che hanno saputo con
l'astuzia aggirare e' cervelli delli uomini; et alla fine hanno superato
quelli che si sono fondati in sulla realtà.

{[}2{]} Dovete adunque sapere come e' sono dua generazione di
combattere: l'uno, con le leggi; l'altro, con la forza. {[}3{]} Quel
primo è proprio dello uomo; quel secondo delle bestie. {[}4{]} Ma perché
el primo molte volte non basta, conviene ricorrere al secondo: pertanto
ad uno principe è necessario sapere bene usare la bestia e lo uomo.
{[}5{]} Questa parte è suta insegnata alli principi copertamente dalli
antichi scriptori, li quali scrivono come Achille e molti altri di
quelli principi antichi, furono dati a nutrire a Chirone centauro, che
sotto la sua disciplina li custodissi. {[}6{]} Il che non vuole dire
altro, avere per preceptore uno mezzo bestia e mezzo uomo, se non che
bisogna ad uno principe sapere usare l'una e l'altra natura; e l'una
sanza l'altra non è durabile.

{[}7{]} Sendo dunque necessitato uno principe sapere bene usare la
bestia, debbe di quelle pigliare la volpe e il lione; perché el lione
non si defende da' lacci, la volpe non si difende da' lupi; bisogna
adunque essere volpe a conoscere e lacci, e lione a sbigottire e lupi:
coloro che stanno semplicemente in sul lione, non se ne intendono.
{[}8{]} Non può pertanto uno signore prudente, ne debbe, osservare la
fede quando tale observanzia gli torni contro e che sono spente le
cagioni che la feciono promettere. {[}9{]} E se li uomini fussino tutti
buoni, questo precetto non sarebbe buono; ma perché sono tristi e non la
observarebbono a te, tu etiam non l'hai ad observare a loro; né mai ad
uno principe mancorono cagioni legittime di colorire la inobservanzia.
{[}10{]} Di questo se ne potrebbe dare infiniti exempli moderni e
monstrare quante pace, quante promesse sono state fatte irrite e vane
per la infedelità de' principi: e quello che ha saputo meglio usare la
volpe, è meglio capitato. {[}11{]} Ma è necessario questa natura saperla
bene colorire et essere gran simulatore e dissimulatore: e sono tanto
semplici gli uomini, e tanto ubbediscano alle necessità presenti, che
colui che inganna troverrà sempre chi si lascerà ingannare.

{[}12{]} Io non voglio delli esempli freschi tacerne uno. Alessandro VI
non fece mai altro, non pensò mai ad altro che ad ingannare uomini; e
sempre trovò subietto da poterlo fare: e non fu mai uomo che avessi
maggiore efficacia in asseverare, e con maggiori iuramenti affermassi
una cosa, che l'observassi meno; nondimeno sempre gli succederono
gl'inganni ad votum, perché conosceva bene questa parte del mondo.

{[}13{]} A uno principe adunque non è necessario avere in fatto tutte le
soprascritte qualità, ma è ben necessario parere di averle; anzi ardirò
di dire questo: che avendole et observandole sempre, sono dannose, e,
parendo di averle, sono utili; come parere piatoso, fedele, umano,
intero, relligioso, et essere: ma stare in modo edificato con lo animo
che, bisognando non essere, tu possa e sappia diventare il contrario.
{[}14{]} Et hassi ad intendere questo, che uno principe e maxime uno
principe nuovo non può observare tutte quelle cose per le quali gli
uomini sono chiamati buoni, sendo spesso necessitato, per mantenere lo
stato, operare contro alla fede, contro alla carità, contro alla
umanità, contro alla religione. {[}15{]} E però bisogna che egli abbia
uno animo disposto a volgersi secondo che e venti della fortuna e la
variazioni delle cose gli comandano; e, come di sopra dixi, non partirsi
dal bene, potendo, ma sapere entrare nel male, necessitato.

{[}16{]} Debba adunque uno principe avere gran cura che non gli esca mai
di bocca cosa che non sia piena delle soprascritte cinque qualità; e
paia, ad udirlo et vederlo, tutto pietà, tutto fede, tutto integrità,
tutto relligione e non è cosa più necessaria, a parere di avere, che
questa ultima qualità. {[}17{]} E li uomini in universali iudicano più
alli occhi che alle mani; perché tocca a vedere ad ognuno, a sentire a
pochi: ognuno vede quello che tu pari, pochi sentono quello che tu se';
e quelli pochi non ardiscano opporsi alla opinione di molti che abbino
la maestà dello stato che li difenda: e nelle actione di tutti li
uomini, e maxime de' principi, dove non è iudizio a chi reclamare, si
guarda al fine.

{[}18{]} Facci dunque uno principe di vincere e mantenere lo stato: e
mezzi sempre fieno iudicati onorevoli e da ciascuno saranno laudati;
perché el vulgo ne va preso con quello che pare e con lo evento della
cosa: e nel mondo non è se non vulgo, e' pochi ci hanno luogo quando gli
assai hanno dove appoggiarsi. {[}19{]} Alcuno principe de' presenti
tempi, il quale non e bene nominare, non predica mai altro che pace e
fede, e dell'una e dell'altra è inimicissimo; e l'una e l'altra, quando
egli l'avessi osservata, egli arebbe più volte tolto e la reputazione e
lo stato.

\quebra\section{DE CONTEMPTU ET ODIO FUGIENDO
{[}In che modo si abbia a fuggire lo essere sprezzato e odiato{]}}

{[}1{]} Ma perché, circa le qualità di che di sopra si fa menzione, io
ho parlato delle più importanti, l'altre voglio discorrere brevemente
sotto queste generalità: che el principe pensi, come in parte di sopra è
detto, di fuggire quelle cose che lo faccino odioso o contennendo; e
qualunque volta egli fuggirà questo, arà adempiuto le parte sua e non
troverrà nelle altre infamie periculo alcuno. {[}2{]} Odioso sopratutto
lo fa, come io dissi, essere rapace et usurpatore della roba e delle
donne de' subditi: da che si debba abstenere. {[}3{]} E qualunque volta
alle universalità delli uomini non si toglie né onore né roba, vivono
contenti: e solo si ha a combattere con la ambizione de pochi, la quale
in molti modi e con facilità si raffrena. {[}4{]} Contennendo lo fa
essere tenuto vario, leggieri, effeminato, pusillanime, irresoluto: da
che uno principe si debbe guardare come di uno scoglio, et ingegnarsi
che nelle actioni sua si riconosca grandezza, animosità, gravità,
fortezza; e circa a' maneggi privati trà subditi volere che la sua
sentenza sia inrevocabile; e si mantenga in tale opinione che alcuno non
pensi né ad ingannarlo né ad aggirarlo.

{[}5{]} Quel principe che dà di sé questa opinione è reputato assai, e
contro a chi è reputato con difficultà si congiura, con difficultà è
assaltato, purché s'intenda che sia excellente e che sia reverito da'
sua. {[}6{]} Perché uno principe debba avere dua paure: una dentro, per
conto de' subditi; l'altra di fuori, per conto de' potentati externi.
{[}7{]} Da questa si difende con le buone arme e con li buoni amici: e
sempre, se arà buone arme, arà buoni amici. {[}8{]} E sempre staranno
ferme le cose di dentro, quando stieno ferme quelle di fuora, se già le
non fussino perturbate da una congiura: e quando pure quelle di fuora
movessino, s'egli è ordinato e vissuto come ho detto, quando egli non si
abbandoni, sosterrà sempre ogni impeto, come io dixi che fece Nabide
spartano.

{[}9{]} Ma circa subditi, quando le cose di fuora non muovino, si ha a
temere che non coniurino secretamente; di che el principe si assicura
assai fuggendo lo essere odiato o disprezzato, e tenendosi el populo
satisfatto di lui: il che è necessario conseguire, come di sopra a lungo
si disse. {[}10{]} Et uno de' più potenti remedii che abbia uno principe
contro alle congiure, è non essere odiato dallo universale: perché
sempre chi coniura crede con la morte del principe satisfare al populo,
ma quando creda offenderlo non piglia animo a prendere simile partito.
{[}11{]} Perché le difficultà che sono dalla parte de' congiuranti sono
infinite, e per esperienza si vede molte essere state le congiure e
poche avere avuto buono fine. {[}12{]} Perché chi congiura non può
essere solo, né può prendere compagnia se non di quelli che creda essere
malcontenti: e subito che a uno malcontento tu hai scoperto lo animo
tuo, gli dai materia a contentarsi, perché manifestamente lui ne può
sperare ogni commodità; talmente che, veggendo il guadagno sicuro da
questa parte, e dall'altra veggendolo dubio e pieno di periculo,
conviene bene o che sia raro amico o che sia al tutto ostinato inimico
del principe, ad observarti la fede

{[}13{]} E per ridurre la cosa in brevi termini, dico che dalla parte
del coniurante non è se non paura, gelosia e sospecto di pena che lo
sbigottisce: ma dalla parte del principe è la maestà del principato, le
legge, le difese delli amici e dello stato che lo difendono. {[}14{]}
Talmente che, adgiunto a tutte queste cose la benivolenzia populare, è
impossibile che alcuno sia sì temerario che congiuri: perché, dove per
lo ordinario, uno coniurante ha a temere innanzi alla execuzione del
male, in questo caso debbe temere ancora poi, avendo per nimico el
populo, seguito lo excesso, né potendo per questo sperare refugio
alcuno.

{[}15{]} Di questa materia se ne potria dare infiniti esempli, ma voglio
solo essere contento di uno seguito a' tempo de' padri nostri. {[}16{]}
Messere Annibale Bentivogli, avolo del presente messer Annibale, che era
principe di Bologna, sendo da' Canneschi, che gli coniurorono contro,
ammazzato, né rimanendo di lui altri che messere Giovanni, quale era in
fasce, subito dopo tale omicidio si levò el populo et ammazzò tutti e
Canneschi. {[}17{]} Il che nacque dalla benivolenzia populare che la
Casa de' Bentivogli aveva in quelli tempi: la quale fu tanta che, non
restando di quella alcuno, in Bologna, che potessi, morto Annibale,
reggere lo stato, et avendo indizio come in Firenze era uno nato de'
Bentivogli, che si teneva fino allora figliuolo di uno fabbro, vennono e
Bolognesi per quello in Firenze e gli dettono il governo di quella
città; la quale fu governata da lui fino a tanto che messer Giovanni
pervenissi in età conveniente al governo.

{[}18{]} Concludo pertanto che uno principe debbe tenere delle congiure
poco conto, quando il popolo gli sia benivolo: ma quando gli sia nimico
et abbilo in odio, debbe temere d'ogni cosa e d'ognuno. {[}19{]} E gli
stati bene ordinati e li principi savi hanno con ogni diligenzia pensato
di non disperare e grandi e satisfare al populo e tenerlo contento:
perché questa è una delle più importanti materie che abbi uno principe.

{[}20{]} Intra e regni bene ordinati e governati a' tempi nostri è
quello di Francia, et in epso si truovano infinite constituzioni buone
donde depende la libertà e sicurtà del re: delle quali la prima è il
Parlamento e la sua autorità. {[}21{]} Perché quello che ordinò quello
regno, conoscendo l'ambizione de' potenti e la insolenzia loro, e
iudicando essere loro necessario uno freno in bocca che gli correggessi;
e da altra parte conoscendo l'odio dello universale contro a' grandi
fondato in su la paura, e volendo assicurargli, -- non volle che questa
fussi particulare cura del re, per tòrgli quel lo carico che potessi
avere con li grandi favorendo e populari, e co' populari favorendo e
grandi.

{[}22{]} E però constituí uno iudice terzo, che fussi quello che sanza
carico del re battessi e grandi e favorissi e minori: né poté essere
questo ordine migliore né più prudente, né che sia maggiore cagione
della sicurtà del re e del regno. {[}23{]} Di che si può trarre un altro
notabile: che e principi le cose di carico debbono fare subministrare ad
altri, quelle di grazia a loro medesimi. {[}24{]} Ed nuovo concludo che
uno principe debbe stimare e grandi, ma non si fare odiare dal populo.

{[}25{]} Parrebbe forse a molti, considerato la vita e morte di alcuno
imperatore romano, che fussino exempli contrarii a questa mia opinione,
trovando alcuno essere vissuto sempre egregiamente e mostro gran virtù
d'animo: nondimeno avere perso lo imperio, o vero essere stato morto da'
sua che gli hanno congiurato contro. {[}26{]} Volendo pertanto
rispondere a queste obiectioni, discorrerò le qualità di alcuni
imperatori, mostrando le cagioni della loro ruina non disforme da quello
che da me si è addutto; e parte metterò in considerazione quelle cose
che sono notabili a chi legge le actioni di quelli tempi. {[}27{]} E
voglio mi basti pigliare tutti quelli imperatori che succederono allo
imperio da Marco filosofo a Maximino, li quali furono: Marco, Commodo
suo figliuolo, Pertinace, Iuliano, Severo, Antonino Caracalla suo
figliuolo, Macrino, Eliogabal, Alessandro e Maximino. {[}28{]} Et è
prima da notare che, dove nelli altri principati si ha solo a contendere
con la ambizione de' grandi et insolenzia de' populi, gl'imperatori
romani avevano una terza difficultà, di avere a sopportare la crudeltà
et avarizia de' soldati. {[}29{]} La quale cosa era sì difficile che la
fu cagione della ruina di molti, sendo difficile satisfare a' soldati et
a' populi; perché e populi amavano la quiete, e per questo e principi
modesti erano loro grati e li soldati amavono el principe d'animo
militare e che fussi crudele insolente, e rapace: le quali cose volevano
che lui exercitassi ne' populi, per potere avere duplicato stipendio e
sfogare la loro avarizia e crudeltà. {[}30{]} Le quali cose feciono che
quelli imperatori che per natura o per arte non avevano una gran
reputazione, tale che con quella e' tenessino l'uno e l'altro in freno,
sempre ruinavono. {[}31{]} E li più di loro, maxime di quegli che come
uomini nuovi venivono al principato, conosciuta la difficultà di questi
due diversi umori, si volgevano a satisfare a' soldati, stimando poco lo
iniuriare el populo. {[}32{]} Il quale partito era necessario: perché,
non potendo e principi mancare di non essere odiati da qualcuno, si
debbano sforzare prima di non essere odiati dalla università, e quando
non possono conseguire questo, debbono fuggire con ogni industria l'odio
di quelle università che sono più potenti. {[}33{]} E però quelli
imperatori che per novità avevano bisogno di favori extraordinarii, si
aderivano a' soldati più tosto che a' populi: il che tornava nondimeno
loro utile, o no, secondo che quel principe si sapeva mantenere reputato
con epso loro.

{[}34{]} Da queste cagioni sopradette nacque che Marco, Pertinace et
Alessandro, sendo tutti di modesta vita, amatori della iustizia, inimici
della crudeltà, umani, benigni, ebbono tutti, da Marco in fuora, tristo
fine. {[}35{]} Marco solo visse e morí onoratissimo, perché lui succedé
allo imperio iure hereditario e non aveva a riconoscere quello né da'
soldati né da' populi; dipoi, essendo accompagnato da molte virtù che lo
facevano venerando, tenne sempre, mentre che visse, l'uno e l'altro
ordine intra e termini suoi, e non fu mai né odiato né disprezzato.
{[}36{]} Ma Pertinace, creato imperatore contro alla voglia de' soldati,
li quali essendo usi a vivere licenziosamente sotto Commodo non poterono
sopportare quella vita onesta alla quale Pertinace gli voleva ridurre;
onde avendosi creato odio et a questo odio aggiunto el disprezzo sendo
vecchio ruinò ne' primi principii della sua administrazione. {[}37{]} E
qui si debbe notare che l'odio si acquista così mediante le buone opere,
come le triste: e però, come io dixi di sopra, uno principe, volendo
mantenere lo stato, è spesso forzato a non essere buono. {[}38{]}
Perché, quando quella università, o populo o soldati o grandi che sieno,
della qual tu iudichi avere per mantenerti, piu bisogno è corrotta, ti
conviene seguire l'umore suo per satisfarlo: et allora le buone opere ti
sono nimiche.

{[}39{]} Ma vegnamo ad Alessandro: il quale fu di tanta bontà che, intra
le altre laude che gli sono attribuite, è questa, che in 14 anni che
tenne l'imperio non fu mai morto da lui alcuno iniudicato: nondimanco,
sendo tenuto effeminato et uomo che si lasciassi governare alla madre, e
per questo venuto in disprezzo, conspirò in lui l'exercito et
ammazzollo.

{[}40{]} Discorrendo ora per opposito le qualità di Commodo, di Severo,
di Antonino Caracalla e Maximino, gli troverrete crudelissimi e
rapacissimi: li quali, per satisfare a' soldati, non perdonorono ad
alcuna qualità di iniuria che ne' populi si potessi commettere. {[}41{]}
E tutti excetto Severo ebbono triste fine; perché in Severo fu tanta
virtù che, mantenendosi e soldati amici, ancora che populi fussino da
lui gravati, poté sempre regnare felicemente: perché quelle sua virtù lo
facevano nel conspetto de' soldati e delli populi sì mirabile che questi
rimanevano quodammodo stupidi e attoniti, e quelli altri reverenti e
satisfatti. {[}42{]} E perché le actioni di costui furono grande e
notabile in uno principe nuovo, io voglio brevemente monstrare quanto é
seppe bene usare la persona del lione e della volpe, le quali nature io
dico di sopra essere necessario imitare a uno principe.

{[}43{]} Conosciuto Severo la ignavia di Iuliano imperatore, persuase al
suo exercito, del quale era in Stiavonia capitano, che egli era bene
andare a Roma a vendicare la morte di Pertinace, il quale da' soldati
pretoriani era suto morto. {[}44{]} E sotto questo colore, sanza
monstrare di aspirare allo imperio, mosse lo exercito contro a Roma e fu
prima in Italia che si sapessi la sua partita. {[}45{]} Arrivato a Roma,
fu dal Senato per timore eletto imperatore e morto Iuliano. {[}46{]}
Restava dopo questo principio a Severo dua difficultà, volendosi
insignorire di tutto lo stato: l'una in Asia, dove Nigro, capo delli
exerciti asiatici, si era fatto chiamare imperatore; e l'altra in
Ponente, dove era Albino quale ancora lui aspirava allo imperio.
{[}47{]} E perché iudicava periculoso scoprirsi inimico a tutti a dua,
deliberò di assaltare Nigro et ingannare Albino: al quale scripse come,
sendo dal Senato electo imperatore, voleva partecipare quella dignità
con lui; e mandogli il titulo di Cesare e per diliberazione del Senato
se lo aggiunse conlega: le quali cose furno da Albino acceptate per
vere. {[}48{]} Ma poi ché Severo ebbe vinto e morto Nigro e pacate le
cose orientali, ritornatosi a Roma, si querelò in Senato come Albino,
poco conoscente de' benefizii ricevuti da lui, aveva dolosamente cerco
di ammazzarlo: e per questo era necessitato di andare a punire la sua
ingratitudine; dipoi andò a trovare infrancia, e gli tolse lo stato e la
vita. {[}49{]} E chi examinerà tritamente le actione di costui, lo
troverrà uno ferocissimo lione et una astutissima golpe, e vedrà quello
temuto e reverito da ciascuno e dalli exerciti non odiato; e non si
maraviglierà se lui, uomo nuovo, arà potuto tenere tanto imperio, perché
la sua grandissima reputazione lo difese sempre da quello odio che li
populi per le sue rapine avevano potuto concipere.

{[}50{]} Ma Antonino suo figliuolo fu ancora lui uomo che aveva parte
excellentissime e che lo facevano maraviglioso nel conspetto de' populi
e grato a' soldati, perché lui era uomo militare, sopportantissimo
d'ogni fatica, disprezzatore d'ogni cibo dilicato e d'ogni altra
mollizie: la qual cosa lo faceva amare da tutti li exerciti. {[}51{]}
Nondimanco la sua ferocia e crudeltà fu tanta e sì inaudita, per avere
dopo infinite occisioni particulari morto gran parte del populo di Roma
e tutto quello di Alessandria, che diventò odiosissimo a tutto il mondo
e cominciò ad essere temuto etiam da quelli che lui aveva dintorno: in
modo che fu ammazzato da uno centurione in mezzo del suo exercito.
{[}52{]} Dove è da notare che queste simili morte, le quali seguano per
diliberazione di uno animo obstinato, sono da' principi inevitabili,
perché ciascuno che non si curi di morire lo può offendere: ma debba
bene el principe temerne meno, perché le sono rarissime. {[}53{]} Debba
solo guardarsi di non fare grave ingiuria ad alcuno di coloro di chi si
serve e che egli ha dintorno al servizio del suo principato; come aveva
fatto Antonino, il quale aveva morto contumeliosamente uno fratello di
quello centurione e lui ogni giorno minacciava, tamen lo teneva a
guardia del corpo suo: il che era partito temerario e da ruinarvi, come
gl'intervenne.

{[}54{]} Ma vegnamo a Commodo, al quale era facilità grande tenere
l'imperio per averlo iure hereditario, sendo figliuolo di Marco: e solo
gli bastava seguire le vestigie del padre, et a' soldati et a' populi
arebbe satisfatto. {[}55{]} Ma essendo di animo crudele e bestiale, per
potere usare la sua rapacità ne' populi, si volse ad intrattenere li
exerciti e fargli licenziosi: dall'altra parte, non tenendo la sua
dignità, discendendo spesso ne' teatri a combattere con gladiatori e
facendo altre cose vilissime e poco degne della maestà imperiale,
diventò contennendo nel conspetto de' soldati. {[}56{]} Et essendo
odiato da l'una parte e disprezzato dall'altra, fu conspirato in lui e
morto.

{[}57{]} Restaci a narrare le qualità di Maximino. Costui fu uomo
bellicosissimo, et essendo gli exerciti infastiditi della mollizie di
Alessandro, del quale ho di sopra discorso, morto lui lo elessono allo
imperio; il quale non molto tempo possedé, perché dua cose lo feciono
odioso e contennendo. {[}58{]} L'una, essere vilissimo per avere già
guardato le pecore in Tracia: la qual cosa era per tutto notissima, li
chi faceva una grande dedignazione nel conspetto di qualunque. {[}59{]}
L'altra, perché, avendo nello ingresso del suo principato differito lo
andare a Roma et intrare nella possessione della sedia imperiale, aveva
dato di sé opinione di crudelissimo, avendo per li suoi prefetti in Roma
e in qualunque luogo dello imperio exercitato molte crudeltà. {[}60{]}
Talmente che, commosso tutto il mondo dallo sdegno per la viltà del suo
sangue e dall'odio per la paura della sua ferocia, si ribellò prima
Affrica, dipoi el Senato, con tutto el populo di Roma e tutta la Italia,
gli conspirò contro; a che si aggiunse el suo proprio exercito, quale,
campeggiando Aquileia e trovando difficultà nella expugnazione,
infastidito dalla crudeltà sua e, per vedergli tanti nimici, temendolo
meno, lo ammazzò.

{[}61{]} Io non voglio ragionare né di Eliogabalo né di Macrino né di
Iuliano, e quali per essere al tutto contennendi si spensono subito, ma
verrò alla conclusione di questo discorso; e dico che li principi de'
nostri tempi hanno meno questa difficultà di satisfare
extraordinariamente a' soldati ne' governi loro: perché, non obstante
che si abbia ad avere a quegli qualche considerazione, tamen si resolve
presto per non avere, alcuno di questi principi exerciti insieme che
sieno inveterati con li governi et administrazione delle provincie, come
erano gli exerciti dello imperio romano. {[}62{]} E però, se allora era
necessario satisfare più alli soldati che a' populi, perché e soldati
potevano più che e populi, ora è più necessario a tutti e principi,
excepto che al Turco et al Soldano, satisfare a' populi che a' soldati,
perché e' populi possono più di quelli. {[}63{]} Di che io ne exceptuo
el Turco, tenendo quello continuamente insieme intorno a sé XII mila
fanti e 15 mila cavagli, da' quali dipende la securtà e la fortezza del
suo regno: et è necessario che, posposto ogni altro respetto, quel
signore se li mantenga amici. {[}63{]} Similmente el regno del Soldano
sendo tutto in nelle mani de' soldati, conviene che ancora lui sanza
respetto de' populi se li mantenga amici. {[}65{]} Et avete a notare che
questo stato del Soldano è disforme a tutti li altri principati, perché
egli è simile al pontificato cristiano, il quale non si può chiamare né
principato ereditario né principato nuovo: perché non e figliuoli del
principe vecchio sono eredi e rimangono signori, ma colui che è eletto a
quello grado da quegli che ne hanno autorità; {[}66{]} et essendo questo
ordine antiquato, non si può chiamare principato nuovo; per che in
quello non sono alcune di quelle difficultà che sono ne' nuovi; perché,
se bene el principe è nuovo, gli ordini di quello stato sono vecchi et
ordinati a riceverlo come se fussi loro signore ereditario.

{[}67{]} Ma torniamo alla materia nostra. Dico che qualunque considerrà
el soprascritto discorso, vedrà o l'odio o il disprezzo essere suti
cagione della ruina di quelli imperatori prenominati; e conoscerà ancora
donde nacque che, parte di loro procedendo in uno modo e parte al
contrario, in qualunque di quegli uno di loro ebbe felice e gli altri
infelice fine. {[}68{]} Perché a Pertinace et Alessandro, per essere
principi nuovi, fu inutile e dannoso volere imitare Marco, che era nel
principato iure hereditario; e similmente a Caracalla, Commodo e
Maximino essere stata cosa perniziosa imitare Severo, per non avere
avuta tanta virtù che bastassi a seguitare le vestigie sua. {[}69{]}
Pertanto uno principe nuovo in uno principato nuovo non può imitare le
actioni di Marco, né ancora è necessario seguitare quelle di Severo: ma
debbe pigliare da Severo quelle parte che per fondare el suo stato sono
necessarie, e da Marco quelle che sono convenienti e gloriose a
conservare uno stato che sia già stabilito e fermo.

\quebra\section{AN ARCES ET MULTA ALIA, QUAE QUOTTIDIE A PRINCIPIBUS FIUNT, UTILIA AN INUTILLIA SINT
{[}Se le fortezze e molte altre cose che ogni giono si fanno da'
principi, per conservazione del loro stato, sono utili o no{]}}

{[}1{]} Alcuni principi per tenere sicuramente lo stato hanno disarmati
e loro subditi; alcuni hanno tenuto divise le terre subiette. {[}2{]}
Alcuni hanno nutrito inimicizie contro a sé medesimi; alcuni altri si
sono volti a guadagnarsi quelli che li erano suspetti nel principio del
suo stato. {[}3{]} Alcuni hanno edificato fortezze; alcuni le hanno
ruinate e destrutte. {[}4{]} E benché di tutte queste cose non si possa
dare determinata sentenza, se non si viene a' particulari di quegli
stati dove si avessi a pigliare alcuna simile deliberazione, nondimanco
io parlerò in quell modo largo che la materia per sé medesima sopporta.

{[}5{]} Non fu mai adunque che uno principe nuovo disarmassi li suoi
subditi: anzi, quando gli ha trovati disarmati, sempre gli ha armati;
perché, armandosi, quelle arme diventano tua, diventano fedeli quelli
che ti sono sospetti, e quelli che erano fedeli si mantengono, e di
subditi si fanno tua partigiani. {[}6{]} E perché tutti li subditi non
si possono armare, quando si beneficano quegli che tu armi, con gli
altri si può fare più a sicurtà: e quella diversità del procedere, che
conoscono in loro, gli fa tua obligati; quelli altri ti scusano,
iudicando essere necessario quegli avere più merito che hanno più
periculo e più obligo. {[}7{]} Ma quando tu gli disarmi, tu cominci ad
offendergli: monstri che tu abbi in loro diffidenzia, o per viltà o per
poca fede, e l'una e l'altra di queste opinioni concepe odio contro di
te; e perché tu non puoi stare disarmato, conviene ti volti alla milizia
mercennaria, la quale è di quella qualità che di sopra è detto: e quando
la fussi buona, non può essere tanta che ti difenda da inimici potenti e
da'subditi sospecti. {[}8{]} Però, come io ho detto, uno principe nuovo,
in uno principato nuovo, sempre vi ha ordinato l'arme: di questi exempli
sono piene le storie. {[}9{]} Ma quando uno principe acquista uno stato
nuovo, che come membro si aggiunga al suo vecchio, allora è necessario
disarmare quello stato, excepto quegli che nello acquistarlo sono suti
tua partigiani: e quegli ancora col tempo e con le occasioni è
necessario renderli molli et effeminati, et ordinarsi in modo che solo
le arme di tutto il tuo stato sieno in quelli tuoi soldati proprii che
nello stato tuo antico vivevano appresso di te.

{[}10{]} Solevano li antiqui nostri, e quelli che erano stimati savi,
dire come era necessario tenere Pistoia con le parti e Pisa con le
fortezze; e per questo nutrivano in qualche terra loro subdita le
differenzie, per possederle più facilmente. {[}11{]}Questo, in quelli
tempi che Italia era in uno certo modo bilanciata, doveva essere bene
fatto: ma non credo già che si possa dare oggi per precepto; perché io
non credo che le divisioni facessino mai bene alcuno: anzi è necessario,
quando el nimico si accosta, che le città divise si perdino subito,
perché sempre la parte più debile si aderirà alle forze externe e
l'altra non potrà reggere.

{[}12{]} Viniziani, mossi come io credo dalle ragioni soprascripte,
nutrivano le sette guelfe e ghibelline nelle città loro subdite; e
benché non li lasciassino mai venire al sangue, tamen nutrivano tra loro
questi dispareri acciò che, occupati quelli cittadini in quelle loro
differenzie, non si unissino contro di loro. {[}13{]} Il che, come si
vide, non tornò loro poi a proposito: perché, sendo rotti a Vailà,
subito una parte di quelle prese ardire e tolsono loro tutto lo stato.
{[}14{]} Arguiscano pertanto simili modi debolezza del principe, perché
in uno principato gagliardo mai si permetteranno simili divisioni:
perché le fanno solo profitto a tempo di pace, potendosi mediante quelle
più facilmente maneggiare e subditi, ma, venendo la guerra, monstra
simile ordine la fallacia sua.

{[}15{]} Senza dubbio e principi diventano grandi quando superano le
difficultà e le opposizioni che sono fatte loro; e però la fortuna,
maxime quando vuole fare grande uno principe nuovo, il quale ha maggiore
necessità di acquistare reputazione che uno ereditario, gli fa nascere
de' nimici e fagli fare delle imprese contro, acciò che quello abbi
cagione di superarle e, su per quella scala che gli hanno pòrta li
inimici suoi, salire più alto. {[}16{]} Però molti iudicano che uno
principe savio debbe, quando egli ne abbia la occasione, nutrirsi con
astuzia qualche inimicizia, acciò che, oppresso quella, ne seguiti
maggior sua grandezza

{[}17{]} Hanno e principi, e praesertim quegli che sono nuovi, trovato
più fede e più utilità in quelli uomini che nel principio del loro stato
sono suti tenuti sospetti, che in quelli che erano nel principio
confidenti. {[}18{]} Pandolfo Petrucci, principe di Siena, reggeva lo
stato suo più con quelli che gli furono sospetti che con li altri.
{[}19{]} Ma di questa cosa non si può parlare largamente, perché la
varia secondo el subietto; solo dirò questo, che quelli uomini che nel
principio di uno principato sono stati inimici, che sono di qualità che
a mantenersi abbino bisogno di appoggiarsi, sempre el principe con
facilità grandissima se gli potrà guadagnare: e loro maggiormente sono
forzati a servirlo con fede, quanto conoscano esseree loro più
necessario cancellare con le opere quella opinione sinistra che si aveva
di loro. {[}20{]} E così el principe ne trae sempre più utilità, che di
coloro che, servendolo con troppa sicurtà, straccurono le cose sua.

{[}21{]} E poiché la materia lo ricerca, non voglio lasciare indrieto
ricordare alli principi che hanno preso uno stato di nuovo, mediante e
favori intrinsichi di quello, che considerino bene qual cagione abbi
mosso quegli che lo hanno favorito, a favorirlo. {[}22{]} E se ella non
è affectione naturale verso di loro, ma fussi solo perché quelli non si
contentavano di quello stato, con fatica e difficultà grande se gli
potrà mantenere amici: perché fia impossibile che lui possa
contentargli. {[}23{]} E discorrendo bene, con quelli exempli che dalle
cose antiche e moderne si traggono, la cagione di questo, vedrà essergli
molto più facile guadagnarsi amici quegli uomini che dello stato innanzi
si contentavono, e però erano sua inimici, che quegli che, per non se ne
contentare, gli diventorono amici e favorironlo ad occuparlo.

{[}24{]} È suta consuetudine de' principi, per potere tenere più
sicuramente lo stato loro, edificare fortezze che sieno la briglia e il
freno di quelli che disegnassino fare loro contro, et avere uno refugio
sicuro da uno subito impeto. {[}25{]} Io laudo questo modo perché egli è
usitato ab antiquo: nondimanco messer Niccolò Vitelli, ne' tempi nostri,
si è visto disfare dua fortezze in Città di Castello per tenere quello
stato; Guido Ubaldo, duca di Urbino, ritornato nella sua dominazione
donde da Cesare Borgia era suto cacciato, ruinò funditus tutte le
fortezze di quella sua provincia e iudicò sanza quelle più difficilmente
riperdere quello stato; Bentivogli, ritornati in Bologna, usorono simili
termini. {[}26{]} Sono dunque le fortezze utili, o no, secondo e tempi:
e se le ti fanno bene in una parte, ti offendano in un'altra. {[}27{]} E
puossi discorrere questa parte così: quel principe che ha più paura de'
populi che de' forestieri, debbe fare le fortezze; ma quello che ha più
paura de' forestieri che de' populi, debbe lasciarle indietro. {[}28{]}
Alla Casa Sforzesca ha fatto e farà più guerra el castello di Milano,
che vi edificò Francesco Sforza, che veruno altro disordine di quello
stato. {[}29{]} Però la migliore fortezza che sia, è non essere odiato
dal populo; perché, ancora che tu abbi le fortezze et il populo ti abbia
in odio, le non ti salvono: perché non mancano mai a' populi, preso che
gli hanno l'arme, forestieri che gli soccorrino. {[}30{]} Nelli tempi
nostri non si vede che quelle abbino profittato ad alcuno principe, se
non alla contessa di Furlí, quando fu morto il conte Ieronimo suo
consorte: perché mediante quella possé fuggire l'impeto populare et
aspettare el soccorso da Milano e recuperare lo stato; e li tempi
stavano allora in modo che il forestiere non posseva soccorrere il
populo. {[}31{]} Ma dipoi valsono ancora a lei poco le fortezze, quando
Cesare Borgia l'assaltò e che il populo, suo inimico, si congiunse col
forestieri. {[}32{]} Pertanto allora e prima sarebbe suto più sicuro a
lei non essere odiata dal populo, che avere le fortezze. {[}33{]}
Considerato adunque tutte queste cose, io lauderò chi farà le fortezze e
chi non le farà; e biasimerò qualunque, fidandosi delle fortezze,
stimerà poco essere odiato da' populi.

\quebra\section{QUOD PRINCIPEM DECEAT UT EGREGIUS HABEATUR
{[}Quello che s' appartenga fare a uno principe per esserre stimato e
reputato{]}}

{[}1{]} Nessuna cosa fa tanto stimare uno principe, quanto fanno le
grande imprese e dare di sé rari exempli. {[}2{]} Noi abbiamo nelli
nostri tempi Ferrando di Aragona, presente re di Spagna; costui si può
chiamare quasi principe nuovo, perché d'uno re debole è diventato per
fama e per gloria el primo re de' Cristiani; e se considerrete le
actioni sua, le troverrete tutte grandissime e qualcuna extraordinaria.
{[}3{]} Lui nel principio del suo regno assaltò la Granata, e quella
impresa fu il fondamento dello stato suo. {[}4{]} Prima, egli la fece
ozioso e sanza sospetto di essere impedito; tenne occupati in quella gli
animi di quelli baroni di Castiglia, e quali, pensando a quella guerra,
non pensavano ad innovazione: e lui acquistava in quel mezzo reputazione
et imperio sopra di loro, che non se ne accorgevano; possé nutrire, con
danari della Chiesa e de' populi, exerciti, e fare uno fondamento, con
quella guerra lunga, alla milizia sua, la quale lo ha di poi onorato.
{[}5{]} Oltre a questo, per potere intraprendere maggiori imprese,
servendosi sempre della religione, si volse ad una pietosa crudeltà,
cacciando e spogliando el suo regno de' Marrani: né può essere questo
exemplo più miserabile né più raro. {[}6{]} Assaltò, sotto questo
medesimo mantello, l'Affrica. Fece l'impresa di Italia. Ha ultimamente
assaltato la Francia. {[}7{]} E così sempre ha fatte et ordite cose
grandi, le quali hanno sempre tenuti sospesi et ammirati gli animi de'
subditi, et occupati nello evento di epse. {[}8{]} E sono nate queste
sua actioni in modo l'una da l'altra, che non ha dato mai infra l'una e
l'altra spazio alli uomini di potere quietamente operarli contro.

{[}9{]} Giova ancora assai ad uno principe dare di sé exempli rari circa
a' governi di dentro, -- simili a quegli che si narrano di messer
Bernabò da Milano, -- quando si ha l'occasione di qualcuno che operi
alcuna cosa extraordinaria, o in bene o in male, nella vita civile: e
pigliare uno modo, circa premiarlo o punirlo, di che si abbia a parlare
assai. {[}10{]} E sopratutto uno principe si debba ingegnare dare di sé
in ogni sua actione fama di uomo grande e di ingegno excellente.
{[}11{]} È ancora stimato uno principe, quando egli è vero amico e vero
inimico: cioè quando senza alcuno respecto egli si scuopre in favore di
alcuno contro ad un altro. {[}12{]} El quale partito fia sempre più
utile che stare neutrale: perché, se dua potenti tua vicini vengono alle
mane, o e' sono di qualità che, vincendo uno di quegli, tu abbia a
temere del vincitore, o no. {[}13{]} In qualunque di questi dua casi ti
sarà sempre più utile lo scoprirti e fare buona guerra; perché, nel
primo caso, se tu non ti scuopri sarai sempre preda di chi vince, con
piacere e satisfazione di colui che è stato vinto; e non hai ragione né
cosa alcuna che ti defenda, né chi ti riceva: perché chi vince non vuole
amici sospetti e che non lo aiutino nelle avversità; chi perde, non ti
riceve per non avere tu voluto con le arme in mano correre la fortuna
sua.

{[}14{]} Era passato in Grecia Antioco, messovi dagli Etoli per
cacciarne Romani; mandò Antioco oratori alli Achei, che erano amici de'
Romani, a confortargli a stare di mezzo: e dalla altra parte e Romani
gli persuadevano a pigliare le arme per loro. {[}15{]} Venne questa
materia a deliberarsi nel concilio delli Achei, dove el legato di
Antioco gli persuadeva a stare neutrali; a che il legato romano rispose:
«Quod autem isti dicunt, non interponendi vos bello, nihil magis alienum
rebus vestris est: sine gratia, sine dignitate praemium victoris
eritis.» {[}16{]} E sempre interverrà che colui che non è amico ti
ricercherà della neutralità, e quello che ti è amico ti richiederà che
ti scuopra con le arme. {[}17{]} Et e principi male resoluti, per
fuggire e presenti periculi, seguono el più delle volte quella via
neutrale, et il più delle volte rovinano.

{[}18{]} Ma quando el principe si scuopre gagliardamente in favore di
una parte, se colui con chi tu ti aderisci vince, ancora che sia potente
e che tu rimanga a sua discrezione, egli ha teco obligo, e' vi è
contratto l'amore: e gli uomini non sono mai sì disonesti, che con tanto
exemplo di ingratitudine e' ti opprimessino; dipoi le vittorie non sono
mai sì stiette che el vincitore non abbia ad avere qualche respetto, e
maxime alla iustizia. {[}19{]} Ma se quello con il quale tu ti aderisci
perde, tu sei ricevuto da lui, e mentre che può ti aiuta, e diventi
compagno di una fortuna che può resurgere.

{[}20{]} Nel secondo caso, quando quelli che combattono insieme sono di
qualità che tu non abbi da temere, di quello che vince,, tanto è
maggiore prudenzia lo aderirsi, perché tu vai alla ruina d'uno con lo
aiuto di chi lo doverrebbe salvare, se fussi savio; e vincendo rimane a
tua discrezione, et è impossibile, con lo aiuto tuo, che non vinca.
{[}21{]} E qui è da notare che uno principe debba advertire di non fare
mai compagnia con uno più potente di sé per offendere altri, se non
quando la necessità ti constringe, come di sopra si dice; perché,
vincendo, rimani suo prigione: e li principi debbono fuggire, quanto
possono, lo stare a discrezione di altri. {[}22{]} E Viniziani si
accompagnorono con Francia contro al duca di Milano, e potevano fuggire
di non fare quella compagnia: di che ne resultò la ruina loro. {[}23{]}
Ma quando e' non si può fuggirla, -- come intervenne a' Fiorentini,
quando el papa e Spagna andorono con li exerciti ad assaltare la
Lombardia, -- allora si debba el principe aderire per le ragioni sopradette. {[}24{]} Né creda mai alcuno stato potere pigliare partiti sicuri, anzi pensi di avere a prenderli tutti dubii; perché si trova questo, nell'ordine delle cose, che mai si cerca fuggire uno inconveniente che non si incorra in uno altro; ma la prudenza consiste in sapere conoscere le qualità delli inconvenienti e pigliare el men tristo per buono.

{[}25{]} Debbe ancora uno principe monstrarsi amatore delle virtù, dando
ricapito alli uomini et onorando gli excellenti in una arte. {[}26{]}
Appresso debbea animare e sua ciptadini di potere quietamente exercitare
li exercizii loro, e nella mercanzia e nella agricultura et in ogni
altro exercizio delli uomini; e che quello non tema di ornare le sua
possessione per timore che le gli sia tolta, e quello altro di aprire
uno traffico per paura delle taglie. {[}27{]} Ma debbe preparare premii
a chi vuole fare queste cose et a qualunque pensa in qualunque modo
ampliare o la sua città o il suo stato. {[}28{]} Debba oltre a questo,
ne' tempi convenienti dello anno, tenere occupati e populi con feste e
spettaculi; e perché ogni città è divisa in arte o in tribù, tenere
conto di quelle università, raunarsi con loro qualche volta, dare di sé
exempli di umanità e di munificenzia, tenendo sempre ferma nondimanco la
maestà della dignità sua.

\quebra\section{DE HIS QUOS A SECRETIS PRINCIPES HABENT
{[}De' secretari ch'e' principi hanno apresso di loro{]}}

{[}1{]} Non è di poca importanza a uno principe la electione de'
ministri, e quali sono buoni, o no, secondo la prudenzia del principe.
{[}2{]} E la prima coniettura che si fa del cervello d'uno Signore, è
vedere li uomini che lui ha dintorno: e quando sono suffizienti e
fedeli, sempre si può reputarlo savio, perché ha saputo conoscerli
suffizienti e sa mantenerli fideli; ma quando sieno altrimenti, sempre
si può fare non buono iudizio di lui: perché el primo errore che fa, lo
fa in questa electione.

{[}3{]} Non era alcuno che conoscessi messer Antonio da Venafro, per
ministro di Pandolfo Petrucci, principe di Siena, che non iudicasse
Pandolfo essere valentissimo uomo, avendo quello per suo ministro.
{[}4{]} E perché sono di tre generazione cervelli, -- l'uno intende da
sé, l'altro discerne quello che altri intende, el terzo non intende né
sé né altri: quel primo è excellentissimo, el secondo excellente, el
terzo inutile, -- conveniva pertanto di necessità che, se Pandolfo non
era nel primo grado, che fussi nel secondo.

{[}5{]} Perché ogni volta che uno ha iudizio di conoscere il bene o il
male che uno fa o dice, ancora che da sé non abbia invenzione, conosce
le opere buone e le triste del ministro e quelle exalta e le altre
corregge: et il ministro non può sperare di ingannarlo e mantiensi
buono.

{[}6{]} Ma come uno principe possa conoscere el ministro, ci è questo
modo che non falla mai: quando tu vedi el ministro pensare più a sé che
a te, e che in tutte le actioni vi ricerca dentro l'utile suo, questo
tale così fatto mai fia buono ministro, mai te ne potrai fidare. {[}7{]}
Perché quello che ha lo stato di uno in mano, non debbe pensare mai a
sé, ma sempre al principe, e non gli ricordare mai cosa che non
appartenga a lui; e dall'altro canto el principe per mantenerlo buono,
debba pensare al ministro, onorandolo, facendolo ricco, obligandoselo,
participandogli gli onori e carichi: acciò che veggia che non può stare
sanza lui, e che gli assai onori non li faccino desiderare più onori, le
assai ricchezze non gli faccino desiderare più ricchezze, li assai
carichi gli faccino temere le mutazioni. {[}8{]} Quando adunque li
ministri, e li principi circa e ministri, sono così fatti, possono
confidare l'uno dell'altro: quando altrimenti, sempre el fine fia
dannoso o per l'uno o per l'altro.

\quebra\section{QUOMODO ADULATORES SINT FUGIENDI
{[}In che modo si abbino a fuggire li adulatori{]}}

{[}1{]} Non voglio lasciare indrieto uno capo importante et uno errore
dal quale e principi con difficultà si difendano, se non sono
prudentissimi o se non hanno buona electione. {[}2{]} E questi sono gli
adulatori, delli quali le corte sono piene: perché li uomini si
compiacciono tanto nelle cose loro proprie, et in modo vi si ingannono,
che con difficultà si difendano da questa peste. {[}3{]} Et a volersene
difendere si porta periculo di non diventare contennendo; perché non ci
è altro modo a guardarsi dalle adulazioni, se non che gli uomini
intendino che non ti offendino a dirti el vero; ma quando ciascuno ti
può dire il vero, ti manca la reverenza. {[}4{]} Pertanto uno principe
prudente debba tenere uno terzo modo, eleggendo nel suo stato uomini
savii, e solo a quelli eletti dare libero arbitrio a parlargli la
verità, e di quelle cose sole che lui gli domanda e non d'altro, -- ma
debbe domandargli d'ogni cosa,-- e le opinioni loro udire: dipoi
deliberare da sé a suo modo; {[}5{]} et in questi consigli e con
ciascuno di loro portarsi in modo che ognuno cognosca che, quanto più
liberamente si parlerà più gli fia accepto: fuora di quelli, non volere
udire alcuno, andare dietro alla cosa deliberata et essere obstinato
nelle deliberazioni sua. {[}6{]} Chi fa altrimenti, o precipita per li
adulatori o si muta spesso per la variazione de' pareri: di che ne nasce
la poca existimazione sua.

{[}7{]} Io voglio a questo proposito addurre uno exemplo moderno. Pre'
Luca, uomo di Maximiliano presente imperatore, parlando di Sua Maestà
dixe come egli non si consigliava con persona e non faceva mai di cosa
alcuna a suo modo. {[}8{]} Il che nasceva dal tenere contrario termine
al sopradetto; perché lo Imperatore è uomo secreto, non comunica e sua
disegni, non ne piglia parere: ma come nel metterli in atto si
cominciano a conoscere e scoprire, gli cominciono ad essere contradetti
da coloro che lui ha dintorno, e quello, come facile, se ne stoglie; di
qui nasce che quelle cose che fa uno giorno, destrugge l'altro, e che
non si intenda mai quello si voglia o che disegni fare, e che non si può
sopra le sua deliberazioni fondarsi.

{[}9{]} Uno principe, pertanto debba consigliarsi sempre, ma quando lui
vuole e non quando altri vuole: anzi debba torre animo a ciascuno di
consigliarlo d'alcuna cosa, se non gliene domanda; ma lui debbe bene
essere largo domandatore, e dipoi circa alle cose domandate, paziente
auditore del vero: anzi, intendendo che alcuno per alcuno rispetto non
gliene dica, turbarsene. {[}10{]} E perché molti existimano che alcuno
principe, il quale dà di sé opinione di prudente, sia così tenuto non
per sua natura, ma per li buoni consigli che lui ha dintorno, sanza
dubio s'ingannano. {[}11{]} Perché questa è una regola generale che non
falla mai: che uno principe, il quale non sia savio per sé stesso, non
può essere consigliato bene, se già a sorte non si rimettessi in uno
solo che al tutto lo governassi, che fussi uomo prudentissimo. {[}12{]}
In questo caso potrebbe bene essere, ma durerebbe poco: perché quel
governatore in breve tempo gli torrebbe lo stato. {[}13{]} Ma
consigliandosi con più d'uno, uno principe che non sia savio non arà mai
e consigli uniti; non saprà per sé stesso unirgli; de' consiglieri,
ciascuno penserà alla proprietà sua; lui non gli saperrà nè correggere
né conoscere: e non si possono trovare altrimenti, perché gl'uomini
sempre ti riusciranno tristi, se da una necessità non sono fatti buoni.
{[}14{]} Però si conclude che li buoni consigli, da qualunque venghino, conviene naschino dalla~prudenza del principe, e non la prudenza del principe da' buoni consigli.

\quebra\section{CUR ITALIAE PRINCIPES REGNUM AMISERUNT
{[}Per qual cagione li principi di Italia hanno perso li stati loro{]}}

{[}1{]} Le cose soprascripte, osservate prudentemente, fanno parere
antico uno principe nuovo, e lo rendono subito più sicuro e più fermo
nello stato, che s'e' vi fussi antiquato dentro. {[}2{]} Perché uno
principe nuovo è molto più osservato nelle sue actioni che uno
ereditario: e quando le sono conosciute virtuose, pigliono molto più
gl'uomini e molto più gli obligano che el sangue antico. {[}3{]} Perché
gli uomini sono molto più presi dalle cose presenti che dalle passate;
e, quando nelle presenti truovono il bene, vi si godono e non cercano
altro; anzi, piglieranno ogni difesa per lui, quando non manchi
nell'altre cose a sé medesimo. {[}4{]} E così arà duplicata gloria, di
avere dato principio a uno principato et ornatolo e corroboratolo di
buone legge, di buone arme e di buoni amici e di buoni exempli; come
quello ha duplicata vergogna che, nato principe, per sua poca prudenzia
lo ha perduto

{[}5{]} E se si considerà quelli signori che in Italia hanno perduto lo
stato ne' nostri tempi, come il re di Napoli, duca di Milano et altri,
si troverà in loro, prima, uno comune difetto quanto alle arme, per le
cagioni che di sopra si sono discorse; dipoi si vedrà alcuni di loro o
che arà avuto inimici e populi, o, se arà avuto il popolo amico, non si
sarà saputo assicurare de' grandi. {[}6{]} Perché sanza questi difetti
non si perdono gli stati che abbino tanto nervo che possino tenere uno
exercito alla campagna. {[}7{]} Filippo Macedone, non il padre di
Alessandro, ma quello che fu da Tito Quinto vinto, aveva non molto stato
rispetto alla grandezza de' Romani e di Grecia, che lo assaltò:
nondimanco, per esser uomo militare e che sapeva intrattenere il populo
et assicurarsi de' grandi, sostenne più anni la guerra contro a quelli;
e se alla fine perdé el dominio di qualche città, gli rimase nondimanco
el regno.

{[}8{]} Pertanto questi nostri principi, e queli erano stati molti anni
nel loro principato, per averlo dipoi perso non accusino la fortuna, ma
la ignavia loro: perché, non avendo mai ne' tempi quieti pensato ch'e'
possono mutarsi, -- il che è comune difetto degli uomini, non fare conto
nella bonaccia della tempesta, -- quando poi vennono e tempi adversi,
pensorono a fuggirsi non a defendersi; e sperorono che e populi,
infastiditi per la insolenzia de' vincitori, gli richiamassino. {[}9{]}
Il quale partito, quando mancano gli altri, è buono, ma è ben male avere
lasciati li altri remedii per quello: perché non si vorrebbe mai cadere
per credere di trovare chi ti ricolga. {[}10{]} Il che o non adviene o,
se li adviene, non è con tua sicurtà, per essere quella difesa suta vile
e non dependere da te: e quelle difese solamente sono buone, sono certe,
sono durabili, che dependono da te proprio e dalla virtù tua.

\quebra\section{QUANTUM FORTUNA IN REBUS HUMANIS POSSIT ET QUOMODO ILLI SIT OCCURRENDUM
{[}Quanto possa la fortuna nelle cose umane e in che modo se li abbia a resistere{]}}

{[}1{]} E' non mi è incognito come molti hanno avuto et hanno opinione
che le cose del mondo sieno in modo governate, dalla fortuna e da Dio,
che li uomini con la prudenzia loro non possino correggerle, anzi non vi
abbino remedio alcuno; e per questo potrebbono iudicare che non fussi da
insudare molto nelle cose, ma lasciarsi governare alla sorte. {[}2{]}
Questa opinione è suta più creduta nelli nostri tempi per le variazione
grande delle cose che si sono viste e veggonsi ogni dí, fuora d'ogni
umana coniettura. {[}3{]} A che pensando io qualche volta, mi sono in
qualche parte inclinato nella opinione loro. {[}4{]} Nondimanco, perché
il nostro libero arbitrio non sia spento, iudico potere essere vero che
la fortuna sia arbitra della metà delle actioni nostre, ma che etiam lei
ne lasci governare l'altra metà, o presso, a noi. {[}5{]} Et assimiglio
quella a uno di questi fiumi rovinosi che quando si adirano allagano e
piani, rovinano li albori e li edifizii, lievono da questa parte
terreno, pongono da quell'altra: ciascuno fugge loro dinanzi, ognuno
cede allo impeto loro, sanza potervi in alcuna parte obstare. {[}6{]} E
benché sieno così fatti, non resta però che gli uomini, quando sono
tempi quieti, non vi potessino fare provedimenti e con ripari e con
argini: in modo che, crescendo poi, o eglino andrebbono per uno canale o
l'impeto loro non sarebbe né si dannoso né si licenzioso. {[}7{]}
Similmente interviene della fortuna, la quale dimonstra la sua potenza,
dove non è ordinata virtù a resisterle: e quivi volta li sua impeti,
dove ella sa che non sono fatti gli argini nelli ripari a tenerla.
{[}8{]} E se voi considerrete la Italia, che è la sedia di queste
variazioni e quella che ha dato loro il moto, vedrete essere una
campagna sanza argini e sanza alcuno riparo: ché, s'ella fussi riparata
da conveniente virtù, come é la Magna la Spagna e la Francia, o questa
piena non arebbe fatto le variazioni grande che la ha, o ella non ci
sarebbe venuta. {[}9{]} E questo voglio basti avere detto quanto allo
opporsi alla fortuna, in universali.

{[}10{]} Ma ristringendomi più a' particulari, dico come si vede oggi
questo principe felicitare e domani ruinare, sanza avergli veduto mutare
natura o qualità alcuna; il che credo che nasca, prima, dalle cagioni
che si sono lungamente per lo adrieto discorse: cioè che quel principe,
che si appoggia tutto in sulla fortuna, rovina come quella varia.

{[}11{]}Credo ancora che sia felice quello che riscontra il modo del procedere suo con le qualità de' tempi: e similmente sia infelice quello che con il procedere suo si discordano e tempi. {[}12{]} Perché si vede gli uomini, nelle cose che gli conducano al fine quale ciascuno ha innanzi, cioè glorie e ricchezze, procedervi variamente: l'uno con respetto, l'altro con impeto; l'uno per violenzia, l'altro con arte; l'uno per pazienza, l'altro con suo contrario; e ciascuno con questi diversi modi vi può pervenire. {[}13{]} E vedesi ancora dua respettivi, l'uno pervenire al suo disegno, l'altro no; e similmente dua egualmente felicitare con diversi studii, sendo l'uno rispectivo e l'altro impetuoso: il che non nasce da altro, se non da la qualità de' tempi che si conformano, o no, col procedere loro. {[}14{]} Di qui nasce quello ho detto, che dua, diversamente operando, sortiscano el medesimo effetto: e dua, egualmente operando, l'uno si conduce al suo fine e l'altro no. {[}15{]} Da questo ancora depende la variazione del bene; perché se uno, che si governa con rispetti e pazienzia, e tempi e le cose girono in modo che il governo suo sia buono, e' viene felicitando: ma se e tempi e le cose si mutano, rovina, perché non muta modo di procedere. {[}16{]} Né si truova uomo sì prudente che si sappi accomodare a questo: sì perché non si può deviare da quello a che la natura lo inclina, sì etiam perché, avendo sempre uno prosperato camminando per una via, non si può persuadere che sia bene partirsi da quella. {[}17{]} E però lo uomo respettivo, quando egli è tempo di venire allo impeto, non lo sa fare: donde rovina; ché se si mutassi di natura con li tempi e con le cose, non si muterebbe fortuna. {[}18{]} Papa Iulio II procedé in ogni sua actione impetuosamente; e trovò tanto e tempi e le cose conforme a quello suo modo di procedere, che sempre sortì felice fine. {[}19{]} Considerate la prima impresa che fe' di Bologna, vivendo ancora messer Giovanni Bentivogli. {[}20{]} Viniziani non se ne contentavono; el re di Spagna, quel medesimo; con Francia aveva ragionamenti di tale impresa. E lui nondimanco con la sua ferocia et impeto si mosse personalmente a quella expedizione. {[}21{]} La quale mossa fece stare sospesi e fermi Spagna e Viniziani, quegli per paura e quell'altro per il desiderio aveva di recuperare tutto el regno di Napoli; e dall'altro canto si tirò drieto il re di Francia perché, vedutolo quel re mosso e  desiderando farselo amico per abbassare Viniziani, iudicò non poterli negare gli exerciti sua anza iniuriarlo manifestamente. {[}22{]} Condusse adunque Iulio con la sua mossa impetuosa quello che mai altro pontefice, con tutta la umana prudenza, arebbe condotto. {[}23{]} Perché, se egli aspettava di partirsi da Roma con le conclusioni ferme e tutte le cose ordinate, come qualunque altro pontefice arebbe fatto, mai gli riusciva; perché il re di Francia arebbe avuto mille scuse, e li altri li arebbono messo mille paure. {[}24{]} Io voglio lasciare stare l'altre sua actioni, che tutte sono state simili e tutte gli sono successe bene: e la brevità della vita non li ha lasciato sentire il contrario; perché, se fussino sopravvenuti tempi che fussi bisognato procedere con respetti, ne seguiva la sua ruina: né mai arebbe deviato da quegli modi alli quali la natura lo inclinava.

{[}25{]} Concludo adunque che, variando la fortuna i tempi e stando li
uomini nelli loro modi obstinati, sono felici mentre concordano insieme
e, come discordano, infelici. {[}26{]} Io iudico bene questo, che sia
meglio essere impetuoso che respettivo: perché la fortuna è donna, et è
necessario, volendola tenere sotto, batterla et urtarla. {[}27{]} E si
vede che la si lascia più vincere da questi, che da quegli che
freddamente procedano: e però sempre, come donna, è amica de' giovani,
perché sono meno respettivi, più feroci e con più audacia la comandano.

\quebra\section{EXHORTATIO AD CAPESSENDAM ITALIAM IN LIBERTATEMQUE A BARBARIS VINDICANDAM
{[}Exortazione a pigliar la difesa di Italia e liberarla dalle mani de' barbari{]}}

{[}1{]} Considerato adunque tutte le cose di sopra discorse, e pensando meco medesimo se al presente in Italia correvano tempi da onorare uno nuovo principe, e se ci era materia che dessi occasione a uno prudente e virtuoso di introdurvi forma che facessi onore a lui e bene alla università delli uomini di quella, mi pare concorrino tante cose in benefizio di uno principe nuovo, che io non so qual mai tempo fussi più atto a questo. {[}2{]} E se, come io dixi, era necessario, volendo vedere la virtù di Moisè, che il populo d'Isdrael fussi schiavo in Egitto; et a conoscere la grandezza dello animo di Ciro, ch'e' Persi fussino oppressati da' Medi; e la excellenzia di Teseo, che li Ateniensi fussino dispersi; {[}3{]} così al presente, volendo conoscere la virtù di uno spirito italiano, era necessario che la Italia si riducessi nel termine presenti, e che ella fussi più stiava che li Ebrei, più serva ch'e' Persi, più dispersa che gli Ateniensi: sanza capo, sanza ordine, battuta, spogliata, lacera, corsa, et avessi sopportato d'ogni sorte ruina.

{[}4{]} E benché insino a qui si sia mostro qualche spiraculo in
qualcuno, da potere iudicare che fussi ordinato da Dio per sua
redemptione, tamen si è visto come dipoi, nel più alto corso delle
actioni sua, è stato dalla fortuna reprobato. {[}5{]} In modo che,
rimasa sanza vita, aspetta quale possa essere quello che sani le sue
ferite e ponga fine a' sacchi di Lombardia, alle taglie del Reame e di
Toscana, e la guarisca da quelle sue piaghe già per lungo tempo
infistolite. {[}6{]} Vedesi come la priega Iddio che li mandi qualcuno
che la redima da queste crudeltà et insolenzie barbare. {[}7{]} Vedesi
ancora tutta pronta e disposta a seguire una bandiera, pur che ci sia
uno che la pigli. {[}8{]} Né ci si vede al presente in quale lei possa
più sperare che nella illustre Casa vostra, la quale con la sua fortuna
e virtù, favorita da Dio e dalla Chiesa, della quale è ora principe,
possa farsi capo di questa redemptione. {[}9{]} Il che non fia molto
difficile, se Vi recherete innanzi le actioni e vita dei sopra nominati;
e benché quelli uomini sieno rari e maravigliosi, nondimeno furono
uomini, et ebbe ciascuno di loro minore occasione che la presente:
perché la impresa loro non fu più iusta di questa, né più facile, né fu
Dio più amico loro che a Voi. {[}10{]} Qui è iustizia grande: «iustum
enim est bellum quibus necessarium et pia arma ubi nulla nisi in armis
spes est». {[}11{]} Qui è disposizione grandissima: né può essere, dove
è grande disposizione, grande difficultà, pure che quella pigli delli
ordini di coloro che io ho proposti per mira. {[}12{]} Oltre a di
questo, qui si veggono extraordinarii senza exemplo condotti da Dio: el
mare si è aperto; una nube vi ha scòrto il cammino; la pietra ha versato
acque: qui è piovuto la manna; ogni cosa è concorsa nella Vostra
grandezza. {[}13{]} El rimanente dovete fare Voi: Dio non vuole fare
ogni cosa, per non ci tòrre el libero arbitrio e parte di quella gloria
che tocca a noi.

{[}14{]} E non è maraviglia se alcuno de' prenominati Italiani non ha
possuto fare quello che si può sperare facci la illustre Casa vostra, e
se, in tante revoluzioni di Italia et in tanti maneggi di guerra, e'
pare sempre che in Italia la virtù militare sia spenta; perché questo
nasce che gli ordini antichi di quella non erano buoni, e non ci è suto
alcuno che abbia saputo trovare de' nuovi. {[}15{]} E veruna cosa fa
tanto onore a uno uomo che di nuovo surga, quanto fa le nuove legge e li
nuovi ordini trovati da lui; queste cose, quando sono bene fondate e
abbino in loro grandezza, lo fanno reverendo e mirabile. {[}16{]} Et in
Italia non manca materia da introdurvi ogni forma: qui è virtù grande
nelle membra, quando non la mancassi ne' capi. {[}17{]} Specchiatevi ne'
duelli e ne' congressi de' pochi, quanto li Italiani sieno superiori con
le forze, con la destrezza, con lo ingegno; ma, come si viene alli
exerciti, non compariscono. {[}18{]} E tutto procede dalla debolezza de'
capi; perché quelli che sanno non sono obediti, et a ciascuno pare
sapere, non ci essendo insino a qui suto alcuno che si sia saputo
rilevato tanto, e per virtù e per fortuna, che li altri cedino.

{[}19{]} Di qui nasce che in tanto tempo in tante guerre fatte nelli passati XX anni, quando gli è stato uno exercito tutto italiano, sempre ha fatto mala pruova; di che è testimone prima el Taro, dipoi Alexandria, Capua, Genova, Vailà, Bologna, Mestri.


{[}20{]}  Volendo adunque la illustre Casa vostra seguitare quelli excellenti
uomini che redimirno le provincie loro, è necessario innanzi a tutte
le altre cose, come vero fondamento d'ogni impresa, provedersi d'arme
proprie, perché non si può avere né più fidi, né più veri, né migliori
soldati: e benché ciascuno di epsi sia buono, tutti insieme
diventeranno migliori quando si vedessino comandare dal loro principe,
e da quello onorare et intrattenere. {[}21{]} È necessario pertanto
prepararsi a queste arme, per potere con la virtù italica defendersi
dalli externi. {[}22{]} E benché la fanteria svizzera e spagnola sia
esistimata terribile, nondimanco in ambedua è difetto per il quale uno
ordine terzo potrebbe non solamente opporsi loro ma, confidare di
superargli. {[}23{]} Perché gli Spagnuoli non possono sostenere e
cavagli, e li Svizzeri hanno ad avere paura de' fanti quando gli
riscontrino nel combattere ostinati come loro: donde si è veduto e
vedrassi, per experienza, li Spagnuoli non potere sostenere una
cavalleria franzese e li Svizzeri essere rovinati da una fanteria
spagnuola. {[}24{]} E benché di questo ultimo non se ne sia visto
intera esperienzia, tamen se ne è veduto uno saggio nella giornata di
Ravenna, quando le fanterie spagnuole si affrontorono con le battaglie
tedesche, le quali servano el medesimo ordine che le svizzere: dove li
Spagnuoli, con la agilità del corpo et aiuto delli loro brocchieri,
erano entrati tra le picche loro sotto, e stavano securi ad offenderli
sanza che Tedeschi vi avessino remedio; e se non fussi la cavalleria
che gli urtò, gli arebbano consumati tutti. {[}25{]} Puossi adunque
conosciuto il difetto dell'una e dell'altra di queste fanterie,
ordinarne una di nuovo, la quale resista a' cavalli e non abbia paura
de' fanti: il che farà la generazione delle arme e la variazione delli
ordini; e queste sono di quelle cose che, di nuovo ordinate, dànno
reputazione e grandezza a uno principe nuovo.


{[}26{]}Non si debba adunque lasciare passare questa occasione, acciò
che la Italia vegga dopo tanto tempo apparie uno suo redemptore.
{[}27{]} Né posso exprimere con quale amore egli fussi ricevuto in tutte
quelle provincie che hanno patito per queste illuvioni externe, con che
sete di vendetta, con che ostinata fede, con che pietà, con che lacrime.
{[}28{]} Quali porte se li serrerebbono? Quali populi gli negherebbano
la obedienza? Quale invidia se li opporrebbe? Quale Italiano gli
negherebbe l'ossequio? Ad ognuno puzza questo barbaro dominio. {[}29{]}
Pigli adunque la illustre Casa vostra questo absumpto, con quello animo
e con quella speranza che si pigliano le imprese iuste; acciò che, sotto
la sua insegna, e questa patria ne sia nobilitata, e, sotto li sua
auspizi, si verifichi quel detto del Petrarca, quando dixe:

Virtù contro a furore

prenderà l'armi, e fia el combatter corto,

che l'antico valore

nelli italici cor non è ancor morto.
