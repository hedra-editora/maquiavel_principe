\textbf{Nicolau Maquiavel} (Florença, Itália, 1469--\textit{id.}, 1527) foi escritor, diplomata e estudioso de política italiana. Conhecido por seu pensamento político graças à recepção e à posterior aplicação da doutrina política exposta na mais famosa de suas obras,\\ \textit{O príncipe}. Filho de um jurista, nascido numa família que não dispunha de grandes recursos, ao que tudo indica não pôde seguir a educação formal. Mesmo assim, iniciou"-se cedo nos estudos, consultando a vasta biblioteca particular de seu pai, na qual havia títulos sobre política, jurisprudência, leis. Em 1512, com a dissolução do governo republicano, Maquiavel é destituído de seu cargo na chancelaria. Com a volta dos Médici ao poder, Maquiavel é preso, acusado sem provas de ter tomado parte num complô para matar Lourenço de Médici. Consegue a liberdade graças à ampla anistia concedida pelo papa Leão \textsc{x}, mas é banido de Florença, e passa os 14 anos restantes da vida retirado em San Casciano, um vilarejo próximo à Florença. Falece em 1527 quando os Médici são depostos e o novo governo republicano reinstaurado em Florença. Escreveu ainda \textit{Discursos sobre a primeira década de Tito Lívio} (1517), \textit{História de Florença} (1525), \textit{A arte da guerra} (1519--1520), e uma comédia, \textit{Mandrágora}, sendo os dois últimos os únicos publicados em vida. 

\textbf{O príncipe} foi escrito em 1513, quando Maquiavel já havia se retirado da vida pública, mas foi publicado apenas em 1532. A doutrina política de Maquiavel, que não deve ser confundida com o que vulgarmente se denomina “maquiavelismo” (“os fins justificam os meios”), trata de estabelecer como o soberano pode ou deve conservar o poder, ainda que isso demande o uso das aparências e da força. 

\textbf{José Antônio Martins} é doutor em Filosofia pela Universidade de São Paulo, onde defendeu tese sobre a noção de corrupção em Maquiavel, e ensina Filosofia Política na Universidade Estadual de Maringá.


