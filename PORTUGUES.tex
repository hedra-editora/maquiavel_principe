\section*{NICOLAUS~MACLAVELLUS \break{}CARTA~DEDICATÓRIA}

{[}1{]} Aqueles que desejam conquistar as graças de um príncipe
costumam, muitas vezes, presenteá-lo com aquelas coisas, entre as suas,
que têm como as mais caras ou das quais o vejam deleitar-se. Donde se
vê, muitas vezes, os príncipes serem presenteados com cavalos, armas,
tecidos bordados de ouro, pedras preciosas e semelhantes ornamentos
dignos da grandeza deles. {[}2{]} Desejando, pois, oferecer-me à Vossa
Magnificência com algum testemunho de minha servidão para convosco, não
encontrei entre os meus bens coisa que eu tenha mais cara ou tanto
estime quanto o conhecimento das ações dos grandes homens, apreendidas
por mim mediante uma longa experiência das coisas modernas e uma
contínua lição das coisas antigas, sobre as quais tenho com grande
diligência longamente refletido e examinado, e reunido agora em um
pequeno volume, que envio à Vossa Magnificência. {[}3{]} Embora julgue
esta obra indigna de estar à Vossa presença, muito confio, todavia, que
pela sua benevolência, ela deva ser aceita, pois considero que não vos
possa ser oferecido maior presente do que a faculdade para poder, em
brevíssimo tempo, entender tudo aquilo que eu, em tantos anos e com
tantos incômodos e perigos, conheci e entendi\footnote{O binômio citado
  revela o modo como Maquiavel entende a aquisição do conhecimento do
  mundo político. O conhecer diz respeito à experiência concreta da vida
  política, mas que não completa todo o, digamos, ``processo do
  conhecimento político''. Faz-se necessário refletir sobre esse
  material bruto da experiência política, principalmente cotejando com o
  conhecimento acumulado da História. Portanto, a expressão ``conhecer e
  entender'' dá a medida de como se realiza o conhecimento político em
  sua plenitude, ou seja, experiência associada à reflexão iluminada
  pela História.}. {[}4{]} Obra à qual eu não adornei e nem preenchi com
períodos longos ou com palavras empoladas e magníficas, ou com qualquer
outro enfeite ou ornamento extrínseco, com os quais comumente muitos
descrevem e adornam as suas coisas, porque desejei ou que nenhuma coisa
a honre ou que somente a variedade da matéria e a importância do assunto
a torne agradável. {[}5{]} Tampouco quero que seja considerado como
presunção que um homem de baixa e ínfima condição ouse discorrer e
regular os governos dos príncipes, porque, assim como aqueles que
desenham os territórios se põem no plano baixo para analisar a natureza
dos montes e dos lugares altos, e, para considerar aquelas coisas de
baixo, põem-se sobre o alto dos montes, igualmente, para conhecer bem a
natureza dos povos, necessita ser príncipe e, para conhecer bem a
natureza dos príncipes, convém ser homem do povo.

{[}6{]} Receba, portanto, Vossa Magnificência este pequeno presente
segundo aquele ânimo que eu vos mando. Obra a qual, se diligentemente
considerada e lida, vos fará conhecer um grande desejo que está no meu
interior: que seja alcançcada aquela grandeza que a fortuna e outras
qualidades vossas prometem. {[}7{]} E se Vossa Magnificência, do ápice
de vossa grandeza, alguma vez voltar os olhos para estes lugares baixos,
entenderá como eu suporto indignamente uma grande e contínua maldade da
fortuna.

\setcounter{secnumdepth}{2}

\quebra\section{\emph{QUOT SINT GENERA PRINCIPATUUM ET~QUIBUS~MODIS~ACQUIRANTUR} {[}Quantos~são~os~gêneros~dos~principados~e~de~que~modos~são~adquiridos{]}}

{[}1{]} Todos os governos\footnote{O termo \emph{stato} no
  \emph{Príncipe} possui ao menos duas acepções diferentes: pode
  significar a condição política, o ``estado'' político do governante,
  em suma seu \emph{status}; pode significar, ainda, o governo adotado
  por uma província, como sinonimo de regime político. Tendo em vista
  essa dualidade de acepções, adotaremos o uso de ``estado'' (com o
  \emph{e} em minusculo), apenas quando esse indicar o ente político ou
  o governo. Quando \emph{stati} fizer referência à condição política
  adotaremos ``status'', pois se trata da condição política de um
  indivíduo. Por fim, utilizaremos ``governo'' para traduzir
  \emph{stato} como regime político ou governo. Sobre esse tema e uma
  análise da bibliografia a respeito cf. Ames, José Luiz. ``A formação
  do conceito moderno de estado: a contribuição de Maquiavel'' in
  \emph{Revista Discurso}, São Paulo, 41, 2011 (293-328).}, todos os
domínios\footnote{Na edição Martelli (2006, p. 63-64, nota 3 e 5) consta
  que \emph{stati ... domini} referem-se respectivamente às repúblicas e
  aos principados do final do período, haja vista que eles estão em
  paralelo pela construção empregada.} que tiveram e têm
autoridade\footnote{Optamos por traduzir \emph{imperio} por
  ``autoridade'' e ``comando militar'' e não ``poder'', justamente para
  não incorrermos na acepção moderna de ``poder'' que pressupõe a noção
  de soberania (Duso, 2005, cap. 1), algo que não se pode afirmar que
  esteja presente em início do século XVI, haja vista a elaboração ainda
  do conceito de soberania entre os pensadores políticos. Por outro
  lado, \emph{imperio} em italiano descende diretamente do latim
  \emph{imperium}, termo esse associado à \emph{autorictas}. Nessa
  acepção, a opção de tradução conserva uma ligação terminológica da
  tradição romana e evita uma aproximação conceitual não presente no
  pensamento maquiaveliano de início do século XVI. Na edição francesa
  de Fournel e Zancarini, eles traduzem \emph{dominii} por
  \emph{seigneuries} (senhorias) e \emph{imperio} por
  \emph{commandement} (comando).} sobre os homens, foram e são ou
repúblicas ou principados\footnote{Maquiavel abre a obra com uma
  definição geral sobre o campo do político: ele é uma autoridade, um
  \emph{imperio} sobre os homens. Depois dessa curta definição faz-se
  uma primeira tipificação dessa autoridade. Como é evidente, tal
  definição é precária, necessitando de uma maior elaboração, algo que
  será feito, no caso dos principados, nessa obra para as repúblicas em
  outra obra. Importa destacar também que o poder político se faz sobre
  homens e não sobre territórios, ou seja, os estados e domínios são
  relação de autoridade sobre pessoas e não territórios definidos, em
  mais uma clara evidência de que não estamos tratando de uma noção
  política que pressupõe soberania típica.}. {[}2{]} Os principados ou
são hereditários, nos quais o sangue do seu senhor foi por longo tempo
príncipe, ou são novos. {[}3{]} Os novos ou são inteiramente novos, como
foi Milão para Francesco Sforza\footnote{Francesco Sforza, 1401-1466,
  conhecido pela sua grande habilidade militar, casado com Bianca Maria,
  filha do Duque de Milão (Filipo Maria Visconti), herdou o governo da
  cidade depois da morte do duque em 1447, que não deixou descendentes,
  e dos conflitos travados contra os partidários da República
  Ambrosiana, assumindo o governo definitivamente em 1448.}, ou são como
membros acrescentados ao Estado hereditário do príncipe que os
conquista, como é o reino de Nápoles para o rei de Espanha\footnote{O
  reino de Nápoles foi conquistado em 1452 pelo rei espanhol Afonso V de
  Aragão. Com a morte de Afonso, em 1496, Nápoles passou ao seu filho
  bastardo Ferdinando. Um tratado secreto de 11 de novembro de 1500
  assegurou a Luiz XII, rei francês, o título de rei de Nápoles. Em
  julho de 1501 o reino foi atacado pelo norte e pelo sul pelos
  franceses: Ferdinando I de Aragão renunciou de imediato a luta e se
  pôs nas mãos de Luiz. Mas, em julho de 1502, o conflito entre
  espanhóis e franceses desembocou numa guerra aberta. No final de 1503
  os franceses foram derrotados, ficando o reino sob o controle
  espanhol.}. {[}4{]} Estes domínios assim conquistados ou estão
habituados a viver sob um príncipe ou estão acostumados a ser livres; e
são conquistados ou com as armas de outros ou com as próprias, ou pela
fortuna ou pela \emph{virtù}\footnote{Após ter dado de início uma
  definição geral sobre o campo do político, Maquiavel passa, no
  restante do capítulo, a fazer uma classificação ou tipificação de cada
  item, à maneira dos tratados medievais e bem ao sabor do estilo da
  ciência aristotélica. Verifica-se, pois, a seguinte divisão:
 Outro aspecto que se nota é que essa divisão dos itens obedece quase
  que integralmente à sequência das temáticas dos capítulos, conforme
  indicado nos parênteses. Isso revela que Maquiavel pretende, neste
  primeiro capítulo, expor a sequência argumentativa da obra, como que
  fazendo um sumário do que conterá a primeira parte do livro, fórmula
  essa também característica dos tratados medievais. Enfim, por tais
  elementos constitutivos da obra, pode-se afirmar com segurança a
  intenção do autor em apresentar um tratado sobre a forma de governo
  principesca, haja vista ainda que o livro se nomeia como \emph{De
  Principatibus}, que, numa tradução literal, deveria ser \emph{Sobre os
  principados}.}.
  %\begin{longtable}[]{@{}lll@{}}
  %\toprule
  %Governos e Domínios & República\tabularnewline
  %\midrule
  %\endhead
  %\begin{minipage}[t]{0.32\columnwidth}\raggedright\strut
  %\strut
  %\end{minipage} &
  %\begin{minipage}[t]{0.32\columnwidth}\raggedright\strut
  %Principados\strut
  %\end{minipage} &
  %\begin{minipage}[t]{0.32\columnwidth}\raggedright\strut
  %Hereditários
%
  %{[}2{]}\strut
  %\end{minipage}\tabularnewline
  %& & Novos\tabularnewline
  %Membros acrescentados & Habituados a viver sobre o\tabularnewline
  %ou conquistados & comando de um príncipe {[}3-4{]}\tabularnewline
  %& Habituados a viverem livremente\tabularnewline
  %& {[}5{]}\tabularnewline
  %& Conquistados com armas alheias\tabularnewline
  %& {[}7{]}\tabularnewline
  %& Conquistados com armas\tabularnewline
  %& próprias {[}6{]}\tabularnewline
  %& Conquistados pela fortuna {[}7{]}\tabularnewline
  %& Conquistados pela \emph{virtù} {[}8-9{]}\tabularnewline
  %\bottomrule
  %\end{longtable}

\quebra\section{\emph{DE PRINCIPATIBUS HEREDITARIIS} {[}Dos principados hereditários{]}}

{[}1{]} Deixarei de lado a discussão sobre as repúblicas, porque, em
outro momento, dissertei longamente sobre elas\footnote{Essa frase tem
  sido fonte de grandes discussões entre os especialistas,
  principalmente no tocante a qual obra Maquiavel faz remissão. Os
  comentadores levantam três hipóteses: a) que haveria um livro sobre as
  repúblicas escrito antes do \emph{Príncipe}; b) que esta obra seria os
  \emph{Discursos sobre a primeira década de Tito Lívio}, embora a data
  mais plausível sobre a composição desse texto seja posterior ao
  \emph{Príncipe}; c) que uma primeira parte dos \emph{Discursos} foi
  composta antes do \emph{Príncipe}. Apesar de não haver um consenso, a
  hipótese mais provável é a terceira conforme demonstra Felix Gilbert
  (1977, p. 225-253) e também Gennaro Sasso (1993, p. 349-359; p.
  563-568).}. {[}2{]} Ocupar-me-ei somente do principado, e retecerei as
urdiduras\footnote{``retecer as urdiduras'' (\emph{ritexendo} ou
  \emph{ritessendo gli orditi}) remete ao primeiro momento do trabalho
  do tecelão que concebe e estabelece a trama dos fios no tear antes de
  começar a confeccionar o tecido. Maquiavel estabelece aqui uma
  analogia entre o trabalho do pensador político e o do tecelão, pois,
  assim como o tecelão, o seu trabalho consiste num movimento de
  construção e desconstrução das tramas das estruturas políticas.
  Confira também uma metáfora semelhante em Dante: \emph{``}{[}...{]}
  \emph{si mostrò spedita / l'anima santa di metter la trama / in quella
  tela ch'io le porsi ordita''.} (\emph{Divina Comédia, Paraíso}, XVII,
  vv. 100-102) e nos seus \emph{Discursos sobre a Primeira década de
  Tito Lívio}, ``\emph{gli uomini possono secondare la fortuna e non
  opporsegli, possono tessere gli orditi suoi e non rompergli}''. (livro
  I, cap. 2, linha 24). Nunca é demais lembrar que a Florença dos
  séculos XV e XVI era famosa por saber trabalhar com tecidos,
  particularmente com a lã, donde essa analogia aproximar o texto do
  contexto de seu tempo.} acima descritas, e demonstrarei como estes
principados podem ser governados e conservados\footnote{Esse parágrafo
  inicial tem uma dupla função na economia do texto: por um lado
  responder a demanda apresentada no início do cap. I, mostrar porque
  não tratará de repúblicas, visto que estas também são governos
  (\emph{stati}), mas somente dos principados. No primeiro período,
  Maquiavel já responde àquela demanda deixada de início e parte para um
  dos temas centrais que percorrerá a sua obra: a conquista, o governo e
  a conservação dos principados. Contudo, parece que temos uma
  redundância ao final, pois governar se confunde muitas vezes com o
  conservar. De fato, ao longo da obra, esse binômio parece se
  confundir, mas convém, desde já, lembrar que governar é estar na
  condição de príncipe, o que implica também em conquista. A conservação
  restringe-se às ações de manutenção da configuração política. No
  limite, temos duas ações necessárias para aqueles que se põem na
  condição de liderar o processo político como será o caso dessa
  personagem também política do príncipe que se apresentará ao longo da
  obra.}.

{[}3{]} Digo, portanto, que nos governos hereditários e acostumados à
dinastia do seu príncipe são muito menores as dificuldades para
conservá-los do que nos novos, porque basta não preterir os
ordenamentos\footnote{O termo \emph{ordine}, aqui empregado no seu
  plural \emph{ordini}, pode ser traduzido por ``ordem'',
  ``ordenamentos'' ou mesmo, no caso do mundo político, por
  ``instituições''. Seu uso remete à disposição das diversas partes
  políticas que compõem um corpo político. Como bem lembram Fournel e
  Zancarini, no \emph{Príncipe,} Maquiavel mobiliza uma série de termos
  derivados de \emph{ordini}, como os verbos \emph{riordinare,
  disordinare, ordinare} e os substantivos \emph{ordinaria},
  \emph{extraordinario} etc., de modo que a tradução de \emph{ordine}
  por ``instituições'' impediria a conservação desse sentido original,
  embora ela seja mais próxima da nossa compreensão. Todavia, ao optar
  em traduzir \emph{ordine} por ``ordenação'', não somente recuperamos
  uma compreensão também presente em nossa língua -- à semelhança do
  emprego de ``ordenamento jurídico'' --, como abrimos a possibilidade
  de conservar, nos demais empregos, o mesmo radical: para
  \emph{riordinare,} reordenar; para \emph{disordinare,} desordenar etc.
  (Cf. \emph{Le Prince,} org. Fournel e Zancarini, 2002, p. 573-578).}
de seus antecessores e posteriormente contemporizar com os
acidentes\footnote{Aqui a expressão tem o sentido de ``governar com as
  circunstâncias''. O termo ``acidente'' é empregado como um imprevisto,
  um acaso, e não numa acepção própria da metafísica, por oposição à
  essência.}, de modo que, se tal príncipe tiver uma indústria
ordinária, sempre conservará o seu estado, a não ser que uma força
extraordinária e excessiva o prive dele\footnote{Convém destacar, aqui,
  o jogo terminológico entre a indústria \emph{ordinária} (uma
  engenhosidade comum, uma criatividade mediana) e a força
  \emph{extraordinária} e excessiva (uma ação acima da média). Ou seja,
  para o governante herdeiro, em seguindo o normal dos acontecimentos,
  sem nada de extraordinário, fora do comum, não há com o que se
  preocupar. Neste caso, a conservação é ordinária e a conquista deve
  ser extraordinária.}. E tendo sido dela privado, reconquista tal
condição na medida em que o conquistador enfrentar alguma
adversidade\footnote{O argumento maquiaveliano neste período destaca a
  pouca dificuldade em governar um regime político herdado: basta ter
  uma indústria ordinária, ou seja, basta não modificar muito, manter os
  ordenamentos como estão. No limite, esse herdeiro não requer muitas
  qualidades políticas, mas uma condição mediana já é o suficiente. A
  força e o prestígio herdados são garantias suficientes para conservar
  sua condição de governante.}.

{[}4{]} Temos na Itália, por exemplo, o duque de Ferrara\footnote{A
  família ducal dos Estensi, em particular Hercules I (duque de 1471 a
  1505) e Afonso I (duque de 1505 a 1534).}, que não resistiu aos
assaltos dos venezianos\footnote{Os venezianos iniciaram uma guerra
  contra a cidade de Ferrara em maio de 1482 e os seus sucessos
  obrigaram os reinos da Itália a formar uma coalizão para a defesa do
  governo da família Estensi.} em 1484, nem aqueles do papa
Júlio\footnote{Giuliano della Rovere, 1443-1513, foi eleito papa em 1503
  sob o nome de Júlio II. Maquiavel se refere aqui à guerra de Julio II,
  aliando-se aos venezianos e aos suíços, contra Ferrara em 15 de
  fevereiro de 1510.} em 1510, por outras razões senão por ser antigo
naquele domínio. {[}5{]} Porque o príncipe natural tem menores razões e
menor necessidade de ofender, donde se segue que seja mais amado. E se
vícios extraordinários não o fizer odiado, é razoável que seja
naturalmente benquisto pelos seus. {[}6{]} E na Antiguidade e
continuação do domínio são extintas a memória e os motivos das
inovações: porque uma mudança sempre deixa o fundamento\footnote{O termo
  \emph{adentellato} tem aqui o sentido de introduzir, entrar, colocar
  algo para dentro, que neste caso vem de encontro da noção de fundação
  política.} para a edificação de outra.


\quebra\section{\emph{DE PRINCIPATIBUS MIXTIS} {[}Dos~principados~mistos{]}}

{[}1{]} Todavia, é no principado novo que residem as dificuldades. Em
primeiro lugar, se o principado não é totalmente novo, mas é como um
membro acrescido\footnote{Maquiavel concebe o principado e as repúblicas
  como corpos políticos, que assim como os corpos naturais, possuem
  membros ou partes. Neste caso, trata-se de um acréscimo ao corpo
  político pré existente, donde a noção seguinte de principado misto ou
  misturado ao corpo político já existente. Martelli (2006), na edição
  comentada, apresenta a hipótese, a partir de uma imprecisão na
  concordância verbal do texto original italiano (\emph{quale è le quali
  sono}) que Maquiavel operou aqui uma fusão de capítulos, no caso, dois
  capítulos que seriam separados: um dedicado aos principados novos e um
  dedicado aos principados mistos ou conquistados. De fato, conforme
  enunciado no capítulo I, um dos objetivos da obra é tratar dos
  principados novos (um ausente capítulo \emph{De principatibus nouis})
  e em seguida dos principados conquistados ou mistos. Contudo, o que se
  verifica nesse início é uma junção, na medida em que Maquiavel
  equipara os desafios dos principados novos com o dos mistos já nesse
  período introdutório. Com efeito, esse capítulo, juntamente com os
  capítulos VII e XIX, perfazem um dos maiores do livro e apresentam
  dois momentos da argumentação: quando ele trata dos principados
  conquistados, como no caso grego, esse um exemplo de principado
  inteiramente novo {[}da linha 01 a 30{]} e, no segundo exemplo {[}da
  linha 31 a 50{]}, o caso da França com as conquistas do rei Luiz, esse
  um exemplo de principados mistos. Note-se adiante que a linha 31
  inicia-se com ``Mas voltemos à França'', o que evidencia claramente o
  movimento argumentativo. No limite, a hipótese de Martelli nos é útil
  para perceber esses dois momentos argumentativos dos capítulos e como
  eles correspondem àquilo que foi indicado no capítulo I. Cf. Martelli
  (2006), \emph{Nota al texto}, ed. Comentada, p. 428-443.} (e o
conjunto destes principados pode ser chamado de misto), as suas
diferenciações nascem, primeiramente, de uma dificuldade natural
presente em todos os principados novos, a saber: eles são como homens
que voluntariamente mudam de senhor, acreditando melhorar, e esta crença
os faz pegar em armas contra este, no que se enganam, porque vêem
posteriormente pela experiência que pioraram. {[}2{]} O que depende de
uma outra necessidade natural e ordinária, a qual faz com que sempre
precise importunar -- com gente armada e outras infinitas injúrias que a
nova conquista traz consigo -- aqueles de quem se tornou novo príncipe.
{[}3{]} De modo que terá como inimigos todos aqueles que tiver
importunado na ocupação\footnote{O verbo \emph{occupare} empregado
  apresenta um sentido mais fraco politicamente que o termo
  \emph{conquista}, o que nos remete ao caráter provisório dessa
  condição política.} daquele principado e não poderá conservar como
seus amigos aqueles que nele o colocaram, por não poder satisfazer-lhes
naquilo que pressupunham e por não poder usar contra eles os remédios
fortes, uma vez que você tem obrigação para com eles. Porque, ainda que
se tenha um fortíssimo exército seu, sempre se precisa da ajuda dos
provincianos para entrar em uma província\footnote{A nomeação dos
  territórios no Renascimento italiano apresenta uma peculiaridade: uma
  pequena vila de casas, que para nós seria um distrito, bairro ou
  cidade pequena é nomeada como \emph{paese}, que pertenciam, do ponto
  de vista político-administrativo, a uma cidade. As cidades, neste caso
  dos territórios italianos desse período, eram repúblicas autônomas ou
  lutavam pela sua autonomia diante das potências políticas, sejam
  outras repúblicas conquistadoras, seja o Papado, sejam os demais
  Impérios de então: o Francês, o Otomano, o Espanhol etc. ``Província''
  era ora empregada como uma parte ou fração do território, que poderia
  incluir várias cidades, como no caso da Toscana, Lombardia e Veneto,
  embora isso não significasse união política ou administrativa, ora
  como uma designação geográfica, sem qualquer caráter político.}.
{[}4{]} Por estes motivos Luís XII\footnote{Luiz XII ocupou Milão em
  setembro 1499 e a perdeu em fevereiro de 1500.}, rei da França,
rapidamente ocupou Milão e rapidamente a perdeu; para perdê-la pela
primeira vez, bastaram as forças próprias de Ludovico: porque aquele
povo que lhe tinha aberto as portas, encontrando-se enganados pela
opinião dele e desiludidos daquele futuro que tinham pressuposto, não
podiam suportar os aborrecimentos de um novo príncipe\footnote{Em abril
  de 1500, as forças de Ludovico foram definitivamente expulsas de
  Milão.}.

{[}5{]} É bem verdade que, conquistando-se pela segunda vez os países
rebelados, eles se perdem com mais dificuldade, porque o senhor,
aproveitando-se da ocasião da rebelião -- para assegurar-se -- tem menos
escrúpulos em punir os delinquentes, identificar os suspeitos,
precaver-se nos seus pontos mais fracos. {[}6{]} De modo que, na
primeira vez, para fazer perder Milão para França, bastou um duque
Ludovico amotinar- se em seus domínios, depois, para fazê-lo perder na
segunda vez, precisou ter todo mundo contra\footnote{``Todo mundo'', no
  caso, a Igreja, a República de Veneza e a Espanha que expulsaram os
  franceses da Itália, entre outubro de 1511 e abril de 1512.} e que os
seus exércitos fossem eliminados ou expulsos da Itália: o que decorre
das razões sobreditas. {[}7{]} Não obstante, tanto na primeira como na
segunda vez lhe foi tirado o ducado. As razões gerais da primeira já
foram discutidas, resta agora falar sobre a segunda e ver que remédios
ele\footnote{O rei francês.} tinha e quais poderia ter alguém nas suas
condições para poder melhor conservar a sua conquista, coisa que não fez
a França.

{[}8{]} Digo, portanto, que estes estados -- que, ao serem conquistados,
agregam-se a um estado mais antigo do que aquele que os conquistou -- ou
são da mesma província e da mesma língua, ou não o são. {[}9{]} Quando o
são, grande é a facilidade em tê-los, mais ainda quando não estão
habituados a viver livremente: e para possuí-los com segurança basta
extinguir a dinastia do príncipe que os dominava, porque, nas outras
coisas conservando-se-lhes as velhas condições e não havendo diferença
de costumes, os homens vivem sossegadamente, como se viu ocorrer na
Borgonha, na Bretanha, na Gasconha e na Normandia, que durante tanto
tempo foram da França; e ainda que haja alguma diferença na língua, os
costumes são, todavia, similares e eles podem facilmente conviver entre
si. {[}10{]} E quem os conquista, querendo tê-los, deve ter dois
cuidados: um, que seja extinta a dinastia do seu antigo príncipe, outro,
de não alterar nem as suas leis nem os seus impostos, de tal modo que, em brevíssimo tempo,
juntamente com seu principado antigo, faça tudo um só corpo.

{[}11{]} Contudo, quando se conquista estados em uma província de
língua, costumes e ordenações diferentes, aqui se encontram as
dificuldades e aqui é preciso ter grande fortuna e grande indústria para
mantê-los. {[}12{]} E um dos maiores e mais eficazes remédios seria que
a pessoa que o conquista vá habitar pessoalmente o lugar; isto tornaria
mais segura e mais durável aquela posse, tal como fez o Turco\footnote{Como
  era chamado o imperador turco Maomé II.} na Grécia: ao qual, com todas
as outras medidas\footnote{\emph{Ordini} tem aqui o sentido de medidas,
  meios, modos de proceder e não de ordenamentos.} observadas por ele
para manter aquele estado, não seria possível mantê-lo se não fosse lá
habitar. {[}13{]} Porque, estando ali, vê-se nascer as desordens e
rapidamente se lhes pode dar remédio; não estando ali, só as percebe
quando já forem grandes e não houver mais remédio. Além disso, a
província não é espoliada por seus oficiais, os súditos se satisfazem
com a possibilidade de recorrer direta e facilmente ao príncipe, donde
têm mais razões de amá-lo, quando querem ser bons, e de temê-lo, quando
querem ser o contrário; e quem do exterior desejasse assaltar aquele
estado, terá mais respeito, pois se o príncipe habitar o lugar, com
grandíssima dificuldade pode perdê-lo.

{[}14{]} Outro remédio melhor é fundar colônias em um ou dois lugares
para que sejam como pontos de apoio para aquele estado conquistador,
pois é necessário ou fazer isto ou manter na província conquistada muita
gente armada e muitos soldados. {[}15{]} Com as colônias não se gasta
muito e, sem suas despesas ou com poucas, pode-se funda-las e mantê-las,
e somente ofende aqueles de quem tira os campos e as casas para dá-las
aos novos habitantes, que são uma parte mínima daquele governo; {[}16{]}
e aqueles que ele (o conquistador) importuna, permanecendo dispersos e
pobres, jamais lhe podem prejudicar; e todos os outros permanecem por um
lado sem serem importunados -- e por isto deveriam acalmar-se -- e, por
outro lado, preocupados em não errar, por temor que aconteça a eles
aquilo que ocorreu com os espoliados. {[}17{]} Concluo que estas
colônias não são dispendiosas, são mais fiéis, importunam menos, e que
os importunados não podem prejudicar, pois são pobres e dispersos, como
já foi dito. {[}18{]} Pelo que há de se notar que os homens devem ser ou
acalentados ou eliminados: porque caso se vinguem das afrontas leves,
não podem se vingar das afrontas graves, pois a afrontas que se faz a um
homem deve ser de tal modo que não tenha de temer a vingança. {[}19{]}
Mas mantendo no território conquistado, em vez das colônias, gente
armada, gasta-se muito mais, tendo de consumir com a guarda toda a renda
daquela condição, de modo que a conquista se torna perda, e afronta
muito mais, porque causa dano a todo aquela condição, transferindo de um
lado para outro alojamentos e exércitos. Todos se ressentem de tal
incômodo e todos se tornam inimigos do conquistador, e são os inimigos
que lhe podem prejudicar permanecendo vencidos em suas próprias casas.
{[}20{]} De qualquer modo, esta proteção é inútil, assim como aquela das
colônias é útil.

{[}21{]} Quem está em uma província diferente deve ainda, como foi dito,
fazer-se chefe e defensor dos vizinhos de menor poder, empenhar-se em
enfraquecer os poderosos daquele lugar e precaver-se de que por um
acidente não entre aí algum forasteiro tão poderoso quanto ele. E sempre
acontecerá que ser introduzido na província por aqueles que aí estão
descontentes, ou por desmedida ambição ou por medo, como já se viu com
os etólios que introduziram os romanos na Grécia, e, em todas as outras
províncias em que os romanos entraram, foram eles introduzidos pelos
próprios provincianos\footnote{Referência aos conflitos entre os romanos
  e a liga Aquéia na passagem do século II para o século I a.C.}.
{[}22{]} E é esta a ordem das coisas: assim que um forasteiro poderoso
entra em uma província, todos aqueles que nela são menos poderosos
aderem a ele, movidos pela inveja que possuem contra quem tem o poder
sobre eles, tanto que, no que diz respeito a estes menos poderosos, ele
não se fadigará em nada para se proteger, porque de imediato todos se
unem voluntariamente, perfazendo um todo com a condição que ele aí
conquistou. {[}23{]} Há somente que tomar cuidado para que eles não
ganhem muita força e muita autoridade, pois facilmente pode, com a sua
força e com o favor deles, diminuir os que são poderosos, para manter o
controle total daquela província. E quem não governar bem esta parte,
perderá rapidamente aquilo que conquistou, e, enquanto a mantiver,
sofrerá infinitas dificuldades e aborrecimentos.

{[}24{]} Os romanos, nas províncias que pilharam, observaram bem estas
normas: fundaram colônias, detiveram os menos poderosos, sem deixar
crescer o poder deles, enfraqueceram os poderosos e não deixaram que os
forasteiros poderosos alcançassem reputação nela. {[}25{]} Como exemplo,
basta-me apenas a província da Grécia: nela os romanos detiveram os
aqueus e os etólios, enfraqueceram o reino dos macedônios e expulsaram
Antíoco da Grécia. Nem os méritos dos aqueus ou dos etólios impediram
que os romanos conquistassem algum governo, nem as tentativas de
persuasão de Felipe os induziram a serem amigos dos macedônios sem
enfraquecê-los, tampouco o poder de Antíoco pôde fazê-los consentir que
ele mantivesse naquela província algum status\footnote{Conforme indicado
  no período anterior, o exemplo histórico expressa a divisão dos papéis
  políticos, pois: os aqueus e os etólios são os menos poderosos, Felipe
  é o poderoso da província e Antíoco é o forasteiro poderoso,
  respectivamente Felipe V da Macedônia (237-179 a.C.) e Antíoco III da
  Síria (242- 187 a.C.), imperador selêucida. Em 192 Antíoco tentou se
  inserir nos conflitos entre os romanos e os etólios, mas foi
  repetidamente derrotado. Esse fato é apresentado por Tito Lívio em sua
  \emph{História de Roma}, livro XXXVI.}. {[}26{]} Porque os romanos
fizeram, nestes casos, aquilo que todos os príncipes sábios devem fazer:
eles devem não somente se resguardar dos escândalos presentes, mas
também dos futuros, prevenir-se destes com toda indústria, porque,
prevendo-se com antecedência, remedia-se facilmente. Contudo, esperando
que se aproximem, os medicamentos não vêm a tempo, porque a doença
tornou-se incurável. {[}27{]} E ocorre aqui aquilo que diz o
médico\footnote{Ou seja, o médico naturalista.} do tísico: que no
princípio de sua doença é fácil curar e difícil reconhecer, mas, com o
passar do tempo, não sendo reconhecida no princípio nem medicada a
tempo, torna-se fácil reconhecê-la e difícil curá-la. {[}28{]} Assim
ocorre nas coisas do estado: porque, reconhecendo com antecedência -- o
que não é um atributo senão de um homem prudente -- as doenças que
nascem ele rapidamente as cura, mas, caso ele não as reconheça, deixando
que cresçam, de modo que todos passam a reconhecê-las, não tem mais
remédio.

{[}29{]} Os romanos, todavia, vendo com antecedência os inconvenientes,
remediaram-nos sempre e jamais deixaram que progredissem para evitar uma
guerra, porque sabiam que a guerra não se evita, mas se adia à vantagem
do outro. Porém entraram em guerra na Grécia com Felipe e Antíoco para
não ter que travá-la com eles na Itália, e podiam naquele momento evitar
uma e outra, o que não quiseram. {[}30{]} Jamais lhes agradou aquilo que
está na boca de todos os sábios de nossos tempos, isto é, gozar o
benefício do tempo, mas sim os benefícios da sua \emph{virtù} e da sua
prudência: porque o tempo traz todas as coisas, e pode conduzir consigo
o bem como mal, e o mal como bem.

{[}31{]} Mas voltemos à França e examinemos se das coisas ditas ela fez
alguma. Falarei de Luiz\footnote{Luiz XII, rei francês de 1498 a 1515.},
e não de Carlos\footnote{Carlos VIII, rei francês de 1470 a 1498.}, como
daquele de quem, por ter mantido por mais tempo seu domínio na Itália,
os seus progressos foram melhor vistos: e verá como ele fez o contrário
daquelas coisas que devem ser feitas para se manter um estado com outra
conformação. {[}32{]} O rei Luiz foi introduzido na Itália pela ambição
dos venezianos, quedesejavam ganhar metade do estado da Lombardia com a
sua vinda. {[}33{]} Não quero criticar esta decisão tomada pelo rei,
porque, desejando começar a por um pé na Itália e não tendo amigos nesta
província, estando-lhe, aliás, fechadas todas as portas em função das
atitudes do rei Carlos, foi forçado a aceitar aquelas amizades que
podia, e teria sido para ele uma decisão bem tomada, se nas outras
manobras não cometesse nenhum erro. {[}34{]} Conquistada, portanto, a
Lombardia, o rei recuperou imediatamente aquela reputação que Carlos lhe
tinha tirado: Genova cedeu; os Florentinos tornaram-se amigos; o marquês
de Mântua\footnote{Francisco Gonzaga, marido de Isabel d'Este.}, o duque
de Ferrara\footnote{Hercules I d'Este.}, os Bentivogli\footnote{Giovanni
  Bentivoglio, senhor de Bologna.}, a senhora de Furlí\footnote{Catarina
  Sforza Riario, senhora de Imola e Forli.}, o senhor de
Faenza\footnote{Astorre Manfredi.}, de Pesaro\footnote{Giovanni di
  Costanzo Sforza.}, de Rimini\footnote{Pandolfo Maltesta.}, de
Camerino\footnote{Giulio Cesare da Varano.}, de Piombino\footnote{Iacopo
  degli Appiani.}, os de Lucca, de Pisa e de Siena, todos lhe foram ao
encontro para serem seus amigos. {[}35{]} E então os venezianos puderam
considerar a temeridade da decisão tomada, e, para conquistar metade da
Lombardia, fizeram do rei francês senhor de dois terços da Itália.

{[}36{]} Considere-se, agora, com quão pouca dificuldade poderia o rei
manter a sua reputação na Itália, caso ele tivesse observado as regras
acima descritas e mantivesse seguros e defendesse todos aqueles que eram
seus amigos -- os quais eram, em grande número, fracos e medrosos --
quer fossem da Igreja, quer fossem os venezianos, sempre necessitavam
estar com ele, e por meio deles poderia facilmente proteger-se daqueles
que ainda eram poderosos. {[}37{]} Mas assim que chegou a Milão, fez o
contrário, auxiliou o papa Alexandre\footnote{O papa Alexandre VI,
  nascido Rodrigo Borgia, governou a Igreja de 10 de Agosto de 1492 a 18
  de agosto de 1503. Era pai de César Borgia, conhecido como o Duque
  Valentino} para que ele ocupasse a Romanha. Não percebeu que se
enfraquecia com esta deliberação, privando-se dos amigos e daqueles que
tinham recorrido à sua proteção, e tornava grande a Igreja,
acrescentando ao poder espiritual, que lhe dava tanta autoridade, muito
poder temporal. {[}38{]} E depois de cometer um primeiro erro, foi
obrigado a prosseguir, na medida em que, para pôr fim às ambições de
Alexandre e para que este não se tornasse senhor da Toscana, foi forçado
a vir à Itália.

{[}39{]} Não bastou ao rei francês ter feito grande a Igreja e perder os
amigos, pois, por desejar o reino de Nápoles, dividiu-o com o rei de
Espanha; e, enquanto antes ele era árbitro da Itália, colocou aí um
sócio, a fim de que os ambiciosos daquela província e os descontentes
com ele tivessem a quem recorrer; e enquanto podia deixar naquele reino
um rei pensionário seu, retirou-o, para colocar um que pudesse
expulsar-lhe da Itália. {[}40{]} É verdadeiramente coisa muito natural e
ordinária desejar conquistar: e sempre quando os homens o fazem e podem
serão louvados ou não serão censurados; porém, quando eles não podem, e
desejam fazê-lo de todo modo, aqui está o erro e a censura. {[}41{]}
Portanto, se a França pudesse com as suas forças assaltar Nápoles,
deveria fazê-lo; se não podia, não deveria dividi-la; e se fez a divisão
da Lombardia com os venezianos, mereceu desculpas por, com eles, ter
colocado os pés na Itália; merece censura por não ter a desculpa daquela
necessidade\footnote{No caso, os venezianos ajudaram os franceses a
  entrar na península itálica, mas eles (os franceses) não tinham a
  necessidade de aliar-se aos venezianos e dividir a Lombardia.}.

{[}42{]} Luiz cometeu, pois, estes cinco erros: eliminou os poderosos
menores, na Itália acrescentou força a um poderoso, nela colocou um
forasteiro poderosíssimo, não foi habitá-la, não pôs nela colônias.
{[}43{]} Erros que, ainda assim, estando ele vivo, não poderiam
prejudicá-lo, se não tivesse cometido o sexto {[}erro{]}: de tirar o
estado dos venezianos. {[}44{]} Porque, se ele não tivesse aumentado o
poder da Igreja e nem posto a Espanha na Itália, era bem razoável e
necessário enfraquecer os venezianos; mas, ao tomar aquelas primeiras
decisões, não devia nunca consentir na ruína deles, porque, sendo os
venezianos poderosos, teriam sempre mantido os outros distantes da sua
façanha na Lombardia, seja porque os venezianos não lhes consentiriam
sem tornarem-se eles seus senhores da Lombardia, seja porque os outros
não desejariam retirá-la da França para dá-la aos venezianos; e não
teriam coragem de afrontar os dois.

{[}45{]} E se alguém dissesse: o rei Luiz cedeu a Alexandre a Romanha e
à Espanha o reino {[}de Nápoles{]} para evitar uma guerra, respondo, com
os argumentos ditos acima, que não se deve nunca deixar progredir uma
desordem para evitar uma guerra, porque não se a evita, mas se a adia a
sua desvantagem. {[}46{]} E se alguns outros alegassem a promessa que o
rei fez ao Papa, de executar por ele aquele feito para obter a
dissolução do seu matrimônio e a nomeação do cardeal de Ruão, respondo
com aquilo que direi acerca da promessa do príncipe e de como se deve
observá-la\footnote{Vide cap. 18.}.

{[}47{]} Por conseguinte, o rei Luiz perdeu a Lombardia por não ter
observado alguns daqueles preceitos observados por outros que
conquistaram províncias e desejaram mantê-las, não há nisso milagre
algum, mas é muito ordinário e razoável. {[}48{]} Sobre este assunto
falei em Nantes com o cardeal de Ruão, quando o Valentino como era
popularmente chamado César Borgia, filho do papa Alexandre\footnote{César
  Borgia (1475-1507), filho do Papa Alexandre VI e Vanossa Catanei,
  tornado cardeal aos 17 anos, recebeu de seu pai a administração de
  territórios da Igreja.} -- ocupava a Romanha. Porque, ao dizer-me o
cardeal de Ruão que os italianos não entendiam de guerra, respondi-lhe
que os franceses não entendiam de estado, porque, se entendessem, não
deixariam que a Igreja viesse a ter tanta grandeza. {[}49{]} E pela
experiência se viu que a grandeza dela e da Espanha, na Itália, foi
causada pela França, e que a sua ruína foi causada por eles. {[}50{]} Do
que se tira uma regra geral, a qual nunca ou raramente falha: que quem
faz alguém poderoso, causa a sua ruína, porque aquele poder é criado por
ele ou com astúcia ou com força, e uma e outra destas duas é suspeita
para quem se torna poderoso.

\quebra\section{\emph{CUR DARII REGNUM QUOD ALEXANDER OCCUPAVERAT A SUCCESSORIBUS SUIS
POST ALEXANDRI MORTEM NON DEFECIT} {[}Por quê razão o reino de Dário, que tinha sido ocupado Alexandre, não se rebelou contra os seus sucessores após a sua morte{]}}

{[}1{]} Consideradas as dificuldades que se tem em manter um estado
recém- ocupado, poderia alguém supreender-se com o fato de Alexandre
Magno\footnote{Alexandre da Macedônia ou Alexandre Magno, conquistou a
  Ásia entre 334 e 327 a.C.} tornar-se senhor da Ásia em poucos anos, e,
mal tendo-a ocupado, ter morrido: donde pareceria razoável que todos
aqueles estados se rebelassem. Todavia, os seus sucessores
mantiveram-nos e não encontraram, para mantê-los, outra dificuldade
senão aquela que surgiu entre eles mesmos, pelas suas próprias ambições.
{[}2{]} Respondo que os principados, dos quais se tem memória,
encontram-se governados por dois modos diversos: ou por um príncipe e
todos os outros lhe servem, os quais, como ministros, por sua graça e
concessão, ajudam a governar aquele reino; ou por um príncipe e por
barões, os quais possuem esta posição não pela graça do senhor, mas pela
antiguidade da dinastia. {[}3{]} Estes barões têm estados e súditos
próprios, os quais os reconhecem como senhores e são naturalmente
afeiçoados a eles. {[}4{]} Aqueles estados governados por um príncipe e
por servos têm o seu príncipe com mais autoridade, porque em toda a sua
província não há ninguém que se reconheça por superior senão ele; e se
obedecem a algum outro, o fazem como a um ministro e a um oficial, e a
ele não nutrem nenhum amor em particular.

{[}5{]} Exemplos destas duas diversidades de governos são, em nossa
época, o Turco\footnote{No caso, o governante do Império Otomano.}e o
rei de França. {[}6{]} Toda a monarquia do Turco é governada por um só
senhor: os outros são seus servos. Ele divide o seu reino em
\emph{sandjacs}\footnote{\emph{Sangiacchie} é a italianização do termo
  político turco \emph{sangiak,} que constituíam as subdivisões do
  Império Otomano. Cada uma dessas partes era governada por um
  administrador, também conhecido como Paxá.}, aos quais envia diversos
administradores, mudando-os e transferindo-os como lhe convém. {[}7{]}
Mas o rei de França se encontra no meio de uma antiga multidão de
senhores, os quais são reconhecidos em seus estados por seus próprios
súditos e amados por eles: eles têm a sua preeminência, e o rei não pode
tirá-los sem perigo para si. {[}8{]} Quem considerar, pois, um e outro
destes estados, encontrará dificuldade para conquistar o estado do
governo turco; todavia, uma vez tendo-o vencido, terá grande facilidade
para mantê-lo. {[}9{]} Ao contrário, encontrará em alguns aspectos mais
facilidade em ocupar o reino de França, porém, grande dificuldade para
mantê-lo.

{[}10{]} As razões destas dificuldades em poder ocupar o reino do Turco
estão em não poder ser convocado pelos príncipes\footnote{O plural de
  \emph{príncipe} empregada por Maquiavel, os \emph{principi,} demonstra
  um dos fundamentos desse conceito, a saber: o príncipe, como a própria
  etimologia da palavra revela, são os primeiros, aqueles que estão na
  vanguarda. Neste caso do regime turco, trata-se dos administradores
  das \emph{sandjacs}, que são lideres ou as principais figuras
  políticas naquele território. Contudo, sua condição de funcionários e
  a sua pouca ou inexistente liderança política os faz pouco confiáveis
  na conquista do governo e não são eles capazes de liderar o povo em
  uma rebelião política.} daquele reino, nem esperar, mediante a
rebelião daqueles que estão ao redor do Turco, facilidades em sua
empresa, o que tem origem nas razões sobreditas: porque, sendo todos
escravos seus e obrigados a ele, com mais dificuldade podem ser
corrompidos e, quando se corrompem, pode-se esperar pouca ajuda deles,
pois, pelas razões assinaladas, eles não podem trazer consigo o povo.
{[}11{]} Donde ser necessário para quem assalta o Império Turco pensar
que haverá de encontrá-lo unido, e convém a ele confiar mais na própria
força do que na desordem dos outros. {[}12{]} Mas, uma vez que o tenha
vencido e derrotado em campanha, de tal modo que não possa recompor os
exércitos, não tem do que recear senão da dinastia do príncipe e, uma
vez que ela esteja extinta, não resta ninguém a temer, pois os outros
não têm crédito junto ao povo; e assim como o vencedor não pode, antes
da vitória, confiar nele, assim também não deve, depois dela, temê-lo.

{[}13{]} O contrário ocorre nos reinos governados, como aquele de
França, porque você pode entrar com facilidade, ganhando para si algum
barão do reino, e porque sempre se encontra alguém descontente e aqueles
desejosos de mudanças\footnote{Um relato semelhante encontramos no texto
  maquiaveliano \emph{Retratos das coisas da França} (linhas de 4 a 6),
  escrito entre 1510 e 1511, no qual Maquiavel faz uma análise da
  dinâmica política do regime francês. Esse exemplo, como outros que se
  seguiram ao longo de \emph{O Príncipe}, revela como Maquiavel emprega
  todos os conhecimentos adquiridos nas duas missões diplomáticas a
  serviço do governo de Florença, tendo como uma de suas incumbências, a
  análise e a descrição atenta da dinâmica política dos governos
  visitados. Neste caso, a análise \emph{in loco} do regime francês
  possibilitou-lhe uma análise aguda da dinâmica política de um regime
  com divisões de poder muito acentuadas.}. {[}14{]} Esses, pelas razões
ditas, podem abrir-lhe o caminho naquele estado e lhe facilitar a
vitória, a qual, depois, se quiser conservá-la, traz consigo infinitas
dificuldades tanto com aqueles que o ajudaram, como com aqueles que
oprimiu. {[}15{]} Não é suficiente para você eliminar a dinastia do
príncipe, porque permanecerão aqueles senhores que se farão líderes de
novas sedições: e, não podendo contentar-lhes nem eliminá-los, perderá
aquele estado tão logo haja ocasião.

{[}16{]} Ora, se levarem em conta a natureza do governo de
Dario\footnote{Dario III, rei da Pérsia entre 337 a 330 a.C., derrotado
  e morto por Alexandre Magno.}, vocês o considerarão semelhante ao
reino do Turco. Para Alexandre, porém, foi necessário primeiro
combatê-lo e derrotá-lo completamente em campanha. {[}17{]} Depois de
tal vitória, estando Dario morto, permaneceu seguro aquele estado para
Alexandre pelas razões acima expostas, e os seus sucessores, se fossem
unidos, poderiam usufruí- lo com ócio, pois naquele reino não surgiram
outros tumultos senão aqueles que eles próprios suscitaram\footnote{Após
  a morte de Alexandre, seu império foi dividido entre os seus generais:
  Antígono, Antipatro, Lisimaco, Perdicca, Seleuco e Ptolomeo.}.
{[}18{]} Mas é impossível possuir com tamanha tranquilidade os estados
ordenados à semelhança daquele de França. {[}19{]} Disso se originaram
as frequentes rebeliões na Espanha, na França e na Grécia dominadas
pelos romanos, por serem muitos os principados nesses estados\footnote{``\emph{Muitos
  principados nesses estados''}, com isso Maquiavel quer destacar os
  inúmeros governos que havia nesses territórios antes da dominação
  romana e, mesmo após essa, havia nesses povos a lembrança desses
  governos próprios, o que dificultou o domínio imposto por Roma nessas
  localidades.} e, enquanto durou a memória deles, os romanos sempre
estiveram incertos daqueles domínios. {[}20{]} Mas, extinta a memória
desses senhores, com a força e a duração do Império, os romanos ficaram
seguros em seu domínio. Combatendo depois entre eles\footnote{No caso,
  as guerras civis entre os próprios romanos (Mario e Sila, César e
  Pompeu, Otaviano e Antônio), que enfraqueceram o controle de Roma
  sobre esses territórios.}, cada um daqueles principados pode também
retomar parte daquelas províncias, conforme a autoridade que haviam
obtido ali, e, aquelas, por terem extinta a dinastia dos seus antigos
senhores, não reconheciam senão os romanos como senhores. {[}21{]}
Considerando, portanto, todas estas coisas, ninguém se surpreenderá com
a facilidade com que Alexandre obteve o estado da Ásia e da dificuldade
que outros tiveram para conservar o conquistado, como Pirro\footnote{Rei
  do Épiro (atual Albânia) de 307 a 303 e de 297 a 272 a.C. Conquistou a
  Macedônia, mas, em

  função de uma insurreição, perdeu-a em poucos meses} e muitos outros,
o que não tem origem na muita ou pouca \emph{virtù} do vencedor, mas na
diversidade das matérias\footnote{O termo italiano \emph{subietto} é
  tradução do latim \emph{subiectum}, que por sua vez, é a tradução do
  termo grego \emph{hypokeímenon}, que é empregado por Aristóteles como
  o substrato do ser. Tal substrato pode ser entendido de vários modos
  (por exemplo, \emph{Metafísica}, Z, III), mas a interpretação mais
  adotada pelos leitores dos textos aristotélicos foi a de associar o
  \emph{hypokeímenon} à matéria, tornado-se o substrato sobre o qual se
  agregam os acidentes. Assim, quando Maquiavel emprega o termo
  \emph{subietto}, como neste caso, está pensando nesse substrato
  material, o povo, que é o fundamento material da cidade, conservando,
  portanto, sua significação grega original. Cf. também Aristóteles,
  \emph{Política}, III, cap. 1, 1274b30-1276a5.}.

\quebra\section{\emph{QUOMODO ADMINISTRANDAE SUNT CIVITATES VEL PRINCIPATUS, QUI,
ANTEQUAM OCCUPARENTUR SUIS LEGIBUS VIVEBANT} {[}De que modo se deve administrar as cidades ou principados que, antes de serem ocupados, viviam segundo as suas próprias leis{]}}

{[}1{]} Quando aqueles estados que se conquistam, como foi dito, estão
habituados a viver segundo as suas próprias leis e em liberdade, para
querer mantê-los são três os modos: {[}2{]} o primeiro, arruiná-los; o
outro, ir habitá-los pessoalmente; o terceiro, deixá- los viver segundo
as suas próprias leis, cobrando um tributo e criando dentro deles um
estado de poucos\footnote{Estado de poucos é uma outra forma de designar
  o regime aristocrático.}, que os conserve seus amigos. {[}3{]} Porque,
sendo esse governo {[}aristocrático{]} criado por aquele príncipe, sabe
que não pode ficar sem a sua amizade e a sua força e há de fazer tudo
para conservá-lo. E, desejando preservá-la, mais facilmente se mantém
uma cidade acostumada a viver livremente por meio de seus
cidadãos\footnote{``\emph{Uma cidade acostumada a viver livremente por
  meio de seus cidadãos}'' é uma expressão típica para caracterizar um
  regime republicano, ou seja, uma cidade que não é dominada ou
  comandada por um outro poder político, e na qual a condução do governo
  está nas mãos dos cidadãos. Convém destacar o uso do vocábulo
  \emph{ciptadino} (cidadãos), o que revela o estatuto político dos
  habitantes, por oposição aos \emph{súditos} de um reino. Como é
  evidente, a liberdade política é um atributo fundamental da cidadania
  para Maquiavel.}, do que de qualquer outro modo.

{[}4{]} Como exemplo temos os espartanos e os romanos. Os espartanos, ao
ocuparem Atenas e Tebas, criaram nelas um estado de poucos\footnote{Como
  já mencionado, trata-se de uma aristocracia (regime virtuoso) ou
  oligarquia (regime desviado), conforme a definição clássica.}, contudo
voltaram a perdê-las. {[}5{]} Os romanos, para manter Cápua, Cartago e
Numância, destruíram-nas e não as perderam; quiseram manter a Grécia,
assim como a mantiveram os espartanos, tornando-a livre e deixaram-na
com suas leis, e não obtiveram sucesso, de modo que, para mantê-la,
foram obrigados a destruir muitas cidades daquela província. {[}6{]}
Porque, em verdade, não havia um modo seguro para apossar-se dela, senão
arruinando-a; e quem se torna senhor de uma cidade acostumada a viver
livremente e não a destrói, pode esperar ser destruído por ela, pois a
cidade sempre encontra guarida na rebelião, em nome da
liberdade\footnote{Aqui ``nome'' com sentido de valor ou ideal, que
  jamais abandona os povos que experimentaram a liberdade política.} e
nas suas antigas ordenações, as quais jamais são esquecidas, nem pela
duração do tempo, nem pelos benefícios realizados. {[}7{]} E qualquer
coisa que se faça ou se proveja, se não se dividirem ou se dispersarem
os habitantes, e se não se esquecerem daquele nome nem daquelas
ordenações, logo, em qualquer imprevisto, acontecerá como fez Pisa cem
anos depois de ter sido posta sob o domínio dos florentinos\footnote{A
  cidade de Pisa foi ocupada pelos florentinos em 1405 e se rebelou em
  1494, com a chegada na Itália de Carlos VIII, rei francês. Foi
  novamente ocupada em 1509 pelos florentinos.}. {[}8{]} Mas, quando as
cidades ou as províncias estão habituadas a viver sob um príncipe e
extingue-se essa dinastia -- estando de um lado habituadas a obedecer e
de outro não tendo o velho príncipe -- não entram em acordo em si para
ter um outro príncipe, nem sabem viver livremente, de modo que demoram
mais para pegar em armas e com mais facilidade um apríncipe pode
conquistá-las e estar seguro com eles.

{[}9{]} Mas, nas repúblicas, há mais vida, mais ódio, mais desejo de
vingança; nem os deixa\footnote{No caso, os cidadãos, que não abandonam
  a memória da liberdade.}, nem pode deixar-se aquietar a memória da
antiga liberdade, de modo que o caminho mais seguro é extingui-las ou
habitá-las\footnote{O período final sintetiza a idéia que percorre e
  comanda o capítulo: a liberdade, característica própria das
  repúblicas, é o valor maior de uma cidade. Logo, no ato da conquista,
  o príncipe tem que saber lidar com esse dado, pois a memória da
  liberdade política dos cidadãos sempre será um empecilho para o seu
  governo. Donde a necessidade de se tomar medidas drásticas, como
  destruir todas as antigas instituições, para que o conquistador
  consiga impor o seu domínio sobre aquele povo. De outro lado, para uma
  república, acostumada com a liberdade, isso implica também lutas e
  conflitos, pois esse é um dado inerente da vida política: o conflito
  político permanente. Tal noção de conflito político, que também é
  expressão da liberdade política, se apresentará melhor nos capítulos
  VIII e IX de \emph{O Príncipe}, mas com mais força e vigor nos
  \emph{Discursos sobre a primeira década de Tito Lívio},
  particularmente nos capítulos III e IV do livro I.}.

\quebra\section{\emph{DE PRINCIPATIBUS NOVIS QUI ARMIS PROPRIIS ET VIRTUTE ACQUIRUNTUR}
{[}Dos principados novos que se conquistam com armas próprias e virtude{]}}

{[}1{]} Ninguém se surpreenda se, na exposição que farei dos principados
completamente novos, dos príncipes e dos estados, eu apresentar
grandíssimos exemplos. {[}2{]} Porque, caminhando os homens sempre pelos
caminhos percorridos por outros e procedendo por imitação nas suas
ações, não podendo em tudo seguir nos caminhos alheios, nem adquirir a
\emph{virtù} daqueles que você imita, deve um homem prudente seguir
sempre pelas estradas percorridas por grandes homens, e imitar aqueles
que foram excelentíssimos, a fim de que, se a sua \emph{virtù} não os
alcançar, ao menos receba deles algum aroma; {[}3{]} e fazer como os
arqueiros prudentes, os quais, parecendo muito distante o lugar que
desejam alvejar e conhecendo bem até que ponto vai a \emph{virtù} do seu
arco, põem a mira muito mais alta que o lugar mirado, não para atingir
tão alto com a sua flecha, mas para poder, com a ajuda de sua mira alta,
alcançar o alvo desejado.

{[}4{]} Digo, pois, que nos principados inteiramente novos, em que haja
um príncipe novo, encontra-se menor ou maior dificuldade para
conservá-los, segundo seja mais ou menos virtuoso\footnote{Eis aqui um
  bom exemplo da distinção operada por Maquiavel entre a sua noção de
  \emph{virtù} e a noção de virtude. Neste caso Maquiavel utiliza
  ``virtuoso'' quando poderia muito bem utilizar \emph{virtù}, contudo,
  em não o fazendo, pretende ressaltar justamente a diferença entre ter
  virtudes e ter \emph{virtú}. Aquele que conquista até pode ter alguma
  ``virtude'', ser virtuoso, mas isso não é o mesmo que ter
  \emph{virtù}, necessária à conquista política.} aquele que os
conquista. {[}5{]} E porque este evento, de passar de cidadão
comum\footnote{O termo \emph{privato,} utilizado aqui por Maquiavel, é
  de difícil tradução para o nosso contexto discursivo, pois
  literalmente \emph{privato} deve ser traduzido por \emph{privado} em
  português. No capítulo IX, ele usará a expressão \emph{privato
  ciptadino} (1) e \emph{ciptadino privato} (20), que poderia ser
  traduzida literalmente por \emph{cidadão particular}. Entretanto, o
  autor está se referindo, aqui, bem como nas demais ocorrências a
  seguir {[}capítulos VI (27), VII (1, 2, 6), VIII (1, 4), IX (1, 20),
  XI (15), XIV (3){]}, ao cidadão comum, não pertencente à família do
  governante, que se torna príncipe de uma cidade. Esse personagem
  político, mais do que os outros, é o caso típico do estatuto político
  de \emph{príncipe} defendido ao longo de toda obra: alguém que se
  torna o líder, o primeiro cidadão que conduz os demais politicamente.
  Tendo em vista essa interpretação, defendida na \emph{Introdução}
  desta edição, entendemos que a melhor tradução para esse termo
  \emph{privato}, bem como para \emph{ciptadino privato}, seja mais
  adequado o termo cidadão comum, que expressa esse personagem político
  de proa que assume o comando da cidade calcado na sua \emph{virtù}.} a
príncipe, pressupõe ou \emph{virtù} ou fortuna, parece que uma ou outra
destas duas coisas mitiga, em parte, muitas dificuldades. Todavia,
aquele que menos se apoiou na fortuna, manteve-se mais. {[}6{]} Gera
ainda mais facilidade o príncipe ser obrigado, por não ter outro estado,
a vir habitá-lo pessoalmente.

{[}7{]} Mas, tratando daqueles que, pela própria \emph{virtù} e não pela
fortuna, tornaram- se príncipes, digo que os mais excelentes são Moisés,
Ciro, Rômulo, Teseu\footnote{Moisés: profeta bíblico, fundador da Israel
  do Antigo Testamento, provavelmente no séc. XIII a.C.; Ciro: fundador
  do Império Persa no séc. VI a.C.; Teseo: unificador da península
  Ática, provavelmente no séc. XII a.C.; Rômulo: fundador de Roma,
  séc.VIII a.C.} e outros semelhantes. {[}8{]} E ainda que não se deva
discutir sobre Moisés, tendo sido ele um mero executor das coisas que
lhe eram ordenadas por Deus, todavia deve ser admirado tão somente por
aquela graça que o tornava digno de falar com Deus. {[}9{]} Mas,
considerando Ciro e os outros que conquistaram ou fundaram reinos,
achá-los-eis todos admiráveis; e se considerarmos as suas ações e os
seus ordenamentos particulares, não pareceram discrepantes em relação
aos de Moisés, que teve tão grande preceptor\footnote{No caso, Maquiavel
  está se referindo ao próprio Deus, que, conforme a narração bíblica,
  ordenou a Moisés conduzir os hebreus à ``terra prometida'' e fundar um
  novo reino.}. {[}10{]} E examinando as suas ações e as suas vidas,
vê-se que não tiveram nada da fortuna senão a ocasião, a qual deu-lhes a
matéria para que uma forma semelhante possa ser introduzida dentro
dela\footnote{Expressão típica da metafísica de origem aristotélica, mas
  de uso corrente no Renascimento italiano, no qual se concebe que a
  forma é inserida na matéria correspondente para formar uma substância,
  um algo. Aqui, a ocasião, o momento, correspondem à matéria e a
  \emph{virtù} à forma, donde o príncipe novo tomar a ocasião adequada
  para inserir, com sua \emph{virtù,} a ordenação política na cidade, o
  que resulta no regime político ou governo.}; e, sem essa ocasião, a
\emph{virtù} do seu ânimo teria sido extinta, e sem essa \emph{virtù} a
ocasião teria vindo em vão.

{[}11{]} Era, portanto, necessário a Moisés encontrar o povo de Israel
no Egito, escravizado e oprimido pelos egípcios, a fim de que ele (o
povo de Israel), para sair da servidão, se dispusesse a segui-lo.
{[}12{]} Convinha que Rômulo não pudesse permanecer em Alba e fosse
abandonado ao nascer, para querer tornar-se rei de Roma e fundador
daquela pátria. {[}13{]} Era necessário que Ciro encontrasse os persas
descontentes com o Império dos medos, e os medos amolecidos e efeminados
pela longa paz. {[}14{]} Não poderia Teseu demonstrar a sua
\emph{virtù}, se não encontrasse os atenienses dispersos. {[}15{]} Estas
ocasiões, portanto, fizeram estes homens felizes e a sua excelente
\emph{virtù} fez com que aquela ocasião se tornasse conhecida, donde a
sua pátria foi dignificada e tornou-se felicíssima.

{[}16{]} Aqueles os quais, por caminho virtuoso, à semelhança destes,
tornaram-se príncipes, conquistam o principado com dificuldade, mas com
facilidade o mantêm; e as dificuldades que eles têm ao conquistar o
principado, nascem, em parte, dos novos ordenamentos e costumes que são
forçados a introduzir para fundar o seu estado e a sua segurança.
{[}17{]} E deve-se considerar que não há coisa mais difícil de tratar,
nem mais duvidosa em obter, nem mais perigosa em manejar, do que
fazer-se chefe para introduzir novos ordenamentos. {[}18{]} Porque o
introdutor tem por inimigos todos aqueles que se beneficiaram dos velhos
ordenamentos, e tem por defensores tíbios todos aqueles que dos novos
ordenamentos se beneficiarão. Tal tibieza nasce em parte por medo dos
adversários, que têm as leis do seu lado, em parte da incredulidade dos
homens, os quais não acreditam na verdade das coisas novas, senão quando
vêem nascida de uma firme experiência. {[}19{]} Donde nasce que algumas
vezes aqueles que são inimigos tem ocasião para assaltar, o fazem
apaixonadamente, e aqueles outros defendem tibiamente, de modo que
juntos com esses se corre perigo.

{[}20{]} É necessário, portanto, querendo discorrer bem sobre esta
parte, examinar se estas inovações se sustentam por si mesmas ou se
dependem de outros, isto é, se para conduzir a sua obra, precisa rezar
ou pode forçar. {[}21{]} No primeiro caso, sempre entendem mal e não
leva a coisa alguma, mas, quando dependem de si próprios e podem forçar,
então é que raras vezes correm perigo. Daqui nasce que todos os profetas
armados venceram e os desarmados se arruinaram. {[}22{]} Porque, além
das outras coisas ditas, a natureza dos povos é variada e é fácil
persuadi-los em uma coisa, mas é difícil sustentá-los nesta persuasão.
Porém, convém ser ordenado de modo que, quando não crêem mais, pode-se
fazer crerem pela força. {[}23{]} Moisés, Ciro, Teseu e Rômulo não
teriam podido fazer observar sua constituição longamente caso estivessem
desarmados, como no nosso tempo sucedeu com o frei Jerônimo
Savonarola\footnote{Jerônimo Savonarola (1452-1498) foi um frei
  dominicano, que ganhou notoriedade em Florença, onde foi enviado para
  pregar em 1475. Ao longo do ano de 1494, seus sermões contribuem para
  a queda do governo da família Médici e ele assume a liderança do novo
  governo em dezembro deste ano, constituindo um governo republicano de
  caráter popular. Seu governo entra no crise em início de 1498, sendo
  ele deposto, julgado pela Inquisição católica e condenado a morrer na
  fogueira neste mesmo ano. Apesar de liderar um governo republicano, o
  regime de Savonarola e seus partidários era adversário do grupo
  político ao qual Maquiavel pertencia, grupo político esse que assume o
  governo em 1498, sob o comando de Pier Soderini e que perdura até
  1512.}, o qual arruinou os seus novos ordenamentos, quando a multidão
começou a não acreditar nele, e ele não tinha como manter firmes aqueles
que haviam acreditado nele, nem fazer crer os descrentes. {[}24{]}
Porém, estes tem grande dificuldade no conduzir, e todos os seus perigos
estão no seu caminho, e convém que os superem com a \emph{virtù}.
{[}25{]} Mas, uma vez superadas essas adversidades, começam a ser
venerados, tendo perdido aquela sua qualidade que lhe tinham invejado,
permanecendo fortes, seguros, honrados e felizes.

{[}26{]} A estes altos exemplos, desejo acrescentar um exemplo menor,
mas terá alguma similitude com aqueles, e desejo que me seja suficiente
para todos os outros similares: e este é Hierão de Siracusa\footnote{Puxar
  a nota do Hierão}. {[}27{]} Este, de cidadão, tornou-se príncipe de
Siracusa; nem lhe conhecem outra ocasião senão a fortuna, porque,
estando os siracusanos oprimidos, elegeram-no para seu capitão, donde
mereceu ser feito príncipe deles. {[}28{]} E foi de tanta \emph{virtù},
ainda que privado pela fortuna, que quem nos escreve sobre ele diz:
\emph{quod nihil illi deerat ad reganandum praeter regnum}\footnote{``que
  nada lhe faltava para reinar exceto um reino''. Citação de memória do
  texto de Justiniano, XXIII, 4. O texto exato é: \emph{ut nihil ei
  regium deesse, praeter regnum videretur}.}. {[}29{]} Este extinguiu a
velha milícia, ordenou uma nova; deixou as antigas amizades, fazendo
novas; e como teve amizades e soldados, pode sobre tais fundamentos
edificar todo o edifício, tanto que lhe deu muito cansaço a conquista e
pouco a conservação.

\quebra\section{\emph{DE PRINCIPATIBUS NOVIS QUI ALIENIS ARMIS ET FORTUNA ACQUIRUNTUR}
{[}Dos principados novos que são conquistados com armas e fortuna alheias{]}}

{[}1{]} Aqueles que somente pela fortuna de cidadão se tornam príncipes,
com pouco esforço conseguem sê-lo e com muito se mantêm. E não têm
nenhuma dificuldade neste caminho, porque voam\footnote{Muito
  ilustrativo e importante aqui o emprego do termo \emph{volare} (voar),
  pois é esta a trajetória daqueles que chegam à condição de príncipe
  apoiados na fortuna. Eles, em função da fortuna, saltam ou ``passam
  por cima'' literalmente dos problemas inerentes à conquista. Enquanto
  \emph{virtù} esta pressupõe esforço e labuta, o recurso à fortuna
  configura-se como um salto dessas etapas, mas que geram, por
  consequência, maiores problemas na conservação, como destacará a
  sequência da exposição.} para esta condição, mas todas as dificuldades
surgem quando a ela chegam. {[}2{]} Estes são aqueles aos quais é
concedido um estado ou por dinheiro ou pela graça de quem o concede,
como sucedeu a muitos na Grécia, nas cidades da Jônia e do
Helesponto\footnote{Cidades gregas da antiga Ásia menor. Helesponto é o
  antigo nome da região dos Dardanelos, atual

  Turquia.}, onde foram feitos príncipes por Dario\footnote{Segundo
  Heródoto, o rei persa Dario I, após conquistar parte da península
  grega, foi derrotado na batalha de Maratona (490 a.C.).}, para que as
conservassem para a sua segurança e glória; como foram feitos ainda
aqueles imperadores que, de cidadãos comuns -- mediante a corrupção dos
soldados --, tomaram o Império.

{[}3{]} Estes estão fundados unicamente na vontade e na fortuna de quem
lhes concedeu tal status, duas muito volúveis e instáveis, e não sabem e
não podem se manter naquele posto: não sabem, porque não sendo\footnote{Como
  destaca Inglese, aqui se encontra um exemplo de uma brusca passagem
  explicativa do plural para o singular.} um homem de grande engenho e
\emph{virtù}, não é razoável que, sempre vivendo como homens de condição
particular\footnote{O texto refere-se aqui, aos homens que vivem em
  \emph{privata fortuna} (fortuna particular), que diz respeito a pouca
  ou nenhuma inserção política ou pública desses indivíduos. No caso
  ``\emph{privata}'' se apresenta como oposto ao ``público''. A
  \emph{fortuna} é aqui um qualificativo desta condição não pública, ou
  seja, reforça o caráter voltado para os interesses particulares desses
  indivíduos. Enfim, mais do que ressaltar o aspecto econômico,
  Maquiavel enfatiza a condição desses homens, que, ao não terem
  experiências de comando político (de vivência pública), não sabem
  comandar, contam apenas com a fortuna para conquistar a condição de
  príncipe. Importa destacar neste argumento a utilização do termo
  \emph{comandar}, uma das incumbências daquele que alcança a condição
  de príncipe.}, saibam comandar; não podem, porque não têm forças que
lhes possam ser amigas e fiéis. {[}4{]} Disso decorre que, os estados
que surgem subitamente, assim como todas as outras coisas da natureza
que nascem e crescem depressa, não podem ter raízes e
ramificações\footnote{\emph{Barbe e correspondenzie} são as raízes e as
  demais partes da planta que estão ligadas à terra, no caso, as demais
  ramificações. Conforme Inglese e Martelli, o termo
  \emph{correspondenzie} é de compreensão exata, pois poderia remeter
  ainda ao tronco, às folhas, ao caule etc. O termo ramificações aqui, a
  nosso ver, expressa melhor a alegoria descrita por Maquiavel.}, de
modo que o primeiro período de adversidade os extingue; exceto no caso
daqueles que, como foi dito, de repente se tornaram príncipes e tenham
tanta \emph{virtù} que saibam rapidamente organizar e conservar aquilo
que a fortuna lhes colocou no colo, e aqueles fundamentos, que os outros
construíram antes de se tornarem príncipes, eles construam depois.

{[}5{]} Desejo aduzir a um e a outro destes modos, de se tornar
príncipe, pela \emph{virtù} ou pela fortuna, com dois exemplos presentes
em nossa memória: Francisco Sforza\footnote{Francisco Sforza
  (1401-1466), conhecido pela sua grande habilidade militar. Casou-se
  com Bianca

  Maria, filha do Duque de Milão (Filippo Maria Visconti), herdando o
  governo da cidade depois da morte do duque em 1447 e dos conflitos
  travados contra os partidários da República Ambrosina. Assumiu
  definitivamente o governo em 1448.} e César Bórgia. {[}6{]} Francisco,
com os meios adequados e com a sua grande \emph{virtù}, passou de
cidadão a duque de Milão, e aquilo que com mil percalços havia
conquistado, com pouco esforço conservou. {[}7{]} Por outro lado, César
Bórgia, chamado pelo povo de duque Valentino, conquistou o governo com a
fortuna do pai\footnote{Como já mencionado (nota 46) o pai de César era
  o papa Alexandre VI.} e com a mesma o perdeu, apesar de ter ele usado
de todos os recursos e ter feito todas aquelas coisas que um homem
prudente e virtuoso deveria fazer para deitar suas raízes naqueles
governos, que as armas e a fortuna de outros lhe haviam concedido.
{[}8{]} Porque, como se disse anteriormente, aquele que não constrói
primeiro os fundamentos, poderia, com uma grande \emph{virtù},
construí-los depois, ainda que se façam com incômodo para o arquiteto e
perigo para o edifício. {[}9{]} Se, então, considerarmos todos os
progressos do duque, veremos que ele construiu grandes fundamentos para
um poder futuro, sobre os quais não julgo supérfluo discorrer, porque
não saberia quais preceitos melhores dar a um príncipe novo, senão o
exemplo de suas ações; e se seus modos de proceder não lhe forem
proveitosos, não será por culpa sua, porque nasce de uma extraordinária
e extrema malignidade\footnote{A expressão ``extraordinária e extrema
  malignidade'' visa ressaltar os excessos da fortuna.} da fortuna.

{[}10{]} Alexandre VI, no desejo de fazer grande o duque, seu filho,
tinha muitas dificuldades presentes e futuras. {[}11{]} Primeiro, ele
não via meios de poder fazer dele senhor de algum estado que não fosse
um estado da Igreja; e, apressando-se em tomar algum estado da Igreja,
sabia que o duque de Milão\footnote{Ludovico Sforza (1452-1508), último
  filho de Francesco Visconti e Bianca Maria, assumiu o título ducal e o
  governo da cidade em 1494.} e os venezianos não lhe consentiriam,
porque Faenza e Rimini já estavam sob a proteção dos venezianos.
{[}12{]} Via, além disso, que as armas da Itália, e em particular
aquelas das quais seria possível se servir, estavam nas mãos daqueles
que deviam temer a grandeza do papa, e por isso não poderia confiar
nelas -- sendo todas elas dos Orsini e dos Colonna\footnote{Os Orsini e
  os Colonna eram as duas famílias mais poderosas de Roma no início do
  século XVI, de

  modo que toda a eleição papal sofria forte influência dos interesses
  delas} e dos cúmplices destes. {[}13{]} Era, portanto, necessário
romper aqueles equilíbrios e desordenar os seus estados da Itália, para
poder assenhorear-se seguramente de parte deles. {[}14{]} O que lhe foi
fácil, porque encontrou os venezianos, que tinham decidido deixar os
franceses reentrar na Itália, rebelados por outros motivos\footnote{Para
  maiores informações sobre a disputa com os milaneses, confirir cap.
  III, 32.}, o que não somente não foi dificultado, mas tornou-se mais
fácil com a dissolução do antigo matrimônio do rei Luiz\footnote{Luiz
  queria se divorciar de sua primeira mulher, Joana, irmã de Carlos
  VIII, para casar com a viúva deste, Ana da Bretanha. O papa Alexandre
  VI concedeu o divórcio em outubro de 1498.}.

{[}15{]} O rei invadiu, pois, a Itália com a ajuda dos venezianos e o
consentimento de Alexandre; não foi primeiro em Milão que o papa obteve
dele gente para a sua empresa na Romanha\footnote{Esse trecho retoma
  aquilo que havia sido dito no cap. III (37): ``mas assim que chegou a
  Milão, fez o contrário, auxiliou o papa Alexandre para que ele
  ocupasse a Romanha. Não percebeu que se enfraquecia com esta
  deliberação, privando-se dos amigos e daqueles que tinham recorrido à
  sua proteção, e tornava grande a Igreja, acrescentando ao poder
  espiritual, que lhe dava tanta autoridade, muito poder temporal''.},
apoio que lhe foi concedido pela reputação do rei. {[}16{]} Conquistada
então a Romanha\footnote{Romanha é uma região ou província da Itália do
  centro-norte, que teve várias cidades conquistadas pelas forças papais
  neste período: Imola (27 de novembro de 1499), Forlì (19 de dezembro
  de 1499), Cesena (2 de agosto de 1500), Rimini (10 de outubro de
  1500), Pesaro (21 de outubro 1500), Faenza (25 de abril de 1501).}
pelo duque\footnote{César Borgia, o duque Valentino.} e batidos os
Colonna, desejando o papa mantê-la e prosseguir adiante, duas coisas o
impediam: uma, as suas armas, que não lhe pareciam fiéis; a outra, a
vontade da França. Caso as armas Orsini\footnote{As milícias comandadas
  por Paolo Orsini.}, das quais se tinha valido, viessem a lhe faltar,
isso não somente o impediria na conquista, mas lhe tiraria o que tinha
conquistado; e, ainda, que o rei não lhe fizesse algo semelhante.
{[}17{]} Dos Orsini teve a confirmação quando, depois da tomada de
Faenza, assaltou Bolonha, porque os viu frívolos neste assalto; acerca
do rei, conheceu o seu ânimo quando, conquistado o ducado de Urbino,
assaltou a Toscana: empresa da qual o rei fê-lo desistir\footnote{Verifica-se
  nas linhas 17 e 18 uma construção em paralelo das duas coisas que
  obstruiam o intento papal, a saber: a) a infidelidade dos exércitos
  que combatiam pelo papado, b) a falta de apoio do rei. Em seguida,
  para \emph{a}, corresponde o pouco empenho dos exércitos comandados
  por Paolo Orsini e, para \emph{b}, o não apoio do rei a sua gestão
  para que o papa não atacasse a região da Toscana.}.

{[}18{]} Assim que o duque deliberou não depender mais das armas e da
fortuna dos outros, a primeira iniciativa que tomou enfraqueceu os
partidários dos Orsini e dos Colonna em Roma, porque ganhou para si
todos os gentis-homens\footnote{Expressão equivalente a aristocratas,
  nobres.} que os apoiavam, tornando-os gentis-homens dele, dando-lhes
grandes pagamentos e honrando-os\footnote{Honrar significa aqui designar
  para um cargo público. Desde a época da Roma antiga, é uma honraria
  ser nomeado para um cargo público.}, segundo a sua qualidade, com
cargos militares e de governo\footnote{\emph{Condotte} remete ao comando
  das forças militares das cidades.}, de modo que, em poucos meses,
perdeu-se, nos seus ânimos, a afeição ao partido, e toda ela se voltou
ao duque. {[}19{]} Depois disto, esperou a ocasião para eliminar os
chefes dos Orsini, tendo já dispersado os da casa dos Colonna: o que lhe
foi conveniente e ele a usou melhor. {[}20{]} Porque os Orsini, tendo
percebido tardiamente que a grandeza do duque e da Igreja era a sua
ruína, fizeram uma reunião\footnote{Reunião aqui traduz \emph{dieta},
  termo que designa a reunião dos príncipes alemães para tomar alguma
  deliberação sobre o Estado. O termo é empregado aqui por Maquiavel com
  a mesma conotação, importada provavelmente por ocasião de sua visita
  às terras alemãs quando funcionário da Chancelaria.} em Magione, na
Perugia; dessa nasceu a rebelião de Urbino e os tumultos da Romanha e os
infinitos perigos para o duque, aos quais superou a todos eles com a
ajuda dos franceses. {[}21{]} E, tendo-lhe retornado\footnote{Este é um
  particípio que não concorda em gênero com o substantivo
  (\emph{reputazione}).} a reputação, não confiando nem na França nem em
outras forças externas, para não ter que experimentá- las contra si,
recorreu às artimanhas; e soube tanto dissimular a sua intenção, que os
Orsini, por intermédio do senhor Paulo, reconciliaram-se com ele -- de
modo que o duque não deixou de se valer de todos os meios de ofício para
assegurar-se disso, dando-lhes dinheiro, roupas e cavalos --, tanto que
a simplicidade\footnote{Mais do que simplicidade, a \emph{simplicità}
  aqui mencionada é uma falta de perspicácia dos partidários de Paolo
  Orsini, que foram ingênuos em aderir a alguém que haviam traído antes.}
deles conduziu Sinigalia às suas mãos.

{[}22{]} Extintos, portanto, esses chefes e convertidos os partidários
deles em seus amigos, tinha o duque lançado muitos bons fundamentos para
o seu poder, tendo toda a Romanha com o ducado de Urbino, parecendo-lhe,
sobretudo, que tivesse conquistado a amizade da Romanha e ganho todos
aqueles povos, porque começavam a provar o seu bem-estar. {[}23{]} E
porque esta parte é digna de nota e de ser imitada por outros, não quero
deixá-la para trás. {[}24{]} Tendo o duque tomado a Romanha e
encontrando-a comandada por senhores impotentes -- os quais tinham mais
rapidamente espoliado os seus súditos do que os corrigido, dando-lhes
motivo para a desunião e não para a união --, tanto que aquela província
estava cheia de latrocínios, de brigas e de todas as outras causas de
insolência, julgou ser necessário, por querer torná-la pacífica e
obediente ao braço régio, dar-lhe bom governo. Porém, designou
Senhor\footnote{Maquiavel utiliza o termo \emph{messere} que era
  utilizado como pronome de tratamento para os detentores de título
  honorífico dado aos juristas ou todo aquele que trabalhavam nos
  negócios jurídicos, tal como utilizado o título de doutor no Brasil
  para os profissionais que trabalham no mundo jurídico (advogados,
  delegados, juízes e promotores). Contudo, a origem do termo
  \emph{messere} é a contração de \emph{mio sire}, sendo \emph{sire} um
  termo antigo equivalente a \emph{signore}, donde \emph{messere}
  significar \emph{mio signore} (meu senhor ou monsenhor). Optamos aqui
  pela utilização de Senhor ao invés de doutor e monsenhor para traduzir
  \emph{messere}.} Remirro de Orco\footnote{Ramiro de Lorqua, mordomo de
  César Borgia, tornou-se governador da Romanha em 1501.}, homem cruel e
diligente, ao qual deu pleníssimos poderes. {[}25{]} Este em pouco tempo
a tornou pacífica e unida, com grandíssima reputação. {[}26{]} Depois
julgou o duque não ser necessária tão excessiva autoridade, porque
receava torná-la odiosa, e instituiu um tribunal civil no centro da
província, com um presidente excelentíssimo, no qual toda cidade tinha
seu advogado. {[}27{]} E porque sabia que a severidade do passado havia
gerado neles algum ódio, queria, para purgar os ânimos daqueles povos e
ganhá-los totalmente, mostrar que, se alguma crueldade havia sido
cometida, não nasceu dele, contudo da natureza cruel do ministro.
{[}28{]} E aproveitou esta ocasião para colocá-lo numa manhã na praça,
em Cesena, dividido em duas partes: com um pau e uma faca ensanguentada
do lado. A ferocidade daquele espetáculo fez aquele povo ficar ao mesmo
tempo satisfeito e estupefato.

{[}29{]} Mas retornemos ao ponto de onde partimos. Digo que,
encontrando-se o duque muito forte e em parte assegurado contra os
perigos presentes, por ter-se armado a seu modo e ter, em boa parte,
eliminado aquelas armas que, próximas, poderiam prejudicá-lo, querendo
prosseguir com a conquista, restava-lhe o respeito do rei de França,
porque sabia o rei, o qual tardiamente tinha se dado conta dos seus
erros, que não continuaria a suportá-lo. {[}30{]} Por isso, começou a
procurar novas amizades e a falhar com a França, na incursão que os
franceses fizeram ao reino de Nápoles contra os espanhóis que assediavam
Gaeta\footnote{Os franceses assediaram Gaeta, cidade napolitana, em
  junho de 1503 e deixaram-na em 1º de janeiro de 1504.}; sua intenção
era assegurar-se contra eles, o que ele teria conseguido rapidamente se
Alexandre estivesse vivo\footnote{O papa Alexandre VI morre em 18 de
  agosto de 1503, enquanto durava o assédio à cidade de Gaeta. Com a
  convocação do Conclave (reunião dos cardeais para a escolha de um novo
  papa) antes da consolidação de suas conquistas, César Borgia não teve
  as condições adequadas para manter o seu poder.}. {[}31{]} Estas foram
as suas ações quanto às coisas presentes.

{[}32{]} Mas quanto às coisas futuras, ele tinha de desconfiar,
primeiro, que um novo sucessor da Igreja não lhe fosse amigo e
procurasse tirar-lhe aquilo que Alexandre tinha lhe dado. {[}33{]}
Contra o que pensou em assegurar-se de quatro modos: primeiro, eliminar
toda a linhagem daqueles senhores que ele tinha espoliado, para retirar
do papa a ocasião de restituir-lhes os domínios; segundo, ganhar para si
todos os gentis- homens\footnote{} de Roma, como foi dito, para poder
com eles ter o papa sob controle; terceiro, tornar o Colégio
Cardinalício\footnote{Isto é, o Colégio dos Cardeais ou Colégio,
  responsável pela eleição do papa.} tão seu quanto possível; quarto,
antes que o papa morresse, conquistar tanta força, que pudesse, por si
mesmo, resistir a um primeiro ataque do novo papa. {[}34{]} Destas
quatro coisas, à época da morte de Alexandre, ele já havia conseguido
três, a quarta quase conseguira, porque dos senhores espoliados matou
tantos quanto pode alcançar e pouquíssimos se salvaram; e ganhou os
gentis-homens romanos; e no Colégio tinha o apoio de uma grandíssima
facção. Quanto às novas conquistas, havia planejado tornar-se senhor da
Toscana e já possuía Perugia e Piombino, e de Pisa tinha conquistado o
apoio. {[}35{]} E como não tivesse mais de ter cuidados com a França --
que não lhe preocupava mais, pois os franceses já tinham sido espoliados
do reino de Nápoles pelos espanhóis, de sorte que cada um deles tinha a
necessidade de comprar sua amizade -- ele teria avançado sobre Pisa.
{[}36{]} Depois disto, Lucca e Siena cederiam seu apoio a ele
rapidamente, em parte pela inveja dos florentinos, em parte por medo; e
os florentinos não teriam remédio. {[}37{]} Se César Borgia o tivesse
conseguido -- e teria conseguido no mesmo ano que Alexandre morreu --,
teria conquistado tanta força e tanta reputação que por si mesmo
ter-se-ia mantido e não mais seria dependente da fortuna e força dos
outros, mas da sua potência e da sua \emph{virtù}\footnote{Para
  Maquiavel o ano de 1503 foi central na sorte política de César Borgia.
  Pelo raciocínio apresentado, se o papa tivesse sobrevivido mais 6
  meses César teria conquistado todos os seus objetivos políticos e
  consolidado seus domínios na Itália. Neste caso, a fortuna lhe faltou
  apesar de toda a sua \emph{virtù.}}.

{[}38{]} Mas Alexandre morreu cinco anos depois que ele começou a usar a
espada: deixando-o somente com o estado da Romanha consolidado e todos
os demais no ar, entre dois poderosos exércitos inimigos e doente de
morte\footnote{No mesmo período da morte do papa Alexandre VI, César
  Borgia também fica gravemente doente e, segundo Biaggio Buonacorsi,
  também diplomata em Florença e amigo da Maquiavel, suspeitava-se que
  ambos, o papa e César, haviam sido envenenados.}. {[}39{]} E havia no
duque tanta ferocidade e tanta \emph{virtù}, tão bem se conhece como os
homens podem conquistar ou perder, e tão válidos foram os fundamentos
que tinha lançado em tão pouco tempo, que, se não tivesse tido aqueles
exércitos contra si, ou estivesse são, teria resistido a todas as
dificuldades.

{[}40{]} E viu-se que os seus fundamentos eram bons: porque a Romanha o
esperou por mais de um mês; em Roma, ainda que semi-vivo, esteve seguro,
e, embora os Ballioni, os Vitelli e os Orsini retornassem a Roma, não
obtiveram seguidores contra ele; e ele conseguiu, se não fazer papa quem
ele queria, ao menos que não se tornasse papa quem ele não desejava.
{[}41{]} Mas se na morte de Alexandre ele estivesse são, todas as coisas
lhe teriam sido fáceis: e ele me disse, nos dias em que Júlio II foi
eleito papa, que pensou no que poderia ocorrer quando seu pai tivesse
morrido, e para tudo havia encontrado remédio, mas nunca pensara que, na
morte de seu pai, também ele estaria morrendo.

{[}42{]} Colhendo, pois, todas as ações do duque, eu não saberia
repreendê-lo, mas antes me parece o caso de, como tenho feito, propô-lo
imitável a todos aqueles que, pela fortuna e com as armas alheias,
ascenderam ao poder, porque ele, tendo grande ânimo e intenção elevada,
não poderia governar de outro modo, e somente se opuseram aos seus
desígnios a brevidade da vida de Alexandre e a sua doença. {[}43{]}
Quem, portanto, julga necessário no seu principado novo assegurar-se
contra os inimigos, ganha para si amigos; vencer ou pela força ou pela
fraude; ser amado e temido pelo povo, seguido e reverenciado pelos
soldados; eliminar aqueles que podem ou devem prejudica-lo; inovar com
novos costumes os antigos ordenamentos; ser severo e grato, magnânimo e
liberal, acabar com a milícia infiel, criar uma nova; manter a amizade
dos reis e dos príncipes de modo que te beneficiem-no com a sua graça ou
ofendam-no com cautela; não se pode encontrar exemplos mais recentes do
que as ações deste\footnote{Ou seja, César Borgia.}.

{[}44{]} Somente se pode acusá-lo na eleição do pontífice Júlio, na qual
o duque fez uma má escolha. {[}45{]} Porque, como foi dito, não podendo
fazer um papa a seu modo, ele poderia impedir que alguém o fosse; e não
deveria nunca consentir o papado àqueles cardeais que ofendeu ou que,
tornando-se papa, tivessem de ter medo dele, porque os homens ofendem ou
por medo ou por ódio. {[}46{]} Aqueles que ele ofendeu eram, entre
outros, os cardeais de São Pedro em Víncula\footnote{Giuliano Della
  Rovere.}, Colonna\footnote{Giovanni Colonna.}, São Jorge\footnote{Rafaello
  Riario.}, Ascânio\footnote{Ascanio Sforza.}; todos os outros, tornados
papa, tinham de temê-lo, exceto o de Ruão\footnote{George d'Ambroise,
  arcebispo de Ruão.} e os espanhóis: estes por parentesco e obrigação,
aquele pelo poder, pois tinha ao seu favor o reino de França. {[}47{]}
Portanto, o duque, antes de tudo, deveria fazer papa um espanhol, e, não
podendo, deveria consentir que o fosse o cardeal de Ruão e não o de São
Pedro em Víncula. {[}48{]} E quem crê que nos grandes personagens os
benefícios novos fazem esquecer as antigas injúrias, se engana. {[}49{]}
Errou, pois, o duque nesta eleição; e foi a razão de sua ruína final.

\quebra\section{\emph{DE HIS QUI PER SCELERA AD PRINCIPATUM PERVENERE}
{[}Daqueles que por atos criminosos conquistaram principados{]}}

{[}1{]} Mas porque há ainda dois modos de se passar de cidadão comum a
príncipe\footnote{Note-se que a frase que abre o capítulo apresenta a
  mesma temática do capítulo precedente: analisar aqueles que passam da
  condição de cidadãos para a condição de príncipe {[}\emph{diventa di
  privato in príncipe}{]}. Esse é, pois, um dos temas centrais do livro,
  conforme exposto na \emph{Introdução.}}, o que não se pode atribuir de
todo ou à fortuna ou à \emph{virtù}, não me parece que deva deixá-las de
lado, ainda que sobre uma delas se possa discorrer mais amplamente em se
tratando de repúblicas\footnote{No caso das repúblicas, trata-se de um
  cidadão que obtém a condição de príncipe em função da sua \emph{virtù}
  com o favor de seus concidadãos.}. {[}2{]} Estes modos são: ou quando
por algum meio criminoso e nefasto alguém ascende ao principado, ou
quando um cidadão comum, com o favor de outros cidadãos, torna-se
príncipe da sua pátria. {[}3{]} E falando do primeiro modo, mostrar-se-á
com dois exemplos, um antigo e outro moderno, sem entrar nos méritos
desta parte, porque eu julgo que basta, a quem for necessário,
imitá-los.

{[}4{]} Agátocles\footnote{Soberano de Siracusa de 316 a 289 a.C.}
siciliano, não só de condição particular\footnote{Essa é a mesma
  expressão do capítulo precedente, cf. nota 66.}, mas também ínfima e
abjeta, tornou-se rei de Siracusa. {[}5{]} Este homem, nascido de um
oleiro, sempre teve, nas diferentes fases da sua vida, uma conduta
celerada; não obstante, associou a seus crimes tanta \emph{virtù} de
alma e corpo, que, ingressando na milícia\footnote{Milicias eram
  formações armadas, contratadas por cidades ou aristocratas para ações
  militares específicas, se distinguindo, dos exércitos regulares. Esse
  tipo de formação era muito utilizado pelas cidades italianas na
  Renascença, haja vista, que muitas delas não possuíam exércitos
  regulares.}, pelos seus diversos graus, chegou a ser pretor\footnote{Cargo
  romano responsável pelo supremo comando militar ou pela magistratura
  suprema numa determinada localidade.} de Siracusa. {[}6{]} Ao ser
investido em tal posto, decidiu tornar-se príncipe e manter com
violência e sem obrigação a outrem aquilo que lhe tinha sido concedido
por um acordo. Tendo acordado este seu plano com o cartaginês
Amílcar\footnote{Conforme explica Inglese, a história é mais complexa do
  que esse rápido relato de Maquiavel. Diz Inglese: ``Os episódios
  narrados por Justino são mais complexos: Agátocles tentou tomar o
  poder, mas foi mandado ao exílio; tomou em vão as armas contra a
  pátria e, então, rogou ao cartaginês Amílcar (não confundí-lo com o
  homônimo Amílcar Barca, general cartaginês) de fazer a pacificação
  entre ele e os siracusanos. Em consequência disto obteve o título de
  pretor de Siracusa e, por fim, com o apoio de cinquenta mil homens de
  Amilcar, tomou o governo da cidade.'' {[}Inglese, 1995, p. 55, nota
  3{]}.} -- o qual militava com os seus exércitos na Sicília --, reuniu
certa manhã o povo e o senado de Siracusa, como se fosse deliberar algo
pertinente à república. {[}7{]} E com um aceno combinado, ordenou aos
seus soldados matarem todos os senadores e os mais ricos do povo. Mortos
estes, ocupou e manteve o principado daquela cidade sem nenhuma
controvérsia civil. {[}8{]} E embora fosse duas vezes derrotado pelos
cartagineses e, por fim, assediado por eles, não somente pode defender a
sua cidade (porém, deixando parte de seu exército na defesa dela), como
com a outra parte assaltou a África e em pouco tempo libertou Siracusa
do assédio e conduziu os cartagineses a um extremo perigo. Em seguida os
obrigou a um acordo com ele: os cartagineses ficaram com a possessão da
África e deixaram a Sicília para Agátocles.

{[}9{]} Quem considerar, portanto, as ações e a vida deste homem verá
pouca ou nenhuma coisa que se possa atribuir à fortuna, porque, como se
disse acima, não pelo favor de alguém, mas pelos postos da milícia -- os
quais conquistou com mil incômodos e perigos -- alcançou o principado; e
este, posteriormente, conservou com muitas resoluções corajosas e
perigosas. {[}10{]} Não se pode também chamar de \emph{virtù} matar os
seus cidadãos, trair os amigos, agir de má-fé, sem piedade, sem
religião: meios estes que permitem conquistar poder, mas não
glória\footnote{Maquiavel toca aqui num ponto central de sua teoria
  política, a necessidade de glória para que haja de fato \emph{virtù}.
  Glória entendida como admiração e louvor perante os outros. No caso de
  Agátocles, ele tinha respeito dos siracusanos, mas não era digno de
  admiração e louvor, donde a falta de glória.}. {[}11{]} Porque, se se
considera a \emph{virtù} de Agátocles ao entrar e ao sair dos perigos, e
a grandeza do seu ânimo ao suportar e superar as coisas adversas, não se
vê porque ele haveria de ser julgado inferior a qualquer excelentíssimo
capitão: todavia, a sua feroz crueldade e desumanidade, com infinitos
crimes, não permitiram que fosse celebrado entre os excelentíssimos
homens. {[}12{]} Não se pode, portanto, atribuir à fortuna ou à
\emph{virtù} aquilo que ele conseguiu sem uma e sem a outra.

{[}13{]} No nosso tempo, reinando Alexandre VI, Oliverotto de
Fermo\footnote{Oliverotto Euffreducci da Fermo nasceu em 1475 e morreu
  estrangulado a mando de César Bórgia em 31 de dezembro de 1502, visto
  que era um dos conspiradores da reunião em Perúgica, conforme narrado
  no capítulo VII {[}20{]}. Ele pertencia a uma família nobre da cidade
  e, conforme Guicciardini era um valente soldado e foi um dos
  auxiliares dos comandantes militares Vitellozzo Vitelli e Paolo
  Orsini.}, tendo ficado, quando era ainda pequeno, sem seus pais, foi
criado por um tio materno, chamado Giovanni Fogliani, e nos primeiros
anos de sua juventude foi destinado a servir sob o comando de Paulo
Vitelli\footnote{Paolo Vitelli, da cidade de Castelo, foi comandante
  militar (\emph{condottiero}) de grande fama, tendo papel de destaque
  em junho de 1498, quando comandou os soldados florentinos na
  reconquista de Pisa.}, a fim de que, pleno daquela disciplina,
alcançasse excelente posto na milícia. {[}14{]} Morto depois Paulo,
serviu sob o comando de Vitellozzo, irmão de Paulo, e, em brevíssimo
tempo, por ser engenhoso e corajoso de corpo e de alma, tornou-se o
primeiro homem\footnote{\emph{Primo uomo} que bem pode ser compreendido
  aqui como aquele que lidera, que está na vanguarda, que principia, ou
  seja, um \emph{príncipe}. Nos capítulos adiante Maquiavel insistirá na
  necessidade do príncipe ser também um comandante militar.} da sua
milícia. {[}15{]} Contudo, parecendo coisa servil estar sob as ordens de
outro, pensou em ocupar Fermo com a ajuda de alguns cidadãos desta
cidade -- aos quais era mais cara a servidão do que a liberdade da sua
pátria -- e com o patrocínio de Vitellozzo. {[}16{]} E escreveu a
Giovanni Fogliani que, por ter estado muitos anos fora de casa, desejava
revê-lo e rever sua cidade, e inspecionar algumas partes do seu
patrimônio. E porque não se tinha esforçado senão em conquistar
honrarias, a fim de que os seus cidadãos vissem como ele não perdera o
seu tempo em vão, queria vir à cidade de forma honrada e acompanhado por
cem cavaleiros de sua amizade e seus auxiliares. Pediu, pois, a seu tio
que fizesse com que eles fossem recebidos honrosamente pelos cidadãos de
Fermo, o que não traria honra apenas a ele, seu tio, mas também ao
próprio Oliverotto, seu discípulo.

{[}17{]} Não faltou Giovanni, portanto, com nenhum de seus deveres em
relação a seu sobrinho: fez os firmianos receberem-no honradamente e se
alojou\footnote{Aqui uma brusca mudança de sujeito; o sujeito é
  Oliverotto, que se aloja em sua própria casa em Fermo.} em sua casa,
onde, passando alguns dias empenhado em tramar secretamente aquilo que
era necessário a seus futuros crimes, fez um convite soleníssimo, com o
qual convidou Giovanni Fogliani e todos os principais homens de Fermo.
{[}18{]} E uma vez consumidas as iguarias e todos os outros
entretenimentos empregados comumente nos banquetes, Oliverotto,
artificialmente, teceu certos argumentos graves, falando da grandeza do
papa Alexandre e de César, seu filho, e dos seus feitos. A tais
raciocínios responderam Giovanni e os outros. De repente Oliverotto se
levantou, dizendo que aquilo era coisa para falar em lugar mais secreto
e retirou-se para uma câmara, onde Giovanni e todos os outros cidadãos
lhe seguiram. {[}19{]} Nem bem tinham se sentado quando, de lugares
secretos daquela câmara, saíram soldados que mataram Giovanni e todos os
outros. {[}20{]} Depois de tais homicídios, Oliverotto montou a cavalo e
atravessou correndo a cidade, e assediou no palácio o supremo
magistrado\footnote{A Magistratura Suprema era composta pelos membros do
  Conselho Geral, a maior instância deliberativa da cidade. Em geral,
  esses conselhos eram dominados pelas aristocracias locais.}, tanto
que, por medo, os firmianos foram constrangidos a obedecê-lo e a formar
um governo do qual se fez príncipe. Mortos todos aqueles, que
descontentes, podiam prejudicá-lo, fortaleceu-se com novos ordenamentos
civis e militares, de modo que, no espaço de um ano depois de ter tomado
o principado, ele não somente estava seguro na cidade de Fermo, mas se
tornou temido por todos os seus vizinhos. {[}21{]} E seria a sua
expulsão difícil, como aquela de Agátocles, se não se tivesse deixado
enganar por César Bórgia, quando em Sinigallia, como acima se
disse\footnote{Capítulos VII {[}20-21{]}}, foram presos os Orsini e os
Vitelli, e onde foi preso ele também. Um ano depois de cometer o
parricídio\footnote{Isto é, o tio Giovanni, que fez as vezes de pai para
  Oliverotto.}, foi estrangulado, junto com Vitellozzo, que foi mestre
da sua \emph{virtù} e de seus crimes.

{[}22{]} Poderia alguém se perguntar como foi possível que Agátocles e
alguns semelhantes, depois de infinitas traições e crueldades, pudessem
viver longamente seguros na sua pátria e se defenderem dos inimigos
externos, sem que seus cidadãos jamais tivessem conspirado contra eles:
e isto, apesar de muitos outros não terem podido, por meio da crueldade,
conservar o governo nem nos tempos de paz, nem nos duvidosos tempos de
guerra. {[}23{]} Creio que isto advenha da crueldade mal usada ou bem
usada. {[}24{]} ``Bem usadas'' se podem chamar aquelas -- se é lícito
falar bem do mal\footnote{Aqui Maquiavel indica um tema que será melhor
  analisado nos capítulos XV e XVII.} -- que se fazem de uma só vez pela
necessidade de assegurar-se, e depois não se insiste mais nelas, mas se
convertem na maior quantidade possível de benefícios para os súditos.
{[}25{]} ``Mal usadas'' são aquelas as quais, ainda que no princípio
sejam poucas, rapidamente crescem com o tempo, em vez de se extinguirem.
{[}26{]} Aqueles que observam o primeiro modo, podem com Deus e com os
homens ter algum remédio para o seu governo, como teve Agátocles; os que
empregam mal a crueldade não conseguem se manter.

{[}27{]} Donde é de se notar que, ao pilhar um governo, deve o invasor
fazer todas aquelas afrontas que são necessárias, e fazê-las de uma só
vez, para não ter de renovar tudo e para poder, não as renovando,
tranquilizar os homens e ganhá-los ao beneficiá- los. {[}28{]} Quem faz
de outro modo, ou por timidez ou por mau conselho, sempre precisa ter a
faca na mão; também não pode nunca se apoiar nos seus súditos, nem podem
estes, pelas injúrias recentes e contínuas, jamais confiar nele.
{[}29{]} Por isso, as injúrias devem ser feitas todas de uma só vez, a
fim de que se saboreiem menos e afrontem menos; os benefícios se devem
fazer pouco a pouco, afim de serem melhor saboreados. {[}30{]} E um
príncipe deve, sobretudo, viver com os seus súditos de modo que nenhum
acidente, mau ou bom, obrigue-o a mudar, porque, advindo as necessidades
em tempos adversos, você não terá tempo para o mal, e o bem que fizer
não lhe serve, porque é julgado forçado, e não terá com ele nenhum
reconhecimento

\quebra\section{\emph{DE PRINCIPATU CIVILI}
{[}Do principado civil{]}}

{[}1{]} Voltando à outra parte\footnote{No início do cap. VIII (2),
  Maquiavel havia apresentado dois modos de se ascender à condição de
  príncipe (não calcado totalmente na \emph{virtù} e nem na fortuna), a
  saber: por meio de atos criminosos -- conforme exposto no capítulo
  VIII, e quando um cidadão, com auxílio de seus concidadãos, torna-se
  príncipe e lidera a cidade. É deste segundo ponto que Maquiavel
  tratará aqui. A conjunção italiana \emph{ma} não possui, portanto,
  valor adversativo e sim consecutivo, donde não ser indicado inseri-la
  na tradução para que não haja uma confusão na interpretação.}, quando
um cidadão comum\footnote{Conforme explicado na nota 66, traduzimos aqui
  \emph{privato ciptadino} por ``cidadão comum''. Entretanto, devemos
  ter claro, não se trata de um cidadão qualquer, mas daquele que assume
  o comando ou a liderança das ações políticas. Neste caso, ao usar o
  termo \emph{privato}, Maquiavel parece querer reforçar essa
  possibilidade de transformação do \emph{status} do homem, que pode
  sair de sua condição de privada, particular, alheia às demandas
  públicas e se engajar na esfera pública, seja ocupando cargos, seja
  liderando politicamente a cidade. A condição particular do cidadão,
  pela expressão construída, enfatiza, neste caso, a antítese da
  condição deste indivíduo em relação à esfera pública. Maquiavel dá a
  entender que escolheu o indivíduo mais avesso ao mundo da política
  para transformá-lo em modelo de ação política, deixando claro que a
  todos está aberta a possibilidade de se inserir na esfera pública,
  desde que busque agir com \emph{virtù} e saiba lidar com os desafios
  da fortuna.}, não por meio de crimes ou outra violência intolerável,
mas com o favor dos outros cidadãos, torna-se príncipe da sua pátria --
que poderia ser chamada de principado civil\footnote{Destaque-se que é a
  primeira vez que Maquiavel se refere a um principado no singular,
  visto que todos os outros exemplos são dados no plural. Trata-se,
  pois, de um caso especial no qual a grande maioria dos comentadores do
  pensamento político de Maquiavel entende ser um modelo de regime mais
  indicado para a crise das repúblicas, crises estas constantes em
  Florença, bem como nas demais repúblicas de seu tempo.}: e para sê-lo
não é necessário toda \emph{virtù} ou toda fortuna, mas, antes, uma
astúcia afortunada\footnote{A \emph{astuzia} (astúcia) se não é
  entendida como sinônimo da \emph{virtù} é certamente uma de suas
  qualidades principais, conforme se verá no cap. XVIII. Essa expressão
  sintetiza a condição principal de um cidadão que deseje tornar-se
  príncipe: ter astúcia e \emph{virtù}, mas que elas venham acompanhadas
  da fortuna, donde a \emph{astuzia afortunada}, ou seja, astúcia e
  \emph{virtù} com fortuna.} --, digo que se ascende a este principado
ou com o favor do povo\footnote{Povo (\emph{populo}) não deve ser
  entendido aqui como sendo somente os pobres, o oposto dos ricos ou dos
  grandes, mas conforme explicou Tafuro, trata-se, para Maquiavel, de
  uma denominação que inclui vários setores sociais, principalmente as
  parcelas medianas, no caso, algo que seria para nós a classe média
  composta de artesãos, comerciantes, pequenos proprietários. Cf.
  Tafuro, A. \emph{La formazione di Niccolò Machiavelli.} Napoli: Dante
  \& Descates, 2003, {[}parte I, 1.2 e 1.3{]}} ou com o favor dos
grandes\footnote{O termo \emph{grande} é sinônimo também de nobre ou
  aristocrata.}. {[}2{]} Porque em toda cidade se encontram estes dois
humores\footnote{Maquiavel recupera aqui uma imagem clássica da
  medicina, mostrando que entende a cidade como um corpo político, tal
  qual o pensamento político clássico sempre a entendeu. Os humores na
  medicina antiga eram quatro: o sanguínio, o fleumático, a bile negra e
  a bile amarela. Da combinação desses humores nascem os diversos
  temperamentos dos corpos. Aqui trata-se de forças políticas que atuam
  na cidade e, à semelhança do que ocorre nos corpos, na predominância
  de um desses humores, a cidade apresentará um certo ``temperamento''
  ou tendência. Essa imagem está também nos \emph{Discursos sobre a
  primeira década de Tito Lívio,} livro I, cap. 4 (5): ``e' sono in ogni
  republica due umori diversi, quello del popolo e quello de' grandi.''}
diversos e nasce, disto, que o povo deseja não ser nem comandado nem
oprimido pelos grandes e os grandes desejam comandar e oprimir o povo.
Destes dois apetites diversos nasce na cidade um destes três efeitos: ou
o principado, ou a liberdade ou a licença\footnote{Nota sobre licença
  como regime político.}. {[}3{]} O principado origina-se do povo ou dos
grandes, segundo que uma ou outra destas partes tenha a ocasião, porque,
vendo os grandes que não podem resistir ao povo, começam a aumentar a
reputação e o prestígio de um dos seus e fazem-no príncipe para poderem,
sob sua proteção, desafogar o seu apetite. O povo, também, vendo que não
pode resistir aos grandes, aumenta a reputação de um e o faz
príncipe\footnote{Neste caso, para aquele que se torna príncipe com o
  favor do povo, não se trata de alguém do próprio povo. Maquiavel não
  apresenta esta restrição, que se verifica no caso dos grandes que
  escolhem alguém de seu grupo social. Para o caso do povo pode ocorrer
  que seja até um grande o escolhido pelo povo, desde que ele defenda o
  povo do desejo de dominação dos grandes.}, para serem defendidos pela
sua autoridade.

{[}4{]} Aquele que chega ao principado com a ajuda dos grandes,
conserva-se com mais dificuldade do que aquele que chega com a ajuda do
povo, porque, como príncipe, encontra-se com muitos ao entorno que se
lhe equiparam como seus iguais e, por isto, não lhes pode nem comandar
nem guiá-los a seu modo. {[}5{]} Mas, aquele que chega ao principado
pelo favor popular, encontra-se sozinho e tem entorno a si ou nenhum ou
pouquíssimos que não estejam dispostos a lhe obedecer. {[}6{]} Além
disto, não se pode, com honestidade, satisfazer os grandes sem injuriar
outros, mas ao povo sim, porque o fim do povo é mais honesto que o dos
grandes, querendo esses oprimir e aqueles não ser oprimidos\footnote{Essa
  é uma máxima de Maquiavel para caracterizar os desejos ou humores
  opostos dos grandes e do povo: os grandes querem oprimir e o povo não
  quer ser oprimido. Essa caracterização é fundamental para a reflexão
  política maquiaveliana, pois dela é que derivam várias consequências
  argumentativas.}. {[}7{]} Além disso, um príncipe inimigo do povo não
pode nunca estar seguro, por serem esses muitos; contra os grandes pode
estar seguro, por serem poucos. {[}8{]} O pior que pode acontecer a um
príncipe inimigo do povo é ser abandonado por ele, mas dos grandes, que
lhe são inimigos, não somente deve temer ser abandonado, mas ainda mais
eles lhe venham contra, porque, tendo estes mais visão e mais astúcia,
sempre antecipam o tempo para se salvarem e buscarem as graças de quem
esperam que vença. {[}9{]} É necessário, também, ao príncipe, viver
sempre com o mesmo povo, contudo, pode muito bem dispensar os mesmos
grandes, podendo fazer e se desfazer deles todo dia, e tirar-lhes e lhes
dar, a seu bel prazer, a sua reputação.

{[}10{]} Para esclarecer melhor esta parte, digo que se devem considerar
os grandes de dois modos principais: ou se governam de maneira que, com
o seu proceder, estejam totalmente vinculados a sua fortuna, ou não.
{[}11{]} Aqueles que se vinculam a você, e não são rapaces, devem-se
honrar e amar. {[}12{]} Aqueles que não se vinculam a você devem ser
examinados de dois modos: ou eles fazem isto por pusilanimidade e
defeito natural de ânimo\footnote{\emph{Difetto naturalle d'animo}
  (defeito natural de ânimo) trata-se de falta de coragem, completando e
  explicando a condição do pusilânime (volúvel).} -- então tu deves
servir-te mais ainda daqueles que são bons conselheiros, porque na
prosperidade te honram e na adversidade não terás que temê-los --,
{[}13{]} ou quando eles não se vinculam a você por artifício e por
motivos ambiciosos, é sinal de que pensam mais em si mesmos que em você.
Deve o príncipe, pois, proteger-se e temê-los como se fossem inimigos
declarados, porque sempre, nas adversidades, colaboram para arruiná- lo.

{[}14{]} Deve, portanto, alguém que se torna príncipe mediante o favor
do povo, conservar-se amigo dele, o que faz facilmente, pois não desejam
eles senão não ser oprimidos. {[}15{]} Mas alguém que, contra o povo,
torna-se príncipe com o favor dos grandes, deve, antes de qualquer outra
coisa, procurar ganhar o povo para si, o que ele faz facilmente, quando
assume a proteção do povo. {[}16{]} E porque os homens ficam mais
ligados a seu benfeitor quando recebem o bem de quem acreditavam receber
o mal, torna-se o povo imediatamente mais benévolo para com ele do que
se tivesse ele chegado ao principado com os favores do povo. {[}17{]} E
pode o príncipe ganhá-los de muitos modos: aos quais, porque variam
segundo a matéria\footnote{Isto é, conforme a lugar, conforme a cidade.},
não se pode dar uma regra certa e constante, e, por isso, deixaremos de
lado essa questão. {[}18{]} Concluo, somente que, a um príncipe, é
necessário ter o povo como amigo, pois, de outro modo, não tem remédio
nas adversidades. {[}19{]} Nabis\footnote{Nabis, tirano de Esparta de
  206 a 192 a.C.}, príncipe dos espartanos, suportou o assédio de toda
Grécia e de um exército romano vitoriosíssimo e defendeu contra eles a
sua pátria e o seu status. Bastou-lhe somente, quando sobreveio o
perigo, assegurar-se contra poucos, o que, caso ele fosse inimigo do
povo, não lhe teria sido suficiente.

{[}20{]} E não me venha alguém refutar esta minha opinião com aquele
provérbio trivial, de que quem se apóia sobre o povo, apóia-se sobre a
lama: porque, com efeito, isto é verdadeiro quando um cidadão comum faz
do povo o seu fundamento, e ilude-se que o povo o libertará quando ele
for oprimido pelos inimigos ou pelos magistrados\footnote{Esse
  raciocínio complementa aquilo que foi dito no início do capítulo.
  Maquiavel não afirmou na linha 3 que o povo escolhe um dos seus para
  príncipe, mas alguém que o defenda da vontade opressora dos grandes,
  sem distinção aqui de grupo social. Agora ele explicita essa ideia ao
  declarar que o cidadão comum (no caso um \emph{privato ciptadino}, que
  assumiu a condição de príncipe) que se apóia tão somente no povo, não
  tem respaldo o bastante para manter-se na condição de comando. Em
  suma, faz-se necessário a este \emph{privato ciptadino} buscar alguma
  forma de apoio dos grandes. No limite, o príncipe tem sempre que
  buscar o apoio do outro grupo político: se ele for membro dos grandes,
  necessita de apoio do povo, se ele for um \emph{privato ciptadino},
  tem que buscar apoio dos grandes.}. {[}21{]} Neste caso se poderia
frequentemente enganar, como em Roma os Gracos\footnote{Ou seja, os
  irmãos Tibério Sempronio e Caio Sempronio, líderes da revolução
  popular de Roma e Tribunos da Plebe, sendo o primeiro assassinado em
  133 a.C., e o segundo em 121 a.C.} e em Florença o senhor Giorgio
Scali\footnote{Giorgio Scali foi eleito membro da \emph{Signoria} em 1
  de setembro de 1378 após os conflitos do \emph{Ciompi}. Por três anos
  foi um dos líderes da cidade, com o apoio das \emph{Artes Menores}, ou
  seja, os artesãos em oposição aos grandes comerciantes, mas caiu em
  desgraça e foi decapitado, por vontade da \emph{Signoria}, em 17 de
  janeiro de 1382.}. {[}22{]} Mas, sendo um príncipe que se funda sobre
o povo, sendo que ele pode comandar e é homem de coragem, que não se
amedronta nas adversidades, nem lhe falta outros preparos e mantém com a
sua coragem e seus ordenamentos todos animados, nunca se encontrará
enganado pelo povo e lhe parecerá ter bem feito os seus fundamentos.

{[}23{]} Costumam estes principados correr perigo quando passam de
principado civil a principado absoluto. {[}24{]} Porque estes príncipes
ou comandam por si mesmo ou por meio dos magistrados: no último caso é
mais débil e mais perigosa a situação deles, porque estão em tudo
submetidos à vontade daqueles cidadãos que são prepostos como
magistrados, os quais, principalmente nos tempos adversos, lhe podem
retirar o estado com grande facilidade ou abandonando-o\footnote{Existe
  uma divergência nas edições de Giorgio Inglese e Mario Martelli: o
  primeiro apresenta a expressão \emph{o con abbandonarlo} e o segundo
  \emph{o con non lo obedire}. Contudo, como explica Martelli em sua
  nota (nota 39, p. 170) ``basta ter em conta o fato que abandoná-lo e
  obedecê-lo são paleograficamente conversíveis um pelo outro''.
  Portanto, seja qual for a expressão utilizada, o sentido
  conservar-se-ia equivalente.}, ou não o obedecendo, ou agindo
ativamente contra o príncipe. {[}25{]} E o príncipe não tem tempo, nos
momentos perigosos, em retomar a autoridade absoluta, porque os cidadãos
e os súditos, que estavam habituados aos comandos dos magistrados, nesta
conjuntura, não são obedientes às suas ordens. {[}26{]} E sempre terá,
nos tempos de penúria, poucos em quem possa confiar, porque um príncipe
assim não pode fundar-se sobre aquilo que vê nos tempos calmos, quando
os cidadãos têm necessidade do governo, porque quando a morte é
distante, todos acorrem, todos prometem e todos desejam morrer por ele,
mas nos tempos adversos, quando o governo necessita dos cidadãos, nesta
hora se encontram poucos. {[}27{]} E, tanto mais é esta experiência
perigosa quanto não se pode tê-la senão uma vez. Porém, um príncipe
sábio deve pensar em um modo pelo qual os seus cidadãos, sempre e em
todas as ocasiões, precisem do governo e dele, e assim sempre lhe serão
fiéis.

\quebra\section{\emph{QUOMODO OMNIUM PRINCIPATUUM VIRES PERPENDI DEBEANT} {[}De que modo se devem considerar as forças de todos os principados{]}}

{[}1{]} Convém fazer, ao examinar as qualidades destes
principados\footnote{Martelli sustenta, com alguma razão, que esse
  capítulo é uma continuação da argumentação do capítulo IX, sendo que
  naquele Maquiavel tratou das coisas internas do principado civil e
  neste ele tratará das coisas externas. Cf. Martelli, M. \emph{Il
  Príncipe}. Roma: Salerno, 2006, p. 171, n. 2.}, uma outra
consideração: isto é, se um príncipe se vê em uma tal condição na qual
possa, se necessitar, regê- la\footnote{O vocábulo \emph{reggere} aqui
  demonstra como Maquiavel concebe a condição do príncipe como alguém
  que ``rege'' e não alguém que ``reina'', donde se comprova mais uma
  vez que não temos presente ainda a noção do príncipe soberano próprio
  do pensamento político da Modernidade. Trata-se, antes, da clássica
  imagem do maestro que rege o coro, alguém que conduz e guia outros,
  mas não se impõe pelo seu poder. Sobre essa distinção no pensamento
  político anterior a Maquiavel, cf. Senellart, M. \emph{As artes de
  governar}. São Paulo: ed. 34, 2006.} por si mesmo, ou, se de fato, tem
sempre necessidade de ser defendido por outros. {[}2{]} E, para
esclarecer melhor esta parte, digo como eu julgo aqueles que
podem\footnote{Como nos lembra Inglese, note-se o paralelismo deste
  \emph{``coloro potersi regere ... che possono''} com aquilo que se
  afirma na linha seguinte, ``\emph{coloro avere ... che non possono''},
  ou seja, Maquiavel está tratando inicialmente ``daqueles que podem''
  e, em seguida, ``daqueles que não podem''. Cf. Inglese, G. \emph{Il
  Príncipe}. Torino: Einaudi, 1995, p. 69, cap. X, n. 2.} reger por si
mesmos, ou por abundância de homens ou de dinheiro, reunir um exército
suficiente para uma batalha campal contra qualquer um que venha
assaltá-los. {[}3{]} E assim julgo que têm sempre necessidade de outros
aqueles que não podem defrontar-se com o inimigo em campo aberto, mas
necessitam refugiar-se dentro dos muros e protegê-los. {[}4{]} Acerca do
primeiro caso, já se discutiu\footnote{No capítulo IX.} e, em seguida,
diremos algo mais a esse respeito. {[}5{]} Acerca do segundo caso, não
se pode dizer outra coisa senão exortar tais príncipes a fortificar e
municiar a própria cidade, e não levar em conta os demais territórios.
{[}6{]} E qualquer um que tiver bem fortificada a sua cidade e, acerca
dos outros governos, tenha-se comportado com os súditos como acima se
disse e adiante se dirá, será sempre atacado com grande respeito, porque
os homens são sempre inimigos das empresas em que vêem dificuldade: e
tampouco podem ver facilidade em assaltar alguém que tenha a sua cidade
fortificada e não seja odiado pelo povo.

{[}7{]} As cidades da Alemanha\footnote{Maquiavel esteve em missão
  diplomática no sul da Alemanha e na Suiça entre 1507 e 1508, momento
  esse em que também escreveu alguns opúsculos com análises das
  condições que encontrou nesses territórios. Nesta parte do capítulo
  encontramos um resumo dessas análises sobre a Alemanha, cuja exposição
  mais ampla se encontra em \emph{Discorso sopra le cose della Magna e
  sopra l'Imperatore}, \emph{Ritratto delle cose della Magna} e
  \emph{Rapporto di cose della Magna}. Cf. Machiavelli, N. \emph{L'Arte
  della guerra/ Scritti politici minori}. Edizione Nazionale delle Opere
  -- I/3. Roma: Salerno, 2001.} são libérrimas, têm poucos territórios e
obedecem ao imperador quando desejam, e não temem nem este nem aquele
poderoso que habita no seu entorno. {[}8{]} Porque elas são de tal modo
fortificadas que qualquer um pensa que sua conquista deve ser tediosa e
difícil, pois todas têm fosso e muros convenientes; têm artilharia
suficiente; têm sempre nos celeiros públicos bebidas, mantimentos e
combustíveis para um ano; {[}9{]} e, além disto, para poder manter a
plebe alimentada e sem prejuízo para o poder público, têm sempre na
cidade trabalhos para dar-lhe por um ano naquelas atividades que são o
nervo e a vida daquela cidade e naquelas atividades com as quais a plebe
se sustenta; têm, ainda, em grande conta os exercícios militares, e
nisto tomam as medidas necessárias para mantê-los.

{[}10{]} Um príncipe, portanto, que tenha uma cidade assim ordenada e
não se faz odiado, não pode ser tomado de assalto, e, ainda que houvesse
quem o tomasse de assalto, partiria com vergonha, porque as coisas do
mundo são tão variadas que é quase impossível que alguém pudesse ficar
com os exércitos ociosos por um ano a sitiá-lo. {[}11{]} E quem a isto
replicasse que o povo tem suas posses fora dos muros, e que ao vê-las
arder, não terá paciência, e o longo assédio e o amor a si próprio o
fará esquecer o amor ao príncipe, respondo que um príncipe prudente e
animoso superará sempre todas aquelas dificuldades, dando aos súditos
ora esperança de que o mal não será longo, ora incutindo o temor da
crueldade do inimigo, ora precavendo-se com astúcia daqueles que lhe
parecem muito ousados. {[}12{]} Além disto, é razoável pensar que o
inimigo deva queimar e arruinar o país em sua investida e nos momentos
em que os ânimos dos homens estão ainda quentes e voluntariosos para a
defesa. Porém, por isso deve o príncipe recear menos ainda, porque
depois de alguns dias, quando as forças estão arrefecidas, os danos já
foram feitos, os males já aceitos, e não há mais remédio. {[}13{]} E
então tanto mais se unirão a seu príncipe, parecendo-lhes que este tem
obrigação para com eles, tendo sido as suas casas queimadas e arruinadas
as suas posses para a defesa dele. E é da natureza dos homens obrigar-se
tanto pelos benefícios que se fazem, como por aqueles que se recebem.
{[}14{]} Donde, se se considera tudo bem, não é difícil a um príncipe
prudente manter, antes e depois, firmes os ânimos dos seus cidadãos na
defesa da cidade, enquanto não lhes faltar nem o necessário para viver,
nem o necessário para se defender.

\quebra\section{\emph{DE PRINCIPATIBUS ECCLESIASTICIS}
{[}Dos principados eclesiásticos{]}}

{[}1{]} Resta somente discorrer, agora, sobre os principados
eclesiásticos, acerca dos quais todas as dificuldades estão antes de se
possuí-los, porque se conquistam ou pela \emph{virtù} ou pela fortuna, e
sem uma e outra se conservam, pois são apoiados pelas antigas ordens da
religião, as quais têm sido tão poderosas e de uma tal qualidade que
conservam os seus príncipes no poder, não importando o modo como
procedam e vivam. {[}2{]} Somente eles têm estados e não os defendem;
têm súditos e não os governam. {[}3{]} Os estados, por serem indefesos,
não lhes são retirados, e os súditos, por não serem governados, não se
preocupam com isso, nem pensam e nem podem rebelar- se contra eles.
{[}4{]} Logo, apenas estes principados são seguros e felizes, mas, sendo
eles regidos por razões superiores, as quais a mente humana não alcança,
deixarei de falar deles, porque sendo engrandecidos e mantidos por Deus,
seria ofício de um homem presunçoso e temerário discorrer sobre eles.
{[}5{]} Todavia\footnote{Neste momento da exposição evidencia-se uma das
  marcas do estilo retórico de Maquiavel, que indica de início que os
  principados eclesiásticos são diferentes dos demais principados,
  contudo, apresenta a seguir a dinâmica política inerente a estes
  principados, igualando-os aos outros. No limite, como declara o
  próprio texto, na esfera temporal e na dinâmica das lutas políticas,
  os principados eclesiásticos são iguais aos demais principados.}, se
alguém me perguntasse por onde a Igreja chegou a tamanha grandeza na
esfera temporal -- uma vez que, antes de Alexandre\footnote{O papa
  Alexandre VI.}, os poderosos italianos, e não somente aqueles que se
chamavam poderosos, mas todos os barões e senhores, ainda que pouco
poderosos no plano temporal, a prezavam pouco, e agora um rei como o de
França a teme, e ela pôde expulsá-lo da Itália e arruinar os venezianos
--, coisa que, ainda que seja conhecida, não me parece supérfluo trazer
parte dela de novo à memória.

{[}6{]} Antes que Carlos, rei de França, invadisse a Itália\footnote{Em
  1494.}, estava esta província sob o império do papa, dos venezianos,
do rei de Nápoles, do duque de Milão e dos florentinos. {[}7{]} Estes
poderosos tinham duas preocupações principais: uma, que um forasteiro
não entrasse na Itália com suas armas; a outra, que nenhum dentre eles
ocupasse mais poder do que os outros. {[}8{]} Aqueles que suscitavam
mais preocupação eram o papa e os venezianos. Para deter os venezianos,
era necessária a união de todos os outros, como aconteceu na defesa de
Ferrara; para limitar o poder do papa, serviram- se dos barões de Roma,
os quais, estando divididos em duas facções -- os Orsini e os
Colonna\footnote{No início do século XVI as famílias Orsini e Colonna
  exerciam grande influência sobre os rumos políticos do papado e de
  Roma, de modo que era muito difícil governar a cidade sem um acordo
  político com alguma das partes.} --, sempre havia motivo de discórdia
entre eles, e, estando com as armas em mãos e com os olhos sobre o
pontífice, conservaram o pontificado débil e enfermo. {[}9{]} E ainda
que surgisse alguma vez um papa corajoso, como foi Sixto\footnote{Sixto
  IV, nascido Francesco Maria Della Rovere, foi papa de 1471 a 1484.}, a
fortuna ou o conhecimento, todavia, não puderam nunca livrá-lo destes
incômodos. {[}10{]} E a brevidade da vida deles era a razão disso,
porque em dez anos que, em média, vivia um papa, mal podia diminuir o
poder de uma das facções. Caso, por exemplo, um deles houvesse quase
eliminado os Colonna, surgia um outro inimigo dos Orsini, que fazia
ressurgir aqueles e os Orsini não tinham tempo de extingui-los. {[}11{]}
Isto fazia com que as forças temporais do papa fossem pouco respeitadas
na Itália.

{[}12{]} Surge depois Alexandre VI, o qual, de todos os pontífices que
já existiram, mostrou quanto um papa, com o dinheiro e com a força,
poderia prevalecer; e fez, usando o duque Valentino como seu instrumento
e pela ocasião da invasão dos franceses, todas aquelas coisas sobre as
quais eu discorri acima\footnote{No cap. VII.} ao tratar das ações do
duque. {[}13{]} E, embora a sua intenção não fosse engrandecer a Igreja,
mas o duque, contudo, o que ele fez trouxe grandeza à Igreja, a qual,
depois de sua morte, desaparecido o duque, foi herdeira de seus
esforços.

{[}14{]} Veio depois o papa Júlio, que encontrou a Igreja grande,
dominando toda a Romanha, eliminados os barões de Roma e, pelas
investidas de Alexandre, anuladas aquelas facções. Ele encontrou o
caminho ainda mais aberto para acumular dinheiro, jamais usado antes de
Alexandre. {[}15{]} Coisas que Júlio não apenas continuou, mas aumentou,
pois pensou em conquistar Bolonha, eliminar os venezianos e expulsar os
franceses da Itália: em todas estas empresas obteve êxito, e com tanto
mais honra para si, quanto fez todas essas coisas para o engrandecimento
da Igreja e não para o de algum cidadão. {[}16{]} Conservou, ainda, os
partidos dos Orsini e Colonna nos limites que os encontrou. {[}17{]} E,
embora entre eles algum chefe fosse propenso a suscitar discórdias, duas
coisas, todavia, detiveram-nos: uma, a grandeza da Igreja, que lhes
amedrontava; outra, não ter eles cardeais, os quais são as origens dos
tumultos entre eles: e nunca ficarão tranquilos estes partidos enquanto
tiverem cardeais, porque estes

alimentam, em Roma e fora dela, os partidos e os barões para
defendê-los; e, assim, da ambição dos prelados nascem as discórdias e os
tumultos entre os barões.

{[}18{]} Sua Santidade, o papa Leão\footnote{Leão X, nascido Giovanni di
  Medice, foi eleito papa em 21 de fevereiro de 1513.}, encontrou,
portanto, este\footnote{Este período final é um excelente indicativo
  temporal do momento da escrita de \emph{O Príncipe}, ou seja, a obra
  foi escrita durante o papado de Leão X, provavelmente no início do
  pontificado e não foi revisada posteriormente, pois nesse caso
  Maquiavel corrigiria essa informação de época. Tudo leva a crer que,
  ao menos essa primeira parte que trata dos principados, foi escrita no
  ano de 1513.} pontificado poderosíssimo, do qual se espera, se aqueles
o fizeram grande com as armas, que este, com a sua bondade e as suas
outras infinitas \emph{virtù}, torne-o grandíssimo e venerável.

\quebra\section{\emph{QUOT SINT GENERA MILITIAE ET DE MERCENNARIIS MILITIBUS}
{[}De quantos são os gêneros de milícias e de soldados mercenários{]}}

{[}1{]} Tendo discorrido detalhadamente sobre todas as qualidades
daqueles principados que no princípio nos propuzemos a pensar,
considerando em alguns pontos as razões do bem e mal-estar deles, e
expostos os modos pelos quais muitos têm procurado conquistá-los e
conservá-los, resta-me agora discorrer, de modo geral\footnote{Maquiavel
  já havia escrito, durante o seu período de trabalho na Chancelaria de
  Florença, alguns opúsculos sobre questões militares. Posteriormente
  ele escreverá uma obra somente sobre esse tema, \emph{A arte da
  guerra} (entre 1516 e 1520), o que comprova os vários testemunhos que
  indicam que Maquiavel era reputado como um grande conhecedor de
  questões militares.}, sobre os ataques e as defesas que podem ocorrer
em cada um dos principados citados anteriormente\footnote{O período
  inicial revela a conclusão daquilo que foi proposto no capítulo I, no
  caso, a exposição sobre os principados, que ocupou a primeira parte do
  livro. Ciente disso, Maquiavel apresenta em seguida a justificativa
  que articula os três capítulos seguintes (XII, XIII e XIV) com o que
  havia sido exposto, eliminando com isso uma quebra na argumentação.}.

{[}2{]} Dissemos acima\footnote{Cap. VII, 4.} quão necessário é a um
príncipe ter os seus fundamentos bons, pois, de outro modo, é forçoso
que se arruíne. {[}3{]} Os principais fundamentos comuns a todos os
estados, tantos os novos como os velhos e os mistos, são as boas leis e
as boas armas: porque não podem ser boas as leis onde não há boas armas,
e onde há boas armas convém que haja boas leis. Deixarei de lado o
raciocínio relativo às leis e falarei das armas\footnote{A noção de arma
  remete à força militar de uma cidade. A importância do fator militar
  na política é um \emph{topos} recorrente no pensamento político de
  Maquiavel, presente em outros escritos políticos, seja nas grandes
  obras, como os \emph{Discursos sobre a primeira década de Tito Lívio},
  seja em opúsculos menores, e principalmente na \emph{Arte da Guerra},
  que, como dito, é dedicado inteiramente ao tema.}.

{[}4{]} Digo, portanto, que as armas com as quais um príncipe defende o
seu estado ou são próprias ou são mercenárias, ou auxiliares ou mistas.
{[}5{]} As mercenárias e auxiliares são inúteis e perigosas. Se alguém
tem o seu estado fundado sobre as armas mercenárias, nunca estará nem
firme nem seguro, porque elas são desunidas, ambiciosas, sem disciplina,
infiéis, valorosas entre os amigos, vis entre os inimigos: não temerosas
a Deus, não confiáveis para com os homens; com elas se adia a ruína
enquanto se adia o ataque; na paz se é espoliado por elas, na guerra
pelos inimigos. {[}6{]} A razão disto é que elas não têm outro amor nem
outra razão que as conserve em campo, senão um pouco de soldo, o qual
não é suficiente para fazer com que queiram morrer por você. {[}7{]}
Querem bem ser seus soldados enquanto você não for à guerra, mas, quando
a guerra vem, ou fogem ou se vão. {[}8{]} E disto é fácil de se
persuadir, porque a atual ruína da Itália não é causada por outra coisa
senão por ter-se, pelo espaço de muitos anos, apoiado inteiramente sobre
as armas mercenárias. {[}9{]} Tais armas já fizeram algumas conquistas
sob o comando de alguns\footnote{Ou seja, ``pela mão de alguém''.} e
pareciam valorosas quando combatiam entre si, mas, quando veio o
forasteiro\footnote{Carlos V, rei da França.}, elas mostraram aquilo que
eram. Por isso para Carlos, rei de França\footnote{Maquiavel já havia
  feito referência a este fato no cap. VII, 18.}, foi lícito riscar a
Itália com giz\footnote{Maquiavel se vale aqui de uma metáfora para
  mostrar a força e o poder do rei francês Carlos V, que numa investida
  rápida conquistou toda a península itálica, como se passasse um giz
  sobre uma lousa. Essa expressão é atribuída a Alexandre VI pelo
  historiador francês Philippe de Commynes, nas suas \emph{Memóires}
  (VII, 14): ``\emph{les Françoys y sont alléz avecques des esperons de
  boys et de la craye en la main des fourriers pour marcher leurs logis,
  sans aultre peyne}''. {[}``Os franceses partiram com esporas de
  madeiras e com giz na mão dos entendentes para marcar as suas
  habitações, sem qualquer outro castigo''{]}. Com o giz eram marcados
  os edifícios das várias cidades da Itália destinadas ao alojamento dos
  franceses. Este teria sido o único trabalho dos franceses na
  conquistada Itália. Confira, Borsellino, N. \emph{Niccolò
  Machiavelli}, in \emph{Letteratura Italiana}. Bari: Laterza, 1973.
  Vol. 4. t.1, pp. 35-180.}; e quem dizia\footnote{Alusão ao frei
  Jerônimo Savonarola que pregou sobre a ruína de Florença no ano de
  1494 como sendo causada pela fraqueza dos próprios florentinos.} que a
causa disto eram os nossos erros dizia a verdade, embora não se tratasse
daqueles que acreditavam ser, mas destes que eu narrei; e porque eram
erros dos príncipes, eles também sofreram as suas penas.

{[}10{]} Vou demonstrar melhor a infelicidade destas armas. Os capitães
mercenários ou são homens excelentes ou não; se o são, não se pode
confiar neles, porque ou sempre aspirarão à própria grandeza ou vão
oprimi-lo, você que é seu patrão, ou oprimirão outros para além de sua
intenção. Porém, se o capitão não é virtuoso, normalmente o arruína.
{[}11{]} E se alguém disser que qualquer um fará isso, tendo as armas em
mão, seja ele mercenário ou não, replicarei que as armas devem ser
manejadas ou por um príncipe ou por uma república: o príncipe deve ir
pessoalmente e fazer ele mesmo o ofício de capitão; a república tem de
mandar um cidadão seu e quando manda alguém que não seja um homem
valente, deve trocá-lo; quando for o caso, detê-lo com as leis, para que
não ultrapasse o prescrito. {[}12{]} E por experiência se vê apenas os
príncipes e as repúblicas armadas fazerem grandíssimos progressos, e as
armas mercenárias não causarem senão danos; e com muito mais dificuldade
se submete à obediência de um só cidadão uma república armada de armas
próprias, do que uma república armada de armas externas\footnote{Digna
  de nota a remissão, neste trecho, às repúblicas e à necessidade que
  tenham forças militares próprias e não mercenárias. O que evidencia a
  necessidade da força militar para a constituição de qualquer poder
  político que se pretenda autônomo.}.

{[}13{]} Estiveram Roma e Esparta por muitos séculos armadas e livres.
Os suíços são armadíssimos e libérrimos. {[}14{]} Das armas mercenárias
antigas são exemplo os cartagineses, que foram oprimidos por seus
soldados mercenários, terminada a primeira guerra com os
romanos\footnote{Primeira Guerra Púnica (241 a 237 a.C.).}, embora os
cartagineses tivessem por chefe seus próprios cidadãos. {[}15{]} Felipe
da Macedônia\footnote{Rei da Macedônia de 359 a 336 a. C., pai de
  Alexandre, o grande.} foi feito pelos tebanos capitão de seus
exércitos, depois da morte de Epaminondas, e tirou deles, depois da
vitória, a liberdade.

{[}16{]} Os milaneses, morto o duque Felipe, assoldadaram\footnote{Apesar
  de ser pouco utilizado, o verbo \emph{assoldadar} confere uma
  significação precisa ao texto. Assoldadar significa contratar alguém
  por meio de \emph{soldo}, cujo sujeito que recebe é o \emph{soldado.}
  A família terminológica aqui sugerida está, justamente, no centro da
  argumentação maquiaveliana, já que, o que se discute é o grau de
  fidelidade ou infidelidade das armas mercenárias, ou o seu grau de
  adesão, de solda, de união ao governo pagante.} Francisco Sforza
contra os venezianos\footnote{Batalha de Caravaggio de setembro de 1448.},
o qual, superados os inimigos em Caravaggio, se associou a eles para
oprimir os milaneses, seus patrões. {[}17{]} Sforza, seu pai\footnote{Muzio
  Attendolo Sforza (1369-1424), grande comandante militar italiano que
  esteve a serviço da rainha Giovanna de Napoli ou Giovanna II,
  rebelando-se em 1420 contra ela e se aliando a Luiz III D'Anjou. Esse
  episódio também é narrado na \emph{Arte da Guerra,} livro I, e nas
  \emph{História de Florença,} livro I, cap. 38.}, sendo soldado da
rainha Giovanna de Nápoles, deixou-a, subitamente, desarmada; e ela,
para não perder o reino, foi constrangida a se atirar no colo do rei de
Aragão. {[}18{]} E se venezianos e florentinos, entretanto, acresceram
os seus impérios com estas armas, e se os seus capitães, porém, não
foram feitos príncipes, mas os defenderam, respondo que os florentinos,
neste caso, foram favorecidos pela sorte\footnote{Um dos poucos casos em
  que Maquiavel usa o termo \emph{sorte} e não \emph{fortuna,} como
  seria conveniente.}: porque, dos capitães virtuosos que podiam temer,
alguns não venceram, alguns enfrentaram oposição, outros voltaram as
suas ambições para outro lugar. {[}19{]} Aquele que não venceu foi
Giovanni Aucut\footnote{John Hawkwood, comandante inglês que de 1390 a
  1392 comandou as tropas florentinas contra as tropas milanesas de Gian
  Galeazzo Visconti.}, de quem, não vencendo, não se podia conhecer a
lealdade, embora qualquer um concorde que, se ele vencesse, estavam os
florentinos à sua mercê. {[}20{]} Sforza teve sempre os
Brancceshi\footnote{Como era conhecida a Companhia militar de Andréa
  Fortebracci, dito Braccio da Montone (1368- 1424), que esteve
  diretamente envolvido nos acontecimentos relativos a Nápoles.} por
antagonistas, que vigiavam um ao outro. {[}21{]} Francisco voltou a sua
ambição para a Lombardia; Braccio\footnote{No caso Braccio da Montone,
  que assumiu o lugar de Muzio Sforza e esteve a serviço de Giovanna II
  e Alfonso de Aragão.} contra a Igreja e o reino de Nápoles. {[}22{]}
Mas vejamos o que aconteceu há pouco tempo. Fizeram os florentinos Paulo
Vitelli\footnote{Confira cap. VIII, 13.} seu capitão, homem
prudentíssimo e que, a partir da fortuna privada, havia conquistado
grande reputação. Se ele tomasse Pisa, ninguém negaria que conviria aos
florentinos tê-lo consigo, porque, se ele se convertesse em soldado dos
inimigos dos florentinos, não teriam remédio; se os florentinos o
mantivessem ao seu lado, teriam de obedecê-lo.

{[}23{]} Os venezianos, se se considera os seus progressos, ver-se-á que
agiram seguramente e gloriosamente enquanto fizeram a guerra com suas
próprias armas -- o que ocorreu antes de empreenderem suas conquistas em
terra --, quando, com os gentis homens e com a plebe armada, agiram de
maneira muito virtuosa. Contudo, quando começaram a combater em terra,
deixaram esta \emph{virtù} e seguiram os costumes das guerras da
Itália\footnote{Entre os séculos IX e XIII os venezianos fizeram uma
  grande expansão marítima, conquistando vários territórios ao longo do
  mar Adriático e Egeu, tornando-se uma grande potência econômica e
  militar. Todavia, ao tentar fazer uma expansão sobre os territórios do
  norte da Itália, principalmente sobre a região do Veneto, essa força
  conquistadora não se verificou, pois, conforme denunciado por
  Maquiavel, não se valeram de seus próprios exércitos, mas de milícias
  mercenárias contratadas.}. {[}24{]} E no princípio da sua expansão por
terra, por não ter nela muitos poderosos e por terem grande reputação,
não tinham muito do que temer os seus capitães. {[}25{]} Mas, quando
eles aumentaram seu poder em terra, o que ocorreu sob o governo de
Carminhola\footnote{Francesco Bussone, conde de Castelnuovo Scrivia
  (1340 a 1432), comandante militar que no século XV obteve várias
  conquistas em favor dos venezianos.}, tiveram uma lição deste erro:
porque, vendo-o poderosíssimo, após ter batido, sob seu comando, o duque
de Milão, e vendo, por outro lado, como ele estava arrefecendo na
guerra, julgaram não poder mais vencer com ele, porque não desejavam;
nem podiam liberá-lo, para não perderem novamente aquilo que haviam
conquistado; foi- lhes necessário, para sua segurança, matá-lo. {[}26{]}
Tiveram, depois, como seus capitães Bartolomeu de Bergamo\footnote{Bartolomeu
  Colleoni (1400-1475), derrotado na Batalha de Caravaggio.}, Ruberto de
San Severino\footnote{Roberto de San Severino (1418-1487), comandante
  veneziano na guerra contra Ferrara (1484).} o Conde de
Pitiglino\footnote{Nicollò Orsini (1442-1510), comandante na batalha de
  Vailate (maio de 1509), na qual os venezianos foram derrotados.} e
semelhantes, com os quais temiam a derrota e não os seus ganhos, como
ocorreu depois em Vailá\footnote{Ou Vailate.}, onde, em uma jornada,
perderam aquilo que em oitocentos\footnote{Aqui Maquiavel faz alusão a
  um período de tempo que não é exato, mas que compõe aquilo que ficou
  conhecido como o ``mito de Veneza'', principalmente em sua semelhança
  com Esparta. Afirmava-se no Renascimento italiano que a república
  espartana havia durado 800 anos, sendo a mais longeva da história. As
  mesmas qualidades de Esparta eram atribuídas a Veneza, e, neste caso,
  Maquiavel comete o equívoco de atribuir uma temporalidade que não é de
  Veneza, mas da mitologia sobre Esparta, o que evidencia o recurso
  retórico de falsa exaltação, pois pretende-se ressaltar o seu
  fracasso.} anos haviam conquistado com tanto esforço: porque destas
armas nascem somente as conquistas lentas, tardias e débeis e as
derrotas rapidíssimas e extraordinárias.

{[}27{]} E porque eu vim parar com estes exemplos na Itália, que foi
governada por muitos anos pelas armas mercenárias, gostaria de discorrer
sobre elas ainda um pouco mais, para que vendo a origem e o progresso
delas, se possa melhor corrigi-las. {[}28{]} Você tem de entender,
portanto, como o Império\footnote{Isto é, o Sacro-Império Romano
  Germânico.} rapidamente começou, nestes últimos tempos, a ser expulso
da Itália e como o papa ganhou mais poder temporal, e a Itália se
dividiu em vários estados porque muitas das cidades grandes pegaram em
armas contra os seus nobres -- os quais, primeiramente favorecidos pelo
imperador, as oprimira-nas --, e a Igreja as favoreceu para dar a si
poder no plano temporal; em muitas outras cidades seus cidadãos se
tornaram príncipes\footnote{Conforme se disse, no cap. IX, dos cidadãos
  comuns que se tornaram príncipes.}. {[}29{]} Assim, tendo a Itália
quase caído nas mãos da Igreja e de algumas repúblicas, e sendo aqueles
padres e aqueles outros cidadãos não habituados ao uso das armas,
começaram a pagar forasteiros. {[}30{]} O primeiro que deu força a este
tipo de milícia foi Alberigo di Conio\footnote{Alberico da Barbiano,
  conde de Cunio, foi o primeiro a constituir uma companhia militar.},
da Romanha: da escola deste descende, entre outros, Braccio e Sforza,
que nos seus tempos foram árbitros da Itália. {[}31{]} Depois destes
vieram todos os outros, que até os nossos tempos têm comandado estas
armas, e o fim da sua \emph{virtù} foi a Itália ter sido devastada por
Carlos, saqueada por Luiz, subjugada por Fernando e vituperada pelos
suíços.

{[}32{]} As táticas que eles seguiram serviram, primeiramente, para dar
reputação a si próprios e tirar a reputação da infantaria; fizeram isto
porque, não tendo um status e vivendo do seu trabalho, poucos infantes
não lhes davam reputação e não poderiam alimentar a muitos; porém, por
isso se limitaram à cavalaria, com a qual, com um número suportável,
estavam bem armados e eram honrados; e as coisas eram reduzidas a tal
ponto que, em um exército de vinte mil soldados, não se encontravam dois
mil infantes\footnote{O destaque para a infantaria nos exércitos faz
  parte de um pressuposto central do pensamento político e militar de
  Maquiavel que entende que o soldado, mais especificamente o
  cidadão-soldado, é o fundamento primeiro da força militar. Esses
  comandantes militares, ao optarem por exércitos predominantemente
  compostos por cavalarianos, enfraqueciam esse elemento político
  central. Há que se destacar, ainda, que apesar de nesse momento já se
  conhecer e usar a pólvora, Maquiavel não faz referência ou atribui
  qualquer importância à artilharia em seus escritos. Esse elemento
  revolucionário das técnicas militares do século XVI não é destacado
  por ele, que pensa as forças militares no interior do quadro da
  dinâmica política.}. {[}33{]} Além disso, haviam usado todos os meios
para isentarem a si e aos seus soldados do cansaço e do medo, não se
matavam nos combates, os prisioneiros eram libertados sem resgate, não
atacavam à noite as cidades amuralhadas\footnote{\emph{Terre} são
  cidades contornadas por uma muralha, como se vê, ainda hoje, em várias
  cidades da Itália, por exemplo Lucca, na Toscana.}, os sitiados não
saiam das muralhas para atacar os assediantes; não faziam, no entorno
dos campos, nem as paliçadas nem as fossas; não guerreavam durante
inverno. {[}34{]} E todas estas coisas eram permitidas nas suas táticas
militares e forjadas por eles para fugir, como foi dito, da fadiga e dos
perigos: tanto que foram eles que tornaram a Itália escrava e vituperada

\quebra\section{\emph{DE MILITIBUS AUXILIARIIS, MIXTIS ET PROPRIIS}
{[}Das milícias auxiliares, mistas e próprias{]}}

{[}1{]} As armas auxiliares, que são outras armas inúteis, são as que se
têm quando se chama um poderoso que, com as suas armas, vem defende-lo,
como fez em tempos recentes o papa Júlio: o qual, tendo visto na empresa
de Ferrara a triste prova das suas armas mercenárias, voltou-se às
auxiliares e fez um acordo com Fernando\footnote{Fernando, fazendo parte
  da ``Santa Liga'' (Espanha, Veneza e Igreja), ajudou o papa a
  conquistar Ferrara em 11 outubro de 1511.}, rei de Espanha, para que
ele, com a sua gente e os seus exércitos\footnote{Convém notar que
  Maquiavel dispõe de um termo próprio para os exércitos
  (\emph{eserciti}), mas não o usacom frequência, preferindo o termo
  \emph{arma}, o que revela uma compreensão mais restrita do primeiro e
  mais ampla do segundo, uma vez que este inclui exércitos, milícias e
  outros tipos de forças armadas.}, tivesse de ajudá-lo. {[}2{]} Estas
armas podem ser úteis e boas para si mesmas, mas são, para quem as
chama, quase sempre danosas, porque, perdendo, você permanece derrotado,
vencendo, fica prisioneiro delas. {[}3{]} Embora as histórias antigas
estejam repletas destes exemplos, não desejo, todavia, afastar-me do
exemplo recente do papa Júlio II, cuja decisão não poderia ser mais
insensata, por querer Ferrara, ficou completamente nas mãos de um
forasteiro. {[}4{]} Mas a sua boa fortuna fez nascer um terceiro fator,
a fim de que não colhesse o fruto de sua má escolha, porque, sendo os
seus auxiliares derrotados em Ravena\footnote{Batalha de Ravena, 11 de
  abril de 1512, onde a ``Santa Liga'' é derrotada pelos exércitos
  franceses.} e aparecendo os suíços que expulsaram os vencedores,
contra toda expectativa sua e de outros, não permaneceu prisioneiro nem
dos inimigos, postos em fuga, nem dos seus auxiliares, tendo vencido com
armas de outros e não com as suas. {[}5{]} Os florentinos, estando
completamente desarmados, conduziram dez mil franceses a Pisa\footnote{Retomada
  de Pisa em maio de 1500.} para tomá-la de assalto, decisão que
comportou mais perigos do que em qualquer outro momento de suas
dificuldades. {[}6{]} O imperador de Constantinopla\footnote{Giovanni VI
  Cantacuzeno reinou de 1347 a 1355.}, para se opor aos seus vizinhos,
colocou na Grécia\footnote{Fazer nota sobre o que era a Grécia antes.}
dez mil turcos, os quais, terminada a guerra, não quiseram partir, o que
foi o princípio da servidão da Grécia\footnote{Grécia, aqui, entendida
  como a região da Península Ática, que compreende mais territórios que
  a Grécia do século XXI.} aos infiéis.

{[}7{]} Aquele, portanto, que não deseja vencer, que se valha destas
armas, porque são muito mais perigosas que as mercenárias. {[}8{]}
Porque, com as armas auxiliares, a ruína é certa: são todas unidas,
voltadas à obediência dos outros; mas as mercenárias, uma vez que tenham
vencido, para prejudicarem-no precisam de mais tempo e de melhores
ocasiões, porque não sendo um só corpo e sendo contratadas e pagas por
você, um terceiro que você eleve ao comando, não pode ganhar rapidamente
tanta autoridade que o prejudique. {[}9{]} Em suma, nas mercenárias, é
mais perigosa a indolência, nas auxiliares, a \emph{virtù}. {[}10{]} Um
príncipe sábio, portanto, sempre fugiu destas armas e voltou- se às
próprias, e preferiu, antes, perder com as suas do que vencer com as
outras, julgando não verdadeira a vitória que se conquista com as armas
alheias.

{[}11{]} Não hesitarei nunca de dar como exemplo César Borgia e suas
ações. Este duque entrou na Romanha com as armas auxiliares, conduzindo
todas as tropas francesas, e com elas tomou Imola e Furli. Porém, não
lhe parecendo, pois, tais armas seguras, voltou-se às mercenárias, e
pagou os Orsini e os Vitelli, cujas armas, depois, ao manobrá-las,
julgou dúbias, infiéis e perigosas, eliminou-as e voltou-se para as
armas próprias. {[}12{]} E pode-se facilmente ver que diferença há entre
uma e outra dessas armas, considerando a diferença de reputação do duque
quando apenas tinha os franceses, quando tinha os Orsini e os Vitelli e
quando ficou com os seus soldados e consigo mesmo: sempre se encontrará
seu poder engrandecido, e nem foi muito estimado, senão quando todos
viram que ele controlava completamente suas armas.

{[}13{]} Eu não quero me distanciar dos exemplos italianos e recentes,
como não deixarei para trás Hierão de Siracusa, sendo um dos acima
nomeados por mim\footnote{Confira cap. VI, 26-28.}. {[}14{]} Este, como
eu disse, feito chefe dos exércitos dos siracusanos, percebeu
imediatamente que aquela milícia mercenária não era útil, por serem os
chefes militares como os nossos italianos. Parecendo-lhe não poder nem
conservar tais armas e nem deixá-las, ele fez com que ela fossem
destroçadas e depois fez guerra com as suas armas e não com as alheias.
{[}15{]} Quero ainda trazer à memória uma figura do Velho Testamento,
que vem a propósito. {[}16{]} Oferecendo-se Davi a Saul para ir combater
com Golias, provocador filisteu\footnote{Cf. I Samuel, 17.}, Saul, para
lhe dar coragem, o armou com suas próprias armas. Assim que Davi as
vestiu, recusou-as, dizendo que com aquelas não se podia valer-se bem de
suas próprias qualidades, e por isso desejava enfrentar o inimigo com a
sua funda e a sua faca. {[}17{]} Enfim, as armas dos outros ou lhe caem
por terra, ou lhe pesam ou lhe apertam.

{[}18{]} Carlos VII\footnote{Reinou de 1422 a 1461.}, pai do rei Luiz
XI, tendo com a sua fortuna e \emph{virtù} libertado a França dos
ingleses\footnote{Com a ``Guerra dos Cem Anos'', que terminou em 1452.},
reconheceu a necessidade de se armar de armas próprias e ordenou no seu
reino o recrutamento da cavalaria e da infantaria. {[}19{]} Depois o rei
Luiz\footnote{Luiz XI, rei de 1461 a 1483.}, seu filho, extinguiu o
recrutamento da infantaria e contratou os soldados suíços, erro que,
seguido por outros, é, como de fato se vê agora, a razão dos perigos
daquele reino. {[}20{]} Porque, tendo dado reputação aos suíços, aviltou
todas as suas armas, tendo em vista que eliminou toda a infantaria e
tornou sua cavalaria dependente da \emph{virtù} de outros, porque, sendo
acostumadas a combaterem com os suíços, não lhe parecia poder vencer sem
eles. {[}21{]} É por isso que os franceses não bastam contra os suíços e
não se colocam à prova contra os outros sem os suíços. {[}22{]} Foram,
portanto, exércitos mistos os da França, em parte mercenário e em parte
próprio: estas armas, todas juntas, são muito melhores do que as
auxiliares sozinhas ou as mercenárias sozinhas, e muito inferior às
próprias. {[}23{]} E que basta o exemplo dado, porque o reino de França
seria insuperável se o recrutamento de Carlos fosse aumentado ou
preservado; mas a pouca prudência dos homens começa uma coisa que, por
parecer então boa, não se percebe o veneno que tem debaixo, assim como
eu disse acerca da febre tísica\footnote{Ou tuberculose, confira cap.
  III, 27.}. {[}24{]} Portanto, aquele que em um principado não conhece
os males quando nascem, não é verdadeiramente sábio, e isso é atributo
de poucos. {[}25{]} E, caso se considerasse a principal causa da ruína
do Império Romano, encontrar-se-ia ter sido somente o começar a pagar os
mercenários godos\footnote{Trata-se da incorporação de 40.000 visigodos
  ao exército romano realizada, uma primeira vez pelo imperador Valente
  em 376 d.C. e, uma segunda vez, pelo imperador Teodósio em 382 d.C.},
porque daquele princípio as forças do império romano começaram a se
debilitar, e toda aquela \emph{virtù} que se extraía dele dava- se aos
godos.

{[}26{]} Concluo, portanto, que sem ter armas próprias, nenhum
principado está seguro. Aliás, fica completamente dependente da fortuna,
não tendo \emph{virtù} que com confiança o defenda na adversidade. E foi
sempre a opinião e a sentença dos homens sábios, \emph{quod nihil sit
tam infirmum aut instabile quam fama potentiae non sua vi
nixa}\footnote{``Não há nada de mais instável e frágil do que a fama de
  uma potência que não se apóia na própria força''. Tácito, Anais, XIII,
  19. Citado de memória por Maquiavel, mas com o mesmo sentido do texto
  original. O texto original é: ``Nihil rerum mortalium tam instabile ac
  fluxum est quam fama potentiae non sua vi nixae.''}. {[}27{]} As armas
próprias são aquelas compostas ou pelos súditos, ou pelos cidadãos, ou
pelos seus criados: todas as outras são ou mercenárias ou auxiliares; e
o modo para ordenar as armas próprias será fácil de se encontrar se
examinar os ordenamentos dos quatro acima nomeados\footnote{Há nesta
  passagem uma discordância entre os comentadores sobre quais seriam
  esses quatro exemplos, pois duas hipóteses se apresentam e ambas
  excluem os romanos, que foram sempre elogiados por Maquiavel por
  possuírem um exército próprio. A primeira hipótese nos remete ao
  capítulo VI, quando ele fala de Moisés, Rômulo, Ciro e Teseu. A
  segunda hipótese, recuperada a partir dos exemplos citados neste
  capítulo, seriam Cesar Bórgia, Hierão, Carlos VII e Davi. O problema
  dessa última hipótese é que Davi não é apresentado aqui como detentor
  de um exército próprio, mas apenas de armamento próprio. Martelli
  formula uma outra hipótese mais controversa, sugerindo que Maquiavel
  havia escrito uma outra obra, na qual cita explicitamente quatro
  personagens ou quatro povos. (Martelli, \emph{edição comentada}, p.
  209, nota 48). Seja como for, não é possível saber exatamente quem são
  esses quatro exemplos mencionados por Maquiavel.} por mim, e se verá
como Felipe, pai de Alexandre Magno, e muitas repúblicas e príncipes se
armaram e ordenaram, cujos ordenamentos eu me remeto em tudo.

\quebra\section{\emph{QUOD PRINCIPEM DECEAT CIRCA MILITIAM}
{[}O que compete a um príncipe no que diz respeito às milícias{]}}

{[}1{]} Deve, pois, um príncipe não ter outro objetivo nem outro
pensamento, nem tomar coisa alguma como arte sua que não seja a guerra,
a organização e a disciplina desta, porque apenas ela é a arte que se
espera de quem comanda e tem tamanha \emph{virtù,} que não somente
conserva aqueles que nasceram príncipes, mas, muitas vezes, faz com que
os homens de fortuna privada alcancem aquele posto. {[}2{]} E, ao
contrário, se vê que, quando os príncipes pensaram mais nas delicadezas
do que nas armas, perderam o seu status, e a principal razão que faz
você perdê-lo é negligenciar esta arte; e a razão que faz você
conquistá-lo é ser professor nesta arte. {[}3{]} Francisco Sforza, por
estar armado, de pessoa comum tornou-se duque de Milão; e os seus
filhos\footnote{Francisco Sforza teve três filhos: Galeazzo Maria,
  Ludovico Moro e Ascânio. O primeiro foi morto numa conjuração em 1476,
  o segundo perde o ducado em 1499-1500 e Ascânio tornou-se cardeal e
  morreu em 1505. Apesar desses fatos, em 1513, quando Maquiavel está
  escrevendo \emph{O Príncipe}, o ducado de Milão está ainda nas mãos da
  família Sforza, com Ercole Massimiliano, filho de Ludovido Moro e,
  portanto, neto de Francisco Sforza. Ora, esses fatos parecem indicar
  um equívoco de Maquiavel. Contudo, desde 1924, Federico Chabod, e,
  depois, toda uma série de outros comentadores que se seguiram, são
  concordes em interpretar essa passagem como sendo uma referência a
  Ludovico Moro principalmente-- este sim que perdeu o governo da cidade
  de Milão por não ter armas próprias e nem saber guerrear --, apesar do
  plural empregado remeter aos outros filhos. Enfim, seja principalmente
  Ludovico, mas também Galeazzo e Ascânio, o fato é que os filhos de
  Francisco Sforza não conservaram o governo da cidade, apesar de seu
  neto retomá-lo depois. De qualquer modo, Maquiavel não foi muito
  preciso em sua referência histórica.}, para fugirem dos incômodos das
armas, de duques tornaram-se pessoas comuns. {[}4{]} Porque, entre
outros males que lhe acarretam, estar desarmado torná-o desprezível, e
isso é uma daquelas infâmias das quais o príncipe deve se guardar, como
adiante se dirá\footnote{Sobretudo no cap. XIX.}. {[}5{]} Porque não há
proporção alguma entre um príncipe armado e um desarmado, e não é
razoável que quem está armado obedeça voluntariamente a quem está
desarmado, e que o desarmado esteja seguro entre serviçais armados,
pois, havendo em um o desprezo e noutro a suspeita, não é possível
agirem bem juntos. {[}6{]} Por isso que um príncipe que não entenda de
milícia, além das outras infelicidades, como foi dito, não pode ser
estimado pelos seus soldados nem fiar- se neles.

{[}7{]} Portanto, nunca deve desviar o pensamento destes exercícios da
guerra, e na paz deve mais exercitá-los que na guerra, o que pode fazer
de dois modos: um com o agir, outro com a mente. {[}8{]} E, quanto ao
agir, além de ter bem organizados e exercitados os seus, deve sempre ir
às caçadas, e, mediante estas, acostumar o corpo aos incômodos e,
paralelamente, apreender a natureza do lugar, conhecer como se erguem os
montes, como se abrem os vales, como se estendem as planícies, entender
a natureza dos rios e dos pântanos, e nisso tudo por grandíssima
atenção. {[}9{]} Tal conhecimento é útil de dois modos: primeiro,
aprende-se a conhecer seu próprio território, e se pode melhor entender
a defesa dele; depois, mediante o conhecimento e a prática nesses
lugares, compreende-se com facilidade todos os outros lugares que
novamente será necessário estudar, porque as colinas, os vales, as
planícies, os rios, os pântanos, que estão, por exemplo, na Toscana, têm
certa semelhança com os de outras províncias, de tal modo que, do
conhecimento do lugar de uma província, pode-se facilmente chegar ao
conhecimento de outra. {[}10{]} E aquele príncipe que é desprovido desta
perícia é desprovido da primeira qualidade que um capitão deve possuir
porque esta lhe ensina a encontrar o inimigo, escolher o melhor lugar
para os alojamentos, conduzir os exércitos, preparar o plano de batalha,
assediar as cidades com vantagem para você.

{[}11{]} Filopomene\footnote{Filopemene de Megalópolis (252 a 184 a.C),
  estrategista da Liga Aqueia.}, príncipe dos Aqueus, entre as outras
glórias que lhe foram dadas pelos escritores\footnote{Tito Lívio, livro
  XXXV, cap. 28.}, há aquela que nos tempos de paz só pensava nos modos
de fazer guerra; e quando estava em campanha, com frequência se detinha
com os amigos e raciocinava com eles: {[}12{]} ``Se os inimigos
estivessem sobre aquelas colinas, e nós nos encontrássemos aqui com os
nossos exércitos, quem de nós teria vantagem? Como se poderia, mantendo
as formações, ir encontrá-las? Se nós quiséssemos bater em retirada,
como faríamos? Se eles batessem em retirada, como iríamos
segui-los?''\footnote{Essa é uma citação da \emph{História de Roma}, de
  Tito Lívio, indicada acima.} {[}13{]} E, assim, propunha a eles todos
os casos que podem ocorrer a um exército, ouvia a opinião deles, dizia a
sua, corroborava-as com os argumentos; tanto que, por estes contínuos
raciocínios, não podia nunca, guiando os exércitos, surgir algum
acidente para o qual ele não tivesse solução.

{[}14{]} Mas, quanto ao exercício da mente, deve o príncipe ler as
histórias\footnote{Aqui se evidencia dois requisitos que perfazem a
  metodologia proposta por Maquiavel desde o início da obra: experiência
  no agir e reflexão ou estudo da História. Esses são os dois
  fundamentos da reflexão política, que não se pode valer apenas da
  experiência e nem dos estudos somente. Esse pressuposto metodológico
  será reiterado no capítulo seguinte.} e nessas considerar as ações dos
homens excelentes, ver como se governaram nas guerras, examinar as
causas das suas vitórias e das suas derrotas, para poder fugir dessas e
imitar aquelas, e, sobretudo, fazer como fizeram antes alguns daqueles
homens excelentes, que tentaram imitar alguém que, antes deles, foi
louvado e glorificado, e cujos feitos e ações sempre mantiveram junto a
si: como se disse que Alexandre Magno imitava Aquiles; César, Alexandre;
Cipião, Ciro\footnote{Esses exemplos encontram-se em: Plutarco
  (\emph{Vida de Alexandre}, 8), Suetônio (\emph{Julio César}, 7) e
  Cícero (\emph{Ad Quintum fratem}, I, 23).}. {[}15{]} E alguém que leia
a vida de Ciro escrita por Xenofonte\footnote{A \emph{Ciropédia}.}
reconhece depois, na vida de Cipião\footnote{Trata-se aqui de Publio
  Cornélio Cipião Africano (235-183 a.C.), general romano que venceu o
  cartaginês Aníbal, na Segunda Guerra Púnica, na batalha de Zama, em 19
  de outurbro 202 a.C.}, o quanto aquela imitação lhe foi gloriosa, e
quanto, na castidade, na afabilidade, na humanidade, na liberalidade,
Cipião se conformou àquelas coisas que Xenofonte escreveu de Ciro.

{[}16{]} Modos semelhantes deve observar um príncipe sábio e nunca ficar
ocioso nos tempos pacíficos, mas, com habilidade, reunir fundos para
poder se valer deles nas adversidades, de modo que, quando a fortuna
muda, encontrar-lo-á pronto para resistir a ela.

\quebra\section{\emph{DE HIS REBUS QUIBUS HOMINES ET PRAESERTIM PRINCIPES LAUDANTUR AUT
VITUPERANTUR}
{[}Das coisas pelas quais os homens, e especialmente os príncipes, são
louvados ou vituperados{]}}

{[}1{]} Resta agora ver quais devem ser os modos e os atos de governo de
um príncipe para com os súditos ou para com os amigos. {[}2{]} E, porque
sei que muitos escreveram sobre isto\footnote{Esta é uma referência
  explícita de Maquiavel à tradição literária dos ``espelhos de
  príncipe'', que, em linhas gerais, concentrava-se em apresentar regras
  e máximas de condutas para o príncipe conduzir o povo, principalmente
  no que diz respeito às virtudes que ele deveria possuir ou adquirir.
  Nota-se, em seguida, que ele não se manterá nessa tradição, pois
  formulará uma análise diferente sobre as qualidades políticas do
  príncipe que não se encaixaram dentro do quadro tradicional da virtude
  cristã preconizada.}, temo, escrevendo eu também, ser considerado
presunçoso, sobretudo porque, ao debater esta matéria, afasto-me do modo
de raciocinar dos outros. {[}3{]} Mas, sendo a minha intenção escrever
coisa útil a quem a entenda, pareceu-me mais convincente ir direto à
verdade efetiva da coisa do que à imaginação dessa\footnote{A expressão
  \emph{verità effetuale della cosa} indica um dos fundamentos da
  reflexão política maquiaveliana, que não pretende se pautar pela
  imaginação ou idealização da política, mas se apóia diretamente na
  realidade da vida política.}. {[}4{]} E muitos imaginaram
repúblicas\footnote{Aqui uma referência direta à \emph{República}, de
  Platão.} e principados que nunca foram vistos, nem conhecidos de
verdade. {[}5{]} Porque há tanta diferença entre como se vive e como se
deveria viver, que quem deixa aquilo que se faz por aquilo que se
deveria fazer apreende mais rapidamente a sua ruína que a sua
preservação, porque um homem que deseja ser bom em todas as situações, é
inevitável que se destrua entre tantos que não são bons. {[}6{]} Assim,
é necessário a um príncipe que deseja conservar-se no poder, apreender a
não ser bom, e sê-lo e não sê-lo conforme a necessidade.

{[}7{]} Deixando, portanto, para trás as coisas imaginadas sobre um
príncipe e discorrendo sobre aquelas que são verdadeiras, digo que todos
os homens, quando falam deles, e mais ainda os príncipes, por estarem em
posição mais elevada, são tachados de algumas destas qualidades que
causam ou a sua ruína ou o seu louvor. {[}8{]} E assim é que um é tido
por liberal, outro por miserável -- usando um termo toscano, porque
``avaro'', em nossa língua, é também aquele que por roubo deseja ter e
``miserável'' chamamos àquele que se abstém muito de usar o que é seu
--; um é considerado pródigo, outro rapace; um cruel, outro piedoso;
{[}9{]} um não confiável, outro fiel; um efeminado e pusilânime, outro
feroz e animoso; um humano, outro soberbo; um lascivo, outro casto; um
íntegro, outro astuto; um rijo, outro fácil; um sério, outro leviano; um
religioso, outro incrédulo e assim por diante. {[}10{]} Sei que todos
afirmaram que seria coisa louvabilíssima encontrar- se em um príncipe,
de todas as sobreditas qualidades, aquelas que são consideradas boas.
{[}11{]} Mas, porque não se podem ter todas, nem observá-las
inteiramente, por causa das condições humanas que não o consentem, é
necessário ser tão prudente que saiba evitar a infâmia\footnote{Neste
  capítulo Maquiavel anuncia um \emph{topus} central da sua exposição
  doravante, a saber: o modo como um príncipe deve lidar com a fama e a
  infâmia. Para isso, convém ter em mente que a fama e a infâmia são as
  percepções que os indivíduos têm de seu governante em função do efeito
  dos seus atos e não necessariamente dos atos em si praticados pelo
  príncipe. Essa dimensão dos efeitos da ação e da não ação é
  propriamente o ponto central da argumentação maquiaveliana, que é um
  dos primeiros a chamar a atenção para esse aspecto do agir político: o
  seu efeito sobre os outros. Não devemos nos esquecer que Maquiavel era
  um teatrólogo; na verdade, sua fama entre os florentinos era mais
  resultado do sucesso de suas peças de teatro do que propriamente dos
  seus escritos políticos. Sobre essa temática, cf. Adverse, Helton.
  \emph{Politica e Retórica em Maquivel}. Belo Horizonte: ed. UFMG,
  2009.} daqueles vícios que lhe tirariam o status, e se guardar, se lhe
é possível, daqueles que não lhe fariam perdê-lo. Mas não podendo, com
menos escrúpulo, pode-se deixar ir. {[}12{]} E ainda não se preocupe em
incorrer na infâmia daqueles vícios sem os quais dificilmente poderia
manter o status, porque, caso se considere tudo muito bem,
encontrar-se-á algo que parece \emph{virtù}, e que, seguindo-a, seria a
sua ruína, e alguma outra que parece vício, e seguindo-a consegue a
segurança e o seu bem-estar.

\quebra\section{\emph{DE LIBERALITATE ET PARSIMONIA}
{[}Da liberalidade e da parcimônia{]}}

{[}1{]} Começando, então, pelas primeiras sobreditas qualidades, digo
que seria bom ser considerado liberal. {[}2{]} Contudo, a liberalidade
praticada de modo que você sejas reconhecido como tal, prejudica-o,
porque, se ela é praticada virtuosamente e como se deve praticá-la, não
lhe será atribuida e não lhe recairá a infâmia do seu contrário; porém,
querendo conservar entre os homens a fama de liberal, é necessário não
deixar de lado nenhuma manifestação de sua suntuosidade, de tal modo que
um príncipe liberal sempre consumirá em semelhantes obras todas as suas
riquezas; {[}3{]} e será necessário, por fim, se quiser conservar a fama
de liberal, tributar o povo extraordinariamente, cobrar impostos e fazer
todas aquelas coisas que se fazem para obter dinheiro, o que começará a
fazê-lo odioso aos súditos ou pouco estimado por todos, pois ficarão
pobres. {[}4{]} De modo que, com esta sua liberalidade, tendo ofendido
muitos e premiado poucos, sente logo qualquer incipiente contrariedade,
e periclita logo no primeiro perigo: percebendo isso e desejando voltar
atrás, incorre, imediatamente, na infâmia de miserável. {[}5{]}
Portanto, se um príncipe não puder usar desta \emph{virtù} de liberal
sem dano para si, para que ela lhe seja reconhecida, deve, se é
prudente, não se importar com a fama de miserável, porque, com o tempo,
será considerado cada vez mais liberal, ao verem que, com a sua
parcimônia, suas receitas lhe bastam, podendo defender-se daqueles que
lhe fazem guerra, podendo fazer obras sem tributar o povo. {[}6{]} De
tal modo que se torna liberal para todos aqueles dos quais não subtrai,
que são infinitos, e miserável para todos aqueles a quem não dá, que são
poucos.

{[}7{]} No nosso tempo não vimos fazer grandes coisas senão aqueles que
foram considerados miseráveis, os outros foram extintos. {[}8{]} O papa
Júlio II, ao se servir da fama de liberal para conseguir o papado, não
pensou, pois, em conservar essa fama, para poder guerrear. {[}9{]} O
atual rei de França\footnote{Luiz XII. Essa expressão é um dado
  fundamental para determinar o limite temporal de quando foi escrito
  \emph{O Príncipe}, no caso, antes de 31 de dezembro de 1514, data da
  morte do rei. Ou seja, o texto estava concluído antes desta data,
  senão Maquiavel teria retirado a expressão ``o atual rei''.} fez
tantas guerras sem impor um imposto extraordinário aos seus, somente
porque administrou as despesas supérfluas com sua grande parcimônia.
{[}10{]} O atual rei de Espanha\footnote{Fernando, o católico.}, se
fosse tomado por liberal, não teria nem vencido nem feito tantas obras.
{[}11{]} Portanto, um príncipe deve preocupar-se pouco -- para não ter
de roubar os súditos, para poder defender-se, para não se tornar pobre e
desprezível, para não ser forçado a tornar-se rapace --, de ter a fama
de miserável, porque este é um daqueles vícios que o faz reinar.

{[}12{]} E se alguém dissesse: César\footnote{Caio Júlio César (100 a 44
  a.C).} com a liberalidade ganhou o império, e muitos outros, por terem
sido e serem considerados liberais, ganharam postos altíssimos,
respondo: ou você já é príncipe ou você está em vias de conquistar o
principado. {[}13{]} No primeiro caso, essa liberalidade é danosa. No
segundo, é bem necessário ser considerado liberal; e César era um dos
que queriam obter o principado de Roma, mas, depois que o alcançou, se
tivesse sobrevivido e não fosse moderado naquelas despesas, teria
destruído aquele império\footnote{Maquiavel faz remissão aqui aos fatos
  ocorridos no ano 44 a.C., quando Julio César entra com suas tropas em
  Roma e exige ser coroado como príncipe dos romanos e sumo pontífice,
  mas é assassinado em 15 de março no senado com punhaladas desferidas
  pelos senadores romanos.}.

{[}14{]} E se alguém replicasse: muitos foram os príncipes considerados
liberalíssimos, que com o exército fizeram grandes coisas, respondo: ou
o príncipe gasta do seu e dos seus súditos, ou de outros. {[}15{]} No
primeiro caso, deve ser comedida. No segundo, não deve deixar de lado
nenhuma manifestação de liberalidade. {[}16{]} E aquele príncipe que vai
com os exércitos, que se alimenta das pilhagens, dos saques e dos
roubos, lida com o que é dos outros; a ele é necessária essa
liberalidade, pois, de outro modo, não seria seguido pelos soldados.
{[}17{]} E, daquilo que não é seu ou dos seus súditos, se pode ser maior
doador, como o foi Ciro, César e Alexandre: porque gastar o que é dos
outros não tira sua reputação, ao contrário, agrega-lhe mais; somente
gastar do que é seu é que o prejudica. {[}18{]} E não há nada que
consuma tanto a si mesma quanto a liberalidade, enquanto você a usa,
perde a faculdade de usá-la e se torna ou pobre ou desprezível, ou, para
fugir da pobreza, rapace e odioso. {[}19{]} E, entre todas as coisas de
que um príncipe se deve guardar, está o ser temido e odiado, e a
liberalidade o conduz a uma coisa e outra. {[}20{]} Portanto, é mais
sábio ter a fama de miserável, que gera uma infâmia sem ódio, do que,
por querer a fama de liberal, ser necessário ter a fama de ladrão, que
gera uma infâmia com ódio.

\quebra\section{\emph{DE CRUDELITATE ET PIETATE; ET AN SIT MELIUS AMARI QUAM TIMERI, VEL
E CONTRA}
{[}Da crueldade e da piedade; e se é melhor ser amado do que temido ou o
contrário{]}}

{[}1{]} Examinando as outras qualidades apresentadas acima, digo que
todo príncipe deve desejar ser tido por piedoso e não cruel, todavia,
deve cuidar em não usar mal esta piedade. {[}2{]} César Borgia era tido
por cruel, não obstante, aquela sua crueldade pacificou a Romanha,
uniu-a, reconduzindo-a à paz e à confiança. {[}3{]} Se se considerar
bem, se verá que ele foi muito mais piedoso que o povo florentino, o
qual, para fugir da fama de cruel, deixou que destruíssem
Pistóia\footnote{A cidade de Pistóia foi controlada pelos florentinos
  desde 1328 e, ao longo do século XV, ficou dividida em dois partidos:
  um de apoiadores da família Médici e outro contrário, controlado pelo
  partido dos Panciatici. Em 1500 e 1501, diante da ameaça de César
  Bórgia em conquistar a cidade, os florentinos deixaram que ela caísse
  em uma guerra civil com a derrota dos Panciatici. Esses fatos, também
  analisados por Maquiavel em \emph{De rebus pistoriensibus} (de março
  de 1502), mostram como os florentinos foram inábeis no controle da
  cidade e permitiram que uma parte exercesse sua crueldade e força
  sobre a outra, sem que isso implicasse ao final na pacificação e
  unidade entre eles.}. {[}4{]} Um príncipe deve, portanto, não se
importar com a má fama de cruel para manter os seus súditos unidos e
confiantes, porque, com pouquíssimas punições exemplares, será mais
piedoso do que aqueles que, por excessiva piedade, deixam que avancem as
desordens, das quais nascem os assassínios ou os roubos; porque estes
costumam ofender uma comunidade inteira, e aquelas execuções, que partem
do príncipe, prejudicam a um particular. {[}5{]} E, entre todos os
príncipes, ao príncipe novo é impossível fugir da fama de cruel, por
serem os estados novos repletos de perigos. {[}6{]} E Virgílio, pela
boca de Dido, disse: \emph{Res dura, et regni novitas me talia cogunt
Moliri, et late fines custode tueri}\footnote{``A difícil circunstância
  e a novidade do meu reino me constrange a usar de tais modos e vigiar
  todas as partes até os confins.'' Eneida, I, 563-564.}. {[}7{]}
Contudo, deve ser prudente no crer e no se mover, não criando medos para
si mesmo, e proceder, de modo temperado, com prudência e humanidade, de
modo que a muita confiança não o faça incauto e a muita desconfiança não
o torne intolerável.

{[}8{]} Nasce disto uma discussão: se é melhor ser amado que temido, ou
o contrário. {[}9{]} Responde-se que se gostaria de ser um e outro, mas,
porque é difícil tê-los juntos, é muito mais seguro ser temido que
amado, quando se deve ser desprovido de um dos dois. {[}10{]} Porque,
dos homens, pode-se dizer isto: que geralmente são ingratos, volúveis,
simuladores e dissimuladores, esquivos aos perigos, cobiçosos de ganho;
e enquanto os beneficia são todos seus: oferecem o sangue, os bens, a
vida, os filhos, como acima se disse\footnote{Confira cap. IX, 26.},
quando a necessidade está longe; mas, quando ela se avizinha de você,
revoltam-se, e aquele príncipe que está todo fundado em suas palavras,
encontrando-se despido de outra predisposição, arruína-se. {[}11{]}
Porque as amizades que se conquistam com dinheiro, e não com grandeza e
nobreza de alma, se compram, mas não se possuem e, quando necessárias,
não se podem gastá-las; e os homens têm menos respeito para ofender a
alguém que se faz amar, do que alguém que se faz temer, porque o amor é
mantido por um vínculo de obrigação, o qual, por serem os homens
decepcionantes\footnote{Uma das poucas ocasiões que Maquiavel emite um
  juízo sobre a natureza humana, coisa que ele fará também nos
  \emph{Discursos} (I, 3), o que não nos permite, contudo, afirmar que
  haja uma antropologia em seus escritos. Cf. Bignotto, N.
  \emph{Antropologia negativa em Maquiavel} in ANALYTICA, Rio de
  Janeiro, vol 12 nº 2, 2008, p. 77-100.}, é rompido em todas as
ocasiões úteis a si próprios, mas o temor mantém-se pelo medo da punição
que não o abandona nunca.

{[}12{]} Contudo, deve o príncipe fazer-se temer de tal modo que, se não
conquista o amor, evite o ódio, porque pode muito bem estarem juntos o
ser temido e o não ser odiado. {[}13{]} O que sempre acontecerá,
enquanto o príncipe se abstiver dos bens dos seus cidadãos, dos seus
súditos e das suas mulheres. E quando lhe for necessário atentar contra
a família de algum deles, fazê-lo desde que haja justificação
conveniente e causa manifesta. {[}14{]} Mas, sobretudo, deve abster-se
dos bens dos outros, porque os homens esquecem mais rápido a morte do
pai do que a perda do patrimônio; depois, nunca faltam motivos para
confiscar bens, e aqueles que começam a viver do roubo sempre encontram
motivos para tomar aquilo que é dos outros, e, inversamente, os motivos
para atentar contra a vida são mais raros e acabam mais depressa.

{[}15{]} Mas, quando o príncipe está com os exércitos e tem sob seu
comando uma multidão de soldados, então é totalmente necessário não se
importar com a fama de cruel, porque sem esta não houve nunca exército
unido nem disposto à ação alguma. {[}16{]} Entre as maravilhosas ações
de Aníbal\footnote{Comandante militar de Cartago, que guerreou contra os
  romanos de 221 a 202 a.C., chegando com os seus exércitos até os muros
  de Roma, mas que teve sua grande derrota na batalha de Zama, com os
  exércitos romanos comandados por Cipião {[}cf. nota 206{]}. Essas
  informações sobre Aníbal, Maquiavel retira da \emph{História de Roma}
  de Tito Lívio (XXI 4, 9 e XXVIII 12, 2-5)} enumera-se esta, na qual,
tendo um exército grandíssimo, composto por inúmeras raças de homens,
levou-os para guerrear em terras alheias, e não surgiu nunca nenhuma
distensão, nem entre eles, nem contra o príncipe, tanto na má como na
boa fortuna. {[}17{]} O que não pode nascer de outra coisa senão daquela
sua crueldade desumana, a qual, juntamente com as suas infinitas
\emph{virtù}, o fez sempre venerável e temível para os seus soldados.
{[}18{]} E sem aquela crueldade, que produziu este efeito, as suas
outras \emph{virtù} não lhe bastariam, e os escritores neste ponto são
pouco perspicazes: de uma parte admiram esta sua ação, de outra condenam
a principal razão desta.

{[}19{]} E que é verdade que as suas outras \emph{virtù} não teriam
bastado, pode-se comprovar por Cipião, exemplo raro não somente nos seus
tempos, mas em toda a memória dos fatos que se sabem, cujos exércitos na
Espanha se rebelaram, o que não aconte ceu por outro motivo senão pela
sua excessiva piedade, a qual tinha dado a seus soldados mais liberdade
do que convinha à disciplina militar. {[}20{]} Coisa que lhe foi
repreendida por Fábio Maximo\footnote{Quinto Fabio Maximo (275-203
  a.C.), general romano.} no Senado, sendo chamado por ele de corruptor
da milícia romana. {[}21{]} Os Locrenses\footnote{Os habitantes de Locri
  Epizefrii, na Calábria, sul da Itália.}, tendo sido arruinados por um
legado de Cipião, não foram por ele vingados, nem a insolência daquele
legado corrigida, nascendo tudo daquela sua natureza permissiva, tanto
que, querendo alguém desculpá-lo no Senado, disse que como ele havia
muitos homens que mais sabiam não errar do que corrigir os erros.
{[}22{]} Esta característica teria com o tempo desonrado a fama e a
glória de Cipião, se ele tivesse perseverado junto com ela no comando
militar\footnote{``Comando Militar'' aqui traduz ``\emph{imperio}" que
  em seu significado original denota, principalmente, o comando militar
  que se possui.}, mas, vivendo sob governo do Senado, esta sua
qualidade danosa não somente ficou escondida, mas foi sua glória.

{[}23{]} Concluo, portanto, voltando ao ser temido e amado -- os homens
amam quando lhes convém e temem o príncipe pelas escolhas deste -- que
um príncipe sábio deve fundar-se sobre o que é seu e não sobre o que é
dos outros; deve somente aplicar-se em evitar o ódio, como foi dito.

\quebra\section{\emph{QUOMODO FIDES A PRINCIPIBUS SIT SERVANDA}
{[}De que modo os príncipes devem conservar a fé\protect\footnote{Maquiavel está
  utilizando aqui o termo \emph{fede} que é traduzido pelo português
  ``fé''. Contudo, não se trata somente da fé como crença em valores
  religiosos, mas da confiança que se deposita na palavra dada a alguém
  ou na confiança depositada em outra pessoa.}{]}}

{[}1{]} Que seja louvável em um príncipe conservar a fé e viver com
integridade e não com astúcia, todos entendem; não obstante se vê, por
experiência em nossos tempos, que aqueles príncipes que fizeram grandes
coisas, tiveram a fé em pouca conta e souberam com a astúcia enganar o
juízo dos homens, e, ao final, superaram os que fundaram suas ações na
lealdade.

{[}2{]} Deveis, portanto, saber que são dois os gêneros de combate: um
com as leis, outro com a força\footnote{Esta ideia já se encontra
  expressa no início do cap. 12.}. {[}3{]} O primeiro é próprio do homem
e o segundo dos animais. {[}4{]} Mas porque o primeiro muitas vezes não
basta, convém recorrer ao segundo: portanto, a um príncipe é necessário
saber bem usar o animal e o homem. {[}5{]} Esta parte foi ensinada pelos
escritores antigos aos príncipes secretamente\footnote{Maquiavel faz
  referência aqui a uma tradição literária denominada de \emph{Arcana
  Imperii} ou os segredos da arte de governar, do qual o gênero
  literário dos ``espelhos de príncipes'' faziam parte na medida em que
  revelavam informações secretas sobre como conduzir os governos. Cf.
  Senellart, M. \emph{As artes de governar.} São Paulo: ed. 34, 2006,
  parte III, cap. 2.}, os quais escrevem como Aquiles, e muitos outros
daqueles príncipes antigos, foram alimentados pelo centauro
Quiron\footnote{Centauro é um monstro fabuloso, metade homem e metade
  cavalo. Quíron era um centauro sábio, filho de Filira e do deus
  Cronos, tutor de Aquiles.}, para que sob a sua disciplina os educasse.
{[}6{]} O que não quer dizer outra coisa ter por predecessor um meio
animal e um meio homem senão a necessidade que um príncipe tem de saber
usar uma e outra natureza, e que uma sem a outra não é durável\footnote{No
  caso, o governo do príncipe, que se perde sem uma dessas naturezas.}.

{[}7{]} Sendo, pois, necessário a um príncipe saber bem usar o animal,
deve destes tomar por modelos a raposa e o leão: porque o leão não se
defende das armadilhas e a raposa não se defende dos lobos. Necessita,
pois, ser raposa para conhecer as armadilhas e leão para amedrontar os
lobos: aqueles que são somente leão não entendem nada de governo.
{[}8{]} Não pode, e nem deve, portanto, um senhor prudente conservar a
fé quando tal observância se lhe volta contra e quando tiverem
desaparecido os motivos que levaram-no a fazer suas promessas. {[}9{]} E
se os homens fossem todos bons, este preceito não seria bom, mas porque
são decepcionantes e não a observariam consigo, você, então, não tem que
observar a fé com eles; nem faltarão ocasiões legítimas para um príncipe
poder colorir a inobservância. {[}10{]} Disto poder-se-ia dar infinitos
exemplos modernos e mostrar quantos tratados de paz, quantas promessas
tornaram-se inválidas e vãs pela infidelidade dos príncipes; e aquele
que soube melhor usar a raposa, teve melhor resultado. {[}11{]} Mas é
necessário saber mascarar esta natureza e ser grande simulador e
dissimulador: e são tão ingênuos os homens, e tanto se sujeitam às
necessidades presentes, que aquele que engana encontrará sempre quem se
deixará enganar.

{[}12{]} Eu não quero, dos exemplos recentes, silenciar um. Alexandre VI
não fez nunca outra coisa, não pensou nunca em outra coisa, senão em
enganar os homens, e sempre encontrou matéria para poder fazê-lo: e
nunca houve homem que tivesse maior eficácia em assegurar, e com os
maiores juramentos afirmasse uma coisa e que os observasse menos.
Todavia, sempre os enganos tiveram os resultados que ele desejava,
porque conhecia bem esta parte do mundo.

{[}13{]} A um príncipe, portanto, não é necessário ter de fato todas as
sobreditas qualidades, mas é muito necessário parecer tê-las. Assim,
ousarei dizer isto: que, tendo-as e observando-as sempre são danosas, e
parecendo tê-las são úteis; como parecer piedoso, fiel, humano, íntegro,
religioso, e sê-lo: mas estar com o ânimo predisposto, para que,
necessitando não sê-lo, você possa e saiba ser o contrário. {[}14{]} E
deve-se entender isto: que um príncipe, e muito mais um príncipe novo,
não pode observar todas aquelas coisas pelas quais os homens são
considerados bons, sendo frequentemente necessário, para conservar o
governo, agir contra a fé, contra a caridade, contra a humanidade,
contra a religião. {[}15{]} E, porém, é necessário que ele tenha um
ânimo disposto a mudar segundo o que lhe ordenem os ventos da fortuna e
as variações das coisas exigirem; e, como acima se disse, podendo não
separar-se do bem, mas, se necessário, saber praticar mal.

{[}16{]} Deve, pois, um príncipe ter grande cuidado para que não lhe
saia jamais da boca uma coisa que não seja plena das cinco sobreditas
qualidades, e que pareça, ao vê- lo e escutá-lo, todo piedade, todo
fiel, todo íntegro, todo humano, todo religioso; e não há coisa mais
necessária de se parecer ter do que esta última qualidade. {[}17{]} Os
homens, em geral, julgam mais com os olhos do que com as mãos, mais
pelas aparências, porque se vêem todos e se conhecem poucos; todos vêem
aquilo que você pareces ser, poucos conhecem aquilo que você é. E
aqueles poucos não se atrevem a opor-se à opinião dos muitos que têm a
majestade do Estado para defende-los; e nas ações de todos o homens,
sobretudo na dos príncipes, quando não há juiz para quem reclamar,
olha-se para os fins.

{[}18{]} Faça, portanto, um príncipe tudo para vencer e conservar o
governo: os meios serão sempre julgados honrosos e por todos serão
louvados, porque o vulgo se deixa levar por aquilo que parece e pelo
resultado das coisas, e no mundo não há senão o vulgo, e os poucos não
têm lugar quando os muitos têm onde se apoiarem. {[}19{]} Certo
príncipe\footnote{Alguns comentadores reconheceram aqui uma alusão ao
  rei espanhol Fernando, o católico.} dos tempos presentes, o qual não é
bom nomear, não prega nunca outra coisa senão a paz e a fé, e de uma e
de outra é inimicíssimo; e uma e outra, se ele as tivesse observado,
ter-lhe-iam muitas vezes tirado a reputação e o governo.

\quebra\section{\emph{DE CONTEMPTU ET ODIO FUGIENDO}
{[}Como se deve evitar ser desprezado e odiado{]}}

{[}1{]} Uma vez que, acerca das qualidades de que acima\footnote{Confira
  cap. XV.} se fez menção, eu falei das mais importantes, quero
discorrer brevemente sobre as outras a partir destes princípios gerais:
que o príncipe pense, como acima foi dito em parte\footnote{Confira cap.
  XVI, 3 e 18; cap. XVII, 12.}, acerca de como evitar aquelas coisas que
o fazem odioso e desprezível; e sempre que ele as evitar, terá cumprido
a sua parte e não encontrará nas outras infâmias qualquer perigo.
{[}2{]} O que o faz odioso, sobretudo, como eu disse\footnote{Confira
  cap. XVII, 13.}, ser rapace e usurpador dos bens e das mulheres dos
súditos, do que se deve abster. {[}3{]} E toda vez que não se tira da
totalidade\footnote{Algumas vezes Maquiavel se refere à
  \emph{universalità delli uomini}, o que poderia ser traduzido como
  universalidade ou totalidade, em ambos os casos conservando a ideia de
  congregação de todos os homens. Neste capítulo, mais do que nos
  outros, ao mobilizar mais vezes o termo \emph{universalle} e
  \emph{universalità,} nota- se claramente que ele pretende apresentar
  normas gerais para a condução do governo, num esforço em constituir
  uma teoria geral da ação política.} dos homens nem os bens e nem a
honra, eles vivem contentes: e somente se tem que combater a ambição de
poucos, a qual se contém de muitos modos e com facilidade. {[}4{]} O que
o faz desprezado é ser considerado volúvel, leviano, efeminado,
pusilânime, irresoluto: do que um príncipe deve se guardar do mesmo modo
que se evita um obstáculo perigoso, e planejar que, nas suas ações, se
reconheça grandeza, animosidade, gravidade, firmeza; e acerca da
administração privada dos súditos, querer que a sua sentença seja
irrevogável, e se conserve em tal opinião, que ninguém pense nem em
enganá-lo e nem em iludi-lo.

{[}5{]} O príncipe que cria em torno de si esta opinião é sempre
estimado, e contra quem é estimado com dificuldade se conjura\footnote{Conjurar
  remete à conspiração feita entre um grupo, o que implica numa
  pactuação e associação entre os membros desses grupos.}, com
dificuldade é assaltado, desde que se compreenda que é excelente e
reverenciado pelos seus. {[}6{]} Porque um príncipe deve ter dois medos:
um de dentro, por conta dos súditos; o outro de fora, por conta dos
poderosos externos. {[}7{]} Deste se defende com as boas armas e com os
bons amigos: e sempre, se tiver boas armas, terá bons amigos. {[}8{]} E
sempre estará firme a situação interna, quando estiver segura aquela
externa, se não estiver já conturbada por uma conjuração; e mesmo que
externamente se movessem contra ele, se ele se organizou e viveu
conforme eu disse, se ele não se amedrontar, sempre suportará todo
ímpeto, como eu disse que fez o espartano Nábis\footnote{Confira cap.
  IX, 19.}.

{[}9{]} Mas, sobre os súditos, quando as coisas externas não os movem,
deve-se temer que não conjurem secretamente, o que o príncipe se
assegura evitando sempre ser odiado ou desprezado, e mantendo o povo
satisfeito com ele: o que é necessário conseguir, como acima se disse
longamente. {[}10{]} E um dos mais poderosos remédios que tem um
príncipe contra as conjurações é não ser odiado pela totalidade dos
homens: porque sempre quem conjura crê com a morte do príncipe
satisfazer o povo, mas quando crê prejudica-lo, não tem coragem para
tomar semelhante decisão. {[}11{]} Porque são infinitas as dificuldades
dos conjurados, e, por experiência, se vê que muitas foram as
conjurações, e poucas chegaram a um bom fim. {[}12{]} Porque quem
conjura não pode estar sozinho, nem pode aceitar a companhia senão
daqueles que crê ser descontentes; e logo que você revele a sua
disposição a um descontente, lhe dá motivos para se contentar, porque
denunciando-o, ele pode tirar vantagem, de tal modo que, vendo a
vantagem assegurada desta parte, e da outra, da conjuração, vendo-a
hesitante e repleta de riscos, acaba acontecendo que raramente o
conjurado seja amigo ou que seja tão obstinado inimigo do príncipe, ao
ponto de se conservar fiel. {[}13{]} E, para reduzir tudo isso em breves
palavras, digo que, da parte dos conjurados, não há senão medo, inveja,
o pressentimento da pena que os amedronta; mas, da parte do príncipe,
está a majestade do principado, as leis, a defesa dos amigos e do estado
que o protegem. {[}14{]} De tal modo que, unida a todas estas coisas a
benevolência popular, é impossível que alguém seja tão imprudente que
conjure: porque, usualmente, um conjurado tem que temer antes da
execução da conjura, e, neste caso, deve temer ainda mais depois, uma
vez executada, tendo por inimigo o povo, nem pode, por isso, esperar
refúgio algum\footnote{Período intricado, no qual Maquiavel equipara os
  perigos anteriores e posteriores a uma conjuração.}. {[}15{]} Desta
matéria se poderiam dar infinitos exemplos, mas quero apenas limitar-me
a um que ocorreu nos tempos de nossos pais. {[}16{]} Senhor Aníbal
Bentivoglio\footnote{Aníbal I (1413-1445), senhor de Bologna de 1443 a
  1445.}, avo do atual Senhor Aníbal\footnote{Aníbal II (1469-1540),
  filho de Giovanni II e neto de Aníbal I, foi \emph{condottiero}
  (comandante militar) e responsável pela restauração do governo dos
  Bentivogli na cidade de Bolonha de maio de 1511 a junho de 1512.}, que
era príncipe em Bolonha, foi assassinado pelos Canneschi\footnote{Canneschi,
  os partidários dos Canetoli, que assassinaram Aníbal I em 24 de junho
  de 1445. Estes fatos são narrados mais extensamente na \emph{História
  Florentina,} livro VI, cap. 9.}, que conjuraram contra ele, não
restando outro descendente dele senão Senhor Giovanni\footnote{Giovanni
  II (1443-1508), filho de Aníbal I, que na época do assassinato tinha
  apenas 2 anos. Governou posteriormente Bolonha de 1463 a 1506.}, que
estava nas fraldas; imediatamente se levantou o povo, após tal
homicídio, e matou todos os Canneschi. {[}17{]} O que aconteceu pela
benevolência popular da qual a casa dos Bentivogli gozava naqueles
tempos, a qual foi tanta, que, morto Aníbal, não restando daquela casa
ninguém em Bolonha que pudesse reger o estado, e tendo indícios de que
em Florença havia alguém nascido dos Bentivogli\footnote{Sante
  (1424-1463), primo em primeiro grau de Aníbal I, governou Bolonha de
  1443-1463.} que tinha sido considerado, até então, filho de um
ferreiro, vieram por isso os bolonheses à Florença e lhe deram o governo
daquela cidade, a qual foi governada por ele até que Senhor Giovanni
chegasse à idade conveniente para o governo\footnote{Sante morre em
  1463, passando o governo a Giovanni II, que tinha na época 20 anos.}.

{[}18{]} Concluo, portanto, afirmando que um príncipe deve dar pouca
atenção às conjurações, quando o povo lhe é benevolente, mas, quando
este lhe é inimigo e odeia- o, deve temer todas as coisas e a todos.
{[}19{]} E os estados bem ordenados e os príncipes sábios pensaram com
toda diligência em não desprezar os grandes e em satisfazer o povo e
mantê-lo contente, porque esta é uma das matérias mais importantes que
tem um príncipe.

{[}20{]} Entre os reinos bem ordenados e governados de nosso tempo, está
o da França, e neste se encontram uma infinidade de boas instituições,
das quais depende a liberdade e a segurança do rei, das quais a primeira
é o parlamento e a sua autoridade. {[}21{]} Porque aquele que ordenou
aquele reino, conhecendo a ambição dos poderosos e a sua insolência, e
julgando ser-lhes necessário um freio na boca que os corrigisse -- e, de
outra parte, conhecendo o ódio do povo\footnote{Aqui o termo
  originalmente utilizado ``\emph{universale''} diz respeito ao povo,
  grupo político antagônico dos grandes conforme apresentado no cap. IX,
  2.} contra os grandes, nascido do medo, e querendo protegê-los --, não
quis que este fosse um remédio particular do rei, para tirar- lhe as
dificuldades que poderia ter com os grandes, ao favorecer a população, e
com a população, ao favorecer os grandes. {[}22{]} Por isso constitui um
terceiro em juiz, que seria aquele que, sem o encargo do rei, submetesse
os grandes e favorecesse os menores: nem poderia ser esta ordenação
melhor nem mais prudente, nem há maior causa para a segurança do rei e
do reino. {[}23{]} Do que se pode tirar um outro princípio digno de
nota: que os príncipes devem atribuir aos outros as tarefas
desagradáveis, e a si mesmos aquelas que lhes geram graças. {[}24{]} E
de novo concluo que um príncipe deve estimar os grandes, mas não se
fazer odiar pelo povo.

{[}25{]} Pareceria a muitos, considerada a vida e a morte de alguns
imperadores romanos, que fossem exemplos contrários a esta minha
opinião, encontrando algum que tenha sempre vivido distintamente e
demonstrado grande \emph{virtù} de alma, e, não obstante, ter perdido o
império, ou mesmo ter sido assassinado pelos seus que conjuraram contra
ele. {[}26{]} Querendo, portanto, responder a esta objeção, discorrerei
sobre as qualidades de alguns imperadores, mostrando as razões das suas
ruínas, não diferentes daquelas que foram antes apresentadas para mim;
e, por outro lado, examinarei aquelas coisas que são notáveis a quem lê
as ações daqueles tempos. {[}27{]} E quero restringir-me a considerar
todos aqueles imperadores que sucederam ao império de Marco, o filósofo,
até Maxímino, os quais foram: Marco, seu filho Cômodo, Pertinax,
Juliano, Severo, Antonino, seu filho Caracala, Macrino, Eliogábalo,
Alexandre e Maxímino\footnote{Pela ordem: Marco Aurélio Antonino
  (161-180 d.C.), Aurélio Cômodo Antonino (180-192 d.C.), Públio Elvio
  Pertinax (janeiro a março de 193 d.C.), Marco Dídio Juliano (março a
  junho de 193 d.C.), Sétimo Severo (193-211 d.C.), Antonino Caracala
  (211-217 d.C.), Opílio Macrino (217-18 d.C.), Eliogábalo (218-222
  d.C.), Severo Alexandre (222-235 d.C.) e Maximino, o Traço (235-238
  d.C.).}. {[}28{]} E é notável, primeiramente, que, enquanto nos outros
principados se deve combater apenas a ambição dos grandes e a insolência
do povo, os imperadores romanos tinham uma terceira dificuldade: ter que
suportar a crueldade e a avareza dos soldados\footnote{Aqui Maquiavel
  apresenta uma novidade conceitual na teoria dos humores políticos:
  agora já basta controlar os desejos antagônicos dos grandes e do povo,
  mas é necessário também lidar com uma outra força política, a saber: a
  crueldade e avareza dos soldados. De tal modo que há, portanto, não
  apenas duas forças políticas concorrendo sobre o governo do príncipe,
  mas também uma terceira força, os militares, que se constituem agora
  como força política relevante.}. {[}29{]} Coisa que, de tão difícil,
foi a ruína de muitos, sendo trabalhoso satisfazer aos soldados e ao
povo, porque o povo amava a quietude, e por isto eram gratos aos
príncipes modestos, e os soldados amavam os príncipes de alma militar e
que fossem cruéis, insolentes e rapaces, qualidades que desejavam que
ele exercesse sobre povo, para poder ter o soldo duplicado e desafogar a
sua avareza e crueldade. {[}30{]} Coisas que fizeram com que aqueles
imperadores, que por natureza ou por arte\footnote{Por artifício, por
  aprendizado.} não tinham uma grande reputação, com a qual refreavam
uns e outros, sempre se arruinavam. {[}31{]} E muitos deles, mais ainda
aqueles que como homens novos chegavam ao principado, quando percebiam a
dificuldade destas duas diversidades de humores\footnote{Note-se que
  agora os humores antagônicos são entre o povo e os militares e não
  mais, como no cap. IX, entre os grandes e o povo, o que demonstra
  claramente a existência de três humores conflitantes na dinâmica
  política apresentada aqui em \emph{O Príncipe.}}, se ocupavam em
satisfazer os soldados, pouco se preocupando em injuriar\footnote{Convém
  lembrar que o termo italinao \emph{iniuriare} advém do latim
  \emph{iniuria}, que é ir contra a lei, violar o direito, a justiça.
  Logo, ``injuriar o povo'' é atacar o seu direito.} o povo. {[}32{]}
Tal decisão era necessária, porque, não podendo os príncipes evitar de
serem odiado por alguns, devem esforçar-se primeiro em não ser odiado
pela totalidade\footnote{Totalidade, aqui, o universo dos cidadãos,
  independente do seu estrato social, contudo, não engloba o conjunto
  dos habitantes.} dos cidadãos, e quando não pode conseguir isto,
devem-se empenhar, com toda astúcia, para evitar o ódio daquela
totalidade que é mais poderosos. {[}33{]} Porém, aqueles imperadores
que, por serem novos, tinham necessitado de favores extraordinários,
procuravam o apoio dos soldados, mais do que do povo, o que, no entanto,
era útil ou não, conforme o príncipe soubesse conservar aquela reputação
entre eles.

{[}34{]} Foi por estas razões sobreditas que Marco, Pertinax e
Alexandre, sendo todos de vida modesta, amantes da justiça, inimigos da
crueldade, humanos e benignos, tiveram todos, com exceção de Marco,
triste fim. {[}35{]} Marco só viveu e morreu honradíssimo porque ele
chegou ao poder por direito hereditário, e não tinha que ser reconhecido
nem pelos soldados nem pelo povo; depois, possuía muitas \emph{virtù},
que o faziam venerando, tendo sempre, enquanto viveu, uma e outra
ordenação\footnote{Aqui Maquiavel se refere ao povo e aos exércitos como
  ``ordenação'', visto serem os fundamentos do governo do príncipe.
  Terminologia esta totalmente coerente com a economia do texto, haja
  vista que os dois grupos apresentam-se como sustentáculos do governo,
  logo, como ordenações políticas, tal qual ele mobiliza o termo em
  todos os seus escritos políticos.} nos seus limites, e não foi nunca
nem odiado nem desprezado. {[}36{]} Mas Pertinax tornou-se imperador
contra a vontade dos soldados -- os quais estavam habituados a viver
licenciosamente sob Cômodo, e não puderam suportar aquela vida honesta à
qual Pertinax queria submetê-los --, daí nasceu o ódio e, a este ódio,
juntou-se o desprezo, por ser ele velho, e arruinou-se logo no início de
sua administração. {[}37{]} E aqui se deve notar que o ódio se conquista
mediante as boas obras, bem como pelas más; porém, como eu disse acima,
um príncipe, querendo conservar o estado, é frequentemente forçado a não
ser bom. {[}38{]} Porque, quando aquela totalidade, ou o povo, ou os
soldados, ou que sejam os grandes, da qual você julga ter mais
necessidade para conservar-se, é corrompida, convém a você seguir-lhe o
humor para satisfazê-la, e então as boas obras lhe são inimigas.

{[}39{]} Mas consideremos Alexandre, o qual foi de tanta bondade que,
entre as outras coisas louváveis que lhe são atribuídas, há esta: que em
quatorze anos que teve o comando militar, ninguém foi morto por ele sem
ser julgado; não obstante, sendo considerado efeminado e homem que se
deixasse governar pela mãe\footnote{Severo Alexandre (222-235 d.C.) foi
  adotado pelo imperador Eliogabalo (218-222 d.C.), sendo sua mãe Giulia
  Mamea. Ambos foram assassinados por ordem de Maximino (235-238 d.C.).},
por isto caiu em desprezo, o exército conspirou contra ele e o
assassinou.

{[}40{]} Discorrendo agora, pelo contrário, acerca das qualidades de
Cômodo, de Severo, Antonino Caracala e Maximino, vemo-nos crudelíssimos
e rapacíssimos, os quais, para satisfazer os soldados, não abstiveram de
nenhum tipo de injúria que pudessem cometer contra o povo. {[}41{]} E
todos, exceto Severo, tiveram triste fim, porque Severo teve tanta
\emph{virtù} que, conservando os soldados amigos, ainda que o povo fosse
oprimido por ele, pode sempre reinar felizmente, porque aquelas suas
\emph{virtù} faziam-no admirável no conceito dos soldados e do povo, já
que este permanecia de certo modo atônito e atordoado, e aqueles outros
reverentes e satisfeitos. {[}42{]} E porque as ações dele foram grandes
e notáveis para um príncipe novo, eu quero mostrar brevemente o quanto
ele soube bem usar a natureza da raposa e do leão, cujas naturezas eu
disse acima\footnote{Confira cap. XVIII, 7-12.} serem necessárias a um
príncipe imitar.

{[}43{]} Tendo Severo conhecido a indolência do imperador Juliano,
persuadiu o seu exército, do qual era capitão na Eslavônia\footnote{Territórios
  eslavos a leste do Mar Adriático, Ilíria para os romanos, atual
  Bálcãs.}, de que era bom ele ir a Roma e vingar a morte de Pertinax, o
qual tinha sido morto pelos soldados pretorianos. {[}44{]} E com este
pretexto, sem mostrar que aspirava ao comando militar, moveu o exército
contra Roma e chegou na Itália antes que soubessem da sua partida.
{[}45{]} Chegando em Roma e morto Juliano, foi por temor eleito
imperador pelo Senado. {[}46{]} Restavam a Severo, depois deste
princípio, querendo assenhorear-se totalmente do estado, duas
dificuldades: uma na Ásia, onde Nigro\footnote{Gaio Pescennio Nigro,
  comandante das legiões na Síria, foi vencido por Sétimo Severo em
  Cizico, Nicéia e Isso em 194, e depois foi morto.}, chefe dos
exércitos asiáticos, se fez nomear imperador; a outra no ocidente, onde
estava Albino\footnote{Décimo Clódio Albino, comandante das legiões na
  Britânia, foi chamado por Severo a Roma para dividir o império, quando
  foi assassinado em Lyon em 197.}, o qual ainda aspirava ao poder.
{[}47{]} E porque julgava perigoso mostrar-se inimigo de todos os dois,
deliberou atacar Nigro e enganar Albino: ao qual escreveu que, tendo
sido eleito imperador pelo Senado, desejava partilhar aquela dignidade
com ele; e mandou-lhe o título de César e, por deliberação do Senado,
associou-o a ele como colega\footnote{Aqui o termo italiano
  \emph{collega} tem o sentido de posto ou grau equivalente, no sentido
  de pares num mesmo ofício.}, coisas que foram aceitas como verdadeiras
por Albino. {[}48{]} Mas, depois que Severo venceu e matou Nigro e
acalmadas as coisas no Oriente, voltando a Roma, lamentou-se, no Senado,
de como Albino, pouco reconhecido dos benefícios por ele recebidos,
havia dolosamente procurado matá-lo; por isto lhe era necessário punir a
sua ingratidão; foi encontrá-lo depois na França e lhe tirou o status e
a vida. {[}49{]} Portanto, quem examinar atentamente as ações deste,
encontrará nele um ferocíssimo leão e uma astuciosíssima raposa, e o
verá temido e reverenciado por todos, e não odiado pelos exércitos; e
não se surpreenderá se ele, homem novo, pode manter tanto poder, porque
a sua grandíssima reputação sempre o defendeu daquele ódio que podia
nascer no povo por causa de seus roubos.

{[}50{]} Mas Antonino, seu filho, foi homem que também tinha qualidades
excelentíssimas e que o tornavam admirável no conceito do povo e grato
para os soldados, porque era homem de guerra, suportava todas as
fadigas, desprezava todo alimento delicado e qualquer outras molezas, o
que o fazia amado por todos os exércitos. {[}51{]} Todavia, a sua
ferocidade e a sua crueldade foram tantas e tão inacreditáveis que, por
ter, depois de infinitos assassinatos individuais, matado grande parte
do povo de Roma e todo aquele de Alexandria, tornou-se muito odiado por
todo mundo e começou a ser temido também por aqueles que estavam perto
dele, de modo que foi assassinado por um centurião no meio do seu
exército. {[}52{]} Deve-se notar aqui que semelhantes mortes, que
decorrem da deliberação de um ânimo obstinado, são, para os príncipes,
inevitáveis, porque qualquer um que não se importe em morrer pode
prejudicá-lo, mas o príncipe deve temê-los bem menos, porque esses são
raríssimos. {[}53{]} Deve apenas guardar-se de não cometer grave injúria
contra algum daqueles dos quais se serve e que ele tem em torno de si a
serviço do seu principado, como fez Antonino, que matou injuriosamente
um irmão daquele centurião que o ameaçava todo dia, e ele, no entanto,
ainda o mantinha na sua guarda pessoal: o que era uma decisão temerária
e ruinosa, como veio a lhe acontecer\footnote{Este centurião que matou
  Antonino se chamava Marcial.}.

{[}54{]} Mas voltemos a Cômodo, para o qual foi muito fácil manter o
poder, por tê-lo por direito hereditário, sendo filho de Marco, e apenas
lhe bastava seguir as pegadas do pai, que os soldados e o povo estariam
satisfeitos. {[}55{]} Mas, sendo de ânimo cruel e bestial, para poder
usar a sua rapacidade sobre o povo, dedicou-se a entreter os exércitos e
torná-los licenciosos; de outra parte, não conservando a sua dignidade,
apresentando-se frequentemente nos teatros para combater com os
gladiadores, e fazendo outras coisas vis e pouco dignas da majestade
imperial, tornou-se desprezível no conceito dos soldados. {[}56{]} E
sendo odiado de um lado e, desprezado de outro, conspiraram contra ele e
o mataram.

{[}57{]} Falta narrar as qualidades de Maximino. Ele foi homem muito
belicoso e, estando os exércitos insatisfeitos com a moleza de
Alexandre, sobre a qual se discorreu acima, assassinado este,
elegeram-no para o Império, o qual não manteve por muito tempo, porque
duas coisas fizeram-no odioso e desprezível. {[}58{]} Uma, ser vilíssimo
por já ter sido pastor de ovelhas na Trácia, o que era muito bem sabido
por todos e lhe trazia grande descrédito no conceito de todos. {[}59{]}
A outra, porque, tendo, no início do seu principado, adiado a ida a Roma
e a tomada da posse da sede imperial, criou para si a imagem de
crudelíssimo, tendo por meio dos seus prefeitos\footnote{Ou ministros
  imperiais, responsáveis pela administração de várias províncias.}, em
Roma e em toda parte do império, praticado muitas crueldades. {[}60{]}
Tanto que, impressionando a todos pelo desdém e pela vileza da sua
dinastia e pelo ódio causado pelo medo da sua ferocidade, primeiro se
rebelou a África, depois o Senado com todo o povo de Roma e toda a
Itália conspiraram contra ele; ao que se acrescentou o seu próprio
exército, o qual, assediando Aquiléia e encontrando dificuldade no
assalto, insatisfeito com a sua crueldade e por ver-lhe tantos inimigos,
temendo-o menos, assassinaram-no.

{[}61{]} Eu não quero discorrer nem sobre Eliogábalo, nem sobre Macrino,
nem sobre Juliano, os quais, por serem em tudo desprezíveis, se
extinguiram de imediato, mas passarei à conclusão deste discurso, e digo
que os príncipes de nosso tempo tem esta dificuldade a menos, de
satisfazer com meios extraordinários os soldados ao lidar com eles,
porque, ainda que se deva ter para com eles alguma consideração, é coisa
que se resolve logo, por não ter nenhum destes príncipes exércitos que
estejam estacionados com o governo e a administração das províncias,
como eram os exércitos do Império Romano. {[}62{]} Outrossim, se então
era necessário satisfazer mais aos soldados que ao povo, era porque os
soldados podiam mais que o povo; agora, é mais necessário a todos os
príncipes, exceto ao Turco\footnote{O imperador otomano Selim I, que
  governou de 1512 a 1520.} e ao Sultão\footnote{O Sultão do Egito,
  Tuman Bey, que foi assassinado pelos turcos em 1517, anexando o Egito
  ao Império Turco.}, satisfazer mais ao povo que aos soldados, porque o
povo pode mais do que aqueles. {[}63{]} Do que eu excluo o Turco, tendo
aquele sempre em torno de si doze mil soldados de infantaria e quinze
mil cavaleiros, dos quais depende a segurança e a força do seu reino; e
é necessário que, adiada qualquer outra consideração, que o Senhor os
conserve amigos. {[}64{]} Igualmente para o reino do Sultão, estando
todo nas mãos dos soldados, convém a ele também que, sem o respeito do
povo, mantenha-os seus amigos. {[}65{]} E deve-se notar que este estado
do Sultão é diferente de todos os outros principados, porque ele é
semelhante ao pontificado cristão, o qual não se pode chamar nem de
principado hereditário nem de principado novo, porque os filhos do
príncipe velho não são os seus herdeiros e não se tornam senhores, mas
aquele que é eleito para o posto o é por aqueles que têm autoridade para
isto; {[}66{]} e sendo este ordenamento antigo, não se pode chamar de
principado novo, pois naquele não há nenhuma daquelas dificuldades que
existem nos novos, porque, se bem que o príncipe seja novo, as
ordenações daquele estado são velhas e ordenadas para recebê-lo como se
fosse seu senhor hereditário.

{[}67{]} Mas retornando à nossa matéria. Digo que qualquer um que
considerar o sobredito discurso verá ou o ódio ou o desprezo ser a
ocasião da ruína daqueles imperadores nomeados anteriormente, e saberá,
ainda, porque parte deles procedendo de um modo e parte de outro modo,
em ambas as situações alguns tiveram um final feliz e outros um triste
fim. {[}68{]} Porque, para Pertinax e Alexandre, por serem príncipes
novos, foi inútil e danoso imitar Marco, que estava no principado por
direito hereditário; e igualmente para Caracala, Cômodo e Maximino foi
coisa perniciosa imitar Severo, por não terem tido \emph{virtù}
suficiente para seguir as suas pegadas. {[}69{]} Portanto, um príncipe
novo em um principado novo não pode imitar as ações de Marco, nem por
isso é necessário seguir aquelas de Severo, mas deve tomar de Severo
aquelas qualidades que são necessárias para fundar o seu governo, e de
Marco aquelas que são convenientes e gloriosas para conservar um governo
que já esteja estabelecido e firme.

\quebra\section{\emph{AN ARCES ET MULTA ALIA QUAE QUOTIDIE A PRINCIPIBUS FIUNT UTILIA AN
INUTILIA SINT}
{[}Se as fortalezas e muitas outras coisas que cotidianamente os
príncipes fazem são úteis ou inúteis{]}}

{[}1{]} Alguns príncipes, para manter em segurança o governo, desarmaram
os seus súditos; alguns outros mantiveram divididas as terras
subordinadas. {[}2{]} Alguns alimentaram inimizades contra si mesmos;
alguns outros tentaram ganhar a confiança daqueles de quem ele
desconfiava no princípio do seu governo. {[}3{]} Alguns edificaram
fortalezas, alguns arruinaram-nas e destruíram-nas. {[}4{]} E ainda que
não se possa dar determinada sentença, senão se conhece as
particularidades daqueles governos nos quais se tivesse que tomar alguma
semelhante deliberação, todavia, eu falarei daquele modo geral que a
matéria por si mesma permite.

{[}5{]} Ora, jamais ocorreu que um príncipe novo desarmasse os seus
súditos, pelo contrário, quando os encontrou desarmados sempre os armou,
porque, quando você os armanda, aquelas armas tornam-se suas, tornando
fiéis aqueles dos quais você suspeitava, e aqueles que eram fiéis
mantêm-se assim, e de súditos, tornam-se seus partidários. {[}6{]} E,
porque não se podem armar todos os súditos, enquanto aqueles que você
arma se beneficiam, em relação aos outros pode proceder sem maiores
considerações: e essa diversidade no proceder para com aqueles, que eles
reconhecem, torna-os gratos a você; estes outros o desculpam, julgando
ser necessário ter mais méritos aqueles que estão sujeitos a maiores
perigos e obrigações. {[}7{]} Mas, quando você os desarma, começa a
contrariá-los, mostra que não confia neles, seja por covardia, seja por
desconfiança, e uma e outra destas opiniões suscita o ódio contra você.
E como você não pode ficar desarmado, convém lhe recorrer à milícia
mercenária, que tem aquelas características das quais acima se
falou\footnote{Confira cap. XIII, 5-6.}; e, ainda que esta fosse boa,
não pode ser tão boa que o defenda dos inimigos poderosos e dos súditos
suspeitos. {[}8{]} Por isso, como eu disse, um príncipe novo em um
principado novo sempre ordenou as suas armas e destes exemplos as
histórias estão repletas. {[}9{]} Mas, quando um príncipe conquista um
novo governo, que se agrega como membro ao seu antigo governo, então é
necessário desarmar este governo, exceto aqueles que, na sua conquista,
foram seus partidários; e, ainda aqueles, com o tempo e com as ocasiões,
é necessário torná-los moles e efeminados, e ordená-los de modo que, em
todo seu governo, as únicas armas que si mantenham sejam aquelas dos
seus próprios soldados, que, no seu governo antigo, vivem próximos de
ti.

{[}10{]} Costumavam os nossos antepassados\footnote{Os líderes políticos
  florentinos.}, e aqueles que eram estimados como sábios, dizer como
era necessário manter Pistóia com os partidários e Pisa com as
fortalezas, e por isto alimentavam, por toda a cidade que lhe estava
submetida, as diferenças entre os seus súditos, para possuí-las mais
facilmente. {[}11{]} Isto, naqueles tempos em que a Itália estava de
certo modo equilibrada\footnote{Alude aos tempos de Lorenzo, O
  Magnífico, quando um delicado e eficaz equilíbrio político-
  diplomático garantia a estabilidade da Itália, entre 1454 e 1492.},
era bem feito, mas não creio que se possa dar hoje este preceito, pois
não creio que as divisões nunca façam bem algum; antes, é forçoso,
quando o inimigo se aproxima, que as cidades divididas se percam logo,
porque sempre a parte mais débil aderirá às forças externas e a outra
parte não poderá governar. {[}12{]} Os venezianos, impelidos, como
creio, pelas razões supracitadas, alimentavam as facções
guélfas\footnote{Grupo político favorável ao Papa.} e
guibelinas\footnote{Grupo político favorável à república ou em algumas
  cidades favorável ao Império. As disputas entre os guelfos e os
  guibelinos marcaram os séculos XII a XIV, nas diversas cidades do
  norte da Itália.} nas cidades dominadas por eles; e ainda que não os
deixassem nunca chegar ao derramamento de sangue, todavia alimentavam
entre eles essas divergências, a fim de que, ocupados aqueles cidadãos
com seus conflitos, não se unissem contra eles. {[}13{]} O que, como se
vê, não correspondeu aos seus propósitos, porque, sendo derrotados em
Vailá\footnote{Cf. cap. XII, 26.}, imediatamente um destes
partidos\footnote{No caso, os guibelinos que eram favoráveis ao império.}
começou a ousar e tirou deles todo o governo. {[}14{]} Provam, portanto,
tais costumes a fraqueza do príncipe, porque, em um principado forte,
nunca se permitirão semelhantes divisões, porque trazem vantagens apenas
em tempos de paz, podendo-se mediante aquelas divisões mais facilmente
manipular os súditos, mas, chegando a guerra, semelhante ordenação
mostra a sua falácia\footnote{Modo de construção ou figura de lógica que
  redunda em falsidade.}.

{[}15{]} Sem dúvida, os príncipes tornam-se grandes quando superam as
dificuldades e as oposições que lhe são feitas, e, por isso, a fortuna,
sobretudo quando deseja fazer grande um príncipe novo -- o qual tem
maior necessidade de conquistar a reputação do que um príncipe
hereditário --, lhe dá inimigos e os faz agir contra ele, a fim de que o
príncipe tenha a oportunidade de vencê-los e, com aquela escada que os
seus inimigos lhe trouxeram, ele ascenda mais alto. {[}16{]} Porém,
muitos julgam que um príncipe sábio deve, quando tem a ocasião, com
astúcia alimentar algumas inimizades, a fim de que, oprimido por essas,
seja maior a sua grandeza.

{[}17{]} Têm os príncipes, e, sobretudo, aqueles que são novos,
encontrado mais fé e mais utilidade naqueles homens que, no princípio do
seu governo, eram considerados suspeitos, do que naqueles que no
princípio eram confiáveis. {[}18{]} Pandolfo Petrucci\footnote{Pandolfo
  Petrucci (1450-1512) apoderou-se gradualmente de Siena a partir de
  1487, governando-a com plenos poderes até a sua morte, em 1512.},
príncipe de Siena, governava mais com aqueles que lhe foram suspeitos do
que com os outros. {[}19{]} Mas, destas coisas, não se pode falar
longamente, porque elas variam segundo a matéria. Apenas direi isto, que
aqueles homens que no início de um principado tinham sido inimigos, e
que estão em uma condição que, para conservar-se, precisam do apoio do
príncipe, este sempre com grandíssima facilidade poderá conquistá-los. E
são eles mais ainda forçados a servi-lo com fidelidade, quando sabem ser
mais necessário apagar com obras a má opinião que se tinha sobre eles.
{[}20{]} E assim o príncipe sempre obtém mais utilidade destes do que
daqueles que, servindo-o com pouca consideração, descuidam das coisas do
príncipe.

{[}21{]} E porque o assunto\footnote{Na linha 19 acima, Maquiavel
  utilizou o termo \emph{subietto}, que nós traduzimos por matéria,
  tendo em vista sua vinculação a noção metafísica de substrato (cf.
  nota 58). Agora ele utiliza o termo \emph{materia}, que não possui o
  mesmo significado do nosso termo português `matéria', mas refere-se ao
  assunto, ao tema a ser tratado.} exige, não quero deixar de lado a
lembrança dos príncipes que conquistaram um novo governo por meio do
favor dos cidadãos, que se considere bem que causa moveu aqueles que
favoreceram esse príncipe a favorecê-los\footnote{Como destacam as
  edições Crítica e Comentada, neste período Maquiavel reitera a ideia
  de \emph{favor} que há entre as partes por meio da repetida utilização
  dos termos \emph{favori, favorito a favorirlo}. Tal redundância tem a
  função estilística de destacar como inicialmente se estabelece as
  relações de trocas políticas entre o príncipe novo e os seus novos
  dirigidos.}. {[}22{]} E se esta causa não é uma afeição natural para
com eles, mas somente porque não se contentavam com aquele governo, com
esforço e grande dificuldade se poderá conservá-los amigos, porque é
impossível que se possa contentá- los. {[}23{]} E discorrendo bem acerca
das razões disto, com aqueles exemplos que se encontram nas coisas
antigas e modernas, ver-se-á que é muito mais fácil conquistar como
amigos aqueles homens que se contentavam com o governo anterior, porém
eram seus inimigos\footnote{Inglese sugere que neste caso Maquiavel está
  citando o seu próprio exemplo em relação à família Médici, do qual foi
  inimigo político durante o governo de Soderini. Cf. Inglese, p. 143,
  nota 3.}, do que aqueles que, por não se contentarem com ele,
tornaram-se amigos e favoreceram o príncipe para ocupá-lo.

{[}24{]} É costume dos príncipes, para poder manter com mais segurança o
seu governo, edificar fortalezas que sejam o arreio e os freios daqueles
que desejassem atacá- lo, e ter um refúgio seguro para um assalto
imprevisto. {[}25{]} Eu louvo este costume porque ele é comum desde a
Antiguidade, todavia, nos nossos tempos, viu-se o Senhor Niccolò
Vitelli\footnote{Niccolò Vitelli (1414-1486), comandante militar da
  cidade de Castelo; apoderou-se de sua cidade em 1462, estando longe do
  poder entre 1474-1482, quando retorna ao governo e permanece até a sua
  morte. A destruição da fortaleza de Castelo é de 1482.} derrubar duas
fortalezas na cidade de Castello para conservar aquele governo; Guido
Ubaldo\footnote{Guido Ubaldo de Montefeltro governou a cidade de Urbino
  de 1482 a 1508.}, duque de Urbino, retornando aos seus domínios, donde
foi expulso por César Borgia\footnote{Entre junho de 1502 a agosto de
  1503.}, destruiu por completo todas as fortalezas da sua província e
julgou que, sem aquelas, muito dificilmente\footnote{Maquiavel usa aqui
  \emph{piú difficilmente}, que traduzindo literalmente seria ``mais
  dificilmente'', ou seja, um superlativo de um advérbio de intensidade,
  que não possui nenhum correlativo em português. Aqui inserimos dois
  advérbios para conservar a mesma ideia do texto.} perderia de novo
aquele governo. O Bentivogli, de volta à Bolonha, usaram meios
semelhantes. {[}26{]} São, pois, as fortalezas úteis ou não, conforme os
tempos: e se fazem lhe bem em uma parte, prejudicam-no em outra.
{[}27{]} E pode-se assim discorrer sobre esta parte: aquele príncipe que
tem mais medo do povo que dos forasteiros deve fazer as fortalezas.
Contudo, aquele que tem mais medo dos forasteiros que do povo, deve
deixá-las de lado. {[}28{]} À casa dos Sforza trouxe e trará mais
guerras\footnote{Não hesitando em vexar o povo, que não o apoiou na luta
  contra os franceses.} ao castelo de Milão, que Francisco Sforza aí
edificou\footnote{A construção do castelo durou de 1450 a 1572.}, que
qualquer outra desordem daquele governo. {[}29{]} Por isso, a melhor
fortaleza que existe é não ser odiado pelo povo, porque, ainda que você
tenhas as fortalezas, mas o povo o odeie, elas não o salvaram, porque
não falta nunca ao povo, uma vez que pegue em armas, forasteiros que lhe
ajudem. {[}30{]} Nos nossos tempos não se vê serem as fortalezas de
algum aproveito para príncipe algum, exceto para a condessa de
Forlí\footnote{Catarina Sforza (1463-1509), filha de Galeazzo Maria
  Sforza, casou-se com Girolamo Riario, senhor de Imola e Forlí. Quando
  este foi assassinado em 1488, ela reagiu com grande coragem e resistiu
  ao ataques até a chegada de um exército milanês.}, quando morreu o
conde Jerônimo, seu marido, porque, por meio delas, ela pode fugir da
fúria popular, esperar o socorro de Milão e recuperar o governo. E os
tempos eram tais que os forasteiros não podiam socorrer o povo. {[}31{]}
Mas, depois pouco lhe valeram as fortalezas, quando César Borgia a
assaltou e o povo, inimigo dela, aliou-se ao forasteiro. {[}32{]}
Portanto, nesta ocasião e antes, seria mais seguro a ela não ser odiada
pelo povo do que ter as fortalezas. {[}33{]} Considerando, portanto,
todas estas coisas, eu louvarei quem construir as fortalezas e quem não
as constrói; e censurarei todo aquele que, confiando nas fortalezas, dê
pouca atenção em ser odiado pelo povo.

\quebra\section{\emph{QUOD PRINCIPEM DECEAT UT EGREGIUS HABEATUR}
{[}O que convém a um príncipe para que seja estimado{]}}

{[}1{]} Nada faz um príncipe ser tão estimado quanto realizar grandes
feitos e dar raros exemplos de si. {[}2{]} Nós temos em nossos tempos
Fernando de Aragão\footnote{Fernando de Aragão, o católico (1452-1516),
  unificador dos reinos espanhóis no novo estado moderno Espanhol que
  veio a se transformar nos séculos seguintes em império colonial.},
atual rei de Espanha. A este se pode chamar quase um príncipe novo
porque, de um rei débil que era, tornou-se, pela fama e pela glória, o
primeiro rei dos cristãos; e se se considerar as suas ações, as
encontraram todas grandíssimas e algumas extraordinárias. {[}3{]} Ele,
no princípio do seu reinado, tomou Granada\footnote{Conquista de
  Granada, 1481-1492.}, e aquela empresa foi o fundamento do seu
governo. {[}4{]} Primeiro, ele o fez tranquilamente e sem levantar
suspeitas para que não fosse impedido, tendo ocupado com isto os ânimos
daqueles barões de Castela, os quais, pensando naquela guerra, não
pensavam em inovações\footnote{Trata-se aqui de mudanças políticas que
  pudessem ocupar a atenção de todos. Maquiavel também utiliza o termo
  com o sentido de fundação política, para se referir as novidades ou
  modificações no

  ordenamento político de uma cidade.}: e ele conquistava, por este
meio, reputação e poder sobre eles, que não se apercebiam; pode
sustentar o exército com o dinheiro da Igreja e do povo e, com aquela
longa guerra, dar o fundamento às suas milícias, as quais depois o
honraram. {[}5{]} Além disto, para poder empreender maiores feitos,
servindo-se sempre da religião, voltou-se a uma piedosa crueldade,
expulsando e espoliando, em seu reino, os mouros\footnote{Modo usual de
  se referir às populações de cultura árabe, tanto para aqueles que
  habitavam a Península Ibérica, quanto para os demais povos do norte da
  África e Península Arábica.}: não pode ser este exemplo mais miserável
e nem mais raro. {[}6{]} Assaltou, com este mesmo pretexto, a
África\footnote{Guerras feitas pelo rei Fernando no norte da África:
  conquista de Orano (1509), Bugia (1510) e Trípoli (1511).}; fez o
mesmo na Itália\footnote{Como relatado nos caps. III, 39 e XII, 1-4,
  Fernando foi protagonista na divisão do reino napolitano.};
ultimamente assaltou a França\footnote{Ataques ao território francês
  entre maio e dezembro de 1512 e conquista de Navarra. Essas afirmações
  são um forte indicativo da época de composição d'\emph{O Príncipe},
  entre os anos de 1513-1514.}. {[}7{]} E assim sempre fez e urdiu
grandes coisas, as quais sempre mantiveram elevada e cheio de admiração
o animo dos súditos, e ocupados no desenrolar destes eventos. {[}8{]} E
de tal modo estas ações nasceram uma das outras, que ele não deu nunca,
entre uma e outra, espaço aos homens para poderem calmamente agir contra
ele.

{[}9{]} Ademais, auxilia muito a um príncipe dar de si exemplos raros
quanto ao governo interno -- semelhante àqueles que se narram a respeito
do Senhor Bernabò de Milão\footnote{Bernabo Visconti (1323-1385).}--,
quando se tem a ocasião de encontrar alguém que faça alguma coisa
extraordinária na vida civil, ou para o bem ou para o mal, e escolher um
modo de como premiá-lo ou puni-lo, sobre o que há muito o que dizer.
{[}10{]} E, sobretudo, um príncipe deve esforçar- se para obter para si,
em todas as suas ações, a fama de grande homem e de excelente
engenhosidade.

{[}11{]} Um príncipe é ainda estimado quando é verdadeiro amigo e
verdadeiro inimigo, isto é, quando, sem hesitação alguma, ele se revela
em favor de alguém contra outro. {[}12{]} Posição essa que é sempre mais
útil do que ficar neutro, porque, se dois poderosos vizinhos seus
começam a combater, ou eles são de tal qualidade que, vencendo um deles,
você tenha que temer o vencedor, ou não. {[}13{]} Em qualquer destes
dois casos, lhe será sempre mais útil decidir-se por um dos lados e
declarar guerra aberta, porque, no primeiro caso, se não se decide, será
sempre presa daquele que vence, com prazer e satisfação daquele que foi
vencido; e não terá razão nem coisa alguma que o defenda nem que o
acolha, porque quem vence não deseja as amizades suspeitas e que não o
ajudaram na adversidade; quem perde, não o acolhe por não ter você, com
as armas em mãos, desejado compartilhar a sua fortuna.

{[}14{]} Tinha chegado à Grécia Antíoco\footnote{Cf. nota 30.}, movido
pelos etólios para expulsar os romanos; mandou ele embaixadores aos
aqueus, que eram amigos dos romanos, para exortá-los a permanecerem
neutros: e de outra parte os romanos os persuadiram a pegar em armas por
eles. {[}15{]} Veio esta matéria para deliberação no concílio dos
aqueus, onde o legado de Antíoco os persuadiu a ficarem neutros, ao que
o legado romano respondeu: «Quod autem isti dicunt non interponendi vos
bello, nihil magis alienum rebus vestris est; sine gratia, sine
dignitate, praemium victoris eritis»\footnote{``Ora, isto que eles vos
  dizem -- de não vos interpor nesta guerra -- não pode ser mais
  contrário a vossos interesses: sem benevolência, sem dignidade, vós
  seríeis a recompensa do vencedor'', Tito Lívio, \emph{História de
  Roma}, l. XXXV, cap. 49, 13. Citado de memória por Maquiavel, já que
  no original é: ``\emph{Nam quod optimum esse dicunt, non interponi vos
  bello, nihil immo tam alienum rebus vestris est: quippe'sine gratia,
  sine dignitate praemium victoris eritis.}''}. {[}16{]} E sempre
ocorrerá que aquele que não é amigo lhe pedirá a neutralidade, e aquele
que é seu amigo lhe pedirá para que se apresente com as armas. {[}17{]}
E os príncipes hesitantes, para fugir dos perigos presentes, seguem
muitas vezes aquela via da neutralidade, e muitas vezes arruínam-se.

{[}18{]} Mas, quando o príncipe se revela bravamente em favor de uma
parte, se aquele a quem você se associas vence, ainda que ele seja
poderoso e que você permaneças à mercê dele, ele tem, em relação a você,
uma obrigação, estabelece-se um contrato de confiança: e os homens nunca
são tão desonestos, com tamanho exemplo de ingratidão, a ponto de
oprimi-lo; depois, as vitórias não são nunca assim tão completas que o
vencedor não tenha que ter alguma consideração, sobretudo em relação à
justiça. {[}19{]} Mas, se aquele com o qual você se associa perde, você
é acolhido por ele, e, enquanto puder, ele ajudará você, e você se torna
companheiro de uma fortuna que pode ressurgir.

{[}20{]} No segundo caso\footnote{Confira linha 12 deste capítulo.},
quando aqueles que combatem um contra o outro são de tal qualidade que
você não tem que temer aquele que vence, é tanto maior a prudência em
associar-se a ele, porque você ajuda na ruína de alguém com a ajuda de
quem deveria salvá-lo, se fosse sábio; vencendo, este permanece à sua
mercê, e é impossível que, com a sua ajuda, não vença. {[}21{]} Deve-se
aqui notar que um príncipe deve cuidar para não fazer nunca campanha com
um mais poderoso do que si mesmo para prejudicar a outros, a não ser
quando a necessidade o constranja, como acima se disse, porque,
vencendo, fica prisioneiro dele, e os príncipes devem evitar o quanto
podem ficar ao capricho de outros. {[}22{]} Os venezianos se associaram
aos franceses contra o duque de Milão\footnote{Em 1499. Cf. cap. III,
  32.}, e podiam evitar fazer aquela campanha, da qual resultou a sua
ruína. {[}23{]} Mas quando não se pode evitá-la -- como ocorreu aos
florentinos\footnote{Em 1511-1512.}, quando o papa e a Espanha, com seus
exércitos, foram assaltar a Lombardia --, então deve o príncipe aliar-se
pelas razões sobreditas. {[}24{]} Nem creia nunca que estado algum possa
sempre tomar decisões seguras, ao contrário, pense que terá de tomar
sempre decisões duvidosas, porque isto está na ordem das coisas, que não
se pode nunca evitar um inconveniente sem incorrer em um outro, mas a
prudência consiste em saber conhecer os tipos de inconvenientes, e tomar
os menos ruins como bons.

{[}25{]} Deve, ainda, um príncipe mostrar-se amante da \emph{virtù},
dando hospitalidade aos homens virtuosos e honrando aqueles que são
excelentes em uma arte. {[}26{]} Deve ainda animar os seus cidadãos a
poder exercitar calmamente as suas atividades: nos negócios, na
agricultura e em todos os outros afazeres dos homens; e que
este\footnote{Apesar de estar no singular, o pronome refere-se aos
  homens do período anterior.} não tema em melhorar as suas propriedades
por temor de que lhe seja tirado, e aquele em abrir um comércio por medo
dos impostos. {[}27{]} Mas deve prever prêmios a quem deseje fazer estas
coisas e a qualquer um que pense, de algum modo, ampliar a sua cidade ou
o seu status. {[}28{]} Deve, além disto, nos tempos convenientes do ano,
manter o povo ocupado com as festas e os espetáculos, e, porque toda
cidade é dividida em artes ou em tribos\footnote{Tribos significa
  quarteirão ou bairros. As cidades romanas eram divididas em
  \emph{tribus}. Essa expressão se encontra também em Dante e na
  \emph{Arte da guerra} de Maquiavel.}, deve ter em conta essas
comunidades, reunir-se com elas algumas vezes, dar de si exemplo de
humanidade e de magnanimidade; todavia, mantendo sempre firme a
majestade de sua dignidade, porque esta não pode faltar em parte
alguma\footnote{A frase final ``\emph{perchè questo non vuol mancare in
  cosa alcuna}'' não se faz presente na Edição Crítica

  de Inglese, embora ele faça referência a esta na nota (cf. p. 153, n.
  8). Contudo, conforme ele mesmo nos informa, essa frase está presente
  num número considerável de manuscritos -- os reunidos no grupo
  \emph{y} -- que, ainda que não componham o escopo de manuscritos
  principais, aqueles que congrega o grupo G e \emph{y}, entende-se que
  tal sentença estava presente no autógrafo de \emph{O Príncipe}. Na
  Edição Comentada de Martelli essa frase está presente. Tendo em vista
  a presença da sentença em vários manuscritos e a sua completa
  adequação ao texto, decidimos inseri-la aqui, ainda que ela não esteja
  na Edição Crítica, por nós adotada como edição de trabalho.}.

\quebra\section{\emph{DE HIS QUOS A SECRETIS PRINCIPES HABENT}
{[}Dos secretários que os príncipes têm{]}}

{[}1{]} Não é de pouca importância para um príncipe a escolha dos
ministros, os quais são bons ou não, segundo a prudência do príncipe.
{[}2{]} E a primeira conjectura que se faz da inteligência de um senhor
é ver os homens que ele tem em torno de si: quando eles são aptos e
fiéis, sempre se pode reputá-lo como sábio, porque soube conhecê-los
suficientemente e sabe conservá-los fiéis. Mas, quando são de outro
modo, não se pode sempre fazer bom juízo dele, porque, ao primeiro erro
que o príncipe faz, o faz nesta escolha.

{[}3{]} Não havia ninguém que conhecesse o Senhor Antonio de
Venafro\footnote{Antonio da Venafro (1459-1530), professor do Studio
  (Colégio) de Siena, depois juiz, ministro e braço direito de Pandolfo
  Petrucci.}, ministro de Pandolfo Petrucci, príncipe de Siena, que não
julgasse Pandolfo homem valentíssimo, por aquele seu ministro. {[}4{]} E
como são de três gêneros de intelectos -- um entende por si, outro
discerne aquilo que outro entendeu, e o terceiro não entende nem por si
nem por outro, do qual o primeiro é excelentíssimo, o segundo excelente
e o terceiro inútil --, portanto, convinha pela necessidade que, se
Pandolfo não era do primeiro grau, que fosse do segundo. {[}5{]} Porque
toda vez que alguém tem discernimento suficiente para conhecer o bem ou
o mal que outro alguém faz ou diz, ainda que por si não tenha um
intelecto inventivo, conhece as obras más e boas do ministro, a estas
exalta e àquelas corrige, e o ministro não pode esperar enganá-lo e
manter-se bem.

{[}6{]} Mas, para que um príncipe possa conhecer o ministro, ele deve
fazer isto, que não falha nunca: quando você vê o ministro pensar mais
em si que em ti, e procura em todas as suas ações aquilo que lhe é mais
útil, este, que assim é, nunca será um bom ministro e nunca poderás
confiar nele. {[}7{]} Porque aquele que tem o governo de alguém nas
mãos, não deve pensar nunca em si, mas sempre no príncipe, e nunca
lembrar de coisas que não pertençam ao príncipe: e, de outro lado, o
príncipe, para bem conservá-lo, deve pensar no ministro, honrando-lhe,
fazendo-o rico, garantindo-lhe o cargo, atribuindo-lhe as honras e os
cargos\footnote{Cargos não somente no sentido de posto, mas de
  responsabilidades na condução do governo.}: a fim de que veja que não
pode estar sem ele, e que as muitas honras não lhe façam desejar mais
honras, as muitas riquezas não lhe façam desejar mais riquezas, os
muitos cargos lhe façam temer as mudanças. {[}8{]} Quando, pois, os
ministros, e os príncipes em relação aos ministros, são deste feitio,
podem confiar um no outro: quando são de outro modo, sempre o fim será
danoso ou para um ou para outro.

\quebra\section{\emph{QUOMODO ADULATORES SINT FUGIENDI}
{[}De que modo se deve evitar os aduladores{]}}

{[}1{]} Não quero deixar de lado um assunto importante e um erro do qual
os príncipes com dificuldade se defendem, se não são prudentíssimos ou
se não fazem boa escolha. {[}2{]} E estes são os aduladores, dos quais
as cortes estão cheias, porque os homens se comprazem tanto nas coisas
deles próprios, e de tal modo aí se enganam, que com dificuldade se
defendem desta peste. {[}3{]} E, para querer se defender, correm o risco
de tornarem-se desprezíveis, porque não há outro modo de proteger-se das
adulações que não seja os homens entenderem que não o ofendem quando
dizem-lhe a verdade; mas, quando todo mundo pode dizer-lhe a verdade,
falta-lhe a reverência. {[}4{]} Portanto, um príncipe prudente deve ter
um terceiro modo, escolhendo no seu governo homens sábios e apenas a
estes deve dar liberdade para dizer-lhe a verdade, e apenas sobre
aquelas coisas que ele lhes perguntar e não de outras -- mas deve
perguntar-lhes sobre todas as coisas --, e escutar as suas opiniões,
depois deliberar por si, a seu modo. {[}5{]} Com estes conselhos e, com
cada um deles, portar-se de modo que todos saibam que, quanto mais
livremente se falar, mais será aceito; fora aqueles sábios, não queira
ouvir ninguém, siga a deliberação tomada e seja obstinado nas suas
deliberações. {[}6{]} Quem faz de outro modo, ou cai por causa dos
aduladores ou frequentemente muda sua decisão pelas variações dos
pareceres, o que leva a ser pouco estimado.

{[}7{]} Eu desejo a este propósito apresentar um exemplo moderno. O
padre Luca\footnote{Luca Rinaldi, bispo de Trieste, foi em várias
  ocasiões embaixador do imperador alemão Maximiliano.}, homem do atual
imperador Maximiliano\footnote{Massimiliano d'Asburgo (1459-1519),
  eleito rei do Sacro Império Germanico em 1486, assumiu plenamente o
  título imperial em 1508.}, falando de Sua Majestade, disse como ele
não se aconselhava com as pessoas e não fazia nunca coisa alguma a seu
modo. {[}8{]} O que resultava de ter contrariado os conselhos
supracitados, porque o imperador é homem reservado, não comunica os seus
planos, não pede parecer: mas, quando os coloca em ação, começa-se a
conhecê-los e a descobri-los, começam a ser contestados por aqueles que
ele tem em entorno de si, e o imperador, que facilmente muda de opinião,
afasta-se de seus planos; daqui decorre que aquelas coisas que faz num
dia, destrói no outro, e que não seja nunca compreendido naquilo que se
deseja ou planeja fazer, e que não se pode fundar sobre as suas
deliberações.

{[}9{]} Um príncipe, portanto, deve aconselhar-se sempre, mas quando ele
deseje e não quando o outro deseja; antes, deve desencorajar qualquer um
que queira aconselhá- lo sobre alguma coisa caso ele não lhe tenha
perguntado, mas ele deve ser assíduo inquiridor, e depois, sobre as
coisas perguntadas, pacientemente ouvir a verdade: aliás, penso que se
alguém, por algum respeito não lhe fale a verdade, ele deve zangar-se.
{[}10{]} E os muitos que avaliam que certo príncipe, o qual cria para si
a imagem de prudente, seja assim considerado não por sua natureza, mas
pelos bons conselhos que tem em torno de si sem dúvida se enganam.
{[}11{]} Porque esta é uma regra geral que não falha nunca: que um
príncipe, o qual não seja sábio por si mesmo, não pode ser bem
aconselhado, a não ser que tenha tido a sorte de poder confiar-se a um
homem prudentíssimo, que pudesse a tudo governar. {[}12{]} Neste caso,
poderia ser bem aconselhado, mas duraria pouco, porque esse governador
em breve tempo lhe tomaria o governo. {[}13{]} Mas, aconselhando- se com
mais de um, um príncipe que não seja sábio não terá nunca os conselhos
concordes, não saberá por si mesmo uni-los, cada um dos conselheiros
pensará no seu interesse; ele não saberá corrigi-los nem distingui-los e
não se podem encontrar de outro modo, porque os homens sempre se mostram
maus, se por uma necessidade não se tornam bons. {[}14{]} Por isso,
conclui se que os bons conselhos, não importa de quem provenham, convém
que nasçam da prudência do príncipe e não a prudência do príncipe dos
bons conselhos.

\quebra\section{\emph{CUR ITALIAE PRINCIPES REGNUN AMISERUNT}
{[}Por que os príncipes da Itália perderam os seus reinos{]}}

{[}1{]} Observadas prudentemente, as sobreditas coisas fazem parecer
antigo\footnote{Aqui antigo com o sentido de hereditário, como se verá
  adiante} um príncipe novo e o tornam imediatamente mais seguro e mais
firme no governo do que se nele fosse antigo. {[}2{]} Porque um príncipe
novo é muito mais observado nas suas ações do que um hereditário, e
quando são conhecidas as suas virtudes, conquista mais ainda os homens e
muito mais se obrigam a ele do que a uma antiga dinastia. {[}3{]} Porque
os homens são muito mais presos às coisas presentes do que às coisas
passadas, e, quando nas presentes encontram o bem, se regozijam e não
procuram outra, antes, o defenderão de toda maneira, quando nas outras
coisas o príncipe não falta a si mesmo. {[}4{]} E assim terá dupla
glória, de ter dado início a um principado, tê-lo honrado e fortalecido
com boas leis, com boas armas e com bons exemplos, assim como tem dupla
vergonha aquele que, tendo nascido príncipe, perdeu o governo por sua
pouca prudência.

{[}5{]} E se se considera aqueles senhores que, na Itália, perderam o
governo em nossos tempos, como o rei de Nápoles\footnote{Confira cap.
  III, 39.}, o duque de Milão\footnote{Confira cap. III, 4.} e outros,
se encontrarão neles, primeiro, um defeito comum quanto às armas, pelas
razões que longamente foram discutidas acima; depois, se verá que alguns
deles ou tiveram por inimigo o povo ou, se tiveram o povo como amigo,
não souberam assegurar o apoio dos grandes. {[}6{]} Porque, sem estes
defeitos, não se perdem os governos que têm tanta força que podem manter
um exército em campanha. {[}7{]} Felipe da Macedônia\footnote{Confira
  cap. III, 21.}, não o pai de Alexandre, mas aquele que foi vencido por
Tito Quinto, não tinha um governo muito forte em relação à grandeza dos
romanos e dos gregos que o assaltaram; todavia, por ser homem de guerra
e que sabia manter o povo consigo e assegurar-se dos grandes, manteve
por anos a guerra contra aqueles, e, se ao final perdeu o domínio de
algumas cidades, restou-lhe, contudo, o reino.

{[}8{]} Portanto, que estes nossos príncipes que estavam há muitos anos
nos seus principados, por tê-los perdido depois, não acusem a fortuna,
mas a sua ignávia, porque, não tendo nunca, nos tempos de paz, pensado
que poderiam mudar -- o que é um defeito comum dos homens, não levar em
conta, na bonança, a tempestade --, quando depois vieram os tempos
adversos, pensaram em fugir e não em se defender, e esperaram que o
povo, insatisfeito com a insolência dos vencedores, os chamasse de
volta. {[}9{]} Esta decisão, quando não há outras, é boa, mas é muito
ruim ter deixado os outros remédios por este, porque nunca se deve
desejar cair, por acreditar que encontrará quem o acolha. {[}10{]} O que
ou não acontece, ou, se acontece, não é seguro para você, por ser esta
defesa vil e não depender de você. E somente aquelas defesas que
dependem de você e de sua própria \emph{virtù} são boas, são certas e
são duráveis.

\quebra\section{\emph{QUANTUM FORTUNA IN REBUS HUMANIS POSSIT ET QUOMODO ILLI SIT OCCURRENDU}
{[}Quanto pode a fortuna nas coisas humanas e de que modo se deve resistir a ela{]}}

{[}1{]} E não desconheço como muitos\footnote{Muitos aqui, conforme os
  editores (Inglese e Martelli), podem ser deste os antigos -- Cícero,
  Teofrasto, Salustio -- como os italianos do renascimento --
  particularmente Dante e Boccacio. De toda forma, é consenso que se
  tratava antes de um provérbio popular ao tempo de Maquiavel,
  largamente difundido entre os florentinos, donde ser essa a origem
  principal.} tiveram e têm opiniões que as coisas do mundo são, de
certo modo, governadas pela fortuna e por Deus; que os homens com a sua
prudência não podem corrigi-las, não havendo, então, remédio algum; e,
por isto, poderiam julgar que não seria necessário cansar-se muito
nessas coisas, mas deixar-se governar pela sorte\footnote{Importa
  recordar que fortuna não tem o mesmo signficado de sorte, como para
  nossa sociedade

  lusófona. Como apresentará Maquiavel, a fortuna está ligada a uma
  concepção cosmológica que ultrapassa a ideia de mero acaso central no
  conceito de sorte, donde o equívoco de se traduzir \emph{fortuna} (em
  italiano) por \emph{sorte} e vice-versa. Convém ainda destacar que
  esse é o capítulo do texto com maiores referências à noções
  cosmológicas e metafísicas do livro, haja vista que o autor mobiliza
  conceitos como \emph{natureza, livre arbítrio, tempo, providência}
  etc.}. {[}2{]} Esta opinião tem muito crédito nos nossos tempos por
causa da grande mudança nas situações que foram vistas e se vêem todos
os dias, que estão além de toda conjectura humana. {[}3{]} Diante do
que, pensando eu algumas vezes, inclinei-me de certo modo pela opinião
deles\footnote{Acredita-se que Maquiavel esteja fazendo remissão aos
  seus escritos \emph{Ghiribizi al Soderini} e \emph{Capitolo}

  \emph{di Fortuna}, ambos escritos em 1506.}. {[}4{]} Todavia, porque o
nosso livre arbítrio não esteja extinto, julgo ser verdadeiro que a
fortuna seja árbitra de metade das nossas ações, mas que ela ainda nos
deixa governar a outra metade, ou quase. {[}5{]} E comparo a fortuna a
um desses rios ruinosos que, quando se iram, alagam as planícies,
arruínam as árvores e os edifícios, levam terra desta parte, põem-na em
outro lugar: qualquer um foge em sua presença, todos cedem ao seu ímpeto
sem poder impedi-lo de modo algum. {[}6{]} E, ainda que sejam assim, aos
homens nada impede que, quando os tempos estão calmos, tomem
providências, com proteções e com diques: de modo que, ao se avolumarem
depois, ou iriam por um canal ou o seu ímpeto não seria nem tão violento
nem tão danoso. {[}7{]} Ocorre o mesmo à fortuna, a qual demonstra o seu
poder onde a \emph{virtù} não é ordenada para resisti-la, e então volta
o seu ímpeto para onde ela sabe que não se fizeram os diques e as
proteções para contê-la. {[}8{]} E se vocês considerem a Itália, que é o
palco\footnote{Maquiavel usa aqui o termo \emph{sedia}, com o mesmo
  sentido de lugar da cena teatral. Não devemos nos esquecer que em vida
  ele foi mais reconhecido como escritor de peças teatrais de sucesso,
  como \emph{Mandragora}. Essa dimensão artística percorre quase todos
  os seus escritos em vários momentos.} destas mudanças e que lhes deu o
movimento inicial, veriam ser ela uma planície sem diques e sem nenhuma
proteção, que, se ela fosse reparada pela conveniente \emph{virtù}, como
é a Alemanha, a Espanha e a França, ou estas planícies não teriam
sofrido as grandes variações que sofreram, ou elas não teriam lá
chegado. {[}9{]} E creio que basta ter dito isto, de modo geral, quanto
ao opor-se à fortuna.

{[}10{]} Mas, restringindo-me mais ao particular, digo como se vê hoje
um príncipe prosperar e amanhã arruinar-se, sem tê-lo visto mudar alguma
natureza ou atributo, o que, creio, decorra primeiro, das razões que
foram longamente discutidas antes\footnote{Confira cap. VII.}, isto é,
que aquele príncipe que se apóia totalmente na sua fortuna, cai quando
ela muda. {[}11{]} Creio, ainda, que seja feliz aquele que conforma o
seu modo de proceder com os atributos do tempo, do mesmo modo creio
infeliz aquele que cujo proceder diverge do tempo. {[}12{]} Porque se vê
os homens, nas coisas que lhes conduzem ao fim ao qual se propunham,
isto é glórias e riquezas, procederem diferentemente: um com respeito,
outro com ímpeto; um com violência, outro com arte\footnote{Engenhosidade,
  criatividade.}; um pela paciência, outro com o seu contrário; e cada
um pode com estes diversos modos alcançar o que deseja. {[}13{]} Vê- se
ainda dois homens prudentes, um chegar a seu intento, o outro não; e
semelhantemente dois igualmente prosperarem com diferentes modos de
agir, sendo um respeitoso e outro impetuoso, o que não decorre de outra
coisa, senão, de um atributo dos tempos que se conformam, ou não, ao seu
proceder. {[}14{]} Daqui advém aquilo que eu disse, que dois, agindo
diferentemente, chegam ao mesmo efeito; e dois igualmente agindo, um
alcança seu fim e o outro não. {[}15{]} Disto dependem ainda as
variações do que se considera bem, porque, se para alguém que governa
com respeito e paciência, os tempos e as coisas giram de modo que o seu
governo seja bom, ele prospera, mas, se o tempo e as coisas mudam, ele
se arruína, porque não muda o seu modo de proceder. {[]}16{]} Nem se encontra homem assim tão prudente que saiba se acomodar a isto: seja porque não pode desviar-se daquilo que a natureza o inclina, seja ainda porque, tendo alguém sempre prosperado, caminhando por uma estrada, não se pode persuadi-lo a deixá-la. {[}17{]} E, assim, o homem ponderado, quando é tempo de agir com ímpeto, não sabe fazê-lo, por isso se arruína; pois, se se mudasse a natureza com os tempos e as coisas, não se mudaria a fortuna.

{[}18{]} O papa Júlio II procedeu em todas as suas coisas impetuosamente
e encontrou tanto o tempo como as coisas de acordo com aquele seu modo
de proceder, que sempre teve bom êxito. {[}19{]} Observe-se primeiro a
ação que empreendeu em Bolonha\footnote{No verão de 1506 o papa atacou
  Perugia, e como essa cidade era um território bolonhês, logo ele
  atacou Bolonha. Em 11 de novembro deste ano, ele fez sua entrada
  vitoriosa na cidade.}, vivendo ainda o Senhor Giovanni Bentivoglio.
{[}20{]} Os venezianos não o viam com bons olhos; o rei de Espanha, o
mesmo; com a França, estava em tratativas sobre tal ação. Todavia, ele,
com a sua ferocidade e ímpeto, moveu pessoalmente aquela expedição.
{[}21{]} Movimento que deixou perplexos e imóveis a Espanha e os
venezianos, estes por medo, aquela pela vontade que tinha de recuperar
todo o reino de Nápoles; e, por outro lado, o Papa arrastou consigo o
rei de França, porque, vendo este aquele rei em marcha e desejando
fazê-lo amigo para subjugar os venezianos, julgou não poder lhe negar os
seus exércitos sem injuriá-lo abertamente. {[}22{]} Fez, portanto,
Júlio, com seu movimento impetuoso, aquilo que nunca outro pontífice,
com toda prudência humana, teria feito. {[}23{]} Porque, se ele tivesse
esperado para partir de Roma com os acordos firmados e todas as coisas
ordenadas, como qualquer outro pontífice teria feito, nunca teria
conseguido, porque o rei de França teria tido mil desculpas e os outros
incutido-lhe mil temores. {[}24{]} Eu não vou tratar das outras ações
suas, que foram todas semelhantes e todas tiveram bons resultados, e a
brevidade da vida não lhe deixou sentir o contrário, porque, se
sobreviessem tempos nos quais fosse necessário proceder com prudência,
daí seria a sua ruína, pois nunca se teria desviado daqueles costumes
aos quais a natureza o inclinava.

{[}25{]} Concluo, portanto, que, mudando a fortuna, e permanecendo os
homens obstinados nos seus costumes, são felizes enquanto a fortuna e os
costumes concordam e, quando discordam, são infelizes. {[}26{]} Acredito
que seja melhor ser impetuoso que ponderado, porque a fortuna é mulher e
é necessário, se se a quer, subjugá-la, submetê- la e bater
nela\footnote{Conforme Inglese (p. 167, nota 2), a expressão
  ``\emph{batterla e urtala''} equivale a \emph{``te-la submetida}'' ou

  ``\emph{possuí-la carnalmente}'', que se adéqua bem melhor ao
  raciocínio. Essa metáfora de possessão carnal da mulher é mais
  conforme ao argumento, tendo em vista que no imaginário feminino
  sugerido por Maquiavel aqui, a fortuna como mulher deseja ser possuída
  carnalmente pelo jovem impetuoso, mais do que ser agredida por ele tão
  somente.}. {[}27{]} E se vê que ela se deixa mais vencer pelos
impetuosos do que por aqueles que friamente procedem, e, por isso, como
é mulher, sempre é amiga dos jovens, porque são menos prudentes, mais
ferozes e comandam-na com mais audácia.

\quebra\section{\emph{EXORTATIO AD CAPESSENDAM ITALIAM IN LIBERTATEMQUE A BARBARIS VINDICANDAM}
{[}Exortação para tomar a defesa da Itália e libertá-la das mãos dos
bárbaros\protect\footnote{Convém lembrar que o termo bárbaro possuia a acepção
  de povo estrangeiro invasor que não possuía os costumes e,
  principalmente, a língua. No caso da Itália do Renascimento, haja
  vista o histórico de invasões de exércitos estrangeiros, conforme
  relatado aqui no \emph{Príncipe}, o título propõe diretamente o fim
  dessas invasões e a constituição de um estado italiano dominado a
  península itálica.}{]}}

{[}1{]} Consideradas, pois, todas as coisas discutidas acima, e pensando
comigo mesmo se na Itália do presente são tempos propícios para honrar
um novo príncipe, e se houvesse matéria que desse ocasião para que
alguém prudente e virtuoso pudesse aí introduzir a forma\footnote{``Introduzir
  a forma'' é uma expressão oriunda das metafísicas clássicas,
  principalmente o aristotelismo, cuja ideia principal remonta a noção
  de que todas as coisas (substâncias) são possuem uma matéria e uma
  forma que se unem para formar o ente ou o composto substancial. No
  caso, o povo é a matéria (\emph{hipokeimenon}) na qual o regime
  confere a forma, aqui em particular, essa é a tarefa do príncipe,
  conferir um regime ou uma forma ao povo. Maquiavel está retomando o
  que foi dito no cap. VI, 10; cf. também nota 69.}, e que o honrasse e
fizesse o bem a todos os homens da Itália, me parece que tantas coisas
concorrem para benefício de um príncipe novo, que eu não saberia qual
tempo seria mais apto para isto. {[}2{]} E se, como eu disse, era
necessário, para se ver a \emph{virtù} de Moisés, que o povo de Israel
fosse escravo no Egito; e para se conhecer a grandeza de ânimos de Ciro,
que os persas fossem oprimidos pelos medos; e para se conhecer a
excelência de Teseu, que os atenienses estivesse dispersos; {[}3{]}
assim no presente, desejando-se conhecer a \emph{virtù} de um espírito
italiano, era necessário que a Itália se reduzisse às condições
presentes, e que ela fosse mais escrava que os hebreus, mais serva que
os persas, mais dispersa que os atenienses: sem chefe, sem ordem,
abatida, espoliada, dilacerada, invadida e tivesse suportado toda sorte
de ruína.

{[}4{]} E se bem que até aqui já tenha aparecido, em alguns, sinal que
pudesse levar a julgar que alguém fosse ordenado por Deus para a sua
redenção, todavia se viu como depois foi reprovado pela fortuna, no
momento mais alto das suas ações\footnote{A referência aqui é para César
  Bórgia, que morreu sem cumprir seu desejo maior de governar toda a
  Itália.}. {[}5{]} De modo que a Itália jaz como sem vida, espera
aquele que possa curar as suas feridas e ponha fim aos saques da
Lombardia, aos impostos no reino de Nápoles e na Toscana, e a cure
daquelas suas chagas, já por um longo tempo supuradas. {[}6{]} Vê-se
como ela roga a Deus que lhe mande alguém que a redima desta crueldade e
insolência bárbaras. {[}7{]} Vê-se ainda toda pronta e disposta a seguir
uma bandeira, desde que haja alguém que a empunhe. {[}8{]} Nem se vê, no
presente, em que ela possa mais esperar senão na sua ilustre
Casa\footnote{Referência direta a família Médici.}, a qual -- com a sua
fortuna e \emph{virtù}, favorecida por Deus e pela Igreja, da qual é
agora o príncipe\footnote{No caso, o papa Leão X (Giovanni di Médici).}
-- pode tornar-se chefe desta redenção. {[}9{]} O que não seria muito
difícil, se mantiver à vista diante de você as ações e a vida dos acima
nomeados\footnote{Moisés, Ciro e Teseu.}, e ainda que aqueles homens
sejam raros e maravilhosos, todavia foram homens, e tiveram cada um
deles menores ocasiões do que a presente, porque o feito deles não foi
mais justo do que este, nem mais fácil, nem foi Deus mais amigo deles do
que do senhor. {[}10{]} Aqui é grande justiça: «iustum enim est bellum
quibus necessarium, et pia arma ubi nulla nisi in armis spes
est».\footnote{``Justas, pois, são as guerras necessárias, e piedosas
  são as armas quando só nelas há esperança.'' Tito Lívio,
  \emph{História}, L. IX, cap. 1, citado de memória por Maquiavel, já
  que o correto é ``\emph{Iustum est bellum, Samnites, quibus
  necessarium, et pia arma, quibus nulla nisi in armis rilinquitur
  spes}.''} {[}11{]} Aqui há grandíssima disposição, e não pode pode
haver, onde há grande disposição, grande dificuldade, desde que esta
casa siga os ordenamentos daqueles que eu tenho proposto como exemplo.
{[}12{]} Além disto, no exemplo de Moisés se vêem exemplos
extraordinários, conduzidos por Deus: o mar que se abriu; uma nuvem
escoltou-o pelo caminho; da pedra jorrou água; aqui choveu o maná. Todas
as coisas concorreram para sua grandeza. {[}13{]} O restante o senhor
deve fazer, Deus não deseja fazer tudo, para não tirar o livre arbítrio
e parte daquela glória que nos cabe.

{[}14{]} E não é surpreendente se nenhum dos pré-nomeados italianos
puder fazer aquilo que se pode esperar que faça a sua ilustre Casa, e
se, em tantas revoluções na Itália e em tantas manobras de guerra,
pareça sempre que, na Itália, a \emph{virtù} militar foi extinta, isto
ocorre as porque suas antigas ordens não eram boas, e não houve ninguém
que tivesse sabido encontrar novas. {[}15{]} E nenhuma coisa dá tanta
honra a um homem novo\footnote{No caso aqui significa o príncipe novo
  que chega ao poder.}\textsuperscript{348} que ascende quanto fazer
novas leis e novas ordens: estas coisas, quando são bem fundadas e há
nelas grandeza, tornam-no reverenciado e admirado. {[}16{]} E na Itália
não falta matéria para introduzir qualquer forma: aqui é grande a
\emph{virtù} nos membros, quando ela não falta nos chefes. {[}17{]}
Observe os duelos e os combates de poucos, quanto os italianos são
superiores na força, na destreza e na astúcia; mas, quando se trata de
exércitos, não têm sucesso. {[}18{]} E tudo se origina da debilidade dos
chefes: porque aqueles que sabem não são obedecidos e qualquer um parece
saber, não havendo por aqui ninguém que tenha sabido sobressair-se pela
\emph{virtù} e pela fortuna, ao qual os outros cedam.

{[}19{]} Daqui advém que, em tanto tempo, com tantas guerras feitas nos
últimos vinte anos, quando houve um exército todo italiano, ele sempre
foi mal, do que é testemunha primeiro o Taro\footnote{Batalha de Fornovo
  sobre Taro (6 de julho de 1495), entre os franceses e a liga de
  estados italianos.}, depois Alexandria\footnote{Batalha de Alexandria
  (28 de agosto de 1499), novamente entre os franceses e os italianos.},
Cápua\footnote{Saque de Cápua pelos franceses (24 de julho de 1501).},
Gênova\footnote{Sufocamento da rebelião de Gênova contra os franceses
  (28 de abril de 1507).}, Vailá\footnote{Confira cap. XII, 26.},
Bolonha\footnote{Abandono da defesa de Bolonha (20 de maio de 1511).},
Mestre\footnote{Incêndio da cidade de Mestre e batalha de Vicenza (7 de
  outubro de 1513).}.

{[}20{]} Querendo, portanto, a sua ilustre Casa seguir aqueles
excelentes homens que redimiram as suas províncias, é necessário, antes
de todas as outras coisas, como verdadeiro fundamento de qualquer
empresa, prover-se de armas próprias, porque não se pode ter nem mais
fiéis, nem mais verdadeiros, nem melhores soldados: e se bem que cada um
destes soldados seja bom, todos juntos tornam-se melhores quando se vêem
comandar por seu príncipe e por ele serem honrados e bem tratados.
{[}21{]} É necessário, portanto, preparar-se para estas armas, para
poder com a \emph{virtù} itálica defender-se dos estrangeiros. {[}22{]}
E ainda que a infantaria suíça e a espanhola sejam consideradas
terríveis, contudo, em ambas há defeitos para o qual uma terceira forma
de organização militar poderia não somente opor-se a elas, mas,
confiante, superá-las. {[}23{]} Porque os espanhóis não podem resistir a
uma carga de cavalaria e os suíços têm medo dos infantes quando os
encontram no combate, obstinados com eles: donde se vê e verá, na
prática, os espanhóis não poderem suportar um ataque da cavalaria
francesa e os suíços serem derrotados pela infantaria espanhola.
{[}24{]} E, ainda que desta última não se tenha tido uma prova completa,
todavia se viu um ensaio na batalha de Ravena\footnote{Confira cap. III,
  6.}, quando as infantarias espanholas se defrontaram com os batalhões
alemães, os quais se utilizaram das mesmas ordenações dos suíços, e na
qual os espanhóis, com a agilidade do corpo e a ajuda de seus
escudos\footnote{No original, \emph{brocchieri,} pequenos escudos
  redondos munidos no centro de uma ponta grossa que servia como arma de
  defesa e ataque.}, entraram sozinhos entre os lanceiros, e estavam
seguros para atacá-los sem que os alemães tivessem como se defender; e
se não fosse a cavalaria, que os atacou, os espanhóis teriam feridos e
matado todos. {[}25{]} Pode-se, portanto, conhecendo o defeito de uma e
de outra destas infantarias, criar uma nova, que resista à cavalaria e
não tenha medo dos infantes: e que será fruto das armas e das mudanças
das ordenações; e estas são daquelas coisas que, novamente ordenadas,
dão reputação e grandeza a um príncipe novo.

{[}26{]} Não se deve, pois, deixar passar esta ocasião, a fim de que a
Itália, depois de tanto tempo, veja aparecer um seu redentor. {[}27{]}
Nem posso exprimir com qual amor ele seria recebido em todas as
províncias que têm sofrido por causa destas invasões estrangeiras, com
que sede de vingança, com que obstinada confiança, com que piedade, com
que lágrimas. {[}28{]} Quais portas se lhe fechariam? Quais povos lhe
negariam obediência? Que invejas se lhe oporiam? Que italiano lhe
negaria o serviço? A todos fede este bárbaro domínio. {[}29{]} Tome,
portanto, a sua ilustre Casa este assunto com aquele ânimo e aquela
esperança com que se tomam as façanhas justas, a fim de que, sob o seu
estandarte, esta pátria seja enobrecida e, sob os seus auspícios, se
realize aquele dito de Petrarca:

\begin{quote}
\emph{Virtù} contra o furor

Tomará as armas, e que seja breve o combate

Que o antigo valor

Nos corações italianos não está ainda morto.\footnote{``Virtù contro a
  furore

  \begin{quote}
  prenderà l'armi, e fia el combatter corto,

  che l'antico valore

  nelli italici cor non è ancor morto''.

  (Italia mia, canzoniere, CXXVIII, 93-96)
  \end{quote}}
\end{quote}